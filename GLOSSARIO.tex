\setcounter{secnumdepth}{-2}

\chapter{Glossário}

\begin{hangparas}{.25in}{1}
{\small

\textbf{Abayé:} Comentarista famoso da Mishná de origem Babilônica, do
século IV.

\textbf{Abodá Zará:} Adoração de ídolos.

\textbf{Agudot:} Aglomerações de pessoas.

\textbf{Ah-Ab:} Rei que matou o sábio Nabot para apossar-se de sua propriedade.

\textbf{Aharé Mot:} Trecho de leitura semanal do 3º livro do Pentateuco chamado Levítico.

\textbf{Akh:} Mas.

\textbf{Al:} Negação.

\textbf{Amá:} Cúbito. Medida equivalente a 48 centímetros.

\textbf{Amalec:} Neto de Esaú, que herdou de seu avô o ódio mortal que
este tinha por seu irmão gêmeo, Jacob, e por todos os seus
descendentes.

\textbf{Amon:} Irmão de Moab, frutos ambos da união de Lot (sobrinho do
patriarca Abraham) com suas duas filhas, os três únicos sobreviventes
de Sodoma e Gomorra. Pensando ter sido o mundo destruído e não haver
mais nenhum homem sobre a terra, as filhas de Lot embebedaram o próprio
pai para se unirem a ele. Dessa união a filha mais velha concebeu um
filho, a quem deu o nome de Moab, e a mais nova teve outro filho, que
chamou de Amon. Ao sair do Egito e atravessar o deserto para atingir a
Terra Prometida, o povo judeu precisava passar por muitos povoados.
Amon e Moab proibiram terminantemente a passagem dos judeus por suas
cidades, e chegaram a contratar o maior mago da terra, Bil-An, para
amaldiçoar o povo de Israel (vide Números, capítulo 22). Diante de tal
atitude, proibiu-se aos judeus que se casassem com os integrantes
daqueles povos.

\textbf{Amonita:} Povo descendente de Amon.

\textbf{Amorita ou Emorita:} Um dos povos que vivia na terra de Canaã.

\textbf{Arakhin:} Capítulo de um dos tratados do Talmud.

\textbf{Ashera:} Deusa da fertilidade entre o povo de Canaã.

\textbf{Atzeret:} Nome dado ao sétimo dia de Pessah (Páscoa) e ao
oitavo dia de Sucot (Festa das Cabanas).

\textbf{Avon:} Pecado.

\textbf{Az-Hará:} Alertar-se.

\textbf{Azariah:} Um dos três sábios que santificaram o nome de Deus na
época de Nabucodonosor penetrando na caldeira de fogo e saindo
totalmente ilesos.

\textbf{Azanechá:} Cinto bélico no qual se pendura todo armamento do
soldado em guerra.

\textbf{Baba Batra:} Capítulo de um dos tratados do Talmud.

\textbf{Baba Kamma:} Capítulo de um dos tratados do Talmud.

\textbf{Baba Metzia:} Capítulo de um dos tratados do Talmud.

\textbf{Bakar:} Gado.

\textbf{Ball:} Nome dado a um ídolo.

\textbf{Bekhorot:} Primogênitos. Também nome de um Capítulo de um dos
tratados do Talmud.

\textbf{Belinta:} Esteira.

\textbf{Berakhot:} Bênçãos. Também nome de um capítulo de um dos
tratados do Talmud.

\textbf{Beit Hashoebá:} Local onde havia uma fonte cuja água era usada
na 2ª noite da festa de Sucot (Cabanas) na época do Templo Sagrado.

\textbf{Betzá:} Capítulo de um dos tratados do Talmud.

\textbf{Bicurim:} Primícias (de Pentecostes).

\textbf{Bitlo ve lo Bitlo:} Não cumpriu e cumpriu.

\textbf{Bitrumath:} Contribuição espontânea entregue ao sacerdote na
época do Templo.

\textbf{Caleb:} Representante chefe da tribo de Judá. Cada uma das 12
tribos tinha o seu representante chefe.

\textbf{Casher:} O que é permitido em matéria de alimento, segundo a
religião judaica.

\textbf{Cohanim:} Plural de Cohen.

\textbf{Cohanim Guedolim:} Plural de Cohen Gadol.

\textbf{Cohen:} Descendente da família sacerdotal.

\textbf{Cohen Gadol:} Sumo sacerdote no Templo Sagrado.

\textbf{Col:} Cada.

\textbf{Cutá:} Mistura de farinha fermentada com leite.

\textbf{Demai:} Capítulo de um dos tratados do Talmud.

\textbf{Derash:} Explicação com comentário.

\textbf{Doresh el hametim:} Evocando os mortos a fim de descobrir o futuro.

\textbf{Ebal:} Nome de um monte sobre o qual metade das tribos recebeu uma parte das leis da Torah (1ª metade).

\textbf{Ed shaker:} Falso testemunho.

\textbf{Ed shav:} Falso testemunho.

\textbf{Edumeu (ou Adomi):} O nome Edom, que significa vermelho, é
atribuído a Esaú, descendente do patriarca Isaac, e Adomi ou Adomen é
a maneira como são conhecidos os seus descendentes.

\textbf{Eduyoth:} Capítulo de um dos tratados do Talmud.

\textbf{Efa:} Medida equivalente a 36,44 litros cúbicos.

\textbf{Efod:} Espécie de manto curto usado pelo Sumo Sacerdote na
época do Templo Sagrado.

\textbf{Eglá Arufá:} Bezerra degolada.

\textbf{Elazar:} Filho de Aharon, o Sumo Sacerdote.

\textbf{Elohim:} Nome divino.

\textbf{Erub:} Confundir fronteiras (no caso do sábado).

\textbf{Erubin:} Capítulo de um dos tratados do Talmud.

\textbf{Etrog:} Tipo de fruta cítrica que faz parte das quatro espécies
usadas na festa de Sucot.

\textbf{Exilarca (Rosh Galut):} Descendente da casa de Davi que era
representante geral e diretamente responsável por toda a coletividade
judaica numa determinada cidade ou num determinado país do exílio (como
a Pérsia e a Babilônia).

\textbf{Guedidá:} Tatuagem.

\textbf{Guemará:} Seria a continuação da Mishná com a qual se forma o
Talmud.

\textbf{Guer toshab:} Habitante estranho que não é nativo do próprio local.

\textbf{Guerizim:} Nome de um monte sobre o qual metade das tribos recebeu uma parte das leis da Torah (2ª metade).

\textbf{Guezerá:} Sentença decretada.

\textbf{Guezerá shavá:} Sentença decretada (porém comparada a outra semelhante).

\textbf{Guid hanashé:} Tendão encolhido sobre a junção da coxa.

\textbf{Guitin:} Capítulo de um dos tratados do Talmud.

\textbf{Habdalá:} Separação, cerimônia de despedida do sábado.

\textbf{Haguigá:} Festividade.

\textbf{Halá:} Nome dado ao pão usado no sábado e nos dias de festa, de
formato diferente do comum.

\textbf{Halachá:} Preceito rabínico.

\textbf{Halachot Guedolot:} Grandes regras. Nome de uma obra talmúdica.

\textbf{Halalá:} Nome que se dava a uma mulher viúva, separada do
marido, ou prostituta, que estavam terminantemente proibidas de desposar
o Sumo Sacerdote ou um simples Sacerdote, durante a existência do Templo
Sagrado.

\textbf{Halel:} Oração de louvor que se recita especialmente nas
festas e nos primeiros dias de cada mês.

\textbf{Halitzá:} Recusa de uma mulher sem filhos que acabou de
enviuvar a unir-se ao irmão solteiro de seu marido.

\textbf{Hametz:} Que contém fermento.

\textbf{Hanayá:} Pouso.

\textbf{Hananiah:} Um dos três sábios que santificaram o nome de Deus
na época de Nabucodonosor penetrando na caldeira de fogo e saindo
totalmente ilesos.

\textbf{Hanucá:} Festa de luzes que se celebra na noite de 25 de
Kislev, comemorando a vitória dos macabeus, cujo símbolo é o candelabro
de oito braços.

\textbf{Haran:} Local em que nasceu o patriarca Abrahão.

\textbf{Hassid:} Caridoso.

\textbf{Helek:} Parte (divisão).

\textbf{Hen:} Graça.

\textbf{Hená:} As mesmas.

\textbf{Heresh:} Surdo.

\textbf{Hezekiel:} Profeta.

\textbf{Hilchot:} Plural de Halachá.

\textbf{Hilchot Rambam:} Preceitos Maimonídicos.

\textbf{Hin:} Medida para líquidos.

\textbf{Hishamer:} Alertar-se.

\textbf{Hober:} Feiticeiro.

\textbf{Hober haber:} Feiticeiro praticante (de grau um pouco mais alto).

\textbf{Horayot:} Capítulo de um dos tratados do Talmud.

\textbf{Hulin:} Ser profano, ou seja, prometer cumprir uma promessa e
não fazê-lo. Também nome de capítulo de um dos tratados do Talmud.

\textbf{Issi (ben Yiehudá):} Grande comentarista talmúdico.

\textbf{Issur bevad ehad:} Proibição de uma só vez.

\textbf{Issur colel:} Proibição geral.

\textbf{Issur mossif:} Proibição a acrescentar.

\textbf{Itamar:} Filho de Aharão, o Sumo Sacerdote.

\textbf{Iyar:} O segundo mês do ano judaico.

\textbf{Kadashim:} Tratado completo do Talmud.

\textbf{Kal vahomer:} Com toda razão.

\textbf{Kalam (os mestres do):} Os eruditos.

\textbf{Kedoshim:} Trecho de leitura semanal do 3º livro
do Pentateuco, chamado Levítico.

\textbf{Kehat:} Filho de Levy (neto do patriarca Jacob).

\textbf{Kenaz (Otniel ben Kenaz):} Grande sábio da época de Josué, em
1272 antes da era comum.

\textbf{Kidush:} Santificação. Bênção que é pronunciada sobre um copo
de vinho no sábado ou em dia festivo.

\textbf{Kidushin:} Capítulo de um dos tratados do Talmud.

\textbf{Kinim:} Capítulo de um dos tratados do Talmud.

\textbf{Kipurim:} Capítulo de um dos tratados do Talmud.

\textbf{Kiriat Sefer:} Metrópole de livros (no caso, comparando Otniel
ben Kenaz que, de tão sábio, dominava até uma metrópole de livros).

\textbf{Kiyemu ve lo kiyemu:} Cumpriram e não cumpriram.

\textbf{Koptim:} Povo egípcio que costumava tatuar o próprio corpo.

\textbf{Korah:} Primo em primeiro grau de nosso mestre Moisés.

\textbf{Kossem:} Pessoa que adivinha o futuro por meio de magia.

\textbf{Kossem Kessamim:} Pessoa que pratica bruxaria.

\textbf{Lav shebikhlalut:} O não total.

\textbf{Levi:} Filho de Jacob, o patriarca.

\textbf{Levita:} Da tribo de Levi.

\textbf{Ló:} Não.

\textbf{Ló tehonem:} Não ter piedade.

\textbf{Ló tefaer:} Não adornar.

\textbf{Ló tishacheb:} Não deitará (emprego do verbo no futuro).

\textbf{Ló tishcab:} Não deitará (emprego do verbo no imperativo).

\textbf{Ló titgodedu:} Não se tatuarão.

\textbf{Log:} Medida líquida equivalente a 506 cm\textsuperscript{3} ou 0,23 kg.

\textbf{Lulav:} Uma das quatro espécies de palmeira usadas na festa de
Sucot (Festa das Cabanas).

\textbf{Maasser Sheni:} Segundo dízimo (dízimo do dízimo) dado no
Templo Sagrado.

\textbf{Maasserot:} Dízimos.

\textbf{Macat mardut:} Bater rebeldemente.

\textbf{Macot:} Capítulo de um dos tratados do Talmud.

\textbf{Madiá:} Aquele que incentiva a praticar o mal.

\textbf{Makhshirin:} Capítulo de um dos tratados do Talmud.

\textbf{Mamzer:} Bastardo.

\textbf{Maneh:} Nome popular de antiga moeda usada na época do Talmud.

\textbf{Marbit:} Usura.

\textbf{Mashkin:} Capítulo de um dos tratados do Talmud.

\textbf{Mashuah Mil-Hama (ou Meshuah Mil-Hama):} O comandante chefe de
uma batalha, que também tem a responsabilidade de preparar sua equipe
psicologicamente e de recusar os que não estiverem preparados para a
luta.

\textbf{Mashukh:} Alguém que teve seu prepúcio puxado para a
frente a fim de cancelar o sinal do pacto de Abraham.

\textbf{Matzah:} Espécie de pão sem fermento, usado apenas na Páscoa, em lugar do pão
comum.

\textbf{Matzebá:} Lápide que se coloca sobre o túmulo.

\textbf{Meguilá:} Relato de um acontecimento verídico.

\textbf{Meilá:} Desfalque.

\textbf{Mekhashef:} Feiticeiro.

\textbf{Mekhiltá:} Compêndio de regras rabínicas relativas ao Êxodo, o
segundo livro do Pentateuco (termo aramaico).

\textbf{Meliká:} A maneira como se devia degolar a ave, na época do Templo Sagrado.

\textbf{Menahesh:} Adivinho.

\textbf{Menahot:} Capítulo de um dos tratados do Talmud.

\textbf{Meonen:} Feiticeiro.

\textbf{Meribá:} Briga.

\textbf{Merkulis:} Deus dos negócios dos romanos.

\textbf{Meshichá:} Recibo ou comprovante de uma transação comercial.

\textbf{Messit:} Incitante.

\textbf{Mezuzá:} Prece protetora que se coloca nos umbrais das portas,
do lado direito.

\textbf{Midot:} Capítulo de um dos tratados do Talmud.

\textbf{Midrashot ou Midrashim (plural de Midrash):} Palestras,
conferências. Também nome da compilação de comentários bíblicos feitos
durante palestras dos mestres a seus alunos.

\textbf{Mikvá:} Reservatório de água para banho ritual.

\textbf{Mikvaot:} Plural de Mikvá.

\textbf{Mishael:} Um dos três sábios que santificaram o nome de Deus na
época de Nabucodonosor penetrando na caldeira de fogo e saindo
totalmente ilesos.

\textbf{Mishná:} Primeira parte do Talmud.

\textbf{Mishpatim:} Processos.

\textbf{Mishrat anabim:} Líquido derivado da uva (vinho).

\textbf{Mitzvot:} Preceitos.

\textbf{Moab:} Irmão de Amon, filho de Lot e sobrinho do patriarca
Abraham (vide ``Amon'').

\textbf{Moabita:} Povo descendente de Moab.

\textbf{Moed Catan:} Pequena festa ou comemoração.

\textbf{Molekh:} Ídolo do povo chamado Amon que costumava adorá-lo por
meio do fogo, ofertando-lhe os próprios filhos.

\textbf{Monte Moriá (Har Hamoriá):} Local onde o patriarca Isaac foi
levado ao altar do sacrifício, pelo próprio pai.

\textbf{Nabot:} Sábio da época de Reis que foi morto pelo rei Ah-Ab que
queria apossar-se de sua propriedade.

\textbf{Nassi:} Presidente.

\textbf{Nazir:} Asceta.

\textbf{Nazirim:} Plural de Nazir.

\textbf{Nebelá:} Impureza.

\textbf{Nedarim:} Capítulo de um dos tratados do Talmud.

\textbf{Negaim:} Capítulo de um dos tratados do Talmud.

\textbf{Neshekh:} Usura.

\textbf{Nessiim:} Presidentes.

\textbf{Nezikin:} Tratado completo do Talmud.

\textbf{Nidá:} Capítulo de um dos tratados do Talmud.

\textbf{Nikebu:} Insultaram.

\textbf{Nissan:} Primeiro mês do ano judaico.

\textbf{Noahid:} Descendente de Noé.

\textbf{Nokeb:} Insultar.

\textbf{Notar:} Sobras.

\textbf{Ob:} Feitiçaria na qual se evocam os mortos para fazer-lhes
perguntas e saber o futuro.

\textbf{Ohalot:} Capítulo de um dos tratados do Talmud.

\textbf{Okatzin:} Capítulo de um dos tratados do Talmud.

\textbf{Olelot:} Pequenos cachos (de uva) em formação.

\textbf{Omer:} Espécie de medida de cevada nova, recém-colhida, a qual
era oferecida no Templo no 2º dia de Páscoa.

\textbf{Oná:} Período.

\textbf{Onen:} Entristecido.

\textbf{Onatá:} Período amoroso.

\textbf{Onkelos:} Grande comentarista Bíblico.

\textbf{Orlá:} Nome que se dá ao fruto de uma árvore antes que ela complete três anos.

\textbf{Otniel (ben Kenaz):} Grande sábio da época de Josué em 1272 antes da era comum.

\textbf{Pará:} Capítulo de um dos tratados do Talmud.

\textbf{Parasanga ou Parsá:} Medida métrica equivalente a 3.840 metros.

\textbf{Patriarca Jacob:} O terceiro patriarca Jacob.

\textbf{Peá:} Sobras abandonadas nos campos para os pobres.

\textbf{Pen:} Para que.

\textbf{Pen tikdash:} Para que te santifiques.

\textbf{Pen tukad esh:} Para que a chama fique acesa (no Templo Sagrado).

\textbf{Pen yi-yeh:} Para que seja.

\textbf{Peor:} Idolo do povo moabita.

\textbf{Perutá:} Nome da moeda israelense anterior à atual.

\textbf{Pessahin:} Capítulo de um dos tratados do Talmud.

\textbf{Pessah:} Páscoa, em hebraico.

\textbf{Pigul:} Nome dado ao sacrifício que era ofertado sem total
intenção, durante a época do Templo.

\textbf{Pinhas:} Neto de Aharão, o Sumo Sacerdote.

\textbf{Pitom:} Nome de um antigo feiticeiro egípcio.

\textbf{Portão de Nicanor -- Shaar Nikanor:} Portão do Templo sagrado
oferecido por Nicanor, um dos homens mais ricos do Egito no último
século antes da era comum, razão pela qual tem seu próprio nome.

\textbf{Quelim:} Capítulo de um dos tratados do Talmud.

\textbf{Quemosh:} Nome de um ídolo.

\textbf{Queretot:} Capítulo de um dos tratados do Talmud.

\textbf{Quessutá:} A vestimenta dela.

\textbf{Quetubot:} Capítulo de um dos tratados do Talmud.

\textbf{Quil-aim:} A mistura, proibida pela Torah, de duas espécies
distintas, como, por exemplo, lã e linho, cavalo e mula etc. Também
nome de capítulo de um dos tratados do Talmud.

\textbf{Quil-ei ha querem:} Mistura do vinhedo.

\textbf{Quil-ei zeraim:} Mistura de plantações, proibida pela Torah,
como, por exemplo, de bananeira com macieira etc.

\textbf{Rabá:} Membro do 4º ciclo dos amoraitas da Babilônia, de
meados do 4º século.

\textbf{Raban Shimeon ben Gamliel:} Membro do 1º ciclo dos tanaitas,
do fim do 1º século.

\textbf{Rabi Abin} ou \textbf{Rabi Ilai:} Membro do
3º ciclo dos amoraitas de Jerusalém, do início do 4º século.

\textbf{Rabi Akiba:} Membro do 3º ciclo dos tanaitas, em meados do 2º
século.

\textbf{Rabi Dossá:} Membro do 1º ciclo dos tanaitas, no 1º século.

\textbf{Rabi Eliezer:} Membro do 4º ciclo dos tanaitas, no fim do 2º século.

\textbf{Rabi Eliezer ben Jacob:} Membro do 1º ciclo dos tanaitas, no 1º século.

\textbf{Rabi Hananya ben Akabya:} Membro do 4º ciclo dos tanaitas do fim do 1º século.

\textbf{Rabi Haniná:} Membro do 1º ciclo dos emoraitas (geração de eruditos que vieram após os tanaitas), no 3º século.

\textbf{Rabi Hisdá:} Membro do 2º ciclo dos amoraitas da Babilônia, no fim do 3º século.

\textbf{Rabi Hiya ben Abun:} Membro do 3º ciclo dos amoritas de
Jerusalém, no princípio do 4º século.

\textbf{Rabi Lanai:} Membro do 1º ciclo dos amoraitas de Jerusalém, em meados do 1º século.

\textbf{Rabi Ilai:} Vide Rabi Abin.

\textbf{Rabi Isaac ben Abdimei:} Grande comentarista talmúdico do
início do 5º século.

\textbf{Rabi Ishmael:} Membro do 3º ciclo dos tanaitas, em meados do 2º
século.

\textbf{Rabi Meir:} Tanaita de grande gabarito do 2º século. É muito
conhecido, inclusive nos dias de hoje, pelos seus milagres.

\textbf{Rabi Nathan:} Membro do 4º ciclo dos tanaitas, do fim do 2º
século.

\textbf{Rabi Shimeon ben Gamliel:} Membro do 1º ciclo dos
tanaitas, do fim do 1º século.

\textbf{Rabi Shimeon ben Lakish} ou \textbf{Resh Lakish:} Membro do 2º
ciclo dos amoraitas de Jerusalém, do fim do 3º século.

\textbf{Rabi Yehoshuá ben Hananya:} Membro do 2º ciclo dos tanaitas, do
início do 2º século.

\textbf{Rabi Yehudá:} Compilou a Mishná no fim do 2º século.

\textbf{Rabi Yehudá ben Betera:} Membro do 1º ciclo dos tanaitas, do
fim do 1º século.

\textbf{Rabi Yohanan:} Membro do 2º ciclo dos amoraitas de Jerusalém,
do fim do 3º século.

\textbf{Rabi Yohana ben Gudgoda:} Membro do 2º ciclo dos tanaitas, do
início do 2º século.

\textbf{Rabi Yossi ben Hanina:} Membro do 2º ciclo dos emoraitas, do fim do 3º século.

\textbf{Rabi Yossi ben Yehudá:} Membro do 5º ciclo dos tanaitas, do início do
2º século.

\textbf{Rabi Yossi Hagalili:} Membro do 3º ciclo dos tanaitas, em meados do 2º século.

\textbf{Rabi Yoshiá:} Membro do 4º ciclo dos tanaitas, do fim do 2º século.

\textbf{Rabiná:} Rabino (do aramaico).

\textbf{Rachil:} Caluniador.

\textbf{Rashá:} Ímpio.

\textbf{Rav:} Rabino.

\textbf{Rebiit:} A 4ª parte de um cálice de vinho.

\textbf{Resh Lakish:} Vide Rabi Shimeon ben Lakish.

\textbf{Ribit:} Usura.

\textbf{Ribit Ketsutsa:} Usura reduzida.

\textbf{Rosh Hashaná:} Festa do ano-novo judaico.

\textbf{Sanhedrin:} Capítulo de um dos tratados do Talmud.

\textbf{Sefer Hamitzvot:} Livro dos preceitos.

\textbf{Selaim (Plural de ``sela''):} Espécie de moeda antiga.

\textbf{Seret:} Tatuar-se.

\textbf{Seritá:} Tatuagem.

\textbf{Shaatnez:} A mistura de lã com linho, que é proibida pela
Torah.

\textbf{Shabatot:} Plural de Shabat.

\textbf{Shabuot:} Festa de Pentecostes ou também festa do recebimento
da Torah. É também o nome de um capítulo de um dos tratados do Talmud.

\textbf{Shebiit:} Não trabalhar a sua terra no 7º ano. Também nome de
um capítulo de um dos tratados do Talmud.

\textbf{Shebuat bitui:} Jurar cumprir e não cumprir.

\textbf{Shebuat shav:} Jurar em vão.

\textbf{Shebuat sheker:} Jurar pela mentira.

\textbf{Shebuot:} Capítulo de um dos tratados do Talmud.

\textbf{Sheerá:} O sustento.

\textbf{Shehitá:} Ato de abater a ave ou o animal segundo os preceitos
da Torah.

\textbf{Shekalim:} Capítulo de um dos tratados do Talmud.

\textbf{Shekel:} Moeda de prata.

\textbf{Shekhiná:} Divindade.

\textbf{Shemá:} Principal oração da religião judaica.

\textbf{Shemini Atzeret:} Oitavo dia de Sucot (Festa das Cabanas).

\textbf{Shemoná Sheratsim:} Capítulo de um dos tratados do Talmud.

\textbf{Sheniyot:} Preceitos rabínicos talmúdicos.

\textbf{Shitim:} Local onde o povo judeu acampou quando saiu do Egito e
atual fronteira jordaniana.

\textbf{Shoel Adam (Mehaberó):} Capítulo de um dos tratados do Talmud.

\textbf{Shoel ob:} Consultar-se com um feiticeiro.

\textbf{Shofar:} Corneta feita de chifre de carneiro que costuma ser
tocada no ano novo judaico.

\textbf{Sido:} Tipo de moeda antiga.

\textbf{Sidrá:} Porção semanal do Pentateuco lida aos sábados.

\textbf{Sidrá tsáv:} Uma das porções semanais que começa o nome Tsáv.

\textbf{Sifrá:} Obra antiga que comenta preceitos rabínicos relativos
ao 3º livro do Pentateuco, o Levítico, escrita por Rabi Yehudá Ilai Z.
L. no 2º século.

\textbf{Sifrei:} Estilo exclusivo literário pelo qual
foi transmitida a Torah sagrada ao nosso mestre Moisés.

\textbf{Sotá:} Capítulo de um dos tratados do Talmud.

\textbf{Sucá:} Cabana coberta com ramos. Também nome de um capítulo de
um dos tratados do Talmud.

\textbf{Sucot:} Festa das Cabanas.

\textbf{Taalé:} Fará subir.

\textbf{Taaniot:} Parágrafo de um dos tratados do Talmud.

\textbf{Taanit:} Capítulo de um dos tratados do Talmud.

\textbf{Taassé:} Fará.

\textbf{Talmud:} Obra composta pela Mishná e pela Guemará.

\textbf{Talmud Torah:} Estudo da Torah.

\textbf{Tamid:} Sempre. É também o nome de um dos tratados da Guemará
que fala sobre a proibição de se plantarem árvores no Templo para
embelezá-lo.

\textbf{Taná (ou Tanaita):} Palavra que vem do aramaico e que significa
professor. O Talmud emprega esse termo para os doutores da lei que se
empenharam de corpo e alma para que a Mishná fosse escrita e
posteriormente impressa.

\textbf{Taná Kamá:} Palavra aramaica que significa o primeiro tanaita,
ou seja, o primeiro que elaborou a primeira lei de uma determinada
parte da Mishná.

\textbf{Tanaim:} Tanaita ou Taná.

\textbf{Tarbit:} Usura.

\textbf{Targum:} Tradução explicativa, com comentários.

\textbf{Tazria:} Trecho de leitura semanal do 3º livro do Pentateuco, o Levítico.

\textbf{Tebel:} Palavra aramaica, que significa algo
impróprio para ser ingerido.

\textbf{Tebul Yom:} Banho ritual diário.

\textbf{Tefilin:} Filactérios vusados diariamente nas preces matinais.

\textbf{Teharot:} Capítulo de um dos tratados do Talmud.

\textbf{Tehorot:} Tratado completo do Talmud.

\textbf{Telussin:} Espécie de joia usada antigamente pelos soldados.

\textbf{Templo Monte (Har Habait):} O local onde foi construído o Templo.

\textbf{Temurá:} Capítulo de um dos tratados do Talmud.

\textbf{Tenahashu:} Praticar a bruxaria.

\textbf{Teonenu:} A prática de prever o futuro por meio de bruxaria.

\textbf{Terefá:} Impróprio para ser ingerido.

\textbf{Terra:} Terra de Israel (Erets Israel).

\textbf{Terumá:} Contribuição oferecida pelo povo aos sacerdotes do Templo.

\textbf{Terumot:} Plural de Terumá. Também o nome de um capítulo de um tratado do Talmud.

\textbf{Tigzol:} Saquear.

\textbf{Tishri:} 7º mês do calendário judeu.

\textbf{Torah:} Pentateuco.

\textbf{Tossafot:} Suplementos de comentários do Talmud que surgiram
bem depois dos comentários, daí o nome de suplementos.

\textbf{Tosseftá}: Palavra aramaica que significa suplementos da Mishná
elaborados pelos tanaitas, de onde o nome Tosseftá.

\textbf{Toshab vesachir:} Habitante estranho (não circuncisado) contratado (novo).

\textbf{Tsedaká:} Caridade.

\textbf{Tsitsit:} Espécie de franjas do xale usado nas preces.

\textbf{Tson:} Pequeno gado.

\textbf{Tzav:} Trecho de leitura semanal do 3º livro do Pentateuco, o
Levítico, que tem como título o próprio nome ``Tzav''.

\textbf{Tzion:} Israel.

\textbf{Ushmartem:} E vocês observarão.

\textbf{Uziah:} Rei de Jerusalém entre 645 e 707 antes da era comum que penetrou no local
sagrado do Templo, onde era proibido entrar. Como castigo ele se tornou
leproso até o fim da vida.

\textbf{Vayehi Bayom Hashemini:} Trecho de leitura semanal do 3º livro do Pentateuco, o Levítico, que começa com essas palavras.

\textbf{Vayikrá:} Trecho de leitura semanal do 3º livro do Pentateuco, o Levítico.

\textbf{Vayigzol:} E extorquiu.

\textbf{Vayitgodedu:} Tatuaram-se.

\textbf{Veshameru:} E observarão.

\textbf{Yadayim:} Capítulo de um dos tratados do Talmud.

\textbf{Yahel:} Profanar.

\textbf{Yain nessech:} Vinho impróprio para ser consumido em ritual religioso judaico.

\textbf{Yarimu:} Refere-se à separação dos donativos.

\textbf{Yebamot:} Capítulo de um dos tratados do Talmud.

\textbf{Yehudá:} Uma das doze tribos.

\textbf{Yideoni:} Que pratica bruxaria.

\textbf{Yidoa:} Este nome é dado a um determinado osso existente nas aves com o qual se praticava a bruxaria.

\textbf{Yom Kipur:} Dia do perdão.

\textbf{Yom Tob:} Capítulo de um dos tratados do Talmud.

\textbf{Yoma:} Capítulo de um dos tratados do Talmud.

\textbf{Zab:} Pessoa doente que tem propensão a expelir o próprio
sêmen sem autocontrole.

\textbf{Zaba:} Feminino de Zab.

\textbf{Zabim:} Capítulo de um dos tratados do Talmud.

\textbf{Zar:} Estranho que não seja descendente da família de Aarão, o
Sumo Sacerdote.

\textbf{Zebahim:} Capítulo de um dos tratados do Talmud.

\textbf{Zeraim:} Tratado completo do Talmud.

\textbf{Zimá:} Depravação.

\textbf{Zimri:} Representante chefe da tribo de Shimeon.

\textbf{Zoná:} Prostituta.

\textbf{Zot Tih-yé:} Trecho da leitura semanal do 3º livro do
Pentateuco, o Levítico, que começa com esse título.
}
\end{hangparas}

