\setcounter{secnumdepth}{-2}

\chapter{Glossário}

\begin{hangparas}{.25in}{1}
{\small

\textbf{Abaié:} Famoso comentarista da Mishná de origem babilônica, do
século \textsc{iv}.

\textbf{Ach:} Literalmente \emph{mas}.

\textbf{Acharei Mot:} Trecho de leitura semanal do terceiro livro da Torá, chamado Levítico.

\textbf{Agudot:} Aglomerações de pessoas.

\textbf{Ahav:} Sétimo rei do reino de Israel, filho de Omri e marido de Jezebel, que matou o sábio Nabot para apossar"-se de sua propriedade (ver \emph{Navot}).

\textbf{Amram:} Literalmente \emph{amigo do Altíssimo}. É o pai de Moisés.

\textbf{Al:} Expressão de negação.

\textbf{Amá:} Medida cúbita equivalente a quarenta e oito centímetros.

\textbf{Amalec:} Neto de Esaú. Herdou de seu avô o ódio mortal que
este tinha por seu irmão gêmeo, Jacob, e por todos os seus
descendentes.

\textbf{Amon:} Irmão de Moab, ambos frutos da união de Lot (sobrinho do
patriarca Abraham) com suas duas filhas, os três únicos sobreviventes
de Sodoma e Gomorra. Pensando ter sido o mundo destruído e não haver
mais nenhum homem sobre a terra, as filhas de Lot embebedaram o próprio
pai para se unirem a ele. A filha mais velha concebeu um
filho, a quem nomeou Moab, e a mais nova teve outro filho, que
chamou Amon. Ao sair do Egito e atravessar o deserto para atingir a
Terra Prometida, o povo judeu precisava passar por muitos povoados.
Amon e Moab proibiram terminantemente a passagem dos judeus por suas
cidades, e chegaram a contratar o maior mago da terra, Bil"-An, para
amaldiçoá"-los (ver Números, capítulo 22). Diante de tal
atitude, proibiu"-se aos judeus que se casassem com os integrantes
daqueles povos.

\textbf{Amonita:} Povo descendente de Amon.

\textbf{Amorita:} Um dos povos que vivia na terra de Canaã.

\textbf{Arachin:} Capítulo de um dos tratados do Talmud.

\textbf{Asherá:} Deusa da fertilidade do povo de Canaã, conhecida por sua força e cuidado. Seu nome por vezes foi traduzido como \emph{árvore sagrada}. Há relatos de que essa árvore foi cortada e queimada fora do Templo, numa atitude de governantes que tentavam \emph{purificar} o culto e dedicar"-se à adoração exclusiva de Deus.

\textbf{Atzeret:} Nome dado ao sétimo dia de Pessach e ao oitavo dia de Sucót.

\textbf{Avodá Zará:} Literalmente \emph{adoração estrangeira}, idolatria. 
É o nome de um dos tratados talmúdicos de Nezikin.

\textbf{Avon:} Pecado.

\textbf{Azhará:} Alertar"-se.

\textbf{Azariá:} Um dos três sábios que santificaram o nome de Deus na
época de Nabucodonosor, penetrando na caldeira de fogo e saindo
totalmente ilesos.

\textbf{Bava Batra:} Literalmente \emph{Último portão}. É um capítulo do tratado talmúdico de Nezikin.

\textbf{Bava Kama:} Literalmente \emph{Primeiro portão}. Também integra Nezikin.

\textbf{Bava Metzia:} Literalmente \emph{Portão do meio}. Também integra Nezikin.

\textbf{Bakar:} Gado.

\textbf{Baal:} Nome dado a um ídolo dos povos do Levante.

\textbf{Bechorot:} Literalmente \emph{primogênitos}. Também é o nome de um capítulo de um dos tratados do Talmud.

\textbf{Belinta:} Esteira.

\textbf{Brachot:} Literalmente \emph{bênçãos}. Também nome de um capítulo de um dos
tratados do Talmud.

\textbf{Beit Hashoevá:} Celebração que acontece durante a segunda noite da festa de Sucót. Na época do Templo Sagrado, era usada uma fonte d'água para a o serviço de \emph{Nisuch hamáim}, literalmente \emph{derramamento da água}.

\textbf{Betzá:} Capítulo de um dos tratados do Talmud.

\textbf{Bicurim:} Literalmente \emph{primícias}. Também é o nome de um capítulo de um dos tratados do Talmud.

\textbf{Bitlú velo bitlú:} Literalmente \emph{não cumpriu} e \emph{cumpriu}.

\textbf{Bitrumá:} Contribuição espontânea entregue ao sacerdote na
época do Templo.

\textbf{Caleb:} Representante chefe da tribo de Judá. Cada uma das 12
tribos tinha o seu representante chefe.

\textbf{Casher:} O que é permitido em matéria de alimento, segundo a
religião judaica.

\textbf{Chaguigá:} Festividade.

\textbf{Chalá:} Nome dado ao pão trançado servido no shabat e dias de festa.

\textbf{Chametz:} É a comida feita de grãos e água. São alimentos 
essencialmente fermentados, proibidos durante a festa 
de Pessach (ver \emph{Pessach}).

\textbf{Chanucá:} Também conhecida como Festa das Luzes, é celebrada 
na noite de 25 de Kislev. Comemora a vitória dos macabeus, 
cujo símbolo é o candelabro de oito braços.

\textbf{Chassid:} Piedoso.

\textbf{Cheresh:} Surdo.

\textbf{Cohanim:} Plural de Cohen.

\textbf{Cohanim Gdolim:} Plural de Cohen Gadol.

\textbf{Cohen:} Descendente da família sacerdotal.

\textbf{Cohen Gadol:} Sumo sacerdote no Templo Sagrado.

\textbf{Col:} Literalmente \emph{cada}.

\textbf{Cutá:} Mistura de farinha fermentada com leite.

\textbf{Daián:} Juiz rabínico.

\textbf{Demai:} Capítulo de um dos tratados do Talmud.

\textbf{Derash:} Literalmente \emph{interpretar}. É um método de explicação 
das escrituras sagradas através de comentários.

\textbf{Doresh el hametim:} Evocação dos mortos a fim de descobrir o futuro. 
Segundo a Guemará de Sanhedrin, a expressão \emph{Doresh el hametim} diz respeito 
a ``aquele que se deixa morrer de fome e passa a noite num cemitério para
que o espírito de um demônio possa descansar nele''.

\textbf{Eval:} Nome de um monte sobre o qual metade das tribos recebeu uma parte 
das leis da Torá (ver \emph{Guerizim}).

\textbf{Ed sheker ou Ed shav:} Literalmente \emph{falso testemunho}.

\textbf{Edumeu ou Adomi:} O nome Edom, que significa vermelho, é atribuído a 
Esaú, descendente do patriarca Isaac e irmão gêmeo de Jacob. Edumeu ou adomi 
é a maneira como são conhecidos os seus descendentes.

\textbf{Eduiót:} Capítulo de um dos tratados do Talmud.

\textbf{Efa:} Medida equivalente a 36,44 litros cúbicos.

\textbf{Efod:} Espécie de manto curto usado pelo sumo sacerdote na
época do Templo Sagrado.

\textbf{Eglá Arufá:} Bezerra degolada.

\textbf{Eleazar:} Filho de Aharon, o sumo sacerdote.

\textbf{Elohim:} Um dos nomes atribuídos a Deus.

\textbf{Eruv:} Ritual haláchico (ver \emph{Halachá}) que cria uma área cercada 
através de fios em postes da cidade e outras barreiras naturais, a fim de ser 
considerada propriedade privada. Isso possibilita que moradores possam circular 
pela área carregando coisas essenciais para seus afazeres sem incorrer em 
proibições ligadas ao shabat, que não permite transportar objetos de lugares 
privados a públicos e vice versa.

\textbf{Eruvin:} Capítulo de um dos tratados do Talmud.

\textbf{Etrog:} Fruta cítrica, espécie de cidra amarela, que faz parte das 
quatro espécies usadas na festa de Sucót. As outras são lulav, hadás e aravá.

\textbf{Exilarca:} Em hebraico \emph{Rosh Galut}. Descendente da casa de Davi, 
representante geral e diretamente responsável por toda a coletividade
judaica em determinada cidade ou país do exílio (como a Pérsia e a Babilônia).

\textbf{Gaon:} Autoridade em academias talmúldicas medievais, aceito como líder espiritual da comunidade judaica.

\textbf{Gaonim:} Plural de Gaon.

\textbf{Guedidá:} Tatuagem.

\textbf{Guemará:} Continuação da Mishná, ambos juntos formam o Talmud. A Guemará 
contém os comentários e análises rabínicas da Mishná.

\textbf{Guer toshav:} Habitante estranho, que não é nativo do local. Também pode ser entendido por um \emph{prosélito residente}, ou seja, um gentio que não adora
ídolos e cumpre as sete leis de Noach, mas que não se converteu oficialmente ao judaísmo.

\textbf{Guerizim:} Nome de um monte sobre o qual metade das tribos recebeu uma 
parte das leis da Torá (ver \emph{Eval}).

\textbf{Guezerá:} Sentença decretada.

\textbf{Guezerá shavá:} Sentença decretada, mas comparada a outra semelhante. Uma \emph{expressão similar}, ou seja, uma analogia entre duas leis, estabelecida com base na congruência verbal dos textos das Escrituras.

\textbf{Guid hanashé:} Tendão encolhido sobre a junção da coxa, o nervo ciático.

\textbf{Guitin:} Capítulo de um dos tratados do Talmud.

\textbf{Havdalá:} Literalmente \emph{separação}. Cerimônia análoga à 
de recebimento do shabat na sexta"-feira à noite, a Havdalá é 
realizada ao anoitecer do sábado em alusão à depedida do shabat 
e início da nova semana.

\textbf{Halachá:} Literalmente \emph{caminho}. É conhecida como a 
Lei judaica, e seus preceitos dizem respeito a orientações, hábitos, 
costumes, modos de agir e práticas.

\textbf{Halachot Gdolot:} São as \emph{grandes regras}, e o nome de 
uma obra talmúdica.

\textbf{Halalá:} Nome que se dava a uma mulher viúva, separada do
marido, ou prostituta, que estavam terminantemente proibidas de desposar
o sumo sacerdote ou um simples sacerdote, durante a existência do Templo
Sagrado. Ou a filha de um sacerdote que se casa com alguém com quem não lhe é
permitido casar"-se.

\textbf{Halel:} Oração de louvor que se recita especialmente nas
festas e nos primeiros dias de cada mês.

\textbf{Halitzá:} Recusa de uma mulher sem filhos que acabou de
enviuvar a unir"-se ao irmão solteiro de seu marido.

\textbf{Hanaiá:} Pouso.

\textbf{Chananiá:} Um dos três sábios que santificaram o nome de Deus
na época de Nabucodonosor penetrando na caldeira de fogo e saindo
totalmente ilesos.

\textbf{Haran:} Local em que nasceu o patriarca Abraham.

\textbf{Helek:} Literalmente \emph{porção}. Também pode ser 
entendido como parte, divisão.

\textbf{Hen:} Graça.

\textbf{Hená:} As mesmas.

\textbf{Hezekiel:} Profeta.

\textbf{Hilchot:} Plural de Halachá.

\textbf{Hilchot Rambam:} Preceitos de Maimônides.

\textbf{Hin:} Medida para líquidos.

\textbf{Hishamer:} Alertar"-se, guardar"-se, manter"-se seguro.

\textbf{Hober:} Feiticeiro.

\textbf{Hober haber:} Feiticeiro praticante (de grau mais experiente).

\textbf{Horaiot:} Capítulo de um dos tratados do Talmud.

\textbf{Hulin:} Ser profano, ou seja, prometer cumprir uma promessa e
não fazê"-lo. Também é o nome de capítulo de um dos tratados do Talmud.

\textbf{Issi (ben Iehuda):} Grande comentarista talmúdico.

\textbf{Issur bevad echad:} Uma proibição simultânea, ou seja, mais de uma proibição que entra em vigor ao mesmo tempo.

\textbf{Issur colel:} Literalmente \emph{proibição geral}. Uma proibição inclusiva, ou seja, uma proibição que se aplica a mais tipos de coisas.

\textbf{Issur mossif:} Uma proibição adicional.
%, ou seja, uma proibição que se aplica a mais tipos de pessoas.

\textbf{Itamar:} Filho de Aharon, o sumo sacerdote.

\textbf{Iár:} O segundo mês do ano judaico.

\textbf{Kodashim:} Tratado completo do Talmud.

\textbf{Kal vachomer:} Literalmente \emph{brando e rigoroso}. Argumento do menor ao maior.

\textbf{Calám:} Os eruditos.

\textbf{Kedoshim:} Trecho de leitura semanal do terceiro livro
da Torá, chamado Levítico.

\textbf{Kehat:} Filho de Levy (neto do patriarca Jacob).

\textbf{Kenaz:} Sobrenome de Otniel ben Kenaz, grande sábio da época de Josué, de
1272 a.C.

\textbf{Kidush:} Literalmente \emph{santificação}. É também o nome da bênção 
pronunciada sobre um copo de vinho no shabat ou outros dias festivos.

\textbf{Kidushin:} Capítulo de um dos tratados do Talmud.

\textbf{Kinim:} Capítulo de um dos tratados do Talmud.

\textbf{Kipurim:} Capítulo de um dos tratados do Talmud.

\textbf{Kiriat sêfer:} Literalmente \emph{metrópole de livros}. No caso, 
uma comparação a Otniel ben Kenaz que, de tão sábio, dominava até 
uma \emph{metrópole de livros}.

\textbf{Kiemú ve lo kiemú:} Literalmente \emph{cumpriram e não cumpriram}.

\textbf{Koptim:} Povo egípcio que costumava tatuar o próprio corpo.

\textbf{Kórach:} Primo em primeiro grau de nosso mestre Moisés.

\textbf{Kossem:} Quem adivinha o futuro por meio de magia.

\textbf{Kossem kessamim:} Quem pratica bruxaria.

\textbf{Lav shebichlalut:} O não total, uma reprimenda negativa geral.

\textbf{Levi:} Filho de Jacob, o patriarca.

\textbf{Levita:} Da tribo de Levi.

\textbf{Lo:} Literalmente \emph{não}.

\textbf{Lo techonem:} Não ter piedade.

\textbf{Lo tefaer:} Não adornar.

\textbf{Lo tishachev:} Não deitará (verbo no futuro).

\textbf{Lo tishcav:} Não deitar (verbo no imperativo).

\textbf{Lo titgodedu:} Não se tatuarão.

\textbf{Log:} Medida líquida equivalente a 506 centímetros cúbicos ou 0,23 quilos. O ``revi'it'' equivale a um quarto de um ``log''.

\textbf{Lulav:} Uma das quatro espécies de palmeira usadas na festa de
Sucót (ver \emph{Etrog}).

\textbf{Maasser sheni:} Segundo dízimo, o \emph{dízimo do dízimo}, dado no
Templo Sagrado.

\textbf{Maasserot:} Dízimos.

\textbf{Macat mardut:} Bater rebeldemente.

\textbf{Macot:} Capítulo de um dos tratados do Talmud.

\textbf{Madiá:} Aquele que incentiva a praticar o mal.

\textbf{Machshirin:} Capítulo de um dos tratados do Talmud.

\textbf{Mamzer:} Bastardo.

\textbf{Manê:} Nome popular de antiga moeda usada na época do Talmud. Equivalente a atuais cem \emph{shekalim}.

\textbf{Marbit:} Usura.

\textbf{Mashkin:} Capítulo de um dos tratados do Talmud.

\textbf{Mashua milchamá:} Comandante que é chefe de
batalha. Também responsável por preparar sua equipe
psicologicamente e recusar os que não estiverem preparados para a
luta.

\textbf{Mashuch:} Alguém que teve seu prepúcio puxado para a
frente a fim de cancelar o sinal do pacto de Abraham (da circuncisão).

\textbf{Matzá:} Espécie de pão sem fermento servido em Pessach, no lugar do pão
comum (ver \emph{Chametz}).

\textbf{Matzebá:} Lápide que se coloca sobre o túmulo.

\textbf{Meguilá:} Literalmente \emph{rolo}. É um tratado completo do Talmud.

\textbf{Meilá:} Desfalque.

\textbf{Mechashef:} Feiticeiro.

\textbf{Mechiltá:} Termo aramaico. É o compêndio de regras rabínicas relativas ao Êxodo, o
segundo livro da Torá.

\textbf{Meliká:} A maneira como se devia sacrificar a ave na época do Templo Sagrado, em que o sacerdote se utilizava da unha para cortar a nuca, e a laringe e o esôfago do pássaro.

\textbf{Menachesh:} Adivinho.

\textbf{Menachot:} Capítulo de um dos tratados do Talmud.

\textbf{Meonen:} Feiticeiro.

\textbf{Merivá:} Briga.

\textbf{Merkulis:} Deus dos negócios dos romanos, Mercúrio.

\textbf{Meshichá:} Recibo ou comprovante de uma transação comercial.

\textbf{Messit:} Incitante.

\textbf{Mezuzá:} Prece protetora que se coloca nos umbrais das portas,
do lado direito.

\textbf{Midot:} Capítulo de um dos tratados do Talmud.

\textbf{Midrashot ou midrashim:} Literalmente \emph{palestras}, ou 
\emph{conferências}. 
Também nome da compilação de comentários bíblicos feitos
durante palestras dos mestres a seus alunos. Plural de \emph{midrash}.

\textbf{Mikvá:} Reservatório de água para banho ritual.

\textbf{Mikvaot:} Plural de Mikvá. Também é um nome de um tratado do Talmud.

\textbf{Mishael:} Um dos três sábios que santificaram o nome de Deus na
época de Nabucodonosor, penetrando na caldeira de fogo e saindo
totalmente ilesos.

\textbf{Mishná:} Primeira parte do Talmud, seguida da Guemará 
(ver \emph{Guemará}). A Mishná
é a redação da tradição oral judaica, também conhecida como a \emph{Torá Oral}.

\textbf{Mishpatim:} Literalmente \emph{leis}. É também o nome de um trecho de 
leitura semanal do segundo livro da Torá, chamado Êxodo.

\textbf{Mishrat anavim:} Líquido derivado da uva, o vinho.

\textbf{Mitzvot:} Preceitos.

\textbf{Moab:} Irmão de Amon, filho de Lot e sobrinho do patriarca
Abraham (ver \emph{Amon}).

\textbf{Moabita:} Povo descendente de Moab (ver \emph{Amon} e \emph{Moab}).

\textbf{Moed katan:} Pequena festa ou comemoração.

\textbf{Molech:} Ídolo dos amonitas. Amon e seu povo costumavam adorá"-lo por
meio do fogo, ofertando"-lhe os próprios filhos.

\textbf{Monte Moriá:} Em hebraico \emph{Har Hamoriá}. Local 
onde o patriarca Isaac foi
levado ao altar do sacrifício, pelo próprio pai.

\textbf{Navot:} Sábio da época de Reis, comerciante de vinho, morto pelo rei Ahav.

\textbf{Nassi:} Cargo de autoridade.

\textbf{Naguid:} Líder e o porta"-voz dos judeus egípcios,
nomeado pelo sultão, que representava a autoridade moral e política
de todas as comunidades judaicas no país.

\textbf{Nazir:} Asceta.

\textbf{Nazirim:} Plural de Nazir.

\textbf{Nevelá:} Impureza.

\textbf{Nedarim:} Capítulo de um dos tratados do Talmud.

\textbf{Negaim:} Capítulo de um dos tratados do Talmud.

\textbf{Neshech:} Usura.

\textbf{Nessi'im (plural de Nassi):} Cargo de autoridade.

\textbf{Nezikin:} Tratado completo do Talmud.

\textbf{Nidá:} Capítulo de um dos tratados do Talmud.

\textbf{Nikebu:} Insultaram.

\textbf{Nissan:} Primeiro mês do ano judaico.

\textbf{Noachid:} Descendente de Noach, significa os não
israelitas ou pagãos que, de acordo com a lei
judaica, são obrigados a obedecer aos sete preceitos seguintes:
(1) estabelecer tribunais de justiça; (2) não praticar idolatria;
(3) não blasfemar; (4) não cometer incesto; (5) não matar;
(6) não roubar; (7) e não comer carne retirada de animais enquanto vivos. Os noachidos que observam estes sete preceitos herdarão uma parte do Mundo Vindouro.

\textbf{Nokev:} Insultar.

\textbf{Notar:} Sobras.

\textbf{Ob:} Feitiçaria na qual se evocam os mortos para fazer"-lhes
perguntas e saber o futuro.

\textbf{Ohalot:} Capítulo de um dos tratados do Talmud.

\textbf{Okatzin:} Capítulo de um dos tratados do Talmud.

\textbf{Olelot:} Pequenos cachos de uva em formação.

\textbf{Omer:} Espécie de medida de cevada nova, recém"-colhida. 
Era oferecida no Templo no segundo dia de Pessach.

\textbf{Oná:} Período, tempo, época.

\textbf{Onen:} Entristecido, também pode significar \emph{luto}.

\textbf{Onatá:} Período de encontro amoroso.

\textbf{Onkelos:} Grande comentarista Bíblico.

\textbf{Orlá:} Nome que se dá ao fruto de uma árvore antes que 
ela complete três anos.

\textbf{Pará:} Capítulo de um dos tratados do Talmud.

\textbf{Parasanga ou parsá:} Medida métrica equivalente a 3.840 metros.

\textbf{Patriarca Jacob:} O terceiro patriarca do povo judeu.

\textbf{Peá:} Sobras abandonadas nos campos para os pobres.

\textbf{Pen tikdash:} Para que te santifiques.

\textbf{Pen tukad esh:} Para que a chama fique acesa (no Templo Sagrado). [para que não seja consumido pelo fogo]

\textbf{Pen i'ié:} Para que seja. [que não haja]

\textbf{Peor:} Ídolo do povo moabita.

\textbf{Perutá:} Nome de moeda antiga.

\textbf{Pessachim:} Capítulo de um dos tratados do Talmud.

\textbf{Pessach:} Também conhecida como Festa da Liberdade, celebra 
a libertação dos hebreus da escravidão no Egito. É uma das mais 
importantes festas dentro do judaísmo.

\textbf{Pigul:} Nome dado ao sacrifício que era ofertado sem total
intenção, durante a época do Templo.

\textbf{Pinchas:} Neto de Aharon, o sumo sacerdote.

\textbf{Pitom:} Nome de um antigo feiticeiro egípcio.

\textbf{Portão de Nicanor:} Em hebraico \emph{Shaar Nikanor}. 
Portão do templo sagrado oferecido por Nicanor, um dos homens mais ricos 
do Egito no último século a.C., razão pela qual 
leva seu próprio nome.

\textbf{Quelim:} Capítulo de um dos tratados do Talmud.

\textbf{Quemosh:} Nome de um ídolo moabita.

\textbf{Queritot:} Capítulo de um dos tratados do Talmud.

\textbf{Quessutá:} Literalmente \emph{a vestimenta dela}.

\textbf{Quetubot:} Capítulo de um dos tratados do Talmud.

\textbf{Quil'aim:} A mistura, proibida pela Torá, de duas espécies
distintas, como, por exemplo, lã e linho, cavalo e mula etc. Também
nome de capítulo de um dos tratados do Talmud.

\textbf{Quil'ei hakerem:} Um vinhedo onde se colocou, juntamente 
com as sementes de uvas, 
outras espécies de sementes, como a de trigo ou as de vegetais.

\textbf{Quil'ei zeraim:} Mistura de plantações, proibida pela Torá,
como, por exemplo, de bananeira com macieira, trigo com aveia.

\textbf{Rabá:} Membro do quarto ciclo dos amoraítas da Babilônia, de
meados do século \textsc{iv}.

\textbf{Raban Shimeon ben Gamliel:} Membro do primeiro ciclo dos tanaítas,
do fim do século \textsc{i}.

\textbf{Rabi Abin ou Ilai:} Membro do
terceiro ciclo dos amoraítas de Jerusalém, do início do século \textsc{iv}.

\textbf{Rabi Akiva:} Membro do terceiro ciclo dos tanaítas, em meados do 
século \textsc{ii}.

\textbf{Rabi Dossá:} Membro do primeiro ciclo dos tanaítas, no século \textsc{i}.

\textbf{Rabi Eliezer:} Membro do quarto ciclo dos tanaítas, 
no fim do século \textsc{ii}.

\textbf{Rabi Eliezer ben Jacob:} Membro do primeiro ciclo dos 
tanaítas, no século \textsc{i}.

\textbf{Rabi Chananiá ben Akabia:} Membro do quarto ciclo 
dos tanaítas do fim do século \textsc{i}.

\textbf{Rabi Chaniná:} Membro do primeiro ciclo dos emoraítas, 
a geração de eruditos que vieram após os tanaítas, no século \textsc{iii}.

\textbf{Rabi Chisdá:} Membro do segundo ciclo dos amoraítas da Babilônia, 
no fim do século \textsc{iii}.

\textbf{Rabi Hiá ben Abun:} Membro do terceiro ciclo dos amoraítas de
Jerusalém, no princípio do século \textsc{iv}.

\textbf{Rabi Lanai:} Membro do primeiro ciclo dos amoraítas de Jerusalém, 
em meados do século \textsc{i}.

\textbf{Rabi Isaac ben Abdimei:} Grande comentarista talmúdico do
início do século \textsc{v}.

\textbf{Rabi Ishmael:} Membro do terceiro ciclo dos tanaítas, em meados 
do século \textsc{ii}.

\textbf{Rabi Meir:} Tanaíta de grande importância do século \textsc{ii}. 
É conhecido até hoje por seus milagres.

\textbf{Rabi Nathan:} Membro do quarto ciclo dos tanaítas, 
do fim do século \textsc{ii}.

\textbf{Rabi Shimeon ben Gamliel:} Membro do primeiro ciclo dos
tanaítas, do fim do século \textsc{i}.

\textbf{Rabi Shimeon ben Lakish ou Resh Lakish:} Membro do segundo
ciclo dos amoraítas de Jerusalém, do fim do século \textsc{iii}.

\textbf{Rabi Iehoshua ben Chananiá:} Membro do segundo ciclo dos tanaítas, 
do início do século \textsc{ii}.

\textbf{Rabi Iehuda:} Compilou a Mishná no fim do século \textsc{ii}.

\textbf{Rabi Iehuda ben Betera:} Membro do primeiro ciclo dos tanaítas, do
fim do século \textsc{i}.

\textbf{Rabi Iohanan:} Membro do segundo ciclo dos amoraítas de Jerusalém,
do fim do século \textsc{iii}.

\textbf{Rabi Iohana ben Gudgoda:} Membro do segundo ciclo dos tanaítas, do
início do século \textsc{ii}.

\textbf{Rabi Iossi ben Hanina:} Membro do segundo ciclo dos emoraitas, 
do fim do século \textsc{iii}.

\textbf{Rabi Iossi ben Iehuda:} Membro do quinto ciclo dos tanaítas, do início do
século \textsc{ii}.

\textbf{Rabi Iossi Hagalili:} Membro do terceiro ciclo dos tanaítas, 
em meados do século \textsc{ii}.

\textbf{Rabi Ioshiá:} Membro do quarto ciclo dos tanaítas, do fim do século \textsc{ii}.

\textbf{Rabiná:} Rabino, em aramaico.

\textbf{Rachil:} Caluniador.

\textbf{Rashá:} Ímpio.

\textbf{Rav:} Rabino.

\textbf{Revi'it:} Medida líquida próxima à quarta parte de um cálice de vinho.

\textbf{Resh Lakish:} Vide Rabi Shimeon ben Lakish.

\textbf{Ribit:} Usura.

\textbf{Ribit ketzutzá:} Usura reduzida.

\textbf{Rosh Hashaná:} Literalmente \emph{cabeça do ano}, é a 
festa do ano novo judaico.

\textbf{Sanhedrin:} Capítulo de um dos tratados do Talmud. 

\textbf{Sêfer Hamitzvot:} Livro dos preceitos.

\textbf{Selaim:} Plural de \emph{selá}, espécie de moeda antiga.

\textbf{Seret:} Tatuar"-se.

\textbf{Simchat Torá:} Literalmente \emph{rezijo da Torá}. É uma festa que ocorre oito 
dias após Sucót, quando encerra"-se a leitura anual da Torá.

\textbf{Seritá:} Tatuagem.

\textbf{Shaatnez:} A mistura de lã com linho, que é proibida 
pela Torá (ver \emph{Quil'aim}).

\textbf{Shabatot:} Plural de shabat.

\textbf{Shavuot:} Festa do recebimento da Torá, conhecida como Festa 
das Colheitas ou Festa das Primícias (ver \emph{Bicurim}). É também o 
nome de um capítulo de um dos tratados do Talmud.

\textbf{Shevi'it:} Nome de um capítulo de um dos tratados do Talmud, 
que anuncia a lei de não trabalhar a terra no sétimo ano do ciclo agrícola.

\textbf{Shevuat bitui:} Jurar cumprir e não cumprir.

\textbf{Shevuat shav:} Jurar em vão.

\textbf{Shevuat sheker:} Jurar pela mentira.

\textbf{Shevuot:} Capítulo de um dos tratados do Talmud.

\textbf{Sheerá:} O sustento.

\textbf{Shechitá:} Ato de abater a ave ou o animal segundo os preceitos
da Torá.

\textbf{Shekalim:} Capítulo de um dos tratados do Talmud.

\textbf{Shekel:} Moeda de prata.

\textbf{Shechiná:} Divindade descrita no livro cabalístico Zohar 
como a presença feminina de Deus.

\textbf{Shemá:} De \emph{Shemá Israel}, principal reza da religião judaica.

\textbf{Shemini Atzeret:} Oitavo dia de Sucót.

\textbf{Shemoná Sheratzim:} Capítulo de um dos tratados do Talmud.

\textbf{Sheniót:} Preceitos rabínicos talmúdicos [Incesto de segundo grau, que foram proibidos por autoridade rabínica.]

\textbf{Shitim:} Local onde o povo judeu acampou quando saiu do Egito, 
atual fronteira jordaniana.

\textbf{Shoel Adam (Mehaberó):} Capítulo de um dos tratados do Talmud.

\textbf{Shoel ob:} Consultar"-se com um feiticeiro.

\textbf{Shofar:} Corneta feita de chifre de carneiro que costuma ser
tocada no ano novo judaico.

\textbf{Sido:} Tipo de moeda antiga.

\textbf{Sidrá:} Trecho de leitura semanal da Torá, feita aos sábados.

\textbf{Sidrá Tzáv:} Um dos trechos de leitura semanais da Torá, 
que inicia a seção de Tzáv.

\textbf{Sifrá:} Obra antiga que comenta preceitos rabínicos relativos
ao terceiro livro da Torá, o Levítico, escrita por Rabi Iehuda Ilai Z.\,L. no século \textsc{ii}.

\textbf{Sifrei:} Estilo literário exclusivo pelo qual
foi transmitida a Torá sagrada a Moisés.

\textbf{Sotá:} Capítulo de um dos tratados do Talmud.

\textbf{Sucá:} Tradicional cabana coberta com ramos (ver \emph{Sucot}). 
Também nome de um capítulo de um dos tratados do Talmud.

\textbf{Sucot:} Também conhecido como Festa dos Tabernáculos ou das Cabanas. 
Relembra os quarenta anos de êxodo dos hebreus no deserto após a sua saída 
do Egito, período em que o povo era nômades e vivia em cabanas temporárias.

\textbf{Ta'ale:} Fará subir.

\textbf{Ta'aniot:} Parágrafo de um dos tratados do Talmud.

\textbf{Ta'anit:} Capítulo de um dos tratados do Talmud.

\textbf{Ta'assé:} Fará.

\textbf{Talmud:} Obra composta pela Mishná e pela Guemará.

\textbf{Talmud Torá:} Estudo da Torá.

\textbf{Tamid:} Literalmente \emph{sempre}. É também o nome de um dos tratados 
da Guemará que fala sobre a proibição de se plantarem árvores no Templo para
embelezá"-lo.

\textbf{Taná (ou Tanaíta):} Professor, em aramaico. O Talmud emprega 
esse termo para os doutores da lei que se
empenharam de corpo e alma para que a Mishná fosse escrita e
posteriormente impressa.

\textbf{Taná kamá:} Palavra aramaica que significa o primeiro tanaíta,
ou seja, o primeiro que elaborou a primeira lei de uma determinada
parte da Mishná.

\textbf{Tanaim:} Plural de tanaíta ou taná.

\textbf{Tarbit:} Usura.

\textbf{Targum:} Tradução explicativa, com comentários.

\textbf{Tazriá:} Trecho de leitura semanal do terceiro livro da Torá, o Levítico.

\textbf{Tebel:} Palavra aramaica, que significa algo
impróprio para ser ingerido.

\textbf{Tebul iom:} Banho ritual diário.

\textbf{Tefilin:} Filactérios usados diariamente nas preces matinais.

\textbf{Teharot:} Capítulo de um dos tratados do Talmud.

\textbf{Tehorot:} Tratado completo do Talmud.

\textbf{Telussin:} Espécie de joia usada antigamente pelos soldados.

\textbf{Monte Templo:} Em hebraico \emph{Har Habait}, 
o local onde foi construído o Templo.

\textbf{Temurá:} Capítulo de um dos tratados do Talmud.

\textbf{Tenachashu:} Praticar a bruxaria.

\textbf{Teonenu:} A prática de prever o futuro por meio de bruxaria.

\textbf{Terefá:} Impróprio para ser ingerido.

\textbf{Terumá:} Contribuição oferecida pelo povo aos sacerdotes do Templo, a \emph{oferenda de elevação}.

\textbf{Terumot:} Plural de Terumá. Também o nome de um capítulo 
de um tratado do Talmud.

\textbf{Tigzol:} Saquear.

\textbf{Tishrei:} Sétimo mês do calendário judaico.

\textbf{Torá:} Também conhecida como Pentateuco, por ser composta por 
cinco livros. São eles: Gênesis (Bereshit), Êxodo (\emph{Shemot}), 
Levítico (\emph{Vaikrá}), Números (\emph{Bamidbar}) e Deuteronônimo (\emph{Devarim}).

%\textbf{Tossafot:} Comentários do Talmud que surgiram
%bem depois dos comentários, chamados por conta disso de \emph{suplementos}.

\textbf{Tosseftá}: Palavra aramaica que diz respeito aos suplementos da Mishná
elaborados pelos tanaítas.

\textbf{Toshav vesachir:} Habitante estranho, não circuncisado, 
temporariamente residente por conta de um contrato de trabalho.

\textbf{Tsedaká:} Caridade, cuja raiz é \emph{tsedek}, ou justiça. O preceito ordena fazer caridade no sentido de fazer justiça.

\textbf{Tsitsit:} Espécie de franjas do xale usado nas preces.

\textbf{Tsón:} Pequeno gado.

\textbf{Tzáv:} Trecho de leitura semanal do terceiro livro da Torá, o
Levítico, que tem como título o próprio nome Tzáv.

\textbf{Tzión:} Israel.

\textbf{Ushamartem:} Literalmente \emph{e vocês observarão}.

\textbf{Uziahu:} Rei de Jerusalém entre 645 e 707 a.C., que penetrou no local
sagrado do Templo, onde era proibido entrar. Como castigo ele se tornou
leproso até o fim da vida.

\textbf{Vaiehí baiom hashemini:} Trecho de leitura semanal do terceiro livro da Torá, 
o Levítico, que começa com essas palavras.

\textbf{Vaikrá:} Trecho de leitura semanal do terceiro livro da Torá, o Levítico.

\textbf{Vaigzol:} Literalmente \emph{e extorquiu}.

\textbf{Vaitgodedu:} Literalmente \emph{tatuaram"-se}.

\textbf{Veshamru:} Literalmente \emph{e observarão}.

\textbf{Iadáim:} Capítulo de um dos tratados do Talmud.

\textbf{Iahel:} Profanar.

\textbf{Iáin nêssech:} Vinho impróprio para ser consumido em ritual religioso judaico.

\textbf{Iarimú:} Refere"-se à separação dos donativos.

\textbf{Ievamot:} Capítulo de um dos tratados do Talmud.

\textbf{Iehuda:} Também conhecido como Judá, que é uma das doze tribos de Israel.

\textbf{Ideoni:} Quem pratica bruxaria.

\textbf{Idoa:} É o nome dado a um determinado osso existente nas aves, 
com o qual se praticava a bruxaria.

\textbf{Iom Kipur:} Dia do perdão.

\textbf{Iom Tov:} Capítulo de um dos tratados do Talmud. Pode também 
se referir ao nome genério dado aos dias festivos judaicos.

\textbf{Ioma:} Capítulo de um dos tratados do Talmud.

\textbf{Zav:} Pessoa doente, com propensão a expelir os 
próprios fluidos sem autocontrole.

\textbf{Zava:} Feminino de zav.

\textbf{Zavim:} Capítulo de um dos tratados do Talmud.

\textbf{Zar:} Estranho que não seja descendente da família de Aarão, o
sumo sacerdote.

\textbf{Zevahim:} Capítulo de um dos tratados do Talmud.

\textbf{Zeraim:} Tratado completo do Talmud.

\textbf{Zimá:} Literalmente \emph{depravação}.

\textbf{Zimri:} Representante chefe da tribo de Shimeon.

\textbf{Zoná:} Literalmente \emph{prostituta}.

\textbf{Zot tihi'ié:} Trecho da leitura semanal do terceiro livro da Torá, 
o Levítico, que começa com esse título.
}
\end{hangparas}

