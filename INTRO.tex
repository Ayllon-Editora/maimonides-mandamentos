\chapter*{}
\thispagestyle{empty}
\begin{verse}
Neste dia o Eterno, teu Deus, te ordena\\
Que cumpras estes estatutos e leis.\\
Deverás cumpri"-los diligentemente\\
Com todo o teu coração e com toda a tua alma.\\[10pt]

Com relação ao Eterno, confessaste, hoje, que Ele \qb{}é teu Deus\\
que andarás em seus caminhos\\
E observarás seus estatutos, preceitos e leis,\\
E que atenderás a suas determinações.\\[10pt]
 
E o Eterno confessou hoje, a teu respeito, que fazes \qb{}parte de seu povo,\\
Como Ele te havia prometido,\\
Portanto, deves observar todos os seus preceitos,\\
Para que possas ser um povo consagrado ao \qb{}Eterno, teu Deus\ldots{}\\[10pt]

E agora, Israel, o que o Eterno, teu Deus, pede de ti\\
A não ser que temas o Eterno, teu Deus,\\[10pt]

Que andes em todos os caminhos e que O ames\\
E que sirvas o Eterno, teu Deus, com todo o teu \qb{}coração e com toda a tua alma;\\
Que observes, para o teu bem, os mandamentos do \qb{}Eterno\\\bigskip
\hfill{}(Deuteronômio 26:16--19, 10:12--13) 
%\enlargethispage{\textheight}
\end{verse}


\chapter[Apresentação, \emph{por Moacir Amâncio}]{Apresentação \subtitulo{A tarefa sem fim de Giuseppe, aliás Youssef, aliás\ldots{}}}

\begin{flushright}
\textsc{moacir amâncio}
\end{flushright}

Youssef Nahaïssi, nasceu em Alexandria, Egito, ano de 1941, numa família
judaica de origem sefardita e oriental. Seu pai falava fluentemente o
ladino, idioma dos judeus de origem hispano"-portuguesa, a mãe pertencia
ao clã Saádia, com raízes no Cairo e faria muitas coisas em sua vida.
Quando a família foi obrigada a deixar o Egito em 1957, por causa das
pressões políticas que se tornavam étnicas --- os judeus nativos eram
apátridas suspeitos. O pretexto dessa vez foi a existência de Israel. O
melhor caminho era a porta de saída. O pai Mussa ou Moïse, Moisés,
conseguiu a cidadania e o passaporte italianos para os Nahaïssi. Foi
então que nasceu o Giuseppe. Há um toque de imponderável nisso. De
início, tentaram ir para a Austrália, mas como por lá andavam
preocupados com a Máfia, não queriam saber de mais italianos, mesmo que
tivessem centenas de anos ou dois mil anos de Egito e nunca tivessem
pisado na Itália. A opção de destino foi o Brasil, onde Giuseppe, como
assumiu e ficou conhecido, tornar"-se"-ia o ``Italiano'', estudaria artes
visuais com extensão pelo Rio e pela Itália, mas acabaria por se tornar
homem de marketing premiado e dono de uma carreira importante, pautada
em primeiro lugar pela experiência. Construiria uma família ao lado da
esposa Carmen, com quem teve os três filhos, Nathan, Mauro e Sílvia.
Além da pintura, cultivava outros hobbies, como os charutos e a música,
que gostava de tocar num teclado, para a família e os amigos reunidos
sobretudo nas festas judaicas inesquecíveis.

No meio de tudo isso, ou seja, atividade profissional, hobbies, amigos e
a família, havia um ponto central em sua vida. Um ponto que o ligava a
uma história de milênios, isto é, a religião que professou com o
entusiasmo digno de uma criança e uma leveza de coração digna de Hilel,
o sábio. Nesse sentido, a melhor definição a respeito de suas convicções
judaicas, sua posição que não deixa, claro, de ser motivo de debate, num
universo marcado em primeiro lugar pela convivência de visões em atrito
ou mesmo choque, ainda mais atualmente, quando os judeus se denominam
pelo menos reformistas, conservadores, ortodoxos, ou uma mistura de tudo
isso, Giuseppe dizia com um sorriso tranqüilo, ao qual não faltava uma
ponta de séria matreirice: ``Eu me considero o mais ortodoxo dos
liberais, e o mais liberal dos ortodoxos''. Se tomarmos o Judaísmo por
seu eixo de vida --- como ele fazia --- ficará mais fácil acompanharmos e
entendermos como construiu aos poucos um sentido para sua existência.
Nessa posição, um homem se torna não anacrônico, mas atemporal. Em vez
de assumir a atitude daquele neto de rabino chamado David Émile Durkheim
--- um dos inventores do laicismo não só judaico --, segundo o qual a
pessoa está destinada e encontrar um molde que a tornará humana dentro
da estrutura ideológica da sociedade, com sua escala de valores capaz de
retirá"-la do caos bestial, ele preferia a recomendação do rabino Iossef
Caro, que em sua obra máxima, a codificação do \emph{Shulchan Aruch},\footnote{Literalmente \emph{A mesa posta.}} feita no século 16 na mítica cidade de Sfat, num alto de montanha da Galiléia.

Determinada recomendação fundamental de Iossef Caro poderia até ter
inspirado Durkheim às avessas. Lê"-se, logo no início desse livro, que
toda manhã ao despertar, a pessoa deve reunir forças a fim de superar"-se
como um leão para não ser derrotado pelas forças escuras, o caos, o
\emph{tohu vavohu} da noite de onde emerge. É preciso tomar consciência
da sua pequenez abraâmica e agradecer a Deus pela vida. Um agradecimento
que se deve realizar em bênçãos e atos tendo o Divino por referência,
inclusive nas suas relações com o próximo, na vida doméstica, no
trabalho etc. Toda a vida do ser humano, enfim, precisa --- independente
de época e lugar --- ser uma dedicação plena ao serviço divino, com
temência e respeito que para resumir numa palavra se resumem à prática
do amor. Isso em consonância direta com a terceiro ensinamento, dos
\emph{Pirqei Avot},\footnote{Literalmente \emph{Capítulos dos Patriarcas}.} uma das mais célebres
súmulas do judaísmo de que se tem notícia. Nessa lição, Antígonos de
Soho ensina que não se deve amar a Deus como quem pretende obter
recompensa, pois este é o amor do servo interesseiro, voltado mais a si
próprio e não ao objeto de amor. Deve"-se amar de maneira incondicional,
pois este é o amor verdadeiro, livre de egoísmo e pequenas ambições.
Dizia Antígonos: ``Não sejais como os escravos que servem ao seu senhor
a fim de receber uma recompensa, mas sede como os escravos que servem ao
seu senhor sem pensar em recompensa, e o temor dos céus esteja sobre
vós.''

\asterisc

No judaísmo, o amor é indistinto, amor que um homem dedica à mulher tem
a mesma natureza do amor que ele dedica aos filhos e a Deus. Mais
delicado do que todas as outras emoções humanas, o amor está exposto a
todo tipo de confusão misturado com os desejos, os temores e também o
seu contrário, o ódio. Todos sabemos que a linha entre a manifestação de
amor e da ira é uma linha facilmente rompida. No caso do amor a Deus,
confundi"-lo com um meio de atingir nossos próprios objetivos, graças à
Sua participação na história, é algo muito freqüente e também faz parte
do comportamento humano. Cada um dedica"-se à fé conforme as suas
próprias forças, e este é um ponto fundamental, digamos, democrático,
pois evita a instituição do elitismo e da intolerância. O assunto ocupou
de maneira profunda o rabino Moisés ben Maimon, o Rambam\footnote{Moisés filho de Maimon, ou Maimônides, na versão grega, cerca de Córdoba c. 1238--Fustat, 1204.} o médico que se dedicou a ensinar aos judeus
--- e pela universalidade do seu pensamento que nunca deixa de ser
judaico, aos não judeus também\footnote{Vide Tomás de Aquino.} do seu e de todos
os tempos sobre a vida material e espiritual pautado pela Torá, a Lei
Escrita, e pela Halachá, a Lei Oral, a Torá Shebeal Pê, a tradição dos
sábios de Israel.

Um dos grandes momentos de suas lições é aquele em que trata do serviço
que o ser humano deve prestar ao Criador, ao qual se refere o trecho
citado. Isso também ocuparia de maneira exaustiva um de grandes
comentadores de Rambam do século 20, o professor Yeshaiahu (Isaías)
Leibowitz. Ele nasceu em Riga, Lituânia, no ano de 1903, foi educado na
Alemanha e Suíça e morreu em Israel no ano de 1994. Formou"-se em
medicina, dedicou"-se ao ensino e à pesquisa em Israel, para onde
dirigiu"-se com a família em 1935. Em função dos seus ensaios judaicos
ele foi convidado pela faculdade de filosofia da Universidade Hebraica
de Jerusalém. A mesma questão surge diversos momentos na obra do
pensador Leibowitz, na verdade está no cerne da sua profissão de fé
judaica renovada dia a dia dentro dos moldes que se convencionou chamar
de ortodoxia.

De acordo com o prof. Leibowitz, o tipo de amor que o homem dedica a
Deus depende, como o amor que dedica a outros objetos, do nível de sua
consciência e aprimoramento espiritual. Num primeiro nível, temos o amor
e, em conseqüência, o serviço do escravo que conserva por alvo, na
verdade, a satisfação de suas próprias necessidades, e da comunidade, o
indivíduo expandido. Neste caso, a religião assume um caráter
antropocêntrico. Num nível mais aprimorado dessa consciência religiosa,
o homem serve a Deus sem nenhuma expectativa, não por renúncia, mas pela
convicção de que seu dever é dedicar"-se ao serviço divino, sendo que sua
recompensa está no ato mesmo de servir a Deus, de acordo com Ben Zomá,
na terceira parte dos \emph{Capítulos}, ensinamento 2. Assim, de modo
diferente do escravo interesseiro, ele pratica os mesmos atos
determinados pelos 613 mandamentos ou preceitos, que o Rambam relacionou
em seu livro, como um gesto de amor a Deus cuja compensação é o próprio
ato de servir, motivado pela consciência da pessoa em relação ao Criador
e mais nada. Um gesto de liberdade na submissão incondicional, de que
temos exemplo máximo no episódio da Aqedá, o Sacrifício de Isaac.\footnote{Gênesis 22.}

Dessa maneira, amar ao próximo como a si mesmo significa tratá"-lo com
dignidade não porque esperamos um gesto recíproco, ou uma dádiva
celeste, mas porque agindo de tal maneira estamos colocando em prática o
amor a Deus. A transcendência está na relação com os demais, no assumir
a responsabilidade com o mundo dos homens. É um aspecto decorrente do
conceito de Torá lishemá (a Torá por sua própria causa) e Torá lelô
lishemá (Torá não por sua própria causa). O tema não é nada simples e
tem um imenso potencial polêmico. Como fica, a partir disso, a teologia
quando confrontada por um fato como a Shoá? Depois da Shoá, o que Deus
significa para as pessoas piedosas que esperavam uma resposta divina
para suas orações? O que significa o mal? Como ficamos diante do desafio
representado pelo Sacrifício de Isaac, que enfim não se consumo e
representa, como interpreta Leibowitz na preciosidade intitulada
\emph{Cinco Livros da Fé} (Hamishá Sifrê Emuná, Keter, Jerusalém, 1995)
a síntese da atitude hebraica e judaica perante o Criador, ou seja, a
aceitação das mitsvot, dos preceitos, que configuram a síntese da
Orientação de Moisés, ou Ensinamentos de Moisés, a Torá?

\asterisc

Questões como essa faziam parte do cotidiano de Giuseppe. O ambiente de
uma cidade portuária como Alexandria, cosmopolita até pela origem, com
certeza provocou reflexos no jovem estudante, filho do sr. Moise, que
era funcionário de um banco estatal. Quem tiver interesse em conhecer
uma obra inspirada nesse ambiente sempre pode recorrer ao famoso
\emph{Quarteto de Alexandria}, do autor britânico Lawrence Durrel --- um
dos volumes da série, \emph{Balthazar}, é dedicado a uma personagem
judaica. Gregos, armênios, outros europeus, africanos e levantinos se
cruzavam naquele cotidiano ensolaradíssimo e empoeirado da década de
1940. Os judeus estavam presentes na cidade desde os tempos antigos. Lá
viveu Fílon, o grande filósofo judeu de expressão grega, uma influência
significativa no desenvolvimento da cultura ocidental.

Portanto, quem pensa que aquele mundo era monocromático, engana"-se.
Presente e passado se misturam pelas ruas, arquitetura e rostos,
história e presente. A riqueza de matizes, como se sabe, torna"-se
característica dominante entre os judeus. A experiência cosmopolita de
Giuseppe começava em casa, onde o idioma dominante era o francês.
Estudava nas escolas comunitárias e fez o Bacalauret. Conhecia os
seguintes idiomas: francês, árabe (clássico, egípcio e sírio"-libanês),
inglês, castelhano e italiano, além do hebraico e do português. Estudou
Torá e Talmud na ieshivá de Alexandria, de onde puxava recordações
judaicas diversas, entre elas a simplicidade do sistema de
\emph{shehitá} (abate ritual de animais permitidos ao judeu para
consumo) da época. Na sexta"-feira, sua mãe, d. Liza, o enviava ao
mercado com uma galinha. Lá estava o \emph{shohet,} que fazia a bênção e
abatia a ave de acordo com a \emph{cashrut}, o sistema das regras
alimentares judaicas, a partir da Torá, e recebia uma moeda por isso.

O interesse pelo Rambam, que seria decisivo em sua vida, manifestou"-se
nesse tempo, ligado por um arco à experiência brasileira de Giuseppe,
que chegou a este país quando contava 17 anos. Bem plantado , estudaria
na Escola de Belas Artes de Florença e na Escola de Belas Artes do Rio
de Janeiro. E aperfeiçoou"-se em pintura acadêmica com o professor Pedro
Algaza. Colecionaria prêmios de marketing e a certa altura dedicou"-se a
um preceito muito especial. Tem a ver mais com os demais do que com a
própria pessoa. Ou seja, criar condições e atrair os demais para as
orações coletivas do Shabat. Isso, na sinagoga da Hebraica, onde foi
também diretor de Cultura Judaica. O norte disso está na Torá:
\emph{Veahavta lereachá} \emph{camoha} (Ama ao próximo como a ti mesmo).

Os \emph{Capítulos dos Patriarcas} têm poucas páginas, mas é uma ponta
de iceberg, sempre há ocasião para citar uma de suas \emph{mishnaiot},
determinações e ensinamentos. Desta vez o rabi Tsadok cita Hilel, que
condena o uso da Torá como instrumento em seu próprio benefício material
(4, 5), como já mencionado. Da mesma maneira que os sábios talmúdicos,
que se dedicavam normalmente a diversas profissões --- sabemos pelo
apelido colado aos nomes de muitos deles que eram ferreiros, padeiros,
comerciantes etc. Eles participavam de reuniões periódicas para debater
os pontos da Lei Oral ao longo da vida. Giuseppe se dedicaria cada vez
mais ao estudo de obras do Rambam até decidir que traduziria o
\emph{Sêfer Mitsvot} para o português, em meio a todas as outras
atividades mantidas simultaneamente nas empresas, no seu ateliê, entre
freqüentes viagens profissionais pelo país e pelo exterior. Não só
traduziu como associou"-se a José Luiz Goldfarb, numa livraria e numa
editora com o mesmo nome, Nova Stella. José Luiz o lembra como ``Mestre
Nahaïssi''.

E Alexandria, e o Egito? As lembranças eram cultivadas nos costumes
tradicionais judaicos, na culinária do dia a dia e das festas
religiosas. Às vezes recebia os convidados com um fez vermelho na
cabeça... E as amizades. Com Salomon Wahba, amigo, conterrâneo e
confidente, gostava de conversar em árabe. Salomon, do Cairo, conheceu
Giuseppe providenciava filmes egípcios em vídeo cassete e, com Giuseppe,
dirigia"-se à casa de outro conterrâneo, George Doss, para ver os filmes,
avivar a memória e matar saudades. ``Giuseppe era uma pessoa muito
especial'', diz Salomon.

\asterisc

Como administrador de empresas e homem de marketing, Giuseppe tinha um
faro especial para perceber o que poderia despertar o interesse do
público mais que possa parecer estranho, essa mesma intuição, aliada ao
seu amor pelo Rambam e pela religião de Israel, levou"-o a cometer um ato
que para muitos soava quixotesco àquela altura. Ou seja, simplesmente
traduzir para o português o \emph{Sêfer Mitsvot}, o Livro dos
Mandamentos, dirigido antes de mais nada à mente judaica, mas também a
quem se interesse por ética, mesmo não sendo religioso. Alaguns viam na
empreitada um passo em falso do profissional capaz de lançar com sucesso
marcas de vinho e de produtos para beleza. Estaria misturando alhos com
alfalhos. A prática mostrou no entanto que eles estavam errados. Sim, a
obra não se tornou um best seller imbatível. Mas vendeu bem e foi
recebida com respeito pela imprensa. Enfim, o pensador nascido em
Córdoba, era restituído de algum modo a um segmento da cultura
peninsular que o relegava ao esquecimento. Giuseppe foi pioneiro nisso.
Mais tarde, apareceram obras como o \emph{Livro do Conhecimento}. Uri
Lam encarregou"-se de verter para o português a obra"-prima filosófica do
Rambam, que é \emph{O Guia dos Perplexos}, aventura ainda incompleta.

No \emph{Livro dos Mandamentos} apresentado por Giuseppe no primeiro
volume, o leitor encontrará, ao tomar conhecimento dos 613 preceitos
destinados ao judeu, os caminhos para uma constante reflexão sobre a
própria conduta e a conduta humana em função de Deus. O prof. Leibowitz,
no seu livro \emph{Emunatô} \emph{shel haRambam} (A Fé Segundo Rambam,
que a Hedra republica), ao falar sobre os preceitos e a Torá (cap. 15),
compara este ensinamento de Maimônides a uma revolução de Copérnico no
universo religioso. De acordo com ele, a ``grande questão da
providência, do conhecimento divino e da opção humana é resolvida porque
o fiel não se pergunta como Deus o conhece, mas como ele conhece Deus''.
E cita o Rambam de \emph{Mishnê} \emph{Torá} (O Livro do Conhecimento),
que diz: ``E, conforme o conhecimento que o homem tem de Deus, assim
será o seu amor a Deus''. Na leitura de Leibowitz, a ``compreensão de
Deus é a tarefa que o homem assume de servir a Deus. É um dom divino em
si, que se consubstancia na prática dos preceitos. Aí o professor vê a
grande importância do pensamento maimonideano para a atualidade. O
modelo primeiro e último do homem de fé está em Abraão, como vemos no
episódio da Aqedá (o Sacrifício de Isaac), e em Moisés, que falou com
Deus face a face. Mas cada um conforme suas próprias forças.

Ao decidir que traduziria o \emph{Livro dos Mandamentos}, Giuseppe,
óbvio, cumpria um preceito. Mais uma vez, com o objetivo de que seu
gesto fosse replicado pelo número de leitores desse mapa de aproximação
ao serviço divino. E mais uma vez agia de acordo com uma recomendação
dos \emph{Capítulos dos Patriarcas}, ou seja, aquela que determina ao
judeu como uma de suas atribuições fundamentais a divulgação da Torá.
Toda a lição, a segunda do conjunto, diz o seguinte: ``Sêde moderados na
lei e formai muitos discípulos e construí uma cerca em volta da Torá''.

Cada qual conforme a sua medida.

Giuseppe Nahaïssi, dedicava"-se aos estudos mas nunca esqueceu, a vida se
dá no ato, e sabia quando confiar na intuição. A respeito do \emph{Livro
dos Mandamentos}, ele acertou. Nunca ouvi nada dele sobre isso, mas
ninguém pode garantir que não lhe tenha passado pela cabeça que a obra
deveria chegar ao formato de bolso, para atingir o maior número possível
de leitores, fenômeno se não único, com certeza raríssimo em todo o
mundo. Quixote, com a cabeça nas nuvens e os pés na terra. Cada qual
conforme a sua medida. Giuseppe Nahaïssi, dedicava"-se aos estudos mas
nunca esqueceu, a vida se dá no ato, e sabia quando confiar na intuição.
A respeito do \emph{Livro dos Mandamentos}, ele acertou. Nunca ouvi nada
dele sobre isso, mas ninguém pode garantir que não lhe tenha passado
pela cabeça que a obra deveria chegaria, em edição anterior, ao formato
de bolso, para atingir o maior número possível de leitores naquela
época. Quixote, com a cabeça nas nuvens e os pés na terra. Um dos
trabalhos inéditos deixados por Giuseppe quando se foi, em 2001, é
exatamente sua tradução dos \emph{Pirqei Avot}. Como o rabi Tarfon
resumiu deste modo a condição do homem de fé, na segunda parte, 17:
``Não te caberá concluir o trabalho, nem estás livre dele.''

