%MAIMÔNIDES
%
%Os 613 Mandamentos
%
%
%\textbf{(TARIAG HA-MITZVOTH)}
%
%MO.SHÉ BEN MAIMON
%
%\textbf{HA RAMBAM}
%
%
%2 2 EDIÇÃO
%
%TRADUÇÃO • BIOGRAFIA
%
%
%\textbf{GIUSEPPE NAHAISSI}
%
%NOVA STELLA
%
%\textbf{Dados de Catalogação, na Publicação (CIP) Internacional\\
%(Câmara Brasileira do Livro, SP, Brasil)}
%
%
%Maimônides, 1135-1204.
%
%Os 613 Mandamentos : Tariag Ha-Mitzvoth / Moshé ben Maimon ; tradução,
%biografia Giuseppe Nahaïssi. --- São Paulo : Nova Stella, 1990.
%
%1. Judaísmo - Doutrinas 2. Maimônides, 1135-1204 3. Seiscentos e treze
%Mandamentos I. Título. II. Título: Tariag Ha-Mitzvoth.
%
%
%CDD-296.092
%
%-296
%
%90-0763 -296.172
%
%
%\textbf{Índices para catálogo} sistemático:
%
%
%\begin{enumerate}
%\def\labelenumi{\arabic{enumi}.}
%\item
% 
 %Judaísmo 296
% 
%\item
% 
 %Mandamentos : 613 : Teologia social judaica 296.172
% 
%\item
% 
 %Teólogos judeus : Biografia e obra 296.092
% 
%\end{enumerate}
%
%MAIMÔNIDES\\
%OS 613 MANDAMENTOS
%
%Título original\\
%TARIAG HA-MITZVOTH
%
%Tradução\\
%GIUSEPPE NAHAÏSSI
%
%Produção e Capa\\
%LUCIANO GUIMARÃES
%
%Revis.o\\
%CRISTIANE REGINA BARBIERI\\
%CRISTINE BAENA FONTELLES\\
%NICOLE WEXLER
%
%Composição, Paginação e Filmes\\
%HELVÉTICA EDITORIAL LTDA.
%
%1? edição: 1990\\
%3000 exemplares
%
%2 edição: 1990\\
%2000 exemplares
%
%Copyright de tradução\\
%GIUSEPPE NAHAÏSSI
%
%NOVA STELLA EDITORIAL\\
%R. Antonio de Souza Noschese, 289\\
%05324 --- São Paulo --- SP\\
%Tel.: 268.4214 --- FAX: 268.0987

\chapter*{}
\thispagestyle{empty}
\begin{flushright}
\begin{vplace}[30]
\versal{DO REI DAVID}

\emph{``Ajuda-me a trilhar os caminhos de Teus Man­damentos,\\ pois é neles
que eu encontro deleite''}
\end{vplace}
\end{flushright}

\chapter*{}
\thispagestyle{empty}
\begin{verse}
\versal{DA TORÁ}\\[10pt]

Neste dia o Eterno, teu Deus, te ordena\\
Que cumpras estes estatutos e leis.\\
Deverás cumpri-los diligentemente\\
Com todo o teu coração e com toda a tua alma.\\[10pt]

Com relação ao Eterno, confessaste, hoje, que Ele é teu \qb{}Deus\\
que andarás em Seus caminhos\\
E observarás Seus estatutos, preceitos e leis,\\
E que atenderás a Suas determinações.\\[10pt]
 
E o Eterno confessou hoje, a teu respeito, que fazes parte \qb{}de Seu povo,\\
Como Ele te havia prometido,\\
Portanto, deves observar todos os Seus preceitos,\\
Para que possas ser um povo consagrado ao Eterno, teu \qb{}Deus...\\[10pt]

E agora, Israel, o que o Eterno, teu Deus, pede de ti\\
A não ser que temas o Eterno, teu Deus,\\[10pt]

Que andes em todos os caminhos e que O ames\\
E que sirvas o Eterno, teu Deus, com todo o teu coração \qb{}e com toda a tua alma;\\
Que observes, para o teu bem, os Mandamentos do \qb{}Eterno \enlargethispage{\textheight}
\end{verse}


%Agradeço
%
%Ao Rabino SHMUEL ZAIONTZ, ROSH YECHIVA, TOMCHEI TMIMIM LUBAVITCH de Nova
%York pela revisão e aconselhamento na preparação desta tradução.
%
%Ao Professor ELIE SOU''CAR pela grande ajuda na preparação do Glossário
%e seus comentários.
%
%A Sra MARY VANSTREELS pela ajuda e colabora­ção em ordenar esta obra,
%sem a qual a mesma se­ria absolutamente impossível.
%
%Dedico estra tradução da obra do Rambam, Rabi Moshé ben Maimon, o Grande
%Maimônides, à minha mulher Sarah e aos meus filhos Moshé,
%Nathan. e Carmelah na esperança de que continuem a
%trilhar o caminho da verdade.


%PREFÁCIO 11

%MAIMÔNIDES VIDA E OBRA 13 OS 14 FUNDAMENTOS 37 OS PRECEITOS POSITIVOS 75
%OS PRECEITOS NEGATIVOS 183 GLOSSÁRIO 349

\chapter{Prefácio}

``De Moisés a Moisés não houve outro igual a Moisés''. Esta frase
singu­lar está gravada na pedra tumular de Maimônides, ou Moisés filho
de Maimon, na cidade de Tiberíades não longe das margens do mar da
Galiléia, onde o grande mestre está sepultado. Ela foi escrita pelos
seus discípulos querendo dizer que desde Moi­sés filho de Amram --- o
maior legislador Hebreu, autor por inspiração divina dos dez mandamentos
e da lei, a Torah --- até Moisés filho de Maimon --- o seu maior
intérprete e autor do \emph{Misbneb Torah ---} não houve outro que
pudesse ser compara­do em grandeza e sabedoria ao primeiro Moisés.

Quando Moisés foi incubido por Deus de empreender a gigantesca tare­fa
de resgatar os filhos de Israel da escravidão do Egito e levá-los à
terra prometida, lá em Canaã, muito mais que libertar escravos era
necessário torná-los aptos a guiar seus próprios destinos como
indivíduos e como nação para os séculos a virem. Era necessário uní-los
sob um estatuto que iria reger seus modos de vida, criar costu­mes e
tradições nacionais, dar a eles um conjunto de leis que deveria servir
tanto na guerra como na paz e conter instruções para a conquista, para o
tratamento a ser dispensado aos inimigos, aos cativos de guerra, aos
estrangeiros e peregrinos e aos aliados da paz e da guerra, conter o
estatuto e divisão da terra, o direito civil de paz, a obrigatoriedade
do ensino, as leis do casamento, as uniões lícitas e ilícitas, as leis
do divórcio e da herança, as instruções de higiene e de saúde e o regime
alimentar. Era também necessário afirmar este povo num código moral e
ético apoia­do num único Deus universal, intangível, justo e onipotente;
um código que dis­serta sobre o bem e o mal, a sua relatividade e seus
impactos sobre a vida humana, sobre a virtude e a promiscuidade, a
verdade e a mentira, a pureza e as impurezas. Exige também a luta
constante contra os inimigos da moral estabelecida, personifi­cados
pelos filhos de Amalec que cultuam o caos e, por outro lado, insiste que
os homens se aproveitem corretamente das dádivas divinas, de serem
férteis, de cres­cerem e multiplicarem-se. Neste código são determinados
os deveres e direitos dos cidadãos, dos sacerdotes, dos reis e dos
juízes. O amor ao próximo só é superado pelo amor a Deus, provedor de
todas as coisas.

Esta obra monumental intitulada ``Torah'' ou ``a Lei'' é a primeira
cons­tituição escrita e distribuída a um povo para lhe servir de
estatuto e guia. Escrita sobre pergaminho e dividida em 5 capítulos ou
livros (Gênesis, Êxodo, Levíticos, Deuteronômio, Números), contém 613
artigos de lei mais conhecidos como os 613 preceitos ou 613 mandamentos.
Estes preceitos são divididos em duas grandes sec­ções: os preceitos
positivos ou ``Farás'' e os negativos ou ``Não farás''. São 248 os preceitos
positivos e 365 os negativos, pois usará as 248 partes que compõem o seu
corpo para fazer os seus deveres para com Deus e seu próximo e se
recusará a fazer o mal os 365 dias do ano. Os dez mandamentos resumem os
613 preceitos.

Esta obra é de tal profundidade que não somente guardou o povo de Israel
por 3500 anos como a mais velha nação do mundo mas influenciou a
huma­nidade com sua essência. Os ``dez mandamentos'' são aceitos e
respeitados em to­dos os tribunais do planeta como a lei magna. Duas das
maiores religiões se inspirá­ram na moral deste código, o Cristianismo e
o Islamismo, que juntas contam hoje com mais de 2 bilhões de fiéis. Os
cinco livros de Moisés são os cinco primeiros livros da Bíblia,
considerada sagrada e divina tanto por cristãos, muçulmanos como judeus
e base de sua ética e suas orações.

Esta lei também chamada de ``Lei de Moisés ou lei Mosaica'', aplicada na
disciplina do povo hebreu durante os 40 anos no deserto de Sinai, foi a
lei dos juízes de Israel, dos profetas; teve sua corte instalada durante
o período dos reis
``De Moisés a Moisés não houve outro igual a Moisés''. Esta frase
singu­lar está gravada na pedra tumular de Maimônides, ou Moisés filho
de Maimon, na cidade de Tiberíades não longe das margens do mar da
Galiléia, onde o grande mestre está sepultado. Ela foi escrita pelos
seus discípulos querendo dizer que desde Moi­sés filho de Amram --- o
maior legislador Hebreu, autor por inspiração divina dos dez mandamentos
e da lei, a Torah --- até Moisés filho de Maimon --- o seu maior
intérprete e autor do \emph{Mishneh Torah ---} não houve outro que
pudesse ser compara­do em grandeza e sabedoria ao primeiro Moisés.

Quando Moisés foi incubido por Deus de empreender a gigantesca tare­fa
de resgatar os filhos de Israel da escravidão do Egito e levá-los à
terra prometida, lá em Canaã, muito mais que libertar escravos era
necessário torná-los aptos a guiar seus próprios destinos como
indivíduos e como nação para os séculos a virem. Era necessário uní-los
sob um estatuto que iria reger seus modos de vida, criar costu­mes e
tradições nacionais, dar a eles um conjunto de leis que deveria servir
tanto na guerra como na paz e conter instruções para a conquista, para o
tratamento a ser dispensado aos inimigos, aos cativos de guerra, aos
estrangeiros e peregrinos e aos aliados da paz e da guerra, conter o
estatuto e divisão da terra, o direito civil de paz, a obrigatoriedade
do ensino, as leis do casamento, as uniões lícitas e ilícitas, as leis
do divórcio e da herança, as instruções de higiene e de saúde e o regime
alimentar. Era também necessário afirmar este povo num código moral e
ético apoia­do num único Deus universal, intangível, justo e onipotente;
um código que dis­serta sobre o bem e o mal, a sua relatividade e seus
impactos sobre a vida humana, sobre a virtude e a promiscuidade, a
verdade e a mentira, a pureza e as impurezas. Exige também a luta
constante contra os inimigos da moral estabelecida, personifi­cados
pelos filhos de Amalec que cultuam o caos e, por outro lado, insiste que
os homens se aproveitem corretamente das dádivas divinas, de serem
férteis, de cres­cerem e multiplicarem-se. Neste código são determinados
os deveres e direitos dos cidadãos, dos sacerdotes, dos reis e dos
juízes. O amor ao próximo só é superado pelo amor a Deus, provedor de
todas as coisas.

Esta obra monumental intitulada ``Torah'' ou ``a Lei'' é a primeira
cons­tituição escrita e distribuída a um povo para lhe servir de
estatuto e guia. Escrita sobre pergaminho e dividida em 5 capítulos ou
livros (Gênesis, Êxodo, Levíticos, Deuteronômio, Números), contém 613
artigos de lei mais conhecidos como os 613 preceitos ou 613 mandamentos.
Estes preceitos são divididos em duas grandes sec­ções: os preceitos
positivos ou ``Farás'' e os negativos ou ``Não farás''. São 248 os preceitos
positivos e 365 os negativos, pois usará as 248 partes que compõem o seu
corpo para fazer os seus deveres para com Deus e seu próximo e se
recusará a fazer o mal os 365 dias do ano. Os dez mandamentos resumem os
613 preceitos.

Esta obra é de tal profundidade que não somente guardou o povo de Israel
por 3500 anos como a mais velha nação do mundo mas influenciou a
huma­nidade com sua essência. Os ``dez mandamentos'' são aceitos e
respeitados em to­dos os tribunais do planeta como a lei magna. Duas das
maiores religiões se inspirá­ram na moral deste código, o Cristianismo e
o Islamismo, que juntas contam hoje com mais de 2 bilhões de fiéis. Os
cinco livros de Moisés são os cinco primeiros livros da Bíblia,
considerada sagrada e divina tanto por cristãos, muçulmanos como judeus
e base de sua ética e suas orações.

Esta lei também chamada de ``Lei de Moisés ou lei Mosaica'', aplicada na
disciplina do povo hebreu durante os 40 anos no deserto de Sinai, foi a
lei dos juízes de Israel, dos profetas; teve sua corte instalada durante
o período dos reis
e sua autoridade excedia a do próprio
monarca. Com a destruição do primeiro
Tem­plo e de Jerusalém houve uma enorme necessidade de se preservar a
lei e seus valo­res e começou a era dos estatutos acadêmicos e a redação
do resto dos livros que . compõem a Bíblia hoje.
Com a volta a Sion, 70 anos mais tarde, reestabeleceu-se a Suprema
Corte ou Sanhedrim, mais conhecida como ``Sinedrio'', onde os juízes
decidiam todas as questões. Com a invasão
helenística, judeus e gregos se influen­ciaram mutuamente; a dialética
da filosofia grega tornou-se presente nas academias israelitas e a moral
judaica invadiu os gregos. Com a dispersão
criada pela invasão romana, era imperativo salvar a lei. Rabbi Yohanan,
filho de Zaccai, vendo desmo­ronar o Templo e a Suprema Corte, conseguiu
escapar do desastre e montou uma academia em Yavne, Israel, formando 72
Mestres ou Rabbis.

O Talmud ou os estatutos da jurisprudência da lei Mosaica foi a coluna
mestre da sustentação da lei no Exílio. As cortes rabínicas espalhadas
pela comuni­dade na Diáspora resolviam as questões entre os judeus e,
desta forma, os rabinos mantiveram a comunidade, o ensino, a moralidade
e a fé.

Em mais de 2500 anos de jurisprudência em todos os estilos, os mestres
da lei Mosaica, os Rabbi, compuseram uma obra monumental que os judeus
con­vencionaram chamar de lei oral, pois ela foi proferida verbalmente
pelos juízes e acadêmicos e, posteriormente, redigida pelos escribas. A
lei oral explica a lei e con­tém os relatos e as opiniões de centenas de
mestres. Maimônides decidiu colocar num só trabalho o sumo da lei oral e
escreveu o \emph{Mishizeh Torah} ou a Torah pela segunda vez. Para
resumir decidiu escrever este trabalho em 2 volumes com curtas
explicações sobre cada preceito e referências de onde encontrar mais
sobre o as­sunto na literatura talmúdica.

Esta é a tradução do livro dos preceitos de Maimônides, intitulado em
hebraico \emph{Sefer Ha-Mitzvoth.} Recomendamos a sua leitura com muita
atenção e me­ditação e, apesar de muitos dos preceitos não caberem mais
nos dias de hoje, pelo menos de maneira literal --- como os serviços dos
sacerdotes no templo, ou dos cas­tigos a serem aplicados ---, no bojo de
cada preceito há' uma lição de vida como; no preceito de como retirar
cinzas do santuário em que está embutido o respeito ao passado: a
juventude foi o fogo de ontem e a velhice tem que ser respeitada,
cui­dada e levada em segurança. Quanto aos castigos por açoitamento, o
exílio ou até a pena de morte, hoje nos servem para dimensionar a
gravidade do crime cometido.

A maneira mais prática de estudar a lei Mosaica é começando pelo livro
dos preceitos de Maimônides e isso foi exatamente o que eu fiz quando,
no ano de 1985, festejava-se os 850 anos do seu nascimento e os Rabinos
do mundo inteiro e especialmente o Rabbi Menahem Meyer Schneersohn,
Rabino chefe do movimento ``chabad'', recomendaram o estudo da obra do
grande Mestre.

No estudo dos textos em hebraico, tendo ao lado a tradução em inglês, eu
tomava notas em português para minha memória. As notas se acumularam e
al­guns amigos sabendo do meu trabalho me recomendaram sua publicação.
Refinei o texto e parti para a tradução. literal. A fidelidade ao
espírito do texto foi funda­mental e muitas vezes achei conveniente
sacrificar o vernáculo a favor do assunto e com a ajuda de Deus e de
todos os que colaboraram comigo nesta tarefa apresen­tamos este
trabalho.

\begin{flushright}
\emph{Giuseppe Nahaïssi}\\
\emph{29 de ADAR 5750\\
26 de março de 1990\\
São Paulo --- Brasil}
\end{flushright}

\chapter{Maimônides: vida e obra}

No ano de 1166, aos 31 anos, desembarca em Alexandria, no Egito, Moisés,
filho de Rabi Maimon, o homem que viria a ser conhecido como o Moi­sés
do Egito e respeitado pelo mundo todo como uma das mais relevantes
figu­ras do pensamento judeu.

``De Moisés a Moisés não houve outro igual a Moisés'': é assim que os
estudiosos costumam se referir a esse grande sábio cujo legado foi
decisivo para a manutenção da fé e da união do povo judeu no século XII.
Sua glória se extendeu aos círculos não judeus, e nos meios cultos de
Bagdad ele passou a ser considerado como um dos mais eminentes homens da
época. Maimôni­des foi o responsável, entre outros feitos, pela
subordinação do valor moral ao valor teórico, e pela análise
contemplativa abstrata como objetivo final, ao invés do julgamento
concreto dos atos, se bem que a introdução da inteligên­cia no espírito
religioso já houvesse sido feita na época tanaica e que o valor
religioso da compreensão talmúdica já fosse conhecido pelo povo há muito
tem­po. A superioridade da contemplação sobre o rito e a moral constitui
o pilar central de seu pensamento e embora o Talmud ensine que não são
as pesquisas e sim o fato o que importa, ele insiste nas pesquisas
porque tem a profunda convicção de que o amor de Deus é tanto maior
quanto mais desenvolvida e aperfeiçoada for nossa inteligência.

Talmudista, codificador da Torah, filósofo, místico, matemático, mé­dico
e dono de um talento literário ímpar, ele iria transformar a comunidade
judaica do Egito, trazer um nova ordem para os judeus do mundo e viria a
ser o único pensador da Idade Média cujas teorias exerceram influência
significati­va sobre os pensadores tanto cristãos quanto muçulmanos e
judeus de sua épo­ca. Sua obra foi, aliás, freqüentemente citada por
filósofos como Tomás de Aqui-no, Alberto, o Grande, Roger Bacon, Inácio
de Loyola, Alexandre de Halle, Ní­colas de Coves, Leibniz Barouch de
Espinoza e muitos outros.

Homem de personalidade densa e complexa, Maimônides estabele­ceu para si
mesmo uma conduta estrita e complicada, mas soube simplificar o que
desejava transmitir de forma tal que seus leitores pudessem
compreendê-lo facilmente. Fanático pela brevidade, Maimônides se
preocupa sempre em cons-
truir parágrafos claros, sem nenhum interesse em engrandecer seus
pensamen­tos nem glorificá-los com uma retórica exagerada. São suas
estas palavras: "Se me fosse possível resumir o Talmud inteiro numa
frase, eu não quereria fazê-lo em duas". Enquanto algumas de suas obras
são muito eruditas, outras são escri­tas de maneira muito fácil e são de
compreensão extremamente simples. Quan­do interpelado sobre o por quê
dessa diferença entre uma obra e outra, ele res­pondeu: "O pão e o leite
são para as crianças, e a carne e o vinho são para os adultos". Fiel a
essa filosofia, Maimônides conduz seu aluno, fazendo-o crescer em suas
mãos, e levando-o a passar por vários estágios de ``pão e leite''
primeiro, para que ele possa chegar a compreender e a apreciar "a carne
e o vinho" da metafísica, a ciência superior que lhe abriu os caminhos
na sua busca da verdade.

Ele acredita que todos os homens devotos, sem exceção, que vivam de
acordo com a virtude e que sigam os mandamentos bíblicos e mantenham
sempre boa conduta, serão recompensados com o mundo futuro,
independen­temente de seu credo ou religião. Respeita e tem íntimos
amigos no mundo islâmico, e costuma afirmar que a doutrina cristã não
tem nenhuma contradi­ção com o judaísmo, pois ela também reconhece a
força e a necessidade dos mandamentos e da moral bíblica, e que seus
adeptos, se quiserem aprofundar-se no estudo contemplativo dos textos,
descobrirão a verdade.

Além de ter sido considerado o maior talmudista do século, esse ho­mem,
que se torna o médico da corte do Egito, servindo ao grão-vizir de
Sala­din, Al Fadil, e depois o sultão Al Afdal, gozava da :eputação de
ser o melhor médico de seu tempo. Os relatos sobre seu saber e sua
competência se esten­dem de tal forma que chegam ao conhecimento do rei
Ricardo Coração de Leão, da Inglaterra, e este o convida para ser seu
médico particular. Maimônides, no entanto, prefere permanecer no Egito,
pois lá ele acumula também o cargo de Naguid, e pode utilizar-se de sua
digna posição para proteger a comunidade ju­daica através do mundo
islâmico. O Naguid era o líder e o porta-voz dos judeus egípcios,
nomeado pelo sultão, e que representava a autoridade moral e políti­ca
de todas as comunidades judias no país dos Fatimitas. Ele era escolhido
den­tre a comunidade rabínica, mas tinha também direito de justiça sobre
os caraí­tas e os samaritanos.

Mas seu brilhantismo e seu sucesso não são fortuitos. Na sua juven­tude,
Maimônides aprende astronomia com o filho do célebre astrônomo Ibn
Aphla, de Sevilha, estuda o Almagesto, o tratado astronômico de
Ptolomeu, as proposições de Algebra, o tratado das secções cônicas, a
geometria, a mecâni­ca, a medicina, tratados astrológicos, bem como
livros teológicos de outras re­ligiões, para adquirir um conhecimento
geral das religiões de seu tempo. Apro­funda-se ainda nas doutrinas
filosóficas de Aristóteles, de Filo, de Afrodisias, de Themistius, de
Alfarabi, de Gazali, do Gaon Saadia, de Bachija, de Rabi Ye­huda Halevi,
de Rabi Abraham bar Chiha e de Rabi Abraham Ibn Esra, mas ba­seia suas
explicações metafísicas mais profundas no pensamento aristotélico.

Já aos 16 anos Maimônides escreve uma introdução à lógica, e aos 23 uma
dissertação matemática e astronômica, tratando das questões principais
da determinação do calendário judaico. Pelos lugares por onde passa
durante o êxodo em que vive durante 20 anos, ele estuda atentamente a
flora dos países e se interessa por suas arquiteturas. No Egito ele
estuda os usos e as particulari­dades da língua, os hábitos e a moral
dos judeus egípcios, e chega a redigir um comentário sobre essas
observações.

O RAMBAM, sigla de Rabi Moisés ben Maimon, ou simplesmente Mai­mônides,
do grego ``filho de Maimon'', nasceu em Córdoba, na Espanha, em 1135,
filho do Dayan ou Juiz Rabínico Rabi Maimon, descendente de uma longa 
linhagem de Dayanim ou juízes, remontando a Rabi Yehuda Hanassi, o
autor da Mishná, sábio que havia atingido a perfeição moral e
intèlectual, e que, por sua vez, era descendente direto da casa real de
Davi.

Tendo ficado sem a mãe ao nascer, Maimônides se revela uma crian­ça que
desde cedo se habitua a entregar-se a meditações profundas sobre a vida
e a morte e a confiar-se sozinho a Deus. Para isso, ele se refugia na
sinagoga durante a semana, na parte reservada às mulheres, para meditar,
onde tem cer­teza de que ninguém virá interrompê-lo.

No ano de 1148, quando o jovem Moisés completa 13 anos, os Al­mohades,
liderados por Abd-el-Mumin, invadem a cidade de Córdoba. Esses
Almohades, ou ``confessores da unidade'', eram uma tribo berbere que
con­quistara o poder na Espánha e no Marrocos, após 20 anos de lutas
sangrentas. Abd-el-Mumin era o sucessor de Ibn Toumert, jovem e ardoroso
muçulmano que vivia no sudoeste do que atualmente é o Marrocos e que,
insatisfeito com os ensinamentos teológicos básicos que ali lhe haviam
sido ministrados, deci­diu aprofundar-se no assunto, indo para isso às
faculdades de Córdoba, de Me­ca e de Bagdad, onde entrou em contato com
os ensinamentos de Gazali. De­pois de ter aprendido a ciência teológica
oriental, Ibn Toumert voltou para sua região natal e; declarando-se
descendente de Maomé, liderou uma guerra santa contra as altas esferas
do governo as quais, segundo ele, eram as responsáveis --- entre outras
tantas coisas inadmissíveis --- pelo relaxamento religioso, pelo luxo e
pela decadência moral da corte e da alta sociedade, pela representação
material de Deus --- o que era uma blasfêmia ---, e pelo ``politeísmo'',
que ele atribuía aos antigos fiéis da África do Norte, os quais
afirmavam, tal como os cristãos, a pluralidade do Ser divino. A
revolução teológica, aliada ao desejo de conquista, levou a um sucesso
sem precedentes, e o reino dos Almohades se estendeu da Síria ao oceano
Atlântico. Eles destruíam as igrejas e as sinago­gas, e aos povos que
não aceitavam converter-se a "verdadeira religião islâmi­ca", propagada
por eles, restava a opção entre a imigração ou a morte.

O jugo dos Almohades se fazia sentir na mesma época em que as Cru­zadas
partiam da França e da Alemanha, a fim de conquistar a Terra Santa e
apossar-se do túmulo de Cristo. Intolerantes, os Cruzados arrasavam, na
sua mar­cha para Jerusalém, tudo o que encontravam de não cristão,
massacrando em seu caminho as populações judias indefesas. Mais uma vez
a história se repetia e os judeus se defrontavam com uma nova e grave
crise de identidade: dobrar-se aos conquistadores islâmicos ou à
barbárie dos cruzados em marcha:. O fana­tismo
religioso imperava tanto no levante quanto no ocidente e continuar
pro­fessando o credo judaico representava um risco de vida. Talvez esse
tenha sido um dos momentos mais difíceis e trágicos da história da
sobrevivência do ju­daísmo; eram necessárias grandes forças para
sustentar a fé, e um dos persona­gens mais importantes dessa época foi
Maimônides, pois a clareza de suas idéias e de seus escritos mantiveram
acesas no povo judeu as chamas da crença e da liberdade da ciência e do
conhecimento; para enfrentar esse período singular e sinistro.

Preferindo o êxodo ao massacre ou à renúncia de sua fé, a família do
Rabi Maimon sai de Córdoba quando os Almohades lá chegam. Depois de 10
anos de vida errante, nos quais passam por diversas cidades do sul da
Espa­nha, eles chegam a Fez, capital do Marrocos --- a África do Norte
sempre fora o local de asilo para os judeus que fugiam das perseguições
religiosas na Espa­nha. Munido de coragem ímpar, Rabi Maimon opta por
Fez, onde os "confes­sores da unidade" haviam instalado sua corte,
porque tem a esperança de ser introduzido ao líder deles, o califa
Abd-el-Mumin, homem que gozava da repu-
tação de interessar-se pelas coisas do espírito e que procurava
cercar-se de sá­bios, a fim de expor-lhe o pensamento judeu relativo a
Deus e tentar assim ob­ter uma mudança na política do governo em relação
aos judeus.

Muitos judeus, no entanto, para escapar a morte ou ao abandono do lar,
optavam pela conversão aparente a doutrina dos "confessores da
unida­de." Essa conversão, que os obrigava a uma vida dupla vergonhosa e
sem dig­nidade, era algo suportável apenas enquanto eles contassem com a
providência divina e enquanto essa situação de sofrimentos tivesse algum
sentido compreen­sível. Contudo, esse conflito se tornava um sofrimento
intolerável quando o sustentáculo moral da fé começava a desmoronar,
abalando sua confiança em Deus e em si próprios. A crença na unidade
absoluta de Deus, apregoada pelos Almohades, parecia ao povo mais
simples, idêntica à doutrina judaica e eles co­meçavam a acreditar que a
missão do povo eleito tinha chegado a seu fim e a se perguntar se o
profeta Maomé não era realmente superior a Moisés.

Extremamente preocupado com essa situação, Rabi Maimon decide escrever,
em 1159, uma carta em árabe que ele envia às comunidades judias da
África do Norte, recordando-lhes a infalibilidade divina, a existência
de uma aliança permanente entre Deus e Israel, a superioridade de Moisés
e o profun­do significado da prece. Nessa carta ele diz: "Um rei, ao
demitir um de seus funcionários, tem o hábito de nomear imediatamente um
outro, a fim de transmitir-lhe o cargo e as funções do primeiro. Um
marido que repudia sua mulher, geralmente coloca uma outra em seu lugar,
e lhe dá os adornos e a ca­ma da primeira. O sinal da mudança consiste
em dar ao sucessor os direitos e as honras do predecessor. Onde está,
afinal, o povo a quem o Eterno apare­ceu, ao qual Ele deu uma Torah e
sobre o qual Ele espalhou sinais de Sua bene­volência, semelhantes
aqueles com os quais Ele favoreceu os judeus?" Rabi Mai­mon refere-se
aqui às maravilhas do Êxodo, quando Deus libertou o povo de Israel da
opressão faraônica, às 10 pragas impostas ao inimigo, à abertura do mar
para a passagem do povo eleito, ao milagre do maná para sua alimentação
diária durante 40 anos, à presença de Suas colunas de fogo para guiá-los
à terra prometida e à entrega de Seus mandamentos a viva voz, desde o
cume do Si­nai, não através de um emissário, nem de um intermediário,
nem sequer de um anjo, mas através d'Ele próprio, em Sua glória. A carta
segue assim: "Enquanto nenhum outro povo puder mostrar sinais de
clemência e de benevolência simi­lares, só se pode considerar como
falatório o abandono de Israel em favor de um outro povo. Ainda que
vivamos incessantemente na angústia, ainda que pe­la manhã desejemos a
chegada da noite, e a noite a chegada da manhã, ainda assim devemos
pensar na seguinte profecia: 'Deus não esquecerá a aliança que ele fez
com teus pais' ".

... Deus não quer destruir, mas purificar Israel. Devemos conside­rar
nossa aflição atual como um ensinamento, como uma prova. Como acredi­tar
na ira do Eterno, no repúdio de Israel? A missão de Moisés, de nosso
incom­parável mestre, prova a eleição de Israel. ... O sucesso material
não prova o valor de uma nação. A preferência de Deus por Moisés e por
Israel, preferência confirmada em várias ocasiões pela benevolência
divina, garante a efetivação das promessas do Senhor, mas não se pode
saber quando ocorrerá sua realiza­ção, trazidas pelo arrependimento e
pela oração. ..."

Seguindo o exemplo de seu pai, aos 24 anos Maimônides decide sair da
vida de estudo e de trabalho solitários que levava até então para
redigir um tratado sobre um julgamento feito por um rabino que condenara
como traido­res do judaísmo os judeus que se convertiam em aparência à
doutrina dos Al­mohades. Maimônides considera que professar a fé
islâmica para continuar vi-

vo não é apostasia, baseado no fato de que outros judeus haviam tido
atitudes similares anteriormente, sem que por isso tenham provocado a
ira do Senhor, e que o mais importante é a sobrevivência do povo de
Israel. Nessa sua primei­ra obra, publicada em Fez, ele diz: "Se as
colunas do mundo, Moisés, Elias, Isaías, e até mesmo um anjo, foram
punidos porque ousaram elevar a voz contra Is­rael, então quanto não
deve ser censurado um homem suficientemente auda­cioso para dizer que
nas comunidades judias há malfeitores, pagãos, homens indignos de
prestar testemunho a Deus, ateus! ... Então esse rabino estrangeiro e de
pouca reflexão não sabia que os que se convertem pela força não pecam
por negligência? ... O Senhor não os abandonará; Ele não os rejeitará:
Ele nun­ca menosprezou a miséria dos infelizes."

Orientando-se pelo provérbio ``Não se consegue nada sem penar'', o
objetivo a que Maimônides se fixa é o de ``compreender'' Deus, até onde
isso for possível ao homem. Para tanto ele julga que deve iniciar pela
lógica, segui­da das ciências matemáticas, das naturais, e por fim da
metafísica, numa pro­gressão do concreto para o abstrato. Assim, ele se
dedica ao estudo de várias ciências para exercitar seu espírito e suas
capacidades intelectuais, a fim de dis­cernir a lógica demonstrativa dos
outros métodos de raciocínio. Ele se consa­gra com zelo ao estudo das
ciências gerais, mas apenas como elementos neces­sários à aquisição de
uma cultura global, e não por uma necessidade interna, pois esta ele
satisfaz através do estudo da Torah.

Já na sua adolescência, Maimônides procura compreender e aprofun­dar-se
nos mistérios proféticos e suas reflexões a esse respeito formam o ponto
culminante de toda sua vida intelectual. Ele tem a convicção de que a
sabedo­ria, a integridade e a modéstia são os atributos que preparam o
espírito do ho­mem para o advento da profecia. Acredita também que, por
mais profundo que possa parecer o saber acumulado por um homem, ele deve
colocar tudo nas mãos do Todo Poderoso, pois o conhecimento é um dom de
Deus. No entan­to, sua tendência especulativa o leva a buscar
incessantemente o sentido da exis­tência individual, já que a crença na
necessidade do pensamento é a idéia con­dutora de sua vida. Para ele o
pensamento é sagrado, e ele só consegue aceitar a crença através da
inteligência e do entendimento, afirmando que a inteligên­cia filosófica
é uma condição ``sine qua non'' para a imortalidade da alma e pa­ra a
participação no reino eterno.


A solução, a resposta, não é o essencial para Maimônides. A disciplina
e a dedicação são as qualidades fundamentais de sua inteligência, daí
ele reafirmar
sempre que não deseja construir um sistema filosófico e sim
apenas facilitar
o caminho para alcançar o conhecimento de Deus. Ele considera como
fraqueza de espírito acomodar-se na crença tradicional toda vez que a lógica
se inclina diante da religião, como por exemplo no caso das posições
dogmáticas, e diz, a esse respeito, o seguinte: "... se alguma coisa não tem um motivo
compreensível e se ela não traz nenhum benefício nem evita nenhum mal, por
que diríamos daquele de quem ela é o objeto de crença ou a regra de conduta,
que ele é sábio e inteligente, e que ele ocupa uma posição elevada? Que haveria
de surpreendente para os povos nisso? ... Diríamos que ... o homem é
mais perfeito que seu criador, pois o homem falaria e agiria visando um determinado
objetivo, enquanto que Deus, ao invés de agir dessa forma, nos ordenaria...
a fazer o que não tem nenhuma utilidade para nós e nos proibiria ações que
não nos trazem nenhum prejuízo." Ele conhece os limites da razão, mas
considera
como imperativo viver sob o império dela, pois para ele a
inteligência não
é um local para descarregar suas dúvidas, mas já faz parte do reino de Deus.
Em 1158, durante sua fuga através da Espanha, ele inicia a redação
de seu \emph{Comentário sobre a Mishná,} obra que leva 7 anos para ser
concluída e na qual, à guisa de prefácio ao décimo capítulo do tratado
\emph{``Sanhedrin''}, ele faz uma exposição da tradição e da doutrina do
judaísmo. Redigida para pro­porcionar uma resposta às dificuldades e às
necessidades do povo judeu na épo­ca, e com o intuito de preservar a
unidade de seu povo, que ameaçava desmo­ronar diante de tantas provações
e de tantos conflitos, essa introdução levou Maimônides a sacrificar
seus princípios liberais e a propor um quadro quase dogmático que
representa, sob seu próprio ponto de vista, o verdadeiro credo do
judaísmo, e que pode ser resumido da seguinte forma:

\begin{enumerate}

\item Eu acredito plenamente que o Criador, que o Seu nome seja bendito, é o
Criador e Guia de todos os seres, que Ele e apenas Ele, criou, cria e
criará todas as coisas.

\item Eu acredito plenamente que o Criador, que o Seu nome seja bendito, é
um e único e que não existe nada mais único do que Ele; que apenas Ele
é nos­so Deus, era, é e será.

\item Eu acredito plenamente que o Criador, que o Seu nome seja bendito, é
eté­reo; que Ele não tem nenhuma propriedade antropomórfica; que nada
é parecido com Ele.

\item Eu acredito plenamente que o Criador, que o Seu nome seja bendito, é
pri­meiro e último.

\item Eu acredito plenamente que o Criador, que o Seu nome seja bendito, é o
único a quem é apropriado rezarmos e que não é apropriado rezar a mais
ninguém.

\item Eu acredito plenamente que todas as palavras dos profetas são
verdadeiras.

\item Eu acredito plenamente que a profecia de Moisés, nosso mestre, que
esteja em paz, foi verdadeira, que foi ele o pai de todos os profetas,
daqueles que o precederam como daqueles que o seguiram (no sentido de
ter sido o maior deles).

\item Eu acredito plenamente que a totalidade da Torah que está em nossas
mãos foi dada a Moisés, nosso mestre, que descanse em paz.

\item Eu acredito plenamente que esta Torah não será modificada e que não
ha­verá outra Torah dada pelo Criador, bendito seja Seu nome.

\item Eu acredito plenamente que o Criador, bendito seja Seu nome, conhece
todas as ações e todos os pensamentos de todos os seres humanos, como
está escrito: .É Ele que amolda o coração de todos,
Ele que capta todas as suas ações" (Salmos 33:15).

\item Eu acredito plenamente que o Criador, bendito seja o Seu nome,
recom­pensa aqueles que observam Seus mandamentos, e pune aqueles que
os transgridem.

\item Eu acredito plenamente na vinda do Messias, ainda que possa tardar, no
entanto espero a cada dia pela sua vinda.

\item Eu acredito plenamente que haverá ressurreição dos mortos no momento
em que assim o desejar nosso Criador, bendito seja Seu nome, exaltada
seja a Sua recordação para todo o sempre.

\end{enumerate}



Incorporados depois à liturgia de várias populações judias, esses
prin­cípios foram recebidos com grande alegria pelas comunidades
carentes, trans­formando-se em hinos de glorificação a Deus. O mais
famoso deles é o Yigdal, de autor desconhecido, com força poética sem
igual e totalmente baseado nas palavras do grande mestre. Este hino é
cantado até os dias de hoje em todas as sinagogas.

Para elaborar seu \emph{Comentário sobre a Mishná,} Maimônides se
inspi­ra, tanto no pensamento como na forma, na própria Mishná, redigida
por seu antepassado, o rabino Yehuda Hanassi. A Mishná, escrita no início do
sécu­lo III, é a obra que condensa as explicações e os resultados dos
estudos e das pesquisas intelectuais que haviam sido feitos até então em
torno da Sagrada Es­critura. Esse trabalho separa o conteúdo da doutrina
daquilo que está direta­mente ligado ao texto da Bíblia, tal como havia
sido transmitido pela tradição, e o reduz a regras e a decisões.

De acordo com seu autor, o \emph{Comentário} deveria trazer novidades e
melhorias aos estudos. Com o passar do
tempo a Guemará, que é um comen­tário elaborado sobre a Mishná, havia
suplantado o estudo desta; grandes sá­bios e profundos conhecedores da
Guemará ignoravam a Mishná, e Maimôni­des os recriminava por isso. Os
objetivos de Maimônides ao redigir o seu \emph{Co­mentário} eram,
portanto o de restabelecer a Mishná na sua posição preponde­rante e o de
fazer um resumo dos debates qué estão nessa obra, de modo a ter-se uma
referência rápida e fácil sobre todas as questões da Lei e de modo a
servir aos debutantes como uma preparação para a dialética superior.

Publicado em 1168, o \emph{Comentário} foi concebido da seguinte for­ma:
as introduções sistemáticas, escritas livremente, se distinguem
totalmente das explicações curtas dos textos da Mishná. A amplidão e a
profundidade da sabedoria do autor aparecem aí de maneira mais forte e
mais clara do que nas partes explicativas, necessariamente limitadas
pelo próprio texto da Mishná. Mai­mônides coloca, nesse seu trabalho, a
seguinte advertência: "Leia várias vezes meu livro e reflita
atentamente. Se sua imaginação lhe fizer crer após a primeira leitura ou
mesmo após a décima que você o compreendeu, então ela o enga­nou. Pois
você não deve fazer a leitura deste livro de maneira rápida: eu não o
escrevi simplesmente, como é o caso às vezes; ele é o fruto de muitas
pesqui­sas e reflexões".

A publicação dessa obra, no entanto, não provoca aparentemente nenhuma
polêmica; Maimônides não possuía as condições habitualmente ne­cessárias
para ser reconhecido como uma autoridade: passar pela escola, tornar-se
professor ou Gaon, ou fazer parte dos trabalhos efetuados numa academia
re­presentativa. Mas ele não queria dever nada a uma posição nem a uma
dignidade.

Com a morte de Abd-el-Mumin em 1163, o qual havia sido de certa maneira
tolerante com a comunidade judia de Fez, seus sucessores retomam as
perseguições cruentas aos judeus, e a família de Maimônides emigra então
para Ceuta, cidade situada à beira-mar, na extremidade norte de
Marrocos, e que naquela época ocupava um lugar preponderante no mundo
das artes e das ciências. Mas os Almohades disputavam acirradamente o
governo de Ceuta e em meio a tumultuados golpes e contra-golpes
políticos, em 1165, Rabi Mai­mon decide partir novamente, desta vez com
destino à Terra Santa.

A Terra Santa era na época o ponto mais cobiçado do ocidente nas.
disputas religiosas. As Cruzadas haviam conquistado o país e quando a
família de Maimônides chega lá o governador é o franco Amaury, homem
ambicioso e ávido de poder e. de riquezas. Contudo Nureddin, governador
muçulmano da região do Tigre, decide partir para a Guerra Santa contra
os cristãos das cru­zadas e faz uma aliança com o Egito e a Síria para
cercar a Terra Santa. É nesse momento que Maimônides se dirige para lá.

Maimônides desembarca na florescente cidade de São João de Acre em 16 de
maio de 1165, em cujas ruas se ouviam todos os idiomas do oriente e do
ocidente, e é asilado pelo Rabi Jafet, que presidia a vida das 200
famílias judias da cidade

Na Terra Santa as antigas tradições judaicas se haviam perpetuado, pois
tinham sido transmitidas de maneira ininterrupta dentro do país. Maimô-
nides descobre ali que a ordem das quatro partes do Pentateuco nos
Teffilin, tal como ele a havia aprendido em Córdoba, difere da ordem
estabelecida de acordo com a opinião de conhecidos Gaonin, que eram os
chefes das grandes academias da Babilônia, e de acordo com antigos
textos do Talmud, e decide então corrigir a ordem de seus Teffilin. Esse
fato é um acontecimento impor­tante, sobretudo por tratar-se ele de um
homem que dedica sua existência à pesquisa, à explicação e à
representação da lei judaica, pois é uma demonstra­ção de grande
humildade e de aceitação do conhecimento daqueles sábios no que se
refere à tradição.

Em outubro de 1165 Maimônides se dirige à Jerusalém para rezar dian­te
do Muro das Lamentações, e de lá ele vai para Hebron, a fim de rezar
sobre o túmulo dos patriarcas.

Mas não tardaria muito para que a depravação que ele observava nos
imigrantes estrangeiros que chegavam à Terra Santa o convencesse a
partir de lá, já que ele acreditava que "É inato no homem curvar-se, com
relação aos seus hábitos e atos, aos costumes dos países e dos amigos ou
companheiros que ele encontra ali". Todo homem, diz ele, deve procurar
assimilar os hábitos e a con­duta dos sábios. Para isso, é preciso que
ele faça tudo que estiver ao seu alcan­ce para viver junto aos justos e
afastar-se dos maus, a fim de não correr o risco de se integrar a eles e
de se sentir inclinado a agir como eles: "Se acontece de vivermos num
lugar onde os habitantes não seguem o caminho correto, é pre­ciso
imigrar para um lugar onde os habitantes sejam devotos e tenham bons
costumes."

Diante disso, ele parte com direção ao Egito onde, ao contrário do que
ocorria nos outros países muçulmanos, os judeus podiam contar com a
to­lerância religiosa dos califas Fatimitas. Dentre os 50.000 habitantes
do país na­quela época havia 3.000 famílias judias que viviam em paz e
gozavam de com­pleta liberdade civil e religiosa. Chegando em Alexandria
em 1166, a família de Maimônides se depara com uma cidade internacional,
que embora não mais fosse a capital do Egito nem a segunda cidade do
mundo, continuava sendo grande e bela, "cidade do comércio de todos os
povos", como descrevia o co­merciante Benjamim de Tudele, para onde se
dirigiam comerciantes tanto da Europa cristã, como do sul da Arábia, da
África do Norte e das Índias, e onde cada nação possuía seu próprio
armazém.

Mas Rabi Maimon pouco desfruta dessa merecida paz, depois de tan­tos
anos de peregrinação, pois vem a falecer poucos meses depois de terem
chegado ao Egito. Davi, seu filho mais moço, assume então a manutenção
ma­terial da família, dedicando-se ao comércio de pedras preciosas, e
liberando as­sim Maimônides dessa preocupação para que ele pudesse
continuar seus estudos.

Se as comunidades judias da Espanha e do Marrocos estavam amea­çadas de
extinção pela fé implacável dos Almohades, e as da Terra Santa pelas
cruzadas e pelos maus costumes, o conforto material e a comodidade em
que viviam os judeus do Egito também representavam uma ameaça, só que
neste caso para a vida intelectual, como soe acontecer toda vez que a
opulência se instala na existência do homem. Eles ali negligenciavam á
observação das leis religiosas, ignoravam os sábios e a falta de
conhecimentos se generalizava, pro­vocando a decadência religiosa, como
se podia observar pelo desenvolvimen­to e prosperidade da doutrina
caraíta no país.

Os caraítas constituíam uma seita judia separatista, que desprezava a
tradição oral legada pelas instituições rabínicas, e que se guiava ao pé
da letra pela Torah. Ao contrário do que ocorria nos outros países, onde
essa seita esta­va em vias de extinção, seu distanciamento da maioria da
comunidade judia fazia 
com que ela prosperasse no Egito, pois lá encontrava um ambiente
favo­rável, junto aos Maometanos, que acreditavam estarem os caraítas
mais próxi­mos do Islam do que os judeus seguidores das leis talmúdicas.
Vários judeus se juntaram aos caraítas e foram recompensados com favores
políticos, já que essa seita gozava da confiança dos Fatimitas xiitas. A
influência deles se fazia sentir, e aos poucos desapareciam na
comunidade os ritos tradicionais, fazen­do com que os próprios rabinos
se sentissem impotentes com relação ao pro­gresso dessa assimilação.

Como estimasse, por tudo isso, que os caraítas eram inadequados para
executar os deveres religiosos dos judeus tradicionais, Maimônides
come­çou a propôr uma cisão de cultos a fim de eliminá-los
definitivamente da vida religiosa judia. Isso provocou a cólera dos
caraítas a tal ponto que ele se viu forçado a partir de Alexandria.

Portanto, por volta de 1168, Maimônides parte para Fostat, onde se erige
atualmente a antiga cidade do Cairo. Dois anos depois de sua chegada,
ele já ocupa ali um rabinato e trabalha intensamente na ajuda aos
necessitados. Sabe-se, por exemplo, que em 1169 ele envia diversas
cartas circulares às co­munidades egípcias a fim de conseguir o dinheiro
para o resgate dos judeus que haviam sido aprisionados por Amaury,
governador franco de Jerusalém, para evitar que eles fossem vendidos
como escravos.

Depois de ter tentado afastar o perigo que os caraítas representavam
para a vida do judaísmo autêntico, Maimônides se lança na reforma dos
costu­mes no que se refere às preces feitas na sinagoga. Ele percebera
que enquanto o ``hazan'' fazia em voz alta, na sinagoga, a oração
silenciosa da comunidade, as pessoas conversavam ao invés de escutar com
recolhimento a oração. Con­siderando isso um desrespeito a Deus, ele
ordenou que se desse início a essa oração primeiramente em voz alta, e
que ela fosse seguida depois por todos em silêncio e com recolhimento,
em vez do que se fazia até então. Essa melho­ria, que ele ousa impor a
despeito da ordem da oração talmúdica, encontra o apoio e o
reconhecimento dos sábios contemporâneos e é aceita no Egito.

Incansável na sua busca da perfeição, ele deseja também unificar os
ritos dos dois grupos em que estavam divididos os judeus do Egito, ou
seja, os Babilonianos e os Jerusalemitas. Os Babilonianos dividiam a
Torah de ma­neira que ela pudesse ser lida completamente num ano
enquanto que os Jerusa­lemitas utilizavam um ciclo de três anos. Cada um
desses grupos tinha sua sina­goga e não tinha outros ritos em comum a
não ser o Simhat Torah e o Sche­vuoth. Essa oposição de usos dentro da
comunidade judia chocava Maimôni­des que, como judeu espanhol, estava
habituado a uma liturgia uniforme, fun­dada na ordem das orações de
Amram. A diversificação dos ritos pareceu-lhe imprópria, e embora ela se
apoiasse na tradição local, Maimônides sentiu que a lógica da lei e do
pensamento deveria substituir a comodidade dos hábitos. Contudo, ele se
defrontou aqui com a oposição de Zuta, o Naguid da época, homem
ambicioso que aproveitou a ocasião para afastar aquele estrangeiro
atre­vido e audacioso, dizendo que a reforma que Maimônides queria
instaurar no Egito devia ser considerada como um ato de hostilidade
contra o governo. Não resta então a Maimônides outra alternativa a não
ser abandonar temporariamente essa luta e afastar-se de Fostat por algum
tempo, tempo esse que ele aproveita para iniciar sua obra magna, o
\emph{Mishneh Torah.}

A morte de seu irmão Davi no naufrágio de um navio, por volta de 1171,
representa um grande golpe para Maimônides. Além de minar
irremedia­velmente sua saúde, essa grande dor desencadeia uma crise
decisiva em sua al­ma e acarreta uma mudança profunda em seu pensamento
e na sua visão do
mundo. Junto com o irmão desaparece também toda a fortuna da família, e
co­mo estimasse que nenhum sábio devesse viver às custas da comunidade
para poder prosseguir seus estudos, "pois nem na Torah, nem nos livros
posterio­res dos sábios não encontra alguma coisa que apóie essa tese"
--- como ele pró­prio diz ---, Maimônides decide então tornar-se médico
e ganhar dessa forma seu sustento. Ele dava, dessa forma, o exemplo e
recomendava também a to­dos os estudiosos sábios que ganhassem seu pão
graças a seu trabalho e não às custas da religião. Médico devoto,
escreve um juramento para todos os mé­dicos, judeus ou não, no qual
reafirma o dever que eles têm de curar, e faz uma oração para que Deus
lhes preste assistência e intervenha por eles. Essa oração diz o
seguinte:

"Oh Deus! O Senhor formou o corpo do homem com uma bonda­de infinita!
--- O Senhor uniu nele inúmeras forças que trabalham incessante­mente
como tantos instrumentos a fim de preservar em seu todo esta casa
ma­ravilhosa, contendo uma alma imortal, e essas forças atuam com toda a
ordem, concordância, e harmonia imaginável. Mas se fraqueza ou paixão
violenta per­turbam esta harmonia, estas forças agem uma contra a outra
e o corpo volta ao pó de onde ele veio. O Senhor então envia ao homem
seus mensageiros, as doenças, que anunciam a aproximação do perigo e
pede que se prepare para vencê-las. A Eterna Providência me apontou para
cuidar da vida e da saúde de Suas criaturas. Que o amor a minha arte me
deixe agir em todos os momentos, que a avareza e a mesquinhez, tanto
como a sede da glória ou da reputação, não tomem conta dos meus
pensamentos, porque sendo inimigos da verdade e da filantropia poderiam
me decepcionar e me fazer esquecer da minha meta de fazer o bem aos Seus
filhos. Me enriqueça com força de alma e mente, para que ambas estejam
prontas a servir ao rico e ao pobre, ao bom e ao mau, amigo e inimigo e
que jamais enxergue o paciente senão como um ser igual, adoecido.

"Se médicos mais cultos que eu desejam me aconselhar, me inspirar
confiança e obediência, aceito de bom grado, pois o estudo da ciência é
mara­vilhoso. Um só ser não pode enxergar tudo. Que eu seja moderado em
tudo, exceto na sabedoria da ciência; que eu seja insaciável até o ponto
certo; permi­ta-me ter sempre a força e a oportunidade de corrigir
minhas aquisições, sem­pre estendendo meu domínio, porque a sabedoria
não tem limite e o espírito do homem também se estende infinitamente,
para que diariamente se enrique­ça com novos conhecimentos. Hoje ele
pode descobrir os erros de ontem, e amanhã ele pode obter nova
iluminação sobre o que ele deu certeza hoje.

"Deus, o Senhor me apontou para cuidar da vida e da morte de Suas
criaturas; eis-me pronto para minha vocação."

Por volta de 1172, comovido e preocupado com a situação dos ju­deus que
viviam no Yemen, acossados que estavam pelos "confessores da uni­dade"
no oeste, pelos xiitas no leste, e desnorteados depois do anúncio que
havia sido feito, por um pobre e ingênuo lunático, da chegada do Messias
para breve, Maimônides lhes envia três cartas alentadoras, que ele
redige em árabe para que pudessem ser compreendidas por todos. Numa
delas ele diz o seguin­te: "Devemos ficar satisfeitos por sofrer todos
esses infortúnios, essas perse­guições, esse exílio, a perda de nossos
bens e as injúrias de todos, pois todas essas misérias são uma honra que
Deus nos concede". Ele diz ainda que todo o mal que se sofria era um
sacrifício que se levava ao altar e lhes recorda que Deus havia
prometido que nenhuma opressão duraria muito tempo e que seu povo nunca
seria destruído. Afirma também que o que eles estavam passando naquele
momento não era um sofrimento, e sim um mal preliminar que anun­ciava o
reino do verdadeiro Messias. Maimônides acredita que a inveja é a ver-
dadeira motivação que leva os outros povos a perseguir e a oprimir
constante­mente os judeus: não podendo atacar-se ao Todo Poderoso por
ter Ele escolhi­do o povo de Israel como herdeiro e guardião de Seu
estatuto e de Sua doutri­na, eles se lançam contra o povo em si, numa
Guerra Santa que dura desde os tempos de Amalec.

Uma vez concluído seu \emph{Comentário sobre a Mishná,} Maimônides
concebe a \emph{Mishná Torah,} uma obra que o ocuparia de 1170 a 1180, e
que de­veria guiar os leitores através das obscuridades e das
imprecisões do Talmud, no qual está contida a vida interior e exterior
do judaísmo.

O Talmud (tanto o da Babilônia quanto o de Jerusalém) se desenvol­veu
durante séculos como o comentário da Mishná. Uma enorme quantidade de
opiniões e de novos conhecimentos foram expressos ali e fixados como
con­tinuação do texto da Mishná. Esse trabalho se terminou no século 5,
mas ficou rapidamente demonstrado que o povo era incapaz de compreender
esse alto ensinamento e é por isso que apenas um pequeno número de
pessoas se consa­grava ao estudo do Talmud. As perseguições e os
sofrimentos que vinham cas­tigando os judeus anos a fio os obrigavam a
deixar os estudos religiosos cada vez mais em segundo plano e a dar
prioridade à preservação das próprias vidas; a sapiência dos sábios e o
raciocínio dos filósofos se perdiam, as explicações sobre o Talmud que
os Gaonim davam, e que eles julgavam estar ao alcance de todos,
começavam a não mais ser compreendidas, assim como os próprios textos do
Talmud, da Sifrá, dos Sifris e da Toseftá, pois a compreensão dessas
obras exige uma grande inteligência, uma alma preparada e extensos e
aprofun­dados estudos.

Maimônides constata que o povo em si não tinha a sua disposição um
código onde pudesse encontrar regras seguras, sem mistura de
controvér­sias e de opiniões. Assim sendo, ele deseja expor em sua obra,
numa linguagem clara e breve, o que é proibido e o que é permitido, o
que é puro e o que é impuro, bem como tudo o que se refere às questões
da Torah, tudo isso para que a lei possa ser conhecida por todos, sem
deixar dúvidas. Ele quer que a Lei esteja, em palavras claras, na boca
de todos os homens e faz sua exposição de maneira direta e didática,
buscando dar um esclarecimento simples e satisfa­tório a questões que,
de outra forma, poderiam ser interpretadas erroneamente pelo povo. É
assim, por exemplo, que ele explica que o preceito que nos orde­na temer
a Deus implica, na realidade, não no medo do Eterno, mas sim no respeito
ao nome de Deus, na reserva em utilizá-lo, no cuidado para não come­ter
uma blasfêmia, glorificando-o e louvando-o cada vez que ele for
pronuncia­do. Portanto, o temor ao Eterno significa a obrigação de
santificar Seu nome e de estar sempre alerta para não profaná-lo, o que
envolve, entre outras coi­sas, o dever de se deixar matar antes de
renegar o Senhor em benefício de um deus pagão ou antes de entregar aos
gentios um de seus irmãos israelitas para que ele seja morto ou
desonrado. Santifica também o nome do Senhor aquele que "se afasta de
toda transgressão ou observa os preceitos sem ser levado a fazer isso
por alguma consideração de ordem profana, terror, temor, ou busca de
reconhecimento", ao passo que profana o santo Nome "todo aquele que
transgride espontaneamente e na ausência de qualquer tipo de pressão,
por des­dém e com o intuito de escandalizar, nem que seja apenas um dos
preceitos anunciado pela Lei". Assim, com este que pode ser considerado
um código me­tódico de referência para o dia a dia, Maimônides tem a
convicção de poder levar seu leitor a descobrir, passo a passo, qual é a
atitude correta e qual é o caminho a ser seguido para alcançar a
perfeição do corpo e da alma que o Se­nhor espera de nós.
A grande dificuldade para a execução de sua obra estava principal­mente
no fato de que não havia nenhuma preparação anterior que pudesse
aju­dá-lo em seu trabalho pois, de acordo com suas palavras, "A
sapiência dos sá­bios de nossa época consiste em julgar a verdade de uma
sentença não de acor­do com seu conteúdo, e sim de acordo com sua
conformidade com a sentença de um predecessor, sem examiná-la."
Maimônides decide então executar sua obra como uma codificação, como um
resumo sistemático, e não como um co­mentário, que era o que o Talmud
fazia com relação a Mishná. Decide também não citar as opiniões
discutidas e refutadas, mas sim fornecer apenas as deci­sões que têm
força de lei. Ele deseja expor todas as doutrinas da Mishná e do Talmud,
sem dar o nome do autor de cada uma delas após cada citação, mas dizendo
apenas que todas as frases da Torah que constituem a lei oral haviam
sido transmitidas por este e aquele, desde Ezra e desde nosso mestre
Moisés.

Seguindo fielmente a tradição judaica, Maimônides se limita estrita e
logicamente ao Talmud, tomando por lei o que ele encontra explicitamente
decidido. Como na maioria dos casos as questões estão dissertadas, sem
apre­sentar uma recomendação firme e final, ele próprio as resolve, e é
exatamente isso o que há de mais importante em seu trabalho, não a
compilação em si. Os princípios e os métodos que ele aplica, com uma
lógica minuciosa, em suas de­cisões pessoais, são de uma elevação de
pensamento que prepara um terreno novo para todos os séculos seguintes.
Ele distingue no Talmud os elementos halachicos obrigatórios e os
elementos hagádicos, que não o são. Isso lhe per­mite ter uma
independência de opinião completa em relação às decisões dos sábios
talmúdicos que não eram de origem religiosa, sempre que ele não pôde
prová-las cientificamente. O equilíbrio resultante da independência e da
fideli­dade de seu espírito original está repleto da mais legítima e
autêntica autorida­de e é a obra-prima de sua atitude intelectual.

Sua objetividade científica faz com que ele atribua a mesma impor­tância
a todas as matérias da Halacha, sem levar em conta sua relação com a
atualidade. Ele fala sobre todos os preceitos, inclusive aqueles que não
estão mais em aplicação desde a destruição do Templo, e lastima que
ninguém mais se interesse em pesquisar ou conhecer essas leis, pois
assim elas terminam por ser esquecidas.

Maimônides tem consciência de escrever um livro definitivo: "Nin­guém
terá necessidade de ajuda para conhecer a lei judaica, se ele tiver
minha obra que forma uma coletânea completa de todas as instituições,
usos e decre­tos, desde Moisés até o fim da redação do Talmud, incluindo
as explicações posteriores dos Gaonim", e é por isso que ele intitula
seu código de ``Mishneh Torah'', que significa ``repetição da lei''. Isso
representa uma revolução e uma reforma no ensino da religião: quem
estudasse primeiro as Escrituras e depois o Código de Maimônides
conheceria toda a doutrina da tradição oral e poderia, em tese, se
abster de pesquisar em outras obras. Através da utilização de seu código
ganha-se o tempo que outrora se dedicaria ao estudo do Talmud, tem­po
esse que podia ser consagrado aos estudos filosóficos. Maimônides deseja
despertar o interesse dos estudantes pela posição filosófica do problema
para depois dirigir seus pensamentos para a metafísica, pois da mesma
forma que ele também prefere a análise contemplativa ao julgamento das
ações, ele prefe­re o estudo da metafísica, ``raízes da doutrina'', ao
estudo da dialética do Tal­mud, ``ramificações da doutrina'', embora ele
considere indispensável as par­tes dialéticas do Talmud e considere que
as partes hagádicas são a fonte da ciência filosófica.

Ele abre mão da discussão, que é o que acontece no Talmud, em
favor da decisão. Na ciência talmúdica as pesquisas analíticas não são
levadas em consideração; sua pedagogia inculca "a doutrina pela
doutrina" e seu estu­do conduz a uma teoria, e não a uma decisão
essencial para a prática. É esse aspecto do ensino e dos métodos de
pensamento que Maimônides pretende reformular.

Aquela era a primeira vez na história que um homem ousava reco­lher numa
só obra a totalidade que constitui a ciência hebraica. As tentativas
feitas nesse sentido até então haviam fracassado, devido a imensidade de
mate­rial disperso existente. O caráter de sua codificação consiste em
citar a idéia em vez do acontecimento, a lei em vez do caso, a coisa em
vez das pessoas, trocando a história pela especulação, a situação
concreta pela abstração.

Maimônides utiliza um método novo para a forma do livro. Ele de­via
repartir um grande número de prescrições, de leis e de decisões
especiais nos 613 ``compartimentos'' de seu livro, correspondentes cada um
deles a um dos preceitos derivados da Torah. Resolve, então, redigir 613
parágrafos, orde­nando-os em 83 seções e distribuindo-os em 14 livros.
Para que sua grande obra pudesse ser lida com maior clareza e para
preservar a unidade do conjunto, Mai­mônides decide escrever um prefácio
onde apresentaria um resumo dos "Tar­yag Mitzvoth", ou dos 613 preceitos
divinos mas, ao estudar as enumerações já existentes desses preceitos,
chega a conclusão de que nunca se haviam esta­belecido normas claras
para se chegar à classificação do que era e do que não era um preceito,
e de que muitas vezes o conteúdo havia cedido lugar à forma. Sendo os
``613 preceitos'' o fio condutor da vida e da crença do povo judeu, já que
eles representam os desejos do Eterno expressos na Torah, era de
supre­ma importância que o povo soubesse com clareza e com precisão
quais eram esses desejos para que pudesse cumpri-los.

Assim sendo, Maimônides inicia a redação de seu \emph{Livro dos
Precei­tos Divinos} ("Sefer Ha-Mitzvoth") com o intuito de esclarecer e
ordenar os 613 preceitos, e de dirimir as dúvidas dos leigos. Esse
livro, escrito em linguagem precisa e acessível, propeorcionava também
aos jovens da época um caminho mais fácil para adquirir algum
conhecimento a respeito dos preceitos não mais em aplicação e das coisas
do Templo e do Santuário, pois Maimônides estava preocupado com o fato
de que não só eles ignoravam esses aspectos da Tradi­ção, como também
não demonstravam nem ao menos alguma curiosidade ou interesse pelas
coisas do passado.

Maimônides divide o livro em 2 partes. Na primeira ele estabelece os 14
Fundamentos ou os princípios lógicos que utiliza para determinar o que é
e o que não é um preceito. Na segunda ele apresenta de maneira detalhada
os 613 preceitos, divididos em 248 positivos e 365 negativos. Ao longo
de to­do o livro Maimônides se empenha em ir esclarecendo seu
raciocínio, cita fon­tes, argumentos, a literatura rabínica, sem no
entanto deixar de apresentar sem­pre uma conclusão clara e decisiva,
para que não restem dúvidas ao leitor quanto ao que deve e ao que não
deve ser feito.

``A grosso modo'', os princípios são apresentados agrupados de acor­do com
os assuntos neles tratados. Assim, vemos agrupadas as obrigações ---bem
como as proibições --- referentes ao homem para com Deus, para com seus
semelhantes, para com sua família, às relativas ao Templo, às impurezas,
às festas religiosas, ao cultivo da terra, à justiça, ao estado etc.

\emph{O Livro dos Preceitos Divinos} serviu de base para muitas outras
obras, inclusive para o \emph{``Livro do Ensino''} (Sefer Ha-Chinuch), a
grande obra do tal­mudista espanhol.do século XIII Aaron Ha-Levi, que
trata especificamente des­ses preceitos.
Uma vez concluído esse livro, em 1170, Maimônides pode então vol­tar a
dedicar-se de corpo e alma ao \emph{Mishneh Torah,} obra que foi
recopiada por escribas profissionais e que se espalhou pelo mundo
inteiro, conquistando sá­bios, estudiosos, rabinos e juízes. Várias
comunidades o adotaram como códi­go. O Talmud era uma obra vasta e
complexa; o \emph{Mishneh Torah} era escrito de maneira ordenada e
clara, com uma linguagem fácil, onde os leitores encontra­vam a verdade
e aprendiam o que a doutrina da moral encerrava de mais pro­fundo.
Maimônides deixou o \emph{Mishneh Torah} como seu grande legado
organi­zador que iria exercer uma influência definitiva na vida de seu
povo. Há muito tempo a voz de um homem não tinha tal influência sobre os
judeus.

Contudo, em breve começam a aparecer os opositores de Maimôni­des, que
colocam em questão sua autoridade como legislador. Eles censuram a obra
de Maimônides porque ele expõe as regras legais sem citar a origem, sem
dar o nome do autor, sem provas nem peças de apoio.. Até então só se
reconhe­ciam como obrigatórias as decisões dos Gaonim pois elas se
apoiavam não ape­nas nas suas qualidades pessoais, mas também na
autoridade das academias, e os judeus reconheciam a instituição e não a
pessoa. Pesa ainda contra ele o fato de ter escrito, na introdução de
sua obra, que depois de ter estudado seu Códi­go se poderia renunciar ao
estudo da literatura pós-bíblica. Isso foi interpreta­do como uma
tendência a afastar o Talmud das escolas, o que representava uma
profanação e um perigo. Maimônides deixa passar a ocasião de contestar a
tem­po essa falsa interpretação de sua doutrina e acaba por começar a
ser conside­rado por alguns como herege!

Depois de ter mantido uma correspondência preliminar com o grande
mestre, na qual solicitava ser aceito como discípulo e poder estudar com
ele, em 1185 chega a Alexandria o jovem Yossef Ibn Yehuda. Maimônides,
que an­tes nunca se havia interessado pelo ensino público e imediato, e
cuja necessi­dade de ensinar estava voltada mais para a redação do que
para a exposição oral, decide no entanto transmitir seus ensinamentos a
esse rapaz que viera de Ceuta, e que acaba por suscitar a simpatia do
mestre. A decisão de concordar em transmitir seus ensinamentos a um
aluno se prendeu principalmente ao fato de que o número dos que se
interessavam pela ciência e pela filosofia no Egito era muito reduzido.
Há muito ele esperava por uma pessoa a quem pudesse trans­mitir seus
ensinamentos e descobertas, e a afinidade e a afeição que ele acaba
desenvolvendo por Ibn Yehuda é tão grande que ele passa a chamá-lo de
``meu filho''.

Esse jovem, para quem Maimônides era o representante de uma ciência mais
elevada, se interessava sobretudo pela questão da interpretação das
Escrituras e desejava iniciar-se aos mistérios da ciência superior.
Embora tendo durado pouco menos de dois anos, o convívio com Ibn Yehuda
foi muito esti­mulante para Maimônides, que procura fazer com que esse
jovem entusiasta e impaciente por atingir a ``ciência interior'' seguisse
o caminho do estudo me­tódico e lento. Ele lhe diz, numa carta: "E
quando você fez comigo seus estu­dos de lógica, eu depositava minha
esperança em você, e o julgava digno de revelar-lhe os mistérios dos
livros proféticos, para que você compreendesse aqui­lo que os homens
perfeitos devem compreender."

Contudo, antes que Maimônides chegasse a introduzir seu discípulo aos
assuntos metafísicos, este se muda para Alep, por motivos ignorados. Mas
não parte sem antes obter de seu mestre a promessa de que ele redigiria
um tratado no qual responderia as questões que o preocupavam e cuja
resposta ti­nha ido buscar junto a ele.

E é assim que, para cumprir a promessa feita ao seu único discípulo
Maimônides redige, de 1187 a 1190, e não sem grande hesitação, a sua
última obra, o \emph{Guia dos Perplexos,} um trabalho que deveria
"explicar os pontos obs­curos da lei e manifestar o verdadeiro sentido
das alegorias, que estão acima da inteligência comum". Esse é um
trabalho difícil para ele, pois trata-se de des­vendar os mistérios que
lhe parecem invioláveis, e se algumas considerações incidentais em suas
obras anteriores já haviam provocado grandes oposições, o que não
poderia resultar da apresentação completa de suas concepções
filo­sóficas? Ele chega, no entanto, à conclusão de que, apesar de tudo,
é importan­te que o livro seja escrito, e explica nele porque tomou essa
decisão: "... eu sou o homem que, vendo-se apertado numa arena estreita
e não encontrando a forma de ensinar uma verdade bem demonstrada, a não
ser de uma maneira que convenha a um só homem notável e que desagrade a
dez mil ignorantes, prefere falar para essa única pessoa, sem prestar
atenção à reprovação da gran­de multidão, e espera tirar esse único
homem notável da confusão em que ele caiu e mostrar-lhe o caminho para
sair de sua perplexidade, a fim de que ele se torne perfeito e de que
obtenha o repouso.' Num outro trecho dessa mes­ma obra ele diz também
que "A verdade não se torna mais verdadeira pelo fato de todo mundo
acreditar nela, tampouco pelo fato de todo mundo discordar dela".

Ele estava preocupado com o desinteresse dos judeus pela filosofia. A
maioria deles sentia uma separação entre a fé e o saber, entre o
conteúdo da revelação e as doutrinas filosóficas. Para Maimônides, no
entanto, a Hagada é também uma das fontes da ciência filosófica: tudo o
que ali se encontra em forma de parábolas coincide com os ensinamentos
da filosofia abstrata. Com o \emph{Guia
dos Perplexos} Maimônides torna possível o acesso da razão àqueles
as­pectos da Torah que não estão ao alcance da capacidade humana. Ele
deseja guiar "o homem religioso, no qual a verdade de nossa lei está
estabelecida na alma e se tornou um objetivo de crença, que é perfeito
na sua religião e nos seus costumes, que estudou as ciências dos
filósofos e conhece os diversos as­suntos delas, e que foi atraído e
guiado pela razão humana, para fazê-lo entrar em seu domínio". Sua
obra-prima filosófica se destina, portanto, apenas aos sábios e aos
estudipsos, pois os leigos e os ignorantes jamais poderiam com­preender
as revelações nela contidas. Na introdução desse livro ele diz: "Meu
pensamento vai guiá-los no caminho do verdadeiro e vai torná-lo mais
fácil. Venham, caminhem pela sua senda, vocês que vagam no campo da
religião! O impuro e o ignorante não passarão por ele; ele será chamado
de caminho secreto."

Essa obra visava, em primeiro lugar, proteger os estudiosos das
co­munidades judias da sedução que as filosofias árabe e grega exerciam
no século XII, e a grande originalidade de Maimônides nesse trabalho foi
a de estabelecer um diálogo entre o mosaísmo e a filosofia, ao invés de
se limitar a utilizar-se de seus conhecimentos filosóficos para fazer a
apologia do judaísmo. Ele não renuncia a nenhuma das tradições do
pensamento judeu, nem tampouco ali­menta a ilusão de poder ``conciliar'' a
verdade bíblica e a verdade filosófica. Ao invés disso, ele confronta as
duas tradições, de maneira a sobrepô-las. As­sim, ele se outorga a
missão de guiar os estudiosos para o conhecimento meta­físico o qual,
segundo ele, é uma possessão original do judaísmo que havia sido perdida
durante o exílio, e é essa perda que torna o exílio tão trágico. Ele tem
a convicção de que o renascimento da compreensão mais elevada, obtida
gra­ças a introdução da filosofia nos estudos religiosos, é o fato
libertador que con­duzirá ao acontecimento messiânico, teoria essa que,
aliás, acredita-se ser ele o primeiro a introduzir naqueles tempos de
exílio.
Seu \emph{Guía dos Perplexos} se espalha extraordinariamente na
literatura mundial e seu sistema é de grande importância para Nicolau de
Coues, Leibniz e Spinoza. A negação da tese da eternidade do mundo em
favor de sua perpe­tuidade, a modificação da doutrina de Aristóteles
sobre Deus, sua profetologia independente, sua explicação do significado
dos preceitos divinos e seu méto­do de explicação da Bíblia entram no
sistema de pensamento dos escolásticos cristãos como Tomás de Aquino.
Alexandre de Halle, Alberto, o Grande e Iná­cio de Loyola também
aceitaram suas teorias como elementos de seus sistemas.

Embora ciente de que essa obra não esclareceria todas as dúvidas,
Maimônides espera que ela afaste as principais. A aspiração a conhecer
Deus, necessidade de seu pensamento presente desde sua juventude, faz
com que ele se lance à procura de um sistema metafísico. Ele adota a
razão como método de meditação, esperando chegar ao conhecimento de Deus
pela eliminação de negativas e das imperfeições. A própria existência do
indivíduo se torna para Maimônides o ponto de partida para o
conhecimento e para a compreensão de Deus. Ele conclui que "Não há na
realidade no ser nada além de Deus e to­das as suas obras; estas são
tudo o que o ser encerra fora d'Ele. Não há nenhu­ma outra forma de
conceber Deus a não ser por Suas obras; são elas que indi­cam Sua
existência e o que se deve crer a Seu respeito".

Uma questão que preocupa muito Maimônides é como ascender à profecia, o
mais alto grau de perfeição possível a um homem.,Ele faz no seu
\emph{Guia} a exposição de seu ponto de vista a esse respeito, e diz que
ela é um dom que Deus concede, quando Ele quer, a alguns poucos homens
que Ele escolhe dentre aqueles que possuem os três requisitos básicos: a
perfeição do pensa­mento, atingida através da especulação e da
concentração da inteligência no estudo e na busca do conhecimento de
Deus; a perfeição da imaginação, que foge ao controle do homem e que
depende de um cérebro perfeitamente for­mado; e, por último, a perfeição
do caráter e da moral, que se alcança libertando-se o espírito do desejo
de se obter satisfações terrenas e erradicando a ambição pelo domínio e
pelo poder. Essas condições são indispensáveis porque a reve­lação
profética é uma emanação de Deus que se manifesta, a um homem justo,
primeiramente por intermédio do ``intelecto ativo'', em relação a
faculdade de pensar, e depois em relação a imaginação.

Se essa emanação divina atingir unicamente a faculdade racional do
homem, seja porque sua faculdade imaginativa não foi perfeitamente
formada, seja porque a emanação foi insuficiente para alcançar a
faculdade imaginativa desse homem, ele pertencerá à categoria dos sábios
que dedicam suas vidas ao estudo e às pesquisas. Se ela agir apenas
sobre a faculdade imaginativa, seja por­que a faculdade racional é
primitiva ou porque ela foi pouco exercitada, o que teremos é um homem
que pertence ao grupo onde se classificam os legislado­res, os
adivinhos, os ocultistas, os videntes e os feiticeiros. Essas pessoas
têm sonhos e visões semelhantes aos dos profetas, e isso faz com que
eles acredi­tem ser um deles. De acordo com as palavras de Maimônides,
elas pensam "que adquiriram ciências sem ter estudado, e provocam uma
grande confusão nas coisas importantes e especulativas, misturando, de
uma maneira surpreenden­te, as coisas verdadeiras e as quimeras. Tudo
isso ocorre porque sua faculdade imaginativa é forte, enquanto a
faculdade racional é fraca e não obteve absolu­tamente nada". Mas se
essa emanação divina se espalhar pela faculdade racio­nal e daí passar à
faculdade imaginativa, tendo ambas atingido a perfeição que explicamos
anteriormente, esse homem estará na categoria dos verdadeiros
profetas. E como recebe o homem uma profecia? À exceção de Moisés, a
quem

Deus falou diretamente, os profetas recebem essas revelações através de
um mensageiro de Deus, um anjo, que lhes fala num sonho ou numa visão,
situa­ção esta em que seus sentidos ficam paralisados e a emanação
divina se espalha por sua faculdade racional, passando daí para a
faculdade imaginativa, fazendo com que ela se aperfeiçoe e entre em
funcionamento. Maimônides escreve que "Às vezes a revelação começa por
uma visão profética; depois essa agitação e essa forte emoção
decorrentes da ação perfeita da imaginação, vão aumentan­do e então
ocorre a revelação".

Convencido, portanto, de que "o vigor humano do espírito não é
suficiente para se alcançar o conhecimento de todas as coisas",
Maimônides crê que a profecia é a única capaz de decifrar todas as
verdades para a limitada inteligência humana, e de conduzir à
compreensão da criação do mundo e da origem das espécies. Para ele o
sábio está no cume da hierarquia humana, uma vez que este se prepara
para receber as revelações de Deus: "pois apenas aque­le que chegou a
perfeição especulativa pode obter a seguir outros conhecimen­tos, quando
o intelecto divino se extravasa sobre ele." O verdadeiro profeta é
aquele que prevê os acontecimentos, com absoluta verdade, a médio e a
lon­go prazo, e cujas previsões de coisas boas sempre se realizam.
Contudo, caso o profeta tenha previsto um mal que não venha a ocorrer,
isso não significa que ele não é mais um profeta, mas sim que Deus, na
sua infinita bondade, teve misericórdia de seu povo e mudou o mal em
bem.

Para ilustrar seu pensamento a esse respeito ele conta no seu
\emph{Guia} uma parábola através da qual afirma que o estudo da
literatura talmúdica, fun­dada na tradição e sagrada para todos os
judeus, não é a chave que abre a porta do ``palácio'' (onde vive o
``soberano'', ou seja, Deus), mas que filosofar a esse respeito,
aprofundando-se cada vez mais nesse estudo, é o caminho que abre essa
porta e que conduz a Deus. Sem dúvida, essa nova ordem das potências
espirituais é extremamente audaciosa, sobretudo vinda de um homem que
con­sagra o melhor de sua existência ao estudo das tradições talmúdicas,
e só pode ser legítima se corresponder a um autêntico desejo dele de
estabelecer uma no­va ordem das coisas. Essa nova ordem é o coroamento
da obra de Maimônides.

Quando escreveu seu \emph{Comentário sobre a Misbná,} ele pensava ain­da
que tudo o que existe no universo só existe em função dos homens.
Contu­do, essa concepção antropocentrista vai se dissipando ao longo dos
anos para dar lugar à crença de que o homem em si não pode ser imaginado
como o últi­mo objetivo do desenvolvimento universal, de que a espécie
humana é bem pouco com relação ao mundo superior das esferas e dos
astros, e de que não há nada que indique que o mundo só existe por causa
do homem: "... se a Ter­ra inteira não passa de um ponto imperceptível
com relação à esfera das estre­las, qual será a relação da espécie
humana com o conjunto das coisas criadas? E como então alguém dentre nós
poderia imaginar que elas existem em seu fa­vor e por sua causa, e que
elas devem servir-lhe como instrumentos?"

Essa nova concepção do homem como mero componente de um cosmo ordenado e
perfeito abre um caminho que o leva a encontrar a explica­ção filosófica
do mal, explicação essa que ele renunciara a descobrir aos 25 anos,
dando-se como justificativa o fato de que os sábios que o haviam
precedido também não o haviam conseguido. Agora, vários anos após a
morte de seu ir­mão David, a impetuosidade de sua juventude desaparece,
deixando lugar a uma calma e ponderação maiores, e essa maturidade de
espírito lhe traz console e paz de espírito. Maimônides destrói o mal
filosoficamente e conclui que sua ocorrência não passa de exceções
esporádicas, quando se leva em conta a har­monia de toda a criação
divina: "Todo ignorante imagina que o universo in-
teiro só existe em favor de sua pessoa, como se não houvesse nele nenhum
outro ser a não ser ele próprio. Portanto, se o que lhe acontece é
contrário aos seus desejos, ele julga que todo o ser é o mal; mas se o
homem considerasse e concebesse o universo, e se soubesse quão pequeno é
o lugar que ele aí ocu­pa, a verdade lhe apareceria claramente".

Maimônides passa a definir o mal como uma simples privação, co­mo algo
que aparece em decorrência da ausência de uma ação. Ora, como uma ação
só pode resultar em algo que venha a existir e não numa privação, o mal
não pode ser resultante de uma ação direta se essa ação não existe, e
portanto esse mal também não existe. Caso assim não fosse, como diz
Maimônides, "se alguém produz uma matéria incapaz de receber determinada
capacidade, poder-se-ia dizer que ele fez tal privação; da mesma forma,
se alguém tivesse sido ca­paz de salvar uma pessoa da morte, mas tivesse
se abstido de fazê-lo e não a tivesse salvo, poder-se-ia dizer dele que
ele a matou". Ele conclui que o bom é o ser, aquilo que existe, e o ruim
é o não ser: "No homem, por exemplo, a morte é um mal e é sua não
existência; da mesma forma, sua doença, sua po­breza, sua ignorância são
males em relação a ele e são privações de capacidade ... A destruição
nada mais é do que a privação da forma".

Ele classifica os males em três tipos: o primeiro é o que recai sobre
determinados indivíduos, devido a nossa própria natureza humana, e que é
o responsável pelas doenças e deformações congênitas ou resultante de
altera­ções ocorridas na natureza, tal como os terremotos, conforme ele
próprio exem­plifica. Esses males são resultantes naturais da própria
imperfeição da matéria da qual o homem é feito, pois "a coisa mais
eminentemente perfeita que possa formar-se a partir do sangue e do
esperma é a espécie humana, com sua bem conhecida natureza de ser vivo,
racional e mortal". Mas levando-se em conta o total da humanidade em
todos os tempos, tais tipos de males não passam de meras exceções. A
segunda categoria abrange os males que os homens se infli­gem uns aos
outros, tal como a tirania e a barbárie provocados, entre outras causas,
pelas paixões e pelas divergências de opiniões e de crenças; segundo
Maimônides: "Esses grandes males ... são todos decorrentes de uma
privação, pois todos eles resultam da ignorância, ou seja, da privação
da ciência ... por­que o conhecimento da verdade faz cessar a inimizade
e o ódio e impede que os homens se façam mal uns aos outros ...". E há,
finalmente, os classificados no terceiro tipo, que são os mais
freqüentes e que constituem os males que acontecem aos seres humanos
como decorrência de seus abusos dos prazeres mundanos, tal como a
bebida, a comida e a luxúria. Ele escreve que "a maioria dos males que
atingem os indivíduos provêm deles próprios, quero dizer, dos indivíduos
humanos, que são imperfeitos. Se sofremos é por causa dos males que nós
mesmos nos infligimos espontaneamente, mas que atribuímos a Deus ..."
Este último tipo de mal é o mais nocivo, já que além de atingir o corpo
ele pre­judica também a alma, seja porque, sendo uma força corporal, ela
é influencia­da diretamente pelas alterações ocorridas no corpo, ou
porque a alma acaba por se familiarizar com o supérfluo e a se habituar
a ele, levando o homem a desenvolver uma ambição sem termo que o faz
buscar riquezas e grandezas não essenciais e totalmente desnecessárias.

Maimônides explica também no \emph{Guia dos Perplexos} que toda vez que
Deus deseja manifestar sua vontade, seja para desencadear algum
aconteci­mento, seja para intervir no desenvolvimento dos fatos, Ele o
faz por meio dos anjos, as ``Inteligências separadas'' que Ele cria e das
quais Ele se serve para reger o universo. Assim sendo, os anjos são
todas as forças propulsoras e todas as faculdades, tal como a força
formadora, que existe no esperma e que dá formas 
ao feto, e como a faculdade que faz com que um animal aja de uma
deter­minada maneira, num dado momento, de acordo com os desígnios de
Deus, assim como se vê na citação que Maimônides faz das Escrituras:
"Meu Deus en­viou Seu anjo e fechou a goela dos leões, que não me
fizeram nenhum mal" (Dan. 6:22).

As Escrituras dizem também que os anjos formam um terço do uni­verso.
Isso significa que eles são uma das três coisas criadas por Deus e
existen­tes fora d'Ele, a saber: as Inteligências separadas (ou anjos),
os corpos das esfe­ras celestes, e a matéria que se encontra abaixo das
esferas celestes, e da qual é feito tudo o que existe no nosso mundo. As
Inteligências separadas, elemen­tos incorpóreos criados por Deus são os
intermediários entre Deus e todos os corpos celestes e fonte de
extravasamento de benefícios e de luz para os cor­pos das esferas
celestes. Quanto a estas esferas, Maimônides diz que elas são entidades
que possuem uma alma, no sentido de que elas foram postas em mo­vimento
pelas Inteligências criadas por Deus e que elas desenvolvem o desejo de
mover-se eternamente em círculos, o ponto mais alto da perfeição que um
corpo possa alcançar, já que esse é o movimento perpétuo.

De acordo com Maimônides, as esferas celestes são em número de nove: uma
que engloba tudo, uma onde estão as estrelas fixas, e uma esfera para
cada um dos sete planetas existentes (de acordo com os conhecimentos da
época). Cada uma dessas esferas também emana benefícios e forças que
re­gem a matéria de nosso mundo, ``o mundo do nascimento e da corrupção'',
como diz Maimônides. Segundo os filósofos, temos a evidência da
influência que a lua exerce sobre as águas e da que os raios do sol
exercem sobre o ele­mento do fogo e do calor. Baseado nisso, ele chega à
conclusão de que cada esfera "pode possuir um dos quatro elementos de
tal forma que tal esfera seja o princípio de força de tal elemento em
particular, ao qual, graças a seu próprio movimento, ela dê o movimento
do nascimento. Assim, portanto, a esfera da lua seria o que move a água;
a esfera do sol o que move o fogo; a esfera dos outros planetas, o que
move o ar (e seus movimentos múltiplos, sua desigual­dade, seu recuo,
sua retidão e sua estação produzem as diversas configurações do ar, sua
variação e sua rápida contração e dilatação); e, finalmente, a esfera
das estrelas fixas seria o que move a terra, e é talvez por isso que
esta última se move com dificuldade, por receber a impressão e a
mistura, porque as estre­las fixas têm o movimento lento".

Ele conclui também que é como se todas essas forças constituíssem a
força de um só corpo, já que o universo todo é um único indivíduo. Tudo
o que nele existe é elaborado não a partir de um ato concreto e
particular, mas sim a partir do que Maimônides chama de "extravasamento
divino", a fonte inesgotável de bondade, de criação e de continuidade do
universo, que se es­palha como numa cascata, derramando-se primeiro
sobre os anjos, que extra­vasam seus benefícios sobre as esferas
celestes as quais, por sua vez, os extrava­sam sobre os corpos
perecíveis. De acordo com ele, "... tal como foi demons­trada a
incorporeidade do Criador, e tal como foi estabelecido que o universo é
obra Sua e que Ele é sua causa eficiente, ... foi dito que o mundo vem
do extravasamento de Deus e que Deus extravasou sobre ele tudo o que
nele ocorre. Da mesma forma, foi dito que Deus extravasou sua ciência
sobre os profetas. Tudo isso significa que essas ações são a obra de um
ser incorpóreo, e a ação de um ser assim é chamada de extravasamento".

Com a mesma clareza que Maimônides utiliza para explicar que o
"ex­travasamento" divino é o ponto de origem e de renovação de todas as
coisas, corpóreas ou incorpóreas, ele escreve também um capítulo no qual
faz a de-
claração, espantosa a princípio, de que Deus não tem nenhum poder sobre
o impossível. ``O impossível'', explica ele, "tem uma natureza estável e
constan­te que não é obra de um agente e que não varia sob condição
alguma; é por isso que não se pode atribuir a Deus nenhum poder a esse
respeito". Ele escla­rece, a seguir, que já que as coisas impossíveis
não são obra de um agente e que já que a existência delas é
inadmissível, o fato de que Deus não tenha po­der sobre elas não pode
significar, conseqüentemente, nenhum tipo de fraque­za por parte d'Ele.
As divergências que possam surgir entre os pensadores a es­se respeito
limitam-se à classificação do que eles consideram como possível e como
impossível. Assim, todos concordam em identificar como impossível a
união de contrários no mesmo instante e sobre o mesmo assunto, a
transfor­mação da substância em acidente e do acidente em substância, a
existência de uma substância corporal sem acidente, e o fato de que Deus
crie um ser seme­lhante a si próprio ou que Ele se corporifique, ou
ainda que se transforme. As divergências aparecem quando se trata de
saber, por exemplo, se é possível pro­duzir algo que possua um corpo,
sem para isso servir-se de uma matéria pré-exis­tente. ``No entanto'', diz
ele, "está claro que, segundo todas as opiniões e to­dos os sistemas, há
coisas impossíveis cuja existência é inadmissível e a respei­to das
quais não se pode atribuir poder a Deus".

Ainda no seu \emph{Guia dos Perplexos} Maimônides faz uma revelação
sur­preendente com relação às bases da idolatria. De acordo com ele, a
idolatria em todos os tempos teve sua origem na personificação dos
astros, feita pelos Sabianos, que acreditavam que os astros eram a
divindade e que o sol era o deus supremo. O próprio Abraão foi educado
dentro dessa religião, e sua opo­sição a ela lhe valeu a prisão, o
confisco de seus bens, e o exílio da Síria. Os Sabianos adoravam os 7
planetas e os 12 signos do Zodíaco, e diziam ainda que Adão era o
apóstolo da Lua, e que Noé foi encarcerado porque não aprovava o culto
dos ídolos e porque se dedicava ao culto de Deus. Assim, eles "ergue­ram
estátuas aos planetas, estátuas de ouro ao sol e estátuas de prata à
lua, e distribuíram os metais e os climas pelos planetas, dizendo que
tal planeta era o deus de tal clima. Eles construíram templos nos quais
colocaram estátuas e afirmaram que as forças dos planetas se derramavam
sobre essas estátuas, de tal forma que elas falavam, compreendiam,
pensavam, inspiravam os homens e lhes davam a conhecer o que lhes era
útil". Eles acreditavam ainda que se uma árvore fosse plantada em nome
de um planeta e consagrada a ele, de acor­do com determinados ritos e
cuidados, a força espiritual desse planeta passava para essa árvore ---
como por exemplo no caso da Ashera e do Baal ---, inspira­va os homens e
lhes falava durante seu sono, dando assim origem aos augúrios, à
feitiçaria, às previsões, à mágica etc.

Por não ser uma ciência completa, a idolatria leva à dúvida e à
su­perstição, transformando aqueles que nela acreditam em vítimas da
ignorância, e sujeitando-os a situações de miséria e de destruição,
tornando-se necessário que o ser humano saiba diferenciar entre os
verdadeiros profetas de Deus e os outros. Foi, portanto, para afastar
esses cultos dos hábitos dos homens que Deus se preocupou em estabelecer
os preceitos relativos à interdição da idolatria, pois "para
aproximar-se do verdadeiro Deus e para se obter a sua benevolência não
se precisa de todas essas práticas penosas, mas... basta amá-Lo e
temê-Lo, duas coisas .que são o verdadeiro objetivo do culto divino".
Assim sendo, não faz nenhum sentido que haja pessoas que imaginem que
seus destinos possam ser regidos por esses astros, a quem se dedicavam
templos e oferendas. Além do mais, se os astros se encarregassem de
dirigir a vida das pessoas, estabelecendo-lhes um destino que variaria
de acordo com a posição deles no momento do
nascimento de cada uma delas, e se elas nada pudessem fazer para
intervir ou mudar isso, a Lei que Deus nos deu não teria nenhuma razão
de ser: tudo já estaria traçado e determinado, de forma definitiva e
inexorável, independente­mente da atitude e do comportamento de cada um,
quer vivesse ele no cami­nho dos justos e dos bons ou na delinqüência e
na depravação. ``Nesse caso'', escreve ele, "toda recompensa e todo
castigo seriam injustiças manifestas que não poderiam ser permitidas nem
entre nós, nem por parte de Deus com rela­ção a nós".

E no entanto, fica evidente que a lei divina tem um objetivo bem claro:
"Cada qual dos 613 preceitos serve para inculcar as atitudes corretas ou
para eliminar algumas concepções errôneas, para estabelecer uma
legislação justa ou para eliminar injustiças, para nos imbuir de
virtudes exemplares ou para nos dissuadir de inclinações nocivas". O
conjunto dos preceitos está, portanto, li­gado a três coisas: às
opiniões, à moral e à prática dos deveres sociais, e visa fazer com que
o homem possa alcançar a perfeição do corpo e a da alma. A perfeição da
alma, objetivo máximo e superior da existência, através da qual o homem
atinge a ``permanência perpétua'', a comunhão com Deus, só pode ser, no
entanto, alcançada numa segunda etapa, depois que o bem-estar do cor­po,
embora segundo em importância, tenha sido atingido, já que "é impossível
que o homem, atormentado por uma dor, pela fome, a sede, o calor ou o
frio, compreenda as idéias que se deseja fazê-lo compreender."

A concepção da virtude maimonidiana, situada no termo médio,
equi­distante do excesso e da escassez, e que ele expunha no tratado de
ética que escrevera na juventude, vinha de Aristóteles. Mas ele agora
faz uma exceção a essa média ideal com relação à humildade, sobre a qual
ele diz: "É preciso atingir seu ponto culminante e exercê-la no seu mais
alto grau. Pois na Escritu­ra, toda vez que se fala da grandeza de Deus,
fala-se também de Sua humildade. E Deus louvou a humildade de Moisés,
que possuía todas as qualidades morais e intelectuais e que era o mestre
da doutrina, da ciência e da profecia".

A filologia é uma preocupação constante para Maimônides, em toda a sua
obra. Ele estuda os textos a fundo e compara vários manuscritos do
Talmud antes de tomar uma decisão, pois muitas vezes eles diferem em
pontos essen­ciais. Ele chega até mesmo a conseguir uma cópia do Talmud
do século 7, es­crita em pergaminho, e parece que sua leitura é de
extrema importância para a interpretação dessa obra.

Ele percebe que várias questões religiosas só podiam receber uma solução
com a ajuda da ciência geral, e é o caráter religioso de sua obra que
lhe dá unidade. O equilíbrio de sua alma transparece em seu estilo e ao
longo de suas milhares de frases, cada uma das partes de seus livros se
integra no to­do, cada linha guarda sua medida, cada palavra está em
harmonia com o resto do livro, tanto no valor como na forma.

Ao mesmo tempo em que escreve seu \emph{Guia} ele trabalha num trata­do
de medicina que viria a ser muito difundido, e no qual ele expõe as
teorias fisiológicas, anatômicas, terapêuticas e higiênicas da medicina.
Em 1186, a pe­dido do sultão Al Malik al Muzaffar, Maimônides redige o
\emph{Livro dos Segredos,} onde enumera vários remédios conhecidos
graças aos mais profundos segre­dos da medicina e que fazem parte da
literatura médica e do Talmud, além de algumas fórmulas resultantes de
suas próprias pesquisas. escreve ainda, a pedi­do de Al Fadil,
grão-vizir de Saladin, um tratado conciso de primeiros socorros para os
casos de envenenamento, que adquire uma grande autoridade nos meios
competentes e que é freqüentemente citado pelos médicos durante toda a
Ida­de Média.
No ano de 1187 Al Fadil, nomeia Maimônides médico da corte. Co­mo
conseqüência desse emprego e da consideração que o cerca ele recebe
pouco depois o título de Naguid. A alta posição política que ele adquire
na qualidade de chefe das comunidades judias, e o respeito que ele havia
adquirido graças a sua personalidade, permitem-lhe defender os judeus
nas diversas regiões do reino. Um de seus primeiros decretos, ao assumir
esse posto, é o de proclamar que nas cidades egípcias apenas os juízes
expressamente qualificados podem promulgar casamentos e divórcios. Esse
decreto visa uma supervisão geral dos registros do estado civil e uma
proteção à mulher, para evitar que os emigran­tes, já casados em seus
países de origem, tornem a casar-se no Egito. Seu novo cargo vai
ajudá-lo a combater a influência dos caraítas de forma decisiva e a
obter sucesso na reforma dos ritos, que ele havia iniciado anos antes.
Ele consi­dera, no entanto, que o dever de guiar os judeus é muito mais
uma carga exte­nuante do que um benefício, pois ele tem consciência de
quanto é fácil ser ca­luniado e mal interpretado.

Sua situação como médico da corte o obriga a passar o dia inteiro no
Cairo, mas essa grande fama não lhe traz nenhum prazer: ele se sente
depri­mido, e o cansaço físico provocado por seu trabalho na corte
durante o dia e como médico e como Naguid para os que o procuram em sua
casa à noite e nos fins-de-semana, o impedem de encontrar tempo para
continuar dedican­do-se aos estudos. Esse cansaço acaba por provocar uma
doença que o deixa de cama durante todo um ano e da qual ele nunca se
recupera totalmente, pois seu organismo fica debilitado e minado.

Após o \emph{Guia dos Perplexos} ele não escreve mais nenhuma obra de
caráter religioso, a não ser alguns conselhos e cartas e uma dissertação
para ex­plicar os três princípios fundamentais da metafísica da época: a
existência de Deus, as relações da origem do mundo com relação a Deus, e
a eternidade ou a criação do mundo. Para curar os doentes ele renuncia
ao seu projeto de es­crever um livro sobre a Hagada, que deveria colocar
em evidência a filosofia do judaísmo, e assim legitimar suas próprias
teorias. Renunciando até mesmo a traduzir para o hebraico os livros que
havia escrito em árabe, a acabar seus comentários sobre o Talmud,
iniciados em sua juventude, e a redigir o livro de referências de que
dependia o futuro de seu \emph{Mishneh Torah,} obra que tan­tas
controvérsias havia provocado após sua publicação, ele consagra seus
últi­mos anos de vida inteiramente à medicina, e é nomeado, em 1198,
médico par­ticular do sultão Al Afdal, sucessor de Al Aziz no trono de
Damasco e da Síria, que reinou depois da morte de Saladin, em 1195.
Portanto, no ponto culmi­nante de sua vida, Maimônides se dirige ao povo
é essa e sua última transforma­ção: ele passa da contemplação à•prática,
da metafísica à medicina. Renuncian­do ao isolamento, Maimônides
consegue agora falar com as outras pessoas sem por isso deixar de pensar
em Deus e de ter seu coração aberto a Ele. Ele não mais sente a
necessidade de pensar nas coisas sagradas para se sentir próximo de
Deus. A superioridade da profecia sobre a filosofia torna-se-lhe clara:
"a pro­fecia não é demonstrada por nenhuma prova, toda tentativa de
exame científi­co deve ser posta de lado. Seria como querer colocar
todas as coisas do mundo num pequeno vaso", escreve ele em seus últimos
anos de vida.

Mas o ``corpus'' de sua doutrina logo começa a ser sacrificado por aqueles
que, não se contentando em ler seus livros, decidem criticá-los, sem no
entanto ter compreendido corretamente seu significado,. A questão
relativa à redenção dos corpos, por exemplo, foi mal compreendida e os
yemenitas en­viam ao Gaon de Bagdad uma carta onde se queixam de que a
ressurreição dos mortos, tal como era concebida pelo povo, havia sido
negada na obra de
Maimônides. Seus adversários conseguem o apoio do poderoso Gaon Samuel
ben Ali. A notícia de que o Gaon, representante legítimo da tradição
judaica, havia tomado partido contra Maimônides e de que o havia
condenado e prova­do claramente que ele negava a redenção, espalha-se
pelo mundo judeu. Para desfazer esse mal-entendido Maimônides se vê
obrigado a redigir um \emph{``Tratado sobre a Ressurreição''}, onde ele
diz: "... eu julguei incorreto estudar apenas os galhos da doutrina e
negligenciar as raízes. E por isso que discuti os princí­pios
fundamentais da fé... Mas como a compreensão das provas dessas
doutri­nas fundamentais pressupõe o conhecimento de muitas ciências, eu
apenas apre­sentei os meios de prová-las, sem citá-los... Eu expressei a
opinião que deve­mos imaginar o mundo futuro sem relação alguma com a
redenção, mas ressal­tei expressamente que a reencarnação dos mortos é
um pilar fundamental da doutrina religiosa. ... Mas dizer que afirmei
que a alma não retorna jamais a seu corpo é uma difamação, porque essa
negação constituiria a negação dos pró­prios milagres e equivaleria ao
repúdio da religião." Uma vez mais, Maimôni­des se vê forçado a fazer
aqui uma conceção dogmática, mas o faz provavel­mente apoiado em estudos
metafísicos que lhe permitem conciliar seus conhe­cimentos místicos com
o que foi estabelecido pela tradição.

Contudo, se Maimônides encontra oposição e controvérsia por par­te de
alguns, ele é admirado e respeitado por muitos outros, como pelos
estu­diosos que vivem nos países cristãos, onde não há nem Gaon nem
Exilarca, descendente do rei Davi no exílio, onde a autoridade das
escolas é fraca em relação a da academia de Bagdad, e onde se fazem
verdadeiras pesquisas. Esse é o caso da região de Provença, na França,
onde florescem a ciência talmúdica e a especulação filosófica, e onde as
obras de Maimônides produzem uma sen­sação sem precedentes na história
dos judeus junto a um grupo de sábios res­peitados, chamados de "Sábios
de Lunel". Em 1195 chega a Fostat um tratado redigido em hebraico por
esses sábios, cujo porta-voz se chama Jonathan Co­hen, no qual se
cantavam as glórias de Maimônides e se fazia uma declaração de admiração
e fidelidade àquele que eles consideram como o maior mestre que havia
aparecido desde a conclusão do Talmud.

Por volta de 1201 chegam da região do Midi da França cartas com várias
assinaturas, pedindo a Maimônides para traduzir ele próprio para o
he­braico o seu \emph{``Guia dos Perplexos'',} mas ele se vê forçado a
recusar, pois não tem mais tempo nem sequer para concluir os livros que
havia iniciado. A tradu­ção é então entregue a Samuel Ibn Tibbon, que
trabalha nela com afinco, pois deseja submetê-la a Maimônides para
verificação e revisão, porém infelizmente ele não consegue terminá-la em
tempo.

Maimônides tem a firme convicção de que a imortalidade é a vida eterna
do espírito que tem conhecimentos, mas "as almas que sobrevivem após a
morte não são a mesma coisa que a alma que nasce com o homem no mo­mento
de seu nascimento; pois aquela que nasce ao mesmo tempo que ele é apenas
algo em potencial e uma disposição, enquanto que a coisa que fica em
separado depois da morte é aquilo em que ela se transformou". Portanto,
a imor­talidade depende da quantidade de conhecimentos adquiridos. A
morte é acon­tecimento mais importante para o sábio que adquiriu o
conhecimento de Deus, pois a compreensão da inteligência se fortifica no
momento em que ela se sepa­ra do corpo, já que "Há um limite para a
compreensão humana: enquanto a alma estiver no corpo ela não poderá
compreender o sobrenatural... A matéria é um grande véu". Uma vez
atingida a separação, essa inteligência fica para sem­pre nesse estado
de plenitude, gozando continuamente dessa que é a verdadei­ra e a grande
felicidade.
Essa importante separação chega para Maimônides na noite
de. 13 de dezembro de 1204. Ele é kvado, de acordo com
seu desejo, para Tiberíades na Terra Santa, e ali é enterrado num bosque
próximo ao local onde descan­sam os grandes talmudistas dos séculos II,
III e IV, e onde seu antepassado, o rabino Yehuda Hanassi, havia estado
por diversas vezes.

Nas paredes de mármore que circundam sua tumba, vários seguido­res,
estudiosos e amigos deixaram gravados seus testemunhos de admiração e de
respeito pelo grande sábio, um dos quais diz o seguinte:

"Aqui jaz um homem --- e, no entanto, ele não foi um homem; Se fostes um
homem, então foram os seres celestes que te criaram".



%\parte{\title}

\chapter{Os 14 fundamentos}

Começarei agora a mencionar os Fundamentos --- em número de qua­torze
--- que nos guiarão na enumeração dos preceitos. Começarei dizendo que a
soma total dos preceitos que nos foram ordenados por Deus, conforme
cons­ta no Rolo da Torah, é de seiscentos e treze. Deles, duzentos e
quarenta e oito correspondem ao número de membros do corpo humano e são
preceitos posi­tivos, e trezentos e sessenta e cinco correspondem aos
dias de um ano solar e são preceitos negativos. Este número está
mencionado no texto do Talmud, no final do Tratado Macot, onde está
dito: "Seiscentos e treze preceitos foram ditos a Moisés no Sinai:
trezentos e sessenta e cinco correspondendo aos dias de um ano solar e
duzentos e quarenta e oito correspondendo aos membros do corpo humano".
A título de ``derash''. eles disseram também, com
relação ao fato de que os preceitos positivos correspondem ao número de
membros, que é como se cada membro dissesse à pessoa: "Cumpra um
preceito comi­go"; e com relação ao fato de que os preceitos negativos
correspondem ao nú­mero de dias no ano solar, eles disseram que é como
se cada dia dissesse à pes­soa ``Não cometa uma transgressão hoje''. O
fato de que eles constituem o nú­mero dos preceitos não passou
desapercebido a nenhum dos que se empenha­ram na enumeração dos
preceitos; mas no processo da enumeração em si eles contaram assuntos
que são produtos de imaginação sem base, como será expli­cado neste
trabalho. Isto deveu-se ao fato de que eles desconheciam estes qua­torze
Fundamentos, os quais passarei a explicar.

O Primeiro \textbf{Fundamento:} Não se deve incluir nesta enumeração os
preceitos adicionais de autoria rabínica.

\textbf{O Segundo Fundamento:} Não devemos incluir nesta enumeração o
que se po­de deduzir das Escrituras por meio de qualquer um dos treze
princípios exegé­ticos, pelos quais se interpreta a Torah, ou por meio
de inclusão.
\textbf{O Terceiro Fundamento:} Não devemos incluir nesta enumeração
preceitos que não sejam obrigatórios por todos os tempos.

\textbf{O Quarto Fundamento:} Não se deve incluir nesta enumeração
ordens que abran­jam todos os preceitos da Torah.

\textbf{Quinto Fundamento:} A razão dada para um preceito não deve ser
 contada como um preceito separado.

\textbf{O Sexto Fundamento:} Quando um preceito contém tanto uma ordem
positiva quanto uma negativa, cada uma das partes deve ser contada
separadamente, uma entre os preceitos positivos e a outra entre os
negativos.

\textbf{Sétimo Fundamento:} Não se deve contar as leis detalhadas de
um preceito.

\textbf{Oitavo Fundamento:} Não se deve incluir entre os preceitos
negativos uma declaração negativa que exclua um caso particular de um
determinado assunto.
 
\textbf{Nono Fundamento:} Esta enumeração não deve ser baseada no
número de vezes que um determinado preceito negativo ou positivo está
repetido nas Es­crituras, mas sim na Natureza da ação proibida ou
ordenada.

\textbf{O Décimo Fundamento:} Não se deve contar os atos estipulados
como prelimi­nares ao cumprimento do preceito.

\textbf{O Décimo Primeiro Fundamento:} Não se deve contar separadamente
os diver­sos elementos que compõem um só preceito.

\textbf{O Décimo Segundo Fundamento:} Não se deve contar separadamente
as etapas sucessivas na execução de um preceito.

\textbf{O Décimo Terceiro Fundamento:} Quando um determinado preceito
tiver que ser cumprido por vários dias não se deve contar um preceito
por cada dia.

\textbf{Décimo Quarto Fundamento: De que forma os tipos de castigo
 devem ser contados como preceitos positivos.}

E agora voltarei a explicar cada um dos fundamentos e trazer provas
sobre eles, se Deus quiser.


% J: Temos que perguntar para o Paulo como está se fazendo subtítulo
\chapter*{O primeiro fundamento\subtitulo{Não se deve incluir nesta enumeração os preceitos adicionais de autoria rabínica}}

Vocês devem saber que não deveria ter sido necessário comentar este
assunto, pois ele está perfeitamente claro. Se o texto do Talmud diz que
"Seiscen­tos e treze preceitos foram ditos a Moisés no Sinai", como
seria possível dizer que algo que vem dos Rabinos se inclui nesta
enumeração? Mas fomos forçados a comentá-lo, pois outros já se enganaram
a este respeito e contaram --- como pre­ceito --- a luz de ``Hanucá'' e a
leitura do Rolo (de Esther) entre os preceitos positi­vos. Também o
recitar de cem bençãos diariamente, o consolo aos que estão de luto, a
visita aos doentes, o enterro dos mortos, o vestir os que estão
despidos, o cálculo das estações, e os dezoito dias nos quais
completamos o ``Halel''.

De fato, deve-se olhar com espanto para quem ouve as palavras "Fo­ram
ditos a Moisés no Sinai" e ainda assim conta a leitura do ``Halel'', com a
qual Davi exaltou ao Eterno, enaltecido seja Ele, como se ela tivesse
sido orde­nada a Moisés e ainda conta a luz de ``Hanucá'', que os Sábios
estabeleceram na época do Segundo Templo, e a leitura do Rolo (de
Esther)! Contudo, não creio que haja alguém que pudesse imaginar ou
ousasse supor que foi dito a Moisés no Sinai que ele deveria nos ordenar
acender a luz de ``Hanucá'', se no final de nossa soberania um determinado
acontecimento relacionado com os gregos ocorresse entre nós.

Parece-me, no entanto, que o que os levou a errar e a enganar-se a esse
respeito é que, ao recitar a bênção, dizemos: "que nos santificasteis
atra­vés de Vossos preceitos e nos ordenasteis com relação à leitura do
Rolo", ou "a acender a luz `Hanucá--- , ou "a completar
o Além disso, o Talmud pergunta: ``Onde nos foi ordenado''? E eles
responderam: "Em Suas palavras `Não te desviarás' (Deuteronômio 17:11)".

Mas se foi essa a razão pela qual contaram dessa maneira, eles tam­bém
deveriam ter contado tudo o que foi imposto pelos Mestres, pois já havia
sido ordenado a nosso mestre Moisés, no Sinai, que nos obrigasse a
executar tudo o que os Sábios nos ordenaram ou proibiram, por Suas
palavras: "Confor­me o mandamento da lei que te ensinarem, e conforme o
juízo que te disse­rem, farás" (Deuteronômio 17:11); também mais adiante
Ele nos proibiu de desobedecê-los em tudo o que decretassem e
estabelecessem, ao dizer: "Não te desviarás da sentença que te
anunciarem, nem para a direita nem para a es­querda" (Ibid.). Mas se
fosse correto contar entre os 613 preceitos tudo o que está revestido de
autoridade rabínica --- levando-se em consideração que se in­clui em
palavras, enaltecido seja Ele, ``Não te desviarás da sentença'' e
"Con­forme o m. damento da lei que te ensinarem, farás" --- por que
foram estes enfa zados? Assim como contaram a luz de ``Hanucá'' e a
leitura do Rolo (de Est r) les deveriam ter contado a lavagem das mãos e
o preceito de ``er i b''., po' em relação a isso
recitamos as bênçãos "que nos santificasteis
através de Vossos preceitos e nos ordenasteis com relação à lavagem das
mãos" ou "com relação ao •receito de 'erub--- , da
mesma forma que recitamos a bên­ção ``com relaç à le ura do Rolo'' ou "com
relação a acender a luz de 'Hanu­cá"'. E tudo iss i vem
\textbf{.e} lei rabínica! Eles dizem explicitamente:
"As primeiras águas são prec= tosa. •mo? Abayé disse: É preceito
obedecer as palavras dos Sábios". Isto é e ante ao que dizem com relação
à leitura do Rolo (de Es­ther) e à luz de anucá': "Onde nos foi ordenado
isso? Nas palavras 'Não te desviarás"'. Também foi deixado claro que
tudo aquilo que ordenaram os pro­fetas --- a paz esteja com eles --- que
vieram depois de Moisés, nosso mestre, também é de autoridade rabínica.
Assim, eles dizem expressamente: "Quando Salomão estabeleceu o 'erubin'
e as mãos., uma voz divina apareceu e disse: `Meu
filho, seja sábio e alegre Meu coração--- . Em outros
trechos eles explica­ram que o 'erubin' é lei rabínica e que as
mãos. é decreto dos Escribas. Assim, foi explicado que
tudo o que foi ordenado depois de Moisés, nosso mestre, é chamado de
``lei rabínica''.

Eu lhes expliquei isso a fim de que não pensem que pelo fato de ter sido
ordenada pelos profetas, a leitura do Rolo (de Esther) deve ser
considera­da ``lei das Escrituras'', pois o ``erubin'' é chamado de "lei
rabínica", embora tenha sido ordenado por Salomão, o filho de Davi, e
seu Tribunal.

Foi isso o que confundiu alguém, que por esse motivo contou vestir os
despidos, pois ele encontrou em Isaías o seguinte: "Quando vires um
despi­do, o cobrirás". Mas ele não sabia que isto está incluído em Suas
palavras ``E lhe emprestarás o suficiente para o que lhe faltar''
(Deuteronômio 15:8). Está sem dúvida alguma claro que o significado
deste preceito é que devemos ali­mentar os famintos, vestir os despidos,
dar um colchão e um cobertor a quem não os tiver, ajudar a casar-se
aquele que não tiver meios para fazê-lo e provi­denciar urna montaria
para quem não a tiver, pois sabe-se, pelo texto do Tal­mud, que tudo
isso está incluído em Suas palavras globais "O suficiente para o que lhe
faltar".

Parece, contudo, que na opinião dessas pessoas a linguagem do Tal­mud
foi composta ``com lábios vacilantes e com língua estranha'', caso
contrá­rio eles não teriam contado a leitura do Rolo (de Esther) e
similares como pre­ceitos ditos a Moisés no Sinai!


A Guemará de Sheb ot diz. "Eu só tenho conhecimento dos preceitos 
que foram ordenados no S que forma fico sabendo dos que foram
destinados a ser estabelecidos ovos artigos, tais como a leitura do Rolo
(de Esther)? Pelas palavras das s uras: 'Os judeus cumpriram e
assumiram' (Esther 9:27) --- eles cumpriram aquilo que já haviam
assumido". Isso significa que eles aceitaram crer em todos os preceitos
que os profetas e os Sábios orde­nassem depois.

Eu estou surpreso. Por que eles contaram preceitos positivos de
au­toridade rabínica, como os que mencionamos, mas não contaram também
pre­ceitos negativos de autoridade rabínica? Assim como eles contaram
entre os pre­ceitos positivos a lei de "Hanuc '" itura do Rolo (de
Esther), as cem bên­çãos e o ``Halel'', eles deveriam gualm. te ter
contado entre os preceitos ne­gativos os vinte casos de "She o vinte
preceitos negativos. 
% A lavagem das mãos antes da refeição .osição às ``águas finais'', que
% são a lavagem das\\
% mãos após a refeição).
  
%  Dos que o povo de ``Israel'' recebeu no monte Sinai.
 
% \item
 
%  Incesto de segundo grau, que foram proibidos por autoridade rabínica.
 
Pois assim como toda e qualquer relação proibida pela Torah constitui um
preceito negativo de lei das Escrituras, assim também cada relação
proibida pelos Mes­tres constitui um preceito negativo de autoridade
rabínica. Eis exatamente o que os Sábios dizem: "As
Sheniyoe. foram estabelecidas pelos Escribas". E es­tá
também explicado no Talmud que as palavras do Mishná, "proibida por um
preceito" se referem às ``Sheniyot''. A esse respeito eles comentaram:
"Que proi­bição por um preceito? O preceito que ordena obedecer as
palavras dos Sá­bios". Da mesma forma, eles deveriam ter contado nesta
enumeração "a irmã de uma mulher (na qual foi feita a)
`Halitzá--- , que é proibido por decreto dos Escribas!

Resumindo, se devêssemos contar todos os preceitos positivos e
ne­gativos impostos pelos Rabinos, o número chegaria a muitos milhares.

Isto está, sem dúvida alguma, claro. Tudo o que for de autoridade
rabínica não deve ser contado na soma total dos 613 preceitos, uma vez
que esses estão todos baseados em versículos da Torah, não havendo entre
eles na­da que seja de autoridade rabínica, como explicamos. Mas o fato
de contar al­guns preceitos de autoridade rabínica e deixar
arbitrariamente outros de fora não pode ser aceito em hipótese alguma,
seja quem for seu autor.

Explicamos, assim, o teor deste fundamento e sua prova, para que ninguém
venha a ter nem sombra de dúvida a este respeito.

\chapter*{O segundo fundamento
\subtitulo{Não devemos incluir nesta enumeração o que se pode deduzir das
escrituras por meio de qualquer um dos treze princípios exegéticos,
pelos quais se interpreta a torah,
ou por meio de inclusão}}

Nós já explicamos no início de nosso \emph{Comentário sobre a Mishná}
que a maioria das leis da Torah são deduzidas por meio dos treze
princípios exegéticos pelos quais se interpreta a Torah, e que uma lei
deduzida dessa for­ma está algumas vezes sujeita a uma diferença de
opiniões. Também explica­mos ali que há leis que são interpretações
tradicionais recebidas de nosso mes­tre Moisés, e portanto não sujeitas
a diferenças de opinião, mas que foram pro­vadas por meio de algum
desses treze princípios. Na realidade, a sabedoria das Escrituras é tal
que é possível encontrar nelas um indício ou uma semelhança que nos
conduza à interpretação recebida.

Assim sendo, conclui-se que nem toda lei deduzida pelos Sábios por meio
de um dos treze princípios pode ser declarada como tendo sido dita a
Moisés no Sinai; da mesma forma, não devemos concluir que pelo fato de
os Sábios do Talmud encontrarem respaldo para uma determinada lei num
dos treze princípios, ela será de autoridade rabínica, visto que é
possível que uma
determinada lei seja uma interpretação recebida.

Resumindo: toda lei que não estiver explicitamente enunciada na To­rah,
mas que tenha sido deduzida do Talmud por meio de um dos treze
princí­pios, deve ser contada se aqueles que receberam a Tradição
declararem expli­citamente que ``ela pertence ao corpo da Torah'' ou que
``ela é lei da Torah''. Mas se eles não disserem ou explicarem claramente
que é assim, então ela é uma lei de autoridade rabínica, uma vez que não
há nenhum versículo que a indique diretamente.

Também com relação a este fundamento um outro se enganou e con­tou o
temor aos Sábios entre os preceitos positivos.

Quer-me parecer que ele o fez por causa das palavras de Rabi Akiba: "
'Ao Eterno, teu Deus, temerás' (Deuteronômio 6:13) inclui os sábios". E
eles pensaram que tudo o que se deduz pela Inclusão é semelhante aquilo
que lhe deu origem.

Mas se fosse correto o que eles pensaram, então por que eles não
contaram também o dever de honrar o padrasto e a madrasta e o irmão mais
velho, além dos pais --- a quem temos o dever de honrar conforme o que
foi determinado através do princípio de Inclusão? (Como os sábios
disseram, " 'A teu pai' (Êxodo 20:21) inclui teu irmão mais velho assim
como teu padrasto. 'E a tua mãe' inclui tua madrasta"). Isso é análogo
ao que os Sábios disseram: " 'Ao eterno, teu Deus, temerás' inclui os
sábios". Então por que eles conta­ram este último e não o anterior?

O fato de não terem o conhecimento necessário já os levou a come­ter um
erro maior do que esse: ao encontrar uma interpretação de algum
versí­culo que exigisse a execução ou a proibição de um determinado ato
--- obriga­ções que são sem dúvida impostas por lei rabínica --- eles as
contaram entre os preceitos, ainda que o significado literal do
versículo não indicasse de ma­neira alguma aquelas obrigações. Isso
contraria o princípio que eles, abençoa­da seja sua memória, nos
ensinaram: ``Um versículo da Torah não perde nunca seu sentido literal''.
Também contraria o processo de raciocínio encontrado em todo o Talmud, e
que está demonstrado pelo fato de que quando os Sábios falam dè um
versículo, que dá origem a tópicos derivados por meio de inter­pretação
e de várias provas, eles perguntam: "Mas de que trata o versículo em
si?"

Mas eles baseados em comparações sem fundamento, contam entre os
preceitos positivos a visita aos doentes, o consolo aos que estão de
luto, e o enterro dos mortos, tudo isso por causa da seguinte
interpretação, mencio­nada com relação às Suas palavras, enaltecido seja
Ele, "E fa-lo-ás saber o cami­nho por onde andarão, e a obra que farão"
(Êxodo 18:20): " 'O caminho' se refere aos atos de bondade; 'andarão' se
refere a visitar os doentes; 'por onde' se refere a enterrar os mortos;
'e a obra' se refere às leis; 'que farão' se refere ao que ultrapassa o
estritamente requerido pela lei". Eles pensaram que cada uma das
obrigações mencionadas constitui um preceito em si, mas eles não sa­biam
que todas essas obrigações, bem como outras semelhantes, estão
incluí­das nos termos de um dos preceitos explicitamente enunciado na
Torah, que é o que está expresso em Suas palavras, enaltecido seja Ele,
"Amarás o teu pró­ximo como a ti mesmo" (Levítico 19:18). De maneira
semelhante eles conta­ram o cálculo das estações do ano como um
preceito, baseados na seguinte in­terpretação dada pelos Sábios ao
versículo "Porque isto é a vossa sabedoria e o vosso entendimento à
vista dos povos" (Deuteronômio 4:6): "Qual é a sabe­doria e qual o
entendimento que está à vista dos povos? Devo dizer que é o cálculo das
estações e das constelações".


Se eles tivessem contado questões ainda mais claras do que esta, e
portanto mais fáceis de supor que é correto enumerá-las --- ou seja, as
leis de­duzidas através de um dos treze princípios de interpretação da
Torah --- o nú­mero de preceitos atingiria vários milhares! Talvez vocês
possam pensar que nós desistimos de contá-las porque elas não são
suficientemente claras, ou por­que haja dúvidas quanto ao fato de
determinada lei deduzida através daquele princípio estar correta ou não,
mas a razão não é essa. O motivo pelo qual não as contamos é que tudo o
que se deduzir dessa forma são ramificações das raí­zes que foram
explicitamente declaradas a Moisés no Sinai e que constituem os 613
preceitos.

Mesmo que tenha sido o próprio Moisés quem as deduziu, elas não devem
ser contadas. A prova de tudo isto é que eles disseram na Guemará de
Temurá: "Mil e setecentas deduções de menor a maior., analogias de
frases e aspectos.especiais nos decretos dos Escribas foram esquecidos
durante os dias de luto por Moisés. Contudo Ataniel, filho de Kenaz, os
reconstituiu com seu raciocínio, como foi dito: 'E Caleb disse: Aquele
que ferir a ``Kiriat Sefer'' e a tomar... E Ataniel, o filho de Kenaz a
tomou' ". E esse número já era tão gran­de, quantas então não seriam as
leis originais aprendidas através de Moisés! Pois é inconcebível que se
tenha esquecido tudo o que foi aprendido. Sendo assim não há dúvida de
que as leis aprendidas por dedução de menor a maior, ou por algum dos
outros princípios, chegavam a vários milhares e que todas elas já eram
conhecidas nos dias de Moisés, nosso mestre, posto que foram esqueci­das
nos dias de luto.

Assim, foi-lhes explicado que mesmo no tempo de Moisés já se fala­va de
``aspectos especiais nos decretos dos Escribas'', pois tudo o que eles não
ouviram explicitamente no Sinai é considerado como "decreto dos
Escribas". Da mesma forma, foi explicado que mesmo na época de Moisés, a
paz esteja com ele, não se contava entre os 613 preceitos ditos a ele no
Sinai nenhuma lei deduzida através dos treze princípios, e que nós
certamente não devemos contar o que tenha sido deduzido num período
posterior. Em vez disso, deve­mos contar o que constitui uma
interpretação formulada em seu nome, desde que os guardiães da Tradição
nos digam claramente que um ato específico nos foi proibido e que essa
proibição é lei da Torah, ou que eles nos digam que "ela é parte da
própria Torah". Nesse caso a contaremos, pois a aprendemos pela Tradição
e não por um dos treze princípios. Em tais casos a referência feita a um
dos treze princípios ou a apresentação de provas por meio deles serve
apenas para demonstrar a sabedoria contida na Torah, como explicamos no
Co­mentário sobre a Mishná.

\chapter*{O terceiro fundamento\subtitulo{Não devemos incluir nesta enumeração preceitos que não sejam obrigatórios por todos os tempos}}

Você deve saber que as palavras "Os 613 preceitos foram declara­dos a
Moisés no Sinai" ensinam que este número constitui a quantidade de pre-
ceitos obrigatórios por todos os tempos, isso porque os que assim não
forem não têm relação específica com o Sinai, quer tenham eles sido ali
proclamados ou não. A expressão ``no Sinai'' significa apenas a Revelação
essencial da Torah, que ocorreu no Sinai. Isto está expresso em suas
palavras, enaltecido seja Ele, "Sobe a Mim, ao monte, e fica ali; e
dar-te-ei..." (Êxodo 24:12). E eles disseram expressamente: "Onde nas
Escrituras está dito que os 613 preceitos foram de­clarados a Moisés no
Sinai? No versículo 'E a Lei que nos ordenou Moisés, heran­ça e...'
(Deuteronômio 33:4). Quer dizer, ele nos ordenou a soma das
letras-nú­meros TORAH, que totaliza 611. Eles ouviram o 'Eu sou o
Eterno, teu Deus' (Êxo­do 20:2) e o 'Não terás outros deuses diante de
Mim' (Ibid., 3) do próprio Todo Poderoso". Com mais esses dois preceitos
se completa o número 613.

O propósito desta rubrica é demonstrar que a Palavra que nos foi
ordenada por Moisés e que nós ouvimos apenas dele é a soma das
letras-núme­ros da palavra TORAH, e que é isso o que Ele declarou ser a
"Herança para a congregação de Jacob." (Deuteronômio 33:4). Um preceito
que não seja obri­gatório por todos os tempos não é ``uma herança'' para
nós, pois "uma heran­ça" é apenas aquilo que perdura para sempre, assim
como foi dito: "Por todos os dias que os céus estiverem sobre a terra"
(Ibid., 11:21). Assim também a afir­mação deles, de que é como se cada
um dos membros de uma pessoa lhe orde­nasse cumprir um preceito e cada
dia do ano o aconselhasse a não cometer uma transgressão, é uma prova de
que este número nunca vai diminuir. Mas se os preceitos que não são
obrigatórios para sempre devessem ser incluídos nesta enumeração, esse
número diminuiria toda vez que um determinado preceito, ao alcançar seu
objetivo, tivesse sido completamente cumprido, e assim aquela afirmação
teria sido correta apenas durante um determinado momento.

Uma vez mais errou o outro com relação a este fundamento, e con­tou ---
ao se ver pressionado --- "E não entrarão para ver, quando cobrirem os
objetos da santidade" (Números 4:20) e "Não farão mais o trabalho de
carre­gamento" (Ibid., 8:25), relativo aos levitas. Mas esses preceitos
eram obrigató­rios apenas no deserto, e não para sempre. Embora digam:
"Há uma insinuação contra roubar um vaso sagrado em 'E não entrarás para
ver"', o termo "insi­nuação" já é prova suficiente para indicar que este
não é o significado literal do versículo; tampouco se inclui esta
transgressão entre as passíveis de morte pelas mãos dos Céus, como foi
explicado na Tosseftá e em Sanhedrin.

De fato, surpreendo-me com quem contou essas proibições. Por que eles
não contaram o versículo relacionado com o maná, "Ninguém deixe so­brar
dele até a manhã" (Êxodo 16:19) assim como o versículo "Não molestes a
Moab, e não faças a ele guerra" (Deuteronômio 2:9), e o versículo
relativo a Amon, "Não os molestes, e não combatas com eles" (Ibid., 19)?
Eles também deveriam ter contado entre os preceitos positivos os
versículos "Faze para ti uma serpente abrasadora e põe-na sobre uma
haste" (Números 21:8) e "Toma um vaso, põe nele a quantia de um 'omer'
de maná. (Êxodo 16:33) da mesma forma como contaram "a
oferenda de elevação do tributo" e a dedicação do altar. E também
deveriam ter contado ``Estejam prontos para o terceiro dia'' (Êxodo
19:15), bem como "Tampouco o rebanho, o gado, aparecerão" (Ibid., 34:3),
``Que não transpassem o termo para subir ao Eterno'' (Ibid., 19:24), e
muitos outros versículos semelhantes.

Nenhum ser racional duvidará que todos esses preceitos --- positi­vos e
negativos --- foram de fato ditos a Moisés no Sinai, só que eles foram
apli­cáveis durante um determinado período de tempo e não são
obrigatórios para sempre, e por isso não devem ser incluídos.

De acordo com este fundamento não devemos contar nem as Bênçãos 
e as Maldições que lhes foram ordenadas no Gerizim e no Ebal, nem a
edi­ficação do altar que nos foi ordenado construir quando entrássemos
na terra de Canaã, porque esses eram todos preceitos aplicáveis a um
determinado pe­ríodo de tempo.

Tampouco devemos contar o preceito positivo de que se desejásse­mos
comer carne de algum animal só poderíamos fazê-lo depois de levá-la
co­mo oferenda de pazes, porque isso foi um decreto especificamente
aplicável no deserto, como aparece em Suas palavras "Os trarão ao
Eterno" (Levítico 17:5), sobre as quais a Sifrá comenta: " 'E os trarão'
constitui um preceito positivo" --- mas obrigatório apenas no deserto,
pois Ele explicou no Deuteronômio a permissão perene de comer uma
refeição de carne ao dizer: "Com todo o dese­jo de tua alma poderás
comer carne" (Deuteronômio 12:20).

Se fosse necessário contar tudo o que está nesta categoria, todos os
preceitos or na. os a Moisés desde o dia em que ele se tornou profeta
até o dia de sua orte incluindo tudo o que lhe foi ordenado no Egito,
durante a Consagr ão., e outros preceitos além desses,
todos contidos na Torah, al­guns posit v o s utros negativos ---
teríamos mais de 300 preceitos, além dos que são vigentes por todos os
tempos. Mas como é impossível contá-los todos, fatalmente não se deve
contar nenhum, e não fazer como fizeram outros, usan­do apenas alguns
deles para assim completar o número que não conseguiram atingir.

Isto é o que desejávamos alcançar com este Fundamento.

\chapter*{O quarto fundamento
\subtitulo{Não se deve incluir nesta enumeração ordens que abranjam todos os
preceitos da torah}}

Há preceitos positivos e negativos na Torah que não se referem a uma
obrigação concreta, mas incluem todos os preceitos, como se o Eterno,
enaltecido seja Ele, estivesse dizendo: "Faça tudo o que eu lhe ordenei
e cuide-se para não fazer todas as coisas que eu lhe proibi", ou "Não se
rebele contra algo que eu lhe ordenei". Tal ordem não deve ser contada
como um preceito separado já que ela não se refere a uma obrigação
específica, o que faria dela um preceito positivo, nem adverte contra um
ato determinado, o que a trans­formaria num preceito negativo.

Assim por exemplo é o que Ele disse: "E de tudo o que vos disse,
guardá-lo-eis" (Êxodo 23:13); ``Meus estatutos guardareis'' (Levítico
19:19); ``Os Meus juízos cumprireis'' (Ibid., 18:4); "E guardardes Minha
aliança" (Êxodo 19:5); ``E guardareis o Meu mandado'' (Levítico 18:30), e
muitas outras afirmações aná­logas.

Já se enganaram com relação a este Fundamento, contando ``Santos sereis''
(Ibid., 19:2) como um preceito positivo, sem saber que os versículos
``Santos sereis'' e "Santificar-vos-ei e sereis santos" (Ibid., 11:44) são
ordens para que cumpramos a totalidade da Torah, como se Ele dissesse:
"Seja santo fazendo tudo o que Eu lhe ordeno e afaste-se de tudo o que
lhe proibi de fazer".

A Sifrá diz: " 'Santos sereis' --- fique distante". Ou seja, fiquem
lon­ge das abominações contra as quais Eu os adverti.

Na Mekhiltá disseram: " 'Homens de santidade sereis para Mim' (Êxo­do
22:30). Issi, o filho de Yehudá diz: a cada novo preceito que o Santo,
enalte­cido seja Ele, impõe a Israel, Ele lhes acrescenta mais
santidade". Quer dizer, esta não é uma obrigação independente, mas está
relacionada aos preceitos que lhes foram ordenados ali, pois todo aquele
que cumprir aquela obrigação será chamado de ``santo''.

Dessa forma não há diferença entre Suas palavras ``Santos sereis'' e
``Cumpre Meus preceitos''. Assim como não diríamos que esta intimação
glo­bal constitui um preceito positivo a ser acrescentado a todos os
outros precei­tos, assim também não podemos dizer que Suas palavras
``Santos sereis'' e ou­tras expressões semelhantes constituem preceitos
separados, uma vez que não há nelas nada de específico além do que já
sabemos.

O Sifrei diz: " 'E sejais santos' (Números 15:40) --- isto se refere à
santidade dos preceitos".

Assim, ficou claro aquilo a que nos propusemos.

A partir deste Fundamento segue-se também que Suas palavras "E tirareis
o entupimento de vosso coração" (Deuteronômio 10:16) significam: Sede
humilde e ouví todos os preceitos que Ele mencionou anteriormente. Da
mes­ma forma, o versículo ``E vossa cerviz não endurecereis mais'' (Ibid.)
significa: Não vos rebeleis, aceitando tudo o que vos ordenei e não o
desobedeçais.

\chapter*{O quinto fundamento\subtitulo{A razão dada para um 
preceito não deve ser contada como um preceito separado}}

Ocasionalmente encontramos razões para os preceitos em forma de
preceitos negativos que poderíamos pensar ser apropriados contar como
pre­ceitos independentes. Assim, por exemplo, são Suas palavras "Não
poderá seu primeiro marido, que a despediu, tornar a tomá-la, para que
seja sua mulher... e não farás condenar a terra" (Deuteronômio 24:4),
onde as palavras ``E não farás condenar a terra'' são as razões da
proibição que as antecede, como se Ele tivesse dito: "Se fizeres assim,
aumentará a corrupção na terra".

Outro exemplo são Suas palavras "Não profanarás a tua filha para fazê-la
prostituta, para que a terra não seja entregue à prostituição" (Levítico
19:29), onde as palavras "Para que a terra não seja entregue à
prostituição" cons­tituem o motivo, como se Ele tivesse dito: "A razão
desta proibição é para que a terra não seja entregue à prostituição".

O mesmo ocorre no versículo "E não vos façais impuros com eles e não
sejais impuros por eles" (Ibid., 11:43). Depois de ter mencionado a
proibição contra comer determinadas coisas, Ele deu a razão para isso
dizendo: "Não se tornem impuros comendo-os", dando a entender que a
transgressão desta proibição causa a impurificação da alma.

O Sifrei diz expressamente, com relação a Suas palavras, enaltecido seja
Ele, sobre a proibição de pedir um resgate pela vida de um assassino: "O
versículo 'E não contaminarás a terra' (Números 35:34) nos ensina que o
derra­mamento de sangue impurifica a terra". Portanto, foi explicado que
esta or­dem negativa constitui uma razão para a proibição anterior e não
uma declara­ção independente.


Da mesma forma dizem, com relação ao seguinte versículo "E do
santuário não sairá e não profanará" (Levítico 21:12): "Mas se sair, ele
profanará".

Também com relação a este Fundamento errou um outro, incluindo
essas ordens, sem compreendê-las. Todavia, se se perguntasse a alguém
que as incluiu qual é a obrigação específica que essa ordem estabeleceu ele
ficaria confuso e não teria resposta. E isso anula a alegação de que elas possam
ser contadas.

\chapter*{O sexto fundamento\subtitulo{Quando um preceito 
contém tanto uma ordem positiva quanto uma negativa,
cada uma das partes deve ser contada separadamente, uma entre os
preceitos positivos e a outra entre os negativos}}

Você deve saber que um assunto pode ser regulamentado por meio de ambos
um preceito positivo e um negativo, de uma destas 3 maneiras:

\begin{enumerate}
\item 
 Se o cumprimento de uma determinada obrigação acarretar um preceito
 positivo e a sua transgressão um preceito negativo, como por exem­plo
 no Shabat, nos festivais e no Ano Sabático, quando se alguém fizer
 certos trabalhos estará violando um preceito negativo e se descansar
 estará cumprin­do um preceito positivo, como será explicado
 posteriormente. Da mesma for­ma, o jejum em ``Yom Quipur'' constitui um
 preceito positivo e comer nesse dia é um preceito negativo.
 
\item
 Se houver um preceito negativo precedido por um preceito posi­tivo,
 tal como se vê em Suas palavras com relação a quem seduziu ou
 maldisse: ``E lhe será por mulher'' (Deuteronômio 22:19), que constituem
 um preceito positivo, e ``Não a poderá despedir por todos os seus dias''
 (Ibid., 29), que cons­tituem um preceito negativo.
 
\item
 Se houver um preceito negativo enunciado primeiro, e depois
 jus­taposto a um preceito positivo, como aparece, por exemplo, em Suas
 palavras ``Não tomarás a mãe estando com os filhos'' (Ibid., 6), que são
 seguidas de "Dei­xarás ir livremente a mãe" (Ibid., 7).
\end{enumerate}


Em cada um destes casos devemos contar a ordem positiva entre os
preceitos positivos e a negativa entre os preceitos negativos, pois os
Sábios falam explicitamente em cada caso do preceito positivo e do preceito
negativo. Assim, eles dizem muitas vezes: "O positivo ou o negativo,
relativos a isto". Isto é perfeitamente compreensível, posto que o
significado do positivo é dife­rente do negativo e portanto eles são
duas obrigações diferentes: num Ele nos dá uma ordem e no outro Ele nos
faz uma advertência.

Eu não me recordo no momento de alguém que tenha se enganado com relação
a este Fundamento.

\chapter*{O sétimo fundamento\subtitulo{Não se deve contar as leis detalhadas de um preceito}}

Saiba que cada preceito está expresso n s E crituras e a partir des base
seguem-se muitas obrigações e advertências •m r lação às leis que reg m
esse preceito. Este é um exemplo disso:
% Jorge: Exemplo de como vamos trabalhar com notas
a "Halit • á"\footnote{Ver o preceito positivo 217.}
o casamento levir o\footnote{Ver o preceito positivo 216.}
são dois preceitos positivos. Não
há nenhuma c o tro érsia quanto a isso. Mas quando estudamos as leis
desses dois preceitos p ■ it os e o que deve ser cum­prido de acordo com
os postulados da lei, verifica-se que algumas mulheres realizarão a
``Halitzá'' e não o casamento levirato e que outras realizarão o
casa­mento levirato e não a ``Halitzá'', enquanto outras ainda farão ou um
ou outro, e outras não farão nem um nem outro. O mesmo se aplica aos
homens, ou seja, aos cunhados. Alguns se submetem à ``Halitzá'' mas não
realizam o casamento levirato, outros se casam, mas não se submetem à
``Halitzá''; outros ainda não fazem nenhum dos dois, e outros podem fazer
ou um ou outro: contrair o casa­mento levirato ou submeter-se à
``Halitzá''. Da mesma forma, verifica-se que algumas das cunhadas devem
fazer a ``Halitzá'' e outras devem casar-se com um dos cunhados; algumas
farão a ``Halitzá'' a todos; a algumas mulheres era permitido que ela se
casasse com o homem que se tornou seu marido, mas não com os irmãos
dele, enquanto que no caso de outra mulher ela poderia ter-se casado com
um dos irmãos de seu marido, mas estava proibida de casar-se com o homem
que de fato se tornou seu marido; em outros casos, tanto o marido como
os irmãos dele eram homens proibidos para ela e, em outro caso ainda,
era-lhe possível casar-se tanto com o homem que se tornou seu marido
como com os irmãos dele.

Se fôssemos considerar cada lei dessas como um preceito indepen­dente,
só as leis do Tratado Yebamot somariam mais de duzentos preceitos!
Contudo, nenhuma delas é por si só um preceito positivo ou um negativo;
ao contrário, devemos dizer que em determinadas circunstâncias uma
cunhada deve fazer a ``Halitzá'' ou o casamento levirato, e que em outras
ela está proibida de casar-se com um determinado cunhado, ou então que
tanto a ``Halitzá'' quan­to o casamento levirato são impossíveis no seu
caso. E assim deve ser necessa­riamente com relação a cada um dos
preceitos.


Sendo assim --- e isto é um assunto acima de qualquer discussão
---conclui-se que ainda que as leis de um preceito estejam
explicitamente enun­ciadas na Torah elas não devem ser contadas. O mero
fato de as Escrituras te­rem explicado as leis ou as condições de um
determinado preceito não signifi­ca que devamos contar cada condição e
cada detalhe da lei como um preceito individual.

Mas muitos já se enganaram a este respeito, contando tudo o que
encontraram nas Escrituras, sem refletir sobre a substância do preceito,
suas leis e condições. A título de exemplo podemos citar que as
Escrituras, no livro de ``Vayikrá'' (Levítico), obrigam uma pessoa que
tenha tornado impuros o San­tuário, suas ofertas consagradas e outras
coisas ali mencionadas, a levar um Sa­crifício de Pecado. Isto
constitui, sem dúvida, um preceito positivo. Logo de­pois as Escrituras
explicam as leis relativas a essa oferenda, dizendo que ela de­ve
constar de uma ovelha ou uma cabra, e que se lhe não tiver posses
suficien­tes para isso deverá levar duas rolas ou dois pombinhos, e se
ele também não puder se permitir isso, deverá levar a décima parte de
uma ``efá'' de farinha. Isto constitui um Sacrifício de Maior ou Menor
Valor. É óbvio que todas estas leis são apenas uma explicação sobre qual
é o sacrifício imposto, e que de for­ma alguma elas devem ser contadas
como três preceitos --- o de oferecer o ani­mal, o de oferecer as aves,
e o de oferecer a décima parte de uma ``efá'' pois elas não são três
ordens e sim apenas um único preceito, a saber, que um transgressor deve
oferecer um sacrifício por pecado e que esse sacrifício deve ser isto ou
aquilo, dependendo de seus recursos.

O mesmo princípio se aplica no caso de um sacrifício por um erro
cometido com relação aos preceitos. Assim, as Escrituras explicaram no
livro ``Vayikrá'' que aquele que transgride involuntariamente um dos
preceitos do Eterno deve levar um Sacrifício de Pecado, e que isto
constitui um preceito positivo se o erro for um dos que acarretam a
extinção se cometido voluntaria­mente, se houver algum ato relacionado
com ele e se ele acarretar a transgres­são de um preceito negativo, como
explicamos no Comentário a Horayot e Que­retot. Em seguida, a Escritura
detalhou as leis relativas a esse sacrifício, cando a isso vários
versículos e dizendo que se a pessoa que cometeu o p for alguém do povo,
ela deve levar uma ovelha ou uma cabra; se for o ele deve levar um bode;
e se for o ``Cohen Gadol'', ele deve levar um se o erro cometido for
especificamente com relação à idolatria, o transg --- seja ele o líder,
alguém do povo ou o ``Cohen Gadol'' --- deve oferece a cabra. Mas o fato
de oferecer diferentes tipos de animais não altera a natureza do
sacrifício em si, que é a oferta a ser levada por um pecado não
intencional, e não o transforma em vários sacrifícios, de modo a dar
origem a vários precei­tos. Se assim fosse, deveríamos da mesma forma
contar Suas palavras "uma ove­lha" ou ``uma cabra'' como dois preceitos
separados, e ``duas rolas'' ou ``duas pombinhas'' como outros dois
preceitos. Obviamente isto não estaria correto, pois o que constitui o
preceito positivo é a obrigação de levar um sacrifício; e o fato de que
uma pessoa ofereça uma cabra como sacrifício, e a outra um bode, é
apenas uma condição dessa oferenda, mas nem toda condição de um preceito
deve ser considerada como um preceito independente.

Compreenda bem este ponto, pois um erro acerca disto pode ser disfarçado
e só será percebido por alguém dotado de compreensão.

Também entra nesta categoria o que Ele disse, enaltecido seja Ele,
em complemento à lei do castigo de uma moça compromet cometer
adultério: que se uma moça comprometida cometer adultéri tigo será
o apedrejamento, e que se ela for a filha de um ``Cohen'', seu o será ser
queimada. Com relação a este assunto,
todos os que consultei se enganaram pois eles contaram um preceito
separado para a mulher casada, um para a moça comprometida, e um para a
filha de um ``Cohen''. Mas isso não está correto, como explicarei a
seguir.

Suas palavras, enaltecido seja Ele, ``Não adulterarás'' (Êxodo 20:14), que
constituem um preceito negativo, são explicadas pela Tradição como
sen­do uma advertência às mulheres casadas. Essa advertência é seguida
por um versículo que declara que quem violar esta proibição está sujeito
à pena de morte. Isso está expresso em Suas palavras "Certamente serão
mortos, o adúltero e a adúltera" (Levítico 20:10). A seguir, as
Escrituras completam a lei desta puni­ção estabelecendo condições,
dizendo que o versículo "Certamente serão mor­tos, o adúltero e a
adúltera" está sujeito às seguintes condições: se a mulher casada for
filha de um ``Cohen'', seu castigo será ser queimada; se ela for uma moça
virgem comprometida, seu castigo será ser apedrejada; e se ela não for
mais virgem e não for a filha de um ``Cohen'', seu castigo será o
estrangulamen­to. As leis detalhadas relativas ao tipo de morte a ser
aplicado não transformam esse único preceito em vários pois, apesar de
todos esses detalhes, não nos afas­tamos da proibição básica imposta à
mulher casada.

Os Sábios dizem explicitamente em Sanhedrin: "Estavam todos in­cluídos
nos termos 'adúltero e adúltera' (Ibid.); as escrituras apenas
especifica­ram que a filha de um israelita está sujeita a ser apedrejada
e que a filha de um ``Cohen'' está sujeita a ser queimada". Com essa
afirmação eles pretenderam dizer o seguinte: a proibição imposta à
mulher casada através de Suas palavras "Certamente serão mortos, o
adúltero e a adúltera" inclui a todos; as Escrituras apenas
estabeleceram uma diferença quanto ao tipo de morte, impondo à quei­ma a
algumas e o apedrejamento a outras.

Se devêssemos contar as leis detalhadas de um preceito por estarem elas
mencionadas nas Escrituras, então deveríamos contar muitos preceitos ao
invés de um só na lei que determina que um homicida não intencional deve
exilar-se numa cidade de refúgio, posto que as Escrituras mencionaram
especi­ficamente os detalhes das leis relativas a esse preceito.
Deveríamos, então, con­tar da seguinte forma: "E se com instrumento de
ferro ferir" (Números 35:16) --- um preceito; "E se com uma pedra que
cabe na mão" (Ibid., 17) --- o segun­do preceito; "Ou se com instrumento
de madeira, com o qual se pode matar, ferir alguém" (Ibid., 18) --- o
terceiro preceito; "O vingador do sangue, matará o homicida" (Ibid., 19)
--- o quarto preceito; ``E se com ódio empurrar alguém'' (Ibid., 20) --- o
quinto preceito; "Ou jogar alguma coisa sobre ele, de embosca­da"
(Ibid.) --- o sexto preceito; ``Ou por inimizade o ferir com a mão''
(Ibid.), --- o sétimo preceito; "E se por acaso, sem inimizade o empurrou"
(Ibid.),  --- o oitavo preceito; "Ou jogou sobre ele algum instrumento sem ser
de emboscada" (Ibid.) --- o nono preceito; "Ou não o vendo... alguma
 pedra que possa causar-lhe a morte" (Ibid., 23) --- o décimo preceito;
 "Jogou sobre ele... e este morrer, não sendo ele seu inimigo" (Ibid.)
 --- o décimo primeiro precei­to; ``E salvará a congregação ao homicida''
 (Ibid., 25) --- o décimo segundo pre­ceito; "E a congregação o fará
 voltar à sua cidade de refúgio" (Ibid.) --- o déci­mo terceiro
 preceito; "E ficará nela até morrer o 'Cohén Gadol"' (Ibid.) --- o
 

10. Através do noivado legal, que acarreta as responsabilidades legais
do casamento.

décimo quarto preceito; ``E se sair o matador fora'' (Ibid., 26) --- o
décimo quinto preceito; "E depois da morte do 'Cohen Gadol' voltará o
homicida" (Ibid., 28) --- o décimo sexto preceito.

Se devêssemos fazer isso em todo e cada preceito, o número de pre­ceitos
ultrapassaria os dois mil! É óbvio, contudo, que isso não seria
racional, já que todas essas leis são apenas detalhes de um preceito, e
que o preceito a ser contado é a lei do homicídio, ou seja, que devemos
lidar com o assunto em questão de acordo com a lei estabelecida nesses
versículos. Na realidade, o Eterno se referiu a eles como ``mishpatim''
(leis) e não como ``mitzvot'' (pre­ceitos). Ele disse: "Então julgará a
congregação entre quem feriu e o vingador do sangue, segundo estas leis"
(Ibid., 24).

O autor do ``Halachot Guedolot'' já se preocupou com relação a es­te
assunto e prestou atenção a ele, mas quando se deparou com complicações
ele começou a contar seções, enumerando ``a seção de herança'', "a seção
de promessas e juramentos", ``a seção do difamador'' e muitas outras
assim. En­tretanto, este conceito não lhe ficou totalmente claro e nem
foi completamen­te compreendido por ele, e assim ele colocou nessas
seções, sem notar, assun­tos que já havia enumerado anteriormente.
Assim, devido ao fato de desconhe­cer este Fundamento, ele contou onze
preceitos com relação à lepra, sem per­ceber que nesse caso há apenas um
preceito e que tudo o que está mencionado nas Escrituras não é mais do
que a enumeração detalhada de suas leis e condições.

O significado disto é o seguinte. Foi-nos ordenado que uma pessoa, ao
tornar-se impura por causa da lepra, deve cumprir todas as obrigações
im­postas aos impuros, ou seja, afastar-se do Santuário e de suas
ofertas consagra­das, e sair do campo da Presença Divina. Contudo, como
ainda não sabemos qual tipo de lepra impurifica uma pessoa e qual não,
as Escrituras começaram, conseqüentemente, a explicar e a detalhar a
lei: se for assim, ele estará puro; se for de outra forma, ele estará
impuro; e se ela for de uma determinada ma­neira, ele deverá afastar-se
por um certo período de tempo.

Os Sábios dizem explicitamente: "Para declará-las puras ou impuras"
(Levítico 13:59) --- assim como é obrigatório declará-la pura, também é
obriga­tório declará-la impura. Assim, é óbvio que o preceito consiste
apenas em declará-lo "impuro'.' ou ``puro'', mas os detalhes que tornam a
pessoa impura ou pura não devem ser contados, uma vez que eles nada mais
são do que as condições e os detalhes da lei. Isso é o mesmo que dizer
que não se deve ofere­cer um animal defeituoso como sacrifício, o que
seguramente constitui um pre­ceito negativo; resta-nos saber apenas o
que é considerado como defeito. Mas devemos contar todo defeito como um
preceito independente? Se assim fosse, o número chegaria a cerca de 70!
Portanto, assim como não contamos os defei­tos --- qual deles é
considerado como defeito e qual não --- mas sim apenas a advertência que
nos previne para não oferecer um animal defeituoso, assim tam­bém não
devemos contar as manifestações da lepra --- qual é impura e qual é pura
--- mas sim contar apenas que a lepra é impura, sendo todo o resto uma
explicação sobre de que consiste a lepra.

É segundo esse método que devemos contar como um único pre­ceito cada um
dos (treze) tipos de impureza, e não contar os detalhes das leis e
condições de um tipo específico de impureza, como será esclarecido em
nos­sa enumeração.

Compreenda este Fundamento, pois ele é uma ``coluna central'' de apoio
neste assunto.

\chapter*{O oitavo fundamento\subtitulo{Não se deve incluir entre os preceitos negativos uma declaração negativa que exclua um caso particular de um determinado assunto}}

Você deve saber que uma proibição é uma das duas partes de uma ordem.
Quer dizer, pode-se ordenar a uma pessoa que faça algo ou que não o
faça, como por exemplo, quando você mandar alguém comer alguma coisa e
lhe disser: ``Coma'', ou quando mandar que ele se abstenha de comer e lhe
disser: ``Não coma''. No idioma árabe, porém, não há uma palavra que
abranja esses dois significados. Os mestres do Kalam dizem, na teoria da
lógica, algo assim: "Em árabe o preceito e a advertência não têm uma
palavra em comum para expressá-los, e por isso fomos obrigados a
referir-nos a ambos com o mes­mo termo, que é o preceito". Assim fica
explicado que uma advertência e um preceito são a mesma coisa. A palavra
usada em árabe no sentido de advertên­cia é ``LA'' (não imperativo). Este
aspecto da comunicação --- a saber, ordenar que uma pessoa faça ou deixe
de fazer alguma coisa --- é encontrado sem dúvi­da alguma em todos os
idiomas. Está, portanto, claro que tanto o preceito posi­tivo quanto o
negativo englobam ordens absolutas: a de fazer determinadas coi­sas e a
de prevenir-nos para não fazer outras. O termo que define as coisas que
somos ordenados a fazer é ``preceito positivo'' e o que define as coisas
que so­mos avisados para não fazer é ``preceito negativo'', e o nome que
abrange os dois juntos no idioma hebreu é ``Guezerá'' (decreto). Dessa
forma os Sábios se referem a todos os preceitos, positivos e negativos,
como sendo "os decre­tos do Rei".

Contudo, uma simples declaração negativa que exclua alguma coisa de um
determinado assunto é algo diferente, pois não há uma ordem com rela­ção
a ela. Esse é o caso, por exemplo, quando se diz: "Aquela pessoa não
co­meu ontem, e aquela outra não bebeu o vinho, e Zaid não é o pai de
Ornar", e outras declarações semelhantes. Todas elas são meras negações
e não têm a menor semelhança com uma ordem.

A palavra usada em árabe para negar uma determinada coisa é quase sempre
``MA'', mas algumas vezes eles usam as palavras ``LA'' ou "LAISA' . Por
outro lado, os hebreus negam a maioria das vezes com a mesma palavra
``LO'', também usada para fazer uma advertência. Eles negam também com a
palavra ``AYIN'', bem como com as formas que essa palavra adquire quando
se lhe acres­centam pronomes como sufixos, tais como ``EINO'' (ele não é),
``EINAM'' (eles não são), ``EINCHEM'' (você não é), e outras.

Encontram-se negações em hebreu com a palavra ``LO'' em versícu­los tais
como: "E não (VE'LO) se levantou mais, em Israel, profeta algum como
Moisés" (Deuteronômio 34:10); "Deus não (LO) é homem, para que minta"
(Nú­meros 23:19); "Não (LO) haverá desgraça duas vezes" (Nahum 1:9); "E
não (VE'LO) ficou homem com ele" (Gênesis 45:1); "E ele não (VE'LO) se
levantou nem foi em sua direção" (Esther 5:9); e em muitos outros casos
como esses. Negações com a palavra ``EIN'' aparecem, por exemplo, nos
versículos "E homem 
não (AYIN) existia" (Gênesis 2:5); "Mas os mortos não (EINAM) sabem
nada" (Ecc. 9:5), e muitos outros versículos além destes.

Ficou, assim, clara a diferença entre uma negação e uma advertên­cia. A
advertência tem um caráter de obrigatoriedade e é, na realidade, o verbo
em sua forma de ordem. Quer dizer, da mesma forma que uma ordem, uma
advertência aparece sempre no futuro; assim como é inconcebível num
idioma que uma ordem seja dada no tempo passado, o mesmo ocorre com uma
adver­tência; assim como é impossível introduzir uma ordem numa sentença
que te­nha um relato ou uma narração --- pois uma sentença assim deve
ter um sujeito e um objeto, enquanto que uma ordem constitui por si só
uma expressão com­pleta, como foi explicado nos livros que tratam desse
assunto --- também não se pode introduzir uma advertência numa narração.
Mas nada disto se aplica a uma negação. Uma declaração negativa pode
entrar numa narração e pode referir-se ao passado, futuro ou presente.
Tudo isso fica evidente quando se reflete a respeito.

Sendo assim, não devemos em hipótese alguma contar entre os pre­ceitos
negativos as declarações que forem apenas negações. Naturalmente, isto é
evidente por si só, sem necessidade de provas, a não ser o que foi
menciona­do com relação ao esclarecimento do conteúdo de certas
expressões a fim de permitir a distinção entre a advertência e a
negação.

Um outro, contudo, não estava a par disto e conseqüentemente con­tou
``Não sairá como saem os escravos'' (Êxodo 21:7), sem perceber que esta
era uma declaração negativa, e não uma proibição.

Vou explicar este assunto. O Eterno decretou que se um senhor fe­rir seu
servo ou serva cananeus, causando-lhes a perda de um de seus órgãos
externos, eles deverão ser libertados. A partir disso poderíamos pensar
que es­sa lei se aplica com toda certeza à serva hebréia, e que caso o
seu senhor lhe cause a perda de um de seus principais órgãos externos,
ela deverá ser liberta­da; é por isso que Ele a excluiu dessa lei,
dizendo: ``Não sairá como saem os escravos'', que é como se Ele estivesse
dizendo que não há obrigação de liber­tá-la caso ele lhe tenha causado a
perda de um de seus órgãos. Dessa forma, o versículo nega a aplicação de
uma determinada lei com relação a ela, mas não é uma proibição. Os
transmissores da Tradição explicaram isto, dizendo o se­guinte na
Mekhiltá: 'Não sairá como saem os escravos' significa que ela não será
libertada por causa da perda de um dos órgãos principais, como acontece
com os escravos cananeus". Foi, assim, explicado que esta é apenas a
negação de aplicar a ela uma determinada lei, mas que não constitui uma
advertência.


Basicamente também não há diferença entre Suas palavras "Não sairá 
como saem os e vos" e "Não procurará o 'Cohen' o pelo louro; impuro
é ele" (Levítico 1 Este versículo constitui igualmente uma negação absoluta, 
e não um•ncia. Quer dizer, Ele nos ensina que ao apresentar um\\
determinado si olamento se torna desnecessário e o ``Cohen'' não deve 
hesitar em o impuro.

Da mesma forma Suas palavras "Ele não morrerá, pois ela não era
libertada" (Levíti e 9:20) não constituem uma advertência, mas sim uma
ne­gação que significa, na realidade, que ele não está sujeito à pena de
morte por­que sua liberdade não estava completa. Não se deve traduzir
esse versículo co­mo ``Elès não deverão ser mortos'' pois assim ele se
transformaria numa adver­tência. Suas palavras "Ele não morrerá, pois
ela não era libertada" são semelhantes 
ao que Ele disse: ``A moça não tem pecado de morte'' (Deuteronômio
22:26). Assim como ele negou a ela a pena de morte porque ela estava sob
coa­ção, Ele também negou a ele a pena de morte por causa da escravidão
da moça, como se tivesse dito que ele não está sujeito à pena de morte
porque ela não era livre.

Um outro caso de negação é o que está em Suas palavras "Para que não
seja como Korah e como sua congregação" (Números 17:5). Os Sábios
ex­plicaram que este versículo é uma negação e interpretam seu
significado dizen­do o seguinte: o Eterno declara que o castigo imposto
a todo aquele que vier e disputar o sacerdócio, reclamando-o para si
próprio não será o mesmo que se abateu sobre Korah e sua congregação ---
a saber, ser tragado pela terra e devorado pelo fogo --- e sim será
"Conforme tinha falado o Eterno, por inter­médio de Moisés", isto é, a
lepra. Isto está expresso em Suas palavras "Leva, por favor, a tua mão
ao teu peito... E a tirou e eis que sua mão estava leprosa como a neve"
(Êxodo 4:6). Eles oferecem como prova disto o que aconteceu com Uziah,
rei de Yehudá. Embora encontremos uma opinião diferente dos Sá­bios na
Guemará de Sanhedrin, afirmando que "Todo aquele que estimula a
discórdia viola um preceito negativo, pois está dito: 'Para que não seja
como Korah e como sua congregação--- , isso é apenas o
significado moral, mas não é o significado literal do versículo. Na
realidade, a advertência quanto a esse conceito está em outro preceito,
como explicarei no lugar apropriado.

Não há nenhuma regra precisa para se fazer a distinção entre urna
declaração negativa e uma advertência, a não ser o sentido da
declaração. Não há urna palavra específica que diferencie a negação da
advertência, pois ambas são expressas em hebreu pela mesma palavra:
``LO''. Assim, é importante que aquele que analisa o assunto profundamente
pese com cuidado em sua mente o significado das palavras e então ele
perceberá facilmente qual expressão ne­gativa constitui uma negação e
qual uma advertência, como já explicamos anteriormente.

Os Sábios, a paz esteja com eles, comentaram este assunto com
refe­rência a uma controvérsia que surgiu entre eles devido a uma de r
inada ex­pressão negativa, para saber se ela era uma simples negação o
urna dvertên­cia. Isso ocorreu por causa do que está expresso em Suas p
vras, e altecido seja Ele, acerca do pássaro de Sacrifício de Pecado: "E
dest • ncará s cabeça pela nuca, porém não o separará" (Levítico 5:8).
``Nosso T ná''I., q e fala na Mishná, é de opinião que é
uma advertência, e portanto d "Se ele o separar completamente ele o
invalidará". Conseqüentemente, esta •claraç.o negativa
constitui um preceito, pois se ele o separar, ele o invalida, a • orno
quem oferece levedura ou mel. Por outro lado, Rabi Elazar, o filho do
Rabi Shimon, é de opinião que este versículo não é urna advertência, e
sim uma negativa, e que as palavras ``Não o separará'' significam que não
é necessário separar a ca­beça e que será suficiente se ele cortar
apenas uma parte dela; dessa maneira, na sua opinião, o sacrifício
também será válido se ele a separar completamente. Os Sábios dizem o
seguinte, na Guemará de Zebahim: "Rabi Elazar, o filho de Rabi Shimon,
costumava dizer: Ouvi dizer que eles separam completamente o pássaro de
oferta de pecado, mas Suas palavras 'não o separará significam que ele
não precisa separá-lo". A respeito dessas palavras os Sábios perguntaram
o seguinte: "Então você também diria que no caso de um Poço, sobre o
qual está dito: 'E não o cobrir' (Êxodo 21:33), isso significa que ele não precisa
cobri-lo?" A resposta a isso foi: "O versículo diz que 'O dono do poço
pagará' (Ibid. 34). Isso deixa claro que ele deve cobri-lo".

Foi, assim, explicado que os Sábios apresentam provas sobre se se trata
de uma negação ou de uma açlvertência a partir do sentido da própria
declaração.

Também foi explicado que Suas palavras ``Não o separará'' consti­tuem um
preceito negativo, como está evidenciado na Mishná. E, finalmente, isso
também deixará claro que Suas palavras com relação ao pássaro de
holo­causto, "E o rasgará pelas asas, mas não o dividirá" (Levítico
1:17) não devem ser contadas, pois todos os Sábios concordam que mesmo
que ele o divida com­pletamente, o sacrifício ainda será válido. Isso
porque, como no caso do ani­mal que se oferece como holocausto Ele diz:
``E o cortará em seus pedaços'' (Ibid., 12), poder-se-ia pensar que a
mesma regra se aplica ao pássaro de holo­causto, por isso Ele diz que
não é necessário dividi-lo, e sim apenas rasgá-lo, e caso ele o divida
completamente, o sacrifício ainda será válido, como será explicado no
devido lugar.

Um outro caso de um versículo que pertence à categoria de declara­ções
negativas é o versículo "Toda a consagração para pagar o resgate da
ava­liação da pessoa condenada à morte, não poderá ser feita" (Ibid.,
27:29). Uma vez que você saiba qual é o significado dessa afirmação,
ficará claro para você que se trata de uma negação, e não de uma
advertência. E o seguinte. As Escri­turas estipularam que um determinado
pagamento seja feito por ``avaliações'', de acordo com a idade da pessoa
avaliada e dependendo se ela for homem ou
mulher. Com relação a este conceito não faz
diferença se alguém disser: "Mi­nha avaliação cabe a mim" ou "A
avaliação de tal pessoa cabe a mim", pois nesses casos vemos o que ela é
e qual a sua idade, e ela pagará de acordo com isso. Mas se a pessoa
avaliada for uma que ficou sujeita à pena de morte pelo Tribunal e foi
julgada culpada, e se alguém então disser: "A avaliação dessa pes­soa
cabe a mim", ela não precisa pagar nada, pois a partir do final do
julgamen­to ela é considerada como morta, e não se avalia os mortos.
Portanto, este é o sentido de Suas palavras ao dizer: "Não poderá ser
feito" (Ibid.): não é preci­so pagar por essa pessoa o resgate que
normalmente aquele que fez o voto de avaliação deveria pagar. Esta é uma
das leis e estatutos sobre as avaliações que foi mencionada nas
Escrituras, mas não é uma advertência.

A Mishná diz: "Não se fazem votos nem avaliações com relação a um
moribundo ou a alguém que foi condenado à morte". O Talmud explica que
isto se aplica a alguém condenado à morte por um tribunal israelita. A
Mekhiltá também diz: "As pessoas sujeitas à morte pelo tribunal não têm
resgate, pois as Escrituras dizem: 'Toda a consagração para pagar o
resgate da avaliação da pessoa condenada à morte, não poderá ser feita".
Avalie a exatidão e a profun­didade das palavras dos Sábios, pois quando
dizem ``não tem resgate'' --- e não ``não deverão ser resgatadas'' --- eles
deixam claro que esta afirmação é uma declaração negativa e não uma
advertência.

Esse mesmo assunto está explicado pelos Sábios na Sifrá, na seção de
Avaliações, onde dizem: "De que forma sabemos, se uma pessoa disser com
relação a um condenado a morte pelo Tribunal: 'Sua avaliação cabe a
mim', que suas palavras não têm efeito?" --- significando que ele não é
obrigado a pagar nada? "Pelas palavras das Escrituras: 'Ele não poderá
ser resgatado"'.

Este assunto foi tão perfeitamente explicado que na minha opinião até
mesmo uma pessoa obtusa não terá mais dúvidas a este respeito.


E já que estamos falando deste assunto, você deve saber que há quatro

palavras na Torah --- a saber, ``Hishamer'' (guarda-te de), ``pen'' (para
que não), ``aI'' (não faça), e ``lo'' (que não haja) --- que são usadas para
estabelecer uma ad­vertência, e tudo o que tiver sido advertido através
de uma dessas palavras se cha­ma preceito negativo. Os Sábios dizem
claramente: "Toda vez que aparece 'guar­da-te de', 'para que não', 'não
faça' e 'que não haja' há um preceito negativo".

Resta-nos explicar o seguinte ponto para que possamos completar o
propósito desta seção. Toda vez que aparece na Torah que somos obrigados
a proclamar que não fizemos um determinado ato, para assim isentarmo-nos
de toda a responsabilidade quanto a ele, esse ato específico deve ser
contado entre os preceitos negativos, ainda que a proibição que aparece
nele seja ape­nas uma negação e não uma advertência. Pois se Ele nos
obriga a isentarmo-nos dizendo: ``Eu não fiz isto ou aquilo'', fatalmente
concluímos que estamos sen­do advertidos para não fazer essas coisas.
Tal é o caso, por exemplo, em que a Torah nos obriga a dizer: "Não comi
do segundo dízimo no primeiro dia de luto, e não comi dele em estado de
impureza, e não o troquei para fazer o se­pultamento de um morto"
(Deuteronômio 26:14). Através dessas palavras fica óbvio que fomos
advertidos para não realizar nenhum desses atos, como será explicado no
devido lugar, quando falarmos desses preceitos.

\chapter*{O nono fundamento\subtitulo{Esta enumeração não deve ser baseada 
no numero de vezes que um determinado preceito negativo ou 
positivo está repetido nas escrituras, mas sim na natureza da ação proibida ou ordenada}}

Você deve saber que todas as obrigações e advertências da Torah se
referem a quatro coisas: as opiniões, os atos, os traços de caráter e o
que se diz. Assim, a Torah nos ordena a acatar certas opiniões, tais
como acreditar na unidade, amar a Deus e temê-lo, enaltecido seja Ele,
ou nos adverte para que não acreditemos em certas' opiniões, tais como
acreditar em e atribuir divinda­de a outro que não Ele. Da mesma forma
ela nos manda realizar certos atos, tais como oferecer os sacrifícios e
construir o Santuário ou nos previne quanto a certas ações, tais como a
advertência para não oferecer sacrifícios a outros que não Ele,
enaltecido seja Ele, ou curvar-se diante de outros que não Ele. As­sim
também ela ordenou que rios conduzamos de acordo com determinados traços
de caráter, tais como a benevolência, a misericórdia, a piedade e o
amor, conforme está no versículo "E amarás o teu próximo como a ti
mesmo" (Leví­tico 19:18), ou nos adverte com relação a determinados
outros traços de cará­ter, tais como guardar rancor, recompensar o mal
ou vingar-se, e outros mais, como explicarei. A Torah nos ordena recitar
certas palavras, tais como expres­sar a nossa gratidão a Ele, orar a
Ele, confessar os pecados e outros assuntos, similares, como 
expl,icarei, ou nos previne quanto a dizer certas
coisas, tais co­mo a advertência contra pronunciar um falso juramento,
fazer intrigas, falar mal dos outros, amaldiçoar, e outros além destes.

Quando estes aspectos forem compreendidos ficará claro que é a na­tureza
do assunto ordenado ou proibido --- quer se refira ele a um ato, a algo
que se disse, a uma opinião ou a um traço de caráter --- que deve ser
contado, e que não devemos levar em consideração o número de vezes que
determina­da ordem ou advertência está repetida --- dependendo se se
trata de um precei­to positivo ou de um negativo ---, pois a finalidade
de todas as repetições é dar maior ênfase, realçando, com a repetição da
lei, o assunto proibido ou ordena­do. Apenas quando você encontrar uma
declaração dos Sábios relativa à divi­são dos assuntos, e quando tiver
sido explicado pelos Intérpretes que cada um desses preceitos negativos
ou positivos contém um assunto específico, não abrangido pelo outro, só
assim deve-se contá-los, mesmo que à primeira vista possa parecer que
eles tratam de um mesmo assunto, pois o objetivo das repe­tições não
será enfatizar e sim dar instruções a respeito de aspectos adicionais.
Somente quando não tivermos escolha, e não encontrarmos apoio nas
palavras dos Intérpretes, 'os guardiães da Tradição, dizendo que o
versículo foi repetido para acrescentar alguma instrução, diremos
necessariamente que ele foi repeti­do para dar maior realce. Mas se
encontrarmos uma Tradição de que uma certa ordem ou proibição se refere
a um determinado assunto, e que a repetição des­sa declaração acrescenta
alguma coisa, será certamente verdadeiro e correto afir­mar que o
versículo foi repetido para ensinar-nos algum princípio novo, e nes­se
caso cada versículo deverá ser contado separadamente. Mas onde nada de
novo tiver sido acrescentado, o objetivo da repetição será dar maior
ênfase, informar-nos que a transgressão daquela lei é muito grave --- já
que as Escritu­ras nos advertem várias vezes a esse respeito ---,
completar a lei de um determi­nado preceito, ou deduzir a partir dela
certa lei para outro preceito, como ex­plica o Talmud ao dizer: "Ela
está repetida a fim de constituir a base para uma analogia ou para
ensinar-nos através da dedução de frases semelhantes".

Verificamos que os Sábios, a paz esteja com eles, comentam este pon­to
no segundo capítulo da Guemará Pessahim ao discutir uma determinada
proi­bição que parece estar repetida porque já havia sido deduzida de
outro precei­to, e que por isso querem aplicá-la como uma instrução
adicional. Eles dizem, como discussão e consideração: "Rabina disse a
Rav Ashei: Talvez seja porque a pessoa transgrediu dois preceitos
negativos". Em outras palavras, por que pro­curar a solução tentando
aplicar esta proibição a algo diferente daquilo que con­cluímos com a
proibição original? Talvez ela tenha sido repetida com referên­cia ao
mesmo assunto, de modo que aquele que a desobedecer será culpado por
transgredir duas proibições. A resposta a isso foi: "Disse-lhe: Sempre
que há uma possibilidade de interpretar o versículo nós o fazemos e não
fazemos dele uma proibição adicional". Portanto, foi deixado claro que
uma proibição que não estabelece alguma instrução nova é chamada de
``adicional'', ou seja, é repetitiva. Assim, embora os Sábios falem de
"culpado por haver transgredi­do dois preceitos negativos", fica claro,
por toda esta explicação, que se trata apenas de uma proibição adicional
e que por essa razão ela não deve ser conta­da. Dessa forma foi deixado
claro que, a enumeração dos preceitos não deve ser baseada no número de
vezes que um determinado preceito negativo ou po­sitivo foi repetido.

Sabe-se que o preceito que nos ordena descansar no Shabat está
men­cionado doze vezes na Torah. Acaso alguém que enumera os preceitos
diria "Entre os preceitos positivos está descansar no Shabat, que
consiste de 12 preceitos"? 
Uma pessoa sensata diria que a proibição de comer sangue
consiste de sete preceitos? Ninguém vai se enganar com relação ao fato
de que descan­sar no Shabat é apenas um dos preceitos positivos e de que
a advertência quan­to a comer sangue é apenas um dos preceitos
negativos.

Você deve saber que mesmo que se encontre uma frase dos Sábios dizendo
que aquele que comete uma certa transgressão viola dessa forma um certo
número de proibições, ou que aquele que deixa de fazer um certo ato
viola por causa disso um certo número de obrigações, não se deve
concluir que devemos contar separadamente cada uma dessas proibições ou
obrigações, pois a natureza da ação é uma só, e não várias. O fato de
dizerem que a pessoa transgride tantos preceitos positivos ou negativos
só se deve à repetição da or­dem ou da advertência feita com relação
àquele preceito em particular, pois ele violou aquele número de
advertências ou ordens. Apenas quando os Sábios disserem que "ele deve
ser açoitado duas vezes" ou que ``ele deve ser açoitado três vezes'', aí
então cada advertência deve ser contada separadamente, já que ninguém é
açoitado duas vezes por violar um único preceito, como explicarei de
acordo com o que se sabe pelos textos do Talmud --- em Macot, Hulin, e
outros trechos. Contudo, administram-se dois açoitamentos por dois
preceitos, isto é, por dois assuntos independentes para os quais haja
advertências separadas.

Portanto essa é a diferença entre eles dizerem "ele violou tantos e
tantos" e ``ele deve ser açoitado duas ou três vezes''.

Encontramos a prova de tudo o que dissemos nas palavras dos Sá­bios
"Aquele que não tiver `tsitsit' em suas vestes violará cinco preceitos
posi­tivos", porque a ordem relativa a eles aparece cinco vezes: "Que
façam para eles 'tsitsit'... e porão sobre os `tsitsit' ... E será para
vós por 'tsitsit' ---  (Números 15:38-39). " 'Tsitsie
farás ra ti e os porás nos quatro cantos de tua vestimen­ta"
(Deuteronômio 22:1 tudo, encontramos uma declaração explícita dos Sábios
com relação ao receit dos ``tsitsit'' dizendo que se trata de um único
preceito, como explic eii. q ando tratar dele.

Similar a isso disseram: "Quem não coloca 'Tefilin' viola 
oito preceitos positi o ", ordem relativa a eles --- isto é, o "Tefi-

lin" da cabeça e o do b aço oito vezes. Também disseram: "O 'Cohen' 
que não subir a platafo la três preceitos positivos" porque a ordem 
relativa a isto está rep ti vezes. Mas ninguém que enumerasse os
preceitos pensaria que a Bênção do ``Cohen'' constitui três preceitos e
que o ``Tefilin'' constitui oito.

Sendo assim, conclui-se que não devemos contar "enganar um pro­sélito"
como três preceitos devido à reiteração da proibição e às palavras dos
Sábios na Guemará de Metzia: "Aquele que enganar um prosélito violará
três proibições e aquele que o oprimir violará três proibições". Ao
contrário, deve­mos contar apenas os dois preceitos segbintes: "E ao
peregrino não o frauda­reis e não o oprimireis" (Êxodo 22:20), sendo os
outros repetições destas proi­bições. Não há dúvidas q to a isto.

Os Sábios diz. .licitamente na Guemará de Metzia: "Por que a
Torah adverte em trinta lugares para que não se trate mal o prosélito?

Porque seu temperamen mau". Alguém pensaria em incluir esses trinta
e seis preceitos nos seis s e treze preceitos? É totalmente
inconcebível.
% Nota
 % Ver o preceito positivo 14.

 % A plataforma onde os ``Cohanim'' fazem a bênção do povo.
  
 % Significando seu instinto do mal.
 
Dessa forma, foi explicado e esclarecido que nem todo preceito ne­gativo
ou positivo encontrado na Torah deve ser contado, pois pode ser que ele
seja apenas uma repetição; somente deve ser contado o conceito da ação
ordenada ou proibida. Apenas um professor --- um dos transmissores do
Co­mentário, a paz esteja com eles --- pode instruir-nos quanto a se um
determina­do preceito positivo ou negativo reaparece a fim de
estabelecer alguma instru­ção adicional ou não.

Também não se deixe confundir por uma proibição que aparece sob
diferentes formas, tal como: ``E tua vinha não rebuscarás'' (Levítico
19:10), "E esqueceres uma gavela no campo, não voltarás a tomá-la"
(Deuteronômio 24:19), e "Quando bateres a tua oliveira (lo tefo'er), não
tornarás a colher o que resta nos ramos" (Ibid.,20). Na realidade, elas
não são duas proibições, e sim uma única advertência relativa a um único
assunto, a saber, que não se deve voltar para buscar o cereal ou as
frutas esquecidas durante a colheita e por isso Ele mencionou dois
exemplos: as uvas e as olivas. O significado da palavra "Lo te­fo'er" é:
não corte o que você esqueceu no fim dos galhos, isto é, os ramos.

Explicarei agora o que deve ser acrescentado a este fundamento. É o
seguinte. O que dissemos, a saber, que devemos contar os conceitos que
nos foram ordenados ou proibidos, está condicionado ao fato de que para
cada um desses conceitos haja um preceito negativo específico ou uma
prova de que os mestres da Tradição separam um conceito do outro,
resultando cada um nu­ma advertência. Mas se uma proibição inclui muitos
assuntos, então contamos apenas essa proibição e não cada um dos vários
conceitos incluídos nela. Essa é a proibição global (Lav shebikhlalut),
cuja violação não acarreta açoitamento, como vamos explicar a seguir.

Ao comentar Suas palavras, enaltecido seja Ele, "Não comereis so­bre o
sangue" (Levítico 19:26), os Sábios dizem: "De que forma sabemos que é
proibido comer da carne de um animal antes que a vida o tenha abandonado
por completo? Pelo v sl lo: 'Não comereis sobr sangue'. Outra
interpre­tação: 'Não comereisr.obre o sangue' --- não com a c• ne
enquanto o sangue ainda estiver na tigel•• Dossá diz: 'De qu form
sabemos que não de­vemos dar alimentos efeição de Conforto
%%%%%%%%
\textsuperscript{17} 
%%%%%%%%
por alguém que foi execu­tado judicialmente?' 'e •
ersículo 'Não comera s sob o sangue'. Rabi Akiba diz: 'De que forma
sabemos que um Sanhedrin - alizou uma execução não deve er nesse dia?'
Pelo versículo 'Não come eis sobre o sangue'. Rabi Yossi, o fi • de abi
Haniná, disse: 'De que forma deduzimos a advertência com rela­çã•o filhe
impertinente e rebelde?' Pelo versículo São comereis sobre o san­gue
%%%%%%%%
\textsuperscript{18}. 
%%%%%%%%
As im, pois, temos cinco assuntos
sujeitos todos eles a uma advertên­cia *ncluí s todos nessa
proibição. Com relação a eles os Sábios
dizem expli­citamente, a Guemará de Sanhedrin: "Nenhum deles acarreta o
açoitamento pois tra se de uma proibição global e não se aplica o
açoitamento por uma proibição global". Mais adiante eles explicam que
uma proibição global é aque­la que dá origem a duas ou três proibições.
Portanto fica, claro que não deve­mos contar cada proibição incluída
nesse preceito negativo como sendo um preceito separado, mas sim como um
único preceito negativo que abrange to­das elas.

 
%  Ou seja, antes do sangue ser aspergido sobre o altar.
 
% \item
 
%  Dada ao enlutado após o funeral.
 
% \item
 
%  Que quer dizer ``Não comereis a comida que acarreta a pena de morte''.
 

Semelhante à proibição ``Não comereis sobre o sangue'' são Suas pa­lavras
``E diante do cego não porás tropeço'' (Levítico 19:14) porque elas
tam­bém incluem muitas proibições, como vamos explicar. Da mesma forma
Suas palavras ``Não dês ouvidos à maledicência'' (Êxodo 23:1) incluem
muitos con­ceitos, como explicaremos. Este é o primeiro dos dois tipos
de proibições globais.

O segundo tipo consiste de um preceito negativo que proíbe várias coisas
juntas e acrescidas umas as outras, tal como quando Ele diz: "Não faça
isto e aquilo". Por sua vez, este tipo se divide em duas partes, e de
acordo com a explicação do Talmud uma delas acarreta o açoitamento por
cada um dos con­ceitos, e a outra causa um único açoitamento, por ser
uma proibição global. Segundo os Sábios, devemos contar como preceitos
separados cada uma das proibições que ocasionam um açoitamento por cada
conceito, e devemos con­tar como um único preceito as proibições que
ocasionam o açoitamento uma única vez, de acordo com o que estabelecemos
neste Fundamento --- que em circunstância alguma se é açoitado duas
vezes por um único preceito. Por ou­tro lado, onde eles estabeleceram
claramente que se está sujeito ao açoitamen­to por cada um dos assuntos
ligados e relacionados entre si, de maneira que aquele que fizer todos
ao mesmo tempo estará sujeito a vários açoitamentos, concluímos com
certeza que eles constituem vários preceitos, a serem conta­dos
separadamente.

Mencionarei agora vários exemplos das duas partes desse segundo tipo. É
possível até que mencione todos os preceitos negativos desta categoria
para que o assunto fique completamente elucidado.

Contamos a proibição expressa em Suas palavras, enaltecido seja Ele,
relativa ao cordeiro Pascal "Não comais dela mal passada no fogo nem
cozida na água" (Êxodo 12:19) como um preceito, e não "Não comais dela
mal passa­da no fogo" como um e ``Não comais dela cozida na água'' como
outro precei­to, pois Ele não expressou especificamente uma declaração
proibitiva em cada assunto dizendo "Não comais dela mal passada nem
tampouco a comais cozida na água". Em vez disso, Ele expressou uma
proibição que inclui dois assuntos ligados e relacionados entre si.

No capítulo de Pessahim os Sábios dizem: "Abayé disse que se ele a comeu
mal passada, ele deve ser açoitado duas vezes; se a comeu cozida na
água, duas vezes; e se a comeu mal passada e cozida na água, três
vezes". Isso se deve ao fato de que ele acredita que se deve ser
açoitado por desobedecer uma proibição global; portanto, quando alguém a
comer mal passada, estará transgredindo duas proibições: a que diz "Não
comais dela mal passada no fo­go" e a que se conclui por dedução, que é
como se Ele tivesse dito: "Coma-a apenas grelhada", ao passo que ele a
comeu de outra forma. E se ele a comeu mal passada e cozida na água, ele
será açoitado três vezes, de acordo com Aba­yé: uma por comê-la mal
passada, outra por comê-la cozida na água, e uma ter­ceira por tê-la
comido sem ser grelhada.

Continuando este debate, disseram ali: "Mas Rabá diz que não se fi­ca
sujeito ao açoitamento por uma proibição global. Há quem diga que se
fica sujeito a pelo menos um açoitamento", isto é, se a comer mal
passada e cozida na água, será açoitado uma vez. "Outros dizem que não
se fica sujeito a açoita­mento algum, pois esta não é tão específica
quanto a proibição de colocar uma focinheira". Este é o preceito "Não
amarrarás a boca ao boi quando estiver de­bulhando" (Deuteronômio 25:4),
que consiste de uma proibição advertindo con­tra fazer uma única coisa,
enquanto que a outra proibição adverte com relação a duas coisas --- mal
passada e cozida em água --- e conseqüentemente não su­jeita o
transgressor a açoitamento algum.

Você já está familiarizado com o que está explicado na Guemará de
Sanhedrin: "Não se fica sujeito ao açoitamento por uma proibição
global". Con­seqüentemente, as palavras de Abayé são rejeitadas, sendo a
opi 1. correta a que diz que se fica sujeito a apenas um açoitamento,
quer se enha c mido mal passada ou cozida na água, ou mal passada e
cozida na á: a Por essa ,azão devemos contar Suas palavras, enaltecido
seja Ele, "Não com. i dela mal pq ssa­da no fogo nem
cozida na água" como um preceito apena

Ali disseram também os Sábios: "Abayé disse qu se \textsuperscript{19}
co .sse a casca da uva, ele seria açoitado duas vezes;
se comesse caro • s de u , duas vezes; se comesse cascas e caroços de
uva, três vezes. Mas R. que não se fica sujeito ao açoitamento por causa
de uma proibição global", fazendo alu­são a Suas palavras "De tudo o que
sai da videira" (Números 6:4) pelo que, na opinião de Abayé, fica-se
sujeito ao açoitamento.

Da mesma forma eles dizem, no quarto capítulo de Menahot: "Aba­yé disse
que aquele que oferecer lêvedo e mel sobre o altar deverá ser açoitado
uma vez pelo lêvedo, uma vez pelo mel, uma vez por ter misturado o
lêvedo e uma vez por ter misturado o mel" (12). Ou seja, a palavra ``col''
inclui duas coisas: que essas coisas não devem ser oferecidas separadas
nem misturadas, seja em que quantidade for. Como você já sabe, tudo isso
está de acordo com a teoria essencial de Abayé de que se está sujeito ao
açoitamento por uma proi­bição global. E os Sábios prosseguem, dizendo:
"Mas Rabá diz que não se fica sujeito a castigo por causa de uma
proibição global. Alguns dizem que se está sujeito a pelo menos um
açoitamento, e outros dizem que não se está sujeito a nenhum, uma vez
que ela não é tão específica como a proibição de colocar uma
focinheira".

Então, como foi explicado que os versículos "Não comais dela mal passada
no fogo nem cozida na água" e ``Não fareis queimar fermento algum ou mel''
(Levítico 2:11) constituem um preceito cada um, assim também conta­remos
cada um dos seguintes versículos como um só preceito: "Não entrará
nenhum amonita e nem moabita" (Deuteronômio 23:4); "A nenhuma viúva ou
órfão afligireis" (Êxodo 22:21); "Não perverterás o juízo do peregrino e
do ór­fão" (Deuteronômio 24:17,); "Sua manutenção, seu vestuário, e seu
direito con­jugal não lhe diminuirá" (Exodo 21:10); cada uma dessas
proibições é idêntica às proibições mencionadas: "Não comais dela mal
passada no fogo nem cozida na água" e "Não fareis queimar fermento algum
ou mel". Não há diferença en­tre elas.

Do mesmo modo, Suas palavras "Não trarás salário de rameira nem preço de
um cão" (Deuteronômio 23:19) constituem um preceito negativo. E esse
também é o caso de Suas palavras "Vinho e bebida forte não bebereis...
quando entrardes à tenda da revelação... e para ensinar" (Levítico
10:9-11). Quer dizer, num único preceito Ele advertiu para que não se
entre no Santuário nem se dê instruções sobre a Torah em estado de
embriaguez. Esta é uma das duas partes do segundo tipo de proibição
global.

A segunda parte trata precisamente do mesmo tipo de expressão que a
primeira, exceto que aqui a instrução da Tradição é de que cada assunto
liga­do e acrescido acarreta um açoitamento em separado e que caso eles
sejam to­dos transgredidos, mesmo que seja todos de uma só vez, fica-se
sujeito ao açoi­tamento por cada um deles. É em casos como estes que
devemos contar cada conceito como uma proibição separada.
Esse é o caso de Suas palavras "Não te será permitido comer em tuas
cidades o dízimo de teus cereais, e de teu mosto, e de teu azeite"
(Deuteronô­mio 12:17), com relação às quais os Sábios dizem na Guemará
Queretot: "Se al­guém comer do dízimo dos cereais, do mosto e do azeite
ele será culpado por cada um deles separadamente". A esse respeito eles
perguntaram: "Mas é-se açoi­tado por uma proibição global?" E a resposta
foi: "O texto é repetitivo. Veja: a Torah já disse: 'Comerás diante do
Eterno, teu Deus,... o dízimo de teu grão, teu mosto e teu
azeite'(Ibid., 14:23). Por que ela determina 'Não te será permiti­do
comer em tuas cidades'? E se você disser que é para estabelecer uma
proibi­ção, então que a Torah diga: 'Não te será permitido comê-los'. P
• • ue ela enun­cia outra vez todos detalhadamente? Só pode ser para
estabele -los e separado.

Está explicado ali que se é culpado por cada uma eparad ente tam­bém no
caso de Suas palavras, enaltecido seja Ele, "E pão, e arinha fei rs de
grãos de espigas verdes, torrada no forno, e grãos verdes de cer
is\textsuperscript{20} não omereis" (Levítico 23:14). Os Sábios dizem:
"Aquele que come pã•rin de grãos de espiga verdes e grãos verdes de
cereais é culpado por cad um deles separada­mente. Mas fica-se sujeito
ao açoitamento por uma proibi ão global? O texto é repetitivo. Que o
Misericordioso escreva uma e as outras serão deduzidas de­la". Depois de
uma discussão a respeito foi explicado que não havia necessida­de de que
Ele mencionasse a ``farinha de grãos de espigas verdes'', e que isso foi
mencionado para estabelecer uma separação: para sujeitar-nos ao
açoitamento no caso dessa farinha separadamente. O Talmud continua o
debate perguntan­do: Talvez se fique sujeito ao açoitamento por causa da
farinha de grãos, uma vez que ela foi mencionada com esse objetivo, mas
será que se fica sujeito a um único açoitamento por comer pão e farinha
de grãos verdes? A resposta foi: "Por que motivo o Misericordioso
escreveu 'farinha de grãos verdes' no meio? Para ensinar-nos que o pão é
como a farinha de grãos, e que esta é como os grãos verdes", de maneira
que se é culpado por cada um individualmente.

O mesmo direi com relação a Sua declaração, enaltecido seja Ele, "Nãó se
achará entre ti quem faça passar seu filho ou sua filha pelo fogo, nem
agoureiro, nem prognosticador, nem adivinho, nem feiticeiro, nem
encanta­dor, nem necromante ou Yideonita, nem quem consulte os mortos"
(Deutero­nômio 18:10-11); cada uma das nove coisas enumeradas são
contadas como um preceito individual e nenhuma delas pertence à primeira
parte do segundo ti­po. Prova disso é que Suas palavras, enaltecido seja
Ele, "Nem prognosticador, nem adivinho" estão no meio da frase, pois já
foi explicado que em Suas pala­vras "Não augurareis e não
prognosticareis' (Levítico 19:26) cada uma dessas proibições constitui
um preceito em si. Assim como o prognosticador e o adi­vinho ---
mencionados no meio --- são casos separados, também todos os ou­tros
casos mencionados antes e depois são semelhantes a eles, tal como
expli­cam os Sábios no caso do "pão, da farinha de grãos e dos grãos
verdes".

Outros se enganaram a respeito désta questão, seja porque suas men­tes
não compreenderam em absoluto estes assuntos, ou então porque eles se
esqueceram e se desviaram do rumo correto. Assim, eles contaram Suas
pala­vras, enaltecido seja Ele, com respeito aos ``Cohanim'', "Mulher
prostituta ou profana não tomarão, nem mulher divorciada de seu marido
não tomarão" (Ibid., 21:7) como um único preceito, embora já estivesse
explicado na Guemará de Kidushin que se fica sujeito a castigo por cada
uma dessas desqualificações, mes­mo que todas elas digam respeito a uma
única mulher, como explicaremos no local apropriado. De fato, poderíamos
encontrar uma desculpa por contar uma

20. Da nova colheita, até que se traga a oferenda do ``omer''.

prostituta e uma mulher profana como um preceito, porque tendo
compreen­dido alguns detalhes da proibição global ele considerou Suas
palavras, enalteci­do seja Ele, "Mulher prostituta ou profana não
tomarão" semelhante a "Não comais dela mal passada no fogo nem cozida na
água" e não percebeu que a primeira proibição estabelece uma separação e
a segunda não. Também não di­ferenciou o versículo "E pão, e farinha
feita de grãos de espigas verdes, torra­das no forno, e grãos verdes de
cereais não comereis" de "Sua manutenção, seu vestuário e seu direito
conjugal não lhe diminuirá". Contudo, não vou cri­ticá-lo em casos como
estes. Mas não há desculpa por ter contado uma mulher divorciada junto
com uma prostituta e uma profana, incluindo-as todas num só preceito,
pois a mulher divorciada constitui claramente uma proibição sepa­rada,
como Ele disse, enaltecido seja Ele: "Nem mulher divorciada de seu
mari­do não tomarão".

Portanto, deixamos claro este grande Fundamento --- isto é, a proi­bição
global --- e explicamos as dúvidas relativas a ele. Também esclarecemos
em que casos ele estabelece uma separação e em que casos há apenas uma
proi­bição global, sujeitando-nos ao castigo apenas uma vez.
Esclarecemos ainda que quando há uma separação deve-se contar todos como
preceitos separados e que quando não há separação deve-se contar um só
preceito. Tenha sempre todo este Fundamento diante de si pois ele é um
guia da maior importância para a enumeração correta dos princípios.

\chapter*{O décimo fundamento\subtitulo{Não se deve contar os atos estipulados como preliminares ao cumprimento do preceito}}

Ocasionalmente a Torah menciona ordens que não constituem pre­ceitos em
si, mas apenas preliminares para o cumprimento de um preceito, co­mo se
Ele estivesse descrevendo a maneira como o preceito deve ser executa­do.
Um exemplo disso é o versículo ``E tomarás a flor da farinha de trigo''
(Le­vítico 24:5). Não seria correto contar o fato de tomar a farinha
como um pre­ceito, e o cozimento dela como outro; o que deve ser contado
é apenas o que Ele diz: "E porás sobre a mesa o pão da proposição diante
de Mim, continua­mente" (Êxodo 25:30), pois o preceito consiste apenas
em que haja sempre pão • diante do Eterno. Depois Ele descreve como deve
ser esse pão e a partir de que ele deve ser feito, dizendo que ele deve
ser feito com flor de farinha e que consiste de doze pães.

Do mesmo modo não devemos contar Suas palavras "Que te tragam azeite de
oliveira puro" (Ibid., 27:20), e sim apenas "para acender a lamparina
contínua" (Ibid.), que é o preceito de manter as lamparinas acesas, como
foi explicado no Tamid.

Assim também não devemos contar Suas palavras "Toma para ti
es­peciarias" (Êxodo 30:34), mas sim a queima diária de incenso, como
dizem as Escrituras: "Pela manhã, quando limpar as lamparinas, o
queimará. E ao acen-
der Aarão os fogos..." (Ibid.,78). É este versículo que constitui o
preceito a ser contado, ao passo que Suas palavras "Toma para ti
especiarias" são apenas uma preparação, explicando como o preceito deve
ser realizado e de que especia­rias deve ser feito esse incenso.

Da mesma forma não devemos contar Suas palavras "Toma para ti
especiarias principais" (Ibid., 23). O que deve ser contado é a ordem
que nos obriga a ungir os ``Cohanim Gadol'', os reis e os utensílios
sagrados com o Óleo de Unção descrito.

Você deve julgar todos os casos semelhantes a esses baseando-se neste
critério para que você não aumente a enumeração com tópicos que não
fazem parte dela. Este é nosso objetivo com este Fundamento, e isso está
perfeitamente claro. Porém nós o mencionamos e o comentamos porque
também com rela­ção a este assunto muitos se enganaram, contando um
preceito e seus atos pre­, liminares como dois preceitos, como ficará
claro para quem observar a enume­ração das seções feitas por Shimon
Kayará, bendita seja sua memória, e por seus seguidores.

\chapter*{O décimo primeiro fundamento
\subtitulo{Não se deve contar separadamente os diversos elementos 
que compõem um só preceito}}

Ocasionalmente um preceito pode consistir de muitas partes, como é o
caso do preceito do ramo de palma (``lulav'' e ``etrog''), que compreende
quatro tipos. Nesse caso não devemos dizer que "o fruto das árvores
formo­sas" e ``os galhos das árvores frondosas'', e "o salgueiro do
regato" e "os ra­mos das palmeiras" constituem cada um um preceito
separado; ao contrário, eles são todos partes de um preceito, pois Ele
ordenou juntá-los e o preceito consiste em segurá-los na mão, todos
juntos, num dia determinado.

Um caso exatamente igual: não devemos contar Suas palavras que dizem que
o leproso deve purificar-se com dois pássaros, um pau de cedro,
carmezin, hissopo, água corrente e uma vasilha de barro como sendo seis
pre­ceitos; o que constitui o preceito é a purificação do leproso
através de todos os elementos estabelecidos --- os já citados e mais a
raspagem --- sendo que os diferentes elementos nos são impostos para
fazer essa purificação, que é fei­ta de tal e tal maneira.

A mesma lei se aplica com relação aos sinais de reconhecimento que nos
foi ordenado fazer no leproso quando ele estiver em estado de impureza,
a fim de que nos mantenhamos afastados dele, como está dito: "Suas
vestes serão rasgadas e seu cabelo não será cortado, e com seu bigode se
cobrirá; e impuro! impuro! clamará" (Levítico 13:43). Cada um desses
atos não constitui um preceito em si; ao invés disso, é o preceito que
consiste no conjunto deles, isto é, que devemos 
fazer com que o leproso possa ser identificado para
que possamos nos manter afastados dele e que essa identificação se
compõe disto e daquilo, tal como nos foi ordenado alegrar-nos diante do
Eterno no primeiro dia dos Tabernáculos, sendo que Ele explicou que essa
alegria consiste em le­var determinados objetos.

Há um aspecto difícil de ser compreendido neste Fundamento e a razão
disso é o que explicarei a seguir: Toda vez que os Sábios disserem, com
relação a um determinado preceito, que um de seus elementos prejudica a
vali­dade de outro, é óbvio que ele constitui um preceito, como por
exemplo as quatro variedades usadas no caso do ramo de palma e o incenso
levado junto com o pão da proposição sobre o qual os Sábios disseram:
"As fileiras e os pra­tos comprometem um a validade do outro"; nesses
casos fica claro que eles constituem um único preceito. Da mesma forma,
toda vez que ficar claro que o objetivo desejado não será obtido através
de um dos elementos, também é óbvio que é o conjunto deles que deve ser
contado. Tal é, por exemplo, o caso explicado sobre o reconhecimento do
leproso, pois se apenas suas roupas fo­rem rasgadas mas se ele não tiver
deixado crescer seu cabelo, não tiver coberto o lábio superior e não
gritar "impuro, impuro", ele não terá efetivado nada; ele só será
identificável quando fizer tudo. Assim também sua purificação não será
alcançada até que ele se utilize de todas as coisas mencionadas: os
pássa­ros, o pau de cedro, o hissopo, o carmezin, e a raspagem. Só assim
sua purifica­ção será alcançada.

Contudo, o aspecto difíc' rge quando os Sábios dizem, com rela­ção aos
elementos, que "eles nã comprometem a validade uns dos outro ". Num
primeiro raciocínio você oncluirá que como cada um desses vário:
ele­mentos não precisa do outro, ada um les deve ser contado como u
pre­ceito independente. Esse é o so, por emplo, de sua afirmação: " 1
zu1\textsuperscript{21} não prejudica a validade do br.
0\textsuperscript{22}, e o branco não compromete o az 1' . Isto pode
levá-lo a concluir que o r. co e azul devem ser contados como dois
preceitos, se não fosse pela afirm xplícita que encontramos na Mekh a de
Rabi Ishmael de que "Poderíamos pensar que estes são dois preceitos ---o
do azul e o do branco ---; por isso as Escrituras afirmam 'E será para
vós por ``tsitsit''' (Números 15:39), mostrando assim que se trata de um
preceito e não de dois".

Assim, foi explicado que mesmo que os elementos não prejudiquem a
validade uns dos outros, eles às vezes são um único preceito, desde que
o significado deles seja um só. Tal é o caso dos ``tsitsit'', onde o
objetivo é ``Para que vos lembreis'' (Números 15:40) e portanto o que deve
ser contado é o con­junto de coisas que vai fazer com que nos lembremos.

Assim sendo, quando formos enumerar os preceitos, resta-nos não prestar
atenção quanto a se cada elemento compromete a validade do outro ou não,
e sim fixarmo-nos em seu conceito para saber se trata-se de um ou de
vários, tal como explicamos no nono dos Fundamentos que estamos tentando
elucidar.


 
%  O cordão azul do ``tsitsit''.
 
% \item
 
%  O cordão branco do ``tsitsit''.
 



\chapter*{O décimo segundo fundamento\subtitulo{Não se deve contar 
separadamente as etapas sucessivas na execução de um processo}}

Às vezes somos ordenados a executar uma determinada ação, e lo­go em
seguida a Torah começa a explicar como essa ação deve ser executada,
elucidando a expressão que usou e definindo o que está incluído nela. Em
casos assim não devemos contar cada ordem contida na explicação como um
preceito individual. Por exemplo: Suas palavras "E me farão um
santuário" (Êxodo 25:8) constituem um dos preceitos positivos, que é o
de que devemos construir uma casa para a qual devemos nos dirigir, para
onde devemos ir a fim de oferecer os sacrifícios e onde as assembléias
terão lugar, durante os festivais. Depois disso Ele começa a descrever
seus detalhes e como eles devem ser executados; esses atos específicos
--- cada um deles precedido pela expressão ``E me farão'' --- não devem
ser contados como preceitos separados.

O mesmo acóntece com relação aos sacrifícios mencionados no "Va­yikrá",
onde um preceito consiste de todo o ritual descrito em cada um dos
vários tipos de sacrifícios. Um exemplo disso é o holocausto, cujo
ritual que nos foi ordenado é o seguinte: que ele seja degolado, que sua
pele seja retirada, que seja cortado em pedaços, que seu sangue seja
derramado de tal e tal manei­ra, é que sua gordura seja queimada,
seguida da queima de toda sua carne, jun­tamente com uma determinada
medida de flor de farinha misturada com óleo e com uma certa quantia de
vinho --- que são as libações --- e que o couro seja dado ao ``Cohen'' que
estiver oficiando. A totalidade deste ritual constitui um preceito
positivo, que é a lei do holocausto, sendo que a Torah nos obriga a
executar cada holocausto dessa maneira.

Um caso semelhante é o do ritual completo do Sacrifício de Pecado:
degolamento, a retirada do couro, a queima do que deve ser queimado, a
lavagem das vasilhas nas quais deve ser derramado parte do sangue, e a
lava­gem ou a quebra das vasilhas nas quais foi cozida a carne. Tudo
isso é a lei do Sacrifício de Pecado e constitui um único preceito.

Da mesma forma, a lei do Sacrifício de Delito constitui um só pre­ceito,
assim como a ``lei do Sacrifício de Oferta de Paz'', oferecido como
sacri­fício de graças, com ou sem pão, quando o ``Cohen'' pega o peito e a
coxa e os levanta, sendo que tudo isso é o ritual do "Sacrifício de
Oferta de Pazes" constitui um só preceito.


Esses compõem a totalidade dos sacrifícios cuja obrigação cabe ao
indivíduo e à congregação, com a exceção do Sacrifício de Delito que é
sempre uma oferta individual, como explicamos na introdução à Ordem de
Kadashim.
O que constitui o preceito positivo é o procedimento nos vários rituais
e não se deve contar cada detalhe desses rituais como um preceito
separado, a não ser que eles sejam ordens que abranjam todos os diversos tipos de
sacrifícios e que não sejam específicos para apenas um deles, excluindo
os outros; tais or­dens devem ser contadas como preceitos em separado,
já que elas não são me­ros detalhes do ritual de algum sacrifício
específico. Esse é o caso, por exem­plo, de Sua advertência para não
trazer um sacrifício defeituoso, ou de sua or­dem para que seja
perfeito, ou de que não ofereçamos um animal que não te­nha atingido a
idade de ser aceito, conforme consta em Suas palavras "Do oita­vo dia em
diante..." (Levítico 22:27); ou de que ofereçamos sal com todos os
sacrifícios, como Ele disse: "Toda tua oferta de oblação temperarás com
sal" (Ibid., 2:13); ou de que não deixemos faltar o sal num sacrifício,
como Ele dis­se: ``Não deixarás faltar o sal'' (Ibid.); ou de que sejam
comidas as partes que devem ser comidas. Cada um desses casos constitui
um preceito independen­te, pois eles não são meros detalhes no ritual de
um sacrifício específico; ao contrário, eles são ordens que abrangem
todos os sacrifícios, como explicare­mos em nossa enumeração.

Está claro que o que o ``Cohen'' toma como sua parte do sacrifício é
apenas um dos detalhes do preceito, tal como mencionamos com relação ao
couro da oferta de Holocausto. Esse também é o caso da primeira tosquia,
on­de a essência do preceito consiste em separar a primeira lã do
carneiro e dá-la ao ``Cohen'', assim como no caso do primeiro dízimo, que
devemos separar e dar ao Levita.

Outros se enganaram a este respeito, e contaram os vinte e quatro tipos
de presentes ao ``Cohen'' como vinte e quatro preceitos, depois de ter
contado alguns preceitos nos quais alguns desses presentes eram meros
deta­lhes, tal como explicamos com relação ao couro da oferenda de
Holocausto e ao peito e coxa do Sacrifício de Paz.

Além disso, devido ao fato de que eles não conheciam este Funda­mento,
não se aperceberam dele, nem lhe prestaram atenção, resultando que eles
contaram, como preceitos separados, verter (óleo nos utensílios),
embe­ber (com óleo), cortar em pedaços, salgar, levar junto ao altar,
levantar, retirar um punhado e queimar, sem saber que todos esses atos
são detalhes do ritual da oblação. Ou seja, depois de ter-nos ordenado
oferecer a oblação de trigo, Ele começou a explicar a que se refere esse
nome --- o da ``lei da oblação de trigo'' --- e disse que se tratava de
flor de farinha, ou pão assado de uma certa forma --- sobre uma chapa,
numa panela a vapor ou no forno ---; depois de­ve-se embebê-la numa
determinada quantia de óleo, separá-la em pedaços e por nela sal e
incenso e deve-se levá-la junto ao altar e erguê-la, tomar um punhado
dela e queimá-la, de acordo com o procedimento que explicamos e
elucidamos no lugar apropriado: o Tratado Menahot. Todos esses são
detalhes do ritual que, quando executado de acordo com todo este
procedimento, é chamado de obla­ção. Assim sendo, o preceito é o
seguinte: somos obrigados a que o ritual do sacrifício do pão \%I da
flor de farinha esteja de acordo com o procedimento assim definido. No
caso do preceito da oblação a oferenda consiste do seguin­te: verter,
embeber, separar em pedaços, salgar, levantar, levar junto ao altar,
pegar um punhado e queimá-la, tal como Ele disse no caso do preceito
único da ``Halitzá'': "E lhe descalçará o sapato do pé, e cuspirá no chão,
diante dele, e responderá dizendo..." (Deuteronômio 25:9), onde não
contamos o fato de tirar o sapato, o de cuspir e o de proferir as
palavras como preceitos indepen­dentes, uma vez que eles estão incluídos
no ritual de ``Halitzá'', que constitui um preceito. Da mesma forma que
nesse caso, também não devemos contar separadamente as seguintes ordens:
``E deitarás sobre ela azeite'' (Levítico 2:6), "E porás sobre ela
incenso" (Ibid., 15), ``Temperarás com sal'' (Ibid., 13), "E
movimentará o sacerdote" (Ibid., 23:20), e a aproximará, ``E tirará'' dali
um pu­nho cheio... e o fará queimar" (Ibid., 2:2).

Isto só passará desapercebido a quem compreender os assuntos
su­perficialmente, sem examiná-los e avaliá-los em sua mente, tal como
dizem os Sábios de abençoada memória: "Ele disse isso por ser
precipitado". Quer di­zer, ele disse isso sem refletir a respeito,
baseado apenas no primeiro pensa­mento que lhe veio à mente.

Assim, este Fundamento nos explicou as leis de todos os sacrifícios e a
maneira como elas devem ser contadas para que não haja nenhum engano nem
confusão, como as explicaremos em nossa enumeração, com a ajuda do Todo
Poderoso.

\chapter*{O décimo terceiro fundamento\subtitulo{Quando um determinado preceito tiver que ser cumprido por vários dias não se deve contar um preceito por cada dia}}

rios durante o transcor­se período é contínuo, eito se sucede dia após
utras vezes ele corres-o de exemplo, se devês­itui um preceito, isto
sig­nificaria que fomos ordenados a levar um sacrifício adicional toda
vez que hou­ver lua nova. Se alguém perguntasse por que não contamos o
sacrifício adicional de cada lua nova como um preceito em si, diríamos
que se assim fosse, você tam­bém deveria contar o Holocausto de cada dia
como um preceito em si, assim como contar queimar o incenso e manter as
lamparinas acesas, os quais são obri­gatórios todos os dias do ano,
também como preceitos individuais. Mas como contamos apenas o conceito
do que nos foi ordenado, sem levar em considera­ção o fator tempo com
relação a sua execução, nós contamos o sacrifício adicio­nal da lua nova
como um único preceito, assim como o sacrifício adicional do Shabat e de
cada um dos cinco festivais, mesmo que eles sejam obrigatórios por
vários dias seguidos. Pois assim como Ele diz: "E vos alegreis diante do
Eterno, vosso Deus, por sete dias" (Levítico 23:40), Ele diz também
"Sete dias oferece­reis ofertas queimadas ao Eterno" (Ibid., 36); assim
como o preceito do ramo de palma é um só, também é um só o preceito do
sacrifício adicional de Páscoa. A mesma regra se aplica aos sacrifícios
adicionais de cada uma das estações.
 
%  Sentar na ``sucá'' durante sete dias.
 
% \item
 
%  O preceito do ``lulav''.
 

Com base neste Fundamento também ficará claro que o sacrifício de festa
é um único preceito, embora ele seja obrigatório nas três estações,
co­mo também o é o de comparecer e o de alegrar-se. Ninguém deve
enganar-se nem pensar de maneira diferente.

Contudo, alguns cometeram um erro extremamente sério e estranho com
relação a este Fundamento: eles contaram todos os sacrifícios adicionais
--- o do Shabat, o das luas novas e o das festas --- como um único
preceito! Se assim fosse, eles deveriam ter contado o descanso em todos
os festivais co­mo um preceito, mas não o fizeram. Mas o Eterno sabe e é
testemunha de que eles não devem ser criticados por isso, pois, de uma
maneira geral, eles não seguiram uma teoria ao fazer suas enumerações;
ao contrário, "Eles subiram até o céu, eles desceram às profundezas"
(Ps. 107:26). A verdade é o que eu lhes mencionei: que cada sacrifício
adicional constitui um preceito indepen­dente, assim como o descanso em
cada um dos festivais constitui um preceito diferente. Esta é a teoria
correta.

\chapter*{O décimo quarto fundamento\subtitulo{De que forma os tipos de castigo devem ser contados
como preceitos positivos}}

Você deve saber que todos os preceitos, positivos e negativos, estão
primeiramente divididos em duas partes, de acordo com o propósito deste
Fun­damento. A primeira parte é aquela em que a Escritura não
especificou castigo algum, mas apenas estabeleceu uma ordem ou uma
proibição, e não sujeitou o transgressor a qualquer castigo nem lhe
designou um castigo determinado por transgredir aquela ordem ou
proibição específica. A segunda parte é a que estabelece a recompensa e
o castigo.

Entre os preceitos nos quais Ele explica o castigo estão os preceitos
que nos ordenam a apedrejar os transgressores de determinados preceitos,
a queimá-los, a executá-los com a espada como foi indicado na explicação
da Tra­dição, a estrangulá-los, e a açoitá-los com uma correia. A
determinados trans­gressores Ele impôs a extinção, isto é, o
transgressor que morrer em estado de pecado não terá um lugar no Mundo
que Há de Vir, conforme explicamos no capítulo ``Helek'' ; a outros Ele
impôs apenas a morte, isto é, Ele fará com que morram por seu pecado e
sua morte lhes trará a absolvição.

Os Sábios explicam no início de Macot que no caso de uma proibi­ção cujo
castigo é a extinção ou apenas a morte pela Mão dos Céus --- se se
concluir que o transgressor pecou premeditadamente diante de testemunhas
e apesar das advertências --- o transgressor está sujeito ao
açoitamento, mesmo que seu castigo principal consista em que seu
julgamento caberá ao Céu. Há também preceitos nos quais Ele nos ordenou
castigar os transgressores de cer­tos preceitos apenas com seu dinheiro,
não com seu corpo, tal como determi-
nou a um assaltante que ele acrescentasse um quinto adicional e a uM
ladrão que pagasse o dobro do que roubou. E também há preceitos em que
Ele nos ordenou que os violadores levem um sacrifício por seu pecado
para serem as­sim perdoados.

As aplicações de todas essas formas de castigo constituem preceitos
positivos, pois fomos ordenados a matar um, a açoitar outro, a apedrejar
aque­le outro, ou a levar um sacrifício pelo que fizemos. Quanto a
inclui-los na enu­meração, devemos contar as quatro penas de morte
impostas pelo Tribunal co­mo quatro preceitos positivos. Tal é, na
realidade, a expressão da Mishná: "Es­te é o preceito dos que devem ser
apedrejados". Eles dizem também: "De que maneira deve ser o preceito de
queimar?", "De que maneira deve ser o precei­to de estrangular?", "De
que maneira deve ser o preceito de decapitar?".

Os Sábios também dizem que Suas palavras, enaltecido seja Ele, "Não
acendereis fogo" ( Êxodo 35:3), são uma advertência para que não se
aplique castigos no Shabat. Ou seja, este versículo nos proíbe de
executar um preceito que nos ordene queimar alguém, pois a expressão "Em
todas as vossas habita­ções" (Ibid.) significa que não se deve acender
fogo no Tribunal, mesmo que isso seja um preceito positivo. Portanto os
Sábios dizem: "Acender um fogo, que está incluído nas categorias de
tarefas proibidas no Shabat, está destacado para ensinar-nos que assim
como as leis do Shabat não podem ser desconside­radas no que se refere
ao tipo de execução especificamente mencionado, elas também não podem
ser desconsideradas no caso de outros tipos de execução judicial". Isto
está claro e ninguém terá dúvidas a respeito. Da mesma forma, devemos
contar o açoitamento com uma correia como um preceito individual.

Entretanto, não se deve fazer o que fizeram outros sem meditar, ou seja,
contar cada castigo em particular como um preceito separado dizendo, a
título de exemplo, que a ordem de apedrejar aquele que profanar o Shabat
é um preceito, que o apedrejamento daquele que pratica o ``Ob'' é um
segun­do preceito, e que o apedrejamento daquele que adora ídolos é um
terceiro preceito, resultando que o número de preceitos corresponderá ao
número de pessoas sujeitas às quatro penas de morte impostas pelo
Tribunal. Se assim fos­se, nós fatalmente teríamos que contar cada
açoitamento em separado, fazendo com que o açoitamento de quem come
``nebelá'' seja um preceito individual, o de quem come carne de porco um
segundo preceito, o de quem come carne cozida no leite um terceiro
preceito, o de quem usa ``shaatnez'' um quarto pre­ceito, resultando assim
que teríamos tantos preceitos positivos quanto o núme­ro de preceitos
negativos que acarretam o açoitamento. Dessa maneira (inú­mero de
preceitos positivos aumentaria e chegaria com certeza a mais de
qua­trocentos! Ao invés disso, assim como não contamos separadamente
todos os que estão sujeitos ao açoitamento, e sim apenas o tipo de
castigo --- a saber, o açoitamento com uma correia ---, devemos contar
também nas penas de morte apenas as formas de execução, que são pelo
fogo, por apedrejamento, por es­trangulamento e por decapitação. Da
mesma forma, não devemos contar sepa­radamente todos os que estão
sujeitos a oferecer um sacrifício, dizendo que o sacrifício de pecado
daquele que viola o Shabat sem querer é preceito, e que o Sacrifício de
Pecado de quem adora ídolos sem querer é preceito; em vez disso devemos
contar apenas o tipo de sacrifício, tal como fizemos nos tipos de pena
de morte.

Você já sabe que o tipo de sacrifício que sé é obrigado a oferecer varia
de acordo com o tipo de pecado que se cometeu. Há pecados pelos quais se
oferece o Sacrifício de Pecado, ou o Sacrifício Suspensivo de Delito, ou
o Sacrifício Incondicional de Delito, ou o Sacrifício de Maior ou Menor
Valor.

É por essa razão que não devemos contar o Sacrifício de Pecado
juntamente com o Sacrifício de Delito; ao invés disso, contaremos a
obrigação do Sacrifí­cio de Pecado, a do Sacrifício Suspensivo de
Delito, a do Sacrifício Incondicio­nal de Delito e a do Sacrifício de
Maior ou Menor Valor como preceitos separa­dos, sendo que as obrigações
são de responsabilidade da pessoa que deve ofe­recer aquele tipo
específico de sacrifício. Não voltaremos nossa atenção para os vários
pecados pelos quais se é obrigado a oferecer um sacrifício específico,
da mesma forma que contamos o açoitamento como um só preceito e
descon­sideramos os vários pecados que acarretam esse castigo. Da mesma
forma, a Escritura dedicou um seção especial a cada tipo.

Outros já fizeram tanta confusão com relação a este Fundamento que se
torna desnecessário refutá-los, nem seria fácil fazê-lo, tamanha é a
desordem que eles implantaram a este respeito.

De fato, devemos espantar-nos e surpreender-nos com uma pessoa que conta
um a um, entre os preceitos negativos, todos aqueles que estão sujei­tos
a alguma das penas de morte aplicadas pelo Tribunal, bem como os que
estão sujeitos à extinção e à morte, além de contar também os atos
proibidos cuja violação implica numa daquelas formas de morte! Foi isso
o que fez o au­tor do Halachot Guedolot. Ele contou "quem profanar o
Shabat" entre aqueles que estão sujeitos à morte por apedrejamento, e
depois contou "Não farás ne­nhuma obra" (Êxodo 20:10). Devemos de fato
concluir que eles pensaram ini­cialmente que a execução judicial
constitui um preceito negativo em si. Mas se assim fosse, como poderiam
eles conter o castigo e a proibição específica pela qual se aplica o
castigo?

Ainda mais surpreendente é o fato de que eles contaram entre os
pre­ceitos negativos os que estão sujeitos à extinção, bem como os que
estão sujei­tos à morte pela Mão dos Céus, os quais não envolvem
execução! Deve ser por­que eles imaginam que ficar sujeito à extinção e
que a aplicação do castigo cons­tituem a natureza desse preceito
específico. De fato, foi assim que o autor do ``Sefer Hamitzvot'' explicou
isso ao resumir o conteúdo do primeiro capítulo nas seguintes palavras:
"Entre estes há trinta e dois conceitos sobre os quais Ele nos informa
que Ele --- abençoado e enaltecido seja ---, e não nós, o fará
seguramente cumprir." ``Entre estes'' significa entre os conceitos
mencionados naquele capítulo. Os ``trinta e dois conceitos'' compreendem,
de acordo com sua enumeração, vinte e três casos sujeitos apenas à
extinção e nove sujeitos à morte pela Mão dos Céus, e procede a sua
enumeração. Com a palavra "segu­ramente"
ele quer dizer que o Eterno, enaltecido seja Ele, assegurou que Ele
aplicará a extinção a este e a morte ao outro. Não há dúvidas de que
esse ho­mem se separou completamente da idéia de que \emph{todos} os
seiscentos e treze pre­ceitos são obrigação \emph{nossa;} em vez disso,
alguns seriam nossa obrigação e ou­tros obrigação d'Ele, enaltecido seja
Ele, tal como ele afirmou claramente: "Ele os fará cumprir, e não nós".
Deus sabe e é testemunha de que na minha opi­nião tudo isto é uma
confusão absoluta e não há necessidade alguma de falar a respeito pois a
invalidez de suas palavras é óbvia. O motivo deste erro é que ao contar
os castigos como preceitos eles se confundiram ,e algumas vezes
con­taram os castigos e também as ações que os acarretam, estabelecendo
tudo co­mo preceito negativo, sem refletir a respeito.

Contudo, a forma correta de enumeração é como eu mencionei: ca­da
\emph{tipo} de castigo constitui um preceito positivo. De acordo com
isso, a lei de restituição referente a um ladrão é um preceito positivo,
a saber, que somos ordenados a impor-lhe uma determinada quantia. Assim
também são as seguin­tes leis: o quinto adicional, a obrigação do
Sacrifício de Pecado, o Sacrifício
Incondicional de Delito, o Sacrifício Suspensivo de Delito e o
Sacrifício de Maior ou Menor Valor. Da mesma forma, cada um dos
seguintes castigos: apedreja­mento, queima, decapitação, estrangulamento
e enforcamento constitui um pre­ceito individual que se aplica a todo
aquele que ficar sujeito a eles, tal como o açoitamento com a correia
constitui um só preceito que se aplica a todo aquele que ficar sujeito a
esse castigo. Isso é o que desejávamos expor neste Funda­mento, e com
ele completamos os Fundamentos, cuja introdução ajudará na­quilo em que
nos empenhamos.

É importante acrescentar a seguinte introdução. Todo pecado, pelo qual a
penalidade é a execução judicial ou a extinção, é necessariamente um
preceito negativo, a não ser o sacrifício de Páscoa e a circuncisão, que
acarre­tam a extinção mesmo sendo preceitos positivos, como explicamos
no início do Tratado Queretot. Não temos nenhum outro preceito positivo
a não ser es­ses por cuja transgressão se fique sujeito à extinção --- e
mais ainda, à execução judicial. Sendo assim, toda vez que a Torah
disser que aquele que cometer um determinado ato deverá ser morto ou
estará sujeito à extinção saberemos per­feitamente que esse ato
específico está proibido e constitui um preceito negativo.

As vezes as Escrituras apresentam uma proibição sem expor o casti­go,
embora tanto o castigo como a advertência estejam claros. Esse, por
exem­plo, é o caso da profanação do Shabat e da adoração de ídolos, com
relação aos quais Ele disse: ``Não farás nenhuma obra'' (Êxodo 20:10) e
"Nem os servi­rás" (Ibid., 5) e depois disso Ele declara que aquele que
trabalha e aquele que serve estão sujeitos ao apedrejamento.

Algumas vezes as Escrituras não mencionam claramente a proibição de um
determinado ato mas declaram apenas a punição, omitindo a advertên­cia.
Contudo, como é regra para nós que "não há nenhum castigo estipulado na
Torah sem que uma advertência o tenha precedido" deve haver de alguma
forma uma advertêcia com relação ao ato que nos sujeita ao castigo .
Isto é o que os Sábios sempre dizem: "Ouvimos o castigo, mas não ouvimos
a proibi­ção. Por isso a Torah diz isto e aquilo". E se a advertência
não estiver explicita­mente enunciada nas Escrituras, eles a deduzem
através de um dos Princípios. Esse é, por éxemplo, o caso do que os
Sábios dizem com relação às proibições de amaldiçoar ou de bater no pai,
as quais não estão de modo algum explicita­mente mencionadas nas
Escrituras, pois em lugar algum está dito "Não amaldi­çoe seu pai" nem
``Não bata em seu pai''; em vez disso Ele declarou que aquele que bater ou
amaldiçoar está .sujeito à morte. Á partir dessas palavras deduzi­mos
que estes atos são proibidos e que os Sábios extraíram delas, através de
um dos princípios exegéticos, as advertências referentes a esses atos,
assim co­mo fizeram em casos semelhantes, em outros lugares.

Este método de dedução de uma advertência não contradiz de for­ma alguma
o que os Sábios frequentemente dizem: "Uma lei derivada por ana­logia
não é considerada uma lei específica pela qual se possa ser castigado",
nem o princípio de que "Acaso se pode castigar pela violação de uma lei
cuja advertência se deduziu por analogia?" O objetivo da declaração "Uma
lei deri­vada por analogia não é considerada uma lei específicg pela
qual se possa ser castigado" é apenas de proibir-nos de derivar por
analogia um assunto com re­lação ao qual não haja proibição
\emph{alguma} mencionada; mas se encontrarmos a punição contra fazer um
determinado ato claramentç exposta na Torah, sabe­remos com certeza que
esse ato está proibido e que fomos advertidos para não fazê-lo. E apenas
para estar de conformidade com a regra de que "Nenhum cas­tigo está
estabelecido na Torah a menos que uma advertência proibitiva o te­nha
precedido" que fazemos deduções a partir de um dos princípios, quando


Ele tiver se referido a essa advertência. E uma vez encontrada a
advertência contra
esse ato, o violador que o fizer fica sujeito à extinção ou à
execução judicial.


Assim, fique ciente desta introdução e recorde-se dela e de todos os
Fundamentos precedentes juntamente com tudo o que tencionamos mencio­nar
a seguir.

Agora começarei a mencionar todos os preceitos, um a um, expli­cando-os
apenas a fim de elucidar o título do preceito, como prometemos no início
de nosso estudo, pois esse é o objetivo deste trabalho. Entretanto,
creio que é aconselhável acrescentar o seguinte ao nosso objetivo.
Quando me refe­rir a um preceito, positivo ou negativo, que acarreta
algum castigo, menciona­rei o castigo dizendo "Aquele que o violar está
sujeito à morte, ou à extinção, ou a oferecer determinado sacrifício, ou
ao açoitamento, ou a uma das penas de morte impostas pelo Tribunal, ou a
pagamento." E todas as vezes que ne­nhum castigo for mencionado você
deverá saber que se for com relação a um dos preceitos negativos a regra
a ser aplicada é, como dizem os Sábios: "Como um homem que viola o
preceito do Rei" e não cabe a nós puní-lo. Mas com relação a todos os
preceitos positivos, quando sua execu - • ainda for aplicável, 
devemos açoitar com uma correia aquele que se ri' fazê-lo até que
ele morra ou cumpra, ou até que passe o momento d ão, pois aquele
que violar o preceito positivo de viver num Taberná•deve ser açoi­tado
por seu pecado depois dos Tabernáculos. Saiba

Além disso, quando eu mencionar os preceitos, positivos ou negati­vos,
que não são obrigatórios para as mulheres, direi: "E este não é
obrigatório para as mulheres". É sabido que as mulheres não estão
qualificadas para julgar nem testemunhar, nem oferecer elas próprias os
sacrifícios, nem tomarem par­te numa guerra opcional. Consequentemente,
não será necessário que eu diga: "E este não é obrigatório para as
mulheres" com respeito a todos os preceitos relativos ao Tribunal, a
testemunhos ou ao Serviço, pois isto seria apenas re­dundante e
desnecessário. •

Também quando eu mencinar os preceito, positivos ou negativos, que são
obrigatórios apenas na terra de Israel ou enquanto existir o Templo,
direi: "E este é obrigatório apenas na terra de Israel, ou enquanto o
Templo existir."

É sabido que todos os sacrifícios eram levados apenas ao Templo, e que
tal ritual está proibido fora do Campo. Da mesma forma, as leis de
puni­ção capital são impostas apenas durante a existência do Templo. A
Mekhiltá diz: "De que forma sabemos que a execução judicial só pode ter
lugar enquanto o Templo existir? Pelo que dizem as Escrituras: "Do meu
altar o tirarás, para que morra" (Êxodo 21:14). Também é sabido que a
profecia e o Reino ficarão desaparecidos de nosso meio até o momento em
que desistamos dos pecados a que nos habittiamos, quando então Deus nos
perdoará e terá misericórdia de nós, de acordo com o que Ele nos
prometeu, e nos restituirá, como disse com relação ao retorno da
profecia: "E virá depois, verterei Meu espírito sobre toda carne, e
vossos filhos e filhas profetizarão" (Joel 3:1); e com relação ao
retorno do reino e do poder Ele disse: "Nesse dia levantarei o
Tabernáculo de Davi que caiu, fecharei suas brechas, levantarei suas
ruínas e o edificarei como nos ve­lhos dias" (Amos 9:11). E também é
sabido que a guerra e a conquista da terra só será feita com um rei, sob
o comando do Grande Sanhedrin e do ``Cohen Gadol'', tal como Ele disse: "E
diante de Elazar, o 'Cohen"' (Números 27:21).

25. Sentar na ``sucá'' durante sete dias.

Como todos estes assuntos são do conhecimento da maioria das pes­soas,
toda vez que houver um preceito positivo ou negativo relativo a
sacrifí­cios, rituais, a penas de morte impostas pelo Tribunal, pelo
Sanhedrin, ou pelo profeta e rei, ou à guerra obrigatória ou opcional,
não será necessário que eu diga que "Este preceito se aplica apenas
durante a existência do Templo", pois isto ficou claro, de acordo com o
que mencionamos. Contudo, nos casos em que possa surgir alguma dúvida ou
engano eu comentarei a respeito, se Deus quiser.

E agora começarei a mencionar todos os preceitos, com a ajuda do Todo
Poderoso.




\part{Os 248 preceitos positivos}

\section{Crer em Deus} %1

Por este preceito somos ordenados a crer em Deus, ou seja, a acredi­tar
que há um Agente Supremo que é o Criador de tudo o que existe. Ele está
expresso em Suas palavras, enaltecido seja Ele, "Eu sou o Eterno, teu
Deus, que te tirei da terra do Egito etc" (Êxodo 20:2).

No final do Tratado Macot está dito: "Seiscentos e treze preceitos
foram comunicados a Moisés no. Sinai, como diz o verso 'A Lei que orde-
nou \textsubscript{n t s}


Moiség'•.(Deuteronômio 33:4)"; ou seja, ele nos
ordenou obe o


preceitos quantos há na soma das letras-número TORAH. A isso que

as letras-números da palavra TORAH somam apenas seiscentos o res-

posta foi: "Os dois preceitos 'Eu sou o Eterno, teu Deus' e ão erás
utros\\
deuses diante de Mim' (Êxodo, 20:3) foram ouvidos do próprio Todo
.roso."

Portanto, foi deixado claro que o versículo "Eu sou o Eterno, teu Deus"
é um dos 613 preceitos, e é o que nos ordena a crer em Deus, como
explicamos.

\section{A unidade de Deus}

Por este preceito somos ordenados a crer na Unidade de Deus, ou seja, a
acreditar que o Criador de todas as coisas existentes e Primeiro Agente
delas é Uno. Este preceito está expresso em Suas palavras, enaltecido
seja Ele, "Escuta, Israel! O Eterno é nosso Deus, o Eterno é Uno!"
(Deuteronômio 6:4)

Em quase todo Midrashot você vai encontrar que essas palavras
sig­nificam que devemos declarar a Unidade do Nome de Deus, ou a Unidade
de Deus, ou algo nesse sentido. A intenção dos Sábios era ensinar que
Deus nos tirou do Egito e nos cumulou de bondade apenas com a condição
de que acre­ditemos em Sua Unidade, e isso é nosso dever.

O preceito de crer na Unidade de Deus está mencionado em muitos lugares,
e os Sábios também o chamam de preceito de crer no Reino dos Céus,
pois eles falam da obrigação "de tomar a nosso cargo a união com o Reino
dos Céus", ou seja, de declarar a Unidade de Deus e de crer n'Ele.

\section{Amar a Deus}

Por este preceito somos ordenados a amar ao Eterno, enaltecido se­ja
Ele, ou seja, a deter-nos e a meditar sobre Seus preceitos, Suas ordens,
e Seus trabalhos, de maneira a obter uma concepção d'Ele, e ao
concebê-Lo, alcançar o júbilo absoluto, e isto é o amor que nos foi
ordenado. Como diz o Sifrei: "Uma vez que dizemos 'E amarás ao Eterno,
teu Deus' (Deuteronômio 6:5), o estu­dioso perguntará: 'Como se deve
manifestar seu amor pelo Eterno?' As Escritu­ras dizem: 'E estarão estas
palavras que Eu te ordeno hoje, no teu coração' (Deu­teronômio 6:6),
porque é através disto que você aprenderá a conhecer Aquele cuja palavra
ordenou ao Universo sua existência".

Desta forma ficou claro que através deste ato de meditação você vai
alcançar a concepção de Deus e alcan o estado de júbilo no qual o amor a
Deus será uma conseqüência nece


Os Sábios dizem que este também inclui a obrigação de con-


vocar todos os descendentes de Ad ra serví-Lo, louvado seja Ele, e ter

fé n'Ele. Porque da mesma forma q e exalta e glorifica alguém a quem

você ama, e convoca os outros hom ara amá-lo, se você ama o Eterno até

a concepção de Sua verdadeira Natu eza, que você já alcançou através do
co­nhecimento, sem dúvida convocará os tolos e os ignorantes a procurar
o co- . nhecimento da Verdade que você já encontrou.

Como diz o Sifrei: " 'E amarás ao Eterno, teu Deus': isso significa que
você deverá fazer com que Ele seja amado pelos homens, como o fez o seu
pai Abraham, como foi dito: 'E as almas que haviam adquirido em Haran' "

(Gênesis 12:5). Ou seja, s esma forma que Abraham, sendo um amante do

Eterno --- como a Tor unha, quando designado pelo Eterno como sen-

do: 'Meu amado Abra pela força de sua concepção de Deus e pelo

seu grande amor por El cone o cou a humanidade a crer, assim você deve
amá­Lo de forma tal a atrai anidade para Ele.

\section{Temer a Deus}

Por este preceito somos ordenados a crer no temor a Deus, enalteci­do
seja Ele, de maneira a não ficar acomodados e auto-confiantes, e sim a
espe­rar sempre Seu castigo.. Este preceito está expresso em Suas
palavras "Ao Eter­no, teu Deus, temerás" (Deuteronômio 6:13, 10:20).

A Guemará do Tratado Sanhedrin comenta da seguinte forma o ver­sículo
"Aquele que insultar (nokeb) o nomé real do Eterno certamente será
mor­to" (Levítico 24:16): "Talvez a palavra `nokeb' devesse significar
'declarar', já que èncontramos em outro lugar 'Estes homens que foram
declarados (nikebu) por nomes' (Números 1:17), provindo a advertência do
versículo 'Ao Eterno, teu Deus, temerás' ". Ou seja, o versículo "Aquele
que insultar o nome real do. Eterno, etc" deve ser
entendido como significando aquele que simplesmente mencionar o Nome do
Eterno, sem louvá-Lo; e se você perguntar: "Que peca-

\begin{enumerate}
\def\labelenumi{\arabic{enumi}.}
\setcounter{enumi}{26}
\item
 
 Os filhos de Adam no contexto da humanidade no seu todo.
 
\item
 
 Isa. 41:8.
 
\end{enumerate}




do há nisso?" responderemos que quem o fizer estará abandonando o temor
ao Eterno, porque faz parte do temor ao Eterno não pronunciar Seu Nome
em vão.

Os Sábios respondem a esta pergunta, e contestam a perspectiva
en­volvida nela, como segue: "Primeiro, para que constitua um insulto, o
Nome deve ser utilizado, e neste caso essa condição está ausente"; ou
seja, ele deve ser culpado de insultar o Nome em nome d'Ele, tal como
eles dizem: 'Deixe Yossi castigar Yossi' ".

"Além do mais, a advertência que você cita está na forma de um pre­ceito
positivo, e é um princípio aceito que tal tipo de advertência não é
váli­da". Quer dizer, sua teoria de que a proibição do mero
pronunciamento do Nome de Deus pode provir do versículo "Ao Eterno, teu
Deus, temerás" é inad­missível porque esse versículo é um preceito
positivo e uma proibição não po­de ser baseada num preceito positivo.

Assim, foi-lhe deixado claro que as palavras "Ao Eterno, teu Deus,
temerás" estipulam um preceito positivo.


\section{Servir a Deus}

Por este preceito somos ordenados a servir a Deus, enaltecido seja Ele.
Este preceito está repetido várias vezes nas Escrituras, como foi dito
em: "E servireis ao Eterno, vosso Deus" (Êxodo 23:25); "e a E e
servireis" (Deute-

ronômio 13:5); ``e a Ele servirás'' (Ibid., 6:13); "e ser ' (Ibid.,
11:13).

Embora este preceito seja da categoria dos tos gerais que es-

tão excluídos dos 613 preceitos pelo Quarto Fundarne
o\textsuperscript{29}, inda assim ele im­põe uma obrigação específica,
que é a Oração. O Sifrei diz. rví-Lo (Ibid. 11:13) significa Oração". Os
Sábios também dizem: " `Serví-Lo' significa estudar a Lei".

Na Mishná de Rabi Eliezer, filho de Rabi Yossi Ha-Galili, está dito: "De
que maneira ficamos sabendo que a Oração é obrigatória? Através do
ver­sículo 'Ao Eterno, teu Deus, temerás, e a Ele s virás' (Ibid.
6:13)". Os Sábios também dizem: " `Serví-Lo' através da Sua T ra e
`serví-Lo' em Seu Santuá­rio", o que significa que devemos aspirar ezar
no Templo ou voltados em sua direção, como disse claramente Salom
o.°

\section{A aproximação de Deus}

Por este preceito somos ordenados a juntarmo-nos e a associarmo-nos com
os homens sábios, a estar sempre em sua companhia, a unirmo-nos a eles e
a seguir seus caminhos através de toda forma possível de
companheiris­mo: comendo, bebendo .e a negócios, com a finalidade de
conseguir ser como eles, quanto as suas ações, e de acreditar nos
conceitos verdadeiros através de suas palavras. Este preceito está
expresso em Suas palavras, enaltecido seja Ele, "E d'Ele te aproximarás"
(Deuteronômio 10:20) que estão repetidas no versí­culo "E
aproximando-vos d'Ele" (Ibid. 11:22). O Sifrei diz: " 'E aproximando-vos
d'Ele' significa que devemos aproximar-nos dos homens sábios e de seus
discípulos".


\begin{enumerate}
\def\labelenumi{\arabic{enumi}.}
\setcounter{enumi}{28}
\item
 
 Ver o Quarto Fundamento.
 
\item
 
 Reis 8:30.
 
\end{enumerate}

Os Sábios também usam as palavras "E d'Ele te aproximarás" como prova de
que é nosso dever casarmo-nos com a filha de um homem sábio, dar nossa
própria filha em casamento a um discípulo de um homem sábio, benefi­ciar
os homens sábios e manter negócios com eles. Como está dito: "Existe a
possibilidade para um ser humano de aproximar-se da Presença Divina,
visto que está escrito 'Porque o Eterno, teu Deus, é um fogo
consumidor'? (Ibid. 4:24). Dessa forma devemos concluir, de acordo com
este versículo, que o casamen­to com a filha de um sábio deve ser
considerado como um meio de aproximar-se do Eterno".


\section{Jurar em nome de Deus}


Por este preceito somos ordenados a jurar exclusivamente em Seu Nome,
enaltecido seja Ele, toda vez que nos for solicitado confirmar ou negar
alguma coisa sob juramento, porque fazendo isso nós O estaremos
exaltando, honrando e magnificando. Este preceito está expresso em Suas
palavras, enal­tecido seja Ele, ``E pelo Seu nome jurarás'' (Deuteronômio
6:13; 10:20), que os Sábios explicam assim: "a Torah diz 'Em Seu nome
jurarás' e novamente `Não jurarás em No d terno, teu Deus, em vão'
(Êxodo 20:7)". Da mesma forma que estamos roibi os de fazer um juramento
sem necessidade, e isto é um preceito negativo\footnote{Ver preceito negativo 62.}, assim
também somos ordenados a fazer um juramen­to quando necessário, e isto é
um preceito positivo.

É por essa razão que não é permitido jurar por nenhuma entidade criada,
tal como os anjos ou as estrelas, exceto quando o juramento é elíptico
como, por exemplo, quando se jura pela realidade do sol, significando
``pela realidade do Deus do sol''. E assim que nossa nação jura pelo nome
de Moisés nosso Mestre --- honrado seja seu nome --- como se aquele que
jura estivesse dizendo ``pelo Deus de Moisés'' ou "por Aquele que mandou
Moisés". Mas quan­do aquele que jurar não tiver a intenção de dizer isso
dessa forma e jurar po um ente criado, acreditando que essa entidade
contém em si própria uma v r­dade tal que se possa jurar por ela, ele
estará cometendo um pecado ao col car alguma outra entidade em pé de
igualdade com o Nome dos Céus; a Tradição\footnote{Sucá 45:b.} diz, a esse
respeito: "Aquele que colocar o Nome dos Céus em pé de igual àde com
algum outro ente deverá ser erradicado da face da terra".

Este era o significado pretendido no versículo "Pelo Seu Nome ju a­ rás
apenas": em nome d'Ele você deve atribuir ma verdade que seja digna de
um juramento.

Está dito no início do Tratado Temurá\footnote{Temurá 3:B.}: "De que modo ficamos sabendo que podemos nos comprometer por um ento a cumprir os preceitos? Pelo versículo 'E pelo Seu Nome jurarás'".

\section{Trilhar os caminhos de Deus}

Por este preceito somos ordenados a assemelhar-nos a Deus, enalte­cido
seja Ele, o tanto quanto nos for possível. Este preceito está expresso
em Suas palavras ``E andares por Seus caminhos'' (Deuteronômio 28:9). Este
preceito foi repetido várias vezes, e disse: "que andes em todos os Seus
caminhos" (Ibid. 11:22).

Com relação a este último versículo os Sábios comentam o seguinte:
"Assim como o Sagrado, enaltecido seja Ele, é chamado Misericordioso,
você também deve ser misericordioso; assim como Ele é chamado
Benevolente, vo­cê também deve ser benévolo; assim como Ele é chamado
Justo, você também deve ser justo; assim como Ele é chamado 'Hassid',
você também deve ser `hassid' ".

Este preceito já apareceu sob outra forma em Suas palavras "Após o
Eterno, vosso Deus, andareis" (Deuteronômio 13:5) que os Sábios explicam
como significando que devemos imitar as boas ações e os elevados
atributos pelos quais o Eterno, enaltecido seja Ele, é figurativamente
descrito, uma vez que Ele é de fato sublime de forma imensuravelmente
superior a toda essa descrição.


\section{Santificar o nome de Deus}


Por este preceito somos ordenados a santificar o Nome de Deus. Es­te
preceito está expresso em Suas palavras, "E serei santificado entre os
filhos de Israel" (Levítico 22:32). O conteúdo deste preceito é que
temos o dever de pregar esta verdadeira religião pelo mundo, sem temer
danos de qualquer es­pécie. Mesmo se um tirano tentar forçar-nos a
negá-Lo, não devemos obede­cer, e ao invés disso devemos preferir a
morte; e não devemos sequer enganar o tirano fazendo-o crer que O
negamos, embora em nossos corações continue­mos a crer n'Ele, enaltecido
seja Ele.

Este é o preceito relativo à Santificação do Nome que foí imposto a cada
um dos filhos de Israel: que devemos estar prontos a morrer nas mãos de
u' ti no por nosso amor a Ele, enaltecido seja Ele, e por nossa fé em
Sua Un de, \textless{} sim como fizeram Hananiah, Mishael e Azariah no
tempo do perverso Nabucodonosor,
 quando ele forçou o povo a prostrar-se diante do ídolo\footnote{Dan. 3:1.} e todos assim o fizeram, inclusive os
 israelitas, e não havia mais nin­g ém lá ara santificar o Nome dos
 Céus, pois estavam todos aterrorizados. Es-t foi tíma grande desgraça
 para Israel, pois este preceito foi desobedecido por todos eles.

Este preceito só se aplica em ocasiões como aquela, quando todo o
mundo estava aterrorizado e era um dever declarar publicamente,
naquela ocasião, a Sua Unidade. O Eterno já havia prometido, através
de Isaías, que Is­rael não seria desgraçada por completo naquele
momento difícil, pois entre eles surgiriam jovens sem medo da morte
que derramariam seu sangue e proclama­riam a Fé, santificando o Nome
publicamente, como ele nos ordenou através de Moisés, nosso Mestre.
Essa promessa está nas palavras "Agora Jacob não ficará
envergonhado, nem ficará seu rosto pálido; quando ele vir que os,
trabalho de Minhas mãos, no meio dele, santificam o Meu nome"\footnote{Isa> 29:22-23.}.

A Sifrá diz: "Eu vos tirei da terra do Egito com a condiç que
vocês santifiquem Meu nome publicamente".


Na Guemará do Tratado Sanhedrin está dito: "Um `Noachid' é obri­gado a
santificar Seu Nome, ou não? Ouçam isto: 'Aos `Noachidos' foi ordenado
que obedecessem sete precei e a eles foi ordenado que santificassem
Seu Nome, então seriam oito"\footnote{O termo ``Noachidos'' (descendentes de Noach) significa os não-israelenses ou pagãos de todos os tempos que, de acordo com a lei judaica, são obrigados a obedecer os sete preceitos seguintes:}.

Assim, ficou claro que st é u e dos preceitos obrigatórios para Is­rael,
tendo os Sábios deduzido esté pr to das palavras "Eu serei santificado
entre os filhos de Israel". As leis detalhadas sobre este preceito estão
expostas no sétimo capítulo de Sanhedrin.

\section{Ler o \textit{Shemá}}

Por este preceito somos ordenados a ler o Shemá diariamente, à noi­te e
pela manhã. Este preceito está expresso em Suas palavras, enaltecido
seja Ele, "E delas falarás sentado em tua casa, andando pelo caminho, e
ao deitar-te e ao levantar-te" (Deuteronômio 6:7).

As leis referentes a este preceito estão explicadas em Berakhot, on­de
está demonstrado que a leitura do ``Shemá'' foi ordenada pela Torah.

A Tosseftá diz: "Da mesma maneira que a Torah ordenou um horá­rio
determinado para a leitura do `Shemá', os Sábios determinaram um horário
para a Oração"; ou seja, os horários da Oração não são determinados pela
To­rah, mas o dever da oração em si é imposto pela Torah, como já
explicamos, e os Sábios determinaram os horários da Oração.

Os Sábios fixaram os horários da Oração de maneira a corresponder aos
horários em que os sacrifícios eram trazidos, nos tempos dos grandes
tem­plos de Jerusalém.

Este preceito não é obrigatório para as mulheres.

\section{O estudo da Torah}

Por este preceito somos ordenados a ensinar e a estudar a sabedoria da
Torah, que é chamada de Talmud Torah. Este preceito está expresso em
Suas palavras ``E as inculcarás a teus filhos'' (Deuteron 6:7).

O Sifrei diz: " 'A teus filhos' significa .studantes: verificamos


que em todo lugar os discípulos de um homemrão ch. mados de seus filhos,


como está dito em 'E os filhos dos profetas saí,: O Sifrei também diz,

no mesmo trecho: " 'E as inculcarás a teus filh significa que elas de-

vem fluir facilmente de sua boca, para que qua d pessoa faça 'uma per-


gunta sobre elas, você não vacile em sua resposta, e esponda com
presteza".\\
Este preceito está repetido muitas vezes: "E ensina-las-eis"
(Deuteronômio
11:19); ``E para que aprendam'' (Deuteronômio 31:12). A importância
deste preceito
e a obrigação de cumprí-lo estão enfatizadas em várias passagens
do Talmud.

36. O termo ``Noachidos'' (descendentes de Noach) significa os
não-israelitas ou pagãos de todos\\
os tempos que, de acordo com a lei judaica, são obrigados a obedecer os
sete preceitos seguintes:

\begin{enumerate}
\def\labelenumi{(\arabic{enumi})}
\item
 
 estabelecer tribunais de justiça;
 
\item
 
 não praticar idolatria;
 
\item
 
 não blasfemar;
 
\item
 
 não comèter incesto;
 
\item
 
 não matar;
 
\item
 
 não roubar, e
 
\item
 
 não comer carne retirada de animais enquanto vivos. Os. Noachidos que
 obser­vam estes 7 preceitos herdarão uma parte do Mundo Vindouro.
 
\end{enumerate}


37. Reis II, 2:3.



As mulheres não são obrigadas a obedecer a este preceito, de acor­do com
Suas palavras "E ensina-las-eis a vossos filhos" (Ibid., 11:19) sobre as
quais os Sábios comentam: " Filhos', mas não filhas", como foi explicado
na Guemará de Kidushin.

12 O \textbf{``TEFILIN''} DA CABEÇA

Por este preceito nos é ordenado o uso do ``Tefilin'' da Cabeça. Este
preceito está expresso em Suas palavras "E serão por marca entre os teus
olhos" (Deuteronômio 6:8).

Este preceito está repetido quatro vezes (Êxodo 13:9; ibid., 16;
Deu­teronômio 6:8; ibid., 11:18).

\section{O \emph{Tefilin} do braço}

Por este preceito nos é ordenado o uso do ``Tefilin'' do Braço. Este
preceito está expresso em Suas palavras, enaltecido seja Ele, "E os
atarás, como sinal na tua mão" (Deuteronômio 6:8). Também este preceito
está repetido qua­tro vezes (Êxodo 13:1; ibid., 16; Deuteronômio 6:8;
ibid., 11:18).

Na Guemará Menahot encontramos a prova de que o uso do "Tefi­lin" da
Cabeça e do ``Tefilin'' do Braço são dois preceitos, quando os Sábios se
mostram surpresos com a idéia de que o ``Tefilin'' dá Cabeça e o do Braço
não possam ser usados um sem o outro, e sim apenas os dois juntos. ``Se'',
di­zem eles, "alguém não puder cumprir dois preceitos, ele não deve
cumprir um?". Quer dizer, se alguém não puder cumprir os dois preceitos
ele não deve cum­prir nenhum deles? Não, ele deve cumprir aquele que ele
puder. Dessa maneira ele deve usar o ``Tefilin'' que possuir.

Assim ficou claro que os Sábios consideram o ``Tefilin'' do Braço e o
``Tefilin'' da Cabeça como sendo dois preceitos.

Estes dois preceitos não são obrigatórios para as mulheres, pois quan­do
explicou sobre sua obrigatoriedade, Ele disse, enaltecido seja Ele,
``Para que esteja a Torah do Eterno em tua boca'' (Êxodo 13:9) e as
mulheres não têm a obrigação de estudar a Torah. Esta é uma explicação
dada na Mekhiltá.

Todas as leis sobre estes dois preceitos estão explicadas no quarto
capítulo de Menahot.

\section{Os \emph{Tsitsit}}

Por este preceito nos é ordenado o feitio dos ``Tsitsit''. Este precei­to
está expresso em Suas palavras, enaltecido seja Ele, "Façam para eles
`Tsitsie sobre as bordas de suas vestes, pelas suas gerações; e porão
sob Tsitsit' da borda, um cordão azul celeste" (Números 15:38).

Este não é contado como dois preceitos, embora s regra e tre nós que o
azu1\textsuperscript{38} não prejudica a validade do
branco\textsuperscript{38}, nem o b anco\textsuperscript{38} ' valida o
azu1\textsuperscript{38}. A razão disso aparece no Sifrei: "Poder-se-ia
pensar que s são dois preceitos --- o preceito do azul e o preceito do
branco; por esse motivo, a To­rah estabelece 'E será para vós por
``Tsitsit'' ' (Ibid., 39), mostrando assim que se trata de um preceito e
não de dois".

Este preceito não é obrigatório para as mulheres, como está explicado no
Início de Kidushin. Todas as leis deste preceito estão explicadas em
Menahot.

\section{A \emph{Mezuzá}}

Por este preceito nos é ordenado a confecção da ``Mezuzá''. Este
preceito está expresso em Suas pai s, enaltecido seja , "E as escreverás
nos umbrais de tua casa e em teu ort es" (Deuteronômio 6:9). Este
preceito está repetido novamente na Tor h.°.


Todas as suas leis estãó : . \textbf{IP} nadas no terceiro capítulo de
Menahot.



\section{A reunião do povo no santuário durante festa dos tabernáculo}

Por este preceito somos ordenados a que odas as pessoas se reu­nam no
segundo dia dos Tabernáculos, depois do finalde cada sétimo ano, e a que
vários versículos de Deuteronômio sejam lidos para elas. Este preceito
está expresso em Suas palavras, enaltecido seja Ele, "Congrega o povo,
os ho­mens e as mulheres, e as crianças, etc." (Deuteronômio 31:12), as
quais são o preceito da Assembléia.

Em Kidushin está dito: "O cumprimento de todos os preceitos posi­tivos
que estão ligados a uma ocasião não é obrigatório para as mulheres". O
Talmud levanta a questão: "Não é a Assembléia um preceito positivo
ligado a uma ocasião e mesmo assim obrigatório para as mulheres?" A
conclusão a que se chegou ao final da discussão foi que "Não se deve
argumentar a partir de uma regra geral".

As normas deste preceito, a saber, como deve ser feita a leitura, quem
deve ler, e que trechos devem ser lidos estão explicadas no sétimo
capítulo do Tratado Sotá.

\section{Um rei deve transcrever o rolo da Torah}

Por este preceito somos ordenados a qu t o rei de nossa nação que ocupar
o trono real transcreva um Rolo da Tora \textsuperscript{42} ara si
mesmo, do qual ele não deve se separar. Este preceito está expresso Suas
palavras, enalteci­do seja Ele, "E quando se sentar sobre o trono de seu
reino, escreverá para si o traslado desta Lei" (Deuteronômio 17:18).

Todas as normas deste preceito estão explicadas no segundo capítu­lo de
Sanhedrin.

\section{Obter um rolo da Torah}



Por este p ecei i. somos ordenados a que todo varão entre nós te-


nha um Rolo da Tor ara si próprio. Se ele o transcrever de próprio pu-

nho, isto será muito ciado e, de preferência, é assim que deve ser feito


\begin{enumerate}
\def\labelenumi{\arabic{enumi}.}
\setcounter{enumi}{38}
\item
 
 Ou seja, nos foi ordenado que fixemos a ``Mezuzá''. Vide Halachot
 Tefilin Cap. 5, lei 7.
 
\item
 
 Deuteronômio 11:20.
 
\item
 
 Festa de Sucot.
 
\item
 
 O Sefer Torah (Pentateuco) que lemos na sinagoga, aos sábados.
 
\item
 
 Sefer Torah
 
\end{enumerate}


PRECEITOS POSITIV 4 S 93

pois os Sábios di ele o transcrever de próprio punho, será conside-

rado pelas Escri u o se ele o tivesse recebido do Monte Sinai". Se ele

próprio não puder o, ele é obrigado a comprar um, ou a contratar alguém

que o transcreva por ele. Este preceito está expresso em Suas palavras,
enalte­cido seja Ele, ``E agora escrevei para vós este cântico''
(Deuteronômio 31:19). Contudo, como não é permitido transc er alguns
trechos dela, as palavras ``este cântico'' devem necessariamen sig m car a
Torah completa, que inclui ``este cântico''.

A Guemará de Sanhedri d 45, ba diz: Ainda que seus pais lhe

tenham deixado um Rolo da Lei, ele d ve transcrever o seu próprio, como
está dito, 'E agora, escrevei para vós este cântico'. Abaye objetou:
Será que trans­crever um Rolo da Torah em seu próprio nome, porque ele
não deve desejar se apoiar nos seus pais, é apenas uma obrigação do Rei
e não do plebeu? A res­posta a isso foi: A regra é necessária apenas
para obrigar o Rei a transcrever dois Rolos, pois nos foi ensinado que
'Ele escreverá para si o traslado desta Lei' (Deu­teronômio 17:18)
significa que ele deve transcrever para si mesmo duas cópias". Ou seja,
a diferença entre o Rei e um plebeu é que cada homem deve transcre­ver
um Rolo da Lei, mas o Rei deve transcrever dois, como está explicado no
segundo capítulo de Sanhedrin.

As regras para a transcrição de um Rolo da Torah e as condições
re­lativas a isto estão explicadas no terceiro capítulo de Menahot, no
início de Ba­ba Batra e em Shabat.

\section{Dar graças após as refeições}

Por este preceito somos ordenados a dar graças ao Eterno, enalteci­do
seja Ele, após cada refeição. Ele está expresso em Suas palavras,
enaltecido seja Ele, "E comerás e te fartarás, e louvarás ao Eterno, teu
Deus" (Deuteronô­mio 8:10).

A Tosseftá diz: "Ficamos sabendo que dar graças após uma refeição é um
preceito imposto pela Torah através do versículo 'E comerás e te
fartarás, e louvarás ao Eterno, teu Deus' ".

As normas deste preceito estão explicadas em vários trechos do Tra­tado
Berakhot.

\section{A construção do santuário}

Por este preceito somos ordenados a construir uma Casa para Seu serviço.
Lá deverão ser oferecidos sacrifícios e deverá arder o fogo perpétuo,
para lá serão feitas as peregrinações e lá terão lugar todos os anos as
festas e assembléias. Este preceito está expresso em Suas palavras,
enaltecido seja Ele, ``E Me farão um santuário'' (Êxodo 25:8).

O Sifrei diz: "Os israelitas foram ordenados a cumprir três preceitos ao
entrar na Terra: designar um rei para si próprios, construir o
Santuário, e destruir os descenden e alec' . Fica, dessa forma, claro
que a constru­ção do Santuário co tui u preceito em si.

Já explic e este preceito global inclui preceitos individuais


\begin{enumerate}
\def\labelenumi{\arabic{enumi}.}
\setcounter{enumi}{43}
\item
 
 Menahot 30:A
 
\item
 
 Sanhedrin 21:B
 
\item
 
 Décimo Segundo Fundamento.
 
\end{enumerate}

e que o Castiçal, a Mesa, o Altar, e as outras coisas são todos partes
do Santuário e que tudo isso junto é chamado ``o Santuário'', embora haja
um preceito espe­cífico para cada uma das partes.

É verdade que Ele disse, com relação ao Altar: "um altar de terra vo­cê
deverá fazer para Mim" (Êxodo 20:21), de maneira que se poderia pensar
que este é um preceito independente, separado do de construir o
Santuário, mas o significado verdadeiro, neste caso é o que vou explicar
a vocês. O senti­do literal do versículo se refere aos tempos em que os
Altos Lugares nos eram permitidos e nós tínhamos autorização para fazer
um Altar de terra em qual­quer lugar, e oferecer sacrifício ne e os
Sábios já declararam que o objetivo do versículo era ordenar-nos const
ção de um Altar ligado à terra, que não fosse móvel, como era no des to
\textsuperscript{47}. sto foi dito por eles na Mekhiltá de Rabi Ishmael,
onde o versículo é in do assim: "Quando você entrar na Terra de Israel
você deverá erguer um altar para Mim ligado à terra". Sendo assim, o
preceito é um dos que são obrigatórios por todas as gerações, e
conclui-se que ele faz parte dos deveres do Templo, e quer dizer que o
altar a ser construí­do deve ser de pedra. Ao explicar as palavras "E se
você Me fizer um altar de pedra" a Mekhiltá diz: "Rabi Ishmael diz:
'Toda palavra 'se' na Torá implica permissão, exceto em três ocasiões,
uma das quais é 'E se você Me fizer um altar de pedra' " . Então os
Sábios disseram: "O versículo 'E se você Me fizer um altar de pedra'
estabelece uma obrigação. Você afirma que isto é uma obri­gação; e se
fôsse apenas uma permissão? A Torah nos diz 'Com pedras inteiras
edificarás o altar do Eterno, teu Deus' (Deuteronômio 27:6)".

As normas relacionadas à construção do Santuário, seu modelo e suas
divisões, a construção do Altar, e as leis relativas, estão explicadas
num Trata­do consagrado especialmente ao assunto, que é o Tratado Midot.
Da mesma forma, o modelo do Castiçal, da Mesa e do Altar de ouro, e suas
posições no Santuário estão explicados na Guemará Menahot e Yoma.

\section{Respeitar o santuário}

Por este preceito somos ordenados a ter uma atitude de grande e profunda
admiração e de temor para com o Santuário, e a venerá-lo em nossos
corações com receio e temor, pois é esse o respeito ao Santuário que foi
im­posto por Suas palavras, enaltecido seja Ele, "E Meu santuário
temereis" (Leví­tico 19:30).

A definição desse respeito está na Sifrá: "O que significa respeito? Que
não se deve entrar no Monte do Templo com seu cajado, ou suas
sandá­lias, ou sua mochila, ou com poeira em seus pés, ou usar o templo
como passa­gem; e de forma alguma se deve cuspir lá dentro". Está
explicado em várias passagens do Talmud que não é permitido a ninguém
ficar sentado no Tribu­nal, com exceção dos Reis da Dinastia de David.
Tudo isso é decorrente de Suas Palavras, enaltecido seja Ele, "E Meu
Santuárió temereis", às quais temos a obri­gação de obedecer por todos
os tempos, até mesmo em, nossos dias, quando, sem virtude de nossos
muitos pecados, ele foi destruído.

A Sifrá diz: "De que modo deduzimos que não somente enquanto o Santuário
existia, mas também depois que ele deixou de existir, o respeito deve
continuar? A Torah diz: 'Meus ``Shabatot'' ' guardareis, e Meu Santuário
te-

47. Números 4:14; Êxodo 27:7.

mereis' (Levítico 19:30 e 26:2), ou seja, assim como a observação do
Shabat é para sempre, da mesma forma o respeito pelo Santuário é para
sempre".

No mesmo trecho lemos: "Seu respeito não deve ser para com o San­tuário,
mas sim para com Ele que nos deu as ordens referentes ao Santuário".

\section{A guarda do santuário}

Por este preceito somos obrigados a manter a guarda ao Santuário e a
vigiá-lo todas as noites e durante toda a noite, e dessa forma honrá-lo,
exaltá-lo e glorificá-lo. Este preceito está expresso em Suas Palavras,
enaltecido seja Ele, a Aarão "E tu e teus filhos contigo, estareis
diante da tenda da assinação" (Nú­meros 18:2) ou seja, você deverá
manter a guarda no Santuário para sempre. Este preceito também é
encontrado sob outra forma: "E manterão o serviço da guarda da tenda da
assinação" (Ibid. 4).

E está escrito no Sifrei: " `Tu e teus filhos contigo, est diante

da tenda do testemunho': os 'Cohanim' dentro, e os Levitas fora". u s
ia, eles devem guardar e vigiar o Santuário e ao redor dele, revezando-
\textbf{48}

A Mekhiltá diz: " 'E manterão o serviço da guarda da te da da
assi­nação' nos dá apenas um preceito posifi • ► e que maneira sabem que
há um preceito negativo envolvido també ? Na orah está escrito: 'E
mantereis o serviço da guarda da santidade' " (Ib •
.5)\textsuperscript{49}. i ica, assim, claro que a guarda do Santuário
constitui um preceito pos tivo.

No mesmo lugar lemos: "Engran • - ce o Santuário que haja guardas nele,
porque um palácio que tem guardas é diferente de um palácio que não os
tem", e é sabido que palácio (Palturim) é um nome para o Santuário. O
signi­ficado disto é que se exalta e se glorifica o Santuário ao se
designar guardas pa­ra vigiá-lo.

Todas as normas deste preceito estão explicadas nos Tratados Ta­mid
(primeiro capítulo) e Midot.

\section{Os serviços dos levitas no santuário}

Por este preceito ordena-se que apenas os Levitas,realizem determi­nados
serviços específicos no Santuário, tal como fechar os portões e cantar
durante a oferta de sacrifícios. Este preceito está expresso em Suas
palavras ``E servirão os Levitas no serviço da tenda de assinação''
(Números 18:23).

O Sifrei diz: "Eu poderia supor que ele pode querer escolher entre
executar o serviço ou não e por isso a Torah diz: 'E servirão os Levitas
no servi­ço', ou seja, eles devem fazê-lo". É, portanto, um dever que
tem que ser execu­tado querendo ou não.

A natureza do serviço dos Levitas está explicada em diversos trechos em
Tamid e em Midot, e no segundo capítulo de Arakhin está explicado que
apenas os Levitas devem cantar.

Este preceito aparece novamente sob outra forma: "Servir em nome do
Eterno, seu Deus, como todos os seus irmãos levitas" (Deuteronômio
18:7),


\begin{enumerate}
\def\labelenumi{\arabic{enumi}.}
\setcounter{enumi}{47}
\item
 
 O revezamento deve ser feito entre os membros de cada grupo apenas, e
 não entre os grupos, ou seja, a guarda do interior do Santuário está
 restrita aos Cohanim e a do exterior aos Levitas.
 
\item
 
 Ver o preceito negativo 67.
 
\end{enumerate}




vo a cada Shabat, juntamente com incenso, e que os ``Cohanim'' devem comer
do pão que havia sido colocado no Shabat anterior.

As normas deste preceito estão explicadas no décimo primeiro capí­tulo
de Menahot.

\section{A queima do incenso}

Por este preceito os ``Cohanim'' são ordenados a colocar incenso
diariamente, duas vezes por dia, no Altar de Ouro. Ele está expresso em
Suas palavras, enaltecido seja Ele, "E Aarão fará queimar sobre ele,
incenso de espe­ciarias; pela manhã, quando limpar as lamparinas, o
queimará" (Êxodo 30:7).

As normas deste preceito e o procedimento a ser seguido na queima diária
do incenso estão explicadas no início de Queretot e em várias passagens
de Tamid.

\section{O fogo perpétuo do altar}

Por este preceito somos ordenados a manter o fogo aceso no Altar, todos
os dias, constantemente. Ele está expresso em Suas palavras "Fogo
con­tínuo estará aceso sobre o altar" (Levítico 6:6), o que só pode
significar que eles são obrigados a colocar lenha no fogo todos os dias,
sem falta, pela manhã e ao anoitecer, como está explicado no segundo
capítulo de Yoma e no Trata­do Tamid.

O Talmud diz claramente: "Embora o fogo venha dos Céus, é um dever
mantê-lo queimando por meios comuns".

As normas deste preceito, que é o preceito relativo à preparação diária
do Fogo sobre o Altar estão expostas no quarto capítulo de Yoma, e no
segun­do capítulo de Tamid.

\section{Remover as cinzas do altar}

Por este preceito os ``Cohanim'' são ordenados a remover as cinzas do
Altar diariamente. Isto é chamado a Retirada das Cinzas, e o preceito
está expresso em Suas palavras, enaltecido seja Ele, "E vestirá o
``Cohen'' a sua tú­nica de linho ... e separará a cinza" (Levítico 6:3).

As normas deste preceito estão explicadas em várias passagens dos
Tratados Tamid e Quipurim.

\section{Retirar os impuros}

Por este preceito somos ordenados a expulsar do Templo as pessoas
impuras. Ele está expresso em Suas palavras, enaltecido seja Ele, "Que
enviem do acampamento todo o leproso, e todo aquele que padece de fluxo
e todo o impuro por ter tocado os ossos de cadáver de pessc■a" (Números
5:2).

A palavra ``acampamento'' aqui significa o Acampamento da Presen­ça
Divina, que, em gerações posteriores, foi o equivalente à Corte do
Santuá­rio, como explicamos no início da Ordem Teharot, em nosso
Comentário so­bre a Mishná. O Sifrei diz: " 'Que enviem do acampamento'
é uma advertência para que pessoas impuras não entrem no Santuário em
estado de impureza".


Este preceito aparece também sob outra forma: "Se houver entre vós

um homem que não estiver puro por causa de derramamento de sêmem, de
noite, sairá para fora do acampamento" (Deuteronômio 23:11). O
"acampamen­to" aqui deve ser entendido como o Acampamento da Presença
Divina, uma vez que o próprio preceito diz: "Fora do acampamento os
enviareis" (Núme­ros 5:3) e que na Guemará de Pessahim se lê: " `Sairá
para fora do acampamen­to' significa o Acampamento da Presença Divina".

A Mekhiltá diz: " 'Ordena aos filhos de Israel que enviem do
acam­pamento': este é um preceito positivo. De que maneira concluímos
que tam­bém 'há um preceito negativo envolvido? A Torah diz: 'Para que
não contami­nem os seus acampamentos"'.


O Sifrei diz: " 'Sairá para fora do acampamento' é um preceito


positivo".

\section{Honrar os ``cohanim''}

Por este preceito somos ordenados a exaltar os descendentes de Aa­rão,
para demonstrar-lhes honra e respeito, e a conferir-lhes alto grau de
santi­dade e dignidade, mesmo contrariando suas próprias objeções. Tudo
isso é pa­ra a glória do Eterno, enaltecido seja Ele, já que Ele os
escolheu para Seu servi­ço e para as ofertas de Seus sacrifícios. Este
preceito está expresso em Suas pa­lavras "E santifica-lo-ás, porque o
sacrifício de teu Deus ele oferece, santo será para ti" (Levítico 21:8)
que os Sábios interpretam assim: " santifica-lo-ás', quer dizer, ele
será o primeiro em todos os assuntos sagrados como, por exemplo, na
leitura da Torah; ele deverá ter a prioridade para ler as Bençãos nas
refei­ções; e ele deverá ser o primeiro a receber uma porção justa".

A Sifrá também diz: " 'E santifica-lo-ás' --- mesmo contra sua
vonta­de"; ou seja, este é um preceito estabelecido para nós, e não
depende da von­tade do ``Cohe ''

Da a forma ele diz: " 'Santos serão para seu Deus' (Ibid., 6) ---

mesmo contra ontade. 'E serão santidade' (Ibid.) --- inclusive os que
tive-

rem um defe' rtanto não devemos argumentar: "Já que este 'Cohen' não

está capacita erecer o sacrifício de seu Deus, por que deveríamos
dar-lhe

prioridade e nstrar-lhe honra e respeito?" Porque Ele disse: "E serão
san-

tidade", significando toda essa honrada família, incluindo tanto os
perfeitos co­mo os defeituosos.

As cláusulas apropriadas nas quais está estabelecido que devemos
tratá-los desta forma estão explicadas em diversos trechos da Guemará de
Ma-cot, Hulin, Bekhorot, Shabat, e em outros trechos.


\section{As vestes dos ``Cohanim''}


Por este preceito os ``Cohanim'' são ordenados a ornar-se com ves­tes de
especial esplendor e beleza antes de servir no Santuário. Ele está
expresso em Suas palavras, enaltecido seja Ele, "E farão vestidos antidade
para Aarão, teu irmão, para o esplendor e para a beleza" (Êx. 4); "E a seus
filhos farás chegar, e os farás vestir as túnicas" (Ibid., 29:8). Estas são as
Vestes dos ``Cohanim'': oito vestimentas para o "Cohen Gado quatro para o


%Ver os preceitos negativos 70 e 71.
 
%C4 Sumo Sacerdote.
``Cohen'' comum. Caso o ``Cohen''celebre o ofício com menos ou com mais
vestimentas do que o número designado para aquele ofício específico, o
ofício fica invalidado, e ele fica sujeito à morte pela mão dos Céus ---
quero dizer, aquele que celebrar o ofício com menos vestes do que o
número designado. Na Guemará de Sanhedrin ele também está entre os que
podem ser mortos pela mão dos Céus. Isto não está explicitamente dito
nas Escrituras, mas é derivado do versículo "E lhes cingirás cintos ...
e será para eles o sacerdócio" (Êxodo 29:9), cuja interpretação é:
"Enquanto usam as vestimentas, eles estão revesti-

dos de seu sacerdócio; quando não usam suas vesti , não estão revesti-

dos de seu sacerdócio", e se transformam em leigi explicado a seguir

que um leigo que oficia está sujeito à pena de mo

A Sifrá diz: " 'E pôs sobre ele o peitoral ico 8:8): esta passa-


gem nos ensina regras que se aplicam a essa ocasião es cífica e também
regras


que se aplicam permanentemente; regras para os diários, e também as

regras para o culto do Dia do Perdão. Todos os d via oficiar com trajes

dourados, mas no Dia do Perdão com vestes de 1 h. branco".

Pela seguinte passagem da Sifrá fica claro que a colocação dessas
ves­tes é um preceito positivo: "Como concluímos que Aarão não colocou
as ves­tes do 'Cohen' apenas para seu próprio enaltecimento, e sim como
aquele que obedece a ordem de seu Rei? Pelas palavras da Torah 'E fez
como ordenou o Eterno a Moisés' "; ou seja, embora essas vestes, com seu
ouro, ônix, jásper e outras pedras preciosas, fossem de beleza
inigualável, não seria o gozo desta beleza que o ``Cohen'' deveria tomar
em consideração, mas apenas o cumpri­mento do preceito que Deus impôs a
Moisés, a saber, que ele deveria sempre usar essas vestes no Santuário.

Todas as normas deste preceito estão explicadas no segundo capítu­lo do
Zebahim e em vários trechos de Quipurim e de Sucá.

\section{Os ``Cohanim'' devem carregar a arca sagrada}



Por este preceito somos ordenados a que os ``Cohanim'' carreguem


a 0. obre os ombros, quando desejarmos transportá-la
de um lugar para

o tr sse preceito está expresso em Suas palavras, enaltecido seja Ele,
"Por-


q e o serviço da santidade estava sobre eles; eles o levavam aos ombros"
(Números
7:9). Embora este preceito tenha sido imposto naquela época aos
Levitas
, isso foi apenas por causa do número limitado de ``Cohanim'', já que
Aarão


foi o primeiro. Na idade, o cumprimento deste preceito compete aos "Co-

hanim", e são e devem carregar, como fica claro no livro de Joshuá e

no livro de Sa quando Davi ordenou trazer a Arca pela segunda vez,

o livro das Crôni gistra: "Assim os ``Cohanim'' e os Levitas se
santificaram

para levantar a Arca do Eterno, Deus Israel. E os filhos dos Levitas
carrega-

ram a Arca de Deus com as barras s' • re eus ombros, como Moisés
ordenara, de acordo com a palavra do Eter


\begin{enumerate}
\def\labelenumi{\arabic{enumi}.}
\setcounter{enumi}{52}
\item
 
 Ver o preceito negativo 74.
 
\item
 
 O ``Cohen Gadol''.
 
\item
 
 Que continha as duas tábuas com os 10 Mandamentos.
 
\item
 
 Josh. 3:14; II Sam. 15:25.
 
\item
 
 I Cron. 15:14-15 e 5:16.
 
\end{enumerate}

Da mesma forma, ao se referir à divisão dos ``Cohanim'' em 24 Gru­pos, o
livro de Crônicas diz: "Essas eram suas posições em seus serviços para
entrar na casa do Eterno, de acordo com as leis dadas a eles pela mão de
Aarão, seu pai, como o Eterno, Deus de Israel, lhe havia ordenado". Os
Sábios expli­cam este versículo como significando que é dever dos
``Cohanim'' carregar a Arca nos ombros e que é isso o que o Eterno, Deus
de Israel, ordenou. O Sifrei diz: " `De acordo com as leis dadas a eles
.... como o Eterno, Deus de Israel, lhe havia ordenado': onde foi que
Ele lhe deu essa ordem? Em 'Porém aos fi­lhos de Kehat não deu; porque o
serviço da santidade estava sobre eles; eles o levavam aos ombros' ".

Fica assim claro que este é um dos preceitos.

\section{O óleo da unção}

Por este preceito somos ordenados a ter óleo feito para nós de acor­do
com uma composição específica, pronto para a Unção de todo "Cohen Ga-

dol" que venha esignado, como Ele diz: "E o 'Cohen Gadol', entre seus

irmãos, sobre c a cab a for derramado o óleo da unção" (Levítico 21:10).
Com esse óleo tamb se de eria ungir alguns dos reis, como está explicado
nas nor­mas deste pre e 0\textsuperscript{58}.

O b lo e todos os seus vasos foram ungidos com este óleo,

mas os vasos n.o serão ungidos com ele no futuro, pois
o Sifrei diz explicita­mente: "Com a unção destes," --- ou seja, dos
vasos do Tabernáculo --- "todos os vasos foram santificados para
sempre", como Ele disse, enaltecido seja Ele: "Oléo de unção de
santidade será este para Mim por vossas gerações" (Exodo 30:31).


As cláusulas deste preceito estão explicadas no início de Queretot.


\section{Os ``Cohanim'' devem oficiar em grupos, revezando-se no serviço}

Por este preceito os ``Cohanim'' são or•enados a oficiar em grupos, sendo
que cada grupo deve oficiar durante u se i ana, e a não oficiar to

ao mesmo tempo, exceto durante os Festivais todos os grupos d. em\\
participar igualitariamente, e quando qualquer m\textsuperscript{59} p
esente pode ofere' er sa-

crifícios. Aparece nas Crônicas que Davi e Sam 1 os d vidiram em 24
os.°,

e na Guemará Sucá está explicado que durante vais todos partici 7am

de forma igual.

O trecho das Escrituras no qual está expresso este preceito é •
se­guinte: "E quando vier o 'Cohen', o qual descende da tribo de Levi,
de alguma das tuas cidades, de todo o Israel, onde ele habita e vier com
todo o desejo de sua alma ao lugar que escolheu o Eterno; e servir em
nome do Eterno, seu Deus, como todos os seus irmãos Levitas, que servem
ali diante do Eterno, igual porção receberão todos" (Deuteronômio
18:6-8). O Sifrei diz: " 'E vier com to-


\begin{enumerate}
\def\labelenumi{\arabic{enumi}.}
\setcounter{enumi}{57}
\item
 
 Ver também o preceito negativo 84.
 
\item
 
 Qualquer ``Cohen''.
 
\item
 
 1 Cron. 24:4-18.
 
\end{enumerate}




do o desejo de sua alma' poderia ser a qualquer momento, por isso nas
Escritu­ras está dito: 'De alguma de tuas cidades', ou seja, quando todo
o povo de Is­rael estiver reunido em uma cidade durante os Festivais.
Poder-se-ia pensar que todos os grupos participavam igualitariamente das
oferendas nos Festivais, mes­mo daquelas que não decorriam
especificamente dos Festivais, por isso a To­rah esclarece: 'Exceto a
parte dos patrimônios paternos' (Deuteronômio 18:8). O que significa o
patrimônio paterno? 'Celebre você durante sua semana, que eu celebrarei
durante minha semana' "; ou seja, eles concordaram com o reve­zamento
dos grupos, e com todo o arranjo do ofício em grupos, com um novo grupo
celebrando a cada semana. O Targum explica o versículo da seguinte
ma­neira: "Exceto o grupo daquela semana, pois assim o decretaram os
pais".


As normas deste preceito estão explicadas no final da Guemará de


Sucá.

\section{Os ``Cohanim'' devem fazer-se impuros pelos parentes mortos}

Por este preceito os ``Cohanim'' são ordenados a fazer-se impuros por
seus parentes lembrados na Torah. Uma vez que as
Escrituras lhes proí­bem, por respeito, de fazer-se impuros pelos
mortos, mas lhes permitem fazê-lo por parentes, poder-se-ia pensar que o
``Cohen'' pode escolher entre fazer-se impuro ou não. Ele lhes impôs uma
obrigação positiva ao dizer: ``Por ela se fará impuro'' (Levítico 21:3).

A Sifrá diz: " 'Por ela se fará impuro' é um preceito positivo. Se ele
não deseja se fazer impuro, deverá fazê-lo contra sua vontade. Isso
aconteceu com o ``Cohen'' Yossi cuja esposa morreu na véspera de ``Pessah''
e ele se re­cusou a se fazer impuro por ela; os Sábios então se
utilizaram de força e o obri­garam a fazer-se impuro, contra sua própria
vontade".


Neste preceito está baseado o dever do luto, ou seja, a obrigação de ovo
de Israel de ficar de luto pelos parentes, que são em número de para
confirmar essa obrigação que Ele declarou expressamente que no


c o ``Cohen'', o qual está normalmente proibido de fazer-se impuro, ele

d verá fazê-lo a qualquer custo, como todos os outros israelitas, de
forma que a lei de luto não seja julgada com leviandade.

Tem sido demonstrado que o luto do primeiro dia está prescrito pe­la lei
das Escrituras. Na interpretação eles disseram explicitamente na Guemará

de Mo tan que o luto não deve ser guardado durante um Festival: "Se o

luto antes do Festival, o preceito positivo abrangendo todo o povo

de Isr sobrepõe ao preceito que lhe foi imposto individualmente". Por-

tanto, aro que as Escrituras obrigam a guardar o luto, mas apenas no
pri-

meiro dia, pois os outros seis dias foram impostos pelos Rabinos; e que
até mes­mo um ``Cohen'' é obrigado a observar o luto no primeiro dia, e
fazer-se impu­ro por seus parentes. Compreenda isso.


As regras detalhadas deste preceito estão expostas no Tratado Mas-

\begin{enumerate}
\def\labelenumi{\arabic{enumi}.}
\setcounter{enumi}{60}
\item
 
 Lembrados ou enumerados na Torah. Levítico 21:2-3.
 
\item
 
 Mãe, pai, filho, filha, irmão, irmã; o marido e a mulher só têm essa
 obrigação por ordem poste­rior contida no Talmud.
 
\item
 
 De que os Festivais devem ser cheios de alegria.
 
\end{enumerate}

hkin, em vários trechos de Berakhot, Quetubot, Yebamot e Abodá Zará, e
na Sifrá, na passagem que começa com "Fala aos `Cohanim' " (Levítico 2 1
: 1 ).

A obrigação do ``Cohen'' de fazer-se impuro por seus parentes não se
estende às mulheres porque o ``Cohen'', que está proibido de fazer-
impu­ro por outros que não sejam seus parentes, tem a obrigação de fazê
o p e r pa-

rentes, mas a mulher da família do ``Cohen'' não está proibida de im-

pura por qualquer pessoa morta, como vou explicar oportuname con-

seqüentemente não tem o dever nem a obrigação de fazê-lo. Ela d v ardar

o luto, mas quanto a fazer-se impura ou não, depende de sua vontade.
Com­preenda isso.

\section{A obrigação do ``cohen gadol'' de casar-se apenas com uma virgem}

Por este prece o ``Cohen Gadol'' é ordenado a casar-se com uma virgem.
Está expresso S s palavras, enaltecido seja Ele, "E ele, mulher em sua
virgindade, toma " (Le ítico 21:13).


Nas Escrit ras\textsuperscript{65} es á explicitamente dito: "Rabi Akiba
afirmava que\\
até mesmo o nasci ento de um filho contrário a este preceito positivo
seria\\
considerado um bastúdo" como exemplo de uma união meramente contrária
a um preceito positive eles citam o caso de um ``Cohen Gadol'' que
tenha\\
um relacionamento com uma mulher que não seja virgem; porque é um
princívado
de um preceito positivo tem a\\
claro que este é um preceito positi`Cohen
Gadol' é obrigado a casar-


\section{O holocausto diário}

Por este preceito somos ordenados a oferecer no Santuário todos os dias
dois cordeiros que são chamados de Oferendas Contínuas. Ele está
ex­presso em Suas palavras, enaltecido seja Ele, "Dois para cada dia, em
holocaus­to contínuo" (Números 28:3).

As normas que regem este preceito, a ordem dos sacrifícios e os mé­todos
a serem seguidos estão explicados no segundo capítulo de Yoma e no
Tratado Tamid.

\section{A oferta diária do alimento pelo ``cohen gadol''}

Por este preceito somos ordenados a que o ``Cohen Gadol'' ofereça todos os
dias uma oblação pela manhã e uma ao anoitecer, chamada de Bolo


\begin{enumerate}
\def\labelenumi{\arabic{enumi}.}
\setcounter{enumi}{63}
\item
 
 No preceito negativo 166.
 
\item
 
 Quetubot 30:A.
 
\item
 
 Horayot 11:B.
 
\end{enumerate}




do ``Cohen Gadol'', conhecida também como oblação do ``Cohen'' ungido. Este
preceito está expresso em Suas palavras, enaltecido seja Ele, "Esta é a
oferta de Aarão e seus filhos" (Levítico 6:13).

As normas deste preceito, bem como a hora e a maneira de fazer a
oferenda estão expostas no sexto e nono capítulos de Menahot e em vários
tre­chos em Yoma e Tamid.

\section{A oferta adicional do shabat}

Por este preceito somos ordenados a oferecer um sacrifício todos os
Shabatot, além do holocausto diário. Ele está expresso em Suas palavras,
enal­tecido seja Ele, "E no dia de Shabat, dois cordeiros de um ano de
idade etc" (Números 28:9).

A ordem dos sacrifícios está explicada no segundo capítulo de Yo­ma e em
Tamid.

\section{A oferta adicional da lua nova}

Por este preceito somos ordenados a oferecer um sacrifício a cada lua
nova, além do holocausto diário, sendo esta a Oferta Adicional da Lua
No­va. Este preceito está expresso em Suas palavras, enaltecido seja
Ele, "E nos princípios de vossos meses oferecereis em holocausto ao
Eterno" (Números 28:11).


\section{A oferta adicional de ``pessah''}


Por este preceito somos ordenados a oferecer um sacrifício em cada um
dos sete dias de ``Pessah'', além do holocausto diário, sendo esta uma
Ofer­ta Adicional do Festival do Azimo. Este preceito está expresso em
Suas pala­vras, enaltecido seja Ele, "E oferecereis por sete dias oferta
queimada ao Eter­no" (Levítico 23:8).

\section{A oblação da nova cevada}

Por este preceito somos ordenados a fazer a oblação da cevada no
sexagésimo dia de Nissan, juntamente com o holocausto de um carneiro de
um ano, sem defeito. Este preceito está expresso em Suas palavras,
enaltecido seja Ele, "Trareis ao 'Cohen' um 'omer' das primícias de
vossa ceifa" (Levítico 23:10).

Esta oblação é chamada ``A Oferta das Primícias'' e ela está mencio­nada
em Suas palavras, enaltecido seja Ele, "E se ofereceres oblação de
primí­cias" (Levítico 2:14). A Mekhiltá diz: "Todo `se' na Torah implica
numa opção, exceto em três casos, em que é usado com relação a uma
obrigação; um deles é este: 'E se ofereceres oblação de primícias'. Você
está certo de que isto é uma obrigação? Talvez seja apenas uma
permissão. E dizem: 'Assim oferecerás a obla­ção de tuas primícias etc.'
(Ibid.), mostrando que isso é uma obrigação e não uma permissão".

Todas as normas deste preceito estão explicadas na íntegra no déci­mo
capítulo de Menahot.

\section{A oferta adicional de ``shabuot''}

Por este preceito somos ordenados a oferecer uma Oferta Adicional também
no quinquagésimo dia após a oferta do ``Omer'', que é no sexagésimo dia de
Nissan. Esta é a Oferta Adicional da Festa das Semanas mencionada no
livro de Números. Este preceito está expresso em Suas palavras "E no dia
das priiinícias, quando oferecerdes oblação nova ao Eterno... E
oferecereis holocaus­to, para ser aceito com agrado pelo Eterno"
(Números 28:26-27).

\section{Levar dois pães em ``shabuot''}

Por este preceito somos ordenados, como está prescrito, a levar ao
Santuário dois pães na Festa das Semanas, juntamente com os sacrifícios
obri­gatórios da Oferenda do Pão, e a oferecer sacrifícios, como
prescrito no livro de Levítico; e depois que esses pães forem movidos,
os ``Cohanim'' devem comê-los juntamente com os cordeiros das Ofertas de
Paz. Este preceito está expres­so em Suas palavras, enaltecido seja Ele,
"De vossas habitações trareis dois pães para serem movidos, de duas
décimas partes de uma `efa' " (Levítico 23:17).

Está explicado no quarto capítulo de Menahot que este sacrifício, que
era um complemento da Oferta do Pão, é uma oferta a parte e diferente da
Oferta Adicional do dia. Já fornecemos explicação suficiente a este
respeito em nosso comentário no Tratado Menahot.

Todas as normas deste preceito estão explicadas em Menahot nos capítulos
quatro, cinco, oito e onze.

\section{A oferta adicional do ano novo}

Por este preceito somos ordenados a oferecer uma Oferenda Adi­cional no
primeiro dia de ``Tishri'', que é a Oferta Adicional do Ano Novo. Este
preceito está expresso em Suas palavras, enaltecido seja Ele, "E no
sétimo mês, no primeiro dia do mês... e oferecereis como holocausto,
para ser aceito com agrado pelo Eterno" (Números 29:1-2).

\section{A oferta adicional do décimo dia de ``tishri''}

Por este preceito somos ordenados a oferecer uma Oferenda Adi­cional no
décimo dia de ``Tishri''. Ele está expresso em Suas palavras, enalteci­do
seja Ele, "E no décimo dia, deste sétimo mês... E oferecereis holocausto
ao Eterno, para ser aceito com agrado" (Números 29:7-8).

\section{O ofício de ``yom quipur''}

Por este preceito somos ordenados a celebrar o Ofício do Dia, ou seja,
todos os sacrifícios e profissões de fé ordenados pelas Escrituras para
o Dia do Perdão, para expiar todos os nossos pecados. Esta é a instrução
que está expressa na porção ``Aharé Mot'' (Levítico 16:1-34).



A prova de que ela constitui, em sua totalidade, apenas um preceito se
encontra no final do quinto capítulo de Quipurim: "Com relação a cada
ce­lebração de `Yom Quipur', mencionada na ordem prescrita, se algum
ofício for celebrado fora da ordem estabelecida é como se nenhum deles
tivesse sido ce­lebrado."

Todas as normas deste preceito estão explicadas no Tratado dedica­do
exclusivamente a este assunto, que é o Tratado de Yoma.

\section{A oferta adicional da festa dos tabernáculos}

Por este preceito somos ordenados a oferecer uma Oferenda Adi­cional no
Festival dos Tabernáculos. Ele está expresso em Suas palavras,
enal­tecido seja Ele, "E oferecereis por holocausto, oferta queimada
para ser aceita com agrado pelo Eterno" (Números 29:13). Esta é a Oferta
Adicional dos Ta­bernáculos.

\section{A oferta adicional de ``shemini atzeret''}

Por este preceito so cional no oitavo dia da Festa oitavo Dia da
Assembléia sol

O que nos faz consi ta em separado, diferente das oferecidas diariamente
durante o Festival dos Ta­bernáculos, é o princípio aceito que o oitavo
Dia da Assembléia Solene é, por si só, um outro Festival. Os Sábios
dizem claramente: "Ele é um Festival separa­do, com ofertas separadas".
Isto prova que sua oferenda é específica, deixando assim o assunto
perfeitamente claro.

\section{As três peregrinações anuais}

Por este preceito somos ordenados subir ao Santuário três vezes por ano.
Ele está expresso em Suas palavras, e o seja Ele, "Três vezes cele-

brarás, pa im, festas no ano" (Êxodo 2 Escrituras deixam claro que

``subir'' ri ica ir até lá com uma oferen te preceito está repetido várias
veze .

„ o Sifrei • "Três preceitos em ser obedecidos num Festi-

val, a sabèr: estejar, recer diante do Eterno, e alegrar-se". A Guemará

de Haguigá também Três preceitos são impostos a Israel num Festival:

festejar, comparecer diante do Eterno e alegrar-se". ``Festejar''
significa levar uma Oferta de Paz, o que não é obrigatório para as
mulheres.

As normas deste preceito estão explicadas no Tratado Haguigá.


\begin{enumerate}
\def\labelenumi{\arabic{enumi}.}
\setcounter{enumi}{66}
\item
 
 Números 29:35-38.
 
\item
 
 Deuteronômio 16:16.
 
\item
 
 Ibid., 15; Êxodo 34:23.
 
\item
 
 Haguigá 6:B.
 
\end{enumerate}

53 COMPARECER DIANTE D TERNO DURANTE OS FESTIVAIS

Por este preceito somos ordenados a compar cer\textsuperscript{71} du
ante os Fes­tivais. Está expresso em Suas palavras, enaltecido seja Ele,
"Três ezes ao ano, aparecerão todos os teus homens diante do Eterno, teu
Deu " euteronômio 16:16). O significado deste preceito é que todo homem
deve subir ao Santuá­rio com todos os seus filhos homens que possam
andar sozinhos e oferecer um holocausto quando subir. Este é chamado o
Holocausto do Comparecimento. Nós já nos referimos às palavras dos
Sábios: "Três preceitos são obrigatórios durante um festival: festejar,
comparecer e alegrar-se".

As normas deste preceito, ou seja, o Preceito do Comparecimento também
estão expostas no Tratado Haguigá. Este preceito também não é
obri­gatório para as mulheres.

\section{Alegrar-se nos festivais}

Por este preceito somos ordenados a alegrar-nos nos Festivais. Está
expresso em Suas palavras, enaltecido seja Ele, "E alegrar-te-ás na tua
festa" (Deuteronômio 16:14). Este é o terceiro dos três preceitos
observados num festival.

A obrigação mais importante imposta por este preceito é a das Ofer­tas
de Paz obrigatórias. Essas são outras Ofertas de Paz do Festival
adicionais às ofertas de Haguigá e são chamadas no Talmud de "Ofertas de
Paz da Alegria".

Com relação a essas Ofertas de Paz nos é dito o seguinte: "As mulhe­res
têm a obrigação de tomar parte na alegria". Como está nas Escrituras: "E
sacrificarás ofertas de pazes e comerás ali; e te alegrarás diante do
Eterno, teu Deus" (Deuteronômio 27:7). As normas deste preceito também
estão expostas no Tratado Haguigá.

As palavras "E alegrar-te-ás na tua festa" incluem o preceito dos
Sá­bios de que devemos alegrar-nos de todas as maneiras possíveis, como
comen­do carne nos festivais, bebendo vinho, vestindo roupas novas,
distribuindo frutas e doces às crianças e mulheres, e alegrando-nos com
instrumentos musicais e dançando no Santuário especificamente, sendo que
essa será a Alegria de ``Beit Hashoeba''. Todos esses tipos de regozijo
estão compreendidos em Suas pala­vras "E alegrar-te-ás na tua festa".

Dentre as maneir egrar-se, a mais obrigatória é a de beber vi-

nho, porque ela está especi ligada à alegria.

Como diz a Gue Um homem tem o dever de fazer com que

seus filhos e sua família se ale rance um Festival... De que maneira?
Com

vinho".

Diz ainda, mais adian : "Foi-nos ensinado: Rabi Yehudá ben Bete­ra diz
que quando o Santuário existia não podia haver outro tipo de regozijo a
não ser o de comer carne, como está escrito: 'E sacrif' 's ofertas de
pa-

zes...'. Mas agora, que o Santuário não mais existe, só há ijo com
vinho,

como está dito 'O vinho faz alegre o coração do hom '. Diz também:


\begin{enumerate}
\def\labelenumi{\arabic{enumi}.}
\setcounter{enumi}{70}
\item
 
 Ao Santuário Sagrado.
 
\item
 
 Pessahim 109:A.
 
\item
 
 Salmos 104:15.
 
\end{enumerate}



"Os hom•maneira apropriada a eles, e as mulheres de maneira apro­priada
a

A rah nos obriga a incluir nesse regozijo os pobres, os necessita­dos e
os estranhos. Suas palavras, enaltecido seja Ele, são: "E alegrar-te-ás
dian­te do Eterno, teu Deus, tu... e o peregrino, e o órfão, e a viúva"
(Deuteronômio 16:11).


\section{Abater a oferta de ``pessah''}


Por este preceito somos ordenados a sacrificar o cordeiro de "Pes­sah"
no décimo quarto dia de Nissan. Está expresso em Suas palavras,
enalteci­do seja Ele, "E o degolará toda a assembléia da congregação de
Israel, pela tar­de" (Êxodo 12:6). Aquele que não cumprir este preceito
e deliberadamente ne­gligenciar a oferenda .deste sacrifício no momento
determinado está sujeito à extinção, seja homem ou mulher, uma vez que
está claramente expresso na Gue­mará de Pessahim que a primeira Oferta
de ``Pessah'' é obrigatória para as mu­lheres da mesma forma que para todo
homem do povo de Israel, e que essa oferta tem prioridade sobre o
Shabat, ou seja, que ela deve ser oferecida no décimo quarto dia de
Nissan, mesmo que esse dia seja Shabat.

A pena de extinção está prescrita em Suas palavras "E o homem que está
puro, e não estiver em viagem, e deixar de fazer o `Pessah', essa alma
será banida de seu povo" (Números 9:13).

Na enumeração dos preceitos --- negativos --- cuja transgressão in­corre
na penalidade de extinção, no início do Tratado Queretot, estão
incluí­dos os preceitos positivos de ``Pessah'' e da circuncisão, como foi
mencionado na introdução.


As regras detalhadas deste preceito estão expostas no Tratado Pessahim.


\section{Comer a oferta de ``pessah''}

Por este preceito somos ordenados a comer a Oferta de ``Pessah'' na décima
quinta noite de Nissan, de acordo com as condições especificadas, ou
seja, ela deve ser grelhada, deve ser comida numa casa, e deve ser
comida com pão ázimo e ervas amargas. Ele está expresso em Suas
palavras, enaltecido seja Ele, "E comerão a carne nesta noite, grelhada
no fogo, e pães ázimos, com ervas amargas comerão" (Êxodo 12:8).

Se alguém perguntar: "Por que você conta comer a Oferta de "Pes­sah", o
pão ázimo e as ervas amargas como um único preceito, e não como três,
uma vez qu omer pão ázimo é um preceito, comer ervas amargas é ou­tro e
comer a me da Oferta de ``Pessah'' é outro?", eu direi que é verdade que
comer ão zimo é um preceito por si só, como explicarei posteriormen
e\textsuperscript{75}; d mesma forma, comer a carne da Oferta de
``Pessah'' tam­bém é por si sou preceito, como já foi mencionado. Mas
comer ervas amar­gas está colocado como dependente de comer a Oferta de
``Pessah'', e não de­ve ser contado como um preceito separado. Isto está
provado pelo fato de que se deve comer a carne da Oferta de ``Pessah'' em
cumprimento ao preceito, quer haja ervas amargas disponíveis oú não, mas
não se comem ervas amargas


\begin{enumerate}
\def\labelenumi{\arabic{enumi}.}
\setcounter{enumi}{73}
\item
 
 Os homens devem alegrar-se de maneira apropriada a eles.
 
\item
 
 Ver o preceito positivo 158.
 
\end{enumerate}

a não ser com a carne da Oferta de ``Pessah'', porque Ele diz, enaltecido
seja Ele, ``Com pães ázimos e ervas amargas comerá o sacrifício'' (Números
9:11). Ao comer ervas amargas sem carne não se cumpre nenhuma obrigação,
e não se pode dizer que se cumpre o preceito de comer ervas amargas. Nas
palavras de Mekhiltá: " 'Grelhada no fogo, e pães ázimos, com ervas
amargas comerão'; isso nos ensina que o preceito referente à Oferta de
``Pessah'' estipula que ela deve ser comida grelhada, com pão ázimo e
ervas amargas", ou seja, o preceito consiste na totalidade destes três.

Também está escrito: "De que modo você conclui que na ausência do pão
ázimo e das ervas amargas a obrigação pode ser cumprida comendo-se
apenas a Oferta de ``Pessah''? Pelas palavras: 'A comerão' ", isto é, a
carne ape­nas. Também se poderia pensar que, da mesma forma que na
ausência de pão ázimo e de ervas amargas, a obrigação pode ser cumprida
comendo-se a Oferta de ``Pessah'', assim também na falta da Oferta da
``Pessah'' a obrigação pode ser cumprida comendo-se pão ázimo e ervas
amargas, argumentando-se que uma vez que comer a Oferta de ``Pessah'' é um
preceito positivo, e comer pão ázi­mo e ervas amargas também é um
preceito positivo, se na ausência de pão ázi­mo e ervas amargas a
obrigação pode ser cump comendo-se apenas a Ofer­ta de ``Pessah'', então
deveria concluir-se qu na lta da Oferta de ``Pessah'' a obrigação pode ser
cumprida comendo-se p o ázi o e ervas amargas. As Es­crituras dizem, a
esse respeito: 'A comerão "\textsuperscript{76}.

Está também escrito: " 'A comerão' dest s palavras deve-se concluir que
a Oferta de ``Pessah'' deve ser comida até sa edade, enquanto que o pão
ázimo e as ervas amargas devem ser comidos antes que o estado de
saciedade seja atingido", uma vez que a essência do preceito consiste em
comer a carne, como Ele disse: ``Comerão a carne nesta noite'', enquanto
que comer as ervas amargas é uma obrigação complementar a comer a carne,
como estas citações deixam claro para os que as compreendem.

Uma prova evidente da exatidão de nosso parecer está numa afirma­ção do
Talmud: "Comer ervas amargas hoje em dia é apenas uma regra
estabe­lecida pelos Rabinos" porque no que diz respeito à Torah, não há
obrigatorie­dade de comê-las sozinhas, mas elas devem ser comidas com a
carne da Oferta de ``Pessah''. Isto constitui prova clara e evidente de
que elas são acessórias ao preceito e não constituem um preceito
individual.


As normas deste preceito também estão expostas no Tratado Pessahim.

\section{Abater a segunda oferta de ``pessah''}


Por este preceito aquele que não tenha podido oferecer a primeira Oferta
de ``Pessah'' é ordenado a abater a Segunda Oferta de ``Pessah''. Este
preceito está expresso em Suas palavras, enaltecido seja Ele, "No
segundo mês, aos 14 dias do mês, pela tare • . celebrará" (Números
9:11).


Aqui novamente s.f. \textbf{•} ia objetar e perguntar: "Por que você
con-


ta a segunda Oferta de "Pessa i ransgredindo dessa forma a regra que vo-

cê estabeleceu no Sétimo Fun. 1. o de que uma lei
pértencente a um preceito



Ou seja, ``comerão'' a Oferta de ``Pessah''.
Como um preceito diferente.


não é por si só considerada como um preceito separado? A pessoa que
assim argumentar deve saber que os Sábios sustentaram diferentes
opiniões sobre a questão da Segunda Oferta de ``Pessah'' ser considerada
como uma continua­ção da primeira ou como um preceito ind endente e a
conclusão a que se che­gou foi que essa obrigação é um prec o di erente
e que, consequentemente, deve ser e . . •s erado em separado.


A G k emará de Pessahim di 78: " a opinião de Rabi, fica-se sujeito

gunda. as Ra PA Nathan diz: fica-se sujeito à extinção por causa da
primeira,\\
à extin .•\textsuperscript{79} po , causa da primeira
e fi a- sujeito à extinção por causa da se-


4\textsubscript{,1} se-

mas não • . segunda; e Rabi Hananyá ben Akabya diz: Não se fica sujeito
à extinção nem por causa da primeira, a menos que se deixe de levar a
segunda"

A Guemará passa a perguntar: "Em que eles diferem?". E respon : "Rabi
sustenta que o segundo é um Festival em separado, e Rabi Nathan sgst
n-ta que ele é apenas complementar ao primeiro". Isso explica nossa
afirma` 80

A Guemará diz ainda, no mesmo trecho: "De acordo com isa , se alguém
deliberadamente negligenciar ambas", ou seja, se alguém delibera mente
deixar de levar tanto a Primeira Oferta de ``Pessah'' quanto a Segunda,
"todos concordam que ele é culpado. Se ele negligenciar as duas
involuntaria­mente, todos concordam que ele não é culpado. Se ele
negligenciar a primeira intencionalmente e a segunda involuntariamente,
ele é culpado e está sujeito à extinção, de acordo com o Rabi e com Rabi
Nathan, é inocente e não está sujeito à punição, de acordo com Rabi
Hananyá ben Akabya. Da mesma forma, se seu erro for intencional no caso
da primeira, e ele trouxer a oferenda na se­gunda, ele é culpado, de
acordo com o Rabi", porque na sua opinião o segun­do ``Pessah'' não é mer
mente complementar ao primeiro, e a lei segue, em tod ses casos, o r do
Rabi.

Este prec é obrigatório para as mulheres porque está expli-

c' do\textsuperscript{81} ue a Segun•pcional para as mulheres.

As norma preceito estão explicadas na Guemará de Pessahim.

58 COMER A SEGUNDA OFERTA DE ``PESSAH''

Por este preceito somos ordenados a comer a carne da segunda Oferta de
``Pessah'' na noite do décimo quinto dia de Iyar juntamente com pão ázimo
e ervas amargas. Ele está expresso em Suas palavras, enaltecido seja
Ele, refe­rentes a esta também: "A comerão com pão ázimo e ervas
amargas" (Números 9: 1 1)


As normas detalhadas deste preceito também estão expostas em Pes-


sahim.

É óbvio que este preceito não é obrigatório para as mulheres já que elas
não são obrigadas a fazer o abatimento, como explicamos, portanto não
resta dúvida de que elas não são obrigadas a comer esta oferenda.


\begin{enumerate}
\def\labelenumi{\arabic{enumi}.}
\setcounter{enumi}{77}
\item
 
 Pessahim 93:A.
 
\item
 
 Punição à pessoa que não abater a oferta de ``Pessah''.
 
\item
 
 De que os Sábios diferem quanto à classificação da Segunda Oferta de
 ``Pessah''.
 
\item
 
 No Tratado Pessahim.
 
\item
 
 A Segunda Oferta de ``Pessah''.
 
\end{enumerate}


\section{Tocar as cornetas no santuário}


Por este preceito somos ordenados a fazer soar as cometas no San­tuário
ao oferecer qualquer um dos sacrifícios sazonais, e ele está expresso em
Suas palavras, enaltecido seja Ele, "E também no dia de vossa alegria,
nas vos­sas solenidades fixas, e nos princípios de vossos meses,
tocareis as cometas so­bre vossos holocaustos etc." (Números 10:10). Os
Sábios dizem explicitamen­te que este é o preceito das cometas.

As normas deste prèceito estão explicadas no Sifrei, em Rosh Hashaná e
Taaniot, uma vez que somos ordenados a tocar as cometas em épocas de
difi­culdades e infortúnios, quando clamamos pelo Eterno, enaltecido
seja Ele, de acor­do com Suas palavras "E quando estiverdes em guerra em
vossa terra, contra o adversário, que vos oprime, tocareis retinindo o
Shofar' " (Números 10:9).

\section{Oferecer gado com idade mínima determinada}

Por este preceito somos ordenados a que todo gado que trouxer­mos como
oferenda tenha 8 dias ou mais de idade, e não menos. Este é o pre­ceito
da oferenda cujo momento de ser aceita ainda não chegou por motivos
físicos e ele está expresso em Suas palavras, enaltecido seja Ele,
``Ficarão por sete dias atrás de sua mãe'' (Levítico 22:27).

Este preceito também nos é dado de uma outra forma: "Sete dias estará
com sua mãe" (Êxodo 22:29). Isto se aplica a todas as oferendas de todos
os tipos, privadas e públicas.

Das palavras "E do oitavo dia em diante serão aceitos por sacrifício,
como oferta queimada ao Eterno" (Levítico 22:27) concluímos que antes
disso eles não seriam aceitáveis. Assim, está claramente proibido
oferecer um animal que não tenha atingido idade para ser aceito; mas
como este é um preceito ne­gativo derivado de um preceito positivo sua
transgressão não acarreta a pena de flagelo, e aquele que trouxer um que
não tenha alcançado a idade certa não será açoitado, como está explicado
no capítulo ``Ele e seus filhotes'', onde tam­bém se lê: "Desconsidere a
oferenda cuja época ainda não tenha chegado, pois a Escritura a justapôs
a um preceito positivo".

As normas deste preceito estão explicadas na Sifrá, no final do Tra­tado
Zebahim.

\section{Oferecer apenas sacrifícios perfeitos}

Por este preceito somos ordenados a oferecer ao Eterno apenas espé­cimes
perfeitos, sem os defeitos mencionados nas Escrituras, e livres de to as
imperfeições consideradas como defeitos pela Tradição. Este preceito est
presso em Suas palavras, enaltecido seja Ele, "Estes deverão ser sem
defe't ra que sejam aceitos" (Levítico 22:21), sobre as quais a Sifrá
diz: " 'Este rão ser sem defeitos para que sejam aceitos': este é um
preceito positi

83. Ver também preceitos negativos 91 a 96.



As palavras "Ser-vos-ão eles sem defeito, igualmente as suas libações"
(Números 28:31) foram citadas como sendo a prova de que os vinhos das
liba­ções e seus óleos e farinha fina devem ser absolutamente perfeitos
e sem qual­quer tipo de defeito.


As normas deste preceito estão explicadas no oitavo capítulo de


Menahot.

\section{Levar sal com c sacrifício}

Por este preceito somos orden..os a oferecer sal com cada sacrifí­cio.
Ele está expresso em Suas palavras, enaltecido seja Ele, "E toda tua
oferta de oblação temperarás com sal" (Levítico
2:13).\textsuperscript{84}


As normas deste preceito estão explicadas na Sifrá e em Menahot.


\section{O holocausto}

Por este preceito somos ordenados quanto ao procedimento a se­guir ao
oferecermos o Holocausto. Ou seja, todo o Holocausto, seja ele uma
oferta privada ou pública, deve ser oferecido de uma maneira
pré-estabelecida. Este preceito está expresso em Suas palavras no
Levítico, "Quando algum de vós oferecer sacrifício ao Eterno ... Se seu
sacrifício for holocausto de gado" etc (Levítico 1:2-3).

\section{O sacrifício de pecado}

Por este preceito somos ordenados a oferecer o Sacrifício de Peca­do,
seja ele de que tipo for, da maneira especificada. Este preceito está
expres­so em Suas palavras, enaltecido seja Ele, "Esta é a lei do
sacrifício de pecado" etc. (Levítico 6:18)

No Levítico está explicado também de que forma o sacrifício deve ser
oferecido, que parte dele deve ser queimada e que parte deve ser comida.

\section{O sacrifício de delito}

Por este preceito somos ordenados a oferecer o Sacrifício de Delito de
uma determinada maneira. Este preceito está expresso em Suas palavras,
enal­tecido seja Ele, no Levítico: "E esta é a lei do sacrifício de
delito" etc. (Leví­tico 7:1).

As Escrituras explicam como este sacrifício deve ser oferecido, que
parte dele deve ser queimada e que parte deve ser comida.

\section{O sacrifício de paz}

Por este preceito somos ordenados a oferecer o Sacrifício de Paz da
forma especificada. Este preceito está expresso em Suas palavras "E se
sacrifí­cio de pazes é sua oferta" etc. (Levítico 3:1) e "É esta a lei
do sacrifício de pazes ... Se por ação de graças a oferecer" etc
(Levítico 7:11-12).

84. Ver também o preceito negativo 99.

Esses quatro rituais --- o Holocausto, o Sacrifício de Pecado, o
Sacri­fício de Delito e o Sacrifício de Paz --- compõem todo o ritual
dos sacrifícios, uma vez que todas as ofertas de animais, sejam elas
trazidas por um indivíduo ou pela congregação, pertencem a uma dessas
quatro categorias, embora o sa­crifício de Delito seja sempre uma oferta
individual, como explicamos em di­versas ocasiões.

O Tratado Zebahim co • em normas destes quatro preceitos e as­suntos
relacionados a eles, e explica quai são as cerimônias obrigatórias, o
que não pode ser feito sem infringir a, lei \textsuperscript{85}, que
invalida um sacrifício e qual é o procedimento correto.

\section{A oblação}

Por este preceito somos ordenados a oferecer a Oblação de acordo com as
regras estipuladas para cada um de seus vários tipos. Este preceito está
expresso em Suas palavras "E quando uma alma oferecer uma oblação ao
Eter­no" etc. (Levítico 2:1); "E se tua oferta for oblação feita na
assadeira" etc (Ibid., 5); "E se oblação de panela é tua oferta," etc.
(Ibid., 7), que são complementa­das pelas seguintes palavras,
encontradas mais adiante: "E esta é a lei da obla­ção" (Ibid., 6:7).

As normas deste preceito, com seus vários aspectos, estão explica­das no
Tratado Menahot, que se dedica especificamente a este assunto.

\section{O sacrifício d um tribunal}


QUE COMETE


Por este preceito o Tribunal so ele tenha tomado uma decisão e

Este preceito está express se a congregação de Israel pecar por da
assembléia" etc. (Levítico 4:13).

As normas e condições deste preceito estão todas explicadas no Tra­tado
Horayot e em vários trechos do Tratado Zebahim.

\section{O sacrifício de pecado}


tenha cometido involuntariamen­rdenado a oferecer um Sacrifício as,
enaltecido seja Ele, "E se uma . (Levítico 4:27).


Esta oferta é um Sacrifício elecido de Pecado, ou seja, um Sa-

crifício de Pecado que deve ser constituído de um animal.

Já explicamos que os pecados passíveis da penalidade de Sacrifício de
Pecado são os mesmos que acarretam a penalidade de extinção, caso sejam


\begin{enumerate}
\def\labelenumi{\arabic{enumi}.}
\setcounter{enumi}{84}
\item
 
 E se a lei for infringida isso implica em punição.
 
\item
 
 Que tenha levado a congregação a cometer um pecado.
 
\item
 
 Ver Mishné Torah Hilchot Shegagot, 1? capítulo, 4! Halachá.
 
\end{enumerate}




cometidos deliberadamente, desde que o pecado implique na violação de um
preceito negativo e que haja algum ato relacionado a ele, como explicado
no início do Tratado Queretot.

As normas deste preceito estão explicadas nos Tratados Horayot e
Queretot, e em vários trechos de Shabat, Shabuot e Zebahim.

\section{O sacrifício suspensivo de delito}

Por este preceito somos ordenados a oferecer um sacrifício determi­nado
em caso de dúvida quanto a um dos pecados capitais que implicam na
penalidade de extinção, se cometidos voluntariamente, e em Sacrifício
estabe­lecido de Pecado, se cometidos involuntariamente. Este sacrifício
é chamado

de Sacrifício Suspensivo de Delito. Um exemplo de um caso de • . que im-

plicar* Sacrifício Suspensivo de Delito é o seguinte: su•nha que uma

pes a tenh diante de si dois pedaços de gordura, uma de r

\textbf{.} e outra de coOção\textsuperscript{89}. e
come um dos dois pedaços e o outro é comido • o outra pes­soá, ou per d
ido. Uma dúvida surge então em sua mente quanto a se o pedaço de go a
que ele comeu foi o permitido ou o proibido. Nesse caso, ele deve
oferecer um sacrifício expiatório, pela dúvida que surgiu, chamado
Sacrifício Suspensivo de Delito. Se posteriormente ele se certificar que
o pedaço de gor­dura que comeu era o de rins, fica confirmado que pecou
involuntariamente e ele deverá então oferecer um Sacrifício Estabelecido
de Pecado.

O versículo referente a esta oferenda está no livro de Levítico: "E se
alguma alma pecar e fizer um dos preceitos do Eterno, daqueles que não
se devem fazer, e não souber, e for culpado, levará a sua iniqüidade. E
trará do rebanho, um carneiro sem defeito, no valor de dois siclos, por
sacrifício de de­lito, ao 'Cohen'; e expiará por ele, o 'Cohen', pelo
erro que cometeu sem sa­ber" (Levítico 5:17-18), ou seja, porque ele não
sabia se tinha pecado ou não. Isto é o que os Sábios chamam de Pecado
Cometido Involuntariamente.


As normas deste preceito estão explicadas no Tratado Queretot.


\section{O sacrifício incondicional de delito}

Por este preceito aquele que cometer certas transgressões é ordena­do a
oferecer um Sacrifício de Delito para obter o perdão. Essa oferenda é
cha­mada de Sacrifício Incondicional de Delito.

As transgressões que requerem este sacrifício são: sacrilégio, roubo,
manter uma ligação com uma escrava prometida em casamento e jurar em
falso no caso de algo entregue sob custódia. Aquele que cometer um
sacrilégio sem intenção, ou seja, que usufruir de uma ``perutá'' com algo
que pertença ao San­tuário, seja dos Objetos Consagrados do tesouro do
Santuário ou dos do Altar; aquele que roubar uma ``perutá'' ou mais de um
amigo e fizer juramento falso a respeito; e aquele que mantiver uma
ligação com uma escrava comprometi­da, seja sem intenção ou
voluntariamente, todos eles terão a obrigação de tra-


\begin{enumerate}
\def\labelenumi{\arabic{enumi}.}
\setcounter{enumi}{87}
\item
 
 A gordura dos rins não é ``Casher''.
 
\item
 
 A gordura do coração é ``Casher''.
 
\end{enumerate}

zer uma oferenda por seus pecados --- não um Sacrifício de Pecado, mas
sim um Sacrifício de Delito chamado Sacrifício Incondicional de Delito.

Com relação ao sacrilégio, Suas palavras, enaltecido seja Ele, são: "E
pec ro nas santidades do Eterno, trará por seu delito" etc. (Levítico
5:15).

E dis E negar ao seu companheiro a coisa que lhe

tódia.. rou em falso ... e como oferta de delito trará

um carneiro sem defeito" etc. (Ibid., 21-22, 25). E d se deitar com uma
mulher ... e ela for escrava despo•a

\section{O sacrifício de maior valor ou de menor valor}


ferecer um Sacrifício de Maior sgressões.


rifício são: impurificar o San-jurar em falso em relação a mpuro por
alguma das fontes trodução à Ordem de Teharot, entrar sem querer no
Santuário, o que implicaria em impurificação do Santuá­rio; se alguém
sem querer comer carne sagrada, o que implicaria em impurifi­cação dos
Objetos Santificados do Santuário; se alguém pronunciar um jura­mento e
involuntariamente deixar de cumpri-lo; se alguém, jurar em falso em
relação a um testemunho, seja sem querer ou intencionalmente, em todos
es­ses casos ele terá que oferecer o que chamamos de um Sacrifício de
Maior ou Menor Valor.

Este preceito está expresso em Suas palavras, enaltecido seja Ele, "E
quando alguma alma pecar, sob juramento ...; se alguma alma tocar em
alguma coisa impura... e lhe for oculto que estava impuro e o souber
depois, e se tor­nar culpado; ou quando alguma alma jurar, pronunciando
com os lábios... e lhe for oculto e o souber depois que foi culpado de
uma dessas coisas ... trará como sacrifício ... E se as suas posses não
lhe permitirem trazer ..." etc. (Levíti­co 5:1-11).

Este sacrifício é chamado um Sacrifício de Maior ou Menor Valor por­que
ele não está especificado, variando de acordo com as posses do
transgres­sor que tem que oferecê-lo.

As normas deste preceito também estão explicadas nos Tratados Que­retot
e Shabuot.

\section{Confessar}

Por este preceito somos ordenados a fazer a confissão oral dos peca­dos
que tivermos cometido contra o Eterno, enaltecido seja Ele, depois de
nos termos arrependido deles. Esta é a maneira de fazer a confissão:
`Oh, Deus,


\begin{enumerate}
\def\labelenumi{\arabic{enumi}.}
\setcounter{enumi}{89}
\item
 
 Com relação ao roubo e falso testemunho em caso de algo entregue sob
 custódia.
 
\item
 
 Com relação a manter uma ligação com uma escrava noiva.
 
\item
 
 Ou seja, alguém perjura que não pode testemunhar quando na realidade
 pode.
 
\end{enumerate}




eu pequei, eu cometi injustiça, eu infringi ...". Deve-se elaborar a
confissão e pedir perdão com toda a eloqüência de que se for capaz.

Deve-se notar que mesmo no caso de pecados pelos quais se deve oferecer
um dos sacrifícios anteriormente especificados, a fim de obter-se o
per­dão prometido pelo Eterno, é preciso confessar-se no momento da
oferta. Isso se depreende de Suas palavras, enaltecido seja Ele, "Fala
aos filhos de Israel: Quando homem ou mulher fizer algum dos pecados do
homem ... E confes­sará os seus pecados que cometerá" (Números 5:6-7).
Comentando esse versí­culo, a Mekhiltá diz: "Uma vez que está escrito:
'Confessará aquilo em que pe­cou' (Levítico 5:5), deduzimos que ele deve
confessar 'junto com o pecado' que cometeu, ou seja, junto ao sacrifício
de pecado enquanto o mesmo ainda estiver vivo, e não depois de abatido.
O versículo não nos diz que o indivíduo deve confessar-se por qualquer
pecado a não ser pelo de entrar no Santuário em estado de impureza".
Isso é assim porque o versículo ``Confessará aquilo em que pecou'' aparece
no trecho ``Vayikra'' das Escrituras que se refere à im­purificação do
Santuário e dos Objetos Santificados, bem como às transgres­sões
mencionadas nele, como explicamos. É por isso que a Mekhiltá diz que
desse versículo se pode deduzir a obrigação de confessar-se apenas no
caso de se ter impurificado o Santuário. "De que modo você conclui sua
aplicação em caso de violação de qualquer um dos outros preceitos? Pelas
palavras das Escri­turas: 'Fala aos filhos de Israel ... E confessará'.
De que modo você conclui que a obrigação se aplica aos casos de pena de
extinção ou morte? Pelas palavras `Seus pecados', ou seja, todos os seus
pecados, estendendo a aplicação aos preceitos negativos. 'Que cometerá'
abrange as transgressões dos preceitos positivos.

A Mekhiltá diz, no mesmo trecho: " 'Algum dos pecados do homem': isto
significa transgressão dos preceitos relacionados com outros homens, co
mo roubo, assalto e maledicência. 'Por falsear em nome do Eterno': isso
inc


ma': isso estende a ri­to à pena de mo e\textsuperscript{93}. estão para
ser c nde­Escrituras dizem:

atória para aquele que

sabe c, ue não cometeu nenhum delito e contra • qual foi prestado falso
testemunho".

Assim, fica claro que em todos os tipos de pecado, sejam eles gran­des
ou pequenos, inclusive os casos de transgressões dos preceitos
positivos, acarr a obrigação de confessar-se.

Como o preceito ``Confessará'' só está mencionado com relação à
\textsubscript{1}6 a rigaçã • de oferecer um sacrifício, poderia
ocorrer-nos a idéia de que a con­ssão não é uma obrigação em si, mas
apenas algo acessório ao sacrifício. Por sso\textsuperscript{95} fo m
obrigados a explicar isso na Mekhiltá da seguinte forma: "Poder­e-ia p
sar que a confissão é necessária quando se oferece um sacrifício; de


aneira sabemos que ela é necessária mesmo quando não há uma oferen­da?
Pelas palavras das Escrituras: 'Fala aos filhos de Israel: ... e
confessará'. Poder-se-ia ainda pensar que a confissão só é obrigatória
na Terra de Israel; de que modo se conclui que ela também é obrigatória
na Diásp9ra? Pelo versículo 'E


\begin{enumerate}
\def\labelenumi{\arabic{enumi}.}
\setcounter{enumi}{92}
\item
 
 Por sentença judicial.
 
\item
 
 Ver preceito positivo 180.
 
\item
 
 Os Sábios.
 
\end{enumerate}

confessarão a sua iniquidade, e a iniquidade d ais' (Levítico
26:40)\textsuperscript{96}.

niel disse: 'A ti, ó Eterno, pertence a justiça'

Fica assim claro, por tudo o que dis • s, que a confissão é por si

só uma obrigação independente e é obrigatória ao transgressor por cada
peca­do que ele cometer, seja na Terra de Israel ou fora dela, quer ele
tenha ofereci­do um sacrifício ou não. Em todos os casos ele fica
obrigado a confessar, de acordo com Suas palavras, enaltecido seja Ele,
``E confessará os seus pecados''.

A Sifrá diz também: " 'E manifestará' (Levítico 16:21): isso significa
confissão oral".


As normas deste preceito estão explicadas no último capítulo de Qui-


purim.


\section{A oferenda ada por um ``zab''}


Por este preceito um " ordenado a levar, ao se curar, um

sacrifício de "duas rolas ou dois po os, um por holocausto e o outro por

sacrifício de pecado". Este é o sacrifíc do ``zab'', cujo perdão não é
alcança­do até que ele o leve. Este preceito está expresso em Suas
palavras, enaltecido seja Ele, "E quando estiver limpo de seu fluxo quem
o tiver... E no oitavo dia tomará para si duas rolas" etc. (Levítico
15:13-14).

\section{A oferenda vada por uma "zab/v'}

Por este preceito uma "z *a"\textsuperscript{99} é • rdenada a levar, ao
se curar, uma oferenda de "duas rolas ou dois po *nho '. Este é o
sacrifício da ``zaba'', cu­jo perdão não é alcançado até que ela • eve.

Talvez se pudesse objetar o seguinte: "Uma vez que o sacrifício de um
``zab'' é o mesmo que o de uma ``zaba'' e que você leva em consideração
apenas o tipo de oferenda envolvido e não se preocupa com o tipo de
trans­gressor --- como acontece no caso do Sacrifício de Pecado, do
Sacrifício Incon­dicional de Delito, do Sacrifício Suspensivo de Delito
e do Sacrifício de Maior ou Menor Valor, onde você enumera cada um deles
como um preceito separa­do, sem se preocupar sobre os tipos de
transgressões pelas quais o sacrifício em questão pode ser requerido ---
por que você não adota o mesmo método neste caso e não deixa de lado o
fato de que são indivíduos diferentes, uma vez que deles se exige o
mesmo tipo de sacrifício?"

Quem usar essa argumentação deve saber que a oferenda do homem ou da
mulher que padecem de fluxo não é levada por causa do pecado, mas é
obrigatória em certas circunstâncias. Se a natureza do fluxo fosse
idêntica no homem e na mulher, como idêntico é o nome, um sendo chamado
de ``zab'' e a outra de ``zaba'', então seria apropriado contar-se os dois
preceitos como um só. Mas este não é realmente o caso, porque o que faz
com que o homem deva levar uma oferenda é a saída do sêmem, enquanto que
se a mulher tivesse algum tipo de emissão de sêmem, ela não seria uma
``zaba''; o que obriga a mu-


\begin{enumerate}
\def\labelenumi{\arabic{enumi}.}
\setcounter{enumi}{95}
\item
 
 Esse versículo se refere à época em que o povo está disperso.
 
\item
 
 Daniel, 9:7. Daniel viveu fora de Israel.
 
\end{enumerate}


cerviz quebrada, sendo cada um deles um preceito diferente, como foi
exposto.


\begin{enumerate}
\def\labelenumi{\arabic{enumi}.}
\setcounter{enumi}{97}
\item
 
 Homem que padece de fluxo.
 
\item
 
 Mulher que padece de fluxo após o período menstrual.
 
\end{enumerate}




lher a levar uma oferenda é um fluxo de sangue, mas um fluxo de sangue
no homem não o obriga ao sacrifício. A palavra ``fluxo'' significa apenas
``fluir'' mas o que flui não é necessariamente a mesma coisa. Os Sábios
dizem explicita­mente: "Um homem transmite a impureza através do sêmem,
e a mulher atra­vés do sangue".

A lei do ``zab'' e da ``zaba'' não é igual a lei do leproso e da leprosa.
Isso está claramente provado pelo que está dito em Queretot: "Há quatro
pes­soas cujo perdão fica pendente: o homem e a mulher com fluxo, a
mulher de­pois do parto, e o leproso". Você verifica que o ``zab'' e a
``zaba'' são contados como dois, porque o fluxo do homem é diferente do
fluxo da mulher, enquan­to qug o leproso e a leprosa não são contados
separadamente.

O versículo referente à oferenda da ``zaba'' é: "E se limpar-se de seu
fluxo... no oitavo dia, tomará para si duas rolas, ou dois pombinhos"
etc. (Leví­tico 15:28-29).

\section{O sacrifício depois do parto}

Por este preceito a mulher que tiver dado à luz a uma criança é
orde­nada a levar uma oferenda, a saber, um cordeiro de menos de um ano
de idade como Holocausto e uma pombinha ou uma rola como Sacrifício de
Pecado. Se ela for pobre, ela pode levar duas rolas ou dois pombinhos:
um como Holo­causto e o outro como Sacrifício de Pecado. Ela faz parte
daqueles cujo perdão só é alcançado depois de ter levado o sacrifício,
de acordo com o que Ele disse, enaltecido seja Ele: "E ao cumprirem-se
os dias de sua purificação, pelo filho ou pela filha, trará um cordeiro
de idade de um ano, por holocausto, e um pom­binho e uma rola por
sacrifício de pecado...". (Levítico 12:6)

\section{O sacrifício levado por um leproso}

ar de sua lepra, é ordenado usto, um Sacrifício de Peca-óleo. Se ele for
pobre, ele e e duas rolas ou dois pombi­nhos, um como Holocausto e o
outro como Sa rifício de Pecado. Ele é a quarta pessoa cujo perdão fica
pendente até que tenha levado seus sacrifícios. Este pre­ceito está
expresso em Suas palavras, enaltecido seja Ele, "E no oitavo dia to­mará
dois cordeiros sem defeito e uma ovelha da idade de um ano, sem
defei­to". (Levítico 14:10).

Pode-se perguntar: "Por que você não conta como um único pre­ceito a
obrigação imposta a todos aqueles cujo perdão fica pendente, uma vez que
a pendência do perdão é comum a todos eles? Se você fizesse isso, esse
seria um dos meios de purificar-se e você poderia dizer: 'Preceito tal é
o que obriga certas pessoas impuras, a saber, um `zab', uma `zaba', uma
mulher de­pois do parto e um leproso a levar um sacrifício para que sua
purificação possa ser considerada completa'. Assim como você conta a
obrigação de ser purifica­do por um 'mikvá' como um único preceito, sem
se preocupar com quem é o impuro ou de que tipo é a sua impureza, você
poderia, da mesma forma, ter

100. Medida líquida que corresponde a 506 cm. ou 0,24
kg.
contado o sacrifício daqueles cujo perdão fica pendente como um único
pre­ceito, sem preocupar-se com o tipo de impureza".

O Eterno sabe, e é minha testemunha, que isso seria perfeitamente válido
se o sacrifício obrigatório àqueles cujo perdão fica pendente fosse o
mes­mo em todos os casos, e nunca fosse alterado, como no caso de ser
purificado pela água, que é um tipo de purificação obrigatória igual
para to.. as pessoas impuras. Mas devido à diversidade de seus
sacrifícios somos orça.. s, como você vêem, a contar cada sacrifício
separadamente, porque ue co pleta a purificação num caso não é o mesmo
que em outro. Este a o 101 é igual ao da água de aspersão, ao da água do
``mikvá'' e ao das quatr c. as las quais o leproso é purificado. Esses são
três preceitos separados, embora eles todos se refiram à purificação de
pessoas impuras, como explicarei depois.

As normas detalhadas referentes às quatro pessoas cujo perdão fica
pendente e às suas oferendas estão explicadas global e detalhadamente no
pri­meiro e no segundo capítulos de Queretot, nos segundos capítulos de
Arakhin e Nezikin, no oitavo capítulo de Nazir, no final de Negaim, no
Tratado Kinim e em diversos trechos do Talmud, mas quase todas elas
estão nos trechos referidos.

\section{O dízimo do gado}

Por este preceito somos ordenados a separar o dízimo dos nossos animais
puros nascidos todo ano, oferecer sua banha e seu sangue e comer o resto
em Jerusalém. Este preceito está expresso em Suas palavras, enaltecido
seja Ele, "E todo o dízimo do gado e do rebanho, todo o que passar
debaixo da vara marcadora, o décimo será santidade ao Eterno". (Levítico
27:32). Esse é o Dízimo do Gado.


As normas deste preceito estão explica • no último capítulo de Bek-


horot. Lá também está explicado que este prec obrigatório também fora

da Terra de Israel e depois da destruição do . Essa é a lei estabelecida

na Torah. Mas para que ele não fosse comid esmo que sem defeito ---

uma vez que não temos Santuário --- os Sábi denaram que este preceito

fosse obrigatório apenas quando o Templo estivesse de pé, e que quando o
Tem­plo fosse reconstruído ele seria obrigatório tanto na Terra quanto
fora dela.


\section{Santificar o primogênito}


•

Por este preceito somos ordenados a santificar os primogênitos, ou seja,
a separá-los e deixá-los de lado para aquilo que deverá ser feito com
eles no momento exato. Ele está expresso em Suas palavras, enaltecido
seja Ele, "Con­sagra para Mim todo primogênito... no homem e no animal."
(Êxodo 13:2). A Torah explica que ``animal'' aqui significa apenas o gado,
os carneiros e os ju­mentos. Este preceito está repetido com relação aos
primogênitos dos animais puros em Suas palavras, enaltecido seja Ele,
``Todo o primogênito que nascer do teu gado'' etc. (Deuteronômio 15:19).

Esta lei do primogênito de animais puros estipula que ele deve ser dado
aos ``Cohanim'', que deverão oferecer sua banha e seu sangue e comer


\begin{enumerate}
\def\labelenumi{\arabic{enumi}.}
\setcounter{enumi}{100}
\item
 
 O caso das quatro pessoas cujo perdão fica pendente.
 
\item
 
 Fora de Jerusalém.
 
\end{enumerate}




o que restar de sua carne. As regras detalhadas deste preceito estão
largamente explicadas no Tratado Bekhorot.

No final do Tratado Halá está explicado que este preceito é obriga­tório
apenas na Terra de Israel. O Sifrei diz: "Poder-se-ia pensar que é
obrigató­rio levar seus primogênitos nascidos fora de Israel até a Terra
de Israel, por -isso as Escrituras dizem: 'E o comerás diante do Eterno,
teu Deus,... o dízimo de teu grão... e os primogênitos do teu gado e do
teu rebanho' ou seja, os primo­gênitos devem ser levados do mesmo lugar
de onde vem o dízimo do grão, e não do exterior, pois o grão não é
levado do exterior".

Fica, portanto, claro que este preceito só é obrigatório na Terra de
Israel. Contudo, um primogênito nascido no exterior, embora não precise
ser levado como sacrifício, deve ser comido somente se for defeituoso,
quer esteja o Templo erguido ou como está agora. Este preceito não é
obrigatório para os Levitas.

\section{Resgatar o primogênito}

Por este preceito somos ordenados a resgatar nossos filhos primogê­nitos
e dar o dinheiro do resgate ao ``Cohen''. Ele está expresso em Suas
palavras, enaltecido seja El , „


• é . ogênito de teus filhos dar-me-ás" (Êxodo 22:28).


A ma ei de "d o primogênito é explicada da seguinte forma: o primogênito
é r sgatado d. ``Cohen'', que o tem por direito e é tirado dele por cinco
"selai "103\textsubscript{.} O \textsubscript{preceito} r

. ceito está expresso em Suas palavras, enaltecido seja Ele, "Porém
esgatarás g s primogênitos do homem" (Números 18:15), sen­do este o
precei o • o Res:ate do filho primogênito.

Esta o rigação não se aplica às mulheres; é um dever do pai em rela­ção
ao seu filho, como está explicado em Kidushin. Todas as normas deste
pre­ceito estão explicadas em Bekhorot.

\section{Resgatar o primogênito de um jumento}


Por este preceito somos ordenados a resgatar o primogênito de um apenas
com um cordeiro --- se não o resgatarmos pelo seu valor e a dar o
cordeiro ao ``Cohen''. Este preceito está expresso em Suas


enaltecido seja Ele, "E todo que abrir a matriz da jumenta, remi-lo-ás
por eiro" (Êxodo 34:20 e 13:13).

As normas deste preceito estão explicadas no Tratado Bekhorot. Es­te
preceito também não se aplica aos Levitas.

\section{Quebrar a cerviz do primogênito de um jumento}

Por este preceito somos ordenados a quebrar a cerviz do primogê­nito de
uma jumenta, caso não se deseje resgatá-lo. Este preceito está expresso


\begin{enumerate}
\def\labelenumi{\arabic{enumi}.}
\setcounter{enumi}{102}
\item
 
 ``Selaim'' é o plural de ``sela'', uma espécie de moeda antiga.
 
\item
 
 A pessoa pode resgatar o jumento com dinheiro e dá-lo ao ``Cohen''.
 
\end{enumerate}

em Suas palavras, enaltecido seja Ele, "E se não o remires,
quebrar-lhe-ás a cer­viz" (Êxodo 13:13 e 34:20). As regras detalhadas
deste preceito também estão explicadas no Tratado Bekhorot.

Poderia ser-me feita a seguinte pergunta: "Por que você conta resga­tar
e quebrar sua cerviz como dois preceitos, e não como um, considerando a
quebra da cerviz como uma das regras detalhadas do preceito, de acordo
com o Sétimo Fundamento?".

O Eterno sabe, e é minha stem nha de que que pergunta teria razão se não
fosse pel prova e que eles são rados encontrada as guintes pala
\textsuperscript{rastos,} `O dever de r dever de quebr a ce z, e o dev r
d asamento levi at dever do `Halit ' "". u seja, da mesma forma que a vi
filhos tem direi o ao cas ento levirato ou ao ``Halitzá'', send sue o
casamen­to levirato é urh p'i o, como mencionado, e o ``Halitzá'' um
outro, assim também o primogênito de uma jumenta pode ser resgatado ou
ter sua cerviz quebrada, sendo cada um deles um preceito diferente, como
foi exposto.

\section{Levar os sacrifícios devidos durante o primeiro festival}


Por este preceito somos ordenados a executar todos os deveres a nós\\
impostos na chegada do primeiro dos três Festivais de forma que com a
passagem
de qualquer um dos três Festivais cada um de nós tenha levado todos
os\\
sacrifícios devidos. Este preceito está expresso em Suas palavras,
enaltecido seja\\
Ele, "E lá ireis. E levareis ali vossos holocaustos" (Deuteronômio
12:5-6). O significado
deste preceito é que quando fordes lá durante cada um dos três
Festivais
, tereis a obrigação de levar todos os sacrifícios que vos tenham
sido impostos.\\
A esse respeito diz o Sifrei: " `E lá ireis. E levareis ali'; por que
foi\\
dito isso? Para estabelecer a.obrigação de levar os sacrifícios no
início do primeiro
Festival". Também diz, no mesmo trecho: "Não se transgride `não
demorarás
em pagá-lo' (Deuteronômio 23:22) até que não tenham terminado os


três Fes --- os Festivais do ano todo". Ou seja, se os três Festivais
tiverem

passa o se tiver levado seu sacrifício, ter-se-á violado um preceito

negat as se apenas um Festival tiver passado, ter-se-á violado apenas

um p ositivo.

a Guemará de Rosh Hashaná lemos: "Rabi Meir diz: Assim que um festiv
terminar, ele terá transgredido o preceito 'Não demorarás em pagá-lo' ".
O Talmud pergunta: "Por que motivo Rabi Meir diz isso?" E a resposta é:
"Por­que está escrito 'E lá ireis. E levareis ali', o que significa que
você deve levá-los quando for lá". Mas os Sábios dizem que esse
versículo constitui apenas um preceito positivo.


Assim, ficou claro que as palavras ``E levareis ali'' são um preceito


positivo, significando que se deve cumprir t as suas obrigações para com

o Eterno e que isso deve ser feito em cada F sti , u• m elas quaisquer

tipos de sacrifícios, ou que sejam donativ s\textsuperscript{108}, alor
coisas consagra-


\begin{enumerate}
\def\labelenumi{\arabic{enumi}.}
\setcounter{enumi}{104}
\item
 
 Bekhorot 13:A.
 
\item
 
 Ver Preceitos Positivos 216 e 217.
 
\item
 
 Ver o preceito negativo 155.
 
\item
 
 Ver o preceito positivo 114.
 
\end{enumerate}




da \textsuperscript{109}oias dedicadas ao Santuár.
\textsuperscript{0110} respigadu. as
\textsuperscript{I I I} fixes esqueci 0s112 e "pe O cumprimento de to
esses tipos igações no prim\\
Festiva e houver é um preceito positivo, como está explicado na Guemará
de Rosh Hashaná.


LEVAR TODAS AS OFERTAS APENAS AO SANTUÁRIO


Por este preceito somos ordenados a oferecer todos os sacrifícios apenas
no Santuário. Ele está expresso em Suas palavras, enaltecido seja Ele,
"Ali oferecerás os teus holocaustos, e ali farás tud• • • e Eu te
ordeno" (Deu­teronômio 12:14). Procurando alguma citação e dei asse
clara a proibição de levar qualquer tipo de sacrifício a outro luga ,
114 ncontraram Suas pala­vras, enaltecido seja Ele, "Guarda-te de
oferecer te s holocaustos em todo o lugar que vires'' (Deuteronômio
12:13).

O Sifrei diz: "Eu sei disso apenas com relação ao Holocausto. De que
forma fico sabendo que é assim para com todos os outros sacrifícios?
Pelas pa­lavras das Escrituras 'E ali farás tudo o que eu te ordeno'.
Aqui novamente eu poderia dizer que apenas no caso do Holocausto há
ambos um preceito positi-

vo e um preceito negativo; co ue o mesmo se aplica a todos os outros

sacrifícios? Pelas palavras das 'Ali farás tudo' ". Eu voltarei a este
as-

sunto mais tarde, nas proibiçõ

O fato de dizer que no o de um Holocausto estão envolvidos um preceito
positivo e um negativo significa que aquele que o levar a outro lugar
estará violando ambos: um preceito positivo e um preceito negativo,
sendo o preceito negativo "Guarda-te de ofereceres teus holocaustos" e o
positivo "Ali

oferecerás", porque os terá levado ``ali''. Os outros sacrifícios pode-

riam abranger apen ito positivo ``E ali farás tudo o que eu te ordeno'',

mas está explicado a também no caso deles se viola um preceito nega-

tivo, ai. ► do ositiv á explicado no final do Tratado
Zebahim que todos

os sac fícios levados outro lugar envolvem um preceito positivo e um
negafi \textsubscript{0}115\textsubscript{,} = a penalidade de extinção.
Assim, por tudo o que expliquei, fica claro q e as alavras "Ali farás
tudo o que Eu te ordeno" são sem dúvida algu­ma um preceito positivo.

\section{Levar ao santuário, desde fora da terra de Israel, todos os sacrifícios devidos}

Por este preceito somos ordenados a levar ao Santuário tudo o que
tenhamos a obrigação de oferecer --- quer seja um Sacrifício de Pecado,
um


\begin{enumerate}
\def\labelenumi{\arabic{enumi}.}
\setcounter{enumi}{108}
\item
 
 Ver o preceito positivo 145.
 
\item
 
 Ver o preceito positivo 121.
 
\item
 
 Ver o preceito positivo 121.
 
\item
 
 Ver o preceito positivo 122.
 
\item
 
 Ver o preceito positivo 120.
 
\item
 
 Os Sábios
 
\item
 
 Ver os preceitos negativos 89 e 90.
 
\item
 
 No Sifrei.
 
\end{enumerate}

Holocausto, um Sacrifício de Delito ou um Sacrifício de Paz --- ainda
que o mo­tivo responsável pelo sacrifício esteja fora da Terra de
Israel, ou seja, embora a obrigação tenha sido ocasionada fora da Terra
de Israel, somos ordenados a levar os sacrifícios ao Santuário, e temos
o dever de fazê-lo, não importa a que distância. Este preceito está
expresso em Suas palavras, enaltecido seja Ele, "De certo, tuas coisas
sagradas, e tuas ofertá de votos, tomarás e levarás ao lugar que o
Eterno escolher" (Deuteronômio 12:26), sobre as quais o Sifrei diz: "
'Tuas coisas sagradas' se refere apenas aos sacrifícios fora da Terra de
Israel. `Tomarás e levarás' nos ensina que devemos nos preocupar com o
transporte dos sacrifícios até o Santuário". Está explicado ali que isso
se aplica apenas nos casos de Sacrifício de Pecado, de Sacrifício de
Delito, de Holocausto e de Sacri­fício de Paz que a pessoa tem a
obrigação de oferecer.


\section{Redimir oferendas defeituosas}


Por este preceito somos ordenados a redimir qualquer oferenda que tiver
se tornado defeituosa, liberando-a assim para uso normal e
permitindo-nos abatê-la e comê-la. Este preceito está expresso em Suas
palavras, enaltecido seja Ele, "Todavia, com todo o desejo de tua alma,
poderás degolar e comer carne, em todas as tuas cidades" (Deuteronômio
12:15), a respeito das quais o Sifrei diz: " 'Todavia, com todo o desejo
de tua alma, poderás degolar e co­mer carne, em todas as tuas cidades':
isto se refere apenas a oferendas defeituo­sas que tiverem sido
redimidas".

As regras detalhadas do preceito de redenção de oferendas estão
ex­plicadas no Tratado Bekhorot e Temurá, e em vários trechos de Hulin,
Arakhin e Meilá.

\section{A santidade de uma oferenda substituída}

Por este preceito somos ordenados a considerar como sagrado um animal
que for substituído por outro. Ele está expresso em Suas palavras,
enalte­cido seja Ele, "Tanto o que for trocado como aquele pelo qual
trocou, serão san­tidade" (Levítico 27:10). Está expressamente declarado
no início do Tratado Te­murá que Suas palavras, enaltecido seja Ele, "E
não trocará" (Ibid., 33) consti­tuem um preceito negativo justaposto a
um preceito positivo. Os Sábios pergun­tam: "Não é o caso da
substituição um exemplo de um preceito negativo justa­posto a um
preceito positivo?". No mesmo texto aparece um argumento suple­mentar
que sujeita aquele que fizer a substituição a ser açoitado, embora esse
seja u receito negativo justaposto a um preceito positivo: "Um preceito
posi­tivo ao ode se sobrepor a dois preceitos negativos." Ou seja, a
proibição de fawr a su stituição está expressa duas vezes, uma em "Não o
mudará" (Ibid., 170)\textsuperscript{117} outra em ``E não o trocará''
(Ibid.), enquanto há apenas um preceito positiv que é "Tanto o que for
trocado como aquele pelo qual trocou, serão santidade" (Ibid., 27:10).
Dessa forma, fica ratificado o que desejávamos provar.

As normas deste preceito, ou seja, o que valida ou invalida a
substi­tuição, quais são suas regras, e como ela deve ser oferecida,
estão explicadas no Tratado Temurá.


\section{Os ``cohanim'' devem comer os resíduos das oblações}

Por este preceito os ``Cohanim'' são ordenados a comer os resíduos das
Oblações. Ele está expresso em Suas palavras, enaltecido seja Ele, "E o
que ficar dela, comerão Aarão e seus filhos; comer-se-á sem fermento"
(Levítico 6:9). A Sifrá diz, a esse respeito: " 'Comer-se-á' é um
preceito positivo, da mesma forma que 'O.irmão de seu
marido estará com ela e a tomará por mulher' (Deu­teronômio 25:5) também
é um preceito positivo. Ou seja, comer os resíduos das Oblações e
casar-se com a viúva de um irmão que morreu sem deixar filhos são dois
preceitos positivos e não meras opções.

As normas destes preceitos estão explicadas no Tratado Menahot. A Torah
determina que este preceito se aplique apenas aos homens, de acordo com
Suas palavras, enaltecido seja Ele, "Todo varão dos filhos de Aarão o
co­merá" (Levítico 6:11).

\section{Os ``cohanim'' devem comer a carne dos sacrifícios consagrados}

Por este preceito os ``Cohanim'' são ordenados a comer a carne dos
Sacrifícios Consagrados, a saber, o Sacrifício de Pecado e o Sacrifício
de Delito, que estão entre os Sacrifícios Mais Sagrados. Este preceito
está expresso em Suas palavras, enaltecido seja Ele, "E comerão das
coisas com que for feita a expia­ção" (Êxodo 29:33).

A Sifrá diz: "De que maneira saber que o fato de que comam os
Sa­crifícios Consagrados concede o perdão para toda Israel? Pelas
palavras da To­rah 'E o Eterno vô-lo deu para levardes a iniquidade da
congregação, a fim de perdoar por eles, diante do Eterno!' (Levítico
10:17). De que forma? O 'Cohen' o come, e Israel recebe o perdão."

Urna das condições deste preceito é que só se deve comê-los duran­te um
dia e uma noite até a meia-noite. Depois disso, fica proibido comer
de­les; ele é uma obrigação apenas durante o espaço de tempo
estabelecido.

Fica clarb que também este preceito se aplica apenas aos elementos
masculinos das famílias dos ``Cohanim'', e não às mulheres, pois as
mulheres não podem comer dos Sacrifícios Mais Sagrados a que se refere
este preceito. Os outros sacrifícios --- a saber, os Sacrifícios Menos
Sagrados--- podem ser co­midos no espaço de dois dias e uma noite,
exceto a Ação de Graças e o carnei­ro dos Nazirim os quais, emb jam
Sacrifícios Menos Sagrados, devem ser comidos em um dia e uma oite at a
meia-noite. Outrossim, as mulheres po­dem comer desses Sacrifí os Menu s
Sagrados.

O fato de com ostra t. bém é parte deste preceito, bem como o de comer a
Oferta de Eleào. Co tudo, comer os Sacrifícios Menos Sagrados e os
Sacrifícios de Elevação úúo é orno comer a carne dos Sacrifícios de
Peca­do e de Delito, pois o ato de comer a carne dos Sacrifícios de
Pecado e de Deli­to completa o perdão de quem ofereceu os sacrifícios,
como explicamos, e o ato de comer é ordenado explicitamente no caso
deles mas não no caso dos

118. O fato dos ``Cohanim'' comerem os Sacrifícios Menos Sagrados.

Sacrifícios Menos Sagrados e dos Sacrifícios de Elevação.
Consequentemente, isso constitui apenas uma parte do preceito que se
aplica aos outros sacrifícios, e ao comê-los ele executa um preceito. O
Sifrei diz: " 'O serviço de vosso sa­cerdócio dei-o como dádiva a vós'
(Números 18:7) faz com que o ato de comer as Coisas Consagradas dentro
da Terra de Israel seja como o serviço no Santuá­rio: assim como ele
tinha que lavar suas mãos antes de iniciar o serviço no San­tuário, ele
também tinha que lavar as mãos antes de comer as Coisas Sagradas fora de
Jerusalém."


As normas deste preceito estão explicadas em diversos trechos de


Zebahim.

\section{Queimar sacrifícios consagrados que se tornaram impuros}

Por este preceito somos ordenados a queimar os Sacrifícios Consa­grados
que se tenham tornado impuros. Ele está expresso em Suas palavras,
enaltecido seja Ele, ``E a carne Sagrada do sacrifício de pazes que tocar
em tudo o que for impuro, não será comida, no fogo será queimada''
(Levítico 7:19).

A Guemará de Shabat discute a questão de por que é proibido usar óleo de
Oferta de Elevação que tenha se tornado impuro para iluminação, em d'


de Fest'


de Festival baseia-se num preceito positivo e num negativo, e um pr
positi o pode se sobrepor a ambos um preceito negativo e um positi

O significado deste trecho é que é proibido trabalhar num dl Festival e
aquele que o fizer estará infringindo um preceito positivo, tendo
anu­lado o preceito de que o dia de Festival ``será para vós descanso
solene'' (Leví­tico 23:24). Ele também estará infringindo um preceito
negativo, porque estará fazendo algo que lhe foi proibido, de acordo com
as palavras ``Nenhuma obra será feita neles'' (Êxodo 12:16), ou seja, nos
dias de Festival. Como a queima de Sacrifícios Consagrados que se
tornaram impuros é apenas um preceito po­sitivo, não é permitido
queimá-los num dia de Festival, de acordo com o prin­cípio já mencionado
de que um preceito positivo não pode se sobrepor a am­bos um preceito
negativo e um positivo. Também está dito: ``Da mesma forma que é
obrigatório queimar Sacrifícios Consagrados que se tornaram impuros,
assim também é obrigatório queimar o óleo da Oferta de Elevação que
tiver se tornado impuro''.

As normas deste preceito estão explicadas em Pessahim e no final de
Temurá.

\section{Queimar as sobras dos sacrifícios consagrados}

Por este preceito somos ordenados a queimar as sobr. Sie está expresso
em Suas palavras, enaltecido seja Ele, "E o que ficar da ca do sacri-


\begin{enumerate}
\def\labelenumi{\arabic{enumi}.}
\setcounter{enumi}{118}
\item
 
 I.e., o preceito de queimar todos os sacrifícios que tenham se tornado
 impuros.
 
\item
 
 Ver os preceitos negativos 323 a 329.
 
\item
 
 Dos Sacrifícios Consagrados.
 
\end{enumerate}




fício, no terceiro dia, no fogo será queimado" (Levítico 7:17). Com
relação a Suas palavras, enaltecido seja Ele, relativas ao cordeiro de
``Pessah'', ``E não fareis so­brar nada dele até a manhã; e a sobra dele,
pela manhã a queimareis no fogo'' (Êxodo 12:10), a Mekhiltá diz: ``O
objetivo das Escrituras é estipular um preceito positivo para o preceito
negativo''. Em vários trechos de Pessahim e Macot, além de outros, está
explicitamente declarado que o preceito negativo relativo às so­bras
está justaposto a um preceito positivo, e dessa forma não há punição por
açoitamento. O preceito ositivo a que nos referimos está expresso nas
palavras já mencionadas, ou s 'E o que ficar da carne... no fogo
queimado".

A lei de R e a das Sobras são semelhan s, co o explicarei

nos preceitos negati , uma vez que a palavra "no usada como\\
recusa.

As normas deste preceito estão explicadas em Pessahim e no final de
Temurá.

\section{O nazir deve deixar crescer seus cabelos}

Por este preceito o Nazir é ordenado a deixar crescer seus cabelos. Ele
está expresso em Suas palavras, enaltecido seja Ele, "Sagrado será ele;
dei­xará crescer o cabelo de sua cabeça" (Números 6:5). A esse respeito
diz a Mek­hiltá: " 'Sagrado será o seu cabelo': ele deixará crescer seu
cabelo em sinal de santidade. 'Deixará crescer... cabeça' é um p
e\textsubscript{a}i

e positivo. De que forma fi-

camos sabendo que também há um preceito gtt o? Pelas palavras das E
turas: 'Lâmina não passará pela sua cabeç "124

Também está dito na Mekhiltá: 0 p eito positivo se aplica
o\textsuperscript{125} que esfregar terra ou aplicar produtos químicos";
ou seja, se ele aplicar pr

tos químicos em sua cabeça que provoquem a queda dos cabelos ele não
estará infringindo o preceito negativo, porque ele não terá colocado em
sua cabeça nada semelhante a uma lâmina, mas estará violando o preceito
positivo ``Dei­xará crescer o cabelo de sua cabeça'' ao não permitir que o
cabelo cresça. E, de acordo com nossos Fundamentos, um preceito negativo
derivado de um pre­ceito positivo é um preceito positivo.

As normas deste preceito estão explicadas no lugar apropriado, no
Tratado Nazir.


\section{A obrigação do nazir de consumar seu voto}


Por este preceito o Nazir é ordenado a raspar sua cabeça e a levar seus
sacrifícios quando os dias de sua consagração estiverem terminados. A
Si­frá diz: "Há três so m que a raspagem da cabeça constitui um preceito
posi­tivo: para o N \textsubscript{t\}}r,\textsubscript{i}rr o leproso e
para os Levitas". Contudo, a raspagem da cabeça dos Lev as
. s era obrigatória enquanto eles estavam no deserto e
dei-


\begin{enumerate}
\def\labelenumi{\arabic{enumi}.}
\setcounter{enumi}{121}
\item
 
 Ver os preceitos negativos 131 e 132.
 
\item
 
 Sobras.
 
\item
 
 Ver o preceito negativo 209.
 
\item
 
 Se aplica ao Nazir.
 
\item
 
 Números 8:7.
 
\end{enumerate}

xou de sê-lo depois disso, enquanto que a raspagem da cabeça do leproso
e do Nazir será sempre obrigatória.

Está claro que o Nazir tem duas ocasiões para raspar sua cabeça: ele
deve fazê-lo se tiver se tornado impuro, como prescrito nas palavras "E
quan­do alguém morrer subitamente, junto a ele" etc. (Números 6:9), e em
estado de pureza, como prescrito nas palavras ``No dia em que se
completarem os dias de seu Nazirado'' (Ibid. 6:13).

Contudo, essas duas obrigações de raspar a cabeça não podem ser contadas
como dois preceitos diferentes, uma vez que a raspagem a ser feita por
causa de uma impureza é um dos detalhes do regulamento relativo ao
pre­ceito do voto do Nazir --- o preceito positivo de deixar crescer seu
cabelo em santidade, como já explicamos. Depois disto as Escrituras
especificam que se o Nazir se tornar impuro ele deve raspar sua cabeça e
trazer um sacrifício e de­pois deixar seu cabelo crescer novamente em
santidade durante o período de Nazirado a que ele se propôs; assim como
no caso do leproso, as duas obriga­ções de raspar a cabeça constituem um
único preceito, como explicarei no lu­gar apropriado.

Também explicarei mais tarde por que, no caso do Nazir, nós conta­mos
raspar a cabeça e trazer seus sacrifícios como um preceito, enquanto que
no caso do leproso nós os contamos como dois.

As normas deste preceito, ou seja, a raspagem da cabeça do Nazir, estão
explicadas no texto a esse respeito do Tratado Nazir.

\section{Cumprir todos os compromissos orais}

Por este preceito somos ordenados a cumprir toda obrigação que
assumirmos através de palavras --- todo juramento, prA e a, oferenda ou
equi-

valente. Ele está expresso em Suas palavras, enaltec Ele, "O que sair de

teus lábios guardarás" (Deuteronômio 23:24). Emb nham analisado mi-

nuciosamente este versículo e tenham explicado cad avra separadamente,

o sentido global de tudo o que eles dizem é que este é um preceito
positivo que obriga o homem a cumprir todo co omisso que ele tenha
assumido, e que deixar de fazê-lo é transgredir um dito negativo. Isto
será explicado quando eu tratar dos preceitos negativo

O Sifrei diz: " 'O que sair dos t lábios' é um preceito positivo".

Você sabe, porém, que não se pode derivar nenhum preceito a partir das
sim­ples palavras ``O que sair dos teus lábios'', portanto o significado
delas deve ser o que mencionei como significado literal das Escrituras,
ou seja, que um homem tem a obrigação de cumprir o que seus lábios
tenham proferido. Este preceito também se encontra em outro trecho em
Suas palavras, enaltecido se­ja Ele, "Como tudo que saiu de sua boca,
assim fará" (Números 30:3).

As normas deste preceito estão explicadas em vários trechos de Sha­buot
e de Nedarim, no final de Menahot, e também no Tratado Kinim; ou seja,
foi deixado claro que devemos ser corretos no cumprimento de qualquer
obri­gação a que tenhamos nos comprometido, e de que forma podemos ser
absol­vidos se tivermos dúvidas quanto ao que nos tenhamos imposto.


\begin{enumerate}
\def\labelenumi{\arabic{enumi}.}
\setcounter{enumi}{126}
\item
 
 Os Sábios.
 
\item
 
 Ver o preceito negativo 157.
 
\end{enumerate}




\section{A revogação de promessas}

Por este preceito somos ordenados a aplicar as leis relativas à
revo­gação de promessas. Este preceito, contudo, não significa que
sejamos obriga­dos a revogar as promessas em todos os casos. Você deve
entender que a mes­ma coisa se aplica a todas as leis que eu enumerar:
um preceito não nos obriga necessariamente a fazer uma determinada
coisa, mas estipula como devemos tratar do assunto em o, de acordo com
as leis.

Evidentem tá explícito nas Escrituras que um marido e um pai

podem fazer a revogaç estão indicados os procedimentos para isso. A Tra-

dição também autoriz u ábio a liberar alguém de uma promessa ou de um\\
juramento, baseada em Suas palavras ``Não profanar a palavra'' (Nú ros\\
30:2), ou seja, "Fle não pode quebrar sua promessa aso. outros po•fa ê-

lo por ele". De uma maneira geral, não há base es ara isso
ele,\textsuperscript{131} i '-

zem: "As regras sobre liberar alguém de uma pro es airam n • na m

nada de concreto para apoiá-las", a não ser a verdadeira Tradiç .

As normas deste preceito estão explicadas no Tratado
\textbf{.} trata es­pecificamente deste assunto, que é
o Tratado Nedarim.

\section{Tornar-se impuro com carcaças de animais}

Por este preceito somos ordenados quanto à impureza da carcaça de um
animal, e ele inclui a impureza de uma carcaça e todas as normas a esse
respeito.

À guisa de prefácio mencionarei algo que você deve ter em mente com
relação a tudo que será dito a seguir, quanto aos vários tipos de
impureza. O fato de que contemos cada um dos vários tipos de impureza
como um pre­ceito positivo nã•significa que seja uma obrigação ou que
seja proibido tornar-se impuro de u ou de outra dessas maneiras, como se
isso fosse um preceito negativo. O e qu remos dizer é que quando a Torah
diz que quem tocar este ou aquele t o\textsuperscript{133} t. na-se
impuro, ou que este ou aquele objeto torna impuro de uma determina• a
forma aquele que o tocar, isto constitui um preceito posi­tivo; ou seja,
est ei que somos obrigados a seguir é um preceito que estabele­ce que
quem tocar determinadas coisas sob determinadas condições se tornará
impuro mas, caso seja sob condições diferentes, ele não se tornará
impuro. Na realidade, o fato de se tornar impuro é opcional: se um homem
quiser, ele se tornará impuro, e se não o quiser, não o fará.

A Sifrá diz: " 'No seu cadáver não tocareis' (Levítico 11:8):
podería­mos pensar que se uma pessoa tocar uma carcaça ele estará
sujeito ao açoita­mento; por isso as Escrituras dizem: 'E por estes vos
tornareis impuros' (Ibid., 24). Poderíamos pensar que se uma pessoa vê
uma carcaça ele deve tocá-la e assim tornar-se impuro. Por isso as
Escrituras dizem: `No seu cadáver não toca­reis'. Como fazer para
conciliar esses dois versículos? Devemos concluir que tocar uma carcaça
é opcional".


\begin{enumerate}
\def\labelenumi{\arabic{enumi}.}
\setcounter{enumi}{128}
\item
 
 Podem revogar as promessas da esposa e da filha, respectivamente.
 
\item
 
 Na Torah escrita.
 
\item
 
 Os Sábios.
 
\item
 
 A Tradição oral.
 
\item
 
 Tipo de impureza.
 
\end{enumerate}
Portanto o conteúdo deste preceito é que todo aquele que toc
r\textsuperscript{134} se torna impuro, e ao se tornar impuro ele está
sujeito a todas as obrigaç "es impostas às pessoas impuras: ele deve
sair do Acampamento da Presença Divi­na; não deve comer nem tocar Coisas
Sagradas, e assim por diante. A essência deste preceito é que uma pessoa
se torna impura ao tocar um determinado ob­jeto ou, em determinadas
circunstâncias, ao estar próximo dele.

Tenha isso em mente com relação aos vários tipos de impureza.


\section{Tornar-se impuro através das carcaças de determinados animais rastejantes}

Por este ceiti somos ordenados quanto à impureza de oito tipos


de criaturas rastejan s\textsuperscript{135}. ste preceito abrange a lei
da impureza de animais rastejantes e as regras" alhadas referentes a
ela.

\section{Tornar-se impuro através de comida e bebida}

Por este preceito somos ordenados a lidar com a im re a da comi­da e da
bebida de acordo com as leis prescritas. Este preceitp incl i todas as
leis relativas à impureza de comida e bebida de todos os
tiOos\textsuperscript{136}

\section{A mulher menstruada}

Por este preceito somos ordenados quanto à impureza a mulher menstruada.
Este preceito inclui todas as regras a esse
respeito\textsuperscript{137}.

\section{Depois do nascimento}


DE UMA CRIANÇA )


Por este preceito somos ordenados quanto à impureza de uma/mu­lher
depois de um parto. Este preceito inclui todas as regras a esse
respeito\textsuperscript{138}.

\section{O leproso}

Por este preceito somos ordenados quanto à impureza de um lepro­so. Este
preceito inclui toda a regulamentação referente à lepra: quais são casos
impuros e quais os puros, quais necessitam segregação e quais não, qúais
requerem, além da segregação, que a cabeça seja raspada, e quais não,
bem co­mo outros detalhes relativos as suas regras e à natureza de sua
impureza\textsuperscript{139}.

\begin{enumerate}
\def\labelenumi{\arabic{enumi}.}
\setcounter{enumi}{133}
\item
 
 Que tocar qualquer tipo de impureza.
 
\item
 
 Esse preceito está expresso em Levítico 11:29-30.
 
\item
 
 Essas leis estão contidas em Levítico 11:34.
 
\item
 
 Esse preceito se encontra em Levítico 15:19-24.
 
\item
 
 Ele está expresso em Levítico 12:2-5.
 
\item
 
 Todas as leis referentes a esse preceito estão contidas em Levítico
 13:1-59
 
\end{enumerate}




\section{As roupas contaminadas pela lepra}

Por este preceito somos ordenados quanto à impureza de uma rou­pa
contaminada pela lepra. Ele inclui toda a regulamentação a esse
respeito: co­mo as roupas se tornam impuras e como elas causam impureza,
quais

ser separadas, rasgadas, queimadas, lavadas ou purificadas, bem como udo
o mais que está prescrito nas Escrituras e o que a Tradição diz a esse
respeito'"

\section{A casa de um leproso}

Por este preceito somos ordenados quanto à impureza da casa de um
leproso. Este preceito inclui todas as regras a esse respeito: que elas
devem ser isola • as, • ais devem ter suas paredes parcialmente
demolidas, ou ser com­pletam me d. olidas como elas se
tornam impuras e como elas causam impur zási41.

\section{O ``zab''}

Por este preceito somos ordenados qua à i ureza de um ``zab''.

Este preceito inclui toda a regulamentação relativa aos si tomas de um
``zab'' e à maneira como ele torna outras pessoas
impu'ras\textsuperscript{142}.

\section{O sêmen}

Por este preceito somos ordenados quanto impu za do sêmen. Este preceito
inclui toda a regulamentação a esse respeito ".

\section{A ``zaba''}

Por este preceito somos ordenados quanto à impu a de ma "za­ba". Ele
inclui as regras referentes aos sintomas que tornam ma mul er uma ``zaba''
e à maneira pela qual ela torna outras pessoas impuras
\textsuperscript{144} a s ter-se tornado uma ``zaba''.

\section{A impureza de um cadáver}
.

Por este preceito somos ordenados quanfo à imp reza de um cadá­ver. Este
preceito inclui todas as regras a esse respeito\textsuperscript{145}.

\section{A lei da água de aspersão}

Por este preceito somos ordenados quanto às regras referentes à água

\begin{enumerate}
\def\labelenumi{\arabic{enumi}.}
\setcounter{enumi}{139}
\item
 
 As regras que regem este preceito estão em Levítico 13:47-59.
 
\item
 
 Os detalhes deste preceito se encontram em Levítico 14:36-48.
 
\item
 
 As regras que regem este preceito estão em Levítico 14:35-54 e
 15:1-12.
 
\item
 
 As regras que regem este preceito estão em Levítico 15:16-18.
 
\item
 
 As regras que regem este preceito estão em Levítico 15:25-30.
 
\item
 
 As regras que regem este preceito estão em Números 19:14-16.
 
\end{enumerate}

de aspersa \textsuperscript{146} a\textsubscript{/}qual, sob certas
circunstâncias, purifica e, sob outras, impuri­fica, como Sermoexplicado
no estudo detalhado deste preceito.

Você deve saber que os treze tipos de impurezas que foram enume­rados
--- a saber, a impureza de uma carcaça, de animais rastejantes, de
comi­das, de uma mulher menstruada, de uma mulher depois do parto, de um
lepro­so, de roupas contaminadas pela lepra, de uma casa contaminada
pela lepra, de um ``zab'', de uma ``zaba'', do sêmen, de um cadáver, e da
água de aspersão --- e a rificação para cada um deles que estão todos
explicados na Torah;

ve s er mbém quê 'Irá \textsubscript{\textbackslash{}}vários textos,
regulamentações e condições r a vos a c a um desses preceit expostos nos
trechos Vayehi Bayom Has , 1.147\\
Tazria\textsuperscript{148}, Zot Tih'ye\textsuperscript{149} e no trecho
Veyikemu Eilecha para Adoma\textsuperscript{15}°

Les­ses quatrojtextos está t o o que se refere aos vários tipos de
impure a. Mas todas as regras e regulamentos relativos a esses tipos de
impurezas estão conti­dos na Ordem de Teharot (Pureza). Três dos
Tratados dessa Ordem, a saber, Teharot, Makhshirin e Okatzin, contém as
impurezas referentes à comida, e tra­tam exclusivamente desse assunto, e
qualquer menção sobre outras impurezas nesses Tratados é meramente
acidental. Da mesma forma, o Nidá contém os regulamentos referentes à
mulher mestruada, à ``zaba'' e à mulher depois do parto, sendo que também
se encontram' algumas das regras referentes a esta úl­tima no Tratado
Queretot. O Tratado Negaim contém todos os regulamentos sobre a lepra
nos homens, roupas e casas; o Tratado Zabim contém os regula­mentos
referentes ao ``zab'', à ``zaba'' e ao sêmen; o Tratado Ohalot trata dos
cadáveres e o Tratado Pará traz as regras sobre a água de aspersão como
um agente de purificação ou de impurificação. Por outro lado, não há
Tratados es­pecíficos sobre a impureza de carcaças e de animais
rastejantes; os regulamen­tos sobre esses assuntos estão espalhados em
vários trechos da Ordem, sobre­tudo nos Tratados Quelim e Teharot. E
várias questões referentes a esses as­suntos são tratadas no Tratado
Eduyot. Nós próprios compusemos um comen­tário sobre a Ordem de Teharot
e não é necessário consultar qualquer outro livro além desse sobre
qualquer assunto relativo à pureza e impureza.

\section{Mergulhar no banho ritual}

Por este preceito somos ordenados a mergulhar nas águas de um ba­nho
ritual, e assim limpar-nos de todo tipo de impurezas que nos tentiam
afeta­do. Este preceito está expresso em Suas palavras, enaltecido seja
Ele, "E o ho­mem ... banhará em água toda a sua carne" (Levítico 15:16),
a respeito das quais a Tradição diz: "Água suficiente para cobrir todo o
seu corpo: isto é, a medida de um banho ritual" --- exceto no caso de
água corrente, para a qual não há medida prescrita, como está explicado
nos detalhes deste preceito.

Uma das cláusulas deste preceito estipula que apenas o ``zab'' deve
purificar-se em água corrente, de acordo com o que diz a Torah: "E
banhará sua carne nas águas vivas" (Ibid., 13).

Ao tratar da imersão como um preceito positivo, não queremos di­zer que
todas as pessoas impuras devam tomar banho de imersão, que todas


\begin{enumerate}
\def\labelenumi{\arabic{enumi}.}
\setcounter{enumi}{145}
\item
 
 Números XIX, 9-21.
 
\item
 
 Levítico IX, 1-11; 47.
 
\item
 
 Levítico XII, 1-13; 59.
 
\item
 
 Levítico XIV, 1-15; 33.
 
\item
 
 Números XIX, 1-22.
 
\end{enumerate}




as pessoas que usam uma vestimenta devem colocar ``Tsitsit'' nela, ou que
to­dos os que tenham uma casa devem fazer um suporte; o que quero dizer
é ape­nas que pela lei da imersão aquele que desejar livrar-se de suas
impurezas não pode atingir seu objetivo a não ser pela imersão em água,
depois do que ele se tornará puro. •

A Sifrá diz: " 'E lavar-se-á nas águas' (Levítico 14:8); • • deríamos
jul­gar isto como sendo um decreto do Rei. Por isso as Escritur.: diz :
'E depois entrará no acampamento' (Ibid.), por causa de sua impura
a\textsuperscript{151}. . sto leva ao princípio que eu
havia explicado, ou seja, que a lei de imersa• se plica apenas àquele
que quiser se purificar, e esta lei é um preceito. Não há, contudo,
ne­nhuma obrigação de banhar-se, e aquele que quiser permanecer impuro e
esti­ver disposto a privar-se de entrar no acampamento da Presença
Divina por al­gum tempo tem a liberdade de fazê-lo.

O Livro da Verdade deixa claro que todo aquele que estiver impuro e
fizer uma imersão se purificará, mas sua purificação não estará completa
até o pôr-do-sol; e de acordo com a interpretação tradicional, ele
deverá estar nu e todo o seu corpo deverá ficar em contato com a água.
Como diz o Talmud, " 'Toda sua carne', ou seja, não haver nada entre sua
carne e a água."

Fica assim claro que. ceito da imersão inclui a
regulamenta­ção do banho ritual, a interposi o ``Tebul yom''. Ele está
explicado nos Tratados Mikvaot e Tebul Yom.

\section{Purificar-se da lepra}

Por este preceito somos ordenados a que a purificação da lepra seja
realizada de acordo com as normas estabelecidas nas Escrituras, ou seja,
com pau de cedro, hissopo, carmezim, dois pássaros vivos e águas vivas,
e que eles sejam empregados como está determinado. O homem será
purificado por esse procedimento, como explicam as Escrituras.

Foi explicado, portanto, que de acordo com nossa Torah há três mé­todos
diferentes pelos quais pode ser realizada a purificação: um geral, e
dois aplicáveis cada um apenas a um tipo específico de impureza. O
método geral é pela água, a qual é indispensável para purificar-se; o
segundo é pela água de aspersão e aplica-se especificamente à impureza
adquirida através de um mor­to; o terceiro, que consiste de pau de
cedro, hissopo, carmezim, dois pássaros vivos e águas vivas, aplica-se
especificamente no caso da lepra.

Todas as normas deste preceito, ou seja, a primeira purificação da
lepra, estão explicadas no Tratado Negaim.

\section{O leproso deve raspar a cabeça}

Por este preceito o leproso é ordenado a raspar a cabeça, e isto
cons­titui o segundo estágio de sua purificação, como está explicado no
final de Ne­gaim. Este preceito está expresso em Suas palavras,
enaltecido seja Ele, ``E ao sétimo dia raspará todo o seu pelo'' (Levítico
14:9). Já nos referimos anterior­mente às palavras dos Sábios: "Três
raspam suas cabeças, e para cada um deles


\begin{enumerate}
\def\labelenumi{\arabic{enumi}.}
\setcounter{enumi}{150}
\item
 
 Entrará no acampamento do qual havia sido excluído por causa de sua
 impureza.
 
\item
 
 A interposição de alguma coisa entre o corpo e a água.
 
\end{enumerate}


\textbf{132 MAIMÔNIDES}

a raspagem constitui um preceito positivo: o Nazir, o leproso e o
Levita". As normas deste preceito estão explicadas no final de Negaim.

Aqui explicarei por que no caso do leproso contamos a raspagem e a
oferta dos sacrifícios determinados cada um como um preceito individual,
enquanto que no caso do Nazir contamos os dois juntos como um preceito.
É que no caso do leproso não há conexão entre o ato de raspar a cabeça e
o de levar seus sacrifícios, e o objetivo atingido pela raspagem é
diferente do al­cançado pelo oferecimento dos sacrifícios, porque sua
purificação depende da raspagem de sua cabeça. No sexto capítulo de
Nazir está dito: "Como um Nazir difere de um leproso? Sua purificação
depende dos dias, enquanto que a purifi­cação do leproso depende da
raspagem de seus cabelos". Tendo raspado a ca­beça, e tendo completado
sua segunda raspagem, o leproso se purifica e cessa de transmitir o tipo
de impureza que é transmitida pelos animais rastejantes, como está
explicado no final de Negaim; e seu perdão fica em suspenso até que ele
leve seus sacrifícios, da mesma forma que os ciutros cujo perdão fica em
suspenso, como está explicado ali.

Assim, a raspagem da cabeça torna o leproso puro a ponto de que ele
cesse de transmitir o mesmo tipo de impureza que é transmitida por um
animal rastejante, quer ele tenha ou não oferecido seus sacrifícios; e a
oferta dos sacrifícios complementa seu perdão, tal como nos outros casos
em que o perdão fica em suspenso --- a saber, o ``zab'', a ``zaba'', e a
mulher depois do parto. Nós já nos referimos às palavras dos Sábios: "Há
quatro pessoas cujo per­dão fica em suspenso" etc.

No caso de um Nazir, como está explicado ali, o perdão não fica
in­completo, e todo o procedimento estabelecido --- raspar a cabeça e
oferecer o sacrifício --- lhe permite beber vinho novamente. E um não é
suficiente sem o outro: a raspagem está ligada ao sacrifício, e o
sacrifício à raspagem, e os dois em conjunto atingem um objetivo único,
que é permitir-lhe as coisas que lhe eram proibidas nos seus dias de
Nazirado. No sexto capítulo de Nazir está dito: "Se ele raspou seu
cabelo depois de um dos sacrifícios e este foi considerado inválido, a
raspagem de seu cabelo também se torna inválida e seus sacrifícios não
contam. = 'm, ficou explicado que a raspagem é uma das condições da
oferenda, nda é uma das condições da raspagem.


foi explicado na Tosseftá que um Nazir que tenha comple-


tado seus á proibido de raspar a cabeça, beber vinho e tornar-se im-

puro pel s s até que ele tenha completado todo o procedimento de ras-

pagem e estado de pureza, o qual, como está explicado no sexto capítulo
de Nazir, im s lica em que a raspagem seja feita diante da porta da
Tenda de Assina­ção, que le jogue se belos debaixo da caldeira e que ele
leve as oferendas, como está explica nas scrituras.

Vocês ontrar o que na maioria dos lugares os Sábios denominam

a oferta dos sacri ios\textsuperscript{154} d ``raspagem'', e eles dizem
explicitamente na Mish­ná: " 'Eu serei u Nazir e e comprometo a raspar a
cabeça etc': a intenção é de levar as ofer das do azir e oferecê-las por
si próprio. Assim foi explica­do que a raspagem um te o alternativo para
seu oferecimento de sacrifícios, e a razão disso é que parte deste
último, como explicamos, e é unicamente com a combinação destes dois que
o Nazirado se completa e que o Nazir pode beber vinho. Mas a raspagem
por impureza é apenas uma das leis do preceito, como explicamos
anteriormente.


\begin{enumerate}
\def\labelenumi{\arabic{enumi}.}
\setcounter{enumi}{152}
\item
 
 Seus dias de Nazirado.
 
\item
 
 Dos Nazirim.
 
\end{enumerate}




\section{O leproso deve ser reconhecível}

Por este preceito somos ordenados á que o leproso seja tornado
re­conhecível, de forma que as pessoas possam se' manter afastadas dele.
Este pre­ceito está expresso em Suas palavras, enaltecido seja Ele, "E
do leproso que tem a chaga, suas vestes serão rasgadas e seu cabelo não
será cortado, e com seu bigode se cobrirá; e impuro! impuro! clamará"
(Levítico 13:45).

A prova de que este é um dos preceitos positivos é encontrada na Sifrá:
"Como foi dito, a respeito do 'Cóhen Gadol' que 'Seu cabelo não deixará
crescer e suas vestes não romperá' (Ibid., 21:10) poder-se-ia pensar que
isto se aplica mesmo se ele tiver contraído uma praga, e que o preceito
'Suas vestes serão rasgadas e seu cabelo não será cortado' se aplica a
todos menos ao 'Co­hen Gadol'. Por isso a Torah diz: 'Em quem estiver a
praga' etc.; mesmo se ele for um 'Cohen Gadol', suas roupas devem ser
rasgadas, o seu cabelo deve ficar solto, e ele deve deixar seu cabelo
crescer."

Está cl e um ``Cohen Gadol'' está proibido, através de um pre-

ceito negativo, • e rasga suas roupas ou de deixar crescer seu cabelo; e
é um princípio acei • entre n o s que toda vez que encontrarmos um
preceito positi­vo e um neg vo\textsuperscript{155}, se e udermos
cumprir os dois será ótimo; caso contrário o preceito p • sítivo se
brepõe ao negativo. Conseqüentemente, uma vez que está estabele id e e u
- m ``Cohen Gadol'' leproso deve deixar crescer seu ca­belo e rasgar suas
roupas, conclui-se que este é um preceito positivo.

Por tradição, outras pessoas impuras também são obrigadas a se tor­nar
reconhecíveis a fim de que os outros possam se manter afastados delas. A
Sifrá diz: "Como sabemos que isto também se aplica a alguém que tenha se
tor­nado impuro através de um cadáver ou de uma relação com uma mulher
mens­truada, e a todos os que possam transmitir impureza aos outros? As
Escrituras dizem: 'E impuro! impuro! clamará' (Ibid., 13:45)". Isto
significa que toda pes­soa impura deve anunciar sua impureza, ou seja,
deve se tornar reconhecível como pessoa impura, que transmite impureza
com seu contato, de forma a que as pessoas possam evitá-la.

Ficou claro que a obrigação de um leproso de se tornar reconhecí­vel não
se aplica às mulheres. ``Um homem'', diz o Talmud, "anda com o cabe­lo
solto e com as roupas rasgadas, mas uma mulher não anda com o cabelo
sol­to e com as roupas rasgadas", embora ela deva cobrir seu lábio
superior, e pro­clamar sua impureza como as outras pessoas impuras.

\section{As cinzas da vaca vermelha}

Por este preceito somos ordenados a preparar a vaca vermelha de maneira
tal que suas cinzas possam ser utilizadas para fazer o que deve ser
feito a fim de remover a impureza de um cadáver, como disse o
Enaltecido: "E a congregação dos filhos de Israel a guardará por água
purificadora" (Números 19:9).

As normas deste preceito estão explicadas no Tratado que trata
es­pecificamente deste assunto, que é o Tratado Pará.

\section{A avaliação de uma pessoa}

Por este preceito somos ordenados quanto à lei da avaliação do ho­mem, a
qual estabelece que quando alguém diz "Eu prometo meu próprio va­lor" ou
``Eu prometo o valor de uma determinada pessoa'', se for um homem, ele
deve pagar uma determinada importância, e se for uma mulher, ele deve
pagar urna determinada importância de acordo com a idade, como está
estabe­lecido nas Escrituras, e de acordo com as posses de que fez a
promessa. Este preceito está expresso em Suas palavras, enaltecido seja
Ele, "Quando alguém fizer um voto, para oferecer preço de almas...
segundo as posses daquele que fez o voto." (Levítico 27:2-8)


As normas deste preceito estão explicadas no Tratado Arakhin.


\section{A avaliação de animais}

Por este preceito somos ordenados quanto à lei da avaliação dos ani­mais
impuros. Ele está expresso em Suas palavras "Fará apresentar o animal
dian­te do 'Cohen'. E o avaliará o 'Cohen' (Levítico 27:11-12).

As normas deste preceito estão explicadas em trechos dos Tratados Temurá
e Meilá.

\section{A avaliação de casas}

Por este preceito somos ordenados quanto à avaliação das casas. Ele está
expresso em Suas palavras: "E quando alguém consagrar a sua casa para
ser santidade ao Eterno, o 'Cohen' a avaliará". (Levítico 27:14)

As normas deste preceito estão explicadas no Tratado Arakhin.

\section{A avaliação dos campos}

Por este preceito somos ordenados quanto à avaliação dos campos. Ele
está expresso em Suas palavras, enaltecido seja Ele, "E se alguém
consagrar uma parte do campo de sua possessão ao Eterno" etc. (Levítico
27:16) ... "E se o campo de sua compra, que não é campo de sua herança,
consagrar ao Eterno" etc. (Ibid., 22). No caso de "campo de sua
possessão", "A tua avaliação será con­forme a semente necessária para
semeá-lo" (Ibid., 16); e no caso de ``campo de sua compra'', "O 'Cohen'
calculará para ele, a conta de sua avaliação" (Ibid., 23).

As normas deste preceito também estão explicadas na íntegra no 'fra­tado
Arakhin.

Que ninguém pense que esses quatro tipos de avaliação têm tanto em comum
que deveriam ser contados como um único preceito. Eles são qua­tro
preceitos separados, cada um com suas regras específicas, embora a
palavra ``avaliação'' seja comum a todos eles. Conseqüentemente não se
deve contar todos os tipos de avaliação como um único preceito, da mesma
maneira que não se contam todos os tipos de oferenda como um único
preceito. Isso se tor­na claro ao se estudar cuidadosamente o assunto.

\section{A restituição por sacrilégio}



comeu acrescido de um quin 0\textsuperscript{156}. E te preceito está
expresso em Suas palavras, enaltecido seja Ele, "E pagará o da coisa
sagrada pela qual pecou, acrescen­tando a quinta parte" (Levítico 5:16)
e "E o homem que comer santidade por er­ro, acrescentará a quinta parte
do valor desta" (Ibid. 22:14). As normas deste pre­ceito estão
explicadas no Tratado Meilá, e também no Tratado Terumot.

\section{A colheita do quarto ano}

Por este preceito somos ordenados a considerar todo o fruto da co­lheita
do quarto ano como sagrado. Este preceito está expresso em Suas
pala­vras "E no quarto ano, será todo o seu fruto santidade de louvores
ao Eterno" (Levítico 19:24). A lei prevê que o proprietário deve levá-lo
a Jerusalém e comê-lo lá, como no caso do Segundo Dízimo. Os ``Cohanim''
não recebem nenhuma parte dele.

O Sifrei diz: " 'E as s tida es de todo o ho ele serão' (Núme-

ros : as Escrituras sep as Coisa g s deram aos 'Co-

ha ceto a Ação de raç Sacrifíci sacrifício de `Pes-

sa Dízimo do Gad i \textsuperscript{159}, o do Dízi ita do quarto ano

da as, os quais decl o • ertencerem aos prietarios."

As normas deste preceito estão explicadas integralmente no último
capítulo do Tratado Maasser Sheni.

\section{``peá'' para os pobres}

Por este preceito somos ordenados a deixar "peábiói
de. ereais, fru­tas e similares. Ele está expresso em
Suas palavras, enaltecidos a Ele, "Para
 pobre e o imigrante os deixarás" (Levítico 19:10).

No Tratado Macot está explicado que ``peá'' envolve um preceito negativo
justaposto a um preceito positivo. O preceito negativo está expresso
em Suas palavras ``Não acabarás de segar o canto de teu campo'' (Ibid. 19.)
e o positivo está em Suas palavras ``Para o pobre e o peregrino os deixarás''.

As normas deste preceito estão explicadas no Tratado Peá. A Torah
limita sua aplicação à Terra de Israel.


\section{A respiga para os pobres}

Por este preceito somos ordenados a deixar respigas. Ele está expresso
em Stias palavras "Nem colhereis a respiga de vossa ceifa; para o pobres
e para
 imigrante os deixareis" (Levítico 23:22).


Este preceito também envolve um preceito negativo justaposto um preceito
positivo, como explicado no Tratado Macot com relação à "pe ".163


\begin{enumerate}
\def\labelenumi{\arabic{enumi}.}
\setcounter{enumi}{155}
\item
 
 Um quinto do valor daquilo que deve restituir.
 
\item
 
 Ver preceito positivo 66.
 
\item
 
 Ver preceito positivo 55 e 56.
 
\item
 
 Ver preceito positivo 78.
 
\item
 
 Ver preceito positivo 128.
 
\item
 
 Sobras abandonadas nos campos para os pobres.
 
\item
 
 Ver o preceito negativo 210.
 
\item
 
 Ver o preceito negativo 211.
 
\end{enumerate}

As normas deste preceito estão explicadas no Tratado Peá. A Torah limita
sua aplicação à Terra de Israel. •

\section{Deixar a gavela esquecida para os pobres}

Por este preceito somos ordenados a deixar a gavela esquecida. Este
preceito está expresso em Suas palavras, enaltecido seja Ele, "Quando
segares a messe no teu campo, e esqueceres uma gavela, no campo, não
voltarás a toma-la; para o imigrante, o órfão, e a viúva será"
(Deuteronômio 24:19). As palavras "Para o imigrante, o órfão, e a viúva
será" constituem o preceito positivo de

deixa vela esquecida, assim como as palavras ``Os deixarás'' (Levítico
19:10)

cons o preceito positivo, no caso das respigas e da ``peá'', como foi ex-


licação deste preceito também está limitada pela Torah à Terra de

Isr

As normas deste preceito também estão explicadas no Tratado Peá.


\section{Deixar as sobras dos cachos de uva para os pobres}

Por este preceito somos ordenados a deixar para os pobres os ca­chos de
uva que sobrarem no momento da colheita, chamados ``olelot''. Tam­bém a
respeito deles as Escrituras dizem: "Para o pobre e o imigrante os
deixa­rás". (Levítico 19:10).

As normas deste preceito estão explicadas no Tratado Peá, e a Torah
limita sua aplicação à Terra de Israel.

\section{Deixar as uvas caídas para os pobres}


Por este preceito cair e ficar separado do c palavras "E o bago de t
deixarás." (Levítico 19:

As normas det e a Torah limita sua apli

mos ordenados a deixar para os pobres o que urante a vindima. Ele está
expresso em Suas não recolherás; para o pobre e o imigrante os

deste preceito estão explicadas no Tratado Peá, à Terra de Israel.

\section{Levar as primícias ao santuário}


Por este preceito somos ordenados a separar as primícias e levá-las ao
Santuário. Ele está expresso em Suas palavras, enaltecido seja Ele, "O
princí­pio das primícias de tua terra trarás à casa do Eterno, teu
Deus." (Êxodo 23:19).


\begin{enumerate}
\def\labelenumi{\arabic{enumi}.}
\setcounter{enumi}{163}
\item
 
 Neste caso, bem como no caso dos preceitos 120 e 121, as leis
 rabínicas determinam que estas obrigações também devem ser cumpridas
 fora de Israel (Mishné Torah, Leis das Dádivas aos Po­bres, Cap. 1,
 Lei 14).
 
\item
 
 Ver preceito negativo 213.
 
\end{enumerate}




Está claro que este preceito só é obrigatório durante existe cia do
Santuário, e que se aplica apenas às ``Sete Categorias'' de prod
•s\textsuperscript{166} q crescem na Ter­ra de Israel, na Síria e na
Transjordânia.

As normas deste preceito estão explicadas atado Bicurim, on-


de foi deixado claro que elas, ou seja, as primícias, são propriedades
do ``Cohen''.


\section{A grande oferta de elevação}

Por este preceito somos ordenados a separar a grande Oferta de
Ele­vação. Ele está expresso em Suas palavras, enaltecido seja Ele, "As
primícias de teu grão... darás a ele" (Deuteronômio, 18:4). De acordo
com a Torah, ele é obrigatório apenas na Terra de Israel, e suas normas
estão explicadas no Tra­tado Terumot.

\section{O primeiro dízimo}

Por este preceito somos ordenados a separar o dízimo do produto da
terra. Ele está expresso em Suas palavras, enaltecido seja Ele, "Porque
os dízimos dos filhos de Israel, que separarem ao Eterno em oferta"
(Números 18:24). As Escrituras explicam que este dízimo pertence aos
Levitas.

As normas deste preceito estão explicadas no Tratado Maasserot. Ele é
chamado de o Primeiro Dízimo, e a Torah só o torna obrigatório dentro da
Terra de Israel.

\section{O segundo dízimo}

Por este preceito somos ordenados a separar o segundo dízimo. Ele está
expresso em Suas palavras, enaltecido seja Ele, "Certamente separarás o
dízimo de todo o produto das tuas sementes, que o campo produzir de ano
a ano." (Deuteronômio 14:22) sobre as quais diz o Sifrei: " 'Ano a ano':
isto nos ensina que os dízimos não devem ser deixados de um ano para o
outro. Contudo, as palavras das Escrituras se referem apenas ao segundo
dízimo; co­mo saber que elas devem ser aplicadas aos outros dízimos
também? Porque a Torah diz: 'Certamente separarás o dízimo etc' ".

A Torah expõe claramente que este dízimo deve ser levado a Jerusa­lém,
para lá ser comido pelo seu proprietário. Nós já nos referimos ao que os
Sábios dizem a este respeito.

As Escrituras dão as leis deste preceito em detalhes, dizendo que
quan­do é impossível levá-lo a Jerusalém devido à distância, ele deve
resgatá-lo e le­var seu valor em dinheiro ao Santuário e ali gastá-lo
exclusivamente com comi­da. Isso está estipulado em Suas palavras,
enaltecido seja Ele, "E se o caminho

te for comprido, de sorte que não o possas levar, por longe de ti," etc.

(Deuteronômio 14:24). Outra norma estabelecida pela que se ele o res-

gatar para si próprio, ele deverá acrescentar um qui
to\textsuperscript{167}. .to está determinado
em Suas palavras, enaltecido seja Ele, "E se quiser a remir o seu
dízi-


\begin{enumerate}
\def\labelenumi{\arabic{enumi}.}
\setcounter{enumi}{165}
\item
 
 Os sete tipos de produtos pelos quais a terra de Israel era famosa:
 trigo, cevada, uvas, figos, romãs, óleo de oliva, e mel de tâmara.
 
\item
 
 Um quinto de seu valor.
 
\end{enumerate}




Está claro que este preceito só é obrigatório durante existê cia do
Santuário, e que se aplica apenas às ``Sete Categorias'' de prod s" q
crescem na Ter-ra de Israel, na Síria e na Transjordânia.

As normas deste preceito estão explicadas atado Bicurim, on-


de foi deixado claro que elas, ou seja, as primícias, são propriedades
do ``Cohen''.


\section{A grande oferta de elevação}

Por este preceito somos ordenados a separar a grande Oferta de
Ele­vação. Ele está expresso em Suas palavras, enaltecido seja Ele, "As
primícias de teu grão... darás a ele" (Deuteronômio, 18:4). De acordo
com a Torah, ele é obrigatório apenas na Terra de Israel, e suas normas
estão explicadas no Tra­tado Terumot.

\section{O primeiro dízimo}

Por este preceito somos ordenados a separar o dízimo do produto da
terra. Ele está expresso em Suas palavras, enaltecido seja Ele, "Porque
os dízimos dos filhos de Israel, que separarem ao Eterno em oferta"
(Números 18:24). As Escrituras explicam que este dízimo pertence aos
Levitas.

As normas deste preceito estão explicadas no Tratado Maasserot. Ele é
chamado de o Primeiro Dízimo, e a Torah só o torna obrigatório dentro da
Terra de Israel.

\section{O segundo dízimo}

Por este preceito somos ordenados a separar o segundo dízimo. Ele está
expresso em Suas palavras, enaltecido seja Ele, "Certamente separarás o
dízimo de todo o produto das tuas sementes, que o campo produzir de ano
a ano." (Deuteronômio 14:22) sobre as quais diz o Sifrei: " 'Ano a ano':
isto nos ensina que os dízimos não devem ser deixados de um ano para o
outro. Contudo, as palavras das Escrituras se referem apenas ao segundo
dízimo; co­mo saber que elas devem ser aplicadas aos outros dízimos
também? Porque a Torah diz: 'Certamente separarás o dízimo etc' ".

A Torah expõe claramente que este dízimo deve ser levado a Jerusa­lém,
para lá ser comido pelo seu proprietário. Nós já nos referimos ao que os
Sábios dizem a este respeito.

As Escrituras dão as leis deste preceito em detalhes, dizendo que
quan­do é impossível levá-lo a Jerusalém devido à distância, ele deve
resgatá-lo e le­var seu valor em dinheiro ao Santuário e ali gastá-lo
exclusivamente com comi­da. Isso está estipulado em Suas palavras,
enaltecido seja Ele, "E se o caminho te for comprido, de sorte que não o
possas levar, por longe de ti," etc. (Deuteronômio 14:24). Outra norma
estabelecida pela orah ' que se ele o res­gatar para si próprio, ele
deverá acrescentar um qui to\textsuperscript{167} to está determi­nado
em Suas palavras, enaltecido seja Ele, "E se quiser pess a remir o seu
dízi-

{,L}


\begin{enumerate}
\def\labelenumi{\arabic{enumi}.}
\setcounter{enumi}{165}
\item
 
 Os sete tipos de produtos pelos quais a terra de Israel era famosa:
 trigo, cevada, uvas, figos, romãs, óleo de oliva, e mel de tâmara.
 
\item
 
 Um quinto de seu valor.
 
\end{enumerate}

mo, acrescentar-lhe-á a quinta de seu preço" (Levítico 27:31). Todas as

regras detalhadas deste prece o esto explicadas no Tratado Maasser
Sheni.

Da mesma forma e é obr atório, pela Torah, apenas com relação aos
produtos da Terra de Isr. 1, e dev ser comido apenas durante a
existência do Santuário. O Sifrei diz: co para o ato de comer os
primogênitos ao segundo dízimo: assim como os pr .gênitos podem ser
comidos apenas du­rante a existência do Santuário, Q se: ndo dízimo
também só pode ser comido durante a existência do Santuário".

\section{O dízimo dos levitas para os ``cohanim'' ou a oferta de elevação}

Por este preceito os Levitas são ordenados a separar um dízimo do dízimo
que eles receberam de Israel, e a dá-lo aos ``Cohanim''. Este preceito
está expresso em Suas palavras, enaltecido seja Ele, "E aos Levitas
falarás e lhes dirás: Quando tomardes dos filhos de Israel o dízimo que
deles vos dei: por vossa herança, dele separareis, uma oferta para o
Eterno, o dízimo do dízimo" (Números 18:26). As Escrituras explicam que
este dízimo, que é chamado de Oferta de Elevação do Dízimo, deve ser
dado ao "Cohen' : "E dareis deles a oferta separada do Eterno, para
Aarão, o 'Cohen' " (Ibid., 28).

As Escrituras explicam que este dízimo deve ser retirado da melhor e
mais selecionada parte: "De todo o melhor delas, a parte consagrada que
lhe é consagrada" (Ibid., 29). As Escrituras ressaltam então que eles
cometem uma transgressão se não fizerem a seleção dentre a melhor parte:
"Não levareis so­bre vós por isso, pecado, quando separardes o melhor
dele" (Ibid., 32). Este é um preceito negativo de exclusão, como se ele
tivesse dito: "Não haverá pe­cado quando separardes do melhor". Daí
deduzimos que se separarmos do pior haverá pecado, e portanto este é um
preceito negativo derivado de um precei­to positivo, e portanto não está
contado entre os preceitos negativos: quer di­zer, o preceito de fazer a
seleção entre o que há de melhor implica que a esco­lha não deve ser
feita dentre a pior parte. O Sifrei diz: "De que modo você con­clui que
se fizerem a seleção de outra parte que não a melhor eles cometem um
pecado? Porque as Escrituras dizem: 'E não levareis sobre vós por isso,
pe­cado, quando separardes o melhor dele' ".

As normas deste preceito estão explicadas nos Tratados Terumot e
Maasserot, e em diversos lugares de Demai.

\section{O dízimo do homem pobre}

Por este preceito somos orden dos a se rar o dízimo para os po­bres no
terceiro ano de cada ciclo de Sha i at\textsuperscript{169}, e • vamente
no terceiro ano depois de cada terceiro ano, ou seja, no se o ano de
cada ciclo de Shabat. Este preceito está expresso em Suas palavras, e
altecido seja Ele, "Ao fim de três anos tirarás todos os dízimos de teu
produt b " etc. (Deuteronômio 14:28).

A Torah também torna este preceito obrigatório não somente na Terra de
Israel. Suas normas estão explicadas nos Tratados Peá e Maasserot, e vá-


\begin{enumerate}
\def\labelenumi{\arabic{enumi}.}
\setcounter{enumi}{167}
\item
 
 A Torah.
 
\item
 
 Deve-se deixar que a terra repouse a cada sete anos, formando assim o
 ciclo de Shabat.
 
\end{enumerate}




rios assuntos ligados a ele estão espalhados em vários trechos dos
outros Trata­dos de Zeraim, e nos Tratados Makhshirin e .Yadayim.

\section{A declaração do dízimo}

Por este preceito somos ordenados a declarar diante d'Ele, enalteci­do
seja Ele, que separamos os dízimos obrigatórios e as Ofertas de
Elevação, e a verbalmente declarar que estamos liberados de nossas
obrigações, assim co­mo nos exoneramos delas de fato. Este preceito,
chamado ``a Declaração do Dízimo'', está expresso em Suas palavras,
enaltecido seja Ele, "E dirás diante do Eterno, teu Deus: Tirei o que é
consagrado, de minha casa, e também o dei ao Levita, e ao imigrante, e
ao órfão, e à viúva" (Deuteronômio 26:13).

As normas deste preceito, a maneira de proceder a separação e o seu
significado estão explicadas no último capítulo de Maasser Sheni.

\section{A narração ao levar as primícias}

Por este preceito somos ordenados, ao levar as primícias, a narrar as
bondades que Deus, enaltecido seja Ele, nos concedeu, como Ele nos
liber­tou dos sofrimentos de nosso Patriarca Jacob, e da escravidão e
opressão dos Egípcios; a agradecer-Lhe por isso; e a implorar-Lhe para
perpetuar Suas bên­çãos. Este preceito está expresso em Suas palavras,
enaltecido seja Ele, "Falarás em voz alta e dirás diante do Eterno, teu
Deus: Arameu errante era meu pai" (Deuteronômio 26:5) e todo o resto
desta passagem. Este preceito é chamado de o Relato das Primícias. Suas
normas estão explicadas no Tratado Bicurim e no sétimo capítulo de Sotá.
Ele não é obrigatório para as mulheres.

\section{A oferta de massa}

Por este preceito somos ordenados a separar uma Torta (Halá) de ca­da
massa, e dá-la ao ``Cohen''. Este preceito está expresso em Suas palavras,
enaltecido seja Ele, "Em primeiro lugar, separareis de vossas massas,
uma torta; como oferta da eira" (Números 15:20).

As normas deste preceito estão explicadas nos Tratados Halá e Orlá, e a
Torah nos obriga a ele somente na Terra de Israel.

\section{Renunciar à produção de sua propriedade no ano de shaba}

Por este preceito • mos ord nados a renunciar a toda a produção de
nossas terras no ano de S batl", e permitir que qualquer pessoa recolha
tudo o que cresce em nossos pos. E te preceito está expresso em Suas
pala­vras, enaltecido seja Ele, "E n - • • e deixa-la-ás de cultivar,
deixa-la-ás de adu­bar e limpar" (Êxodo 23:11).

170. O sétimo ano no ciclo de sete anos

A Mekhiltá diz: "Se o vinhedo e o olival estão incluídos, porque eles
estão mencionados especificamente? Para servirem como analogia: assim
co­mo a obrigação é um preceito positivo específico, cuja violação
acarreta tam­bém a transgressão de um preceito negativo, a violação de
qualquer preceito positivo acarreta a infração de um preceito negativo".

O significado disto é o que explicarei a seguir. O preceito "No séti­mo
deixa-la-ás de cultivar, deixa-la-ás de adubar e limpar" abrange toda a
pro­dução da terra durante o Ano de Shabat: figos, uvas, azeitonas,
pêssegos, ro­mãs, trigo, cevada, e os outros. Conseqüentemente, é um
preceito positivo tra­tar todos esses tipos de produtos dentro dos
termos da lei de Shabat. Mas as Escrituras depois especificam: "Assim
farás com tua vinha e teu olival", embo­ra estes já estejam incluídos em
``todos'' os produtos da terra. O preceito não se aplic ecificamente ao
vinhed • o olival, mas nos é ordenado por causa

da adv ia das Escrituras qua ecolher o produto do vinhedo, a qual

está n ras "As uvas sepa9 ti, da tua vinha, não colherás" (Levíti-

co 25 ssim como no cas hedo, em que é um preceito positivo

decla dono e não faz m preceito negativo, assim está clara-

mente sso que, no caso •e udo que crescer no sétimo ano, é um pre-

ceito positivo declará-lo sem dono eixar de fazê-lo é um preceito
negativo.

Portanto, o caso do olival é o mesmo que o do vinhedo no que se refere a
um preceito positivo e um negativo, e o caso do olival é o mesmo que o
dos outros produtos. Portanto ficou claro por tudo o que foi exposto que
a renúncia a toda a produção do sétimo ano é um preceito positivo.

As normas deste preceito estão explicadas no Tratado Shebiit e a To­rah
só o torna obrigatório para a produção da Terra de Israel.

\section{O pousio da terra durante o ano de shabat}

Por este preceito somos ordenados a deixar de cultivar a terra du­rante
o sétimo ano. Ele está expresso em Suas palavras, enaltecido seja Ele,
"Mes­mo no tempo de arar e ceifar descansarás" (Êxodo 34:21). Ele está
repetido vá­rias vezes, como em Suas palavras "E no sétimo ano, sábado
de descanso para a terra" (Levítico 25:4). Nós já mencionamos que, de
acordo com as palavras dos Sábios, benditas sejam suas memórias, a
palavra ``descanso'' (Shabaton) de­termina um preceito positivo. E Ele,
louvado seja, também diz "Descansará a terra, descanso em nome do
Eterno" (Ibid., 2).

As normas deste preceito estão explicadas no Tratado Shebiit e a To­rah
não o torna obrigatório a não ser na Terra de Israel.

\section{Santificar o ano do jubileu (50 anos)}

Por este preceito somos ordenados a santificar o quinquagésimo ano, ou
seja, a deixar de cultivar a terra durante esse ano, assim como no Ano
de Shabat. Este preceito está expresso em Suas palavras, enaltecido seja
Ele, ``E santificareis o ano quinquagésimo'' (Levítico 25:10), sobre as
quais se comen­ta: "A Lei do Sétimo Ano é a mesma da do Jubileu"; ou
séja, as Escrituras as colo-


\begin{enumerate}
\def\labelenumi{\arabic{enumi}.}
\setcounter{enumi}{170}
\item
 
 Ver o preceito negativo 223.
 
\item
 
 Não fazê-lo é transgredir um preceito negativo.
 
\end{enumerate}




cam em pé de igualdade com relação a relação ao negativo, como vou
explic

As leis do Jubileu e do Ano at são iguais no que se refere a

deixar de cultivar a terra e a declarar sem o tudo o que nela crescer.
Essas

duas leis estão compreendidas em Suas palavras "E santificareis o ano
quinqua­gésimo". As Escrituras explicam que sua santidade consiste em
não ter dono seus frutos e produtos, estando essa obrigação expressa em
Suas palavras "Por­que jubileu é ele; santidade será para vós; do campo
comereis seu produto" (Ibid. 25:12).

O Jubileu é observado apenas na Terra de Israel, e com a condição de que
cada tribo permaneça em seu próprio lugar, ou seja, que cada uma
per­maneça no seu território da Terra de Israel, e que não se misturem
umas com as outras.

os ordenados a fazer soar o ``Shofar'' no décimo proclamar por toda nossa
terra a liberdade dos escravos e a liberação, s ento, de todo escravo
hebreu nesse déèimo dia de ``Tishri''. Este preceito está expresso em Suas
palavras, enaltecido seja Ele, "E farás soar a voz do Shofar' aos dez
dias do sétimo mês; no dia das expiações fa­reis soar o Shofar' em toda
a vossa terra" (Levítico 25:9) e em Suas palavras "E proclamareis
liberdade em toda a terra, para todos os seus moradores" (Ibid., 10).

Tem sido explicado que o Jubileu é como Rosh Hashaná no que se refere a
fazer soar o ``Shofar'' e às Bençãos. As normas relativas a fazer soar o
``Shofar'' em Rosh Hashaná estão explicadas no Tratado Rosh Hashaná.

É sabido que a intenção de fazer soar o ``Shofar'' no ano do Jubileu é
divulgar amplamente a libertação, e é parte da proclamação, como aparece
em Suas palavras ``E proclamareis a liberdade em toda a terra''. Sua
finalidade é diferente em Rosh Hashaná, quando se faz soar o ``Shofar''
como ``lembrança diante do Eterno'', enquanto que no Jubileu é pela
liberação dos escravos, co­mo explicamos.


\textbf{138 A DEVOLUÇÃO DA ERRA\\
NO ANO DO JU E}


Por este preceito somos ord ados a qu todas as terras compradas retornem
aos seus proprietários nesse no\textsuperscript{175}, e e elas sejam
entregues pe­. los compradores sem compensação m rietária. E e preceito
está expresso em Suas palavras, enaltecido seja Ele: "Em toda- rra de
vossa possessão, reden­ção concedereis à terra" (Levítico 2 5 : 2 4),
tendo ficado claro pelas Suas Pala­vras "Neste ano do jubileu, voltareis
cada um a sua possessão" (Ibid., 13) que o resgate deve ocorrer nesse
Ano.


\begin{enumerate}
\def\labelenumi{\arabic{enumi}.}
\setcounter{enumi}{172}
\item
 
 Ver os preceitos negativos 223 e 226.
 
\item
 
 O ano do Jubileu
 
\item
 
 O ano do Jubileu
 
\end{enumerate}

As Escrituras explicam as regras detalhadas deste preceito, e deixam
claro os direitos do vendedor e do comprador caso ele queira resgatar
sua hé­rança antes do início do Ano do Jubileu. Também deixam claro que
esta lei es­pecífica se aplica apenas às terras localizadas fora das
muralhas de uma cidade, e que aldeias e casas construídas em campo
aberto estão sob a mesma lei, uma vez que não estão dentro das muralhas,
assim como pomares e jardins. Essas são as ``casas das aldeias'' a
respeito das quais as Escrituras dizem: "E as casas das aldeias que não
têm muro ao redor, como os campos da terra serão consi­deradas; redenção
haverá para elas e no Jubileu sairão do poder do compra­dor" (Ibid.,
31).

As normas deste preceito estão explicadas em Arakhin. Ele também só é
obrigatório na Terra de Israel, e apenas enquanto a lei do Jubileu
estiver em aplicação.

\section{O resgate de propriedades dentro das muralhas da cidade}

Por este preceito somos ordenados a que o resgate das posses ven­didas
dentro das muralhas de uma cidade seja válido apenas por um ano
com­pleto e que depois de um ano elas se tornem propriedade do
comprador, e não sejam devolvidas no Jubileu. Este preceito está
expresso em Suas palavras, enal­tecido seja Ele, "E quando o homem
vender casa de moradia numa cidade mu­rada" (Levítico 25:29).

Este preceito é a ``lei das casas numa cidade murada''. Suas normas estão
explicadas no Tratado Arakhin, e ele é obrigatório apenas na Terra de
Israel.

\section{Contar os anos até o jubileu}

Por este preceito somos ordenados a contar os anos a partir do mo­mento
em que conquistamos a terra e nos tornamos seus proprietários, uma
contagem feita em ciclos de sete anos, até o Ano do Jubileu. Este
preceito, ou seja, a contagem dos anos dos ciclos Sabáticos, é de
responsabilidade do Tribu­nal, ou seja, do Grande Sanhedrin, o qual deve
contar os cinqüenta anos, ano a ano, assim como cada um de nós tem que
contar os dias do ``omer''. Este preceito está expresso em Suas palavras,
enaltecido seja Ele, ``Contarás para ti sete semanas de anos'' (Levítico
25:8).

A Sifrá diz: "Poder-se-ia pensar que se poderia contar os sete anos de
Shabat sucessivos e proclamar o Jubileu. Por isso a Torah diz 'Sete
vezes sete anos'. Esses são dois versículos e a lei só pode ser
compreendida através dos dois juntos". Quer dizer, a maneira como este
preceito deve ser executado só pode ser entendida através dos dois
versículos: o Sanhedrin tem que contar os anos e os ciclos de Shabat ao
mesmo tempo.

Uma vez que as Escrituras dizem que a lei só pode ser entendida atra­vés
dos dois versículos, conclui-se que, definitivamente, só há um preceito;
por­que se houvesse dois preceitos --- um para a contagem dos anos e
outro para a contagem dos ciclos sabáticos --- não haveria razão pata
dizer "A não ser atra­vés dos dois versículos juntos", porque dois
preceitos são sempre derivados cada um de um versículo, e a expressão "A
não ser através dos dois versículos juntos" só é usada com relação a um
único preceito, cujas normas só podem



ser compreendidas por completo através de dois textos. Um exemplo disso
é o caso do primogênito, onde as Escrituras dizem "Todo o que abre a
matriz será para mim" (Exodo 34:19), ensinando-nos que todo primogênito,
macho ou fêmea, pertence ao Eterno; o versículo "Separarás... macho,
para o Eterno" (Ibid., 13:12) nos ensina que todos os machos, sejam
primogênitos ou não, per­tencem ao Eterno; e a partir desses dois
versículos concluímos o significado do preceito --- que ele se aplica
apenas ao primogênito macho --- como está explicado na Mekhiltá.

\section{Cancelar as dívidas no ano de shabat}

Por este preceito somos ordenados a cancelar todas as nossas
recla­mações de dinheiro no Ano de Shabat. Ele está expresso em Suas
palavras, enal­tecido seja Ele, "O que tiveres em poder de teu irmão, o
deixarás" (Deuteronô­mio 15:3) e em Suas palavras "Este é o modo do ano
sabático: que todo o cre­dor, que emprestou a seu companheiro, o
deixará" (Ibid., 2).

A Tosseftá diz: "As Escrituras falam de dois tipos de desistência: a
desistência de terra e a desistência de dinheiro".

A Torah ordena a desistência de dinheiro apenas quando a lei refe­rente
à desistência de terras estiver em vigência, e nesse momento ela a torna
obrigatória em todo lugar.

As normas deste preceito estão explicadas no último capítulo do Tra­tado
Shebiit.

\section{Cobrar as dívidas dos idólatras}

Por este preceito somos ordenados a cobrar as dívidas do id e a
pressioná-lo para que pague, da mesma forma que somos ordenad
misericordiosos para com o israelita e proibidos de exigir o pagamento
Ele está expresso em Suas palavras, enaltecido seja Ele, "Do estrangeir
tra reclamarás" (Deuteronômio, 15:3), a respeito das quais diz o Sifrei:
estrangeiro idólatra reclamarás' é um preceito positivo".

\section{A parte do ``cohen'' de cada animal puro que se abate}

Por este preceito somos ordenados a dar ao ``Cohen'' o quarto dian­teiro,
as duas faces e o estômago de todo animal puro que abatermos. Este
pre­ceito está expresso em Suas palavras, enaltecido seja Ele, "E este
será o direito dos `Cohanim' sobre o povo: os que oferecerem sacrifício,
seja boi ou cordei­ro" (Deuteronômio 18:3).

As normas deste preceito estão explicadas no décimo capítulo de Hu­lin;
ele não é obrigatório para os Levitas.

\section{A primeira tosquia deve ser dada ao ``cohen''}

Por este preceito somos ordenados a separar a primeira tosquia e dá-la
ao ``Cohen''. Este preceito está expresso em Suas palavras, enaltecido
seja Ele, "A primícia da tosquia de tuas ovelhas, darás a ele"
(Deuteronômio, 18:4) e é obrigatório apenas na Terra de Israel. Suas
normas estão explicadas no décimo primeiro capítulo de Hulin.

\section{As coisas consagradas}

Por este preceito somos ordenados quanto à lei das coisas consagra­das,
ou seja, aquele que consagrar alguma coisa que lhe pertence, dizendo:
"Seja isto consagrad\&', deve entregá-la ao ``Cohen'', a menos que ele
acrescente ex­plicitamente ``a Deus'', e nesse caso ele deve entregá-la
para ser guardada no Santuário, pois tudo o que for declarado
consagrado, em termos gerais, perten­ce ao ``Cohen''. Este preceito está
expresso em Suas palavras, enaltecido seja Ele, "No entanto, toda
consagração que uma pessoa fizer ao Eterno, de tudo o que lhe pertencer,
seja homem, ou animal etc." (Levítico 27:28).

Suas palavras "E será o campo, quando sair livre 'no Jubileu, santida­de
ao Eterno, como campo consagrado para o 'Cohen'; a possessão dele
per­tencerá aos 'Cohanim' " (Ibid., 21) mostram que todas as coisas
consagradas em termos gerais pertencem ao ``Cohen''.

As normas deste preceito estão explicadas no oitavo capítulo de
Arak­hin, e no início de Nedarim.

\section{``Shehitá''}

Por este preceito somos ordenados a matar os animais de uma ma­neira
determinada, antes de comer sua carne, pois só assim ela será alimento
permitido. Este preceito está expresso em Suas palavras, enaltecido seja
Ele, "Poderás degolar do teu gado, e do teu rebanho, ... como te ordenei
(Deutero­nômio 12:21), a respeito das quais diz o Sifrei: " oderás
degolar': tal como as ofertas consagradas devem ser abatidas de a
m...leira determinada, assim também os animais
abatidos como alimento d vem se k abatidos dessa maneira. `Como te
ordenei' nos ensina que Moisés f i orderQdo quanto ao esôfago e à
traquéia, e quanto à maior parte de um del, s\textsuperscript{177} no
pássaros e à maior parte de ambos no gado".

Todas as normas e leis sobre este pre o estão explicadas no Trata­do que
lida especificamente com este assunto, que é o Tratado Hulin.

\section{Cobrir o sangue de pássaros e animais abatidos}

Por este preceito somos ordenados a cobrir o sangue de um pássaro ou
animal depois de abatido. Este preceito está expresso em Suas palavras,
enal­tecido seja Ele, "Derramará o seu sangue e o cobrirá com pó."
(Levítico 17:13).




As normas deste preceito estão explicadas no sexto capítulo de Hulin.


\section{Liberar a mãe quando se pegar seus filhotes}

Por este preceito somos ordenados a deixar partir do ninho. Ele está
expresso em Suas palavras, enaltecido seja Ele, "Mas deixarás ir
livremente a mãe, e os filhos poderás tomar para ti" (Deuteronômio
22:7).


As normas deste preceito estão explicadas no último capítulo de


Hulin.

\section{Procurar os sinais de pureza determinados no gado e nos animais}

Por este preceito somos ordenados a procurar certos sinais em ani­mais
domésticos e selvagens, ou seja, que eles ruminem o alimento e tenham o
casco totalmente fendido, pois isso faz deles alimento permitido.
Procurar esses sinais nos animais é um preceito positivo, expresso em
Suas palavras, enal­tecido seja Ele, ``Estes são os animais que comereis''
(Levítico 11:2).

A Sifrá diz: " 'Esses comereis' (Ibid., 3): apenas 'esses' podem ser
co­midos, e não os animais impuros"; quer dizer, todo animal que tiver
esses si­nais é alimento permitido, e o animal que não os tiver é
proibido. Temos aqui um preceito negativo derivado de um preceito
positivo, que tem força de um preceito positivo, de acordo com o
princípio que explicamos. É por essa razão que depois dessa sentença a
Sifrá o4z: "Sei apenas que há um preceito positivo; de que modo eu
concluo que ta '.bé = preceito negativo? Porque o Tal-

mud diz: 'O camelô, que rumin casco fendido' " (Ibid., 4), como\\
explicarei nos preceitos negativ

Portanto, ficou claro avras ``Esses comereis'' são um pre-

ceito positivo, cujo significado , como e pliquei, que somos obrigados a
pro­curar esses sinais em todo animal, e doméstico ou selvagem, e só
então ele pode ser comido. O que o preceito prescreve é a observação
deste pro­cedimento.


As normas deste preceito estão explicadas nos Tratados Bekhorot


e Hulin.

\section{Procurar os sinais de pureza determinados nos pássaros}

Por este preceito somos ordenados a procurar os sinais nos pássa­ros,
pois apenas alguns deles são alimento permitido. No caso dos pássaros,
os sinais não estão estipulados na Torah, mas foram obtidos através de
estudo. Quan­do examinamos todos os tipos declarados individualmente
proibidos, encon­tramos certos elementos comuns a todos eles e esses são
os sinais dos pássaros

178. Ver o preceito negativo 172.


146

MAIMÔNIDES


impu preceito, de examinar os pássaros e determinar que um é pu-

ro e d impuro, é um preceito positivo.

Sifrei diz: " 'Toda ave pura, podereis comer' (Deuteronômio 14:11)este é
um preceito positivo. Ficou claro, portanto, o que nós assinala­mos
acima.

As normas deste preceito estão explicadas no Tratado Hulin.

\section{Procurar os sinais de pureza determinados nos gafanhotos}

Por este preceito somos ordenados quanto aos sinais também nos
gafanhotos. Eles estão descritos na Torah com as seguintes palavras:
``Que tem pernas por cima dos pés'' (Levítico 11:21).

A explicação que demos sobre o preceito anterior também é válida para
este. O versículo das Escrituras que se refere a ele é: "De todo o
réptil alado... deles comereis estes" (Ibid., 21-22).

As normas deste preceito estão explicadas no terceiro capítulo do
Tratado Hulin.


\section{Procurar os sinais de pureza determinados nos peixes}


Por este preceito somos ordenados quanto aos sinais nos peixes, que
estão expressos na Torah, em Suas palavras "Isto comereis de tudo o que
está nas águas " etc. (Levítico 11:9). A Guemará de Hulin diz
explicitamente: "Aquele que come um peixe impuro viola um preceito
positivo e um preceito negati­vo", já que de Suas palavras "Isto
comereis" eu concluo que outros peixes não devem ser comidos, e um
preceito negativo derivado de um preceito positivo tem força de preceito
positivo. Fica, portanto, claro que as palavras "Isto co­mereis" são um
preceito positivo. Isto significa, como eu disse, que somos or­denados a
decidir, baseados nesses sinais, que um peixe é alimento permitido e
outro não, como as Escrituras dizem claramente: "E fareis separação
entre o quadrúpede puro e o impuro" (Levítico 20:25).

A separação só pode ser feita através dos sinais, e portanto os sinais
de cada uma das quatro categorias --- animais domésticos e selvagens,
pássaros, gafanhotos e peixes --- constituem um preceito separado e
diferente. Já mostra­mos que cada um deles foi considerado como um
preceito positivo.

As normas deste preceito --- a saber, o preceito referente aos sinais
nos peixes --- também estão explicadas no terceiro capítulo do Tratado
Hulin.

\section{Determinar a lua nova}

Por este preceito o Enaltecido nos ordena quanto ao cálculo dos me­ses e
dos anos. Este é o ``preceito da Santificação da Lua Nova'' e ele está
ex­presso em Suas palavras, enaltecido seja Ele, "Este mês seja para
vós, princípio dos meses" (Êxodo 12:2). Para explicar isto está dito nos
escritos: "Este teste-



munho será entregue a vocês"; ou seja, este preceito não é imposto a
todos, como é o caso do Shabat, da Criação, quando todas as pessoas são
obrigadas a contar seis dias, e a descansar no sétimo. Não cabe a cada
um, ao ver a lua nova, decidir que esse dia é o primeiro do mês, nem
fixar o primeiro dia do mês baseado em cálculos aprendidos, nem
intercalar um mês baseado numa primavera tardia ou em outras
considerações que mereçam ser levadas em con­ta. Esse dever nunca deve
ser cumprido por ninguém a não ser o Grande Tri­bunal e deve ser
executado na Terra de Israel e em nenhum outro lugar. Por­tanto, como
não existe nenhum Grande Tribunal hoje em dia, a observação cessou
atualmente entre nós porque não há nenhum Grande Tribunal, assim como
cessou a oferenda de sacrifícios porque o Templo não existe mais.

Neste ponto enganaram-se os descrentes, que aqui no Oriente são chamados
Caraítas; e até alguns Rabanitas não conseguiram perceber o signifi­cado
deste ponto e começaram a tatear na mais densa escuridão a esse
respeito. Vocês devem saber que não se permite que os cálculos que
fazemos hoje em dia, e pelos quais podemos determinar as luas novas e as
festividades, sejam feitos em algum outro lugar a não ser na Terra de
Israel; mas num caso de emer­gência, e na ausência dos Sábios da Terra
de Israel, foi permitido que um Tri-

bunal, que tenha sido habilitado na Ter Israel, intercale um mês no ano

e determine luas novas fora da Terra d orno o Talmud registra ter feito
 Rabi Akiba. Esse procedimento, co 1 udo, e tá repleto de grandes
 dificulda-


des, e é sabido que quase sempre hou erra de Israel, e que foram eles,

quando estiveram reunidos, que dete m as luas novas e intercalaram um

mês nos anos, de acordo com os m to corretos.

Um grande e fundamental s rincípio de nossa fé, que não pode ser
conhecido ou entendido corretamente a não ser através de reflexão
profunda, é que quando hoje em dia, estando fora da Teria de Israel,
calculamos pela ta­bela de ano bissexto que temos em mãos, e
determinamos que um dia é de Lua Nova e um outro é de Festival, nós o
fazemos não com base em nosso próprio cálculo, e sim porque o Grande
Tribunal da Terra de Israel designou esse dia como o primeiro do mês, ou
como um dia de Festival. Esse dia se tornou o primeiro dia do mês ou um
dia de Festival porque eles assim o decretaram, quer tenha sido sua
decisão baseada em cálculos ou na observação. Isso está de acordo com
nossa Tradição, que interpreta o versículo "Estas são as solenidades do
Eterno, as santas convocações que proclamareis no seu tempo determinado"
(Levítico 23:4) como significando:"Eu não conheço outras solenidades a
não ser essas", ou seja, aquelas que foram declaradas como sendo
``solenidades'', ainda que isso tenha sido feito involuntariamente, ou sob
coação, ou por enga­no, como a Tradição nos diz. Hoje nós fazemos
cálculos apenas para saber que dia foi fixado pelos habitantes da Terra
de Israel, pois é por este método e não pela observação da lua nova que
eles determinam e estabelecem atualmente.
 na decisão deles que confiamos e não nos nossos cálculos, que nada
 mais são do que constatações. Isto deve ser bem compreendido.


Darei uma explicação adicional sobre este assunto. Suponhamos, por
exemplo, que os habitantes da Terra de Israel desaparecessem --- que
Deus nos livre de tal coisa, pois Ele nos prometeu que não tirará nem
apagará da terra
 remanescente da nação --- e que não houvesse mais Tribunal lá, e que
 fora da Terra não houvesse mais nenhum Tribunal que tivesse sido
 habilitado na Terra de Israel: nesse caso nossos cálculos não nos
 seriam de nenhuma utilidade


\textbf{180.} Quase sempre houve Sábios na Terra de Israel.


porque não dev novas, a não Tzion sairá a L diz a este respei e não
deixa dúvi

zer cálculos fora da Terra, nem intercalar ou fixar luas condições
mencionadas, como explicamos, pois "De alguém em sã consciência examinar
o que o Talmud ará a conclusão de que nossa interpretação está correta


Nas escrituras há indicações que estabelecem um funda to para

os princípios em que nos baseamos para reconhecer as luas no s anos

bissextos. Baseado num desses trechos, mais especificamente, " rás es-

te estatuto em seu prazo, de ano em ano" (Êxodo 13:10), foi d to nos

ensina que não devemos acrescentar nenhum mês a não ser n do ano

próxima às solenidades". Foi dito ainda: "De que modo conclui e só de-

vemos acrescentar um dia ao mês ou santificar a lua nova durante o dia?
Pelas palavras das Escrituras `mi-yamim yamima', a duração de um ano em
`dias' ". Sobre as palavras das Escrituras "Este mês seja para vós,
princípio dos meses" (Ibid. 12:2) foi dito: "Você calcula um ano pelos
meses mas não pelos dias", significando que o que se deve acrescentar é
um mês completo. Também foi dito, a respeito das palavras ``Porém um mês''
(Números 11:20): "Você calcula um mês pelos dias, não pelas horas"; e
também foi explicado que o versículo "Estejas alerta desde antes que
chegue o mês da primavera" (Deuteronômio 16:1) nos ensina que em nossos
anos devemos levar em consideração as esta­ções do ano, e que portanto
eles devem ser anos solares.

Todas as normas deste preceito estão explicadas na íntegra no pri­meiro
capítulo de Sanhedrin, no Tratado Rosh Hashaná, e também em Berakhot.

\section{Descansar no shabat}

Por este preceito somos ordenados a descansar no Shabat. Ele está
expresso em Suas palavras ``E no sétimo dia descansarás'' (Êxodo 34:21), e
está repetido várias vezes; o Enaltecido nos diz que descansar de todo
trabalho é uma obrigação aplicável a nós, a nosso gado, e a nossos
empregados.

As normas deste preceito estão explicadas no Tratado Shabat e no Tratado
Yom Tob.

\section{Proclamar a santidade do shabat}

Por este preceito somos ordenados a recitar determinadas palavras no
início e no final do Shabat, mencionando a grandeza e a alta nobreza do
dia, e a diferença desse dia com relação aos dias da semana que o
precedem e os que o sucedem. Este preceito está expresso em Suas
palavras, enaltecido seja Ele, "Estejas lembrado do dia de sábado para
santificá-lo" (Êxodo 20:8): ou seja, comemorá-lo proclamando sua
santidade e sua grandeza. Este é o pre­ceito de ``Kidush'', santificação.
A Mekhiltá diz: " `Estejas lembrado do dia de sábado para santificá-lo':
santificá-lo com uma bênção". E os Sábios dizem ex­plicitamente:
"Recorda-o sobre o vinho", e também: "Santifica-o na sua chega­da e na
sua partida", referindo-se à ``habdala'', que faz parte de recordar o
Sha­bat como nos foi ordenado.


\begin{enumerate}
\def\labelenumi{\arabic{enumi}.}
\setcounter{enumi}{180}
\item
 
 Isa. 2:3.
 
\item
 
 Foi dito na Mekhiltá.
 
\end{enumerate}




As normas deste preceito estão explicadas no final de Pessahim e em
vários trechos de Berakhot e de Shabat.

\section{Retirar o fermento}

Por este preceito somos ordenados a remover o fermento de nossas\\
propriedades no décimo quarto dia de Nissan. Este é o preceito da
Retirada do\\
Fermento, e ele está expresso em Suas palavras, enaltecido seja Ele,
"Mas no\\
primeiro dia cessareis de ter fermento em vossas casas" (Êxodo 12:15).
Os Sábios
também o chamam de ``a queima do pão'' (bi'ur), ou seja, o ato de
queimar
o pão fermentado. A Guemará de Sanhedrin no Talmud de Jerusalém
diz:\\
"O pão fermentado envolve ambos um preceito positivo e um negativo: o
preceito
positivo é queimá-lo --- 'Cessareis de ter fermento em vossas
casas' --- e\\
o negativo está em 'Levedura não será encontrada em vossas casas' "
(Ibid., 19).\\
As normas deste preceito estão explicadas no início de Pessahim.


\section{Narrar o êxodo do egito}

Por este preceito somos ordenados a narrar a história do Êxodo do Egito,
com toda a eloqüência de que formos capazes, na véspera do décimo quinto
dia de Nissan. Deverá ser elogiado aquele que discorrer sobre este tema,
contan­do a miséria que nos impuseram os Egípcios e os sofrimentos que
eles nos cau­saram, e sobre a maneira como o Eterno Se vingou deles,
agradecendo-Lhe, enal­tecido seja Ele, por todo o bem com que Ele nos
brindou; como foi dito, "To­dos aqueles que narram longamente a fuga do
Egito merecem ser louvados."

Nas Escrituras este preceito está expresso em Suas palavras, enalteci­do
seja Ele, ``E anunciarás a teu filho naquele dia'' (Êxodo 13:8). Sobre
isto foi comentado: " 'E anunciarás a teu filho': poder-se-ia pensar que
se deve contar a história do primeiro dia do mês em diante; por isso a
Torah diz 'Naquele dia'. As palavras 'Naquele dia' poderiam ser
interpretadas como significando duran­te o dia; por isso a Torah diz
'Por isto' --- uma expressão que não seria usada a não ser no momento em
que o pão ázimo e as ervas amargas estivessem dian­te de você".
Portanto, é uma obrigação contá-la somente depois do anoitecer.

A Mekhiltá diz: "Como foi dito que 'E será quando te perguntar teu filho
amanhã etc.' (Ibid., 14) poder-se-ia pensar que você deve narrar a
história a seu filho apenas se ele lhe perguntar. Por isso as Escrituras
dizem: 'Anunciarás a teu filho' --- mesmo que ele não pergunte.
Novamente poder-se-ia pensar que isto só se aplica àquele que tiver um
filho; de que forma concluímos que ele se aplica também a quem estiver
só ou entre pessoas estranhas? Pelas palavras das Escrituras: 'E disse
Moisés ao povo: Recordai este dia' (Ibid.,3)"; ou seja, Deus nos mandou
que recordassemos, tal como ele nos ordenou com as pala­vras "Estejas
lembrado do dia de sábado para santificá-lo" (Ibid., 20:8).

Você já está familiarizado com as palavras: "Mesmo se fôssemos to­dos
eruditos, homens com conhecimentos, versados na Lei, ainda assim seria
nossa obrigação narrar a fuga do Egito".


As normas deste preceito estão explicadas no final de Pessahim.


\section{Comer pão ázimo na véspera do décimo quinto dia de nissan}

Por este preceito somos ordenados a comer pão ázimo na véspera do décimo
quinto dia de Nissan, quer tenha o cordeiro de ``Pessah'' sido oferecido ou não. Ele está expresso em Suas palavras, enaltecido seja Ele,
"Pela noi­te, comereis pães ázimos" (Êxodo 12:18), sobre as quais
comentam: " 'Pela noite, comereis pães ázimos': as Escrituras apresentam
isto co ma obrigação."

Está explicado em Pessahim que comer pã•obrigatório na primeira noite
(da festividade), e opcional depois dis

As normas desse preceito estão explicadas atado Pessahim.

\section{Descansar no primeiro dia de ``pessah''}

Por este preceito somos ordenados a descansar no primeiro dia de
``Pessah''. Ele está expresso em Suas palavras, enaltecido seja Ele, "O
primeiro dia, de santa convocação será para vós" (Levítico 23:7).
Inicialmente, você de­ve saber que toda vez que o Eterno ordenou
.uma santa convocação" isso é

interpretado como significando que o dia deve ser cado, o que quer di-

zer que nenhum tipo de trabalho deve ser feito nel ser o que se relacio-

na com comida, como está explicado nas Escritur

Nós já nos referimos ao fato de que foi r ito q e a palavra "Shaba­ton",
``descanso solene'' constitui um preceito pos como se Ele tivesse dito
``descansa'' ou ``descansarás'', que são maneiras de impor a abstenção do
trabalho. Todos os dias das ``épocas determinadas'', ou seja, dos
Festivais, são chamados de ``Sábados do Eterno'' (Levítico 23:38).

Está afirmado em vários trechos do Talmud que "há um preceito po­sitivo
e um preceito negativo com relação aos Festivais"; quer dizer, abster-se
do trabalho durante todos os Festivais é um preceito positivo, e
executar traba­lhos proibidos durante um Festival é um preceito
negativo. De acordo com is­so, todo aquele que fizer um trabalho
proibido estará violando ambos um pre­ceito positivo e um negativo.

As normas deste preceito, ou seja, o de descansar, estão explicadas no
Tratado Yom Tob.

\section{Descansar no sétimo dia de ``pessah''}

Por este preceito somos ordenados a descansar no sétimo dia de
"Pes­sah". Ele está expresso em Suas palavras, enaltecido seja Ele, "O
sétimo dia, de santa convocação é" (Levítico 23:8).

\section{Contar o ``omer''}

Por este preceito somos ordenados a contar o ``omer''. Ele está ex­presso
em Suas palavras, enaltecido seja Ele, "E contareis para vós desde o dia
seguinte ao primeiro dia festivo, desde o dia em que tiverdes trazido o
`omer' da movimentação; sete semanas completas serão" (Levítico 23:15).


\begin{enumerate}
\def\labelenumi{\arabic{enumi}.}
\setcounter{enumi}{182}
\item
 
 Não é obrigatório comer-se pão ázimo nos outros dias de "Pessah , mas
 é proibido comer-se pão levedado.
 
\item
 
 Em Êxodo 12:16.
 
\end{enumerate}




Você deve saber que da mesma forma que o Tribunal é obrigado a contar os
anos do Jubileu, ano por ano, e ciclo de Shabat por ciclo de Shabat,
como expliquei anteriormente, assim também nós somos obrigados a contar
os dias do ``omer'', dia por dia, e semana por semana, como está
determinado em Suas palavras ``Contareis cinqüenta dias'' (Levítico 23:16)
e ``Sete semanas contarás para ti'' (Deuteronômio 16:9).

Assim como contar os anos e os ciclos de Shabat constitui um único
preceito, também no caso do ``omer'' um só preceito determina a contagem.
Todos os meus antecessores o contaram corretamente como um preceito; e
você não deve se deixar confundir pelas palavras: "É obrigatório contar
os dias, e é obrigatório contar as semanas" e considerar que há dois
preceitos separados, porque quando o preceito tem várias partes, o
cumprimento de cada uma delas é um preceito. Contudo, haveria dois
preceitos se tivesse sido dito: "Contar os dias é um preceito, e contar
as semanas é um preceito". Isto não passará desapercebido a alguém que
costuma usar as palavras no seu sentido preciso. Se você disser "é
obrigatório fazer isto e aquilo", isto não significa necessaria­mente
que essa determinada ação seja um preceito separado.

Há uma prova clara disto no fato de que ao contar o ``omer'' nós
enumeramos as semanas todas as noites, dizendo "tantas semanas e tantos
dias", enquanto que se as semanas fossem um preceito separado,
deveríamos men­cionar o número de semanas apenas nas noites em que se
completa cada sema­na; e nesse caso haveria duas bênçãos: uma "Bendito
sejas, o Eterno ... que ... nos ordenastes contar os 'dias' do `omer' "
e outra "contar as 'semanas' do `omer' ". Mas não é esse o caso, e o
preceito determina a contagem do ``omer'', seus dias e suas semanas, de
acordo com Seu preceito.

Este preceito não é obrigatório para as mulheres.

\section{Descansar no dia de ``shabuot''}

Por este preceito somos ordenados a descansar no dia de ``Shabuot''. Ele
está expresso em Suas palavras "E proclamareis nesse mesmo dia, e haverá
para vós convocação de santidade" (Levítico 23:21).

\section{Descansar no dia de ``rosh hashanáá''}

Por este preceito somos ordenados a descansar no primeiro dia de
``Thishri''. Ele está expresso em Suas palavras "No sétimo mês, será para
vós descanso solene" (Levítico 23:24).

Nós já mencionamos que foi dito que a palavra ``Shabaton'', "des­canso
solene", constitui um preceito positivo.

\section{Jejuar no dia de ``yom quipur''}

Por este preceito somos ordenados a jejuar no décimo dia do "This­hri".
Ele está expresso em Suas palavras, enaltecido.seja
Ele, "Afligireis vossas almas " etc. (Levítico 16:29)„ que a Sifrá
interpreta desta forma: " 'Afligireis vossas 
almas': aflição com relação àquilo de que dependa a vida, ou seja,
abstinên­cia de comer e beber".

A tradição também proíbe lavar-se, untar-se, usar sapatos e manter
relações conjugais e diz que devemos cessar todas essas atividades
porque, co­mo foi dito, "Sábado solene é para vós, e afligireis vossas
almas" (Ibid., 31), o que equivale a dizer que é obrigatória a
abstinência tanto de trabalho de toda espécie como de alimentação e
cuidados com o corpo, e é isso que se usa a expressão ``Shabat Shabaton''
--- ``um Shabat de descanso solene''.

A Sifrá diz: "De que forma concluímos que é proibido lavar-se, untar-se
e manter relações conjugais em Tom Quipur'? Pelas palavras da Sifrá:
'Sába­do solene'; ou seja, devemos abster-nos de todas essas coisas a
ponto que elas nos aflijam.

\section{Descansar no dia de ``yom quipur''}

Por este preceito somos ordenados a descansar do trabalho de to­dos os
tipos nesse dia. Ele está expresso em Suas palavras " Shabat Shabaton' é
para vós" (Levítico 16:31).

Já explicamos várias vezes que foi dito que a expressão ``Shabaton'',
``descanso solene'', constitui um preceito positivo.

\section{Descansar no primeiro dia de ``sucot''}

Por este preceito somos ordenados a descansar no primeiro dia da Festa
dos Tabernáculos. Ele está expresso em Suas palavras "No primeiro dia
haverá santa convocação" (Levítico 23:35).

\section{Descansar no dia de ``shemini atzeret''}

Por este preceito somos ordenados a descansar no oitavo dia da Fes­ta
dos Tabernáculos. Ele está expresso em Suas palavras, enaltecido seja
Ele,

o o vo dia, haverá santa convocação para vós" (Levítico 23:36).


Você deve saber que a mesma lei se aplica a cada um dos seis


as\textsuperscript{185}, d rante os quais somos obrigados a descansar, e
que nenhum deles está

eito a Igum tipo de restrição que não se aplique aos outros. Também
pode-

o parar comida em cada um deles. Portanto, as mesmas regras com rela­ção
ao ``descanso'' se aplicam a todos os Festivais. Todas as normas a esse
res­peito estão explicadas no Tratado Yom Tob.

Contudo, deve ser ressaltado que o descanso imposto em Shabat e em "Yom
Quipur" acarreta todas as abstinências e muitas outras mais, já que
nesses dois dias não podemos preparar comida. Há outras coisas que nos
são permitidas num dia de Festival e que nos são proibidas no Shabat,
embora não estejam rela­cionadas com a preparação de comida, como está
explicado no Tratado Yom Tob.



\section{Morar numa cabana durante os dias de ``sucot''}

Por este preceito somos ordenados a morar numa cabana por sete dias,
durante toda a Festa. Ele está expresso em Suas palavras, enaltecido
seja Ele, ``Nas cabanas habitareis por sete dias'' (Levítico 23:42).

As normas deste preceito estão explicadas no Tratado Sucá. Ele não é
obrigatório para as mulheres.

\section{Pegar um ``lulav'' no ``sucot''}

Por este preceito somos ordenados a pegar um ramo de palmeira e a
alegrar-nos com ele diante do Eterno durante sete dias. Este preceito
está ex­presso em Suas palavras, enaltecido seja Ele, "E tomareis para
vós, no primeiro dia, (o fruto da árvore formosa, palmas de palmeira, e
ramos de murta e de sal­gueiro de ribeiras, e vos alegrareis diante do
Eterno vosso Deus, por sete dias.)" (Levítico 23:40).

As normas deste preceito estão explicadas no terceiro capítulo do
Tratado Sucá. Lá está explicado que é apenas no Santuário que este
preceito é obrigatório por sete dias; nos outros lugares ele é
obrigatório apenas no pri­meiro dia, de acordo com a Torah. Ele não é
obrigatório para as mulheres.

\section{Ouvir o ``shofar'' no dia de ``rosh hashaná''}

Por este preceito somos ordenados a ouvir o som do ``Shofar'' no primeiro
dia de ``Tishri''. Ele está expresso em Suas palavras "Dia de toque do
Shofar', será para vós" (Números 29:1).

As normas deste preceito estão explicadas no Tratado Rosh Hasha­ná. Ele
não é obrigatório para as mulheres.


\section{Dar meio ``shekel'' anual te}


or este preceito somos ordenados a dar meio "sheke os os

a o \textsuperscript{87}. El está expresso em Suas palavras, enaltecido
seja Ele: " a um

o sgate • e sua alma ao Eterno" (Êxodo 30:12), e ``Isto dará'' (Ib É aro •
e este preceito não é obrigatório para as mulheres, porque ras • izem:
"Cada um que passa para o número dos que são cont

As normas deste preceito estão explicadas no Tratado que 1 cificamente
com este assunto, a saber, o Tratado Shekalim. Lá está ex que o preceito
é obrigatório apenas durante a existência do Santuário.


\section{Acatar o que dizem os profetas}


Por este preceito somos ordenados a ouvir todo profeta, que a paz esteja
com eles, e a fazer o que quer que ele ordene, mesmo que isso seja con-


\begin{enumerate}
\def\labelenumi{\arabic{enumi}.}
\setcounter{enumi}{185}
\item
 
 Moeda de prata.
 
\item
 
 Ao Santuário.
 
\item
 
 E o senso militar não incluía mulheres.
 
\end{enumerate}

trário a um ou mais preceitos, desde que isso seja temporário e que não
repre­sente uma adição ou subtração permanente, como explicamos na
introdução de nosso ``Comentário sobre a Mishná''. O versículo das
Escrituras pelo qual Ele nos impõe isto, enaltecido seja Ele, é: "A ele
ouvireis" (Deuteronômio 18:15), sobre o qual o Sifrei diz: " 'A ele
ouvireis': ainda que ele lhe diga para violar temporariamente um dos
preceitos impostos pela Torah, você deve atendê-lo". Todo aquele que
transgredir este preceito está sujeito à pena de morte pela mão dos
Céus, como estipulado em Suas palavras, enaltecido seja Ele, "E qualquer
homem que não ouvir as minhas palavras, que ele falar em Meu Nome, Eu
lhe pedirei contas" (Ibid., 19).

Está dito em Sanhedrin: "Três pessoas estão sujeitas à morte pela mão
dos Céus: aquele que desobedece um profeta, um profeta que desobedece
seu próprio preceito, e o que oculta sua profecia". Tudo isto se deduz
das palavras "E qualquer homem que não ouvir elo yishma') as Minhas
palavras etc.", lendo-se ``lo yishma'' também como ``lo yishama'', "não
obedecerá", aplicável ao pro­feta que desobedecer seu próprio preceito,
e como ``lo yashmia'', ``não se fará ouvir'', aplicável ao profeta que
ocultar sua profecia.


As normas deste preceito estão explicadas no final de Sanhedrin.



\section{Nomear um rei.}

Por este preceito somos ordenados a nomear um rei sobre nós, um
Israelita, que unirá toda a nossa nação e' será nosso líder. Este
preceito está ex­presso em Suas palavras, enaltecido seja Ele, "Poderás,
certamente, pôr sobre ti o rei" (Deuteronômio 17:15).

Nós já nos referimos às palavras do Sifrei: "Três preceitos foram
im­postos aos Israelitas para quando eles chegassem à Terra de Israel:
nomear um rei para si mesmos, construir o Santuário, e aniquilar os
descendentes de Amalec".

O Sifrei diz ainda: " 'Poderás, certamente, pôr sobre ti o rei' é um dos
preceitos positivos", e explica que isso significa que ele deve ser
temido, e que nosso respeito por ele e estima pela sua grandeza e
supremacia devem

ser tão grandes que o coloquem num nível onra superior ao de todos os

profetas de sua geração. O Talmud diz ex ente: "O Rei tem prioridade

sobre o profeta"; e quando esse Rei der em que não for conflitante

com um preceito da Torah, nós deve er seu comando, e ele tem

o direito de matar com a espada todo a ele que o desobedecer. Nossos
ante-

passados aceitaram isso sobre si mesmo 'Todo aquele que se rebelar

contra teu comando... deverá ser mort a de todo aquele que se re-

belar contra a autoridade real, seja ele está entregue ao rei devida-

mente nomeado de acordo com a Torah.

As normas deste preceito estão explicadas no segundo capítulo de
Sanhedrin, no início de Queretot, e no sétimo capítulo de Sotá.

\section{Obedecer o grande tribunal}

Por este preceito somos ordenados a obedecer ao Grande Tribunal e a tudo
o que ele nos ordene com relação ao que é proibido e ao que é permi-



tido. Quanto a isso não há diferença entre uma decisão baseada na
Tradição,uma a qual eles tenham chegado pela aplicação de uma das leis
de interpretação da Torah, e uma sobre a qual eles concordaram a fim de
delimitar alguma determi­nação da Lei, ou a fim de ir ao encontro de
alguma situação através de uma medida que lhes pareça correta e
calculada, e que reforce a Torah: em todos esses casos somos obrigados a
executar o que eles decidirem e a agir de acordo com suas ordens --- não
podemos desobedecê-los. Este preceito está expresso em Suas palavras,
enaltecido seja Ele, "Conforme o mandado da lei que te ensi­narem"
(Deuteronômio 17:11), a respeito das quais diz o Sifrei: " 'Conforme o
juízo que te disserem, farás' é um preceito positivo".


As normas deste preceito estão explicadas no final de Sanhedrin.


\section{Aceitar a decisão da maioria}

Por este preceito somos ordenados a seguir a maioria caso haja uma
diferença de opinião entre os Sábios com relação a qualquer uma das leis
da Torah. Da mesma forma, se num litígio particular --- por exemplo, num
caso entre Reuben e Simeon --- surgir uma diferença de opiniões entre os
juízes da cidade quanto a ser Simeon ou Reuben o devedor, nós devemos
seguir a maio­ria. Este preceito está expresso em Suas palavras,
enaltecido seja Ele, "Inclina-te à maioria" (Êxodo 23:2). Foi dito
explicitamente: ``A maioria é lei decisiva''.

As normas e regulamentos deste preceito estão explicados em vá­rios
trechos de Sanhedrin.

\section{Nomear juízes e oficiais do tribunal}

Por este preceito somos ordenados a nomear juízes que devem im­por o
cumprimento dos preceitos da Torah, forçar aqueles que se desgarraram a
voltar a trilhar o caminho da verdade, ordenar que se execute o que é
bom e que se evite o que é ruim e aplicar as penalidades sobre os
transgressores, a fim de que os preceitos e as proibições da Torah não
fiquem entregues à von­tade do indivíduo.

Uma das condições deste preceito é que esses juízes devem ser de graus
diferentes, da seguinte forma. Para cada cidade com um número
suficien­te de habitantes se nomeia 23 juízes --- para constituir o
Sanhedrin Menor ---que devem se reunir todos no portão da cidade. Em
Jerusalém deve ser nomea­do o Grande Tribunal, com setenta juízes, e
acima deles o Chefe da Assembléia, que também é chamado pelos Sábios de
``Nassi'', e eles devem se reunir num local especificamente designado para
eles. Numa cidade cuja população seja nu­mericamente insuficiente para
ter um Sanhedrin Menor serão de qualquer for­ma nomeados três juízes que
deverão julgar os casos menores e enviar os casos importantes para o
Tribunal Superior.

Também deverão ser nomeados inspetores para visitar os mercados e
supervisionar a conduta das pessoas em suas transações, a fim de que
eles não cometam injustiças nem mesmo em assuntos insignificantes.

Este preceito está expresso em sua ordem, enaltecido seja Ele, "Juí­zes
e policiais, designarás para ti, em cada uma de tuas tribos"
(Deuteronômio 16:18). O Sifrei diz: "De que forma sabemos que devemos
nomear um tribunal para todo povo de Israel? Pelas palavras das
Escrituras: 'Juízes e policiais, desig­narás para ti'. De que forma
sabemos que devemos nomear um acima de todos 
os outros? Pelas palavras 'Designarás para ti'. De que forma sabemos
que deve ser nomeado um tribunal para cada tribo? Pelas palavras 'Em
cada uma das tuas portas'. Raban Shimeon ben Gamliel diz: " 'Em cada uma
de tuas por­tas; e julgarão': é obrigatório que cada tribo tenha seu
próprio tribunal porque a Torah diz: ``e julgarão o povo'' --- mesmo
contra sua vontade' ".

O preceito que nos ordena nomear 70 anciãos está repetido em Suas
palavras, enaltecido seja Ele, a Moisés, "Ajunta-me (li) setenta homens
dos an­ciãos de Israel " etc. (Números 11:16) e disseram que toda vez
que está dito para Mim (li), isso quer dizer que é para sempre.
Portanto, "E Me (li) servirão" (Êxodo 28:41) significa que este preceito
é obrigatório para sempre; não é um preceito temporário, mas sim
obrigatório de geração em geração.

Você deve saber que a nomeação de todos esses tribunais, a saber, o
Sanhedrin Maior e Menor, o Tribunal de Três, assim como as outras
nomea­ções, só podem ter lugar na Terra de Israel, ou não terão
validade. Mas os juízes ordenados na Terra de Israel podem julgar tanto
dentro como fora dela; e esse é o significado das palavras "O Sanhedrin
tem jurisdição dentro e fora da Ter­ra". Contudo, eles não podem julgar
casos de pena capital nem dentro nem fora da Terra a não ser durante a
existência do Santuário, como explicamos no início deste trabalho. A
respeito de Suas palavras, enaltecido seja Ele, --- relati­vas ao
homicídio acidental --- "E serão estes para vós por estatuto de
julgamen­to para as vossas gerações, em todas as vossas moradas"
(Números 35:29), o Sifrei diz: " 'Em todas as vossas moradas' significa
tanto na Terra como fora dela. Poderíamos pensar que as leis sobre as
Cidades de Refúgio também são obrigatórias fora da Terra; por isso a
Torah diz 'E serão estes etc.': estas leis, referentes aos Tribunais,
são obrigatórias tanto dentro como fora da Terra de Israel; as outras
referentes às Cidades de Refúgio, são obrigatórias apenas na Terra de
Israel". •


Todas as normas deste preceito estão explicadas no Tratado Sanhedrin.


\section{Tratar as partes com igualdade perante a lei}

Por este preceito os juízes são ordenados a tratar com igualdade to­das
as partes, e a permitir que cada um diga o que tem a dizer, quer ele
fale longa ou brevemente. Este preceito está expresso em Suas palavras
"Com justi­ça julgarás o teu próximo" (Levítico 19:15), que a Sifrá
explica da seguinte for­ma: "E proibido permitir a uma pessoa que diga
tudo o que quiser e ordenar a outra que seja breve." Este é um dos
aspectos incluídos neste preceito.

Outro aspecto dele é que todo homem que for conhecedor da Lei é obrigado
a proceder a um julgamento se as partes tiverem começado a argüir diante
dele. Os Sábios dizem explicitamente: "De acordo com as palavras da
Torah, até mesmo uma única pessoa tem competência para julgar casos de
dívi­da, pois está dito: 'Com justiça julgarás o teu
próximo--- .

Outro aspecto ainda é que um homem é obrigado a julgar seu próxi­mo com
uma inclinação em seu favor, e a sempre interpretar seus atos e
pala­vras como sendo bons e caritativos.


As intenções deste preceito estão explicadas em diversos trechos do


Talmud.



\section{Testemunhar no tribunal}

Por este preceito somos ordenados a dar ao Tribunal toda e qual­quer
prova que tivermos, quer ela arruíne a pessoa julgada ou salve sua vida
ou seu dinheiro. Somos obrigados a prestar testemunho sobre cada aspecto
e a dizer aos juízes o que vimos ou ouvimos. Os Sábios citam como prova
da obrigação de prestar testemunho as Suas palavras, enaltecido seja
Ele, "Sendo testemunha de um fato, por ter visto ou sabido" (Levítico
5:1). Aquele que vio­lar este preceito e ocultar provas comete pecado
grave, de acordo com Suas palavras, enaltecido seja Ele, "Se não o
denunciar, levará seu pecado" (Ibid.).

Este é o princípio geral. Contudo, se o testemunho que ele omitir é
relativo a dinheiro, e ele o omitir sob juramento, ele será obrigado a
oferecer um Sacrifício de Maior ou Menor Valor, como determinam as
Escrituras, de acor­do com as condições expostas em Shabuot.


As normas deste preceito estão expostas em Sanhedrin e em Shabuot.


\section{Investigar o depoimento das testemunhas}

Por este preceito somos ordenados a investigar os depoimentos pres­tados
pelas testemunhas e examiná-los cuidadosamente antes de aplicar um
cas­tigo ou apresentar uma decisão. Temos que ter a máxima cautela para
não che­gar a uma conclusão mal ponderada e precipitada que venha a
prejudicar um inocente. Este preceito está expresso em Suas palavras,
enaltecido seja Ele, "E indagarás, e investigarás, e perguntarás bem; e
se for verdade, e se for certa a coisa" (Deuteronômio 13:15).

As normas deste preceito e suas subdivisões --- como devem ser
con­duzidas as indagações e averiguações, quão cautelosos devemos ser e
de que forma as evidências devem ser aceitas ou rejeitadas, com base nas
investiga­ções --- estão explicadas no Tratado Sanhedrin.

\textbf{180} CONDENAR AS TESTEMUNHAS QUE PRESTAREM FALSO TESTEMUNHO

Por este preceito somos ordenados a punir as testemunhas que pres­tarem
falso testemunho com a pena que elas pensaram que seria aplicada pelo
seu testemunho. Este preceito está expresso em Suas palavras, enaltecido
se* Ele, ``Fareis a eles como pensavam fazer a seu irmão'' (Deuteronômio 1
• 9). Esta é a Lei do Falso Testemunho: se seu testemunho foi calculado
par. . nde-

nar a uma perda monetária, devemos aplicar-lhes uma perda no mes alor;

se foi calculado para condenar à morte, elas deverão morrer daquela
ra\textsuperscript{190};

e se foi calculado para condenar ao açoitamento, elas deverão sofrer é
igo.

As normas deste preceito, as dúvidas que surgiram com r lação a e e, e a
maneira de provar que os testemunhos são falsos, e que estão, portanto,
sujeitos a esta lei, estão explicadas no Tratado Macot.

\section{``E g lá arufá''}

Por este preceito somos ordenados a quebrar o pescoço de urna vaca se
encontrarmos num campo o corpo de um homem assassinado, e se não se
souber quem foi o assassino. Este preceito está expresso em Suas
pala­vras, enaltecido seja Ele,"Quando for achada uma pessoa
assassinada, caída no campo etc." (Deuteronômio 21:1). Esta é a Lei de
Quebrar o Pescoço de uma Vaca. Suas normas estão explicadas no último
capítulo do Tratado Sotá.

\section{Separar seis cidades de refúgio}

Por este preceito somos ordenados a separar seis Cidades de Refú­gio que
estejam prontas para receber a quem matar uma pessoa involuntaria­mente,
e a construir estradas que levem a elas e a nivelar essas estradas, não
deixando nelas nada que possa atrapalhar o fugitivo em sua fuga. Este
preceito está expresso em Suas palavras, enaltecido seja Ele,
"Prepararás o caminho e dividirás em três partes a área de tua terra
etc." (Deuteronômio 19:3).

As normas deste preceito estão explicadas em Sanhedrin, Macot, She­kalim
e Sotá. Nós já citamos do Sifrei que as leis relativas às Cidades de
Refúgio são obrigatórias apenas na Terra de Israel.

\section{Designar cidades para os levitas}

Por este preceito somos ordenados a dar aos Levitas cidades para que habitem
nelas\footnote{Elas eram em número de 42, independentemente das seis Cidades de Refúgio mencionadas no preceito anterior.}, porque que eles não receberam nenhum pedaço da
Terra. Ele e Suas 'palavras, enaltecido seja Ele, "Que dêem aos
Levitas... cidades para habitar'' (Números 35:2).

Essas cidades dos Levitas também eram usadas como Cidades de Refúgio, oferecendo asilo
sob condições especiais, como está explicado no Tratado Macot.

\section{Eliminar o perigo de nossas moradias}


Por este preceito somos ordenados a eliminar todos os obstáculos e
possibilidades de perigo dos lugares em que vivemos: ou seja, construir
mu­ros ou parapeitos nos telhados, poços, fossos e similares, para que
ninguém caia neles ou deles. Da mesma forma, toda estrutura perigosa
deve ser reconstruída ou consertada a fim de afastar todo tipo de
perigo. Este preceito está expresso em Suas palavras "Farás um parapeito
no teu telhado" (Deuteronômio 22:8),



a respeito das quais diz o Sifrei: " 'Farás um parapeito no teu telhado'
é um preceito positivo".


As normas deste preceito estão explicadas no Tratado Baba Kamma.


\section{Destruir todo tipo de idolatria na terra de israel}

Por este preceito somos ordenados a destruir todo tipo de idolatria e
seus templos por todas as maneiras possíveis de destruição e
aniquilação: que­brar, queimar, demolir e rasgar usando, para cada
objeto, o meio apropriado para que a destruição seja feita o mais
completa e rapidamente possível, pois a intenção é que não reste nem
traço dele. Isso está expresso em Suas palavras, enaltecido seja Ele,
``Certamente destruireis dos lugares'' (Deuteronômio 12:2), em "Mas assim
fareis com elas: seus altares derrubareis etc." (Ibid., 7:5) e
nova­mente em ``Porém seus altares derrubareis'' (Êxodo 34:13).

A Guemará de Sanhedrin registra casualmente que a menção de um preceito
positivo relativo à idolatria provocou a seguinte pergunta: "Como pode
conceber um preceito positivo com relação à idolatria?" E o Rabi isdá
citou, como explicação: ``Seus altares derrubareis''.

O Sifrei diz: "De que modo se conclui que se cortar uma
eraI\textsuperscript{92}

e se ela tornar a crescer dez vezes, deve-se cortá-la novamente? Pel.. p
alavras da Torah 'Certamente, destruireis' ". Também está dito ali: " 'E
far is • arecer os seus nomes daquele lugar' (Deuteronômio 12:3): o
preceito de destrui ido­latria se aplica apenas na Terra de Israel".

\section{A lei da cidade apóstata}

Por este preceito somos ordenados a matar todos os habitantes de uma
Cidade Apóstata e a queimá-la com tudo o que houver nela. Esta é a Lei
da Cidade Apóstata, e ela está expressa em Suas palavras, enaltecido
seja Ele, "E queimarás no fogo, a cidade e todo o seu despojo,
inteiramente" (Deuteronômio 13:17).


As normas deste preceito estão explicadas no Tratado Sanhedrin.


\section{A guerra contra as sete nações hereges}

os ordenados a exterminar as Sete Nações que orque eles constituíram a
raiz e primeiro fun­eito está expresso em Suas palavras, enaltecido
Deuteronômio 20:17). Está explicado em vários textos que o objetivo
disso era evitar que imitássemos sua heresia. Há vários trechos nas
Escrituras que nos incitam e insistem veementemente para que os
exterminemos, e a guerra contra eles é obrigatória.


Poder-se-ia pensar que este preceito nãò é obrigatório para sempre, uma
vez que as sete nações há muito deixaram de existir, mas essa idéia só
seria concebida por alguém que não tivesse compreendido a diferença
entre os pre-


\begin{enumerate}
\def\labelenumi{\arabic{enumi}.}
\setcounter{enumi}{191}
\item
 
 Uma árvore ou bosque devotado à idolatria:
 
\item
 
 I.e. os hiteus, os girgasheus, os emoreus, os cananeus, os periseus,
 os hiveus, os jebuseus, que eram os idólatras habitantes originais da
 Terra de Israel.
 
\end{enumerate}

ceitos que são obrigatórios através das gerações e os que não o são. Não
se po­de dizer que não seja obrigatório para sempre um preceito que
tenha sido cum­prido por completo --- alcançando seu objetivo --- mas
cujo cumprimento não tenha sido ligado a um limite determinado de tempo,
porque ele será obrigató­rio para cada geração em que surja a
possibilidade de executá-lo. Se o Eterno destriiísse e exterminasse
completamente os Amalequitas --- e que isso ocorra em breve em nossos
dias, de acordo com Sua promessa, enaltecido seja Ele, "Pois extinguirei
totalmente a memória de Amalec" (Êxodo 17:14) --- podería­mos então
dizer que o preceito ``Apagarás a memória de Amalec'' (Deuteronô­mio
25:19) não seria mais obrigatório através das gerações? Não poderíamos;
o preceito é obrigatório através das gerações, e enquanto existirem
descenden­tes de Amalec eles deverão ser eliminados. Da mesma forma, no
caso das Sete Nações, sua destruição e exterminação é obrigatória, assim
como a guerra con­tra elas: temos o dever de exterminá-las e
persegui-las através de todas as gera­ções até que elas sejam destruídas
completamente. Assim fizemos até que sua destruição foi completada por
Davi, e seus remanescentes se dispersaram e se misturaram com outras
nações de tal forma que não restou mais nenhum traço deles. Mas embora
elas tenham desaparecido, isso não significa que o preceito de
exterminá-las não seja obrigatório para sempre --- assim como não
podemos dizer que a guerra contra Amalec não é obrigatória para sempre
--- mesmo de­pois delas terem sido extinguidas e destruídas. Não há
nenhuma indicação es-

pecífica de tempo ou lugar este preceito, como no caso dos preceitos

especialmente estipulado m cumpridos no deserto ou no Egito. Ao

contrário, ele está ligado em ele é imposto, e eles devem cumpri-lo

enquanto exista algum

De um mod ropriado entender e discernir a diferença

entre um preceito e a oc sia espeito da qual ele nos é ordenado. Um
precei-

to pode ser obrigatório para sempre ainda que as ocasiões deixem de
existir durante um determinado período; mas a falta de ocasião não faz
com que ele deixe de ser um preceito obrigatório através das gerações.
Um preceito deixará de ser obrigatório para sempre quando a situação for
inversa, ou seja, quando tiver sido obrigação nossa executar, sob certas
condições, um determinado ato ou uma determinada ordem que não seja mais
obrigação nossa atualmente, em­bora essas condições ainda persistam. Um
exemplo disso é o caso do Levita idoso que foi desqualificado para o
serviço no deserto e que hoje está qualifica­do entre nós, como está
explicado no lugar apropriado. Você deve entender este princípio e
segui-lo à risca.

\section{A extinção de Amalec}

Por este preceito somos ordenados a exterminar, dentre os descen­dentes
de Esaú, apenas a semente de Amalec, homens e mulheres, jovens e
ve­lhos. Este preceito está expresso em Suas palavras, enaltecido seja
Ele, "Apaga­rás a memória de Amalec" (Deuteronômio 25:19).

Nós já citamos do Sifrei: "Três preceitos foram impostos aos Israeli­tas
quando eles entraram na Terra de Israel: nomear um rei para si mesmos,
construir o Santuário, e aniquilar os descendentes de Amalec". A guerra
contra Amalec também é obrigatória.


As normas deste preceito estão explicadas no oitavo capítulo de Sotá.


194. I.e., enquanto exista algum daqueles contra quem este preceito é
dirigido.



\section{Recordar os atos nefastos de Amalec}

Por este preceito somos ordenados a recordar o que Amalec nos fez quando
nos atacou sem ter sido provocado. Devemos falar nisso sempre e inci­tar
o povo a fazer guerra contra ele, e ordenar-lhe que o odeie, a fim de
que esse assunto não seja esquecido e que o ódio por ele não se
enfraqueça nem diminua com o passar do tempo. Este preceito está
expresso em Suas palavras, enaltecido seja Ele, "Recorda-te do que fez
Amalec" (Deuteronômio 25:17). A esse respeito diz o Sifrei: "
'Recorda-te do que te fez Amalec': com palavras; `Não te esquecerás'
(Ibid., 19): com o coração"; ou seja, você deve falar dessas coisas para
assegurar-se de que o ódio contra Amalec não seja afastado do cora­ção
dos homens. E a Sifrá diz: " 'Recorda-te do que te fez Amalec':
poder-se-ia pensar que isto significa 'Em teu coração'. Mas 'Não te
esquecerás' se refere

ao esquecimento do coração: então como se pode obedecer o prece'. e 're-

cordar'? Com a palavra". Veja como o profeta Samuel cumpriu e to:

primeiro ele recordou com palavras e depois deu ordens para qu fossem
r-tos, segundo Suas palavras "Eu recordo o que Amalec fez a I rae

\section{A lei da guerra não ob atória}

Por eito somos ordenados quanto a guerras não obrigató-

rias contra naç C'so entremos em guerra contra eles,.somos obrigados

a fazer um aco eles para poupar suas vidas se eles fizerem as pazes co-

nosco e nos ent eg em suas terras, e nesse caso eles deverão nos pagar
tribu­tos e ser nossos súditos. Este preceito está expresso em Suas
palavras, enalteci­do seja Ele, ``Te será tributário ou te servirá''
(Deuteronômio 20:11). A esse res­peito diz o Sifrei: "Se eles disserem
'nós concordamos com os tributos mas re­cusamos a servidão', ou,
'concordamos com a servidão, mas nos recusamos a pagar os tributos', não
devemos concordar: eles devem aceitar as duas condi­ções". Isto
significa que eles devem pagar um tributo anual a ser determinado pelo
rei daquela ocasião, e obedecer suas ordens com temor e humildade, co­mo
convém aos súditos. Contudo, se eles não fizerem a paz conosco, somos
ordenados a matar toda a população masculina, jovens e velhos, e a tomar
tudo o que lhes pertence, inclusive suas mulheres. Este preceito está
expresso em Suas palavras, enaltecido seja Ele, "E se não fizer paz
contigo etc." (Ibid. 12); tudo isso está na lei da guerra não
obrigatória.

As normas deste preceito estão explicadas no oitavo capítulo de So­tá, e
no segundo capítulo de Sanhedrin.

\section{Nomear um ``cohen'' para a guerra}

Por este preceito somos ordenados a nomear um ``Cohen'' para fa­zer ao
povo o discurso referente à guerra, quando eles forem partir para a
luta,


\begin{enumerate}
\def\labelenumi{\arabic{enumi}.}
\setcounter{enumi}{194}
\item
 
 1 Sam. 15:2.
 
\item
 
 Guerras contra outras nações além daquelas contra quem somos ordenados
 a guerrear.
 
\end{enumerate}

e para mandar de volta todo homem que não estiver apto para a batalha,
seja porque ele tem o coração fraco ou porque seus pensamentos estão
ocupados com alguma outra coisa que possa impedi-lo de se concentrar na
luta, ou seja, com uma das três coisas especificadas nas Escrituras. Só
depois disso é que eles devem entrar em luta. Esse ``Cohen'' é chamado de
"Mashuah Mil-Hama" para a Guerra. Em seu discurso ele deve dizer o que
está escrito na Torah e acres­centar palavras que incitem as pessoas à
guerra, e que as induzam a entregar suas vidas pelo triunfo da Fé no
Eterno, e pela punição dos ímpios que arrui­nam a ordem social. Este
preceito está expresso em Suas palavras, enaltecido seja Ele, "E quando
vos aproximardes à luta, o 'Cohen' se chegar-1" (Deutero­nômio 20:2).

O ``Cohen'' então ordena que seja proclamado nas linhas de comba­te que
devem retornar às suas casas todos aqueles que forem fracos de coração e
os que tiverem construído uma casa e não tenham morado nela, ou que
tive­rem plantado um vinhedo e não tenham comido seus frutos, ou que
tenham prometido casamento a uma mulher e não a tenham desposado. Isso
está de acordo com as palavras das Escrituras ``E falarão os policiais''
(Ibid., 5-8), sobre as quais a Guemará diz: "O 'Cohen' fala, e o
policial faz ouvir suas palavras".

Todo este procedimento --- o discurso do "Mashuah Mil-Hama" pa­ra a
guerra e sua proclamação entre as linhas de combate --- só é obrigatório
em caso de uma guerra não obrigatória, pois só a ela se aplica esta lei.
No caso de uma guerra obrigatória não há tal procedimento --- nem
discurso nem pro­clamação --- como se pode verificar no oitavo capítulo
de Sotá, onde as nor­mas deste preceito estão explicadas.

\section{Preparar um lugar separado do acampamento}


or este preceito somos ordenados a que quando nossas tropas fo-


rem erra devemos preparar um local fora do acampamento ao qual

eles ra que eles não o façam indiscriminadamente em qualquer lu-

gar o tendas, como fazem as outras nações. Este preceito está expres-

so em S .alavras, e tecido seja Ele, "E um lugar (yad) terás para ti,
fora

do acampamento, e sai 's fora" (Deuteronômio 23:13), sobre as quais o
Si-frei diz: " `Yad' sig ica ap nas um lugar, como está dito 'E vede!
ele instalou um lugar (yad) par. si'\textsuperscript{198}"

\section{Incluir uma estaca entre os utensílios de guerra}

Por este preceito somos ordenados a que cada homem do exército se muna
de um instrumento para cavar como parte de seus utensílios de guer­ra,
com a qual ele deverá cavar a terra e cobrir o excremento depois de ter
feito suas necessidades no local designado para esse fim, para que não
se veja ne­nhum traço dos excrementos no solo do acampamento, como Ele
ordenou no início do trecho que começa com as palavras "Quando te
acampares contra


\begin{enumerate}
\def\labelenumi{\arabic{enumi}.}
\setcounter{enumi}{196}
\item
 
 Para fazer suas necessidades.
 
\item
 
 I Sam. 15:12.
 
\end{enumerate}




os teus inimigos" (Deuteronômio 23:10). Este preceito está expresso em
Suas palavras, enaltecido seja Ele, "E uma estaca, terás para ti, entre
os objetos de teu uso (azenecha)" (Ibid., 14), sobre as quais o Sifrei
diz: " 'Azenecha' signifi­ca apenas o lugar de tuas armas".

\section{Um ladrão deve devolver o objeto roubado}

Por este preceito somos ordenados a fazer devolver o objeto que alguém
tenha roubado, se el. 'nda existir, acrescentando um
quinto de seu

valor, ou a reembolsar o se , caso ele tenha sofrido alguma alteração.
Es-

te preceito está expresso Su.. palavras, enaltecido seja Ele, "Devolverá
o

que roubou" (Levítico 5:

Está explicado no ado Macot que o preceito negativo referente

ao latrocínio é um precei • negativo justaposto a um preceito positivo.
"O Mi­sericordioso", diz, `ordenou `Não extorquirás' (Ibid., 19:13), e
'Devolverá o que roubou' ".

As normas deste preceito estão explicadas nos últimos capítulos de Baba
Kamma.

\section{``tsedaká''}

Por este preceito somos ordenados a dar "Tsed sustentar

os necessitados e a aliviar suas cargas. Este preceito está ex r várias
ma-

neiras em Suas palavras, como por exemplo "Abrirás tua mã teu irmão,

para teu pobre" (Deuteronômio 15:11), e também "Deterás sua decaída
mes­mo se ele é peregrino ou estrangeiro morador da terra e viverá
contigo" (Leví­tico 25:35), e ainda ``E viverá teu irmão contigo'' (Ibid.,
36). O significado de todas essas frases é o mesmo, ou seja, que devemos
ajudar os nossos pobres e sustentá-los de acordo com suas necessidades.

As normas deste preceito estão explicadas em vários lugares,
princi­palmente em Quetubot e em Baba Batra.

De acordo com a Tradição, mesmo o homem pobre que vive de "Tse­daká" tem
a obrigação de cumprir este preceito; quer dizer, ele deve dar
"Tse­daká", ainda que mínima, a alguém que seja mais pobre do que ele ou
tão po­bre quanto ele próprio.

\section{Bonificar o servo que recobrar sua liberdade}

Por este preceito somos ordenados a bonificar um servo e a ajudá-lo
quando ele for libertado, a fim de que ele não saia de mãos vazias. Este
preceito está expresso em Suas palavras, enaltecido seja Ele,
"Carrega-lo-ás, fornecendo-lhe do teu rebanho, de tua eira, e do teu
depósito de vinho; e do que te aben-


\begin{enumerate}
\def\labelenumi{\arabic{enumi}.}
\setcounter{enumi}{198}
\item
 
 A lei que diz que se deve acrescentar um quinto do valor se apliéa se
 o ladrão tiver jurado em falso a esse respeito.
 
\item
 
 Foi usado o termo Hebraico ``Tsedaká'' e não caridade, pois a raiz da
 palavra ``Tsedaká'' é ``Tsedek'', ou justiça, o que é mais forte do que
 fazer uma simples caridade. O preceito ordena fazer caridade no
 sentido de fazer justiça.
 
\end{enumerate}

çoou o Eterno, teu Deús, lhe darás."(Deuteronômio 15:14).


As normas deste preceito estão explicadas no primeiro capítulo de


Kidushin.


\textbf{197 EMPRESTAR DINHEIRO AOS POBRES}


Por este preceito somos ordenados a emprestar ao homem pobre para
ajudá-lo a aliviar sua situação. Esta é uma obrigação maior e de maior
peso do que a ``Tsedaká'' porque o mendigo, cuja necessidade o obriga a
pedir es­mola abertamente, não sofre tão grande angústia quanto aquele
que nunca teve que fazer isso e que precise de ajuda para que sua
pobreza não seja descoberta. Este preceito está expresso em Suas
palavras, enaltecido seja Ele, "Se empresta­res dinheiro a Meu povo, ao
pobre que está contigo" (Êxodo 22:24).

A Mekhiltá diz: "Todo `se' na Torah implica uma opção, com exce­ção de
três deles, um dos quais está no versículo `Se emprestares dinheiro a
Meu povo' ". ``Se emprestares dinheiro'', dizem os Sábios, "envolve uma
obri­gação. Caso você questione isto, e sugira que esta seja uma simples
permissão, as Escrituras dizem mais adiante `Lhe emprestarás o
suficiente para o que lhe faltar' (Deuteronômio 15:8), o que é uma
obrigação, não simplesmente uma ques­tão de opção".

As normas deste preceito também estão explicadas em vários trechos de
Quetubot e Baba Batra.

\textbf{198 COBRAR JUROS DO IDÓLATRA}

Por este preceito somos ordenados a exigir juros do dinheiro que
emprestarmos a um idólatra, de maneira a não ajudá-lo, nem ser amável
com ele, mas ao contrário, prejudicá-lo, mesmo se lhe fizermos um
empréstimo, o

que nos é proibido fazer no caso de um filho de Isr te preceito está ex-

presso em Suas palavras, enaltecido seja Ele, "Do iro poderás cobrar

juros" (Deuteronômio 23. 1), e de acordo com a i ção tradicional este

é um preceito positivo nã uma questão de op o é o que diz o Si-

frei a respeito: " 'Do E range o poderás cobrar j preceito positivo;

`E a teu irmão não pag ás'\textsuperscript{202} um preceito negati o ste
preceito também

há certas condições est ulad s pelos Sábios que estão explicadas no
Tratado Baba Metzia.

\textbf{199 DEVOLVER O PENHOR AO}

\textbf{PROPRIETÁRIO NECESSITADO}

Por este preceito somos ordenados a devolver um penhor ao seu
proprietário israelita no momento em que ele o necessitar. Quando o
penhor for algo de que ele precise durante o dia --- como por exemplo,
as ferramentas de seu trabalho ou ocupação --- ele lhe deve ser
devolvido para uso durante o dia, e guardado em caução durante a noite;
se for algo de que ele precise à noite --- como por exemplo roupa de
cama ou roupas côm as quais ele durma


\begin{enumerate}
\def\labelenumi{\arabic{enumi}.}
\setcounter{enumi}{200}
\item
 
 Aparentemente o autor traduz este versículo como significando "Você
 emprestará com juros".
 
\item
 
 Ver 0 preceito negativo 235.
 
\end{enumerate}




--- ele deve ser devolvido para ser usado à noite, e guardado em caução
duran­te o dia. A Mekhiltá diz: " 'Até pôr-se o sol a devolverás' (Êxodo
22:25) se refe­re a uma vestimenta usada durante o dia, que deve ser
devolvida para o dia inteiro. De que forma concluímos que uma vestimenta
usada durante a noite deve ser devolvida para a noite inteira? Pelas
palavras das Escrituras: 'Restituir-lhe-ás o penhor ao por-do-sol'
(Deuteronômio 24:13). Assim, conclui-se que uma vestimenta de dia deve
ser guardada como penhor durante a noite e de­volvida para ser usada
durante o dia, e que uma vestime noturna deve ser guardada como penhor
durante o dia e devolvida pa ada a noite.


Está explicado na Guemará de Macot que as pala ras "Não entra-


rás em sua casa para lhe tomar o seu penhor" (Ibid. tém um precei-

to negativo justaposto a um preceito positivo, estan o mo expresso em

Suas palavras "Restituir-lhe-ás". E o Sifrei diz: "As pala
estituir-lhe-ás' nos

ensinam que o que é usado durante o dia deve ser devolvido para o dia, e
o que é usado durante a noite deve ser devolvido para a noite: um
cobertor du­rante a noite, e um arado durante o dia".


As normas deste preceito estão explicadas no nono capítulo de Baba


Metzia.

\section{Pagar os soldos no dia}

Por este preceito somos ordenados a pagar a diária do trabalhador no
mesmo dia, e não adiar o pagamento para outro dia. Este preceito está
ex­presso em Suas palavras, enaltecido seja Ele, "No seu dia, lhe
pagarás a sua diá­ria" (Deuteronômio 24:15). 1 ac • do com as normas
deste preceito, um ope­rário que trabalha de dia p • - req erer seu
pagamento a qualquer momento da noite, e um trabalhado • turno a
qualquer momento do dia, como expli­carei nos preceitos negat
v.,\textsuperscript{204}.

As normas des e p eceit o estão explicadas no nono capítulo do Tra­tado
Baba Metzia, onde fica cl • • ue ele é obrigatório no caso dos
trabalhado­res diaristas, gentis ou Israelitas, e que é um preceito
positivo pagar no momento certo.

\section{Um empregado deve poder comer daquilo com que ele trabalha}

Por este preceito somos ordenados a permitir que um trabalhador coma
durante seu trabalho daquilo com o qual ele está trabalhando, desde que
isso ainda esteja unido ao solo. Este preceito está expresso em Suas
palavras, enaltecido seja Ele, "Quando entrares na vinha de teu
companheiro, poderás comer uvas... quando entrares na seara de teu
companheiro, poderás colher espigas com a tua mão" (Deuteronômio
23:25-26). A Guemará de Baba Metzia explica que desses dois versículos
nós deduzimos que ele pode comer daquilo com que ele estiver trabalhando
e que ainda esteja ligado ao solo; e que ne­nhum dos dois versículos é
suficiente sem o outro, como no caso mencionado


\begin{enumerate}
\def\labelenumi{\arabic{enumi}.}
\setcounter{enumi}{202}
\item
 
 Ver o preceito negativo 239.
 
\item
 
 Ver o preceito negativo 238.
 
\end{enumerate}

anteriormente, a respeito do qual citamos que "estes são dois textos
diferentes e a lei só pode ser compreendida através dos dois juntos".
Nesse caso o precei­to positivo de que um trabalhador deve ter permissão
para comer o que ainda está unido ao solo se deriva de dois versículos,
e os Sábios dizem claramente: "Eles podem comer de acordo com a lei das
Escrituras etc".

As normas deste preceito estão explicadas no sétimo capítulo do Tra­tado
Baba Metzia.

\section{Descarregar um animal cansado}

Por este preceito somos ordenados a descarregar um animal que te­nha
sucumbido no campo sob o peso de sua carga. Este preceito está expresso
em Suas palavras, enaltecido seja Ele, "Quando vires o asno daquele que
te abor­rece, prostrado debaixo de sua carga, não te recusarás a
ajudá-lo; auxiliá-lo-ás" (Êxodo 23:5), a respeito das quais a Mekhiltá
diz: " 'Auxiliá-lo-ás' se refere a descarregar o peso". Também está
escrito ali: "As palavras 'Aux' - o 's' nos ensinam que se infringe
ambos um preceito positivo e um nega vo". •u seja, somos ordenados a
descarregar o animal e somos proibidos• deixá-1 pros­trado sob sua
carga, como explicaremos nos preceitos negat os\textsuperscript{205} ; e
\textless{} quele que o deixar caído estará infringindo um preceito
positivo e u negati o. Por­tanto foi explicado que as palavras
"Auxiliá-lo-ás" contém um p ceito ositivo.

As normas deste preceito estão explicadas no segu d. pítulo de Baba
Metzia.

\section{Ajudar o próximo a levantar sua carga}

Por este preceito somos ordenados a carregar uma carga sobre o ani­mal
ou sobre o homem, sé ele estiver só, depois que ela tiver sido
descarregada por nós ou por outra pessoa. Assim como somos ordenados a
ajudar a descar­regar, somos ordenados a ajudar a carregar. Este
preceito está expresso em Suas palavras, enaltecido seja Ele, "Mas
ajudarás a levantá-los" (Deuteronômio 11:4), sobre as quais diz a
Mekhiltá: " 'Ajudarás a levantá-los' se refere a carregar".

As normas deste preceito estão explicadas no segundo capítulo de Baba
Metzia, onde foi deixado claro que a Torah,obriga tanto a carregar como
a descarregar.

\section{Devolver a seu dono o que ele tiver perdido}

Por este preceito somos ordenados a devolver a seu dono o que ele tiver
perdido. Este preceito está expresso em Suas palavras "Devolvê-lo-ás"
(Êxo­do 23:4) e ``Mas os restituirás a teu irmão'' (Deuteronômio 22:1). Os
Sábios di­zem explicitamente: "A devolução de um pertence perdido é um
preceito po­sitivo". Eles ainda dizem o seguinte, com respeito a um
pertence perdido: "So­mos ensinados que violamos um preceito positivo e
um negativo". Explicare-

205. ver o preceito negativo 270.

PRECEI • S POSITIVOS 167


ao negativo referente a pertences perdidos no lugar apro-


pr'ádo.ob.

A\textsubscript{)}5 normas deste preceito estão explicadas no segundo
capítulo de \textbf{Be}ba Metzia.

\textbf{205 REPREENDER O PECADOR}

Por este preceito somos ordenados a repreender quem estiver pe­cando ou
estiver inclinado a cometer um pecado, a fim de proibi-lo de agir des­sa
forma e a repreendê-lo. Um homem não pode dizer: "Eu não vou pecar; e se
outra pessoa pecar, isso é um assunto entre ele e Deus". Tal atitude é
contrá­ria à Torah. Somos ordenados a não pecar e não permitir que
ninguém de nos­sa nação o faça. Se alguém desejar pecar é dever de cada
um de nós repreendê-lo e evitar que ele o faça, ainda que não haja
evidência de que ele será castigado por isso. Este preceito está
expresso em Suas palavras, enaltecido seja Ele, 'Re­preenderás a teu
companheiro" (Levítico 19:17).

Está incluído neste preceito repreender aquele que nos tenha ofen­dido e
não lhe guardar rancor nem nutrir maus pensamentos a seu respeito. Somos
ordenados a repreendê-lo diretamente, para que nada perdure em nos­so
coração contra ele. A Sifrá diz: "Como sabemos que mesmo qúe já o
tenha­mos repreendido por quatro ou cinco vezes, ainda devemos tornar a
repreendê-lo? Porque a Torah diz `Repreenderás' --- ainda que mil vezes.
Poder-se-ia pen­sar que ao repreendê-lo se poderia humilhá-lo; o Talmud
diz 'E não levarás so­bre ti pecado' ".

Os Sábios explicam que este preceito é obrigatório para todos, de forma
que até um subalterno é obrigado a repreender um superior, e mesmo que
ele seja amaldiçoado e insultado ele não deve desistir nem deixar de
fazê-lo até que batam nele --- como disseram aqueles que transcreveram a
Tradição: ``Até que ele seja esbofeteado''.

As condições e regulamentos relativos a este preceito estão explica­das
em lugares dispersos do Talmud.

\textbf{206 AMAR O PRÓXIMO}

Por este preceito somos ordenados a amar uns aos outros da mesma forma
que amamos a nós mesmos, e que o amor e a compaixão de alguém por seu
irmão de fé deve ser igual ao amor e compaixão que ele tem por si mesmo
com relação a seu dinheiro, seu corpo, e a tudo o que ele possui e
deseja. Tudo aquilo que eu desejar para mim, devo desejar também para
ele; e tudo o que eu não quiser para mim nem para meus amigos também não
devo desejar para ele. Este preceito está expresso em Suas palavras,
enaltecido seja Ele, ``Amarás o teu próximo como a ti mesmo'' (Levítico
19:18).

\textbf{207 AMAR O PROSÉLITO}

Por este preceito somos ordenados a amar o prosélito. Ele está ex­presso
em Suas palavras, enaltecido seja Ele, ``E amareis ao prosélito''
(Deute­ronômio 10:19). Embora esse prosélito esteja incluído em Israel
--- de tal forma

206. Ver o preceito negativo 269.

que as palavras ``Amarás o teu próximo como a ti mesmo'' (Levítico 19:18)
se apliquem também a ele ---, o Eterno ordenou que lhe seja dedicado um
or maior porque ele se converteu à Fé, e acrescentou um
preceito\textsubscript{.} e\textsubscript{2:}

seci eu

favor, assim como fez no caso da advertência quanto a eng or-

dena: ``E não enganareis cada um ao seu companheiro'' (1 • ,
\textsubscript{5}


de-


pois acrescenta: ``Ao prosélito não fraudareis'' (Êxodo 22
0\textsuperscript{208}. ção

dada na Guemará é que "Aquele que engana um proséli o culpa d duas

violações: 'Não enganareis cada um ao seu companheira' e•rosélito não
fraudareis'. Também nossa obrigação de amá-lo está em ambos 'Amarás o
teu próximo como a ti mesmo' e 'Amareis ao prosélito' " . Isto está
claro acima de qualquer dúvida, e não conheço ninguém que, ao enumerar
os preceitos, te­nha se enganado a este respeito.

Na maioria dos Midrashot está explicado que o Eterno nos ordenou com
relação aos prosélitos o que Ele nos ordenou com relação a Si mesmo ao
dizer "E amarás ao Eterno, teu Deus" (Deuteronômio 6:5) e "Amareis ao
prosélito".

\section{A lei dos pesos e medidas}

Por este preceito somos ordenados a ter pesos, balanças e medidas
exatos, e a regulá-los com extrema precisão. Ele está expresso em Suas
palavras "Balanças justas, pesos justos, 'efá' justa e 'hin' justo
tereis para vós" (Levítico 19:36), sobre as quais diz a Sifrá: "
'Balanças justas' significa que as balanças deve ser absolutamente
precisas: 'pesos justos' significa que os pesos devem ser absolutamente
exatos; " 'efá" justa' significa que a 'efá' deve ser absoluta­mente
exata; e " 'hin" justo' significa que o 'hin' deve ser absolutamente
exa­to". Você sabe que a ``efá'' é uma medida para sólidos e o ``hin'' uma
medida para líquidos. A mesma regra se aplica a todos esses casos,
embora o tipo de medidas possa variar, porque o que é pesado ou medido é
simplesmente a quan­tidade de alguma coisa. Todo esse tipo de coisas ---
balanças, pesos, medidas para sólidos e medidas para líquidos --- são
chamadas medidas, e o preceito que nos obriga a regular cada uma delas
com precisão, de acordo com os pa­drões aprovados, é chamado de preceito
das Medidas.

A Sifrá diz: "Eu vos tirei da terra do Egito com a condição de que vocês
aceitem sobre si mesmos o preceito sobre as Medidas; pois aquele que
reconhece o preceito da Medidas reconhece através dele o Êxodo do Egito,
e aquele que o negar estará negando (também a autenticidade do) o Êxodo
do Egito".


As normas deste preceito estão explicadas no quinto capítulo de Ba-


ba Batra.

\section{Honrar os eruditos e os idosos}

Por este preceito somos ordenados a respeitar os eruditos e a
levantar-nos diante deles a fim de honrá-los. Ele está expresso em Suas
palavras, enalte­cido seja Ele, "Diante das cãs te levantarás e honrarás
as faces do velho" (Leví-


\begin{enumerate}
\def\labelenumi{\arabic{enumi}.}
\setcounter{enumi}{206}
\item
 
 Ver o preceito negativo 251.
 
\item
 
 Ver o preceito negativo 252.
 
\end{enumerate}




tico 19:32), sobre as quais a Sifrá diz: "Te levantarás e honrarás ---
levantar-se para demonstrar respeito". As normas deste preceito estão
explicadas no pri­meiro capítulo de Kidushin.

Você deve saber que embora este preceito para respeitar os Erudi­tos
seja um dever igual para todos, inclusive para um Erudito com relação a
outro de igual conhecimento --- como dizem os Sábios, "Os eruditos da
Babi­lônia estavam habituados a levantar-se uns diante dos outros" ---
ele é especial­mente e sobretudo obrigatório para um discípulo, pois ele
deve um respeito muito maior a seu mestre do que a qualquer outro
Erudito, assim como ele tem a obrigação de temê-lo, pois os Sábios
afirmam claramente que nosso dever pa­ra com nosso mestre é maior do que
nosso dever para com nosso pai, a quem as Escrituras nos obrigam a
honrar e a temer. E os Sábios dizem explicitamente: "Seu pai e seu
mestre --- seu mestre tem prioridade".

Os Sábios também deixam claro que um discípulo está proibido de
contestar seu mestre e por contestar quero dizer se opor a sua decisão,
re'

sua opinião e lecionar e instruir sem sua permissão. Ele também está
Óibi de brigar e discutir com ele, e de j negativamente, ou seja, atri u
r mot\\
vações ruins a seus atos ou palavas, p• s é possível que suas inten •
es' \textsuperscript{09} nã

sejam aquelas. No capítulo "Hel. s Sábios dizem: "Discordar e se

tre é como discordar da 'Shekhi mo está dito em 'Quando fizeram brigar o
povo contra o Eterno' (Núme os 26:9); brigar com seu mestre é como
brigar com a Shekhiná', como está dito em 'Estas são as águas de Meribá
porque bri­garam os filhos de Israel com o Eterno' (Ibid., 20:13 ue
ar-se de seu mestre

é como queixar-se da Shekhiná', como está dito e sobre nós vossas

queixas, senão sobre o Eterno' (Êxodo 16:8); atri eu mestre é como

atribuir à Shekhiná', como está dito em 'E falou o p ntra Deus e contra

Moisés' (Números 21:5)". Tudo isto está perfeitamente claro, pois embora
a dis­cussão de Korah, a briga dos filhos de Israel, e suas acusações e
suspeitas fos­sem na realidade dirigidas contra Moisés, que era o Mestre
de Israel, as Escritu­ras ós consideram como tendo sido dirigidas contra
Deus. E os Sábios dizem expressamente: "Que o temor a seu mestre seja
como o temor aos Cé

Tudo o que foi exposto foi deduzido do preceito das Escrit s de honrar
os Eruditos e os pais, e ficou claro, pela linguagem do Talmud, e212 não
é um preceito independente. Você deve compreender isto.

\section{Honrar os pais}

Por este preceito somos ordenados a honrar nossos pais. Ele está
ex­presso em Suas palavras, enaltecido seja Ele, "Honrarás a teu pai e a
tua mãe" (Êxodo 20:12). As normas deste preceito estão explicadas em
vários trechos do Talmud especialmente em Kidushin

A Sifrá diz: "O que signi hon r? Assegurar-lhes comida e bebi-

da, roupas e calor, e guiar seus pas os213.


\begin{enumerate}
\def\labelenumi{\arabic{enumi}.}
\setcounter{enumi}{208}
\item
 
 É possível que as intenções do mestre não sejam as que se pensa.
 
\item
 
 Do Tratado Sanhedrin.
 
\item
 
 Atribuir maldade a seu mestre.
 
\item
 
 Que o temor por seu mestre.
 
\item
 
 Quando os pais estiverem velhos e fracos.
 
\end{enumerate}


1.( MAIMÔNIDES

\section{Respeitar os pais}

Por este preceito somos ordenados a temer nossos pais, a olhá-los com o
respeito devido a alguém de quem se teme o castigo, como o Rei, e a
tratá-los como tratamos aqueles a quem tememos e receamos descontentar.
Ele está expresso em Suas palavras, exaltado seja Ele, "Cada um a sua
mãe e a seu pai temerá" (Levítico 19:3), sobre as quais diz a Sifrá: "O
que significa temer? Não se sentar em seu assento, nem falar em seu
lugar, nem contradizer suas palavras".


As normas deste preceito estão explicadas em Kidushin.


\section{``frutificar e multiplicar''}

Por este preceito somos ordenados a frutificar-nos e multiplicar-nos
para perpetuar a espécie. Esta é a lei da multiplicação; ela está
expressa em Suas palavras, enaltecido seja Ele, "Frutificai e
multiplicai" (Gênesis 1:28). O Talmud diz que por ocasião do casamento
com uma virgem o noivo está dispensado de recitar o ``Shemá'' porque ele
estará ocupado cumprindo um preceito.

As normas deste preceito estão explicadas no sexto capítulo de Ye­bamot.
Ele não se aplica às mulheres: o Talmud diz explicitamente "O dever de
frutificar e multiplicar-se recai sobre o homem, não sobre a mulher".

\section{A lei da consagração pelo casamento}

Por este preceito somos ordenados a desposar uma mulher através de uma
cerimônia de compromisso: seja dando-lhe alguma coisa, ou entregando-lhe
uma certidão de consagração pelo casamento, ou por relação carnal. Este
é o preceito relativo à cerimônia da consagração pelo casamento. Está
dito o seguinte: "Quando um homem tomar uma mulher e se casar com ela,
etc." (Deu-

teronômio 24:1) nos ensina que se pode ter uma mulher através da ção
car-

nal. "E tendo ela saído da sua casa, poderá ir tornar-se mul ,2) nos

ensina que assim como sua partida se faz por um docu ent
\textsuperscript{14}, e a pode tornar-se esposa de um homem também por
um documen ndemos que uma mulher pode ser adquirida por dinheiro pelas
Sua • avras relativas à serva hebréia ``Sem dar dinheiro'' (Êxodo 21:11),
a respeito das quais diz o Talmud: "Este senhor não recebe dinheiro, mas
há outro senhor que recebe dinheiro, que é seu pai". Contudo, a
consagração pelo casamento ordena pela Torah é a da relação carnal, como
explicado em vários trechos dos Tratados Quetubot, Kidushin, e Nidá.

As normas deste preceito estão explicadas na íntegra no Tratado que lida
especificamente com este assunto, que é o Tratado Kidushin.

Os Sábios dizem especificamente que a consagração pelo casamento feita
através da relação carnal é ordenada pela Torah. Assim fica claro que a
ori-


\begin{enumerate}
\def\labelenumi{\arabic{enumi}.}
\setcounter{enumi}{213}
\item
 
 O documento que concede o divórcio.
 
\item
 
 O documento que atesta a consagração pelo casamento.
 
\end{enumerate}




gem do preceito relativo à cerimônia da consagração pelo casamento está
nas Escrituras.

\section{O marido deve dedicar-se a sua esposa durante um ano}

Por este preceito o marido é ordenado a dedicar-se a sua esposa du­rante
um ano inteiro, durante o qual ele não deverá sair do país para viajar
ou ir à guerra, nem assumir qualquer outra obrigação desse tipo. Ele
deverá alegrar-se com sua esposa durante um ano inteiro a partir do dia
de seu casamento. Este preceito está expresso em Suas palavras,
enaltecido seja Ele, "Livre estará para cuidar de sua casa por um ano, e
alegrará a mulher que tomou" (Deutero­nômio 24:5).


As normas deste preceito estão explicadas no oitavo capítulo de Sotá.


\section{A lei da circuncisão}

Por este preceito somos ordenados a circuncidar. Este preceito está
expresso em Suas palavras, enaltecido seja Ele, a Abraham "Será
circuncidado, entre vós, todo varão" (Gênesis 17:10). Torah decreta
explicitamente a ex­tinção para aquele que violar este pre• - to
ositivo, com as palavras, enalteci­do seja Ele, "E o varão incircunciso,
ue nã circuncidar a carne de seu prepú­cio, essa alma será cortada"
(Ibid. 4)216

As normas deste preceito tão plicadas no décimo nono capítulo de Shabat,
e no quarto capítulo de Yè..mot. A obrigação de circuncidar um filho
recai sobre o pai e não sobre a mãe, como está explicado em Kidushin.

\section{A lei do casamento levirato}

Por este preceito somos ordenados a que um cunhado tome por es­posa a
viúva de seu irmão quando este tiver morrido sem deixar descendentes.
Ele está expresso em Suas palavras, enaltecido seja Ele, "O irmão de seu
mari­do estará com ela" (Deuteronômio 25:5).


As normas deste preceito estão explicadas no Tratado Yebamot.


\section{``halitzá''}

Por este preceito a mulher de um irmão falecido fica ordenada a
exe­cutar ``Halitzá'' em seu cunhado, se ele não se casar com ela. Ele
está expresso em Suas palavras, enaltecido seja Ele, "E lhe descalçará o
sapato do pé" (Deute­ronômio 25:9).

As normas deste preceito estão explicadas no Tratado que lida
espe­cialmente com este assunto, que é o Tratado Yebamot.

216. Ou seja, todo Israelita incircunciso se torna culpado e está
sujeito à extinção, quando ele atin­ge a maioridade. O pai, contudo, não
está sujeito a essa penalidade por não ter incluído seu filho no pacto
de Abraham, embora ele assim esteja transgredindo o mesmo preceito
positivo.
Você já está familiarizado com a regra estabelecida de que "O dever de
casar-se com a esposa de um irmão falecido tem prioridade sobre o dever
da `Halitzá' ". É por esse motivo que o Tratado é chamado ``Yebamot'',
embo­ra ele inclua ambas as leis do casamento levirato e as de
``Halitzá''.

\section{Um violador deve casar-se com a moça que violentou}

Por este preceito um homem é obrigado a casar-se com a moça que ele
tiver violentado. Ele está expresso em Suas palavras, enaltecido seja
Ele, "E ela lhe será por mulher, porquanto a afligiu, e não a poderá
despedir por todos os seus dia ' Deuteronômio 22:29). A Guemará de Macot
afirma que o precei­to negatiativo à violação, que é "E não a poderá
despedir por todos os

seus • a é um preceito negativo precedido por um positivo, e assim

disser mal isso é um preceito negativo precedido por um preceito po-

sitivo". nto, fica claro que as palavras ``E ela lhe será por mulher''
consti-

tui um preceito positivo.

As normas deste preceito estão explicadas no terceiro e quarto
capí­tulos de Quetubot.

\section{A lei sobre aquele que difama sua esposa}

Este preceito estabelece a lei relativa a um homem que difam Este
preceito ordena que ele seja açoitado e que fi vez que com relação a
isso foi dito "E lhe será por dir, por todos os seus dias" (Deuteronômio
22:19

A Guemará de Macot afirma que este é u ceito negativo precedido
por um positivo, assim como no caso do violentador.

As normas deste preceito estão explicadas no terceiro e quarto
capí­tulos de Quetubot.

\section{A lei sobre o sedutor}

Por este preceito somos ordenados quanto à lei sobre o sedutor. Ele está
expresso em Suas palavras, enaltecido seja Ele, "E quando enganar um
ho­mem a uma virgem etc." (Êxodo 22:15).

As normas deste preceito estão explicadas no terceiro e quarto
capí­tulos de Quetubot.

\section{Lei sobre a mulher cativa}

P r este preceito somos ordenados quanto à lei de uma bela
\textsubscript{mu}i \textsubscript{r}221\textsubscript{.} El está
expresso em Suas palavras, enaltecido seja Ele, "E vires en-


\begin{enumerate}
\def\labelenumi{\arabic{enumi}.}
\setcounter{enumi}{216}
\item
 
 Ver o preceito negativ6358.
 
\item
 
 Os Sábios.
 
\item
 
 Difamar a moça com quem ele tenha se casado.
 
\item
 
 Ver também o preceito negativo 359.
 
\item
 
 Que tenha sido capturada durante uma guerra.
 
\end{enumerate}




tre os cativos uma mulher formosa" (Deuteronômio 21:11).


As normas deste preceito estão explicadas no início de Kidushin.


\textbf{222} A LEI DO DIVÓRCIO

Por este preceito somos ordenados a que, se desejarmos divorciar-nos de
uma mulher, o façamos unicamente através de um documento de divór­cio.
Este preceito está expresso em Suas palavras, enaltecido seja Ele,
"Escrever-lhe-á uma carta de divórcio" (Deuteronômio 24:1).

As normas deste preceito, que é a lei do divórcio, estão explicadas por
inteiro no Tratado que lida especificamente com este assunto, e que é o
Tratado Guitin.


SUSPEITA DE ADULTÉRIO

Por este preceito somos ordenados quanto à lei da mulher suspeita

tido adultério. Ele está expresso em Suas palavras, enaltecido seja ho
em, quando se desviar sua mulher etc.' (Números 5:12). normas deste
preceito --- a maneira pela qual ela deve ser forçada levar seu
sacrifício, e as outras condições --- estão explicadas no lida
especificamente com este assunto, que é o Tratado Sotá.


\section{Açoitar os transgressores de determinados preceitos}

Por este preceito somos ordenados a açoitar com uma correia os
vio­ladores de determinados preceitos. Ele está expresso em Suas
palavras, enalte­cido seja Ele, "O juiz o fará deitar e o fará açoitar
na sua presença" (Deuteronô­mio 25:2). Quando lidarmos com os preceitos
negativos nós indicaremos quais são os preceitos cuja violação é punida
com o açoitamento.

As normas deste preceito estão explicadas no Tratado Macot.

\section{A lei do homicídio involuntário}

Por este preceito somos ordenados a exilar de sua cidade um homi­cida
involuntário para uma cidade de refúgio. Ele está expresso em Suas
pala­vras, enaltecido seja Ele, "E ficará nela até morrer o 'Cohen
Gadol' " (Números 35:25). sobre as quais diz o Sifrei: " 'Ficará nela':
ele nunca poderá sair de lá porque a palavra 'nela' significa que lá ele
deverá viver, lá deverá morrer, e lá deverá ser enterrado".

As normas deste preceito estão explicadas no Tratado Macot.

\section{Executar com a espada os transgressores de determinados preceitos}

Por este preceito somos ordenados a executar com a espada os vio­ladores
de determinados preceitos. Ele está expresso em Suas palavras,
enalte­cido seja Ele, ``Serão certamente vingados'' (Êxodo 21:20). quando
tratarmos dos preceitos negativos mostraremos quais são os preceitos
cuja violação é pu­nida com a decapitação.

As normas deste preceito estão explicadas no sétimo capítulo do Tra­tado
Sanhedrin.

\section{Estrangular os transgressores de determinados preceitos}

Por este preceito somos ordenados a estrangular os violadores de
determinados preceitos. Ele está expresso em Suas palavras, enaltecido
seja Ele, ``Certamente serão mortos'' (Levítico 20:10). Quando tratarmos
dos preceitos negativos mostraremos quais são os preceitos cuja violação
é punida com o estrangulamento.

As normas deste preceito estão explicadas no sétimo capítulo do Tra­tado
Sanhedrin.

\section{Queimar os transgressores de determinados preceitos}

Por este preceito somos ordenados a queimar os violadores de
de­terminados preceitos. Ele está expresso em Suas palavras, enaltecido
seja Ele, ``No fogo queimarão a ele e a ela'' (Levítico 20:14). Quando
tratarmos dos pre­ceitos negativos mostaremos quais são os preceitos
cuja violação é punida com a morte pelo fogo.

As normas deste preceito estão explicadas no sétimo capítulo do Tra­tado
Sanhedrin.

\section{Apedrejar os transgressores de determinados preceitos}

Por este preceito somos ordenados a apedrejar os violadores de
de­terminados preceitos. Ele está expresso em Suas palavras, enaltecido
seja Ele, "E os apedrejareis, e morrerão" (Deuteronômio 22:24). Quando
tratarmos dos preceitos negativos mostraremos quais são os preceitos
cuja violação é punida com a morte por apedrejamento.


As normas deste preceito estão explicadas no sexto capítulo de


Sanhedrin.



\section{Pendurar os corpos de certos transgressores depois de executados}

Por este preceito somos ordenados a pendurar certos transgressores
executados por ordem do Tribunal. Ele está e esso em Suas palavras,
enalte­cido seja Ele, "O pendurarás num madei • (De eronômio 21:22).
Quando tratarmos dos preceitos negativos mostr .emos
qu s são os preceitos cuja vio­lação acarreta que o corpo seja pend
ado\textsuperscript{223}.


As normas deste preceito e o expl adas no sexto capítulo de


Sanhedrin.

\section{A lei do enterro}

Por este preceito somos ordenados a enterrar no dia da execução aqueles
que tiverem sido mortos por ordem do Tribunal. Ele está expresso em Suas
palavras, enaltecido seja Ele, "Certamente enterra-lo-ás no mesmo dia"
(Deuteronômio 21:23), sobre as quais o Sifrei diz: 'Certamente
enterra-lo-ás' é um preceito positivo.

A mesma lei é obrigatória com relação a todos os outros mortos: to­do
Israelita deve ser enterrado no dia de sua morte. É por essa razão que
quan­do não há ninguém para assistir ao enterro de um corpo ele é
chamado de Corpo de Obrigação Religiosa; ou seja, um corpo cujo enterro
é o dever de cada pessoa, de acordo com Suas palavras, enaltecido seja
Ele, "Certamente enterra-lo-ás".


As normas deste preceito estão explicadas no sexto capítulo de


Sanhedrin.

\section{A lei do servo hebreu}

Por este preceito somos ordenados quanto à lei do servo hebreu. Ele está
expresso em Suas palavras, enaltecido seja Ele, "Quando comprares um
escravo hebreu etc." (Êxodo 21:2).

As normas deste preceito estão claramente explicadas nos versícu­los da
Torah e todos os seus regulamentos estão no Tratado Kidushin.

\section{O casamento de uma serva hebréia com seu amo ou com o filho dele}

Por este preceito o homem que comprar uma serva hebréia ou o fi­lho dele
são ordenados a casar-se com ela. Este é o preceito relativo aos
espon­sais. Os Sábios dizem explicitamente: "O dever dos esponsais tem
prioridade sobre o dever de resgate, porque o Enaltecido diz: 'Que não a
consagrou para si deve remi-la' (Êxodo 21:8)".


Você deve estar ciente, de que as leis referentes ao servo e à serva

hebreus só estarão em vigor durante a vigência da lei do Jubileu.

As normas deste preceito estão explicadas no primeiro capítulo do
Tratado Kidushin.

\section{O resgate de uma serva hebréia}

Por este preceito somos ordenados quanto ao resgate de uma serva
hebréia. Ele está expresso em Suas palavras, enaltecido seja Ele, "Deve
remi-la" (Êxodo 2 1 :8).

Há muitas normas, condições e regras referentes a este dever de
res­gate. Elas estão todas explicadas no Tratado Kidushin, onde a lei
sobre a serva hebréia está exposta na íntegra.

Explicando Suas palavras, enaltecido seja Ele, "E se não lhe fizer
es­tas três coisas" (Ibid., 11), a respeito da serva hebréia, a Mekhiltá
diz que seu dono deve casar-se com ela, ou casá-la com seu filho, ou
remi-la.

\section{A lei sobre o escravo cananeu}

Por este preceito somos ordenados quanto à lei ,s.obre
uru escravo cananeu; ela diz que ele deve ser escravo para sempre e
que.não po adquirir sua liberdade a não ser por causa
de um dente ou um olho\textsuperscript{224}, ou por causa de qualquer
outro órgão do corpo que não torne a crescer, de ac rdo com a
interpretação tradicional. Este preceito está expresso em Suas palavras
"Perpe­tuamente vos fareis servir deles" (Levítico 25:46) e "E quando
ferir um homem o olho de seu escravo etc." (Êxodo 21:26).

A Guemará de Guitin diz: "Todo aquele que liberar seu escravo pa­gão
estará violando um preceito positivo pois está escrito 'Perpetuamente
vos fareis servir deles' " . Contudo, a Torah diz que ele poderá obter a
liberdade por causa de um dente ou de um olho.

As normas deste preceito estão explicadas por completo em Kidus­hin e em
Guitin.

\section{A penalidade por causar ferimentos}

Por este preceito somos ordenados quanto à lei sobre alguém que fere seu
companheiro. Ele está expresso em Suas palavras, enaltecido seja Ele, "E
quando brigarem homens, e ferir um homem a seu próximo etc." (Exodo 21 :
18). Essas são chamadas leis sobre Penalidades todas elas têm sua base
nas Escrituras nas Suas palavras, enaltecido seja Ele, "Conforme ele
fez, assim lhe será feito" (Levítico 24:19), cujo significado é que um
homem deve pagar a im­portância equivalente ao dano que causou a seu
companheiro. A Tradição de­termina que mesmo que ele o tenha apenas
envergonhado, ele deve ser multa­do na importância equivalente.

Você deve saber que todas essas leis relativas a Penalidades se apli­cam
a ferimentos feitos por um homem a outro homem. Da mesma forma exis-

224. Se o dono do escravo lhe causar a perda de um dente ou de um olho.



tem também leis relativas a ferimentos causados por um animal a uma
pessoa, e vice-versa. Apenas um Tribunal assentado na Terra de Israel
poderá julgar e pronunciar uma sentença referente a essas leis.

A regulamentação deste preceito está explicada no primeiro capítu­lo de
Baba Kamma.

\section{A lei sobre ferimentos causados por um boi}

Por este preceito somos ordenados quanto à lei do boi. Ele está
ex­presso em Suas palavras, enaltecido seja Ele, "Quando marrar um boi a
um ho­mem ou a uma mulher etc." (Êxodo 21:28), e "E quando ferir o boi
de um ho­mem ao boi de seu companheiro etc." (Ibid., 35).

A regulamentação desta lei está explicada nos primeiros seis capítu­los
de Baba Kamma.


\section{A LEI SOBRE FERIMENTOS CAUSADOS POR UM POÇO}

Por este preceito somos ordenados quanto à lei do poço. Ele está em Suas
palavras, enaltecido seja Ele, "E quando um homem abrir etc." (Êxodo
21:33).

A regulamentação deste preceito está explicada no terceiro e quarto de
Baba Kamma.

\section{A lei sobre o roubo}


Por este preceito somos ordenados quanto à lei do ladrã


cobrar dele u que podemo mos vendê-16\textsuperscript{227} que estão
s\textsubscript{i}

Todos os detalhes desta lei estão explicados no sétimo capítulo de Baba
Kamma, no oitavo capítulo de Sanhedrin, no terceiro capítulo de Baba
Metzia, e em alguns trechos de Quetubot, Kidushin, e Shabuot.

\section{A lei sobre os prejuízos causados por um animal}

Por este preceito somos ordenados quanto à lei sobre o animal que
destrói colheitas. Ele está expresso em Suas palavras, enaltecido seja
Ele, "Quan­do um homem fizer pastar num campo ou numa vinha etc." (xodo
22:4).

A regulamentação de toda esta lei está explicada no segundo e sexto
capítulos de Baba Kamma, e no quinto capítulo de Ghitin.


\begin{enumerate}
\def\labelenumi{\arabic{enumi}.}
\setcounter{enumi}{224}
\item
 
 Êxodo 21:37.
 
\item
 
 Ibid., 22:1.
 
\item
 
 Caso ele não possa fazer a restituição, como lhe é ordenado. Ibid.,
 22:2.
 
\end{enumerate}




\section{A lei sobre os prejuízos causados pelo fogo}

Por este preceito somos ordenados quanto à lei do fogo. Ele está
ex­presso em Suas palavras, enaltecido seja Ele, "Quando houver fogo, e
pegar nos espinhos etc." (Êxodo 22:5).

A regulamentação desta lei está explicada no segundo e no sexto
ca­pítulos de Baba Kamma.

\section{A lei sobre o depositário não remunerado}

Por este preceito somos ordenados quanto à lei de um depositário não
remunerado. Ele está expresso em Suas palavras, enaltecido seja Ele,
"Quando o homem der ao seu companheiro, dinheiro ou objetos para
guar­dar etc." (Êxodo 22:6).

Os detalhes desta lei estão explicados no nono capítulo de Baba Kam­ma,
no terceiro capítulo de Baba Metzia, e no oitavo capítulo de Shabuot.

\section{A lei sobre o depositário remunerado}

Por este preceito somos ordenados quanto à lei de um depositário
remunerado ou de um arrendador, sendo que uma só lei se aplica a ambos,
co­mo foi explicado pelos Sábios, os quais dizem que há três leis para
regulamen­tar quatro tipos de depositários. Este preceito está expresso
em Suas palavras, enaltecido seja Ele, "Quando der o homem a seu
companheiro, asno, boi, car­neiro etc." (Exodo 22:9).

Todos os detalhes desta lei estão explicados no sexto e no nono
ca­pítulo de Baba Kamma, no terceiro e no sexto capítulos de Baba
Metzia, e no oitavo capítulo de Shabuot.

\section{A lei sobre quem pede emprestado}

Por este preceito somos ordenados quanto à lei sobre aquele que pede
emprestado. Ele está expresso em Suas palavras, enaltecido seja Ele, "E
quando um homem pedir emprestado de seu companheiro etc." (Êxodo 22:13).

A regulamentação desta lei está explicada no oitavo capítulo de Ba­ba
Metzia e no oitavo capítulo de Shabuot.

\section{A lei de compra e venda}

Por este preceito somos ordenados quanto à lei de compra e venda; ou
seja, o procedimento a seguir pelo vendedor e pelo comprador ao efetuar
uma transação. Aprendemos este procedimento através de Suas palavras,
enal­tecido seja Ele, "E quando fizerdes uma venda a vosso companheiro,
ou com­prardes da mão de vosso companheiro etc." (Levítico 25:14), que
os Sábios in-



terpretam como refer do-se a "uma mercadoria comprada de mão em mão, ou
seja, por `meshichá'\textsuperscript{228}

Foi demonstra que a aquisição por meio de dinheiro fica assegu­rada pela
lei das Escrituras e que a ``meshichá'' no caso de bens móveis é ape­nas
uma regulamentação dos Sábios, da mesma forma que entregá-la o vende­dor
ao comprador ou levantá-la o comprador. O Talmud diz explicitamente:
-"Assim como eles instituíram a `meshichá' para compradores, eles também
ins­tituíram a `meshichá' para depositários".

Assim, foi deixado claro que a norma da ``meshichá'' numa venda foi
instituída pelos Sábios, como está explicado no lugar apropriado; mas as
outras formas de procedimento através das quais são adquiridas terras e
outras coisas, a saber, documentos e delimitações, estão baseadas no
versículo.

Os detalhes dessa lei --- ou seja, o procedimento a seguir em cada caso,
ao efetuar uma venda --- estão eplicados no primeiro capítulo de
Kidus­hin, no quarto e oitavo capítulos de Baba Metzia, e no terceiro,
quarto, quinto, sexto e sétimo capítulos de Baba Batra.

\section{A lei sobre os litigantes}

Por este preceito som

e o acusado. Ele está expresso m Su toda coisa de delito ... a respeito
da qu respeito a Mekhiltá diz: " 'É e\$te'\textsuperscript{229}

Esta lei inclui todos os caso vam confissões ou desmentidos.

Os detalhes desta lei estão explicados no terceiro capítulo de Baba
Kamma, no início e no oitavo capítulo de Baba Metzia, e no quinto, sexto
e sétimo capítulos de Shabuot; muitas perguntas também são encontradas
espa­lhadas em vários lugares do Talmud.

\section{Salvar a vida do perseguido}

Por este preceito somos ordenados a salvar uma pessoa do persegui­dor
que tiver a intenção de matá-la, até mesmo tirando a vida do
perseguidor; ou seja, devemos matar o perseguidor se não pudermos salvar
o perseguido de nenhuma outra forma. Este preceito está expresso em Suas
palavras, enalte­cido seja Ele, "Cortar-lhe-ás a mão, o teu olho não
terá piedade dela (Deutero­nômio 25: 1 1-1 2). A esse respeito o Sifrei
diz: " 'Pelas suas vergonhas' (Ibid.): assim como o ato aqui
especificado, por envolver perigo de vida, justifica cortar-se a mão da
mulher, o mesmo princípio deve ser aplicado toda vez que houver risco de
vida. Contudo este versículo nos diz apenas que o homem deve ser salvo
cortando-se a mão da mulher. Como saber se no caso de um homem que não
possa ser salvo cortando-se a mão de alguém, devemos salvá-lo tirando
uma vida? Pelas palavras 'O teu olho não terá piedade' ".

Portanto, o significado deste preceito foi deixado claro, sendo que as
palavras ``a mulher de um'' são usadas apenas neste caso específico, e
sendo que o verdadeiro significado é que a vida do perseguido deve ser
salva a custa


\begin{enumerate}
\def\labelenumi{\arabic{enumi}.}
\setcounter{enumi}{227}
\item
 
 Recibo ou comprovante de uma transação comercial.
 
\item
 
 O acusado.
 
\end{enumerate}

dos membros do perseguidor, e que quando é impossível salvá-lo a não ser
ma­tando imediatamente o perseguidor, isso deve ser feito.


As normas deste preceito estão explicadas no oitavo capítulo de


Sanhedrin.

\section{A lei sobre as heranças}

Por este preceito somos ordenados quanto à lei sobre as heranças. Ele
está expresso em Suas palavras, enaltecido seja Ele, "Quando um homem
morrer e não tiver filho etc." (Números 27:8).

Uma das normas desta lei é sem dúvida alguma que o filho nito herde o
dobro dos outros, pois esta é uma das leis das heranç

As normas deste preceito estão explicadas no oitavo e no los de Baba
Batra.


%J: Essa seção nao pode ser numerada e deve contar nos sumários. Sugiro
%   colocar o numerador até aqui
\section{Comentários finais de maimônides sobre os preceitos positivos}


Você deve saber que quando digo, a respeito de cada preceito, "suas
normas estão explicadas em tal-e-tal lugar" eu não estou querendo dizer
que o capítulo ou tratado mencionado contém todas as normas daquele
preceito, em seus mínimos detalhes. Eu estou apenas indicando o local
onde se encon­tram as principais regras e a maioria das normas daquele
preceito, embora haja muitas outras referências relativas a regras
espalhadas em outras partes do Tal­mud, que eu não menciono
especificamente.

Se você examinar todos os preceitos apresentados até agora verá que
alguns são obrigatórios a toda a congregação de Israel, de maneira
coletiva, e não a cada pessoa individualmente, como por exemplo a
construção do Tem­plo, a nomeação de um rei, e a exterminação da semente
de Amalec. Outros são obrigatórios ao indivíduo que realizou um
determinado ato, ou a quem acon­teceu alguma coisa, como por exemplo os
sacrifícios oferecidos por quem pe­cou sem querer, ou um ``zab''; e é
possível que um homem não faça e nem lhe ocorra nenhuma dessas coisas em
toda sua vida. Há entre esses preceitos, como explicamos, determinadas
leis, como a do servo hebreu, e da serva he­bréia, a do servo cananeu, a
do depositário não remunerado, a de quem pede emprestado, e outras
mencionadas acima, que pode ser que nunca se apliquem a um determinado
homem, e que pode ser que ele nunca chegue a executar, em toda a sua
vida. Outros preceitos são obrigatórios apenas durante a existên­cia do
Templo, como por exemplo as ofertas dos festivais, o comparecimento
diante do Eterno, e a reunião do povo, que nós apresentamos uma a uma.
Ou­tros são obrigatórios apenas para quem tem bens, como por exemplo os
dízi­mos, os sacrifícios de elevação, os presentes prescritos para o
``Cohen'', e par­te para os pobres, tais como as respigas, a gavela
esquecida, ``peá'', e os cachos de uva imperfeitos; e é possível a um
homem ficar isento deles se ele não tiver bens, e passar a vida toda sem
ser obrigado a realizar nenhum dos preceitos desse tipo. A caridade, no
entanto, não pertence a essa categoria, porque ela é uma obrigação até
mesmo para um homem pobre que vive ele próprio de caridade, como
explicamos. Outros preceitos, ainda, são definitivamente obri-



gatórios a todos os homens, em todos os tempos, em qualquer lugar e em
quais­quer circunstâncias, como por exemplo os ``tsitsit'', os
filactérios, e o Shabat. A esses nós chamamos de preceitos
incondicionais, porque sua obrigatorieda­de recai sobre todo israelita
adulto, sempre, em qualquer lugar, e em quaisquer circunstâncias.

Se você refletir sobre os 248 preceitos positivos, descobrirá que os
preceitos ``incondicionais'' são 60, desde que a pessoa sobre quem eles
recaiam se encontre nas mesmas circunstâncias que a maioria das pessoas,
ou seja, que more numa casa da cidade, que se alimente como a maioria
das pessoas, ou seja, com pão e carne, que negocie com as outras
pessoas, que se case com uma mulher eprocrie.

Os 60 preceitos positivos, de acordo com a ordem de nossa enume­ração,
são: os preceitos positivos 1, 2, 3, 4, 5, 6, 7, 8 e 9. O 10? não é
obrigató­rio às mulheres, nem o 11? Os 12, 13, 14, 15 e 18 também não
são obrigatórios para as mulheres. Os 19 e 26 são obrigatórios apenas
aos varões ``Cohanim''. O 32, 54, 73, 94, 143, 146, 147, 149, 150, 152,
154, 155, 156, 157, 158, 159 e 160; o 161 não é obrigatório às mulheres.
O 162, 163, 164, 165, 166 e 167; os 168, 169 e 170 não são obrigatórios
às mulheres. O 172, 175, 184, 195, 197, 206, 207, 208, 209, 210 e 211; o
212 não é obrigatório às mulheres. O 213. Os 214 e 215 só são
obrigatórios aos varões.


Uma mnemônica para o número de preceitos incondicionais é a se-


guinte: ``Sessent s rainhas'' (Cânticos 6:8), e o número dos não obrigató-

rios para as m ode ser lembrado pela expressão "Que o braço (``yad'')

... está se fo ' (Deuteronômio 32:36): a palavra ``nashim'' (mulher)

perde seu " então o nú s que são obrigatórios às mulheres,

46, pode sei pelo versíc bém para você, pelo sangue ("be-

dam") de seu (Zacarias 9: seja, a palavra ``bedam'' indica o

número (46) dos preceitos que sã icionalmente obrigatórios às mulheres
e constituem seu pacto espe íf

Essas são as observações que achamos necessário registrar na enu­meração
dos preceitos positivos.


\begin{enumerate}
\def\labelenumi{\arabic{enumi}.}
\setcounter{enumi}{230}
\item
 
 O valor numérico das letras do termo hebraico usado, ``yad'', é 14.
 
\item
 
 Isso significa que também ela será redimida pelo mérito do sangue do
 pacto (o preceito da circuncisão). O valor numérico das letras do
 termo hebraico usado aqui, ``bedam'', é 46.
 
\end{enumerate}


\part{Os 365 preceitos negativos}



\section{Não crer e nem atribuir divindade a outro que não ele}

Por esta proibição somos proibidos de crer em ou atribuir divinda­de a
outro que não Ele, enaltecido seja. Ela está expressa em Suas palavras
---embora não se possa atribuir palavras a Seu Ser
transcendental\textsuperscript{233} --- "Não te­rás outros deuses diante
de Mim" (Êxodo 20:3).

Foi deixado claro no final de Macot que esta proibição é um dos 613
preceitos, pois está dito ali: "Seiscentos e treze preceitos foram dados
a Moisés no Sinai etc.". Nós já explicamos este assunto no primeiro
preceito positivo.

\section{Não fazer imagens para adorá-las}

Por esta proibição somos proibidos de fazer imagens que venham a ser
adoradas, e não há diferença entre fazê-las nós mesmos ou mandar que
outros as façam. Esta proibição está expressa em Suas palavras,
enaltecido seja Ele, "Não farás para ti imagem de escultura etc." (Êxodo
20:4).

Todo aquele que transgredir este preceito negativo estará sujeito ao
açoitamento, seja por ter feito o ídolo ou por ter mandado que outra
pessoa o faça, mesmo que ele não o adore.

\section{Não fazer um ídolo para que outros o adorem}

Por esta proibição somos proibidos de fazer dm ídolo, mesmo que seja
para que outros o adorem, e ainda que a pedido de um idólatra. Ela está

233. Já que Deus não é um corpo, e que portanto não se pode atribuir a
Ele a ``fala'' física.

expressa em Suas palavras, enaltecido seja Ele, "Nem fareis vós ídolos
para vós mesmos" (Levítico 19:4), a respeito das quais diz a Sifrá: "
'Nem fareis vós': ainda que seja para os outros".

Também está dito ali: "Aquele que fizer um ídolo para uso próprio
transgride dois preceitos negativos". Ou seja, ele transgride a
proibição de fazê-lo ele próprio, ainda que para uso de outros, que está
contida neste terceiro pre­ceito, e também a proibição de adquirir um
ídolo e de guardá-lo, ainda que ou­tra pessoa o tenha feito para ele, e
que está contida no preceito precedente. Dessa forma ele está sujeito a
ser açoitado duas vezes.

As normas deste preceito e do precedente estão explicadas no Tra­tado
Abodá Zará.

\section{Não fazer figuras de seres humanos}

Por esta proibição somos proibidos de fazer figuras de seres huma­nos de
metal, pedra, madeira e similares, ainda que elas não sejam feitas para
serem adoradas. A finalidade disso é impedir-nos de fazer qualquer
imagem pa­ra que não pensemos, como fazem as massas, que elas possuem
poderes sobre­naturais. Esta proibição está expressa em Suas palavras,
enaltecido seja Ele, "Não fareis diante de Mim, deuses de prata nem
deuses de ouro para vós" (Exodo 20:23).

Para explicar esta proibição a Mekhiltá diz: " 'Não fareis ... deuses de
prata'; a fim de que você não diga: Vou fazê-los apenas como enfeites,
como outros fazem em vários outros países, as Escrituras dizem 'Não
fareis... para vós' ".

A transgressão deste preceito negativo é punida com o açoitamento.

As normas deste preceito --- que figuras nos é permitido ou proibi­do
confeccionar, e de que forma, e assim por diante --- estão explicadas no
ter­ceiro capítulo de Abodá Zará.

Está explicado em Sanhedrin que este preceito negativo --- ou seja, Suas
palavras "Não fareis diante de Mim, deuses de prata etc" --- também
abrange outros aspectos que vão além do limite destes preceitos; mas o
sentido literal deste versículo é o que expusemos, como está explicado
na Makhiltá.

\section{Não se curvar diante de um ídolo}

Por esta proibição somos proibidos de curvar-nos diante de um ído­lo, e
está claro que o termo ``ídolo'' significa qualquer outro objeto de
adora­ção que não o Eterno. Esta proibição está expressa em Suas
palavras, enalteci­do seja Ele, "Não te prostarás diante deles, nem os
servirás" (Êxodo 20:5). A intenção não é de proibir unicamente o ato de
curvar-se, excluindo os outros; foi mencionada apenas uma forma de
adoração, que é a prostração, mas esta­mos da mesma forma proibidos de
oferecer sacrifícios e de queimar incenso diante de um ídolo; todo
aquele que fizer uma dessas coisas proibidas, isto é, que se curvar,
oferecer sacrifícios, oferecer uma libação ou queimar incenso estará
sujeito ao apedrejamento.

A Mekhiltá diz: " 'Aquele que sacrificar aos deuses, será morto' (Êxo­do
22:19). Ouvimos, dessa forma, a penalidade, mas não ouvimos a
advertên­cia. Por isso as Escrituras dizem: 'Não te prostarás diante
deles, nem os servirás'.



Oferecer sacrifícios, que já está incluído\textsuperscript{234}, está
destacado aqui para ensinar-nos a seguinte lição: O sacrifício, que é
algo que se realiza em sinal de adoração a Deus, é um pecado, quer seja
ele normalmente adorado dessa forma quer não; portanto também no caso de
qualquer outro ato executado a serviço de Deus, é um pecado, quer seja
ele normalmente adorado dessa maneira, quer não". O significado dessas
palavras é que todo aquele que realizar diante de um ídolo qualquer um
desses quatro atos de adoração --- a saber, curvar-se, oferecer
sa­crifícios, queimar incenso e verter uma libação --- que nós temos a
obrigação de realizar a serviço de Deus --- está sujeito ao
apedrejamento, mesmo que o ídolo não seja normalmente adorado daquela
maneira. O que se quer dizer pela expressão "não seja normalmente
adorado" é o seguinte: embora alguém não tenha adorado o ídolo da
maneira pela qual ele é costumeiramente adorado, adorá-lo de uma das
formas mencionadas faz com que ele fique sujeito ao ape­drejamento, se
ele tiver pecado voluntariamente, e à extinção, se seu pecado não tiver
sido testemunhado ou se ele não tiver sido punido por isso. Contudo, se
o pecado foi cometido involuntariamente, o pecador deve oferecer um
Sa­crifício Determinado de Pecado. Isto também se aplica a quem deificar
um ob­jeto qualquer.

Esta proibição --- ou seja, a proibição de prestar homenagem a um ídolo
através de qualquer uma dessas quatro formas de adoração, mesmo que o
ídolo não seja normalmente adorado assim --- está repetida em Suas
palavras, enaltecido seja Ele, "E não oferecerão mais seus sacrifícios
aos `se-irim' " (Leví­tico 17:7), a respeito das quais a Sifrá diz: "
Se-irim' significa demônios".

Na Guemará de Pessahim está explicado que esta proibição se aplica em
especial ao caso de alguém que tenha abatido uma oferenda para um ídolo,
mesmo que ele não seja normalmente adorado dessa forma: "Como sabemos
que se alguém sacrificar um animal a `Merkulis' ele está sujeito a ser
punido? Por­que está escrito 'Não oferecerão mais seus sacrifícios aos
demônios'. Uma vez que há uma redundância no que se refere às formas
comuns de adoração, que aparece no versículo 'De que modo serviam estas
nações a seus deuses' (Deute­ronômio 12:30) deve-se considerar este como
referindo-se a formas não comuns de adoração". Dessa maneira, a
transgressão voluntária desta proibição é puni­da com ambas a extinção e
o apedrejamento, como explicado acima, e aquele que a transgredir
involuntariamente deve oferecer um sacrifício. As palavras das
Escrituras relativas a isto são: "Aquele que sacrificar aos deuses será
morto".


As normas deste preceito estão explicadas no sétimo capítulo de


Sanhedrin.

\section{Não adorar ídolos}

Por esta proibição somos proibidos de adorar ídolós ainda que de outras
formas além das quatro especificadas antes, desde que aquele ídolo
es­pecífico seja normalmente adorado dessa maneira, como por exemplo,
evacuar para Peor ou jogar uma pedra para Merkulis. Esta proibição está
expressa em Suas palavras, enaltecido seja Ele, ``Nem os servirás'' (Êxodo
20:5), a respeito das quais a Mekhiltá diz: " 'Não te prostrarás diante
deles, nem os servirás': aqui há dois pecados separados e independentes
--- oferecer um sacrifício e prostrar-se". Dessa forma, aquele que
atirar uma pedra para Peor ou evacuar para Mer-

234. Na advertência ``nem os servirás''.


kulis não terá cometido um pecado, pois essas não são as maneiras comuns
de adorá-los, de acordo com Suas palavras, enaltecido seja Ele, "De que
modo ser­viam estas nações a seus deuses, do mesmo modo também farei eu"
(Deutero­nômio 12:30).

A transgressão voluntária a esta proibição é punível com o
apedreja­mento e a extinção, e aquele que a violar involuntariamente
deve oferecer um sacrifício.

As normas deste preceito também estão explicadas no sétimo capí­tulo de
Sanhedrin, onde se lê: "Por que a extinção está mencionada três vezes
pela idolatria? Ela está prescrita uma vez pela maneira usual, uma vez
pela ma­neira não usual, e uma vez por Molekh". Ou seja, aquele que
adorar qualquer ídolo, seja de que forma for, estará sujeito à extinção,
se a maneira de adoração for a usual, tal como evacuar para Peor, jogar
pedras para Merkulis, ou afastar o cabelo diante de Quemosh. Da mesma
forma, aquele que adorar qualquer ídolo de uma das quatro maneiras
especificadas está sujeito à extinção, mesmo se a forma de adoração não
for a usual, como por exemplo se ele oferecer sacri­fícios a Peor, ou
prostar-se diante de Merkulis, o que seria uma forma ``não usual'' de
adoração. A terceira extinção se aplica àquele que faz com que seus
descen­dentes passem pelo fogo em sinal de adoração a Molekh, como
explicarei.

\section{Não entregar parte de sua descendência a molekh}

Por esta.proibição somos proibidos de colocar uma parte de nossos
descendentes nas mãos do ídolo conhecido, na época da entrega da Torah,
co­mo Molekh. Ela está expressa em Suas palavras, enaltecido seja Ele,
"E da tua semente não entregarás nenhum, para fazê-la passar pelo fogo,
a Molekh" (Le­vítico 18:21).

Esta forma de idolatria, como está explicado no sétimo capítulo de
Sanhedrin, consistia em acender um fogo e abanar suas chamas, quando
então\textsuperscript{235} entregava parte de seus descendentes ao
sacerdote a serviço daquele ídolo, e fazia com que passassem através do
fogo de um lado para o outro.

A proibição de tal conduta está repetida em Suas palavras "Não se achará
entre ti, quem faça passar seu filho ou sua filha pelo fogo"
(Deuteronô­mio \textbf{18:10).}

Aquele que violar voluntariamente esta proibição estará sujeito ao
apedrejamento ou à extinção, se não for apedrejado; aquele que pecar
involun­tariamente deverá oferecer um Sacrifício Determinado de Pecado.

As normas deste preceito estão explicadas no sétimo capítulo de

Sanhedrin.

\section{Não praticar a feitiçaria do ``ob''}


Por esta proibição somos proibidos de praticar a feitiçaria de um ``ob'' o
qual, depois de ter queimado um determinado incenso e realizado um
deter­minado ritual, imagina ouvir uma voz falando de debaixo de suas
axilas que

235. O pai (Hilchot Rambam, Abodá Zará, 6? capítulo, Halachá 3).


responde a suas perguntas, sendo que essa prática é considerada como um
tipo de idolatria. Esta proibição está expressa em Suas palavras,
enaltecido seja Ele, ``Não vos voltareis para as magias'' (Levítico
19:31), a respeito das quais a Sifrá diz: " '013' é o Piton que fala de
debaixo de suas axilas".

Aquele que violar voluntariamente esta proibição --- ou seja, que
pra­ticar isto ele próprio, e que realizar o ritual --- estará sujeito
ao apedrejamento ou à extinção, se ele não for apedrejado; aquele aue
cometer o pecado invo­luntariamente deve oferecer um Sacrifício
Determinado de Pecado.


As normas deste preceito estão explicadas no sétimo capítulo de


Sanhedrin.


\section{Não praticar a feitiçaria do ``yideoni''}


Por esta proibição somos proibidos de praticar a feitiçaria de um
"yi­deoni", sendo que esta também é uma espécie de idolatria. O
``yideoni'' colo­ca na boca o osso de um pássaro chamado ``yidoa'', queima
incenso, recita de­terminadas palavras e cumpre um determinado ritual
até ficar como se tivesse desmaiado e cair num transe, durante o qual
ele\textsuperscript{236} prediz o futuro. Os Sábios dizem: " 'yideoni'
--- aquele que coloca o osso do `yidoa' em sua boca, o qual fala por si
próprio". A proibição deste prática está nas palavras "Não vos
volta­reis para as magias e para as feitiçarias (yideoni)" (Levítico
19:31). Isto não deve ser considerado como um ``Lav shebikhlalut'' porque
ao falar dos castigos a serem aplicados Ele separa os dois, dizendo ``ob''
ou ``yideoni'', e ordenando apedrejamento e extinção no caso de violação
voluntária de cada um deles. Suas palavras, enaltecido seja Ele, são "E
homem ou mulher que fizerem, magia ('ob') ou feitiçaria ('yideoni'),
serão mortos" (Levítico 20:27). A Sifrá diz: Em suas pa­lavras 'E homem
ou mulher que fizerem magia ou feitiçaria" está o castigo, mas não a
advertência. Por isso as Escrituras dizem: 'Não vos voltareis para as
ma­gias e para as feitiçarias' " (Ibid., 19:31).

Também neste caso aquele que transgredir a proibição involuntaria­mente
deverá levar um Sacrifício Determinado de Pecado.


As normas deste preceito estão explicadas no sétimo capítulo de


Sanhedrin.

\section{Não estudar as práticas da idolatria}

Por esta proibição somos proibidos de interessar-nos pela idolatria ou
de estudar suas práticas, ou seja, indagar a respeito de seus disparates
e super-tições ensinados pelos seus fundadores, como por exemplo que um
espírito po­de descer de uma determinada maneira e se comportará de uma
determinada forma; ou que se se queimar incenso para uma determinada
estrela e se seus ado­radores se colocarem diante dela numa determinada
posição, ela agirá de uma determinada maneira; e assim por diante. O
simples fato de pensar a respeito desses assuntos e de se informar sobre
essas ilusões leva os tolos a aproximar-se dos ídolos e a adorá-los. O
versículo das Escrituras que contém a proibição des­tas práticas é: "Não
vos volteis aos ídolos" (Levítico 19:4), a respeito do qual diz a Sifrá:
``Se você se voltar para eles você os endeusará''. A Sifrá cita também

236. O osso do pássaro.

as palavras de Rabi Yehudá: "Não vos volteis para vê-los", ou seja, nem
sequer olhe para o ídolo ou estude sua forma de idolatria, para não
perder nem um momento sequer com qualquer coisa relacionada com isso.

No capítulo ``Shoel Adam'' os Sábios dizem: "O que está escrito abai­xo de
um quadro ou de uma estátua não deve ser lido no Shabat. Quanto à
estátua em si, não se deve olhá-la nem mesmo durante os dias da semana
por­que foi dito 'Não vos volteis aos ídolos'. Como isto deve ser
interpretado? Rabi Yohanan disse: Não vos volteis ao que vossa própria
mente concebe".

A proibição de pensar a respeito dos ídolos está repetida em Suas
pa­lavras "Guardai-vos, não suceda que o vosso coração vos seduza, e vos
desvieis, e sirvais" (Deuteronômio 11:16). Quer dizer, se sua mente
cometer o erro de pen­sar em ídolos, isso o levará a extraviar-se do
caminho correto e a adorá-los. Ele ainda diz, sobre o mesmo assunto, "E
quiçá levantes os teus olhos para os céus, e vendo o sol, a lua e as
estrelas etc." (Ibid., 4:19). Ele não proíbe que se levante a cabeça e
se observe os corpos celestiais com os olhos; o que Ele proíbe é que se
olhe com os olhos da mente para aquilo que seus adoradores atribuem a
eles. Por isso também Suas palavras "E não indagues acerca dos seus
deuses, dizendo: De que modo serviam estas nações a seus deuses? Do
mesmo modo também fa­rei eu" (Deuteronômio 12:30) são uma advertência
para que não indaguemos a respeito de suas formas de adoração, ainda que
nós próprios não os adoremos, porque tudo isso nos conduz a seguí-los
por seus maus caminhos.

Você deve saber que todo aquele que transgredir esta proibição está
sujeito ao açoitamento. Isso foi deixado claro no final do primeiro
capítulo do Erubin, ao mencionar que o castigo de açoitamento está
prescrito na lei das Escrituras para aquele que violar as lei de ``erub''
dos Limites. Para confirmar isso foram citadas as seguintes palavras das
Escrituras: "Não saia ninguém (al) de seu lugar" (Êxodo 16:29); e quando
alguém objetou: "É o açoitamento o castigo pela violação da proibição
expressa por 'ai' e não por 'Io'?" foi-lhe res­pondido: "Se o castigo
pela desobediência da proibição expressa por 'al' não fosse o
açoitamento, o castigo pela desobediência da proibição 'Não nos (al)
volteis aos ídolos' não seria o açoitamento". Isto mostra que a
transgressão desta proibição é punida com o açoitamento.

\section{11 não erguer um pilar que as pessoas se reunirão para reverenciar}

Por esta proibição somos proibidos de erguer um pilar que as pessoas se
reunirão para reverenciar, mesmo que ele seja erguido com o propósito de
ado­ração ao Eterno. A razão disso é que não devemos, no serviço do
Eterno, imitar os idólatras, que tinham o hábito de erguer pilares e
colocar seus ídolos sobre eles. Esta proibição está expressa em Suas
palavras, enaltecido seja Ele, "Não levan­tarás para ti `matzebá',
porque o Eterno, teu Deus, odeia" (Deuteronômio 16:22).

A transgressão desta proibição é punida com, o açoitamento,

\section{Não esculpir pedras para prostrar-se sobre elas}

Por esta proibição somos proibidos de esculpir pedras para prostrar-nos
sobre elas, mesmo se elas tiverem sido feitas a serviço do Eterno. Isto
tam­bém é condenado para que não imitemos os idólatras, cujo hábito era
colocar
uma pedra magnificamente esculpida aos pés de um ídolo e prostrar-se
sobre ela em adoração. Suas palavras, enaltecido seja Ele, "Assoalho
sagrado de pe­dras, não poreis em vossa terra para prostrar-vos sobre
ele" (Levítico 26:1).

A punição pela transgressão desta proibição é o açoitamento.

A Sifrá diz: " São poreis em vossa terra': você está proibido de
prostrar-se sobre as pedras em sua terra, mas pode prostrar-se sobre as
pedras no Santuário".


As normas deste preceito estão explicadas na Guemará de Meguilá.

\section{Não plantar árvores no santuário}

Por esta proibição somos proibidos de plantar árvores no Santuário ou
junto ao Altar com a finalidade de adorno ou embelezamento, mesmo que
seja em adoração ao Eterno, porque era costume dos idólatras plantar
árvores bonitas e simétricas em honra aos ídolos, em suas casas de
adoração. Suas pala­vras, enaltecido seja Ele, são: "Não plantarás para
ti, nenhuma `ashera' nem ár­vore junto ao altar do Eterno, teu Deus"
(Deuteronômio 16:21).

A transgressão desta proibição é punida com o açoitamento.

As normas deste preceito estão explicadas na Guemará de Tamid, onde foi
deixado claro que é proibido todo tipo de planta no Santuário.

\section{Não jurar por um ídolo}

Por esta proibição somos proibidos de jurar por um ídolo mesmo estando
com idólatras, nem devemos fazê-los jurar em nome de um ídolo, de acordo
com as palavras da Mekhiltá "Você não deve levar um pagão a jurar por
sua divindade.' Esta proibição está expressa em Suaspalavras, enaltecido
seja Ele, ``E o nome de outros deuses não mencionareis'' (Exodo 23:13):
você não deve fazer com que o idólatra jure em nome de sua divindade.

No mesmo trecho a Mekhiltá também diz: " 'Não mencionareis' ---isso
significa que não se deve fazer um juramento em nome de um ídolo".

Em Sanhedrin está dito: " 'Não mencionareis' --- isto significa que não
se deve dizer a seu amigo 'Espere por mim junto a tal ídolo' ".

Todo aquele que transgredir esta proibição --- ou seja, que jurar em
nome de qualquer objeto criado que pessoas mal orientadas consideram
como divindade, como se fosse um ser superior\textsuperscript{237} ---
está sujeito ao açoitamento. A Guemará de Sanhedrin diz o seguinte, com
relação à proibição feita pelos Rabi­nos de beijar ou abraçar um ídolo,
de varrer o chão diante dele, ou de realizar outros atos semelhantes que
indiquem respeito e amor por ele: "O ofensor não será açoitado por
nenhum desses atos, a não ser por fazer uma promessa ou um juramento em
nome de um ídolo.


As normas deste preceito estão explicadas no sétimo capítulo de


Sanhedrin.

\section{Não convocar pessoas para a idolatria}

Por esta proibição somos proibidos de convocar pessoas para a práti­ca
da idolatria, isto é, convocar e estimular pessoas a adorar ídolos,
ainda que o próprio incitador não os adore e não faça nada além de
convocar outros a fazê-lo. Aquele que iludir uma comunidade é chamado de
``madiá'', de acordo com Suas palavras, enaltecido seja Ele, "Saíram
homens ímpios do meio de ti, e per­verteram (va-yedihu) os moradores da
sua cidade dizendo etc" (Deuteronômio 13:14). Aquele que iludir um único
indivíduo é chamado de ``messit'', de acordo com Suas palavras, enaltecido
seja Ele, "Quando te incitar (ye-sithkha) teu irmão de pai, ou teu irmão
de mãe... em segredo, dizendo etc." (Ibid., 7). No presente preceito
falamos apenas do ``madiá'' cuja ação está proibida por Suas palavras,
enaltecido seja Ele, ``Não seja ouvido de tua boca'' (Êxodo 23:13).

A Guemará de Sanhedrin diz: " São seja ouvido de tua boca' é a proibição
contra o `messit'. A isso foi objetado que há uma proibição explícita
contra desviar alguém do bom caminho nas palavras das Escrituras "E todo
Is­rael ouvirá e temerá, e não voltará a fazer uma coisa má como esta"
(Ibid., 12) e essa é a proibição contra o `madiá' ". A Mekhiltá de Rabi
Ishmael diz: " São seja ouvido de tua boca' é uma advertência contra o
`madiá' ".

O castigo pela transgressão desta proibição é o apedrejamento, co­mo
lemos em Sanhedrin: "Aqueles que levarem uma cidade apóstata pelo mau
caminho serão punidos com o apedrejamento".


As normas deste preceito estão explicadas no décimo capítulo de


Sanhedrin.

\section{Não tentar persuadir um israelita a adorar ídolos}

Por esta proibição somos proibidos de desencaminhar, ou seja, ten­tar
persuadir um israelita a adorar ídolos. Aquele que fizer isso será
chamado de ``messit'', como foi explicado anteriormente. A proibição está
expressa em Suas palavras, enaltecido seja Ele, "E não voltará a fazer
uma coisa má como esta, no meio de ti" (Deuteronômio 13:12).

A punição pela transgressão desta proibição --- ou seja, por
desenca­minhar um israelita --- é o apedrejamento, como está dito nas
Escrituras: ``Mas certamente o matarás'' (Ibid., 10). E o homem que o
``messit'' desejava desenca­minhar é aquele que deverá matá-lo, como Ele
claramente enunciou, enaltecido seja Ele: "A tua mão será a primeira
contra ele para o matar" (Ibid.), a respeito de que o Sifrei diz:
"Aquele que foi desencaminhado tem por obrigação matá-lo".


As normas deste preceito estão explicadas no sétimo capítulo de


Sanhedrin.

\section{Não amar a pessoa que deseja seduzi-lo para a idolatria}

Por esta proibição quem tiver sido desencaminhado\textsuperscript{239}
está proibido



de amar aquele que procura enganá-lo ou de prestar atenção ao que ele
diz. Ela está expressa em Suas palavras, enaltecido seja Ele, "Não lhe
cederás (to­veh)" (Deuteronômio 13:9).

O Sifrei diz: "Eu poderia pensar, em virtude do princípio geral 'E
amarás o teu próximo' (Levítico 19:18)\textsuperscript{240}, que somos
ordenados a amá-lo; por isso as Escrituras dizem 'Não lhe cederás
(toveh)' ".

\section{Não diminuir nossa aversão pelo enganador}

Por esta proibição quem tiver sido desencaminhado está proibido de
diminuir seu ódio pelo ``messit''. É sou dever incondicional odiá-lo e se
ele não o fizer estará infringindo um preceito negativo. Esta proibição
está expres­sa em Suas palavras, enaltecido seja Ele, ``E não o ouvirás''
(Deuteronômio 13:9), que estão explicadas da seguinte forma: "Como está
escrito 'Auxiliá-lo-às' (Êxodo 23\textsubscript{:}5.241) 
eu poderia pensar que somos ordenados a ajudar o 'inessie; por isso

as Escrituras dizem: 'E não o ouvirás' ".

\section{Não salvar a vida do enganador}

Por esta proibição quem tiver sido desencaminhado está proibido de
salvar a vida do `messit" se ele se encontrar em perigo. Ela está
expressa em Suas palavras, enaltecido seja Ele, "E os teus olhos não
terão piedade dele" (Deuteronômio 13:9), que são explicadas da seguinte
forma: "Uma vez que es­tá escrito 'Não sejas indiferente quando está em
perigo o teu próximo' (Levíti­co 19:16), eu poderia pensar que é
proibido ficar indiferente caso um `messit' esteja em perigo; por isso
as Escrituras dizem: 'E os teus olhos não terão pieda­de dele' .

\section{Não defender um enganador}

Por esta proibição quem tiver sido desencaminhado está proibido de
defender o enganador, e mesmo que ele tenha conhecimento de algum
ar­gumento em seu favor ele está proibido de sugerí-lo ao ``messit'' ou de
expô-lo ele mesmo. Esta proibição está expressa em Suas palavras,
enaltecido seja Ele, ``Não o pouparás'' (Deuteronômio 13:9), que são
explicadas como significan­do ``Você não deve advogar em seu favor''.

\section{Não omitir uma evidência que seja desfavorável ao enganador}

Por esta proibição quem tiver sido desencaminhado está proibido de
omitir qualquer coisa de que ele tenha conhecimento que seja
desfavorável


\begin{enumerate}
\def\labelenumi{\arabic{enumi}.}
\setcounter{enumi}{241}
\item
 
 Ver o preceito positivo 206.
 
\item
 
 Ver o preceito positivo 202.
 
\end{enumerate}

ao ``messit'' e que contribua para puní-lo. Ela está expressa em Suas
palavras, enaltecido seja Ele, ``E nem esconderás a sua culpa''
(Deuteronômio 13:9), que são explicadas como significando "Se souberes
algo desfavorável a ele não te é permitido omití-lo".

\section{Não tirar proveito de ornamentos que enfeitaram um ídolo}

Por esta proibição somos proibidos de tirar proveito de ornamentos com
os quais um ídolo tenha sido enfeitado. Ela está expressa em Suas
pala­vras, enaltecido seja Ele, "Não cobiçarás a prata e o ouro que está
sobre eles" (Deuteronômio 7:25). A Sifrá explica que são proibidas as
roupas que adorna­ram um ídolo e baseia essa proibição em Suas palavras,
enaltecido seja Ele, ``Não cobiçarás a prata e ouro que está sobre eles''.

A punição pela desobediência a esta proibição é o açoitamento. As normas
deste preceito estão explicadas no terceiro capítulo de Abodá Zará.

\section{Não reconstruir uma cidade apóstata}

Por esta proibição somos proibidos de reconstruir uma Cidade Após­tata.
Ela está expressa em Suas palavras, enaltecido seja Ele, "E será um
montão de ruína para sempre; não será reconstruída jamais" (Deuteronômio
13:17). O castigo pela reconstrução de qualquer parte dela --- ou seja,
por reedificá-la co­mo ela era antes --- é o açoitamento.


As normas deste preceito estão explicadas no décimo capítulo de


Sanhedrin.

\section{Não tirar proveito dos pertences de uma cidade apóstata}

Por esta proibição somos proibidos de usar ou tirar proveito dos bens de
uma Cidade Apóstata. Ela está expressa em Suas palavras, enaltecido seja
Ele, ``E não haverá na tua mão nenhuma coisa do anátema'' (Deuteronômio
13:18).

O castigo por pegar alguma coisa é o açoitamento.


As normas deste preceito estão explicadas no décimo capítulo de


Sanhedrin.

\section{Não aumentar nossa fortuna com qualquer coisa que provenha da idolatria}

Por esta proibição somos proibidos de aumentar nossa fortuna com
qualquer coisa que provenha da idolatria; ao contrário, devemos fugir
dela, de seus templos e de seus pertences. Esta proibição está expressa
em Suas palavras, enaltecido seja Ele, ``E não trarás abominação à tua casa''
(Deuteronômio 7:26).


O castigo por beneficiar-se de algum dos pertences é o açoitamento.


Foi deixado claro no final de Macot que aquele que acender lenha de
``Ashera''\textsuperscript{244} está sujeito a dois açoitamentos: um por
"E não trarás abomi­nação à tua casa", e um por ``E não haverá''
(Deuteronômio 13:18). Isto deve ser observado.

As normas deste preceito estão explicadas no terceiro capítulo de Abodá
Zará.

\section{Não fazer profecias em nome de um ídolo}

Por esta proibição um homem fica proibido de profetizar em nome de um
ídolo, ou seja, dizer que o Eterno lhe ordenou adorar um ídolo ou que o
próprio ídolo lhe ordenou adorá-lo, prometendo recompensá-lo e
ameaçan­do puni-lo, como pretextam os profetas de Baal e os de Ashera.

As Escrituras não contém uma proibição precisa e específica a esse
respeito, quer dizer, uma proibição contra profetizar em nome de um
ídolo, mas prescrevem um castigo, que é a morte, por essa ofensa através
de Suas pa­lavras, enaltecido seja Ele, "Que falar em nome de outros
deuses, este profeta morrerá" (Deuteronômio 18:20). Essa morte será por
estrangulamento porque temos uma regra geral de que quando as Escrituras
prescrevem a pena de mor­te sem maior especificação significa que deve
ser por estrangulamento.

Você já está familiarizado com a regra que expliquei nos quatorze
Fundamentos expostos na Introdução a este trabalho, e que foi
estabelecida pelos Sábios pelas palavras "As Escrituras nunca prescrevem
uma punição sem antes estabelecer uma proibição". Aqui a proibição se
deriva das palavras ``E o nome de outros deuses não mencionareis'' (Êxodo
23:13) pois não é impossí­vel que um único preceito negativo sirva como
advertência para várias proibi­ções, embora ele não seja um "lav
shebikhlalut", já que foi estabelecido um castigo diferente para cada
caso. Darei esclarecimentos sobre este princípio no momento apropriado.

As normas deste preceito estão explicadas no décimo primeiro capí­tulo
de Sanhedrin.

\section{Não fazer falsas profecias}

Por esta proibição somos proibidos de fazer profecias falsas, isto é,\\
de divulgar, em nome do Eterno, uma profecia que Ele, enaltecido seja,
não\\
tenha dito ou que tenha dito a outro que não aquele que se gaba de tê-la
ouvido
e que diz falsamente que o Eterno a comunicou a ele. Este preceito
negativo\\
está expresso em Suas palavras "Mas, o profeta que propositadamente
falar alguma
coisa em Meu Nome, que não lhe ordenei falar" (Deuteronômio
18:20).\\
Também neste caso o castigo pela desobediência da proibição é o\\
estrangulamento: o Talmud inclui ``um falso profeta'' na lista dos
transgressores
que devem ser estrangulados. Também está escrito no mesmo lugar:
"Há\\
três que estão sujeitos à morte pela mão do homem: 'Que propositadamente


244. Ver o preceito negativo 13.

falar alguma coisa em Meu Nome' significa aquele que profetiza o que ele
não ouviu; 'Que não lhe ordenei falar', subentendendo que isso foi
ordenado a ou­tro, significa aquele que profetiza o que não foi dito 'a
ele'; 'Ou que falar em nome de outros deuses' (Ibid.) significa aquele
que profetiza em nome de um ídolo. Com
referência a todos eles está escrito: 'Este profeta morrerá' , e toda
vez que a pena de morte for ordenada pelas Escrituras sem especificação,
signi­fica que deve ser por estrangulamento.

As normas da lei relativa a um falso profeta estão explicadas no déci­mo
primeiro capítulo de Sanhedrin.

\section{Não ouvir as profecias de quem profetiza em nome de um ídolo}

Por esta proibição somos proibidos de ouvir as profecias de alguém que
profetiza em nome de um ídolo, ou seja, não devemos entrar em debate com
ele nem fazer-lhe perguntas, dizendo-lhe "Qual é o teu milagre e que
pro­va tens dele?", como faríamos no caso de alguém que profetizasse em
nome do Eterno. Quando ouvirmos uma pessoa profetizando em nome de um
ídolo devemos repreendê-lo, pois é nosso dever repreender todo pecador,
e caso ele insista na sua afirmação devemos aplicar-lhe o castigo que
merece, de acordo com a lei das Escrituras, sem levar em consideração
seus milagres ou suas provas.

Esta proibição está expressa em Suas palavras, enaltecido seja Ele, "Não
obedecerás as palavras daquele profeta" (Deuteronômio 13:4).

As normas deste preceito estão explicadas no décimo primeiro capí­tulo
de Sanhedrin.

\section{Não ter piedade de um falso profeta}

Por esta proibição somos proibidos de ter pena de um falso profeta ou de
deixar de matá-lo porque ele profetiza em nome do Eterno. Não deve­mos
temer estar cometendo algum pecado, uma vez que sua falsidade nos foi
provada. Esta proibição está expressa em Suas palavras, enaltecido seja
Ele, ``Não o temerás'' (Deuteronômio 18:22), a respeito das quais diz o
Sifrei: " 'Não o temerás': não deixe de declará-lo culpado".

As normas deste preceito estão explicadas na Introdução ao nosso
``Comentário sobre a Mishná''.

\section{Não adotar os hábitos e costumes dos descrentes}

Por esta proibição somos proibidos de trilhar os caminhos dos
des­crentes e de adotar seus costumes, inclusive quanto a suas roupas e
suas reu­niões sociais. Esta proibição está expressa em Suas palavras,
enaltecido seja Ele, "E não andareis nos costumes da nação que Eu hei de
expulsar de diante de vós" (Levítico 20:23) e está repetida em Suas
palavras ``E não andeis segundo os seus costumes'' (Ibid., 18:3), a
respeito das quais comenta a Sifrá: "Eu ordenei ape­nas aquilo que foi
estabelecido como costumes para eles e seus ancestrais".

A Sifrá diz: " 'E não andeis segundo os seus costumes': não sigam seus
costumes sociais, ou seja, as coisas que se transformaram em hábitos
para eles, tal como teatros, circos e arenas --- sendo que esses são
vários tipos de locais de reunião onde eles se juntam para a adoração de
ídolos. Rabi Meir diz: Esses se referem aos 'costumes dos Amoritas'
enumerados pelos Sábios. Rabi Yehudá ben Betera diz que você não deve se
barbear ao redor da cabeça ou deixar crescer a franja de seus cabelos ou
raspar o cabelo de sua testa".

A punição por qualquer uma dessas ofensas é o açoitamento.

Esta proibição está repetida sob outra forma, em Suas palavras
"Guarda-te não te unires a elas com receio de que sejas levado a
seguí-las" (Deu­teronômio 12:30) sobre as quais diz o Sifrei: "
'Guarda-te' é um preceito nega­tivo; 'Com receio de que' é um preceito
negativo; 'Sejas levado a seguí-las' ---com receio de que te compares a
elas e sigas seus costumes e elas se transfor­mem numa armadilha para
ti. Você não deve dizer: Assim como eles se vestem em púrpura, eu me
vestirei em púrpura; assim como eles usam 'telusin' --- um tipo de
adorno usado pelos soldados --- eu também usarei o 'telusin' ". E você
conhece as palavras do profeta "Todo que estiver vestido com ornamentos
es­trangeiros"\textsuperscript{245}. A finalidade de tudo isto é que
devemos evitar os idólatras e menosprezar todos os seus hábitos, até
mesmo o de suas roupas.

As normas deste preceito estão explicadas no sexto capítulo de Sha­bat,
e também no Tosseftá desse Tratado.

\section{Não fazer adivinhações}

Por esta proibição somos proibidos de fazer adivinhações, isto é, fa­zer
uso de qualquer um dos vários meios de estimular a capacidade de
suposi­ção, pois todos aqueles que têm o poder de predizer o futuro o
fazem porque a capacidade de suposição está altamente desenvolvida
neles, e de uma manei­ra geral ela opera corretamente; conseqüentemente,
eles têm um pressentimento do que vai acontecer, sendo que alguns deles
são superiores a outros, assim como entre todos os homens alguns superam
outros numa determinada capa­cidade da alma.

Contudo, os que têm esses poderes de suposição tentam estimulá-los e
ativá-los por um ou outro meio. Um deles baterá várias vezes no chão com
sua bengala, emitirá gritos estranhos, e se concentrará durante um longo
perío­do de tempo, até que caia num tipo de transe e comece a predizer o
futuro. Uma vez eu vi isso no extremo Oeste. Outro jogará seixos num
pedaço de pele, olhará para eles por um longo tempo e então fará a
profecia --- uma prática co­mum em todos os lugares que visitei. Outro
ainda jogará um longo cinto de couro no chão, o observará, e fará a
profecia. O objetivo de tudo isto é estimu­lar os poderes que ele
possui; seu ritual não produz nenhum efeito nem lhe fornece nenhuma
informação.

É nesse ponto que a maioria das pessoas se engana. Quando algumas
predições se tornam realidade, eles pensam que essas práticas realmente
reve­lam o futuro e persistem nesse erro ao ponto de chegar a acreditar
que algumas dessas práticas são a causa dos acontecimentos que se
seguem, tal como os as­trólogos costumam acreditar. A arte da astrologia
é, na realidade, semelhante a isto, no sentido de que ambos são meios de
estimular essa capacidade. Portanto não há 
dois homens iguais no que se refere à veracidade de suas
profe­cias, embora muitos possam ser iguais quanto a seu conhecimento da
arte.

Aquele que se envolve com uma destas práticas, ou outras práticas do
mesmo tipo, é chamado de ``kossem'', adivinho; e o Eterno, enaltecido seja
Ele, diz: "Não se achará entre ti... nem adivinho (kossem kessamim)"
(Deutero­nômio 18:10). A este respeito o Sifrei diz: "O que é um
`kossem'? Aquele que tomar sua bengala em suas mãos e disser: 'Devo ou
não devo ir?' " . É com rela­ção a este tipo de adivinhação, comum
naquela época, que o profeta disse: "Meu povo pede conselho a sua vara e
sua bengala se manifesta a eles"\textsuperscript{246}.

Aquele que cometer esta transgressão --- ou seja, que praticar
adivi­nhações, e fizer coisas que lhe permitam predizer o futuro ---
estará sujeito ao açoitamento. Isto não se aplica àquele que consulta o
adivinho, embora o ato de consultá-lo seja censurável ao extremo.

As normas deste preceito estão explicadas em vários trechos na Gue­mará
de Sanhedrin, na Tosseftá de Shabat, e no Sifrei.

\section{Não orientar nossa conduta pelas estrelas}

Por esta proibição somos proibidos de orientar nossa conduta pelas
estrelas, ou seja, não devemos dizer "Este dia é favorável para um
determinado empreendimento e nós vamos realizá-lo", ou "Este dia é
desfavorável para um determihado empreendimento e não o executaremos".
Esta proibição está ex­pressa em Suas palavras, enaltecido seja Ele,
"Não se achará entre ti... nem prog­nosticador (meonen)" (Deuteronômio
18:10) e está repetida em Suas palavras "E não prognosticareis
(teonenu)" (Levítico 19:26), a respeito das quais diz a Sifrá: " 'E não
prognosticareis': isto se refere àqueles que predizem os tempos ---
sendo que a palavra ``teonenu'' é derivada de ``oná'' (que significa tempo,
época). Isto significa que não deverá ser encontrado entre vocês um
vidente que decide que um momento é favorável e outro não.

O castigo pela transgressão desta proibição --- ou seja, por aconse­lhar
quanto aos momentos -\/- é o açoitamento. Isto não se aplica àquele que
faz a consulta, mas tal tipo de consulta também está proibida, além de
ser uma fraude. Todo aquele que deliberadamente escolher uma determinada
época para fazer algo, baseado numa previsão de boa sorte ou sucesso,
também deve ser açoitado porque ele terá realizado uma ação.

Esta proibição também se extende aos truques de ilusão de ótica. Os
Sábios dizem: " `Meonen' se refere a alguém que engana com ilusões de
óti­ca"; isto inclui um grande número de truques realizados por
prestidigitação, que dão aos homens a ilusão de verem coisas que não
existem. Essas pessoas têm seus truques habituais, como pegar uma corda
e diante dos olhos dos es­pectadores, guardá-la num canto de seu traje
para então retirar dali uma cobra; ou jogar um anel para cima e depois
retirá-lo da boca de uma das pessoas que estiver diante dele; e há
outros truques de mágica similares, muito populares. Todos os truques
desse tipo são proibidos e todo aquele que os pratica é cha­mado de
enganador; e como seus truques são um tipo de feitiçaria, sua punição é
o açoitamento. Ele também é um enganador de pesoas e causa grandes
da­nos ao fazer com que aquilo que não pode realmente existir pareça
possível


aos olhos de tolos, homens, mulheres e crianças, habituando-os assim a
aceitar como possíveis coisas impossíveis. Isto deve ser observado.

\section{Não praticar a vidência}

Por esta proibição somos proibidos de praticar a vidência, como fa­zem
as pessoas que dizem: "Como eu interrompi minha viagem, não terei
su­cesso"; ou "A primeira coisa que vi hoje foi isto e aquilo: este será
certamente um dia proveitoso para mim". Tais exemplos são extremamente
comuns entre as pessoas de nações atrasadas.

Todo aquele que permitir que sua conduta seja influenciada por
pres­ságios estará sujeito ao açoitamento, de acordo com Suas palavras,
enaltecido seja Ele, "Não se achará entre ti... nem adivinho, nem
feiticeiro (menahesh)" (Deuteronômio 18:10). Isso está repetido nas
palavras "Não augurareis (tena­hashu)" (Levítico 19:26); e o Sifrei diz:
" `Menahesh': como aquele que diz 'O pão caiu de minha boca', 'O bastão
caiu de minha mão', 'Uma cobra passou pela minha direita', ou 'Uma
raposa passou pela minha esquerda' ". E na Sifrá lê-se: " 'Não
augurareis', como fazem aqueles que tiram presságios de uma do­ninha, ou
dos pássaros, ou das estrelas, e assim por diante". '

As normas deste preceito também estão explicadas no sétimo capí­tulo de
Shabat e na Tosseftá desse Tratado.

\section{Não praticar feitiçaria}

Por esta proibição somos proibidos de praticar feitiçaria. Ela está
ex­pressa em Suas palavras, enaltecido seja Ele, "Não se achará entre
ti... nem feiti­ceiro (mekhashef)" (Deuteronômio 18:10).

Aquele que deliberadamente desobedecer esta proibição está sujei­to ao
apedrejamento. Aquele que a transgredir involuntariamente deve levar um
Sacrifício Determinado de Pecado. As Escrituras dizem: "Feiticeira não
dei­xarás viver" (Êxodo 22:17).


As normas deste preceito estão explicadas no sétimo capítulo de


Sanhedrin.

\section{Não praticar a arte do encantador}

Por esta proibição somos proibidos de praticar a arte do encantador
(``hober''), ou seja, dizer palavras de encantamento que se supõe que
tenham determinados efeitos positivos ou negativos. Ela está expressa em
Suas pala­vras, enaltecido seja Ele, "Não se achará entre ti... nem
encantador (hober ha­ber)" (Deuteronômio, 18:10-11), sobre as quais o
Sifrei diz: " `flober haber' significa um encantador de serpentes ou de
escorpiões"; quer dizer, ele recita palavras de encantamento para eles
para que --- acredita ele --- eles não o mor­dam, ou se ele já tiver
sido mordido, para diminuir a dor.

O castigo pela desobediência desta proibição é o açoitamento.

As normas deste preceito também estão explicadas no sétimo capí­tulo de
Shabat.

\section{Não consultar um necromante que use o ``ob''}

Por esta proibição somos proibidos de consultar um necromante que use o
``ob'', e de tentar obter informações dele. Ela está expressa em Suas
pala­vras, enaltecido seja Ele, "Não se achará entre ti... nem
necromante (shoel ob)" (Deuteronômio 18:10-11).

A desobediência a esta proibição --- ou seja, consultar um necromante
que se utilize de um ``ob'' --- não é punida com a morte, mas mesmo assim
a sua prática é proibida.


\section{Não consultar um feiticeiro que se utilize do``yidoa''}


Por esta proibição somos proibidos de consultar um feiticeiro que se
utilize do ``yidoa'' e de tentar obter informações dele. Ela está expressa
em Suas palavras, enaltecido seja Ele, "Não se achará entre ti... ou
yideoni" (Deute­ronômio 18:10-1 1). A Sifrá diz: '"Não vos volteis para
as magias (ob) e para as feitiçarias (yideoni)' (Levítico 19:31): o
'ob', ou seja, o piton que fala de debaixo de suas axilas, e o
`yideoni', que fala de dentro de sua boca, são punidos com o
apedrejamento e aquele que os consulta é punido com o
açoitamento\textsuperscript{247}.

\section{Não tentar obter informações com os mortos}

Por esta proibição somos proibidos de tentar obter informações com os
mortos --- com aqueles que se pensa que estão mortos embora comam e
te­nham sensações --- pensando que se alguém fizer determinadas coisas e
se vestir de uma determinada maneira, os mortos virão durante seu sono e
responderão suas perguntas. Esta proibição está expressa em Suas
palavras, enaltecido seja Ele, "Não se achará em ti... nem quem consulte
os mortos (doresh el hametim) (Deuteronômio 18:10-11), a respeito das
quais a Guemará de Sanhedrin diz: " 'Do­resh el hametim' significa
aquele que se deixa morrer de fome e passa a noite num cemitério para
que o espírito de um demônio possa descansar nele".

O castigo pela transgressão desta proibição é o açoitamento.


\section{As mulheres não devem usar roupas ou adornos masculinos}


Por esta proibição também somos proibidos de seguir ps costumes dos
hereges no que se refere a mulheres usarem roupas ou adornos
masculi­nos. Ela está expressa em Suas palavras, enaltecido seja Ele,
``Não haverá traje de homem na mulher'' (Deuteronômio 22:5).

Toda mulher que usar um adorno que é sabido ser usado só pelos homens
naquela região está sujeita ao açoitamento.

247. Se ele se conduzir da maneira indicada pelo feiticeiro.


\section{Os homens não devem usar roupas ou adornos femininos}

Por esta proibição os homens também estão proibidos de se enfeita­rem
com adornos femininos. Ela está expressa em Suas palavras, enaltecido
se­ja Ele, ``E não usará o homem vestido de mulher'' (Deuteronômio 22:5).
O ho­mem que colocar adornos ou trajes sabidos ser de uso exclusivamente
femini­no naquela região também é punido com o açoitamento.

Você deve saber que este costume --- ou seja, as mulheres se enfeita­rem
com adornos masculinos e os homens com adornos femininos --- é algu­mas
vezes adotado com o intuito de despertar o desejo carnal, como é comum
entre as nações, e algumas vezes com a finalidade de adoração de um
ídolo, como está explicado nos livros dedicados a esse assunto. Também é
uma práti­ca comum estipular, com relação à confecção de determinados
talismãs, que se aquele que os fizer for um homem ele deve usar trajes
femininos e enfeitar-se com ouro, pérolas e coisas desse tipo, e se for
mulher, ela deve usar armadu­ra e rodear-se de armas. Isto é bem sabido
daqueles que são conhecedores do assunto.

\section{Não fazer marcas em nossos corpos}

Por esta proibição somos proibidos de fazer qualquer marca --- azul,
vermelha ou de qualquer outra cor --- em nossos corpos, assim como fazem
os idólatras, como é comum entre os Koptim até hoje. A proibição está
expres­sa em Suas palavras, enaltecido seja Ele, "E escrita de tatuagem
não poreis em vós" (Levítico 19:28).

O castigo pela desobediência desta proibição é o açoitamento. As normas
deste preceito estão explicadas no final do Tratado Macot.


\section{Não usar roupas de lã e linho}


Por esta proibição somos proibidos de usar uma roupa tecida com lã e
linho, como os sacerdotes dos ídolos costumavam fazer naquela
época\textsuperscript{248}. Ela está expressa em Suas palavras,
enaltecido seja Ele, "Não te vestirás com estofos misturados (shaatnez)
de lã e linho juntamente" (Deuterondmio 22:11). Este costume é comum
entre os monges Coptas do Egito àtualmente.

O castigo por transgredir esta proibição é o açoitamento.

As normas deste preceito estão explicadas nos Tratados Quilaim e Shabat,
e no final de Macot.

\section{Não raspar os cabelos das têmporas}

Por esta proibição somos proibidos de cortar o cabelo das têmpo­ras. Ela
está expressa em Suas palavras, enaltecido seja Ele, "Não cortareis o

248. Isto é, na época da revelação da Torá.

cabelo de vossa cabeça em redondo" (Levítico 19:27). O objetivo desta
proibi­ção também é para que não imitemos os idólatras, porque eles
tinham o costu­me de raspar apenas o cabelo de suas têmporas. De acordo
com isso, os Sábios julgaram necessário explicar no Tratado Yebamot que
"Barbear toda a cabeça é considerado como 'arredondar' ", para que não
se argumente que o objetivo desta proibição é impedir-nos de raspar os
cabelos das têmporas assim como o resto do cabelo (da cabeça), como
fazem os sacerdotes idólatras, mas que ras­par todo o cabelo não é
imitação deles. Por esse motivo os Sábios nos dizem que em circunstância
alguma nos é permitido raspar os cabelos das têmporas, quer se raspe
apenas as têmporas ou todo o cabelo da cabeça; e que cada têm­pora é
punível com o açoitamento, de forma que aquele que barbear toda sua
cabeça deve ser açoitado duas vezes. Apesar disso, não contamos esta
proibi­ção como dois preceitos, porque há apenas uma negação e não duas.
Se Ele tivesse dito: "Você não deve arredondar o canto direito da
cabeça, nem o can­to esquerdo", e víssemos que os Sábios prescreveram
dois castigos, teria sido possível considerá-la como dois preceitos; mas
como é um único assunto abran­gido numa única expressão, trata-se na
realidade de um único preceito. Embo­ra esta proibição seja interpretada
como abrangendo várias partes do corpo, e estejamos sujeitos a
açoitamento por cada uma dessas partes, separadamente, isso não a
transforma necessariamente em mais de um preceito.

As normas deste preceito estão explicadas no final de Macot. Ele não é
obrigatório para as mulheres.

\section{Não raspar a barba}

Por esta proibição somos proibidos de raspar a barba, que tem cin­co
partes: a maxila superior direita, a maxila superior esquerda, a maxila
infe­rior direita, a maxila inferior esquerda e a ponta da barba. A
proibição comple­ta está expressa nas palavras "E não raspareis os
cantos de vossa barba" (Levíti­co 19:27) porque todas as partes da barba
estão incluídas no termo ``barba''. As Escrituras não dizem "Nem raspareis
vossa barba", mas sim "Nem raspareis 'os cantos' de vossa barba",
significando que não se deve raspar nem um canto da barba, a qual, de
acordo com a Tradição, compõe-se de cinco ``cantos'', co­mo explicado
acima, e fica-se sujeito a cinco açoitamentos se se raspar toda a barba,
ainda que se raspe a barba toda de uma só vez.

A Mishná diz: "Cinco vezes pela barba: duas vezes pelo lado direito,
duas vezes pelo esquerdo e uma vez pela parte inferior. Rabi Eliezer
diz: Se ele tirou toda a barba num só movimento, ele está sujeito a um
único castigo; disso o Talmud conclui: "Portanto, Rabi Eliezer considera
o todo uma proibição". Assim, temos uma prova clara que o Primeiro
Sábio\textsuperscript{249} é de opinião que há cin­co proibições, e essa
é a lei.

Esse era também um costume dos sacerdotes idólatras, e é sabido que
atualmente os sacerdotes Europeus raspam suas barbas.

A razão pela qual isso não deve ser contado. como cinco preceitos é que
a proibição trata de um único assunto numa única expressão, como
ex­plicamos no preceito precedente.

As normas deste preceito estão explicadas nó final de Macot. Ele não é
obrigatório para as mulheres.

249. Que discorda de Rabi Eliezer.

\section{Não fazer cortes em nossa carne}

Por esta proibição somos proibidos de fazer cortes em nossa carne, como
fazem os idólatras. Ela está expressa em Suas palavras, enaltecido seja
Ele, "Não fareis cortes em vossa carne (lo titgodedu)" (Deuteronômio
14:1), e tam­bém, sob outra forma, em Suas palavras "E incisões (seret)
por um morto, não fareis em vossa carne" (Levítico 19:28).

A Guemará de Yebamot explica que " titgodedu' é obrigatório

pelo seu próprio contexto, já que o Todo Misericordioso disse "Não vos
feri­reis por causa dos mortos".

A Guemará de Macot diz: " `Seritá' (cortar com a mão) e `guedidá'
(cortar com um instrumento) são a mesma coisa". Também está explicado lá
que quem o fizer por causa dos mortos, seja com a mão ou com um
instrumen­to, estará sujeito ao açoitamento; na prática da idolatria,
ele estará sujeito ao castigo se o fizer com um instrumento, mas estará
isento se o fizer com a mão, pois no livro dos profetas encontramos o
seguinte: "Eles se cortam (vayitgode­du) de acordo com seus métodos, com
espadas e lanças"\textsuperscript{25}°.

O Talmud diz que este preceito negativo também proíbe dividir as pessoas
e criar facções e discórdia, interpretando ``lo titgodedu'' como "Não
deveis separar-vos em facções (agudot)". O verdadeiro significado do
versícu­lo, contudo, é, como explicam os Sábios, "Não vos ferireis por
causa dos mor­tos"; o outro é meramente um ``derash''.

Da mesma forma, o fato de eles dizerem que "aquele que for inflexí­vel
numa discussão viola um preceito negativo, pois está dito que "Para que
não seja como Korah, e como sua congregação" (Números 17:5), também é um
``derash'', pois o verdadeiro objetivo do versículo é o de dissuadir. Da
forma como os Sábios explicam o versículo, ele contém uma declaração
negativa e não uma proibição, sendo que a interpretação deles é que o
Eterno declara que todo aquele que no futuro contestar a autoridade dos
``Cohanim'' e reivindicar o sacerdócio para si mesmo não será castigado
com a punição determinada pa­ra Korah --- ou seja, não será tragado pela
terra --- mas será punido "conforme tinha falado o Eterno, por
intermédio de Moisés" (Números 17:5), isto é, com a lepra, como quando
Ele disse a Moisés: "Leva, por favor, a tua mão ao teu peito". (Êxodo
4:6) e como está dito do rei Uziah\textsuperscript{251}.

Para voltar ao assunto deste preceito, suas normas estão explicadas no
final de Macot, e o castigo pela desobediência desta proibição é o
açoitamento.

\section{Não se fixar na terra do egito}

Por esta proibição somos proibidos de estabelecer-nos na terra do Egito,
a fim de que não aprendamos a heresia dos Egípcios nem sigamos seus
costumes, que são repulsivos para a Torah. Esta proibição está expressa
em Suas palavras, enaltecido seja Ele, ``Nem fará voltar o povo ao Egito''
(Deuteronômio


\begin{enumerate}
\def\labelenumi{\arabic{enumi}.}
\setcounter{enumi}{249}
\item
 
 1. Reis 18:28.
 
\item
 
 II. Cron. 26:16-21.
 
\end{enumerate}


17:16). Ela aparece três vezes nas:crituras e, de acordo com os Sábios,
"Em três lugares a Torah avisa Israel pa não retornar ao Egito; mas três
vezes eles retornaram e três vezes foram punidos". Nós já mencionamos o
primeiro dos três avisos; o segundo está expresso em Suas palavras "Pelo
caminho que te tenho dito: Não voltarás mais para vê-lo" (Deuteronômio
28:68); e o terceiro em Suas palavras "Porque os Egípcios que vedes hoje
não volvereis a vê-los nunca mais" (Êxodo 14:13). Embora de acordo com
seu significado literal esta seja uma afirmação, ela é tradicionalmente
entendida como sendo uma proibição.

No final de Guemará de Sucá está explicado que Alexandria está in­cluída
entre os lugares onde é proibido estabelecer-se e que a totalidade da
ter­ra do Egito, na qual não podemos viver, compreende uma área de 400
parasan­gas quadradas medidas a partir do mar em Alexandria. Contudo, é
permitido atravessar esta área para fins comerciais, ou passar por ela a
caminho de outro país. Está dito explicitamente no Talmud de Jerusalém:
"Você não deve retor­nar para estabelecer-se, mas pode retornar para
fins de comércio, negócios e conquista do país".

\section{Não aceitar opiniões contrárias às ensinadas na torah}

Por esta proibição somos proibidos de exercer a liberdade dos
pen­samentos no que se refere a aceitar opiniões contrárias às que nos
são ensina­das pela Torah; devemos, ao contrário, limitar nossos
pensamentos, e levantar uma barreira em torno deles, formada pelos
preceitos positivos e negativos da Torah. Esta proibição está expressa
em Suas palavras, enaltecido seja Ele, "E não errareis indo atrás de
vosso coração e atrás de vossos olhos, atrás dos quais vós andais
errando" (Números 15:39), sobre as quais diz o Sifrei: " 'E não
erra­reis indo atrás de vosso coração' --- isto significa heresia, pois
está escrito: 'E se eu encontrar mais amargo do que a
morte'\textsuperscript{252}. 'E atrás de vossos olhos' ---isto significa
prostituição, pois está escrito: 'E Sansão disse a seu pai'
"\textsuperscript{253}. ``Prostituição'' aqui significa perseguir e pensar
constantemente em prazeres físicos e indulgências.

\section{Não fazer uma aliança com as sete nações idólatras de canaã}

Por esta proibição somos proibidos de fazer uma aliança com os he­reges,
ou seja, com as Sete Nações\textsuperscript{254}, e de deixá-los
tranqüilos em sua here­sia. Ela está expressa em Suas palavras,
enaltecido seja Ele, ``Não farás aliança alguma com elas'' (Deuteronômio
7:2).


Já explicamos, ao tratar do preceito positivo 187 que a guerra con-

\begin{enumerate}
\def\labelenumi{\arabic{enumi}.}
\setcounter{enumi}{251}
\item
 
 Eclesiástico, 7:26.
 
\item
 
 Juízes 14:3.
 
\item
 
 As Sete Nações de Canaã: os hiteus, os guirgasheus, os emoreus, os
 cananeus, os periseus, os hiveus e os jebuseus.
 
\end{enumerate}



tra as Sete Nações, e os outros preceitos relativos a elas devem ser
incluídos\textsuperscript{255}, e não são de tempo limitado.

\section{Não poupar a vida de um homem das sete nações idólatras}

Por esta proibição somos proibidos de poupar a vida de qualquer homem
que pertença a uma das Sete Nações para evitar que eles corrompam as
pessoas e as levem para o caminho errôneo da idolatria. Esta proibição
está expressa em Suas palavras, enaltecido seja Ele, "Não deixarás com
vida todo que tiver alma" (Deuteronômio 20:16). Matá-los constitui um
preceito positi­vo, como explicamos ao tratar do preceito positivo 187.

Todo aquele que transgredir esta proibição, deixando de matar to­do
aquele que ele poderia ter morto estará infringindo um preceito
negativo.

\section{Não demonstrar compaixão para com os idólatras}

Por esta proibição somos proibidos de demonstrar compaixão para com os
idólatras ou de elogiar qualquer coisa que lhes pertença. Ela está
ex­pressa em Suas palavras, enaltecido seja Ele, "E não terás
misericórdia deles (lo tehonem)" (Deuteronômio 7:2), que são
tradicionalmente interpretadas co­mo significando: "Não lhes concedereis
nenhuma graça (hen)". E ainda que um idólatra tenha uma boa aparência,
somos proibidos de dizer que ``Ele tem boa aparência'' ou "ele tem um belo
rosto", como está explicado em nossa Guemará.

A Guemará de Abodá Zará no Talmud de Jerusalém diz que há um preceito
negativo que proíbe conceder-lhes alguma graça.

\section{Não permitir que idólatras residam em nossa terra}

Por esta proibição somos proibidos de permitir que idólatras resi­dam em
nossa terra, para que não aprendamos suas heresias. Ela está expressa em
Suas palavras, enaltecido seja Ele, "Não morarão em tua terra, quiçá te
fa­çam pecar contra Mim" (Êxodo 23:33). Se algum idólatra desejar
permanecer em nossa terra não devemos permitir que o faça a menos que
ele renegue a idolatria; nesse caso ele poderá se tornar um residente.
Tal pessoa é conhecida como um ``guer toshab'', o que significa que ele é
um prosélito apenas no sen­tido de que lhe é permitido residir em nossa
terra. Assim, os Sábios dizem: "Quem é um Suer toshab'? De acordo com
Rabi Yehudâ, é aquele que rene­gou a idolatria".

"Contudo, um adorador de ídolos não deve residir entre nós, nem podemos
vender-lhe um imóvel ou alugar a\_ele: isto é rigorosamente
interpre­tado como significando 'não permitirás que se fixem (hanaya) na
terra' ".



As normas deste preceito estão explicadas em Sanhedrin e Abodá Zará.


\section{Não se unir pelo matrimônio a hereges}

Por esta proibição somos proibidos de unir-nos pelo matrimônio a
hereges. Ela está expressa em Suas palavras, enaltecido seja Ele, "E não
te apa­rentarás com elas" (Deuteronômio 7:3). O termo ``aparentarás'' é
explicado da séguinte forma: "Tua filha não darás a seu filho, e sua
filha não tomarás para teu filho" (Ibid.). Está exposto claramente no
Tratado Abodá Zará que "A To­rah proíbe a união pelo matrimônio".

O castigo por desobedecer esta proibição varia de acordo com as
circunstâncias. Se um homem mantiver publicamente um relacionamento com
uma mulher pagã, aquele que o matar quando ele estiver cometendo a
trans­gressão estará desse modo aplicando o castigo que Pinhas aplicou a
Zimri. Por isso a Mishná diz: "Se um homem coabitar com uma mulher pagã,
ele será pu­nido por fanáticos". Isto, contudo, só é permitido sob
certas condições, ou se­ja, quando a transgressão for feita abertamente,
e enquanto o ato estiver acon­tecendo, como naquele
caso\textsuperscript{256}. Contudo, quando o ato não for cometido
pu­blicamente ou não for punido pelos fanáticos naquele momento, o
transgres­sor está sujeito à extinção, embora isto não esteja prescrito
pela Torah. Apare­ce no Talmud a pergunta: "E se os fanáticos não o
punirem?" e a resposta é que ele está sujeito à extinção, como consta
nas palavras das Escrituras: "Pois Yehudá tinha profanado a santidade do
Eterno que o amava, e tinha se casado com a filha de um deus estranho. O
Eterno destruirá o homem que fizer isso, o mestre e o
aprendiz"\textsuperscript{257}. Está dito: "Isto mostra que ele está
sujeito à extin­ção". Portanto, quando ficar provado que um homem teve
relação com uma mulher pagã diante de testemunhas, apesar de ter sido
categoricamente adver­tido, ele está sujeito a açoitamento pela
autoridade da Torah. Você deve obser­var isso.

A lei a respeito de todos esses assuntos está explicada em Abodá Za­rá e
Sanhedrin.


\section{Não se unir pelo matrimônio a um homem amonita ou moabita}


Por esta proibição fica proibido o casamento com um varão amoni­ta ou
moabita, mesmo depois que ele tenha se tornado prosélito. Ela está
ex­pressa em Suas palavras, enaltecido seja Ele, "Não entrará nenhum
Amonita, e nem Moabita na congregação do Eterno" (Deuteronômio 23:4).

O castigo pela contravenção desta proibição é o açoitamento, ou se­ja,
se um homem amonita ou moabita prosélito se casar com uma mulher
israe­lita ambos estão sujeitos ao açoitamento, de acordo com a lei das
Escrituras.

As normas deste preceito estão explicadas no, oitavo capítulo de
Ye­bamot, e no final de Kidushin.


\begin{enumerate}
\def\labelenumi{\arabic{enumi}.}
\setcounter{enumi}{255}
\item
 
 Como no caso de Pinhas e Zimri (Números 25:7).
 
\item
 
 Mal. 2:11-12.
 
\end{enumerate}



\section{Não excluir os descendentes de esaú}

Por esta proibição somos proibidos de excluir os descendentes de
Esaú\textsuperscript{258} depois que eles tiverem se tornado prosélitos,
ou seja, de recusarmos a unir-nos a eles pelo matrimônio. Esta proibição
está expressa em Suas pala­vras, enaltecido seja Ele, "Não abominarás o
Edumeu, porque é teu irmão" (Deu­teronômio 23:8).


\section{Não afastar os descendentes dos egípcios}


Por esta proibição somos proibidos de afastar os egípcios e de
recu­sarmos a unir-nos a eles pelo matrimônio, depois que eles tiverem
se tornado prosélitos. Esta proibição está expressa em Suas palavras,
enaltecido seja Ele, ``Nem abominarás o egípcio'' (Deuteronômio 23:8).

As normas destes dois preceitos --- relativos aos egípcios e aos
edu­meus --- estão explicadas no oitavo capítulo de Yebamot, e no final
de Kidushin.

\section{Não oferecer a paz a amon nem a moab}

Por esta proibição somos proibidos de oferecer a paz a Amon ou Moab. O
Eterno nos ordenou que quando estivermos a ponto de sitiar uma ci­dade
devemos pedir a seus habitantes, antes de iniciar as hostilidades, que
eles se submetam e não guerreiem conosco; e se eles nos entregarem a
cidade, fica­mos proibidos de entrar em guerra com eles ou matá-los,
como explicamos ao tratar do preceito positivo 190. Mas no caso de Amon
e de Moab não devemos seguir esse procedimento; o Eterno nos proibiu de
oferecer-lhes a paz e de pe­dir para que se submetam. Esta proibição
está expressa em Suas palavras, enal­tecido seja Ele, "Não lhes
procurarás nem paz, nem bem (Deuteronômio 23:7). A este respeito o
Sifrei diz: "Eu poderia pensar que a regra que diz: 'Quando te
aproximares de uma cidade para pelejar contra ela, oferecer-lhe-às a
paz'(Ibid., 20:10) deve ser aplicada também neste caso. Por isso as
Escrituras dizem: 'Não lhes procurarás nem paz'. E como está escrito:
'No lugar que escolher (Ibid., 23:17), eu poderia pensar que aqui
novamente a regra também se aplica. Por isso as Escrituras dizem: 'Nem
bem, em todos os teus dias para sempre' ".


\section{Não destruir árvores frutíferas durante um assédio}


Por esta proibição somos proibidos de destruir árvores frutíferas
du­rante um cerco a fim de causar escassez e sofrimento aOs habitantes
da cidade sitiada. Ela está expressa em Suas palavras, enaltecido seja
Ele, "Não destruirás

258. Irmão gêmeo de Jacob (ou Israel).

o seu arvoredo... pelo que não o cortarás" (Deuteronômio 20:19). Toda
des­truição está incluída nesta proibição; por exemplo, todo aquele que
queimar uma roupa ou quebras um recipiente desnecessariamente estará
desobedecen­do a proibição ``Não destruirás'', e estará sujeito ao
açoitamento.

No final de Macot está explicado que aquele que cortar ``árvores boas''
está sujeito ao açoitamento. A esse respeito os Sábios comentam: "A
ad­vertência quanto a isso está expressa nas palavras das Escrituras
"Podeis comer de seus frutos, mas não deveis cortá-las".

As normas deste preceito estão explicadas no segundo capítulo de Baba
Batra.

\section{Não temer os hereges em tempos de guerra}

Por esta proibição somos proibidos de temer os hereges em tempos de
guerra, ou de recuar diante deles; ao contrário, é nosso dever ser
corajosos e reunir todas as nossas forças para resistir na linha de
batalha. Todo aquele que fugire recuar infringirá o preceito negativo
expresso em Suas palavras "Não te aquebrantarás. diante deles"
(Deuteronômio 7:21) e também em Suas pala­vras ``Não os temais'' (Ibid.,
3:22).

O preceito contra recuar ou ceder terreno em batalha está repetido
muitas vezes nas Escrituras porque esta é uma das situações em que lhe é
possí­vel defender a verdadeira fé.


As normas deste preceito estão explicadas no oitavo capítulo de Sotá.


\section{Não esquecer o que Amalec nos fez}

Por esta proibição somos proibidos de esquecer o que Amalec nos fez e de
como ele nos atacou sem ter sido provocado\textsuperscript{259}. Nós já
explicamos, ao falar do preceito positivo 189, que é um preceito
positivo lembrar o que Amalec nos fez e manter vivo nosso ódio por ele.
Da mesma forma somos proi­bidos, por um preceito negativo, de
negligenciar ou esquecer este assunto. Es­ta proibição está expressa em
Suas palavras, enaltecido seja Ele, "Não te esque­cerás" (Deuteronômio
25:19). O Sifrei diz: "Lembra-te (Ibid.,17) --- com pala­vras; 'Não te
esquecerás' --- dentro de teu coração". Quer dizer, não devemos abrandar
nosso ódio por Amalec, nem retirá-lo de nossos corações.


\section{Não blasfemar o grande nome}


Por esta proibição somos proibidos de blasfemar o grande Nome,
enaltecido seja Ele muito, muito acima de todas as palavras dos hereges.
Isto é o que é chamado de ``abençoar o Nome''. As Escrituras prescrevem
expressa­mente o apedrejamento como punição pela desobediência da
proibição expressa em Suas palavras, enaltecido seja Ele, "E aquele que
blasfemar o nome real do Eterno, certamente será morto; toda a
congregação o apedrejará" (Levítico 24:16). Mas as Escrituras não
destacam este pecado como uma proibição ex-

259. Ver Êxodo 17:8-13.

pressa; ele está incluído na proibição geral expressa em Suas palavras
"Aos juí­zes não maldigas" (Êxodo 22:27). A Mekhiltá diz: "Nas palavras
das Escrituras `Aquele que blasfemar o nome real do Eterno, certamente
será morto' ouvimos a penalidade por este pecado, mas não ouvimos a sua
proibição. Por isso as Escrituras dizem 'Aos juízes não maldigas' ". E,
de acordo com a Sifrá, "A pe­nalidade pelo Nome Especial é a morte e por
um dos adjetivos é o açoitamen­to". A Mekhiltá diz ainda: " 'Aos juízes
não maldigas': esta é a proibição contra blasfemar".


As normas deste preceito estão explicadas no sétimo capítulo de


Sanhedrin.

Você deve saber que este tipo de proibição, que engloba dois ou três
assuntos específicos, não está incluída na categoria de `lav
shebikhlalut" por­que as Escrituras especificam a punição por cada
transgressão separadamente; conseqüentemente, sabemos que cada um deles
é proibido e é o objeto de um preceito negativo, como explicamos na
Introdução deste livro. Como é nosso princípio que não se prescreve
nenhum castigo a menos que uma proibição o tenha precedido, somos
obrigados a procurar a proibição, a qual algumas ve­zes descobrimos
através de uma das Leis de Interpretação, e outras vezes en­contramos
numa passagem que trata de outro assunto, como explicamos na
In­trodução. Um ``lav shebikhlalut'' subsiste apenas quando não
encontramos ne­nhuma outra base para a proibição de qualquer um dos atos
em questão a não ser naquele determinado preceito negativo, como
explicamos no Nono Funda­mento. Contudo, quando nos tiver sido ensinado
que isto ou aquilo é proibido --- já que as Escrituras dizem que aquele
que fizer uma determinada coisa incor­rerá num determinado castigo ---
não importa se a advertência foi mencionada explicitamente ou se foi
deduzida por raciocínio específico ou geral. Você de­ve compreender este
princípio pois haverá muitos preceitos mais aos quais ele se aplica.

\section{Não violar um shebuat bitui"}

Por esta proibição somos proibidos de violar um ``shebuat bitui''. Ela
está expressa em Suas palavras, enaltecido seja Ele, "E não jurareis
falso em Meu nome" (Levítico 19:12).

O termo ``shebuat bitui'' significa um juramento através do qual ju­ramos
fazer ou não fazer algo que a Lei não ordene nem proíba. Devemos
cum­prir um juramento desse tipo e as palavras "Não jurareis falso em
Meu nome" nos proíbem de violá-lo.

A Guemará de Shebuot diz: "O que é um `shebuat sheker'? Jurar o
contrário. Isto foi corrigido para: Jurar e inverter"; ou seja, jurar
fazer alguma coisa e fazer o contrário daquilo que se jurou.

A Guemará explica no terceiro capítulo de Shebuot e também no Tra­tado
Temurá que ``shebuat sheker'' (um falso juramento) é o não cumprimento de
um ``shebuat bitui''. Essa explicação está exposta da seguinte forma:
"Esse falso juramento, de que tipo é ele?" quer dizer, de acordo com o
contexto, o que se quer dizer por um falso juramento que não acarreta
nenhuma ação? "Devemos dizer que significa jurar não comer e depois
fazê-lo? Mas neste caso a ação foi realizada. Devemos, então, concluir
que o que se quer dizer é jurar comer e não comer. Mas existe uma
penalidade de açoitamento? Seguramente foi-nos dito ... etc".


O castigo pela transgressão voluntária desta proibição é o açoita-


mento; se alguém a transgredir involuntariamente ele deve oferecer um
Sacrifí­cio de Maior ou Menor Valor, como explicamos no preceito
positivo 72. Isto baseia-se na seguinte passagem do terceiro capítulo de
Shebuot: "Este é um `she­buat bitui', por cuja violação voluntária
fica-se sujeito ao açoitamento; se ela for violada inconscientemente
deve-se oferecer um Sacrifício de Maior ou Me­nor Valor". As normas
deste preceito estão explicadas nessa passagem.

Quando afirmei que o castigo pela transgressão voluntária deste
pre­ceito é o açoitamento você deve saber que eu não quis dizer que há
um pecado punido com o açoitamento, mesmo que ele não tenha sido
cometido delibera­damente. Toda vez que você me ouvir afirmar que uma
determinada transgres­são é punível com o açoitamento --- quer seja no
que precede ou no que se segue --- você deve saber que isto se aplica
unicamente a um pecado cometido voluntariamente, na presença de
testemunhas, e desafiando uma advertência formal, como está explicado no
Tratado Sanhedrin com relação às determina­ções sobre testemunhas e à
advertência formal. Aquele que pecar sem querer ou sob coação, ou em
virtude de uma falsa informação em circunstância algu­ma estará sujeito
ao açoitamento ou à extinção, e menos ainda à execução judi­cial. Isto
se aplica a todos os preceitos, e deve ser registrado.

No caso de alguns preceitos, nós realmente afirmamos que a viola­ção
voluntária é punível com o açoitamento ou a morte, porque o mesmo
pe­cado cometido involuntariamente acarreta oferta de um sacrifício. A
razão dis­so é que nem todos os pecados cometidos involuntariamente
acarretam a ofer­ta de um sacrifício. Mas toda vez que a penalidade por
uma transgressão for o açoitamento, a extinção ou a execução judicial,
não se fica sujeito à punição a menos que o pecado tenha sido cometido
na presença de testemunhas e desafiando-se uma advertência formal. É
sabido que o único objetivo da adver­tência formal é para que se possa
fazer a distinção entre a violação por ignorân­cia ou proposital.


Você deve conhecer este princípio, e não espere que eu torne a


repetí-lo.

\section{Não fazer um ``shebuat shav''}

Por esta proibição somos proibidos de fazer um ``shebuat shav'' (um
juramento em vão). Ela está expressa em Suas palavras, enaltecido seja
Ele, "Não jurarás em nome do Eterno, teu Deus, et• vão" (Êxodo 20:7).
Ela nos proíbe de jurar que um objeto existente é o que de fato ele não
é, ou que algo impossí­vel existe, ou de jurar violar qualquer um dos
preceitos da Torah. Da mesma forma, jurar um fato evidente, que nenhuma
pessoa instruída negaria ou ques­tionaria, como por exemplo jurar-se
pelo Eterno que tudo aquilo que for dego­lado morrerá, isso também é um
caso de tomar o nome do Eterno em vão. Nas palavras da Mishná: "O que é
um `shebuat shav'? Jurar o contrário dos fatos conhecidos pelo homem".

A punição pela transgressão deliberada desta proibição é o açoita­mento;
aquele que a transgredir involuntariamente está isento, assim como
to­dos os demais que forem culpados de transgredir um preceito negativo,
como explicamos anteriormente. Está dito em Shebuot que o castigo por um
"she­buat shav" é o açoitamento se a ofensa for deliberada, e que não há
punição se ela for inconsciente. As normas deste preceito estão ali
explicadas.

\section{Não profanar o nome de deus}

Por esta proibição somos proibidos de profanar o Nome. Isto é o
contrário da santificação do Nome, que nos é ordenada pelo nono preceito
po­sitivo, e que explicamos ali. Esta proibição está expressa em Suas
palavras, enal­tecido seja Ele, "E não profanareis o nome de Minha
santidade" (Levítico 22:32)

Este pecado abrange três tipos de ações, duas das quais são possí­veis
para qualquer um e a terceira apenas para certas pessoas.

O primeiro tipo de ação, que é possível para qualquer pessoa, é es­te.
Todo aquele que, em época de perseguição, é forçado a transgredir um dos
preceitos, se seu perseguidor tem em mente fazê-lo cometer um pecado de
maior ou menor gravidade, ou todo aquele que for forçado, mesmo sem ser
em épo­cas de perseguição, a cometer o pecado de idolatria, incesto ou
derramamento de sangue, deve sacrificar sua vida e submeter-se à morte
em vez de transgre­dir, como explicamos em relação ao nono preceito
positivo. Se ele cometer o pecado e escapar assim da morte, ele terá
profanado o Nome e infringido este preceito negativo; e se isso
acontecer em público, isto é, na presença de dez israelitas, ele estará
profanando o Nome publicamente e pecando contra Suas palavras,
enaltecido seja Ele, ``Não profanareis o nome de Minha santidade'' e seu
crime será muito grave. Mas ele não será punido com o açoitamento,
por­que agiu sob coação, e um tribunal só tem o direito de impor o
açoitamento ou a morte a alguém que tenha pecado intencionalmente, de
sua livre vontade, diante de testemunhas e desafiando uma advertência
formal. Como diz o Sifrei daquele que entrega um filho a Molekh: "'Porei
Eu a Minha ira contra aquele homem' (Levítico 20:5): os Sábios dizem
'aquele homem', mas não o horrlem que pecar sob coação ou sem intenção
de fazê-lo, ou devido a uma falsa infor­mação". Assim foi deixado claro
que aquele que adora ídolos sob pressão não está sujeito à extinção, e
menos ainda à execução judicial, mas é culpado de ter profanado o Nome.

O segundo tipo de ação, que também é possível a qualquer um, é cometer
uma transgressão que, embora não motivada por cobiça ou desejo de lucro,
demonstre indiferença e relaxamento de comportamento. Um homem que agir
dessa forma também é culpado de profanar o Nome e está sujeito ao
açoitamento; por conseguinte as Escrituras dizem: "E não jurareis falso
em Meu nome, profanando o nome de vosso Deus" (Levítico 19:12), porque
ele demons­tra indiferença, embora não obtenha nenhum benefício
material.

O tipo de ação possível apenas para certos indivíduos é aquele
reali­zado por um homem de religiosidade e virtudes conhecidas que
pareça ao po­vo ser uma transgressão e ser impróprio de um homem tão
devoto, embora na realidade seja algo permitido. Tal procedimento também
é uma profanação do Nome de acordo com os Sábios, que dizem: "O que
constitui uma profana­ção do Nome? Se por exemplo eu levar carne do
açougue e não pagar imediata­mente por ela. Rabi Yohanan diz: No meu
caso, andar quatro cúbitos sem a Torah ou sem os `tefilin' ".

Esta proibição também se encontra em'outro lugar nas palavras das
Escrituras "E não profanarás o nome de teu Deus; Eu sou o Eterno"
(Levítico 18:21).


As normas deste preceito estão explicadas ern Pessahim e no final


de Yoma.

\section{Não testar suas promessas e advertências}

Por esta proibição somos proibidos de testar Suas promessas e amea­ças,
enaltecido seja Ele, transmitidas a nós por Seus Profetas, lançando
dúvidas sobre elas, depois de saber que aquele que as enunciou é um
verdadeiro Profe­ta. Esta proibição está expressa em Suas palavras "Não
experimentareis ao Eter­no, vosso Deus" (Deuteronômio 6:16)

\section{Não demolir casas •de adoração ao eterno}

Por esta proibição somos proibidos de demolir casas de adoração ao
Eterno, de destruir livros de profecia, ou de apagar os Nomes Sagrados e
coisas semelhantes. Esta proibição está expressa em Suas palavras "Não
proce­dereis de modo semelhante para com o Eterno, vosso Deus"
(Deuteronômio 12:4). Ao ordenar-nos destruir os ídolos, apagar seus
nomes e demolir comple­tamente seus templos e
altares\textsuperscript{260}, Ele acrescenta a proibição "Não
procedereis de modo semelhante para com o Eterno, vosso Deus".

O castigo pela transgressão de qualquer pormenor desta proibição --- tal
como demolir uma parte qualquer do Santuário, ou o Altar, ou algo
se­melhante, ou apagar qualquer um dos Nomes do Eterno --- é o
açoitamento. A Guemará, depois de explicar no final de Macot que queimar
lenha pertencen­te ao Santuário é punível com açoitamento, acrescenta:
"Esta proibição se en­contra nas palavras: 'E suas árvores idolatradas,
queimareis no fogo... Não pro­cedereis de modo semelhante para com o
Eterno, vosso Deus' (Ibid., 3-4)". As­sim também, depois de explicar que
a penalidade por apagar o Nome Divino é o açoitamento, ela continua: "A
proibição se encontra nas palavras 'Fareis perecer os seus nomes daquele
lugar. Não procedereis de modo semelhante para com o Eterno' (Ibid.)".


As normas deste preceito estão explicadas no quarto capítulo de


Shebuot.

\section{Não deixar o corpo de um criminoso pendurado durante 
toda a noite após sua execuçao}

Por esta proibição somos proibidos de deixar o corpo, depois da
exe­cução, pendurado durante toda a noite numa árvore, para que sua
visão não dê origem a pensamentos sacrílegos. O enforcamento só é
praticado entre nós nos casos do blasfemador e do idólatra, a respeito
de quem também foi dito: "Ao

PRECEITOS NEGATIVOS 223

Eterno ele blasfema" (Números 15:30). Esta proibição está expressa em
Suas palavras, enaltecido seja Ele, "Não pernoitará seu cadáver no
madeiro" (Deute­ronômio 21:23), a respeito das quais o Sifrei diz: "
'Não pernoitará seu cadáver no madeiro' é um preceito negativo".


As normas deste preceito estão explicadas no sexto capítulo de


Sanhedrin.

\section{Não interromper a vigilância do santuário}

Por esta proibição somos proibidos de interromper a vigilância do
Santuário, que deve ser continuamente patrulhado durante toda a noite.
Esta proibição está expressa em Suas palavras "E fareis o serviço
(ush'martem) d.. guarda da santidade" (Números 18:5). Nós já explicamos,
ao tratar do preceito positivo 22, que manter vigilância e patrulhar o
Santuário é um preceito positi­vo e aqui mostraremos que negligenciar
isto é infringir um preceito negativo. A Mekhiltá diz: " 'E farão o
serviço da guarda da tenda da assinação' (Números 18:4) é apenas um
preceito positivo; de que modo concluímos que há um pre­ceito negativo?
Pelas palavras das Escrituras: 'E fareis o serviço da guarda da
santidade' ".


As normas deste preceito estão explicadas no início de Tamid e Midot.


\section{O ``cohen gadol'' não deve entrar no santuário 
em outras ocasiões além das ESTABELECIDAS}

Por esta proibição o ``Cohen Gadol'' fica proibido de entrar no San­tuário
a todo e qualquer momento, por causa do respeito devido ao Santuário e
do temor que se deve ter da Presença Divina. Esta proibição está
expressa em Suas palavras, enaltecido seja Ele, "Que não venha a toda a
hora a santida­de" (Levítico 16:2).

Esta proibição inclui várias restrições. O ``Cohen Gadol'' está proi­bido
de entrar no Santíssimo até mesmo em ``Yom Quipur'' a não ser nos
mo­mentos determinados para o serviço\textsuperscript{261}; e todos os
``Cohanim'' estão proibi­dos de entrar no Santuário a qualquer momento
durante o ano a não ser no momento do serviço. Em resumo, fica proibido
a qualquer ``Cohen'' entrar no local que lhe é permitido entrar --- ou
seja, no Santuário Interno, no caso do ``Cohen Gadol'', e no Santuário
externo, no caso de ``Cohen'' comum --- a não ser durante o serviço.

Aquele que violar esta proibição entrando fora da hora de serviço está
sujeito à morte se ele entrar no Santíssimo, e ao açoitamento se ele
entrar no Santuário.

A Sifrá diz: " 'Que ele não venha a toda a hora' se refere a `Yom
Qui­pur'; 'A santidade' abrange também o resto do ano; 'para dentro da
cortina'
(Ibid.) faz com que a proibição se aplique a todo o Santuário.
Poder-se-ia pen­sar que entrar em qualquer lugar do Santuário é punível
com a morte; por isso as Escrituras dizem: 'Diante do propiciatório que
está sobre a arca para que não morra' (Ibid.). O que significa isto?
Entrar 'Diante do propiciatório' é punível com a morte, e, no resto do
Santuário, com o açoitamento".

A Guemará de Menahot diz explicitamente: "Fica-se sujeito ao
açoi­tamento por entrar no Santuário".

\section{Um ``cohen'' com um defeito não deve entrar em nenhuma parte do santuário}

Por esta proibição um ``Cohen'' com um defeito está proibido de entrar em
qualquer parte do Santuário, ou seja, no Altar, ou no espaço entre o
Pórtico e o Altar, ou no próprio Pórtico, ou no Santuário propriamente
dito. Esta proibição está expressa em Suas palavras, enaltecido seja
Ele, "Somente até o véu não virá, e o altar não se chegará" (Levítico
21:23).

Está explicado no início de Teharot que aquele que tiver um defei­to, ou
cujo cabelo estiver solto está proibido de entrar no espaço entre o
Pórti­co e o Altar ou em qualquer outra parte do Santuário. A Sifrá
também explica que essas duas proibições, ``Somente até o véu não virá'' e
``Ao altar não se chegará'' não seriam suficientes uma sem a outra, e que
ambas são necessárias para complementar a lei a esse respeito, que é a
lei que define o local onde eles estão proibidos de entrar. Aquele que
voluntariamente entrar além do Altar es­tá sujeito ao açoitamento, mesmo
se ele não entrar com a finalidade de minis­trar o serviço.

\section{Um ``cohen'' com um defeito não deve ministrar no santuário}

Por esta proibição um ``Cohen'' que tiver um defeito físico está proi­bido
de ministrar. Ela está expressa em Suas palavras "(O homem..) em que
hou­ver algum defeito, não se chegará para oferecer" (Levítico 21:17);
quer dizer, não o deixem aproximar-se para realizar o serviço.

Um ``Cohen'' defeituoso que ministrar está sujeito ao açoitamento, como a
Sifrá diz: "Um 'Cohen' defeituoso está sujeito não à morte, mas apenas
ao açoitamento".

\section{Um ``cohen'' com um defeito temporário não deve ministrar no santuário}

Por esta proibição um ``Cohen'' com um defeito passageiro fica proi­bido
de ministrar enquanto ele estiver com o defeito. Esta proibição está
expressa em Suas palavras, enaltecido seja Ele, "Todo o homem em que
houver algum defeito, não se aproximará" (Levítico 21:18), sobre as
quais diz a Sifrá: " 'Todo o homem em que houver algum defeito': isto me
fala apenas de um 'Cohen'
permanentemente defeituoso; como saber que o mesmo se aplica àquele que
tiver um defeito passageiro? Pelas palavras das Escrituras 'Todo o homem
em que houver algum defeito, não se aproximará' ".

Um ``Cohen'' com um defeito passageiro que ministrar também, es­tá sujeito
ao açoitamento.

As normas deste preceito, com relação a defeitos passageiros e
tem­porários nos homens, estão explicadas no sétimo capítulo de
Bekhorot.

\section{Os levitas e os ``cohanim'' não devem realizar as tarefas uns dos outros}

Por esta proibição os Levitas ficam proibidos de executar qualquer uma
das tarefas designadas aos ``Cohanim'', e os ``Cohanim'' de realizar
qual­quer uma das tarefas designadas aos Levitas porque cada uma dessas
duas famí­lias, isto é, os ``Cohanim'' e os Levitas, tem sua tarefa
específica no Santuário. Portanto, há uma advertência conjunta Dele,
enaltecido seja Ele, a ambos, para que nenhum deles execute o trabalho
do outro, e para que cada um faça aquilo de que foi incumbido, como Ele
disse, ``Cada um no seu ofício e no seu cargo'' (Números 4:19).

A proibição a esse respeito está expressa em Suas palavras, dirigidas
aos Levitas, "Salvo aos objetos da santidade e ao altar não se chegarão
para que não morram" (Números 18:3). Depois disso, Ele diz aos ``Cohanim''
"Para que não morram, tanto ele como vós" (Ibid.), sendo que as palavras
``Como vós'' significam "Esta proibição se aplica também a vós, os
`Cohanim'. Assim como eu os proibi de realizar vosso trabalho também
vocês estão proibidos de reali­zar os deles".

O Sifrei diz: " 'Aos objetos da santidade e ao altar não se chegarão' é
a proibição; 'Para que não morram' enuncia o castigo. O versículo me diz
apenas que os Levitas estão sujeitos à punição e estão proibidos de
executar o trabalho dos `Cohanim'. Como saber que os `Cohanim' também
estão proibi­dos de executar o trabalho dos Levitas? Pelas palavras das
Escrituras 'Tanto eles'. Como saber também que um grupo não deve fazer o
trabalho de outro gru­po?262. Pelas palavras 'Como vós'. Aconteceu uma
vez que Rabi Yehoshuá ben Hananyá quis ajudar Rabi Yohanan ben Gudgoda;
então este lhe disse: 'Volte! Você já foi privado de sua vida, pois eu
sou um dos guardiães do portão e você é um dos cantores' ".

Assim, foi deixado claro que todo Levita que executar uma tarefa que não
lhe tiver sido designada estará sujeito à morte pela mão dos Céus. Da
mesma forma, os ``Cohanim'' não devem se ocupar das tarefas dos Levitas;
mas em caso de transgressão, eles não estão sujeitos à morte, e sim
apenas ao açoitamento.

A Mekhiltá diz: " 'Salvo aos objetos da santidade e ao altar não se
che­garão': poder-se-ia pensar que eles estão sujeitos à morte meramente
por tocar, por isso as Escrituras dizem 'Salvo' (ach); eles só ficam
sujeitos à pena se realiza­rem o trabalho. Aqui novamente, o versículo
me fala apenas sobre os Levitas que realizarem o trabalho dos `Cohanim';
de que forma fico sabendo sobre os
`Cohanim' que realizam o trabalho dos Levitas? Pelas palavras 'Como vós'
". Também está dito ali: "Levitas que realizarem o trabalho dos
`Cohanim' estão sujeitos à morte, mas `Cohanim' que realizarem o
trabalho dos Levitas são cul­pados apenas de transgredir um preceito
negativo".


\section{Não entrar no santuário nem pronunciar uma 
sentença sobre uma lei da torah estando intoxicado}

Por esta proibição somos proibidos de entrar no Santuário ou de
pro­nunciar uma sentença a respeito de qualquer uma das Leis da Torah,
se estiver­mos intoxicados. Ela está expressa em Suas palavras,
enaltecido seja Ele, "Vi­nho e bebida forte não bebereis ... quando
entrardes à tenda da revelação ... e para ensinar aos filhos de Israel"
(Levítico 1 0:8-1 1). O Talmud diz: "Aquele que tiver bebido um
quarto\textsuperscript{263} não deve pronunciar uma sentença legal".

Há diferentes castigos por esta proibição. Todo aquele que se embe­bedar
com vinho está proibido de entrar entre o Pórtico e o Altar ou em
qual­quer parte do próprio Santuário e, se ele o fizer, estará sujeito à
punição por açoitamento. Se ele realizar o serviço enquanto estiver
intoxicado a penalidade é a morte pela mão dos Céus; mas se a
intoxicação tiver sido ocasionada por outro intoxicante que não o vinho
o castigo será apenas o açoitamento, não a morte. Entretanto, se alguém,
``Cohen'' ou israelita, pronunciar uma senten­ça estando bêbado, seja com
vinho ou outro intoxicante, ele transgredirá um preceito negativo.

A Sifrá diz: " 'Vinho ... não bebereis': isto me fala apenas do vinho;
como fico sabendo que isto se aplica da mesma forma a outras bebidas
intoxi­cantes? Pelas palavras das escrituras 'E bebida forte'. Se é
assim, por que as Es­crituras mencionam especificamente o vinho? Porque
o vinho sujeita à morte, enquanto que qualquer outra bebida intoxicante
não".

Também está escrito no mesmo lugar: "Como saber que não se fica sujeito
ao castigo a não ser durante o serviço? Pelas palavras das Escrituras
'Tu e teus filhos contigo, quando entrardes à tenda de assinação, e não
morrereis' ". Mais adiante lemos: "Poder-se-ia pensar que um israelita
que pronunciar uma sentença está sujeito à morte; por isso as Escrituras
dizem: 'Tu e teus filhos con­tigo ... e não morrereis'. Tu e teus filhos
estareis sujeitos à morte, mas um israe­lita não está sujeito à morte
por pronunciar uma sentença".


As normas deste preceito estão explicadas no quarto capítulo de


Queretot.

\section{Um ``zar'' não deve oficiar no santuário}

Por esta proibição um ``zar'' fica proibido de oficiar. Por ``zar'' eu quero
dizer todo aquele que não é descendente de Aarão. Esta proibição está
expressa em Suas palavras, enaltecido seja Ele, "E o estranho (zar) não
se apro-

263. Um quarto de um ``log'' de vinho.


ximará de vós" (Números 18:4). As Escrituras expõem claramente que
aquele que violar esta proibição está sujeito à morte pela mão dos Céus:
``E o estranho que se aproximar será morto'' (Ibid., 7). A esse respeito o
Sifrei diz: " 'E o es­tranho que se aproximar será morto': ouvimos a
penalidade, mas não ouvimos a proibição. Por isso as escrituras dizem:
'E o estranho não se aproximará de vós' ".

A proibição e a penalidade relacionadas com este assunto estão
re­petidas em Suas palavras "E não se aproximarão mais os filhos de
Israel ... para que não levem sobre si, pecado, e morram" (Ibid., 22).

A Guemará de Yoma detalha os serviços cuja realização por um ``zar'' lhe
acarretam a morte: "Há quatro serviços por cuja execução um `zar' fica
su­jeito à morte: aspergir, queimar, a libação do vinho, e a libação da
água".

As normas deste preceito estão explicadas nesse lugar e no último
capítulo do Tratado Zabahim.


\section{Um ``cohen'' impuro não deve oficiar no santuário}


Por esta proibição um ``Cohen'' que estiver impuro fica proibido de
realizar o serviço. Ela está expressa em Suas palavras, enaltecido seja
Ele, ende­reçadas aos ``Cohanim'', "Que se separem quando estão impuros
das santida­des dos filhos de Israel ... e não profanem o nome de Minha
santidade" (Levíti­co 22:2).

No nono capítulo de Sanhedrim lemos: "Como sabemos que ao rea­lizar o
serviço estando impuro ele está sujeito à morte? Porque está escrito
'Fa­la a Aarão e a seus filhos, para que se separem ... e não profanem'.
Está dito em outro trecho 'Para não ... pois morrerão por isto quando o
profanarem' (Ibid., 8-9); assim como no caso anterior, 'profanação'
envolve a penalidade de morte pela mão dos Céus, portanto aqui também as
palavras São profanem o nome da Minha santidade' significam que se eles
profanarem o Nome, realizando o serviço enquanto estiverem em estado de
impureza, eles estão sujeitos à morte pela mão dos Céus".

\section{Um ``cohen'' que praticou um ``tebul yom'' não deve oficiar no santuário}

Por esta proibição um ``Cohen'' que tenha praticado um ``Tebul yom'' fica
proibido de oficiar até o pôr-do-sol, ainda que ele já tenha se
purificado.". Ela está expressa em Suas palavras,
enaltecido seja Ele, relativas aos ``Cohanim'', "Não profanarão o nome de
seu Deus" (Levítico 21:6).

Todo aquele que violar esta proibição --- ou seja, que oficiar após um
``Tebul yom'' --- está sujeito à morte pela mão dos Céus.

Ela não está claramente enunciada nas Escrituras, contudo ela é
tra­dicionalmente interpretada assim. Está escrito no nono capítulo de
Sanhedrin, com relação à interpretação de Suas palavras, enaltecido seja
Ele, "Santos serão
para Deus, e não profanarão o nome de seu Deus": "Uma vez que isto não
po­de se referir a um 'Cohen' que oficiar impuro, o que está proibido
por um ou­tro versículo, como já foi explicado, apliquem-no a quem
oficiar depois de um `Tebul yom'. E pode-se deduzir uma analogia do uso
da palavra 'profanação' neste caso e no caso da oferta de elevação". E
se oficiar depois do ``Tebul yom'', ele estará contado ali entre os
transgressores que estão sujeitos à morte.


\section{Uma pessoa impura não pode entrar em nenhuma parte do santuário}


Por esta proibição toda pessoa impura fica proibida de entrar em
qual­quer parte do Santuário --- o que equivale, em gerações
posteriores\textsuperscript{269}, a todo o Campo do Santuário, que
começa com o Campo dos Israelitas, a partir do Portão de Nicanor. Esta
proibição está expressa em Suas palavras "Para que não contaminem os
seus acampamentos" (Números 5:3), significando o Acampa­mento da
Presença Divina.

A Guemará de Macot diz: "Com relação a uma pessoa impura que entrar no
Santuário as Escrituras expressam ambos a penalidade e a proibição. A
penalidade: 'Será banida aquela alma ... porque o santuário do Eterno
conta­minou' (Número 19:20). A proibição: 'Para que não contaminem os
seus acam­pamentos' ". E a Mekhiltá diz: " 'Ordena aos filhos de Israel
que enviem do acampamento ...' (Ibid.,5:2) é um preceito
positivo.". Como sabemos que há um preceito negativo?
Pelas palavras das Escrituras 'Para que não contaminem' ".

Esta proibição está repetida de outra forma em Suas palavras,
refe­rentes a uma mulher depois do parto, ``E no santuário não entrará''
(Levítico 12:4).

A Sifrá diz: "As palavras 'E separareis os filhos de Israel de suas
im­purezas, e não morrerão .,. (Ibid., 15:31) poderiam ser compreendidas
como aplicáveis em caso de ambos o interior e o exterior, significando
que todo aquele que tocasse o exterior do Santuário em estado que
impureza também estaria sujeito à extinção. Por isso as Escrituras dizem
'E no Santuário não entrará' ". Ali também está explicado que a lei para
uma mulher depois do parto é a mes­ma que para as pessoas impuras em
geral, no que se refere a esta proibição.

A Sifrá também diz, com referência a Suas palavras, enaltecido seja Ele,
"E se não lavar e não banhar sua carne, levará sobre si a sua
iniqüidade" (Levítico 17:16): "O que significa isto? Que se não se
lavar, ele estará sujeito à extinção; que se não lavar suas roupas, ele
estará sujeito ao açoitamento. E como sabemos que o versículo se refere
apenas à contaminação do Santuário e de suas coisas sagradas? Porque
ambas a proibição e a penalidade estão enun­ciadas" etc..

Está explicado em outro lugar que aquele que deliberadamente vio­lar
esta proibição está sujeito à extinção, e aquele que a violar
involuntariamen­te deve levar um Sacrifício de Maior ou Menor Valor,
como explicamos no pre­ceito positivo 72.

As normas deste preceito estão explicadas no início de Shebuot, em
Horayot, em Queretot, e em vários trechos do Tratado Zebahim.


\begin{enumerate}
\def\labelenumi{\arabic{enumi}.}
\setcounter{enumi}{264}
\item
 
 No Templo de Jerusalém (Beis Hamikdash).
 
\item
 
 Ver o preceito positivo 31.
 
\end{enumerate}



\section{Uma pessoa impura não pode entrar no acampamento dos levitas}

Por esta proibição toda pessoa impura fica proibida de entrar no
acampamento dos Levitas --- o que equivale, em gerações
posteriores\textsuperscript{267}, ao Monte do Templo, como explicamos no
início do Tratado Quelim, onde se fa­la da exclusão de pessoas impuras
do Monte do Templo. A proibição das Escri­turas a este respeito está
expressa em Suas palavras, relativas àquele que "está impuro por causa
daquilo que aconteceu com ele à noite": "Não entrará em nenhum
acampamento" (Deuteronômio 23:11).

A Guemará de Pessahim também diz: " 'Sairá para fora do acampa­mento':
isto é, para fora do Acampamento da Presença Divina", como explica­mos
ao falar do preceito positivo 31. " São entrará em nenhum acampamen­to':
isto é, dentro do Acampamento dos Levitas. A isto Rabina objetou:
Supo­nha que ambos se refiram ao Acampamento da Presença Divina e que
ele esteja violando, dessa forma, um preceito positivo e um negativo. Se
fosse assim, as Escrituras diriam 'Sairá para fora do acampamento' e
'Não entrará nele'; quer dizer, ele teria dito simplesmente 'Não entrará
nele'. Qual é a finalidade de re­petir 'o acampamento'? É para
determinar outro acampamento", que é o Acam­pamento dos Levitas: também
nesse ele não deve entrar.

O Sifrei diz: " São entrará em nenhum acampamento' é um precei­to
negativo".

As normas deste preceito estão explicadas em nosso ``Comentário'', no
primeiro capítulo do Tratado Quelim.


\section{Não construir um altar com pedras que tenham sido tocadas por ferro}


Por esta proibição somos proibidos de construir um Altar com pe­dras que
tenham sido tocadas por ferro. Ela está expressa em Suas palavras,
enal­tecido seja Ele, ``Não o edificarás de pedras lavradas'' (Êxodo
20:25). Um altar construído com tal tipo de pedras é inadequado e não se
deve colocar sacrifí­cios sobre ele.


As normas deste preceito estão explicadas no terceiro capítulo


de Midot.

\section{Não subir ao altar por degraus}

Por esta proibição somos proibidos de subir ao Altar por degraus, para
não darmos passos largos para chegar até ele e para que subamos com um
pé seguindo próximo ao outro. Esta proibição está expressa em Suas
palavras, enaltecido seja Ele, ``E não subas por degraus sobre Meu altar''
(Êxodo 20:26),

267. No templo de Jerusalém (Beit Hamikdash).

a respeito das quais a Mekhiltá diz o seguinte: "O que as Escrituras
querem di­zer com 'Para que não seja descoberta tua nudez sobre ele'
(Ibid.)? Que ao su­bir ao altar não se deve dar passadas largas, mas sim
andar com um pé seguindo próximo ao outro".

O modelo do plano inclinado." e a maneira de
construí-lo estão ex­plicados no terceiro capítulo de Midot.

A punição por dar passadas largas e expor sua nudez no altar é o
açoitamento.

\section{Não apagar o fogo do altar}

Por esta proibição somos proibidos de apagar o fogo do Altar. Ela está
expressa em Suas palavras, enaltecido seja. Ele, "Fogo contínuo estará
ace­so sobre o altar; não se apagará" (Levítico 6:6), sobre as quais a
Sifrá diz: " 'Não se apagará': isto nos ensina que aquele que o apagar
estará transgredindo um preceito negativo". E aquele que violar esta
proibição, extinguindo nem que seja uma única brasa ardente de cima do
Altar, será punido com o açoitamento.


As normas deste preceito estão explicadas no décimo capítulo de


Zebahim.

\section{Não oferecer nenhum tipo de sacrifício sobre o altar de ouro}

Por esta proibição somos proibidos de oferecer todo e qualquer ti­po de
sacrifício sobre o Altar de Ouro no Santuário. Ela está expressa em Suas
palavras, enaltecido seja Ele, "Não oferecereis sobre ele incenso
estranho, nem holocausto, nem oblação; e libação não derramareis sobre
ele" (Êxodo 30:9).

Aquele que oferecer ou espalhar sobre ele sangue de qualquer outro tipo
de sacrifício que não seja aquele que lhe corresponde está sujeito ao
açoitamento.

\section{Não fazer óleo igual ao óleo de unção}

Por esta proibição somos proibidos de fazer óleo igual ao Óleo de Unção.
Ela está expressa em Suas palavras, enaltecido seja Ele, "E da mesma
composição não fareis outro como ele" (Exodo 30:32).

A punição pela contravenção voluntária desta proibição é a extin­ção,
como está nas Escrituras: "Todo homem que fizer semelhante a ele ..."
(Ibid., 33). Se o pecado for cometido involuntariamente, o transgressor
deve levar um sacrifício Determinado de Pecado.


As normas deste Preceito estão explicadas no primeiro capítulo de


Queretot.


\section{Não ungir ninguém a não ser os ``cohanim guedolim'' e os reis 
com o óleo de unção preparado por moisés}

Por esta proibição somos proibidos de ungir qualquer outra pessoa a não
ser os ``Cohanim Guedolim'' e os Reis com o Óleo de Unção que Moisés
preparou. Ela está expressa em Suas palavras, enaltecido seja Ele,
``Sobre carne de homem não será untado'' (Êxodo 30:32). Ficou claro que
todo aquele que deliberadamente se ungir com Óleo de Unção está sujeito
à extinção, como es­tá prescrito nas palavras: "E que usá-lo num
estranho, será exterminado de seu povo" (Ibid., 33); e que todo aquele
que o fizer involuntariamente deve levar um Sacrifício Determinado de
Pecado.


As normas deste preceito estão explicadas no início de Queretot.


\section{Não fazer incenso igual ao usado no santuário}

Por esta proibição somos proibidos de fazer incenso igual ao usado no
Santuário; quer dizer, fazer incenso usando os mesmos ingredientes, nas
mes­mas quantidades, com a intenção de queimá-lo. Esta proibição está
expressa em Suas palavras, enaltecido seja Ele, "Como a sua composição,
não fareis para vós" (Êxodo 30:37).

Somos claramente avisados que todo aquele que violar proposital­mente
esta proibição, fazendo incenso similar com a intenção de aspirá-lo,
está sujeito à extinção, pois Ele diz: "O homem que fizer igual a este
para o cheirar será banido do seu povo" (Ibid., 38); e aquele que o
fizer involuntariamente deve levar um Sacrifício Determinado de Pecado.


As normas deste preceito estão explicadas no início de Queretot.


\section{Não retirar as varas das argolas da arca}

Por esta proibição somos proibidos de remover as varas das argolas da
Arca. Ela está expressa em Suas palavras, enaltecido seja Ele, "E nas
argolas da Arca estarão as varas; não se tirarão dela" (Êxodo 25:15). A
punição pela con­travenção é o açoitamento.

No final de Macot, ao enumerar os transgressores que estão sujeitos ao
açoitamento, os Sábios perguntam: "Por que não incluir também aquele que
remover as varas da Arca, já que essa proibição está expressa nas
palavras 'Não se tirarão dela'?". Assim, foi deixado claro que este é um
preceito negativo, e a punição por sua transgressão é o açoitamento.

\section{Não desprender o peitoral do ``efod''}

Por esta proibição somos proibidos de remover o Peitoral do
``Efod''\textsuperscript{269}. Ela está expressa em Suas palavras,
enaltecido seja Ele, "E não se

269. Roupa usada pelo ``Cohen Gadol'' quando está ministrando o serviço no
Santuário.

desprenderá o peitoral de cima do efod" (Êxodo 28:28), e permanecerá
amar­rado a ele.

No final de Macot, ao enumerar os transgressores que estão sujeitos ao
açoitamento, os Sábios perguntam: "Por que não incluir também aquele que
desprende o Peitoral, já que a proibição está expressa em Suas palavras
'E não se desprenderá o peitoral de cima do efod'?". Assim, fica claro
que desprendê-lo é punido com o açoitamento.

\section{Não rasgar a orla do manto do ``cohen gadol''}

Por esta proibição somos proibidos de rasgar a borda do manto do "Cohen
Gadol"; a orla deve estar inteira e intacta. Esta proibição está
expressa em Suas palavras, enaltecido seja Ele, "Como abertura de malha
será debruada, para que não se rasgue" (Êxodo 28:32). Cortar o manto com
tesoura ou similar será punido com o açoitamento.

\section{Não oferecer nenhum sacrifício fora do campo do santuário}

Por esta proibição somos proibidos de oferecer qualquer sacrifício fora
do Campo do Santuário\textsuperscript{270}, o que é chamado de
``sacrificar fora''. Esta proibição está expressa em Suas palavras,
enaltecido seja Ele, "Guarda-te de ofe­receres teus holocaustos em todo
o lugar que vires" (Deuteronômio 12:13).

O Sifrei diz: "Isto me fala apenas dos Holocaustos; como saber quanto
aos outros sacrifícios? Pelas palavras das Escrituras 'E ali farás tudo
o que eu te ordeno'. Contudo, eu poderia dizer que enquanto no caso do
Holocausto há um preceito positivo\textsuperscript{27}' e um negativo,
no caso de todas as outras ofertas há apenas um preceito positivo. Por
isso as Escrituras dizem: 'Ali oferecerás os teus holocaustos'. Por que
o Holocausto está destacado, se ele está incluído no enunciado geral?
Para permitir-lhe deduzir por analogia que assim como no ca­so do
Holocausto --- que está especificamente mencionado --- há um preceito
positivo e um negativo, assim também em todos os outros casos, onde está
es­tipulado apenas um preceito positivo, há um preceito negativo
envolvido também".

Vou apresentar agora uma explicação deste texto --- embora ele seja
simples --- para que o assunto fique claramente entendido. No caso do
Holo­causto, as Escrituras proíbem expressamente oferecê-lo fora, pelas
Suas pala­vras, enaltecido seja Ele, "Guarda-te de ofereceres teus
holocaustos"; a seguir há uma ordem expressa para que ele seja oferecido
no Santuário, em Suas pala­vras, enaltecido seja Ele, "Ali oferecerás os
teus holocaustos", que são um pre­ceito positivo para oferecer o
Holocausto ``No lugar que escolher o Eterno'' (Ibid.). Mas, com relação às
outras ofertas consagradas, há apenas o preceito


\begin{enumerate}
\def\labelenumi{\arabic{enumi}.}
\setcounter{enumi}{269}
\item
 
 Ver o preceito negativo 77.
 
\item
 
 Ver o preceito positivo 84.
 
\end{enumerate}



positivo para que sejam oferecidas no Santuário, expresso em Suas
palavras ``Ali farás tudo o que eu te ordeno'' (Ibid.,14). Entretanto as
palavras ``Ali farás etc'' implicam que não devemos fazê-lo fora, e é um
princípio aceito entre nós que um preceito negativo derivado de um
preceito positivo tem a força de um pre­ceito positivo. Assim sendo, as
palavras citadas acima: "Contudo eu poderia dizer ... que no caso de
todas as outras ofertas há apenas um preceito positivo" devem ser
compreendidas da seguinte forma: aquele que oferecer qualquer
sa­crifício fora transgride apenas um preceito negativo derivado de um
preceito positivo e por isso as Escrituras dizem: "Ali oferecerás os
teus holocaustos", para permitir-nos argumentar, por analogia, que assim
como o oferecimento do Holocausto fora infringe um preceito negativo, o
mesmo acontece quando se leva qualquer uma das outras ofertas.

A violação voluntária desta proibição é punida com a extinção; aquele
que a violar involuntariamente deve levar um Sacrifício Determinado de
Peca­do. O castigo de extinção por ``sacrificar fora'' está estabelecido
na parte cha­mada ``Aharé Mot''\textsuperscript{273} das Escrituras: "Que
oferecer holocausto ou sacrifício, e à porta da tenda da assinação
trouxer, para oferecê-los ao Eterno, será banido aquele homem de seu
povo" (Levítico 17:8-9), a respeito das quais diz a Sifrá: "Será banido
aquele homem de seu povo': já ouvimos a penalidade; de onde vem a
proibição? Das palavras das Escrituras 'Guarda-te de ofereceres teus
ho­locaustos' ". Ou nas palavras da Guemará de Zebahim: "A penalidade
foi enun­ciada, assim como a advertência. A penalidade: 'À porta da
tenda da assinação trouxer ... será banido aquele homem de seu povo'; a
advertência: 'Guarda-te de ofereceres etc' "

As normas deste preceito estão explicadas no décimo terceiro capí­tulo
de Zebahim.

\section{Não degolar nenhum dos sacrifícios sagrados fora do campo do santuário}

Por esta proibição somos proibidos de degolar qualquer um dos
sa­crifícios Sagrados fora. Isso é chamado de ``degolar fora'' e na
enumeração de todas as transgressões que acarretam a extinção, feita no
início de Queretot, ``degolar fora'' e ``sacrificar fora'' são contadas
separadamente.

O princípio de que aquele que degola do lado de fora fica sujeito à
extinção a partir do momento do degolamento, mesmo se ele não fizer a
oferen­da depois, está expressa na Torah em Suas palavras, enaltecido
seja Ele, "Que de­golar boi ou cordeiro ou cabra, no acampamento, ou
degolar fora do acampa­mento, e não os trouxer à porta da tenda da
assinação para o oferecer como sa­crifício ao Eterno, derramador de
sangue será considerado aquele homem; san­gue derramou, e será banido
aquele homem dentre seu povo" (Levítico 17:3-4). Contudo, a proibição de
degolar os sacrifícios sagrados fora não está explicita­mente enunciada,
mas se deduz do princípio de que não se prescreve uma puni­ção a menos
que uma proibição a preceda, de acordo com os Fundamentos que
apresentamos na Introdução a estes preceitos. A Guemará de Zebahim diz:
'Aquele que degolar e oferecer fora é culpado por degolar e por
oferecer. Isso está corre­to no que se refere ao sacrifício, a respeito
do qual as Escrituras prescrevem a punição 
e enunciam a proibição, sendo a punição: 'Será banido' (Ibid., 9),
e a proi­bição: 'Guarda-te (hishamer) de ofereceres' (Deuteronômio
12:13), a qual deve ser compreendida à luz das palavras de Rabi Abin, em
nome de Rabi Ilai, de que toda vez que as Escrituras dissérem
'guarda-te' (hishamer), ou 'para que não' (pen), ou 'não' (a1), há um
preceito negativo. Mas no caso do degolamento, em­bora as Escrituras
prescrevam reconhecidamente o castigo nas palavras 'E não os trouxer à
porta da tenda da assinação ... será banido aquele homem', onde é que
encontramos a proibição?" Depois de uma longa discussão a questão foi
assim resolvida: "Ao dizer 'Ali oferecerás (taaleh)' ... e 'ali farás
(taaseh)' (Deute­ronômio 12:14), as Escrituras nos permitem argumentar o
seguinte, por analo­gia de 'oferecer' a 'fazer': assim como no caso de
'oferecer', também no caso de 'fazer' estão implicados tanto o castigo
quanto a proibição".

``Ali oferecerás'' e ``Ali farás'' se referem às Suas palavras, enaltecido
seja Ele, ``Ali oferecerás os teus holocaustos'', que significam
sacrificar, i.e., quei­mar no fogo, e as palavras "Ali farás tudo o que
eu te ordeno", que incluem ambos queimar e degolar, uma vez que Ele nos
deu ordens quanto ao degola-mento também.

Você deve saber que aquele que degolar fora involuntariamente tam­bém
tem a obrigação de levar um Sacrifício Determinado de Pecado.

Também é importante para você saber que todo aquele que ofere­cer
atualmente um sacrifício sagrado fora está sujeito à extinção. Os sábios
di­zem claramente: "Se alguém sacrificar atualmente, Rabi Yohanan diz
que ele é culpado", e essa é a lei, pois os sacrifícios ainda são
válidos, de acordo com o princípio bem estabelecido de que "Podemos
oferecer sacrifícios embora não haja Santuário".

As normas deste preceito também estão explicadas no décimo ter­ceiro
capítulo de Zebahim.

\section{Não destinar animais defeituosos para serem oferecidos sobre o altar}

Por esta proibição somos proibidos de dedicar animais com defeito sobre
o Altar. Ela está expressa em Suas palavras, enaltecido seja Ele, "Todo
aquele que tiver defeito não oferecereis" (Levítico 22:20), a respeito
das quais a Sifrá diz: " 'Todo aquele que tiver defeito não
oferecereis': isto se refere à consagração".

\section{Não degolar animais defeituosos para oferecê-los como sacrifício}

Por esta proibição somos proibidos de degolar: animais com defeito como
Sacrifício. Ela está expressa em Suas palavras, enaltecido seja Ele,
relativas a animais com defeito ``Não oferecereis ao Eterno'' (Levítico
22:22), a respeito das quais diz a Sifrá: " 'Não oferecereis ao Eterno':
isto se refere ao degolamento".


\section{Não aspergir o sangue de animais defeituosos sobre o altar}

Por esta proibição somos proibidos de aspergir sobre o Altar o san­gue
de animais com defeito. Ela está expressa em Suas palavras, enaltecido
seja Ele, também relativas aos animais defeituosos, "Não oferecereis ao
Eterno" (Le­vítico 22:24), as quais a Tradição interpreta como uma
proibição quanto a es­palhar o sangue de um animal com defeito. Essa é a
opinião do ``Taná kamá'', e essa é a lei. Contudo, Rabi Yossi ben Yehudá
diz que ela proíbe apenas o re­cebimento do sangue. Essa opinião,
aparece na Sifrá: " 'Não oferecereis ao Eter­no": isto se refere a
receber o sangue".

A Guemará de Temurá diz: "Do ponto de vista do .Fana
kamá', por que as Escrituras dizem 'Não oferecereis'? Isso é necessário
para o caso de se aspergir o sangue de um animal defeituoso. Mas é das
palavras 'sobre o altar' que deduzimos essa proibição?", referindo-se às
palavras das Escrituras "E ofertas queimadas, não dareis destas coisas
sobre o altar, ao Eterno" (Levítico 22:22), que nos ensinam que tudo
aquilo que for oferecido sobre o altar não deve pro­vir de animais
defeituosos. A resposta é : "É assim que foi escrita a
Torah"\textsuperscript{274}. Quer dizer, a proibição "E ofertas
queimadas, não dareis destas coisas" se refe­re apenas à queima das
partes de sacrifícios e nada pode ser deduzido do uso da expressão
``sobre o altar'' porque o versículo não poderia ter sido escrito de outra
forma, pois como poderia ele dizer simplesmente "E ofertas queima­das,
não dareis destas coisas"? Ele ficaria incompleto!

Por tudo o que precede, aparece claramente que Suas palavras "Não
oferecereis ao Eterno" são a proibição de espalhar o sangue.

\section{Não queimar as partes de sacrifício de um animal defeituoso sobre o altar}

Por esta proibição somos proibidos de queimar as partes de sacrifí­cio
de um animal defeituoso. Éla está expressa em Suas palavras, enaltecido
se­ja Ele, "E ofertas queimadas, não dareis destas coisas sobre o altar"
(Levítico 22:22), sobre as quais diz a Sifrá: " 'E ofertas queimadas,
não dareis destas coi­sas' se refere à gordura. 'Não dareis destas
coisas' significa apenas a queima de todas; como fico sabendo que se
trata de uma parte qualquer delas? Pelas pala­vras 'destas coisas', que
significam nem mesmo qualquer parte delas".

Assim, foi deixado claro que aquele que sacrificar um animal defei­tuoso
transgride quatro proibiçõçs, se contarmos a queima das partes de
sacri­fício como apenas um preceito; mas se o contarmos como dois, como
fez o ``Taná''\textsuperscript{275}, haverá cinco preceitos transgredidos,
pois o ``Taná'' conta "qual­quer uma" das partes de sacrifício como um
preceito, e ``todas'' elas como ou­tro, sustentando que ``destas coisas''
significa qualquer parte delas, embora ha­ja apenas uma proibição.
Parece que o ``Taná'' é de opiniãd que a violação de um


\begin{enumerate}
\def\labelenumi{\arabic{enumi}.}
\setcounter{enumi}{273}
\item
 
 Sendo que, de acordo com a gramática ou com o assunto, é necessário
 escrever dessa forma.
 
\item
 
 O sábio da Sifrá.
 
\end{enumerate}



``lav shebikhlalut'' é punível com o açoitamento! Por isso ele diz na
Sifrá: "Aquele que destinar para o Altar um animal defeituoso transgride
cinco proibições: a de destinar, degolar, aspergir o sangue, queimar as
partes de sacrifício, e quei­mar qualquer uma dessas partes".

A Guemará de Temurá diz: "De acordo com Abayé, se alguém ofe­recer sobre
o altar os membros de animais defeituosos será punido tanto pela
violação da proibição de queimar o animal inteiro como pela violação da
proi­bição de queimar uma parte qualquer dele. Rabá diz que não há pena
de açoita­mento pela violação de um `lav
shebikhlalut'\textsuperscript{276}, mas contra ele foi citada a
afir­mação 'Aquele que destinar para o Altar um animal defeituoso
transgride cinco proibições, etc', que demonstra que há uma pena de
açoitamento pela violação de um 'lav shebikhlalut'. Rabá foi, por
conseguinte, contestado".

Foi assim deixado claro que quem diz que se transgride cinco proi­bições
tem essa opinião apenas porque ele sustenta que a violação de um "lav
shebikhlalut" é punível com o açoitamento, e conseqüentemente conta como
duas a proibição referente ao todo e a qualquer parte. É sabido que
Abayé ado­ta esse ponto de vista em todos os lugares, como explicamos no
nono dos Fun­damentos expostos no começo deste trabalho. Mas de acordo
com Rabá, que diz que não há pena de açoitamento pela violação de um
``lav shebikhlalut'', há apenas um açoitamento pela queima, como dissemos.
Nós deixamos claro que este é o ponto de vista correto, como explicado
na Guemará de Sanhedrin, e como enfatizado por nós no nono Fundamento;
conseqüentemente, há ape­nas quatro proibições, como foi enunciado nas
Escrituras, e aquele que dedicar e sacrificar um animal defeituoso está
sujeito a quatro açoitamentos pela viola­ção dessas quatro proibições,
como explicamos.

Todas essas proibições se aplicam a animais permanentemente
de­feituosos, tais como os mencionados nas Escrituras nas seguintes
palavras: "Que tenham membros ou maior que o outro, ... e de testículos
machucados, ou moí­dos ou despendidos, ou cortados" (Levítico 22:23-24),
pois todos esses são de­feitos permanentes.

Todos os defeitos de animais, tanto permanentes como temporários, estão
detalhados no sexto capítulo de Bekhorot; e as normas das quatro
proibi­ções que tratam especificamente da oferta de animais defeituosos
estão expli­cadas em diversos trechos dos Tratados Zebahim e Temurá.


\section{Não sacrificar um animal com um defeito temporário}


Por esta proibição somos proibidos de sacrificar um animal com um
defeito passageiro. Ela está expressa em Suas palavras no Deuteronômio
"Não sacrificarás ao Eterno, teu Deus, boi, ou cordeiro que tenha
defeito etc". (Deu­terónômio 17:1), que são explicadas pelo Sifrei como
referindo-se a um animal com um defeito passageiro.

Também neste caso o açoitamento é o castigo por desobedecer a proi­bição
de sacrificar.


\section{Não oferecer sacrifícios defeituosos de um gentio}

Por esta proibição somos proibidos de oferecer sacrifícios defeituo­sos
de um gentio. Não devemos dizer: "Como ele é um gentio, um sacrifício
imperfeito pode ser oferecido em seu favor". Esta proibição está
expressa em Suas palavras, enaltecido seja Ele, "E da mão do estrangeiro
não oferecereis ne­nhuma dessas coisas" (Levítico 22:25).

A punição por sacrificar transgredindo esta proibição também é o
açoitamento.

\section{Não fazer com que uma oferta se torne defeituosa}

Por esta proibição somos proibidos de fazer com que uma oferta se torne
defeituosa. Isso é chamado de ``deformar ofertas consagradas'' e é
puní­vel pelo açoitamento, desde que seja feito quando o Santuário
estiver de pé e caso a oferta seja aceitável, como explicado na Guemará
de Abodá Zará. Essa proibição está expressa em Suas palavras relativas
aos sacrifícios "Estes deve­rão ser sem defeito" (Levítico 22:21), sobre
as quais a Sifrá diz: " 'Estes deve­rão ser sem defeito', ou seja, não
os tornem defeituosos".

\section{Não oferecer fermento ou mel sobre o altar}

Por esta proibição somos proibidos de oferecer fermento ou mel so­bre o
Altar. Ela está expressa em Suas palavras, enaltecido seja Ele, "Porque
não fareis queimar fermento algum ou mel algum como oferta queimada ao
Eter­no" (Levítico 2:11), e exposta sob outra forma em Suas palavras
"Nenhuma obla­ção que oferecerdes ao Eterno será preparada com fermento"
(Ibid.).

Já explicamos no Nono Fundamento que oferecer ambos fermento e mel é
punível com um açoitamento apenas, e não com dois, porque essa
proi­bição é um ``lav shebikhlalut'', como explicamos ali, e ficou
claramente estabe­lecido que a violação de um ``lav shebikhlalut''
acarreta açoitamento apenas uma vez. Assim, por exemplo, aquele que
oferecer mel é punido uma vez com açoitamento, da mesma forma que aquele
que oferecer fermento, ou fermento e mel juntos.

99 NÃO OFERECER UM SACRIFÍCIO SEM SAL

Por esta proibição somos proibidos de oferecer um sacrifício sem sal.
Ela está expressa em Suas palavras, enaltecido seja Ele, "Não deixarás
faltar o sal do pacto de teu Deus, em tua oblação" (Levítico 2:13). Como
somos proi­bidos de deixar faltar o sal, segue-se que é proibido
oferecer qualquer sacrifício sem sal, e, que todo aquele que oferecer um
sacrifício ou oblação sem sal está sujeito ao açoitamento.


As normas deste preceito estão explicadas no sétimo capítulo de


Zebahim.


\section{Não oferecer no altar o salário de uma rameira ou o preço de um cão}

Por esta proibição somos proibidos de oferecer no Altar o salário de uma
rameira ou o preço de um cão. Ela está expressa em Suas palavras,
enal­tecido seja Ele, "Não trarás salário de rameira, nem preço de cão à
casa do Eter­no, teu Deus" (Deuteronômio 23:19).

As normas deste preceito estão explicadas no sexto capítulo do Tra­tado
Temurá. Todo aquele que oferecer uma destas coisas, embora seu
sacrifí­cio seja desqualificado, está sujeito ao açoitamento, de acordo
com a lei relati­va ao sacrifício de um animal com defeito.

\section{Não degolar a mãe e seu filhote no mesmo dia}

Por esta proibição somos proibidos de degolar a mãe e seu filhote no
mesmo dia, seja para serem usados como sacrifício ou como simples
alimento. Ela está expressa em Suas palavras, enaltecido seja Ele, "A
ela e a sua cria não degolareis no mesmo dia" (Levítico 22:28).

A punição por degolar transgredindo esta proibição é o açoitamento. As
normas deste preceito estão detalhadamente explicadas no quin­to
capítulo de Hulin.


\section{O colocar azeite de oliva sobre a oblação de um pecador}
% NOTA: \textbf{102}N\_ \_ .4+

Por esta proibição somos proibidos de colocar azeite de oliva sobre a
oblação de um pecador\textsuperscript{277}. Ela está expressa em Suas
palavras, enaltecido se­ja Ele, ``Não porá sobre ela azeite'' (Levítico
5:11). Colocar azeite sobre ela se pune com o açoitamento.


\section{Não levar incenso junto com a oblação de um pecador}


Por esta proibição somos proibidos de levar incenso junto com a oblação
de um pecador. Ela está expressa em Suas palavras, enaltecido seja Ele,
``Nem porá sobre ela incenso'' (Levítico 5:11). Colocar incenso sobre ela
se pu­ne com o açoitamento.

A Mishná diz: "Um homem torna-se culpado por causa do óleo por si só e
por causa do incenso por si só", porque estes são sem dúvida alguma dois
preceitos negativos independentes.

As normas deste preceito, relativo à oblação de um pecador, estão
explicadas no quinto capítulo de Menahot.

277. Ver o preceito positivo 72.


\section{Não misturar azeite de oliva com a oblação de uma mulher suspeita de adultério}

Por esta proibição somos proibidos de misturar azeite de oliva com a
oblação de uma mulher suspeita de adultério. Ela está expressa em Suas
pala­vras, enaltecido seja Ele, ``Não derramará sobre ela azeite''
(Números 5:15).

A punição por oferecer a oblação com óleo é o açoitamento.

\section{Não colocar incenso sobre a oblação de uma mulher suspeita de adultério}

Por esta proibição somos proibidos de colocar incenso sobre a obla­ção
de uma mulher suspeita de adultério. Ela está expressa em Suas palavras,
enaltecido seja Ele, ``Nem porá sobre ela incenso'' (Números 5:15), a
respeito das quais o Sifrei diz: "Isto significa que aquele que puser
incenso em tal tipo de oferenda transgredirá um preceito negativo, pois
o que se aplica ao azeite se aplica também ao incenso". Portanto a
transgressão da proibição se pune também pelo açoitamento.

A Mekhiltá diz: " 'Não derramará sobre ela azeite, nem porá sobre ela
incenso': isto significa que há duas proibições diferentes".

\section{Não trocar um animal que tenha sido consagrado como oferenda}

Por esta proibição somos proibidos de trocar um animal que tenha sido
consagrado como oferenda. Isso é chamado de ``substituição''. A proibi­ção
está expressa em Suas palavras, enaltecido seja Ele, "Não o mudará e não
o trocará" (Levítico 27:10)\textsuperscript{278}.

Há uma proibição específica no caso do dízimo\textsuperscript{279}. A
razão para is­so está exposta na Sifrá: "O dízimo já está incluído no
gera1\textsuperscript{280}: então por que ele foi especificamente
destacado? Para permitir-nos argumentar por analogia que, da mesma forma
que o dízimo --- cuja substituição está proibida --- é uma das coisas
sagradas do altar, assim também todas as coisas sagradas --- cuja
subs­tituição está proibida por Suas palavras São o mudará' --- são
apenas as coisas sagradas do altar", e sua substituição se pune com o
açoitamento.

As normas deste preceito estão explicadas no Tratado Temurá.


\begin{enumerate}
\def\labelenumi{\arabic{enumi}.}
\setcounter{enumi}{277}
\item
 
 Ver o preceito positivo 87.
 
\item
 
 Ver o preceito positivo 78.
 
\item
 
 I.e., na proibição geral ``Não o mudará e não o trocará''.
 
\end{enumerate}



\section{Não fazer qualquer trabalho com um animal consagrado}

Por esta proibição somos proibidos de fazer qualquer trabalho com um
sacrifício. Ela está expressa em Suas palavras, enaltecido seja Ele,
"Não fa­rás nenhum serviço com o primogênito de teu boi" (Deuteronômio
15:19) e nós deduzimos a proibição de trabalhar com qualquer outro
animal consagra­do ao sacrifício a partir desta, relativa ao
primogênito.

Está explicado no final do Tratado Makoth que aquele que trabalhar com
um animal dedicado ao sacrifício é punido com o açoitamento.

\section{Não tosquiar um animal consagrado}

Por esta proibição somos proibidos de tosquiar um sacrifício. Ela es­tá
expressa em Suas palavras, enaltecido seja Ele, "Nem tosquiarás o
primogê­nito do teu rebanho" (Deuteronômio 15:19) e nós deduzimos a
proibição de tosquiar todos os outros animais consagrados ao sacrifício
a partir desta, relati­va ao primogênito.

As normas destes dois preceitos, relativos a tosquiar e a trabalhar com
sacrifícios, estão explicadas no Tratado Bekhorot. Todo aquele que
tosquiar qualquer sacrifício também será punido com o açoitamento.

\section{Não degolar o sacrifício de ``pessah'' enquanto tivermos pão levedado em nosso poder}

Por esta proibição somos proibidos de degolar o cordeiro de "Pes­sah"
enquanto tivermos pão levedado. Ela está expressa em Suas palavras,
enal­tecido seja Ele, "Não sacrifiques, tendo pão levedado, sangue de
Meu sacrifí­cio" (Êxodo 23:18) e aparece também em outro
trecho\textsuperscript{286}. Significa que a par­tir do momento do
degolamento da oferenda de ``Pessah'', que é a tarde\textsuperscript{287},
não deverá haver pão levedado em possessão daquele que degolar a
oferenda, daquele que aspergir seu sangue, daquele que queimar suas
partes de sacrifício, nem daqueles que fazem parte da
companhia\textsuperscript{288}; e todo aquele que estiver de posse de
pão levedado naquela ocasião será punido com o açoitamento.

A Mekhiltá diz: " 'Não sacrifiques, tendo pão levedado, sangue de Meu
sacrifício': isto é, não deverás degolar o cordeiro de ``Pessah'' enquanto
ainda houver pão levedado".


As normas deste preceito estão explicadas no quinto capítulo de
Pessahim.

\begin{enumerate}
\def\labelenumi{\arabic{enumi}.}
\setcounter{enumi}{285}
\item
 
 Êxodo 34:25.
 
\item
 
 No 14? dia de Nissan.
 
\item
 
 O grupo que se reuniu para juntos comerem a carne do sacrifício de
 ``Pessah''.
 
\end{enumerate}



\section{Não deixar as partes do sacrifício da oferenda de ``pessah'' 
de um dia para o outro}

Por esta proibição somos proibidos de deixar de oferecer as partes do
sacrifício da oferenda de ``Pessah'' até que elas deixem de ser adequadas
ao sacrifício e se transformem em ``notar''. Esta proibição está expressa
em Suas palavras, "E não ficará sebo do Meu sacrifício, até a manhã"
(Exodo 23:18), sobre as quais a Mekhiltá diz: "O objetivo das Escrituras
é explicar-nos que os pedaços de gordura se tornam impróprios ao
sacrifício se ficarem de um dia para o outro no chão".

Está proibição está repetida de uma outra forma em Suas palavras "E não
ficará para a manhã o sacrifício do cordeiro de 'Pessah' " (Ibid.
34:25).

\section{Não deixar ficar nenhuma parte da carne da oferenda de ``pessah'' 
até a manhã seguinte}

Por esta proibição somos proibidos de deixar ficar alguma parte da carne
da oferenda de ``Pessah'' até a manhã do décimo quinto dia. Ela está
ex­pressa em Suas palavras ``E não fareis sobrar nada dele até a manhã''
(Êxodo 12:10).

Nós já explicamos\textsuperscript{289} que este preceito negativo é um
dos que es­tão justapostos a um preceito positivo, o qual está expresso
em Suas palavras "E a sobra dele, pela manhã a queimareis no fogo"
(Ibid.). A Mekhiltá diz: " 'E a sobra dele': as Escrituras tencionam
acrescentar um preceito positivo ao pre­ceito negativo, para assim
indicar-nos que ele não está sujeito ao açoitamento.

\section{Não deixar sobrar carne do sacrifício do festival do décimo
quarto do ``nissan'' ate o terceiro dia}

Por esta proibição somos proibidos de deixar ficar até o terceiro dia
qualquer parte da carne do sacrifício de Festival que for levado no
décimo quarto dia (como está explicado no sexto capítulo de Pessahim).
Ela deve ser comida em dois dias. Esta proibição está expressa em Suas
palavras "E não ficará da carne do cordeiro de 'Pessah' que sacrificares
no primeiro dia à tarde, até pela manhã" (Deuteronômio 16:4), cuja
interpretação tradicional é a seguinte: " 'E não ficará da carne... até
pela manhã': as Escrituras falam aqui do sacrifício de Festival que é
levado em suplemento ao sacrifício de 'Pessah', e determinam que ele
pode ser comido durante dois dias. Poder-se-ia pensar que ele pode ser
comido apenas num só dia, mas quando as Escrituras dizem 'até pela
manhã' isso significa até a manhã do segundo dia do Festival". E a
respeito deste sacri­fício de Festival que Ele diz, enaltecido seja Ele,
"E sacrificarás 'Pessah' ao Eter­no, teu Deus, do rebanho (tson) e do
gado (bakar)" (Ibid., 2).

Qualquer porção de carne deste sacrifício do Festival do décimo quar­to
que tiver ficado até o terceiro dia deverá ser queimada porque ela se
trans­forma em ``notar'', e conseqüentemente não há penalidade de
açoitamento por isso.

As normas deste preceito, relativo ao sacrifício do Festival do déci­mo
quarto dia, estão explicadas em vários trechos dos Tratados Pessahim e
Haguigá.

\section{Não deixar sobrar carne do segundo sacrifício de ``pessah'' até a manhã seguinte}


Por esta proibição somos proibidos de deixar sobrar qualquer parte da
carne do segundo sacrifício de ``Pessah''." até pela
manhã. Ela está expressa em Suas palavras "Não deixarás nada dele até
pela manhã" (Números 9:12).

Assim como a anterior, esta também é uma proibição justaposta a um
preceito positivo\textsuperscript{291}.

\section{Não deixar sobrar carne do sacrifício de graças até a manhã seguinte}

Por esta proibição somos proibidos de deixar sobrar qualquer parte de um
Sacrifício de Graças até pela manhã. Ela está expressa em Suas palavras,
relativas ao Sacrifício de Graças: "Não deixareis ficar dele até pela
manhã" (Le­vítico 22:30).

Por esta proibição chegamos à conclusão de que tudo o que ficar, de
qualquer outro sacrifício, além do tempo determinado para o seu consumo
se transformará em ``notar'' e deverá ser queimado, uma vez que a
proibição está justaposta a um preceito positivo. Queimá-lo é um
preceito positivo, co­mo explicamos ao tratar do preceito positivo 91.

\section{Não quebrar nenhum osso do sacrifício de ``pessah''}

Por esta proibição somos proibidos de quebrar qualquer um dos os­sos do
sacrifício de ``Pessah''. Ela está expressa em Suas palavras "Nem o osso
quebrarão" (Êxodo 12:46), e sua transgressão é punida com o açoitamento,
co­mo está explicitamente exposto no Talmud: "Aquele que quebrar um osso
de um sacrifício de Tessah' puro será punido com açoitamento".


\begin{enumerate}
\def\labelenumi{\arabic{enumi}.}
\setcounter{enumi}{289}
\item
 
 Ver o preceito positivo 58.
 
\item
 
 Ver o preceito positivo 91.
 
\end{enumerate}


PRECEITOS NEGATIVOS 245

\section{Não quebrar nenhum osso do segundo sacrifício de ``pessah''}

Por esta proibição também somos proibidos de quebrar qualquer um dos
ossos do segundo sacrifício de ``Pessah''\textsuperscript{292}. Ela está
expressa em Suas pa­lavras, enaltecido seja Ele, "E osso algum não
quebrará dele" (Números 9:12), e sua contravenção também é punida com o
açoitamento.

A Guemará de Pessahim diz: "Uma vez que está dito, em relação ao segundo
sacrifício de `Pessah', 'E osso algum não quebrará dele', o que é
des­necessário, já que também está dito 'Segundo todo o estatuto de
``Pessah'', o fará' (Ibid.), devo concluir que isso se refere a qualquer
tipo de osso, quer ele contenha tutano ou não."

As normas relativas à quebra de um osso estão explicadas no sétimo
capítulo de Pessahim.

\section{Não retirar o sacrifício de ``pessah'' do lugar onde ele é comido}

Por esta proibição somos proibidos de retirar qualquer porção da car­ne
do sacrifício de `Pessah" do lugar onde a companhia\textsuperscript{293}
se reúne para comê-lo. Esta proibição está expressa em Suas palavras,
enaltecido seja Ele, "Não leva­rão para fora da casa a carne" (Êxodo
12:46), sobre as quais a Mekhiltá diz: " 'Pa­ra fora': isto é, para fora
do lugar onde ela deve ser comida". Qualquer porção da carne que for
retirada é classificada como ``terefá'' e não pode ser comida.

A Guemará de Pessahim diz: "Se carne do sacrifício de ``Pessah'' for
levada de uma companhia\textsuperscript{293} para outra, embora isso
infrinja um preceito ne­gativo, a carne permanecerá pura, mas todo
aquele que a comer estará trans­gredindo um preceito negativo". Também
está dito, no mesmo trecho, que "Aquele que levou carne de sacrifício de
``Pessah'' de uma companhia\textsuperscript{293} para outra não é culpado
a menos que ele a deixe lá, pois a expressão 'levarão' tem o mesmo
significado aqui que no caso do Shabat". Mas se a deixar lá, então ele
estará sujeito ao açoitamento.


As normas deste preceito estão explicadas no sétimo capítulo de


Pessahim..

\section{Não cozer as sobras de uma oblação de cereal com levedo}

Por esta proibição somos proibidos de cozer as sobras de uma obla­ção de
cereal com levedo. Ela está expressa em Suas palavras, enaltecido seja


\begin{enumerate}
\def\labelenumi{\arabic{enumi}.}
\setcounter{enumi}{291}
\item
 
 Ver o preceito positivo 57.
 
\item
 
 O grupo que se reuniu para juntos comerem a carne do sacrifício de
 ``Pessah''.
 
\end{enumerate}


Ele, "Não será cozido levedado, isto é igual à porção das minhas ofertas
quei­madas, que lhe tenho dado" (Levítico 6:10), o que equivale a dizer
que a por­ção deles, que é a sobra da oblação, não deve ser cozida com
levedo. Todo aquele que a cozer com levedo estará sujeito ao
açoitamento, como a Mishná enuncia claramente: "Fica-se sujeito ao
açoitamento".


As normas deste preceito estão explicadas no quinto capítulo de


Menahot.

\section{Não comer o sacrifício de ``pessah'' cozido nem cru}

Por esta proibição somos proibidos de comer o sacrifício de "Pes­sah"
cozido ou cru: ele deve ser assado. Esta proibição está expressa em Suas
palavras, enaltecido seja Ele, "Não comais dela mal passada no fogo nem
cozi­da na água" (Êxodo 12:9).

Já expliquei no Nono Fundamento deste trabalho que a contraven­ção a
esta proibição é punida com o açoitamento.

\section{Não permitir que um ``guer toshab'' coma do sacrifício de ``pessah''}

Por esta proibição somos proibidos de permitir que um "guer
tos­hab"\textsuperscript{294} coma do sacrifício de ``Pessah''. Ela está
expressa em Suas palavras, enaltecido seja Ele, "O forasteiro (toshab) e
o mercenário estrangeiro não co­merão dele" (Êxodo 12:45).

\section{Uma pessoa incircuncisa não deve comer do sacrifício de ``pessah''}

Por esta proibição a pessoa incircuncisa fica avisada para não comer do
sacrifício de ``Pessah''. Ela está expressa em Suas palavras, "E nenhum
incir­cunciso comerá dele" (Êxodo 12:48). Toda pessoa incircuncisa que
comer de­le será punida com o açoitamento.

\section{Não permitir que um israelita apóstata coma do sacrifício de ``pessah''}

Por esta proibição somos proibidos de permitir que um israelita
após­tata coma do sacrifício de ``Pessah''. Ela está expressa em Suas
palavras, enalte­cido seja Ele, ``Nenhum estrangeiro comerá dele'' (Êxodo
12:43), que o Targum

(Onkelos), de acordo com a Tradição, traduz da seguinte forma: "Nenhum
is­raelita apóstata". A Mekhiltá também diz: " `Nenhum estrangeiro' se
refere a um israelita apóstata que adora ídolos".

\section{Uma pessoa impura não deve comer comida santificada}

Por esta proibição uma pessoa impura fica proibida de comer co­mida
santificada. Ela está expressa em Suas palavras, relativas a uma mulher
depois do parto, ``Em nenhuma santidade tocará'' (Levítico 12:4), sobre as
quais a Sifrá diz: " `Em nenhuma santidade tocará, e no santuário não
entrará': assim como entrar no Santuário em estado de impureza acarreta
o castigo de extinção, o mesmo acontece se se comer carne santificada em
estado de impureza".

Esta aplicação da proibição ``não tocará'' (em nenhuma santidade) quanto a
``comer'' voluntariamente está baseada no princípio estabelecido em Macot
sobre a interpretação de Suas palavras ``Em nenhuma santidade tocará''. O
que a Guemará de Macot diz é o seguinte: "Com relação ao impuro que
co­meu carnè santificada, eu admito que a penalidade está expressa em 'E
a alma que comer carne de sacrifício de oferta de paz... e tiver a sua
impureza sobre si, será banida de seu povo" (Ibid., 7:20). Mas onde
encontramos a proibição necessária? Ela se encontra no texto: 'Em
nenhuma santidade tocará' ".

A Guemerá diz ainda: " `Em nenhuma santidade tocará' é a proibi­ção de
comer. Você diz que essa é uma proibição de 'comer'? Será que ela não é
apenas uma proibição de `tocar'? Não. As Escrituras dizem: `Em nenhuma
san­tidade tocará, e no santuário não entrará', tornando assim
equivalentes a 'santi­dade' e o 'santuário'. Assim como o Santuário
acarreta a perda de uma alma (extinção), assim também todas as
santidades acarretam como penalidade a perda de uma alma. Se você disser
que é o 'tocar', você conhece algum exemplo em que o 'tocar' acarrete a
perda de uma alma? Portanto, o significado deve ser `comer' ".

A razão para que o Misericordioso tenha usado a palavra ``tocar'' em
relação a ``comer'' é para mostrar-nos que o tocar é igualado ao comer.

Por essas declarações fica claro que comer carne santificada é puni­do
com a extinção, se a ofensa for cometida deliberadamente, mas se for
invo­luntariamente, o transgressor deve levar um Sacrifício de Maior ou
Menor Va­lor, como explicamos ao tratar do preceito positivo 72.

As normas deste preceito estão explicadas no décimo terceiro capí­tulo
de Zebahim.

\section{Não comer carne de sacrifícios consagrados que se tornaram impuros}

Por esta proibição somos proibidos de comer a carne de sacrifícios
consagrados que se tornaram impuros. Ela está expressa em Suas palavras,
enal­tecido seja Ele, "E a carne sagrada do sacrifício de paz que tocar
em tudo o que for impuro, não será comida" (Levítico 7:19). A penalidade
por comer o que esta proibição determina é o açoitamento. Na Tosseftá de
Zebahin está ex­plicado que uma pessoa pura que comer carne impura
recebe os quarenta açoites; 
e no segundo capítulo da Guemará de Pessahim está dito: "A impureza
da `pessoa' é punida com a extinção, mas a impureza da 'carne' é um
preceito negativo".

As normas deste preceito já foram explicadas no décimo terceiro
ca­pítulo do (Tratado) Zebahim.

\section{Não comer ``notar''}

Por esta proibição somos proibidos de comer ``notar'', quer dizer, carne
de sacrifícios que foi deixada além do tempo determinado para seu
consumo.

Na Torah não consta nenhuma proibição expressa quanto a comer ``notar'',
mas ela prescreve a pena de extinção para todo aquele que o faça
atra­vés de Suas palavras, enaltecido seja Ele, na parte de Kedoshim,
relativas ao Sacrifício de Paz, "E a sobra, até o terceiro dia, no fogo
será queimada. E se for comido no terceiro dia... e será banida aquela
alma de seu povo" (Levítico 19:6-8). Fica, assim, claro que se alguém o
fizer propositalmente, ele será puni­do com a extinção e se o fizer
involuntariamente, deverá levar um Sacrifício Determinado de Pecado. O
castigo está explicitamente enunciado nas Escritu­ras, mas a proibição
se deduz de Suas palavras, relativas à Consagração, "E o estranho não
comerá delas pois elas são santidade" (Êxodo 29:33), onde a pala­vra
``elas'' inclui qualquer parte de um sacrifício que se estrague e que,
como o ``notar'', não deva ser comida.

No Tratado de Meilá há o seguinte comentário sobre a afirmação na Mishná
de que ``pigul'' e ``notar'' não devem ser contados juntos, pois são duas
coisas diferentes: "O princípio se aplica apenas quanto a impurificar as
mãos, de acordo com a lei Rabínica, mas eles devem ser contados juntos
no que se refere a comer, pois a Mishná diz, em nome de Rabi Eliezer:
'Ele não comerá delas pois elas são santidade': um preceito negativo
proíbe comer todas as san­tidades que se tornarem impróprias". Como
``pigul'' e ``notar'' são ambos san­tidades que se tornaram impróprias, a
proibição de comer qualquer um deles está expressa em Suas palavras "Ele
não comerá delas pois elas são santidade". Já foi explicado que o
castigo por comer ``notar'' é a extinção.

\section{Não comer ``pigul''}

Por esta proibição somos proibidos de comer ``pigul''. ``Pigul'' sig­nifica
um sacrifício que se tornou impróprio porque, no momento em que foi
degolado ou oferecido, a pessoa que o ofertou teve intenções inadequadas
quan­to à sua finalidade, tendo pensado em comê-lo ou em só queimar as
partes que devem ser queimadas depois de expirado o prazo para fazê-lo,
como explica­mos claramente no segundo capítulo de Zebahim.

A proibição de comer ``pigul'' está expressa em Suas palavras "Ele não
comerá delas pois são santidade" (Êxodo 29:33), como explicamos ao
tra­tar do preceito precedente, e deduzimos o castigo de Suas palavras
na parte de Tzav, relativas ao ``pigul'', "E, se na hora de sacrifiar,
pensou em comer da carne do sacrifício de ofertas de paz no terceiro
dia, este não será aceito, nem será levado em conta aquele que o
oferecer; impuro será (pigul), e quem comer dele, a sua iniqüidade
levará" (Levítico 7:18). A Tradição explica que este versículo se refere
a um sacrifício que se tornou inválido devido a intenções
inadequadas quanto à sua finalidade, tidas no momento da oferta, sendo
tal sa­crifício conhecido como ``pigul'', e que Suas palavras "Pensou em
comer da carne" se referem apenas à intenção de comê-lo no terceiro dia.
Portanto a Gue­mará diz: "Ouça com atenção! O versículo se refere àquele
que nesse momen­to tiver a intenção de comer a carne de seu sacrifício
no terceiro dia", e nos diz que tal intenção desqualifica o sacrifício e
que todo aquele que comer dele após ter tido essa intenção fica sujeito
à extinção, pois está dito: "E quem co­mer dele, a sua iniqüidade (avon)
levará"; e a respeito do ``notar'' está dito: "E aquele que o comer,
levará sobre si sua iniqüidade (avon)... e será banida aquela alma de
seu povo" (Ibid., 19:8).

A Guemará de Queretot diz: "Nunca negligencie um `guezerá sha­vá'! A lei
do `pigur , que é um dos preceitos essenciais da Torah, foi deduzida
apenas através de um `guezerá shavá'. Um versículo diz: 'A sua
iniqüidade (avon) levará' e o outro diz: 'Levará sobre si sua iniqüidade
(avon)'; assim como num caso há extinção, também há extinção no outro.

Aquele que comer ``pigul'' involuntariamente também fica obriga­do a levar
um Sacrifício Determinado de Pecado.

As normas relativas ao "pigur.' e ao ``notar'' estão explicadas em vá­rios
trechos da Ordem de ``Kadashim''.

\section{Um ``zar'' não deve comer ``terumá''}


Por esta proibição um ``zar''\textsuperscript{295} fica proibido de comer
qualquer "terumá
"\textsuperscript{298}. Ela está expressa em Suas palavras "E todo o
estranho não comerá\\
da santidade" (Levítico 22:10), nas quais a expressão ``santidade''
significa "terumá
", bem como primícias, pois elas também são chamadas de ``terumá'',
como
explicarei; foi isso o que eu quis dizer quando falei de "qualquer
`terumá' ".\\
A mesma lei se aplica a` todo aquele que cometer sacrilégio
deliberadamente.\\
Aquele que comer ``terumá'' voluntariamente está sujeito à morte\\
pela mão dos Céus, mas não fica obrigado a acrescentar uma quinta parte,
como
está explicado no sexto e sétimo capítulos do Tratado Terumá. No
nono\\
capítulo de Sanhedrin, um ``zar'' que comer ``terumá'' está incluído na
lista dos\\
pecadores sujeitos à morte pela mão dos Céus, e isso se justifica por
Suas palavras
"E não levarão sobre si pecado, pois morrerão por isto quando o
profanarem
" (Ibid., 9), que são seguidas por "E todo o estranho não comerá da
santidade
". Da mesma forma, a Mishná diz no segundo capítulo de Bicurim:
"Podese
incorrer na pena de morte pelo Sacrifício de Elevação e pelas
primícias; eles\\
estão sujeitos ao quinto adicional e são proibidos aos que não são
`Cohanim' ".\\
Rav discorda de todas essas leis da Mishná e diz que um ``zar'' que\\
come ``terumá'' é punido com o açoitamento\textsuperscript{297}, e é
sabido que, sendo um Taná
, Rav pode discordar\textsuperscript{298}. Já explicamos em nosso
"Comentário sobre a Mishná
" que em todas as discussões que não afetem o procedimento, e sim
apenas

\begin{enumerate}
\def\labelenumi{\arabic{enumi}.}
\setcounter{enumi}{294}
\item
 
 Ver o preceito negativo 74, onde Maimônides explica que quando ele
 escreve ``zar'', está se referindo a qualquer um que não seja
 descendente de Aarão.
 
\item
 
 A Oferenda de Elevação (ver o preceito positivo 126).
 
\item
 
 E não com a morte pela mão dos Céus.
 
\item
 
 Ou seja, apesar de ter vivido após os Tanaim (os responsáveis pela
 redação e,impressão da Mishná), Rav era considerado uma autoridade tão
 alta que podia discordar da opinião unânime dos Tanaim.
 
\end{enumerate}


a opinião, não darei uma decisão em favor de um ponto de vista ou de
outro. Sendo assim eu me absterei de dizer se o correto é o conceito de
Rav ou o da Mishná anônima, uma vez que todos concordam que ele está
sujeito ao açoita­mento. Isto é conseqüência da regra explicada na
Introdução deste trabalho, segundo a qual todos aqueles que estão
sujeitos à morte pela mão dos Céus por ter violado qualquer um dos
preceitos negativos também estão sujeitos ao açoitamento. A mesma lei se
aplica àquele que deliberadamente cometer sacri­légio ao aproveitar-se
de objetos sagrados, como está demonstrado pelo que se diz sobre o caso
de um menino próximo da maioridade religiosa que faz uma consagração:
"Se ele o consagra e outros o comem, Rabi Yohanan e Resh La­kish são
ambos de opinião que eles devem ser punidos com •o açoitamento.

\section{Um servo ou um criado de um ``cohen'' não devem comer ``terumá''}

Por esta proibição até mesmo um servo ou um criado israelita de um
``Cohen'' ficam proibidos de comer ``terumá''. Ela está expressa em Suas
pala­vras "Aquele que mora com o 'Cohen' e o jornaleiro, não comerá da
santida­de" (Levítico 22:10). Aquele que o comer será tratado da mesma
forma que um ``zar''\textsuperscript{299}.

\section{Um ``cohen'' incircunciso não deve comer ``terumá''}

Por esta proibição um homem incircunciso fica proibido de comer
``terumá'', e a mesma lei se aplica no caso de todas as outras santidades:
um homem incircunciso está proibido de comê-las. Esta proibição não está
expres­samente enunciada nas Escrituras, mas provém de um "guezerá
shavá"\textsuperscript{300}, e os guardiães da Tradição explicam ainda
que esta é uma proibição da Torah e não simplesmente uma proibição
Rabínica. A explicação se encontra em Ye­bamot: "De onde se conclui que
um incircunciso não pode comer `terumá'? A Torah usa a expressão `toshab
vesakhir' (alguém que reside temporariamente e um criado) no caso do
cordeiro de Tessah' e no caso do `terumá'. Portanto, assim como o
cordeiro do Tessah' --- em relação ao qual foi usado `toshab ve­sakhir'
--- está proibido ao incircunciso, assim também o `terumá' --- em
rela­ção ao qual também foi usado `toshab vesakhir' --- está proibido ao
incircunci­so", e a mesma lei se aplica a todos os outros sacrifícios
consagrados. Encon­tramos o mesmo texto na Sifrá, onde também lemos:
"Rabi Akiba diz: O uso da expressão 'Todo o homem' (Levítico 22:4)
significa que o homem incircun­ciso está incluído".

Na Guemará de Yebamot também está explicado que a Torah per­mite que um
``mashukh''\textsuperscript{301} coma ``terumá'', mas os Sábios proíbem
isso, por­que ele parece incircunciso.


\begin{enumerate}
\def\labelenumi{\arabic{enumi}.}
\setcounter{enumi}{298}
\item
 
 Alguém que não é descendente de Aarão.
 
\item
 
 ``Expressão similar'', ou seja, uma analogia entre duas leis
 estabelecida com base na congruência verbal dos textos das Escrituras.
 
\item
 
 Alguém que teve seu prepúcio puxado para a frente a fim de cancelar o
 sinal do pacto de Abraham.
 
\end{enumerate}


PRECEITOS NEGATIVOS 251

Ficou assim claro que um homem incircunciso está proibido de co­mer pela
Torah, e um ``mashukh'' pela lei Rabínica. Isto deve ser compreendido.

No mesmo trecho lemos que de acordo com a lei Rabínica um "mas­hukh"
deve ser circuncidado novamente.

\section{Um ``cohen'' impuro não deve comer "terumá'}

Por esta proibição um ``Cohen'' que estiver impuro fica proibido de comer
``terumá''. Ela está expressa em Suas palavras "Todo homem da semen­te de
Aarão, que for leproso, ou que tiver fluxo, das santidades não comerá
até que se purifique" (Levítico 22:4).

Na Guemará de Macot lemos: "De que forma deduzimos a proibição
necessária relativa à Oferenda de Elevação? Das palavras 'Todo o homem
(pes­soa) da semente de Aarão, que for leproso...' E que coisas são
permitidas de forma igualitária à 'semente de Aarão'? Você é obrigado a
dizer: a Oferenda de Elevação", sendo que "de forma igualitária à
semente de Aarão" significa que todos os que vêm de sua semente, tanto
machos como fêmeas, podem comer.

Esta proibição aparece novamente em Suas palavras, abençoado se­ja Ele,
"E guardarão (veshameru) este Meu mandado" (Ibid.,9).

O castigo pela contravenção desta proibição é a morte pela mão dos Céus.
No nono capítulo de Sanhedrin um ``Cohen'' impuro que comer uma Ofe­renda
de Elevação pura está incluído na lista dos pecadores que estão sujeitos
à morte pela mão dos Céus, e isto se justifica pelas Suas palavras "E
guardarão este Meu mandado e não levarão sobre si pecado".

\section{Uma ``halalá'' não. deve comer alimento sagrado}

Por esta proibição uma ``halalá''.°.
fica proibida de comer alimento sagrado que lhe teria sido permitido
comer, ou seja, o peito e a coxa. Ela está expressa em Suas palavras "E
a filha do 'Cohen', quando se casar com um ho­mem estranho, ela não
comerá daquilo que se separa das santidades" (Levítico 22:12).

Na Guemará de Yebamot lemos: " 'Quando se casar com um homem estranho':
assim que ela passar a viver com um homem desqualificado, ele a
desqualificará". As palavras "Ela não comerá do que é separado das
santida­des" são interpretadas como significando "Ela não comerá daquilo
que se se­para das oferendas consagradas", ou seja, o peito e a coxa.

No mesmo trecho está dito: "A Torah poderia ter dito 'Ela não co­merá
das oferendas consagradas'. Por que 'das (bitrumath) santidades'? É para
ensinar-nos duas coisas", a saber, que se a filha de um ``Cohen'' viver
com um homem desqualificado, ele a desqualifica no que se refere a comer
"teru­má"\textsuperscript{303}, e que se ela esteve casada com um
``zar''." e ele morreu, ela recupera o direito a
Oferenda de Elevação mas não o direito ao peito e a coxa.


\begin{enumerate}
\def\labelenumi{\arabic{enumi}.}
\setcounter{enumi}{301}
\item
 
 A filha de um ``Cohen'' que se casa com alguém com quem não lhe é
 permitido casar-se.
 
\item
 
 A Oferenda de Elevação. •
 
\item
 
 Alguém que não seja descendente de Aarão.
 
\end{enumerate}



Conseqüentemente, esta proibição, a saber, "Não comerá das santi­dades"
abrange dois assuntos: primeiro, ela proíbe uma ``halalá'' de comer
ali­mento sagrado; segundo, ela proíbe a filha de um ``Cohen'' que foi
casada com um ``zar'' de comer o peito e a coxa, mesmo que seu marido
tenha morrido ou se divorciado dela.

Contudo, a proibição de comer ``terumá'' enquanto ela viver com seu
marido, se ele for um ``zar'', não está baseada neste versículo, ela foi
dedu­zida por aqueles que interpretam a Torah de Suas palavras "E todo o
estranho (zar) não comerá da santidade" (Ibid., 10). Enquanto ela viver
com um ``zar'', ela própria será uma ``zar'' e a ela aplicaremos a lei do
``zar''.

Você deve compreender isto, e deve saber que ela também está su­jeita ao
açoitamento se violar esta proibição.

\section{Não comer a oblação de um ``cohen''}

Por esta proibição somos proibidos de comer a Oblação de cereal de um
``Cohen''. Ela está expressa em Suas palavras, enaltecido seja Ele, "E
to­da a oblação de `Cohen' será queimada totalmente, não será comida"
(Levítico 6:16) e está repetida com relação aos bolos assados do "Cohen
Gadol", que também são uma oblação. A transgressão desta proibição
também é punida com o açoitamento.

A Sifrá diz: " 'Será queimada totalmente': isso significa que somos
proibidos por um preceito negativo de comer qualquer coisa que deva ser
quei­mada totalmente".

\section{Não comer carne de sacrifícios de pecado cujo sangue tenha sido
levado para dentro do santuário}

Por esta proibição os ``Cohanim'' ficam proibidos de comer a carne dos
Sacrifícios de Pecado que devem ser oferecidos no Santuário. Ela está
ex­pressa em Suas palavras, enaltecido seja Ele, "E todo o sacrifício de
pecado cu­jo sangue for trazido à tenda da assinação para expiar na
santidade, não será comido; no fogo será queimado" (Levítico 6:23).
Comê-la é punido com o açoitamento.

A Sifrá diz: " `Não será comido; no fogo será queimado': isto signifi­ca
que somos proibidos, por um preceito negativo, de comer qualquer coisa
que deva ser queimada".

\section{Não comer sacrifícios consagrados que tenham sido invalidados}

Por esta proibição somos proibidos de comer sacrifícios consagra­dos que
tenham sido invalidados em virtude de um defeito causado
proposital­mente,(como foi explicado no Tratado Bekhorot), depois do
degolamento, por
qualquer uma das formas que fazem com que um sacrifício consagrado se
tor­ne alimento proibido. Esta proibição está expressa em Suas palavras
"Não co­merás nada do que for abominável" (Deuteronômio 14:3), sobre as
quais diz o Sifrei: " 'Não comerás nada do que for abominável' se refere
a sacrifícios con­sagrados que foram invalidados". Diz também: "Rabi
Eliezer ben Jacob diz: De que forma sabemos que, se alguém fizer um
corte na orelha do primogênito de um animal e comer dele, ele estará
violando um preceito negativo? Pelas palavras das Escrituras: 'Não
comerás nada do que for abominável' ".

Comê-los é punido com o açoitamento.


As normas deste preceito estão explicadas no Tratado Bekhorot.

\section{Não comer o segundo dízimo de cereais não remido fora de jerusalém}

Por esta proibição somos proibidos de comer o segundo dízimo de cercais
fora de Jerusalém. Ela está expressa em Suas palavras, enaltecido seja
Ele, • 'Mas não te será permitido comer em tuas cidades o dízimo de teus
ce­reais" (Deuteronômio 12:17).

A punição por comer o segundo dízimo não remido é o açoitamento, de
acordo com a explicação expressa no final de Macot, ou seja, se ele for
comi­do fora de Jerusalém depois que ele tenha "visto a fachada do
Templo", ou seja, depois que ele tenha sido levado para dentro das
muralhas de Jerusalém. Isto está expresso no Talmud: "A partir de quando
se fica sujeito à penalidade de açoitamento? A partir do momento em que
ele 'vir a fachada do Templo' "

\section{Não consumir o segundo dízimo de vinho não remido fora de Jerusalém}

Por esta proibição somos proibidos de consumir o segundo dízimo de vinho
fora de Jerusalém. Ela está expressa em Suas palavras, enaltecido seja
Ele, "Mas não te será permitido comer em tuas cidades o dízimo de teus
ce­reais, e de teu mosto" (Deuteronômio 12:17).

O açoitamento é a punição por consumí-lo, desde que isso seja feito nas
mesmas condições que no caso do dízimo dos cereais.

\section{Não consumir o segundo dízimo de azeite não remido fora de Jerusalém}

Por esta proibição somos proibidos de consumir o segundo dízimo de
azeite fora de Jerusalém. Ela está expressa em Suas palavras "Mas não te
será permitido... e de teu azeite" (Deuteronômio 12:17). Consumí-lo é
punido com o açoitamento, desde que seja feito nas mesmas condiçõés que
no caso do dízi­mo dos cereais.

Se você está surpreso por contarmos as proibições relativas aos dízi­mos
de cereais, da vindima e do azeite como três preceitos, você deve saber
que aquele que comer os três ao mesmo tempo está sujeito a um
açoitamento por cada um deles pois a proibição expressa neste versículo
("Não te será permi­tido comer em tuas cidades o dízimo de teus cereais,
e de teu mosto, e de teu azeite") não é um "lav
shebikhIalut"\textsuperscript{305}, pelo qual não se aplica a penalidade
de açoitamento. Ao contrário, este texto indica uma divisão. Está
explicitamente dito na Guemará de Macot: "Se alguém comer do dízimo de
cereais, de vinho e de azeite, ele estará sujeito a um castigo por cada
um deles separadamente. Mas aplica-se o açoitamento por uma proibição
coletiva? O texto é redundante. Veja bem: na Torah já estava dito 'E
comerás diante do Eterno, teu Deus... o dízimo de teu grão, teu mosto, e
teu azeite' (Ibid., 14:23); então por que ela os expõe de novo,
detalhadamente? Deve ser para estabelecê-los separadamente.

A Guemará de Macot diz: "Veja bem: se já estava escrito 'E comerás
diante do Eterno, teu Deus... o dízimo de teu grão, teu mosto, e teu
azeite', não poderia o Todo Misericordioso, ter simplesmente dito o
seguinte: 'Não de­ves comê-los dentro de teus portões'? Que outro
objetivo poderia Ele ter ao enunciá-los novamente, em detalhe, a não ser
o de enfatizar separadamente a proibição relativa a cada caso?"

Assim foi deixado claro que cada um dos assuntos mencionados neste
versículo é objeto de um preceito negativo diferente. Voltarei a este
assunto e complementarei o exame das outras proibições expressas neste
versículo.

\section{Não comer um primogênito sem defeito fora de Jerusalém}

Por esta proibição somos proibidos de comer um primogênito sem defeito
fora de Jerusalém. Ela está expressa em Suas palavras "Mas não te será
permitido comer dentro de tuas cidades... nem os primogênitos de teu
gado, e de teu rebanho" (Deuteronômio 12:17), sobre as quais o Sifrei
diz: " 'Os pri­mogênitos' se refere à primeira cria e o objetivo do
texto é ensinar-nos que um `zar'\textsuperscript{306} que comer um
primogênito, seja antes ou depois de seu sangue ter sido aspergido,
estará dessa forma transgredindo um preceito negativo".

Assim, fica claro que esta proibição abrange dois assuntos: ela proí­be
um ``zar'' de comer um primogênito sem defeito, e um ``Cohen'' de comer um
primogênito sem defeito fora de Jerusalém. Esses dois assuntos
constituem a lei relativa ao primógênito sem defeito.

A transgressão desta proibição é punida com o açoitamento.

\section{Não comer o sacrifício de pecado e o sacrifício de delito fora 
do campo do santuário}

Por esta proibição somos proibidos de comet o Sacrifício de Pecado e o
Sacrifício de Delito fora do Campo do Santuário, e esta proibição se
aplica até mesmo aos ``Cohanim''. Ela está expressa em Suas palavras, no
mesmo ver­sículo, ``De teu gado e de teu rebanho'' (Deúteronômio 12:17). É
como se Ele


\begin{enumerate}
\def\labelenumi{\arabic{enumi}.}
\setcounter{enumi}{304}
\item
 
 Uma proibição negativa geral (ver o nono fundamento).
 
\item
 
 Alguém que não é descendente de Aarão.
 
\end{enumerate}


tivesse dito: "Não deves comer dentro de teus portões o dízimo de teus
ce­reais, de teu gado, nem de teu rebanho"; e o Sifrei explica: " 'De
teu gado e de teu rebanho': o versículo se refere apenas ao caso de uma
pessoa que viola um preceito negativo ao comer um Sacrifício de Pecado
ou um Sacrifício de Delito fora das cortinas\textsuperscript{307}" e é
punida com o açoitamento.

Da mesma forma, aquele que comer os sacrifícios menos sagrados fora das
muralhas também será punido com o açoitamento, como está explica­do na
Guemará de Macot, porque comer qualquer uma das santidades fora do local
designado para isso está incluído na proibição ``Não te será permitido''
(Ibid.).

\section{Não comer carne de um holocausto}

Por esta proibição somos proibidos de comer a carne de um Holo­causto.
Ela está expressa em Suas palavras, enaltecido seja Ele, "Mas não te
será permitido... nem os teus votos que ofereceres" (Deuteronômio
12:17). Isto é como se Ele tivesse dito: "Não deves comer os votos que
ofereceres"; e o Si-frei explica: " 'Nem os teus votos': o objetivo do
versículo é apenas mostrar-lhe que aquele que comer um Holocausto, seja
antes ou depois de aspergir seu sangue, seja dentro ou fora das
cortinas, estará violando um preceito negativo".

Este preceito negativo serve como proibição contra qualquer tipo de
sacrilégio.

Todo aquele que desobedecer este preceito voluntariamente --- ou seja,
que comer a carne de um Holocausto, ou que se tornar culpado de
sacrilé­gio por obter um proveito qualquer de qualquer um dos outros
sacrifícios con­sagrados, como explicado no Tratado Meilá --- será
punido com o açoitamento. Aquele que o desobedecer involuntariamente
fica obrigado a levar um Sacrifí­cio de Delito por
sacrilégio." e a pagar de acordo com o valor daquilo
de que ele usufruiu, acrescido da quinta
parte.°., como explicado no Tratado
Meilá.

No nono capítulo de Sanhedrin lemos: "Se ele cometeu sacrilégio
deliberadamente, Rabi Yehudá diz que ele está sujeito à morte, mas os
Sábios dizem que ele está sujeito ao açoitamento"; e os Sábios citam em
apoio a seu ponto de vista: " 'Pois morrerão por isto' (Levítico 22:9),
significando que mor­rerão por 'isto', mas não por causa do sacrilégio".

\section{Não comer sacrifícios menos sagrados antes de aspergir seu sangue sobre o altar}

Por esta proibição somos proibidos de comer dos sacríficios menos
sagrados antes de aspergir seu sangue. Ela está expressa em Suas
palavras "Mas não te será permitido comer em tuas cidades... nem tuas
ofertas voluntárias" (Deuteronômio 12:17). É como se Ele tivesse dito:
``Não deverás comer tuas ofertas voluntárias'' e, de acordo com a
Tradição, "o versículo se refere apenas


\begin{enumerate}
\def\labelenumi{\arabic{enumi}.}
\setcounter{enumi}{306}
\item
 
 Fora das cortinas do Tabernáculo.
 
\item
 
 Ver o preceito positivo 71.
 
\item
 
 Ver o preceito positivo 118.
 
\end{enumerate}

a alguém que viola um preceito negativo ao comer o Sacrifício de Graças
ou o de Paz antes de aspergir seu sangue"; ele também será punido com o
açoitamento.

\section{Um ``cohen'' não pode comer as primícias fora de Jerusalém}

Por esta proibição um ``Cohen'' fica proibido de comer as primícias fora
de Jerusalém\textsuperscript{311}. Ela está expressa em Suas palavras
"Não te será permiti­do comer em tuas cidades... as oferendas (terumá)
de tua mão" (Deuteronômio 12:17) pois, de acordo com a Tradição, "A
expressão 'As oferendas de tua mão' significa as primícias". Como este
versículo menciona explicitamente tudo que deve ser levado, e como ele
inclui ``as oferendas de tua mão'', não pode haver dúvida de que está dito
que elas devem ser levadas.". Mas nós sabemos que a
``terumá'' não precisa ser levada; então como pode Ele ter-nos proibido de
comê-la ``em tuas cidades''?

De acordo com o Sifrei, "Este versículo se refere apenas ao caso de uma
pessoa que, ao comer as primícias sem declamar sobre
elas\textsuperscript{312}, viola um preceito negativo". E está explicado
no final de Macot que alguém se torna cul­pado
apenas\textsuperscript{313} antes de colocá-las no Campo do Santuário,
mas uma vez que as colocar ali ele estará inocente ainda que não tenha
declamado sobre elas.

Uma das condições impostas com relação ao segundo dízimo se aplica
também às primícias, a saber, que aquele que as comer fora de Jerusalém
não será culpado a menos que elas tenham ``visto a fachada do Templo''.
Aquele que as comer depois de elas terem visto ``a fachada do Templo'',
mas antes de que elas tenham sido colocadas no Campo do Santuário,
estará sujeito apenas ao açoitamento, se ele for um ``Cohen''; mas se for
um israelita, ele estará sujei­to à morte pela mão dos Céus por ter
comido as primícias, ainda que ele tenha declamado sobre elas primeiro.

A Mishná diz explicitamente: "A Oferenda de Elevação e das primí­cias
podem causar a pena de morte, sujeitam ao quinto adicional e são
proibi­dos para quem não for 'Cohen' ". Assim, se ele as comer
intencionalmente es­tará sujeito à morte, e se o fizer
involuntariamente, ele deve acrescentar uma quinta parte, como no caso
da Oferenda de Elevação. Pois como as Escrituras se referem a elas como
"a oferenda (terumá) de tua mão", segue-se que elas estão sujeitas à
mesma lei que a terurna".

É importante que você compreenda bem isto para que não se enga­ne a este
respeito. A lei é a seguinte: um ``Cohen'' que comer das primícias de­pois
que elas tenham ``visto a fachada do Templo'' mas antes de que tenham sido
colocadas no Campo do Santuário está sujeito ao açoitamento, sendo que a
proibição está expressa em Suas palavras "Não te será permitido comer em
tuas cidades... as oferendas de tua mão", de acordo com o que está
explicado em Macot, assim como no caso de um israelita que está sujeito
ao açoitamento por comer o segundo dízimo fora de Jerusalém, embora ele
lhe pertença. Mas um israelita que comer as primícias em qualquer local,
depois que elas tenham

310 Ver os preceitos positivos 125 e 132.

311 Deuteronômio 26:2-4.

312 Ver o preceito positivo 132.

313. A pessoa só se torna culpada ao comè-las antes de colocá-las no
Campo do Santuário.


``visto a fachada do Templo'', está sujeito à morte pela mão dos Céus,
estando\\
a proibição expressa nas palavras "E todo o estranho não comerá da
santidade
" (Levítico 22:10), como explicamos ao tratar do número 133 destes
preceitos.\\
As normas deste preceito estão explicadas na Guemará de Macot.

\section{Um ``zar'' não pode comer os sacrifícios mais sagrados}

Por esta proibição um ``zar''\textsuperscript{314} fica proibido de comer
dos sacrifí­cios mais sagrados. Ela está expressa em Suas palavras "O
estranho não comerá delas pois é santidade" (Êxodo 29:33).

Não se fica sujeito ao açoitamento a menos que se coma dentro do Campo
do Santuário e depois de ter aspergido o sangue do sacrifício.

\section{Não comer o segundo dízimo impuro não remido, nem mesmo em Jerusalém}

Por esta proibição somos proibidos de comer, mesmo estando em Jerusalém,
um segundo dízimo que tenha se tornado impuro, até que ele seja remido.
O princípio aceito a respeito é que um segundo dízimo que se tornou
impuro deve ser remido até mesmo em Jerusalém, como está explicado em
Ma-cot. A proibição está expressa em Suas palavras "Não comi dele, em
estado im­puro" (Deuteronômio26:14) que significam, de acordo com a
Tradição: "Nem quando eu estava impuro e ele puro, nem quando eu estava
puro e ele impuro".

A Guemará de Macot explica ainda que é proibido comer o segundo dízimo
ou as primícias que se tornaram impuros e que uma pessoa que se tor­nou
impura está sujeita ao açoitamento se ela os comer, desde que ela coma o
dízimo não remido em Jerusalém em estado de impureza; somente nesse
ca­so ela estará sujeita ao açoitamento, como dissemos.

As normas deste preceito estão explicadas no final de Macot.

\section{Não comer o segundo dízimo durante o período de luto}

Por esta proibição somos proibidos de comer o segundo dízimo du­rante o
período de luto\textsuperscript{315}. Ela está expressa em Suas
palavras, enaltecido seja Ele, "Não comi do segundo dízimo no primeiro
dia de luto" (Deuteronômio 26:14). A Mishná diz que o dízimo e as
primícias devem ser levados a Jerusa­lém, devem ser
declarados\textsuperscript{316} e são proibidos a um
``onen''\textsuperscript{317}. Da mesma for­ma, uma pessoa de luto fica
proibida por este versículo de comer dos sacrifí­cios consagrados, pois
também está escrito na Torah: "E me aconteceram tais coisas; se eu
tivesse comido do sacrifício de pecado do dia, agradaria aos olhos do
Eterno?" (Levítico 10:19).

314 Alguém que. não seja descendente de Aarão.


\begin{enumerate}
\def\labelenumi{\arabic{enumi}.}
\setcounter{enumi}{314}
\item
 
 Ver o preceito positivo 37.
 
\item
 
 Ver o preceito positivo 131.
 
\end{enumerate}


317 Alguém que esteja de luto.

As leis do luto estão explicadas no oitavo capítulo de Pessahim e no
segundo capítulo de Zebahim.

Aquele que comer dos sacrifícios consagrados ou do dízimo duran­te o
luto estará sujeito ao açoitamento.

\section{Não gastar o dinheiro do resgate do segundo dízimo a não ser com comida e bebida}

Por esta proibição somos proibidos de gastar o dinheiro do segun­do
dízimo em outras coisas que não comida e bebida. Ela está expressa em
Suas palavras, enaltecido seja Ele, "E não o troquei para fazer o
sepultamento de um morto" (Deuteronômio 26:14), a respeito das quais diz
o Sifrei: ``Não usei nada dele para um caixão ou uma mortalha''. Aquele
que gastar qualquer parte do dinheiro em outras coisas deve gastar uma
quantia equivalente em comida, co­mo está explicado no lugar apropriado.

``O morto'' está mencionado para dar maior ênfase, como se Ele ti­vesse
dito: "Embora importante, você não deve gastar o dinheiro do segundo
dízimo para esse fim".

Parece-me que uma vez que o Enaltecido nos ordena usar o dinhei­ro do
segundo dízimo apenas para comida, pelas Suas palavras "E darás este
dinheiro" (Deuteronômio 14:26), gastá-lo em outras coisas que não
alimento equivale a dá-lo aos mortos, já que eles não podem usufruir
dele.

\section{Não comer ``tebel''}

Por esta proibição somos proibidos de comer ``tebel'', isto é, um pro­duto
do qual não se tenha separado a Oferenda de Elevação e os dízimos. Ela
está expressa em Suas palavras, enaltecido seja Ele, "E não profanarão
as santi­dades dos filhos de Israel, que eles separarem (yarimu) para o
Eterno" (Levítico 22:15).


A transgressão desta proibição --- ou seja, comer o ``tebel'' é punida\\
com a morte pela mão dos Céus, como se pode deduzir pelo fato de que
aqui\\
Ele diz: ``Não profanarão'', e de que no. caso da Oferenda de Elevação Ele
diz\\
igualmente: ``As santidades dos filhos de Israel não profanareis''
(Números 18:32).\\
A referência à ``profanação'' em ambos os casos indica que aqui, assim
como\\
no caso da Oferenda de Elevação, a penalidade é a morte, como
explicamos.\\
A Guemará de Sanhedrin diz: "De que forma sabemos que aquele\\
que comer tebel' está sujeito à morte? Pelo versículo "Não profanarão as
santidades
dos filhos de Israel'. O versículo se refere àquilo que ainda vai
ser oferecido
, e a identidade da lei se conhece pelo uso da palavra 'profanação'
neste\\
caso bem como no da Oferenda de Elevação". A expressão "aquilo que
ainda\\
vai ser oferecido" significa que é como se Ele tivesse dito: "Não deves
profanar
as santidades que as pessoas 'ainda vão separar' para o Eterno". É
por isso\\
que Ele diz, enaltecido seja Ele, ``et asher yarimu'' (que eles
separarão), usando\\
o verbo no futuro. No versículo seguinte a este Ele diz: "A fim de que
não levem
sobre si delito de culpa comendo as suas santidades" (Levítico
22:16).\\
A Guemará de Macot diz: "Eu poderia pensar que se é culpado apenas
por comer o `tebel' do qual nenhuma contribuição foi separada ainda.
Mas\\
e no caso da grande Oferenda de Elevação ter sido separada, mas a
Oferenda



de Elevação do dízimo\textsuperscript{318} ainda não, ou quando o
primeiro dízimo tiver sido separado, mas o segundo dízimo não, ou ainda
o dízimo dos pobres, de que forma\textsuperscript{319} comer esses
produtos? Através dos seguintes textos instrutivos: 'Mas não te será
permitido comer ``em tuas cidades'' o dízimo de teus cereais etc'
(Deuteronômio 12:17); e mais adiante está dito: 'A fim de que os comam e
se fartem' (Ibid., 26:12). Assim como neste último, no versículo
anterior faz-se re­ferência ao dízimo do pobre, e o Misericordioso
ordena 'Não te será permitido comer' ".

Contudo, tudo isto se refere apenas ao açoitamento. O castigo de morte
só é decorrente da grande Oferenda de Elevação e da Oferenda de
Ele­vação do dízimo. Pois aquele que comer do primeiro dízimo antes que
a Ofe­renda de Elevação do dízimo tenha sido retirado dele está sujeito
à morte, de acordo com Suas palavras, enaltecido seja Ele, aos Levitas,
ao ordenar-lhes que separassem um dízimo do dízimo, "E as santidades dos
filhos de Israel não pro­fanareis e não morrereis" (Números 18:32). Esta
é a proibição de comer do pri­meiro dízimo que ainda é ``tebel'' e sua
violação acarreta a morte, como está explicado no Tratado Demai.

O essencial de todo este debate é o seguinte: aquele que comer "te­bel"
do qual ainda não se tenha separado a grande Oferenda dg Elevação e a
Oferenda de Elevação do dízimo está sujeito à morte, e a proibição está
expres­sa nas palavras "Não profanarão as santidades dos filhos de
Israel...", como ex­plicamos ao tratar deste preceito. Aquele que comer
``tebel'' depois de ter se­parado as Oferendas de Elevação mas antes de
separar todos os dízimos está sujeito ao açoitamento, e a proibição está
expressa nas palavras "Mas não te será permitido comer em tuas cidades o
dízimo de teus cereais...". Você deve se lembrar disso e não se enganar
a esse respeito.

As normas relativas ao ``tebel'' estão explicadas em vários trechos dos
Tratados Demai, Terumot e Maasserot.

\section{Não alterar a ordem prescrita para separar o dízimo da colheita}

Por esta proibição somos proibidos de mudar a ordem das doações dos
produtos; devemos separá-las de acordo com a ordem determinada. A
ex­plicação é a seguinte: quando, por exemplo, o trigo tiver sido
debulhado e ar­rumado numa pilha uniforme e tiver se tornado ``tebel'',
deve-se separar dele primeiro a grande Oferenda de Elevação que é uma
quinquagésima parte, e de­pois um décimo do que restar, que será o
primeiro dízimo, e finalmente um décimo do restante, que será o segundo
dízimo. A grande Oferenda de Eleva­ção deve ser entregue ao ``Cohen'', o
primeiro dízimo ao Levita e o segundo dízimo deve ser comido pelo seu
proprietário, em Jerusalém. Esta é a ordem segundo a qual as doações
devem ser separadas, e a proibição de separar pri­meiro o que deve ser
separado por último e retardar o que deve ser separado primeiro está
expressa em Suas palavras, enaltecido seja Ele, "Não tardarás em
oferecer da plenitude de tua colheita e do que sai de tuas prensas"
(Êxodo 22:28), que é como se Ele tivesse dito: "Não demorarás em traze,
da plenitude de tua


\begin{enumerate}
\def\labelenumi{\arabic{enumi}.}
\setcounter{enumi}{317}
\item
 
 O dízimo que o levita deve separar para dar ao
 .Cohen".
 
\item
 
 I.e., de que forma tomamos conhecimento da proibição de comer tais
 produtos?
 
\end{enumerate}

colheita e do que sai de tuas prensas, aquilo que deve ser oferecido em
primei­ro lugar".

Na Mishná de Terumot lemos: "Se alguém oferecer a Oferenda de Elevação
antes das primícias, ou o primeiro dízimo antes da Oferenda de
Eleva­ção, ou o segundo dízimo antes do primeiro, sua ação será válida,
embora ele esteja transgredindo um preceito negativo, pois está dito:
"Não tardarás em ofe­recer da plenitude de tua colheita e do que sai de
tuas prensas".

A Mekhiltá diz: " 'A plenitude de tua colheita': ou seja, as primícias
retiradas das colheitas plenas. 'E o que sai de tuas prensas': ou seja,
a Ofereficla de Elevação. São tardarás': não deixe que o segundo preceda
o primeiro, o primeiro preceda a Oferenda de Elevação, ou a Oferenda de
Elevação preceda as primícias". E acrescenta: "A partir disso deduz-se o
princípio de que se al­guém oferecer a Oferenda de Elevação antes da
primícias, ou o segundo dízi­mo antes do primeiro, embora ele esteja
violando um preceito negativo, seu ato ainda será válido". E está
explicado no primeiro capítulo de Temurá que aquele que proceder fora de
ordem não está sujeito ao açoitamento.

\section{Não adiar o pagamento de promessas}

Por esta proibição somos proibidos de adiar promessas, sacrifícios
voluntários e outras oferendas a que nos tenhamos comprometido. Ela está
ex­pressa em Suas palavras, enaltecido seja Ele, "Quando fizeres algum
voto ao Eterno, teu Deus, não demorarás em pagá-lo" (Deuteronômio23:22).
E, de acor­do com a Tradição, não se transgride esta proibição até que
três Festivais te­nham se passado.


As normas deste preceito estão explicadas no início do Tratado Rosh


Hashaná.

\section{Não comparecer a um festival sem um sacrifício}

Por esta proibição somos proibidos de ir ao Santuário num Festival sem
um sacrifício para ser oferecido ali. Ela está expressa em Suas palavras
enal­tecido seja Ele, "E ninguém aparecerá diante de Mim com as mãos
vazias" (Exodo 23:15).

As normas deste preceito estão explicadas no Tratado Haguigá. Esta
proibição não se aplica às mulheres.

\section{Não deixar de cumprir uma obrigação oral, mesmo que não se tenha 
feito um juramento}

Por esta proibição somos proibidos de infringir uma obrigação à qual
tenhamos nos comprometido oralmente, ainda que sete ter feito um
juramen­to. O que se tem em mente aqui é um voto, como quando alguém
diz: ``Se isto ou aquilo acontecer'', ou ``Se eu fizer isto ou aquilo'', não
tocarei em nada de que cresce ``no mundo'', ou ``nesta cidade'', ou "num
determinadQ tipo", como
por exemplo vinho, ou leite, ou peixe, ou algo assim; ou ainda se alguém
dis­ser: ``Eu renunciarei meus direitos conjugais'', ou se fizer qualquer
outro tipo de promessa que envolva uma obrigação do tipo das que estão
citadas como exemplo no Tratado Nedarim. Se alguém fizer isso, ele será
obrigado a cumprir essa promessa\textsuperscript{32}° e estará proibido
de quebrar sua palavra, de acordo com Suas palavras, enaltecido seja
Ele, "Não profanará (yahel) a sua palavra" (Números 30:3), que são
interpretadas como significando: "não fará as suas palavras pro­fanas"
(hulin), ou seja, ele não deixará de cumprir o que ele se comprometeu a
fazer. Nas.palavras da Guemará de Shebuot: "Os votos
estão sob a proibição `Não profanará a sua palavra' ".

No Sifrei lemos: " 'Não profanará' nos diz que se transgride duas
proi­bições --- 'Não profanará' e 'Não demorarás em pagá-lo'
(Deuteronômio 23:22)". Quer dizer, se alguém tiver prometido um
sacrifício ou algo semelhante, como por exemplo uma doação para o
tesouro do Templo, ou para a caridade, ou para uma sinagoga, ou similar,
e não tiver cumprido sua promessa até depois da passagem dos três
Festivais, ele será culpado por ambos São demorarás em pagá-lo' e 'Não
profanará'. E alguém que cometer uma transgressão ao fazer al­go que ele
se compremeteu a não fazer é punido com o açoitamento.


As normas deste preceito estão explicadas em detalhe no Tratado


Nedarim.

\section{Um ``cohen'' não pode casar-se com uma ``zoná''}

Por esta proibição um ``Cohen'' fica proibido de tomar uma ``zoná'' como
esposa. Ela está expressa em Suas palavras "Mulher prostituta (zoná) ou
profana não tomarão" (Levítico 21:7). Se um ``Cohen'' se chegar a uma
``zoná'' ele estará sujeito ao açoitamento.

\section{Um ``cohen'' não pode casar-se com uma ``halalá''}

Por esta proibição um ``Cohen'' fica proibido de tomar uma
"hala­lá"\textsuperscript{321} como esposa. Ela está expressa em Suas
palavras "Mulher prostituta ou profana (halalá) não tomarão" (Levítico
21:7). Se um ``Cohen'' se chegará ela, ele estará sujeito ao açoitamento.

\section{Um ``cohen'' não pode casar-se com uma mulher divorciada}

Por esta proibição um ``Cohen'' fica proibido de casar-se com uma mulher
divorciada. Ela está expressa em Suas palavras "Nem mulher d4vorcia­da
de seu marido não tomarão" (Levítico 21:7).


\begin{enumerate}
\def\labelenumi{\arabic{enumi}.}
\setcounter{enumi}{319}
\item
 
 Ver o preceito positivo 94.
 
\item
 
 A filha de um ``Cohen'' que se casou com alguém com quem estava proibida
 de casar-se.
 
\end{enumerate}


MAIMÔNIDES


\section{Um "cohen gadol não pode casar-se com uma viúva}


Por esta proibição o ``Cohen Gadol'' (e apenas ele) fica proibido de
ca­sar-se com uma viúva. Ela está expressa em Suas palavras, enaltecido
seja Ele, "Viúva ou divorciada, ou profana ou prostituta, a estas não
tomará" (Levítico 21:14).

Na Guemará de Kidushin se encontra a razão pela qual Ele proíbe
novamente o ``Cohen Gadol'' de casar-Ne com uma mulher divorciada, uma
"ha­lalá" ou uma ``zoná''. É que se um ``Cohen Gadol'' se chegar a uma
mulher que seja ao mesmo tempo viúva, divorciada, ``halalá'' e ``zoná'', ele
será puni­do por açoitamento quatro vezes, enquanto que um ``Cohen'' comum
será pu­nido apenas três. No mesmo trecho lemos: "Se com uma viúva, uma
mulher divorciada e profana, e uma meretriz" --- que está explicado como
significan­do uma única mulher que seja todas essas coisas --- "e se
elas forem nessa or­dem, ele é culpado por cada relação".

O significado das palavras ``Se elas forem nessa ordem'' é que as
des­qualificações devem ter acontecido à mulher na ordem em que foram
mencio­nadas no versículo, ou seja, que primeiro ela tenha se tornado
uma viúva, de­pois uma divorciada, depois uma ``halalá'', e finalmente uma
``zoná''. Somos a confirmar que isso é assim, porque desejamos que ele
fique sujeito ao açoita­mento quatro vezes por ter-se chegado a uma
mulher, em uma ocasião. E um princípio aceito que uma proibição não
vigora sobre algo que já está proibido por uma outra proibição, a menos
que ela seja um "issur mossif '\textsuperscript{322}, um "is­sur
colei"\textsuperscript{323}, ou um "issur bevad
ehad"\textsuperscript{324}, como explicamos no lugar apro­priado em
nosso ``Comentário sobre a Mishná'' de Queretot; e cada uma delas só será
um "issur mossif ' apenas se elas ocorrerem nessa ordem, como está ali
exposto. Contudo, se várias mulheres estiverem envolvidas como por
exemplo se ele se chegar a uma mulher viúva, a uma outra que for
``halalá'', a uma outra que for ``zoná'' e a uma outra que for divorciada
--- não há dúvidas de que ele está sujeito ao açoitamento por cada uma
delas separadamente.

Você pode objetar e argumentar o seguinte: "Uma vez que é um prin­cípio
aceito que não se fica sujeito ao açoitamento pela violação de um `lav
she­bikhlalut'\textsuperscript{325}, por que ele ficaria sujeito ao
açoitamento por cada violação sepa­radamente, já que elas são todas
proibidas por um único preceito negativo?" A resposta é que o objetivo
de repetir a proibição que impede o ``Cohen Gadol'' de casar-se com uma
mulher divorciada, com uma ``zoná'' ou com uma ``halalá'' é exatamente o de
deixar claro que nesse caso ele está sujeito à mesma regra que um
``Cohen'' comum, e é punido com o açoitamento separadamente. Sabe­mos que
um ``Cohen'' comum está sujeito ao açoitamento por cada violação pe­lo
fato de que uma delas foi mencionada especificamente, a saber, "Nem
mulher divorciada de seu marido não tomarão" (Levítico 21:7), o que
demonstra que há um castigo separado para cada transgressão; assim como
ele está sujeito ao acoitamento por causa de apenas uma mulher que for
divorciada, visto que esta proibição foi especificamente mencionada,
assim também ele está sujeito ao açoi­tamento por apenas uma que for
``zoná'', e por apenas uma que for ``halalá''.

322. Uma proibição adicional, ou seja, uma proibição que se aplica á
mais tipos de \emph{pessoas.} 323 Uma proibição inclusiva, ou seja, uma
proibição que se aplica a mais tipos de \emph{coisas.}


\begin{enumerate}
\def\labelenumi{\arabic{enumi}.}
\setcounter{enumi}{323}
\item
 
 Urna proibição simultãnea, ou seja, duas proibições que entram em
 vigor ao mesmo tempo.
 
\item
 
 Reprimenda negativa geral.
 
\end{enumerate}



Esse é o significado das palavras da Guemara de Kidushin 'Assim como
uma\\
mulher divorciada é diferente de uma ``zoná'' e de uma ``halalá'', em
relação\\
a um 'Cohen' comum, ela também é diferente em relação ao 'Cohen Gadol'
".

Também está explicado ali que caso haja várias mulheres envolvidas há
uma penalidade de açoitamento por cada violação separadamente, sem
le­var em conta se foram na ordem acima mencionada ou
não\textsuperscript{326}

Ficou, assim, claro que a proibição de chegar-se a cada uma delas
constitui um preceito negativo separado, e consequentemente, sua
violação acar­reta uma pena de açoitamento por cada uma delas,
separadamente.

Também está explicado ali que um ``Cohen'' comum não está sujei­to ao
açoitamento por cada uma das violações a menos que ele tenha se casado
com a mulher e se tenha, por conseguinte, chegado a ela. Isto está
exposto no Talmud da seguinte forma: "Se ele se chegar a ela, ele está
sujeito ao açoita­mento; se ele não se chegar a ela, ele não o estará,
pois as Escrituras dizem: 'A estas não tomará... e não profanará...'
(Levítico 2 1 :1 4-1 5). que significam: Por que ele não as tomará? Para
que possa não pi.ofanar".

As normas destes quatro preceitos\textsuperscript{32}' estão explicadas
por comple­to em Yebamot e Kidushin.

\section{Um ``cohen gadol'' não pode chegar-se a uma viúva}

Por esta proibiçào um ``Cohen Gadol'' fica proibido de chegar-se a uma
viúva, mesmo sem casar-se com ela. Ela está expressa em Suas palavras,
enal­tecido seja Ele, "E não profanará sua semente entre seu povo' '
(Levítico 21:15). • Esta é a explicação para isso. Um ``Cohen'' comum é
proibido de casar-sei .. através da proibição "Não
tomarão" (Ibid., 7), sendo que 'tomarão" sig­nifica tomar por esposa,
mas ele não fica sujeito à pena de açoitamento até que se chegue a ela,
como explicamos antes. Se ele se chegar a ela fora do casamento ---
embora ele seja proibido de fazê-lo e seja avisado para não fazê-lo,
pois além de tudo ele ainda prejudica a categoria sacerdotal dela ---
ele também não fica sujeito ao açoitamento, pois isso não está
explicitamente proibido. Contudo, no caso do ``Cohen Gadol'' as Escrituras
mencionam duas proibições: a primei­ra é ``A estas não tomará'' (Ibid.,
14), e a segunda é "E não profanará sua semen­te". a qual o proíbe
chegar-se a ela até mesmo fora do casamento.

A Guemará de Kidushin diz: "Rabá admite, com relação a um 'Co­hen Gadol'
com uma viúva, que se ele viver com ela sem casar-se, está sujeito ao
açoitamento, pois o Todo Misericordioso diz: `Não profanará sua
semente', e ele a terá profanado". Também está dito, no mesmo lugar: "Se
um `Cohen Gadol'\textsuperscript{329} com uma viúva ele estará sujeito
ao açoitamento duas vezes, uma por causa de `A estas não tomará' e outra
por causa de `Não profanará' ". O motivo pelo qual isto está mencionado
no caso de uma viúva é que ela foi especifica­mente proibida apenas para
um ``Cohen Gadol'', sendo permitida aos "Coha­nim", mas se ele se chegar a
ela, ele a profana e faz com que ela não possa mais se casar com um
``Cohen'' comum. Mas com relação às três mulheres --- a sa-

i

326 Sem levar em consideração se as d6qualificaçc \textgreater CS
ocorreram na ordem mencionada.


\begin{enumerate}
\def\labelenumi{\arabic{enumi}.}
\setcounter{enumi}{326}
\item
 
 1.e., preceitos negativos 158 a 161.
 
\item
 
 Com uma .zona", uma ``halalá'' ou uma mulher
 divorciada.
 
\item
 
 Se ele viver com uma viúva.
 
\end{enumerate}

ber, a mulher divorciada, a ``zoná'', e a ``halalá'' --- a lei é a mesma que
no caso do ``Cohen'' comum, ou seja, que cada uma delas está desde o
início proibida a todos os ``Cohanim'' e a única razão para a repetição da
proibição no caso do ``Cohen Gadol'' é a que foi explicada.

\section{Um ``cohen'' não pode entrar no santuário com o cabelo solto}

Por esta proibição os ``Cohanim'' ficam proibidos de entrar no San­tuário
com o cabelo solto tal como os que estão de luto, que não aparam nem
arrumam seus cabelos. Ela está expressa em Suas palavras, enaltecido
seja Ele, ``Não deixeis o cabelo de vossas cabeças solto'' (Levítico
10:6), que o Targum traduz por: ``Não deixeis crescer vosso cabelo'', e
que Hezekiel explica dizen­do: "Nem permitirão que suas madeixas
cresçam"\textsuperscript{330}. A mesma expressão é usada com relação aos
leprosos: ``O cabelo de sua cabeça deixará solto'' (Ibid., 13:45), a
respeito da qual a Sifrá diz: ``Ele deverá deixar seu cabelo crescer''. A
Sifrá diz ainda: " 'Não deixeis o cabelo de vossas cabeças solto': não
deixeis que ele cresça".

Com relação ao ``Cohen Gadol'', a proibição está repetida em Suas palavras
``Seu cabelo não deixará crescer'' (Ibid., 2 1:1 0). O objetivo da
repeti­ção é para que você não pense que Suas palavras a Elazar e a
Itamar "Não dei­xeis o cabelo de vossas cabeças solto" foram ditas
apenas por causa do falecimento\textsuperscript{33}I, e que um ``Cohen''
pode fazê-lo se não for em sinal de luto. Por isso Ele deixa claro que
no caso do ``Cohen Gadol'' ele deve usar seus ca­belos curtos em virtude
de suas funções sacerdotais.

Desobedecer esta proibição oficiando com o cabelo não cortado é punido
com a morte\textsuperscript{332}. Deduzimos que os de cabelo não cortado
estão incluí­dos entre os que estão sujeitos à morte através de Suas
palavras: ``Para que não morrais'' (Ibid. 10:6). Mas aquele que entrar no
Santuário com o cabelo com­prido e não oficiar está sujeito apenas ao
açoitamento, e não à morte.

\section{Os ``cohanim'' não podem usar vestes rasgadas ao entrar no santuário}

Por esta proibição os ``Cohanim'' ficam proibidos de entrar. no San­tuário
usando vestes rasgadas. Ela está expressa em Suas palavras "E vossos
ves­tidos não rompais" (Levítico 10:6), sobre as quais a Sifrá diz: " 'E
vossos vesti­dos não rompais': não rasgem suas vestes". Esta proibição
também está repeti­da com relação ao ``Cohen Gadol" em Suas palavras "E
suas vestes não rompe­rá" (Ibid., 21:10).

Você deve saber que não é permitido ao ``Cohen Gadol'' rasgar suas vestes
por um morto, mesmo que ele não esteja oficiando; e é por causa dessa
restrição adicional que a proibição é repetida. A Sifrá diz: " Seu
cabelo não


\begin{enumerate}
\def\labelenumi{\arabic{enumi}.}
\setcounter{enumi}{329}
\item
 
 Ezeq. 44:20.
 
\item
 
 Pela morte de seus irmãos Nadab e Abihu. Vide Levítico 10:1-7.
 
\item
 
 A morte pela mão dos Céus.
 
\end{enumerate}



deixará crescer e suas vestes não romperá' (Ibid.): nem mesmo por um
parente falecido, da maneira como as pessoas comuns rasgam e deixam seus
cabelos crescerem, quando estão de luto. De que maneira? O ``Cohen Gadol''
rasga suas vestes de baixo\textsuperscript{333}, mas as pessoas comuns
rasgam as de cima".

Quem oficiar usando vestes rasgadas também está sujeito à morte, porque
à mesma regra se aplica ao caso do cabelo comprido e ao das vestes
rasgadas. Mas quem entrar no Santuário dessa maneira transgride uma
proibi­ção. Apenas um ``Cohen Gadol'' está permanentemente proibido de
deixar cres­cer seus cabelos e de rasgar suas vestes, mesmo que ele não
entre no Santuário, e essa é a diferença em relação ao ``Cohen'' comum.

\section{Os ``cohanim'' não podem sair do santuário enquanto estiverem oficiando}

Por esta proibição os ``Cohanim'' ficam proibidos de sair do Santuá­rio
enquanto estiverem oficiando. Ela está expressa em Suas palavras "E da
en­trada da tenda da assinação não saireis" (Levítico 10:7). Esta
proibição também está repetida com relação ao ``Cohen Gadol'', através de
Suas palavras "Do San­tuário não sairá" (Ibid., 21:12).

A Sifrá diz: " 'Da entrada da tenda da assinação': eu poderia pensar que
um 'Cohen' comum não pode sair do Santuário, quer ele esteja oficiando
quer não; por isso as Escrituras dizem: 'Do santuário não sairá e não
profanará'. Isso mostra que é enquanto ele estiver oficiando. 'Porque o
azeite da unção do Eterno está sobre vós' (Ibid., 10:7): isto me diz que
apenas Aarão e seus fi­lhos, que foram ungidos com o Azeite da Unção,
estão sujeitos à morte se eles saírem enquanto estiverem oficiando; como
fico sabendo quanto a todos os `Cohanim', por todos os tempos? Porque as
Escrituras dizem: 'Porque o azeite da unçãb do Eterno está \emph{sobre
vós' .}

Você deve saber que no caso do ``Cohen Gadol'' há uma proibição adicional,
a saber, que ele não pode acompanhar o esquife de um parente pró­ximo.
Este é o significado literal das palavras das Escrituras "E do santuário
não sairá", e é dessa forma que o texto está explicado no segundo
capítulo de Sa­nhedrin, onde ele está citado como prova de que se um
parente próximo do ``Cohen Gadol'' morrer, ele não poderá acompanhar seu
ataúde. O mesmo tex­to ensina ainda que o ``Cohen Gadol'' pode oficiar no
dia do falecimento de um parente. As palavras dos Sábios em Sanhedrin
são as seguintes: " 'E do san­tuário não sairá, e não profanará', mas
qualquer outro 'Cohen' que não sair do Santuário o estará profanando"
--- referindo-se a um ``Cohen'' comum a quem não é permitido oficiar
porque nesse caso ele é um ``onen''. A advertência con­tra oficiar durante
o luto se deriva do texto em questão. Esta regra de que, ao contrário de
um ``Cohen Gadol'', um ``Cohen'' comum não pode oficiar en­quanto ele for um
``onen'' está explicada no final de Horayot.


Dessa forma, foi deixado claro que Suas palavras ``E não profanará''\\
são uma negativa, e não uma proibição. Isto é, é simplesmente para
declarar\\
que ele não estará profanando o Santuário se oficiar, embora sendo um
``onen''.\\
Conseqüentemente, o significado literal do versículo é que Suas palavras
``E não profanará'' são a razão para a proibição anterior, da
seguinte for-


333. Pois o rasgão nesse local não é tão feio.

ma: ``E do santuário não sairá'', a fim de ``não profanar o santuário''. Mas
nas duas teorias segue-se que esta proibição não deve ser contada como
um precei­to negativo separado, como ficará claro para todo aquele que
compreendeu os Fundamentos\textsuperscript{334} pré-estabelecidos para a
execução deste trabalho.

Também foi deixado claro que estas três proibições --- a saber, "Seu
cabelo não deixará crescere suas vestes não romperá" (Ibid., 2 1:1 O) e
``E do santuário não sairá'' (Ibid., 1 2) --- estão repetidas com relação
ao "Cohen Ga­dol" com uma finalidade específica, assim como as
proibições que o impedem de chegar-se a uma mulher divorciada, a uma
``zoná'' ou a uma .halalá". Ficou também claro que o
conteúdo dessas três proibições é idêntico ao que foi proi­bido por Suas
palavras "Não deixeis o cabelo de vossas cabeças solto, e vossos
vestidos não rompais... e da entrada da tenda de assinação não saireis"
(Ibid. 10:6-7). E também que nosso Mestre Moisés, a paz esteja com ele,
as deu a co­nhecer a Elazar e a Itamar dizendo: "As coisas que lhes são
proibidas não se tornam permitidas por causa de seu luto pela grande
perda; vocês ainda conti­nuam proibidos de deixar crescer seus cabelos,
de rasgar suas vestes, e de dei­xar o Santuário enquanto estiverem
oficiando". A razão pela qual a proibição está repetida com relação ao
``Cohen Gadol'' é para explicar que ela se aplica apenas enquanto durar o
ofício e que só nesse caso eles estarão sujeitos à mor­te, como se pode
ver pelo fato de que para explicar Suas palavras "Da entrada da tenda de
assinação não saireis" os Sábios citaram Suas palavras "E do san­tuário
não sairá". E embora cada proibição, repetida com relação ao "Cohen
Gadol", amplie assim o seu alcance, como explicamos, fica claro para
todos aqueles que entenderam nossa Introdução que essas proibições não
são.precei­tos adicionais, uma vez que o objetivo das
Escrituras é de não permitir-lhe fa­zer qualquer uma dessas coisas
enquanto estiver comprometido com os ofícios. Você deve compreender
isto.

\section{Um ``cohen'' comum não pode tornar-se impuro por nenhuma pessoa morta
a não ser pelas que estão determinadas na torah}

Por esta proibição um ``Cohen'' comum fica proibido de tornar-se impuro
por qualquer pessoa morta a não ser pelos parentes especificados na
Torah. Ela está expressa em Suas palavras, enaltecido seja Ele, "O
'Cohen' , por um morto entre seu povo, não se faça impuro" (Levítico
21:1).

Aquele que infringir esta proibição, fazendo-se impuro por qualquer
pessoa morta que não esteja entre os cinco parentes pelos quais ele deve
guar­dar luto\textsuperscript{335} estará sujeito ao açoitamento. A
proibição não se aplica às mulhe­res. A Tradição diz: " 'Filhos de
Aarão' (Ibid.), mas não as filhas de Aarão".

334 Ver o oitavo Fundamento. 335. Ver o preceito positivo 37.


\section{Um ``cohen gadol'' não deve ficar sob o mesmo teto que um cadáver}

Por esta proibição um ``Cohen Gadol'' fica proibido de permanecer sob o
mesmo teto que um cadáver, mesmo que seja de um dos
obrigatórios\textsuperscript{336}, ou seja, seus parentes. Esta
proibição está expressa em Suas palavras "E às al­. mas mortas não se
chegará" (Levítico 2 1:1 1).

Aquele que se fizer impuro dessa forma, mesmo que seja pelo cadá­ver de
seu pai ou de sua mãe, será punido com o açoitamento.

168 UM ``COHEN GADOL'' NÃO PODE FAZER-SE IMPURO POR NENHUMA PESSOA MORTA

Por esta proibição o ``Cohen Gadol'' fica proibido de fazer-se impu­ro por
qualquer pessoa morta, seja de que forma for, quer seja por contato ou
por carregar seu corpo. Ela está expressa em Suas palavras "Por seu pai,
e por sua mãe, não se fará impuro" (Levítico 2 1:1 1).

Possivelmente você poderá pensar que esta proibição e a preceden­te são
uma só e que Suas palavras "Por seu pai, e por sua mãe, não se fará
impu­ro" são apenas uma explicação. Mas não é esse o caso: elas são duas
proibições separadas. Como diz a Sifrá: "Ele será considerado culpado
por "E às almas mor­tas não se chegará' e também por 'Por seu pai...' ".
E um ``guezerá shavá''\textsuperscript{337} faz ainda com que isto também
se aplique a um ``Cohen'' comum: "Assim co­mo o 'Cohen Gadol' que está
proibido por qualquer pessoa morta é culpado duas vezes, assim também o
'Cohen' comum, que está proibido de fazer-se im­puro por qualquer pessoa
morta está sujeito ao castigo pela violação de 'E às almas mortas... ".
Contudo, não consideramos que seja assim no caso no "Co­hen"
comum\textsuperscript{338} pelo motivo explicado no Segundo Fundamento
mas\textsuperscript{339} con­tamos essas duas proibições como preceitos
diferentes porque elas estão ex­pressas em dois versículos diferentes e
porque o significado de "E às almas mor­tas..." não é o mesmo que o de
"Por seu pai...", pois aqueles que transcreve­ram a Tradição disseram:
"Ele está sujeito a castigo por 'E às almas mortas...' e também por 'Por
seu pai...' ".

\section{Os levitas não podem adquirir um pedaço da terra de Israel}

Por esta proibição toda a tribo de Levi está proibida de tomar um pedaço
da terra de Israel. Ela está expressa em Suas palavras "Os sacerdotes-


\begin{enumerate}
\def\labelenumi{\arabic{enumi}.}
\setcounter{enumi}{335}
\item
 
 I.e., um dos parentes cujo funeral ele seria normalmente obrigado a
 acompanhar.
 
\item
 
 Uma ``expressão similar'', ou seja, uma analogia entre duas leis,
 estabelecida com base na con­gruência verbal dos textos das
 Escrituras.
 
\item
 
 Não consideramos que ele tenha transgredido dois preceitos diferentes.
 
\item
 
 No caso do ``Cohen Gadol''.
 
\end{enumerate}

levitas, com ou sem defeitos, de toda a tribo de Levi, não terão parte
nem he­rança com Israel" (Deuteronômio 18:1).

\section{Os levitas não podem receber nenhuma parte da pilhagem da conquista
da terra de Israel}

Por esta proibição toda a tribo de Levi fica também proibida de re­ceber
lima parte da pilhagem obtida na conquista da terra de Israel. Ela está
expressa em Suas palavras "Os sacerdotes-levitas, com ou sem defeitos,
de to­da a tribo de Levi, não terão parte nem herança com Israel"
(Deuteronômio 18:1), a respeito das quais o Sifrei diz: " 'Parte' da
pilhagem; 'herança' da terra".

Você poderia talvez objetar e perguntar-me: "Por que você conside­ra
essas duas proibições --- a de pegar uma parte da terra e a de pegar uma
parte da pilhagem --- como dois preceitos separados? Com certeza essa é
uma ``lav shebikhlalut''." e como tal ela é considerada
como um único preceito, de acordo com o princípio que você estabeleceu"

A resposta é que esta proibição está de fato dividida em duas pelas
palavras ``Mas herança não terão'' (Ibid., 2), havendo assim duas
proibições ex­pressas de maneira diferente: a primeira, que os proíbe de
pegar algo da pilha­gem, e "Os sacerdotes-levitas... não terão parte"; e
a segunda, que os proíbe de ter uma parte da Terra, e "Herança não
terão".

Esta proibição dupla aparece novamente com relação aos ``Cohanim'' em Suas
palavras, enaltecido seja Ele, a Aarão .*Mas herança
não terão em suas terras, nem terão parte no meio de seus irmãos"
(Números 18:20) que são in­terpretadas da seguinte forma: " 'Herança não
terão em suas terras': isto é, quan­do eles dividirem as terras; 'Nem
terão parte no meio de seus irmãos': isto é, parte .da pilhagem".

Caso você pense que estas duas proibições, relativas especificamen­te
aos ``Cohanim'', constituem dois preceitos que deveriam ser acrescentados,
você deve saber que uma vez que esta proibição está expressa em termos
gené­ricos, referindo-se a ``toda a tribo de Levi'', os Cohanim" já estão
incluídos nela, e que ela foi repetida com relação aos ``Cohanim'' apenas
para dar maior ênfase. Nos casos como este, em que uma lei de aplicação
geral está repetida fazendo referência especificamente a um determinado
caso, o objetivo da re­petição é apenas o de enfatizar e complementar a
lei que pode não ter sido for­mulada por completo na primeira proibição.

Se contássemos Suas palavras a Aarão "Mas herança não terão em suas
terras, nem terão parte no meio de seus irmãos", em adição a Suas
pala­vras "Os Sacerdotes-levitas... não terão parte... " , nós
obrigatoriamente deve­ríamos ter contado, por analogia, as proibições
que impedem o "Cohen Ga-dor de chegar-se a uma mulher divorciada, a uma
``halalá'' e a uma ``zoná'', como três preceitos adicionais, além daqueles
que se aplicam da mesma forma ao "Cohen Gadol---  e aos
``Cohanim'' comuns.


Entretanto, caso alguém insista em que realmênte deveríamos ter feito



isso, nós lhe responderemos o seguinte:\textsuperscript{34}' um "Cohen
Gadol ' que se tivesse chegado a uma mulher divorciada estaria
forçosamente sujeito a dois castigos, um por ser ele um ``Cohen'', a quem
uma mulher divorciada está proibida, e outro por ser um ``Cohen Gadol'', a
quem ela está igualmente proibida sob os termos de outro preceito. Mas
está explicado no Tratado Kidushin que ele está sujeito a castigo uma
única vez; conseqüentemente, apenas a proibição geral deve ser contada e
o objetivo de qualquer outra proibição, menos genérica, do mesmo tipo de
conduta serve apenas para ensinar-nos alguma norma espe­cífica ou para
complementar o enunciado da lei, como explicamos ao tratar do preceito
negativo 161.

A essa categoria de leis pertence também a proibição que ordena aos
``Cohanim'': "Não farão calva em sua cabeça, nem a sua barba rasparão e em
sua carne não farão incisões" (Levítico 21:5). Essas três proibições já
foram im­postas a todo o povo de Israel através de Suas palavras "Não
cortareis o cabelo de vossa cabeça em redondo" (Ibid., 19:27); "Não vos
fareis raspar a cabeça" (Deuteronômio 14:1); "E incisões por um morto
não fareis em vossa carne" (Levítico 19:28). Elas estão repetidas com
relação aos "Cohanim' apenas com a finalidade de complementar o
enunciado da lei, como foi deixado claro no final de Macot, onde as
normas desses três preceitos estão explicadas. Se elas fossem preceitos
separados a serem aplicados aos ``Cohanim'' e não simples­mente
complementos do enunciado da lei, um ``Cohen'' estaria sujeito ao
açoi­tamento duas vezes por cada uma das transgressões, uma vez por ser
israelita, e outra por ser ``Cohen''. Mas esse não é o caso. Na realidade,
ele está sujeito a um único açoitamento, como o resto de Israel, como
está explicado no lugar apropriado.


Este princípio deve ser compreendido na sua totalidade.


\section{Não arrancar nosso cabelo pelos mortos}

Por esta proibição somos proibidos de arrancar nossos cabelos por causa
dos mortos, como fazem os loucos. Ela está expressa em Suas palavras
"Não vos fareis raspar a cabeça entre os olhos por causa de um morto"
(Deute­ronômio 14:1).

Esta proibição está repetida, com relação aos ``Cohanim'', em Suas
palavras ``Não farão calva em sua cabeça'' (Levítico 21:5), a fim de
complemen­tar o enunciado da lei. Poderíamos argumentar, em virtude das
palavras ``entre os olhos'', que a proibição se aplica apenas a raspar a
testa, por isso Ele diz: ``Não farão calva em sua cabeça'', mostrando
assim que a proibição se aplica a toda a cabeça assim como à testa. E se
Ele tivesse dito apenas ``Não farão calva em sua cabeça'' poderíamos
argumentar que a proibição se aplica tanto em re­lação aos mortos como
aos outros, por isso Ele explica dizendo 'por causa de um morto".

Todo aquele que deixar em sua cabeça um buraco de calvície do ta­manho
de um grão de feijão, por ter arrancado seus cabelos por causa de um
morto, seja ele ``Cohen'' ou israelita, será punido com o açoitamento, uma
vez por cada buraco de calvície. Aqui novamente
o objetivo da repetição com relação aos
"Cohanim., nas palavras "Nem a sua barba rasparão e em
sua carne não farão incisões" (Le­vítico 2 1 :5), é para complementar a
Iei do preceito, como está explicado no final de Macot.

\section{Não comer um animal impuro}

Por esta proibição somos proibidos de comer um animal impuro, doméstico
ou selvagem. Ela está expressa em Suas palavras "Estes não come­reis dos
que ruminam...: o camelo, e a lebre, e o coelho... e o porco"
(Deutero­nômio 14:7-8).

As Escrituras não proibiram explicitamente que se coma outros ani­mais
impuros, mas pelas Suas palavras "E todo animal de casco fendido, e que
tem a unha separada em dois de cima até embaixo, e que rumina, entre os
ani­mais, esses comereis" (Ibid.,
14:6).4. sabemos que todos os que
não possuam essas duas características não são alimento permitido.
Contudo, este é o caso de um preceito negativo derivado de um preceito
positivo o qual, de acordo com o que foi estabelecido, possui a força de
um preceito positivo, e é um princípio aceito que a infração de um
preceito negativo deste tipo não é puni­da com o açoitamento. Mas
através de um ``kal vahomer''\textsuperscript{343} deduzimos que estamos
proibidos de comer outros animais impuros ou selvagens e que esta­mos
sujeitos ao açoitamento por comê-los; portanto, se comer um porco ou um
camelo, que tem uma das duas características dos animais puros, é
puní­vel com .o açoitamento, conclui-se que comer outros animais
domésticos ou selvagens que não tenham nenhuma dessas características é
punível com o açoitamento.

Agora observem as palavras da Sifrá a este respeito: " 'Esses come­reis'
--- 'esses' podem ser comidos, mas os animais impuros não podem ser
co­midos. Isso me apresenta apenas um preceito positivo; de que forma
fico sa­bendo que também há um preceito negativo? Pelas palavras das
Escrituras: 'Mas estes não comereis dos que ruminam' (Levítico 1 1
:4-7). Isso me diz apenas que `esses' são alimentos proibidos; como fico
sabendo com relação a todos os ou­tros animais impuros? Eu o concluo por
analogia: se esses, que têm uma das características do animal puro, são
alimento proibido, não é lógico pensar que um preceito negativo nos
proíbe de comer outros animais impuros que não tenham nenhuma dessas
características? Dessa forma fica estabelecido que o camelo, a lebre, o
coelho e o porco são\textsuperscript{344} pelas Escrituras, enquanto que
ou­tros animais impuros o são por força de um `kal vahomer'. Fica também
estabe­lecido que o preceito positivo vem das Escrituras enquanto que o
preceito ne­gativo relativo a eles é derivado de um `kal vahomer' ".

Entretanto, esse ``kal vahomer'' é meramente utilizado para esta­belecer
algo que as Escrituras deixaram claro, como no caso da
filha\textsuperscript{345}, que explicaremos no local apropriado.
Portanto, aquele que comer o equivalente ao tamanho de uma oliva de
carne de um animal impuro doméstico ou selva­gem, estará sujeito ao
açoitamento, de acordo com a Torah. Você deve com­preender isso.


\begin{enumerate}
\def\labelenumi{\arabic{enumi}.}
\setcounter{enumi}{341}
\item
 
 Ver o preceito positivo 149.
 
\item
 
 ``Com toda razão''.
 
\item
 
 I.e., são proibidos explicitamente pelas Escrituras.
 
\item
 
 Ver o preceito negativo 336.
 
\end{enumerate}



\section{Não comer um peixe impuro}

Por esta proibição somos proibidos de comer um peixe impuro. Ela está
expressa em Suas palavras, relativas a esses tipos de peixe, "E
abominação serão para vós; de sua. carne não comereis" (Levítico 11:11).

Comer o equivalente ao tamanho de uma oliva de sua carne é puni­do com o
açoitamento.

\section{Não comer nenhuma ave impura}

Por esta proibição somos proibidos de comer uma ave impura. Ela está
expressa em Suas palavras, relativas a essas espécies, "E isto
abominareis das aves: não serão comidas" (Levítico 11:13).

Comer o equivalente ao tamanho de uma oliva de sua carne é puni­do com o
açoitamento.

As normas deste e dos dois preceitos anteriores estão explicadas no
terceiro capítulo de Hulin.

\section{Não comer nenhum inseto alado}

Por esta proibição somos proibidos de comer qualquer inseto ala­do, tal
como moscas, abelhas, vespas e insetos similares. Ela está expressa em
Suas palavras no Deuteronômio "E todo animal rastejante alado, impuro
será para vós: não serão comidos" (Deuteronômio 14:19), sobre as quais o
Sifrei diz: " 'Todo animal rastejante alado etc' é um preceito
negativo".

Comê-los será punido com o açoitamento.

\section{Não comer nada que rasteje sobre a terra}

Por esta proibição somos proibidos de comer qualquer coisa que ras­teje
sobre a terra, tal como vermes, escaravelhos e coleópteros, que são
chama­dos ``coisas que rastejam sobre a terra''. Ela está expressa em Suas
palavras, enal­tecido seja Ele, "E todo o animal rastejante que se move
sobre a terra é abomi­nação, não será comido" (Levítico 11:41).


Comer qualquer uma dessas coisa será punido com o açoitamento.



\section{Não comer nenhuma criatura rastejante que se reproduza em matéria deteriorada}

Por esta proibição somos proibidos de comer qualquer criatura
ras­tejante que se reproduza em matéria deteriorada ou desfeita, mesmo
que ela não pertença a nenhuma espécie característica e não seja
resultante da união de um macho e de uma fêmea. Esta proibição está
expressa em Suas palavras "E não façais impuras vossas 
almas com todo animal rastejante que se
arrasta sobre a terra" (Levítico 11:44), sobre as quais a Sifrá diz: "
'Todo animal raste­jante que se arrasta sobre a terra' --- ainda que ele
não se reproduza".

Esta é a diferença entre Suas expressões "Todo o animal rastejante que
\emph{se move} sobre a terra" (Ibid., 41) e "Todo animal rastejante que
\emph{se arrasta} sobre a terra". A criatura que \emph{se move} tem o
poder de gerar seres semelhantes e se multiplica sobre a terra, enquanto
que as criaturas que \emph{se arrastam} se re­produzem em matéria
deteriorada ou desfeita e não têm o poder de gerar seres como elas.

Também neste caso comê-las é punido com o açoitamento.

\section{Não comer criaturas vivas que se reproduzam em sementes ou frutas}

Por esta proibição somos proibidos de comer criaturas vivas que se
reproduzam em sementes ou frutas uma vez que elas tenham saído da
semente ou da fruta e se movam sobre elas; mesmo se as encontrarmos
depois em nossa comida somos proibidos de comê-las e aquele que o fizer
será punido com o açoitamento. Esta proibição está expressa em Suas
palavras "Todo o animal ras­tejante que se move sobre a terra, não os
comereis porque abominação são eles" (Levítico 11:42), sobre as quais a
Sifrá diz: "Isto inclui também os que estive­ram sobre a terra e que
tornaram a entrar\textsuperscript{346}".

\section{Não comer nenhuma espécie de criatura rastejante}

Por esta proibição somos proibidos de comer qualquer espécie de\\
criatura rastejante, seja ela um ser alado, ou rasteje na água ou sobre
a terra.\\
Esta imbibição está expressa em Suas palavras, enaltecido seja Ele, "Não
tomeis\\
abomináveis vossas almas com nenhum animal rastejante que se move e
não\\
vos façais impuros com eles e não sejais impuros por eles" (Levítico
11:43).\\
Esta é uma proibição separada, cuja transgressão é punida com o açoitamento
e ela é semelhante a um ``issur colei''. De acordo com isso,
aquele que\\
comer qualquer tipo de animal que se arrasta sobre a terra é punido com
o açoitamento
duas vezes: uma por causa da proibição "E todo o animal
rastejante\\
que se move sobre a terra é abominação, não será comido" (Ibid., 41) e
outra\\
por causa da proibição "Não torneis abomináveis vossas almas com
nenhum\\
animal rastejante que se move". Da mesma forma, aquele que comer
qualquer\\
tipo de animal rastejante alado será punido com o açoitamento duas
vezes: uma\\
por causa da proibição "E todo animal rastejante alado, impuro será para
vós:\\
não serão comidos" (Deuteronômio 14:19), e outra por causa da proibição
"Não\\
torneis abomináveis vossas almas etc". Além disso, aquele que comer um
inseto
que tenha asas e que também se arraste pelo chão e que, dessa forma,
seja\\
ao mesmo tempo um animal rastejante alado e um animal que se move pelo\\
chão, será punido com o açoitamento quatro vezes: Se além disso, o
inseto


346 Tornaram a entrar na semente ou na fruta.


também se arrastar na água,. comê-lo será punido seis vezes: a quinta
vez por ser um peixe impuro, .a respeito do qual está dito: "De sua
carne não come­reis" (Levítico 11:11) e a sexta por causa de "Não tomeis
abomináveis vossas almas etc". Esta última abrange também tudo o que se
arrasta dentro da água, uma vez que não temos nenhum versículo proibindo
que se coma animais que rastejem na água a não ser o que diz: "Não
tomeis abomináveis vossas almas com nenhum animal rastejante que se
move". De acordo com esses princípios, a Guemará de Macot diz: "Se
alguém comer uma enguia ele está sujeito ao açoi­tamento por quatro
vezes; se comer uma formiga, por cinco vezes; se comer um marimbondo,
por seis vezes".

Entretanto, todas as vezes que ouvi alguém interpretar esse trecho, ou
seja, ``Se alguém comer uma enguia etc'' e que li os livros que consultei
a esse respeito, a interpretação foi invariavelmente a acima mencionada.
E ela é uma interpretação incorreta, sustentável apenas se os princípios
verdadeiros cla­ramente estabelecidos no Talmud fossem derrubados.
Porque se você exami­nar o que dissemos acima, você vai chegar à
conclusão de que, baseado no pre­ceito negativo "Não tomeis abomináveis
vossas almas com nenhum animal ras­tejante que se move", eles determinam
o açoitamento por três vezes; mas está demonstrado que esse ponto de
vista está errado, pois em circunstância algu­ma alguém pode ser punido
duas vezes por causa de uma proibição, como está explicado na Guemará de
Hulin e como nós próprios já explicamos em diver­sas ocasiões e como
ilustraremos com exemplos a seguir.

A verdadeira interpretação, que não admite dúvidas e que não vai
enganá-lo é a seguinte. Aquele que comer uma criatura alada que rasteja
tanto na água quanto sobre a terra estará sujeito apenas a três castigos
por açoitamento: um por ser uma criatura rastejante alada, a cujo
respeito há uma proibição especí­fica; outro por ser uma criatura que
rasteja sobre a terra contra a qual também há uma proibição específica;
e outro por causa de "Não tomeis abomináveis vos­sas almas", que proíbe
comer criaturas que rastejam na água, abrangidas pela ex­pressão "Nenhum
animal rastejante que se move" que está na proibição "Não tor­neis
abomináveis vossas almas". Mas se alguém comer um animal que se arrasta
apenas sobre a terra, ou uma criatura rastejante alada, ou uma que se
arrasta ape­nas na água, o castigo em cada caso será o açoitamento pela
desobediência refe­rente a cada proibição, sendo que no último caso a
proibição está expressa em "Não tomeis abomináveis vossas almas com
nenhum animal rastejante que se mo­ve". O simples fato de que esta
proibição também abrange a criatura que rasteja sobre a terra não nos
sujeita a dois açoitamentos por uma criatura desse tipo, pois ainda que
houvesse mil proibições expressas relativas às criaturas que rastejam
sobre a terra, ainda assim seríamos açoitados uma única vez por tê-las
desobede­cido, pois todas elas seriam repetições de uma única coisa.
Mesmo se Ele tivesse dito mil vezes: "Não comereis insetos que rastejam
sobre a terra", ``Não comerás insetos que se movem sobre a terra'',
estaríamos sujeitos a um único açoitamento. Vocês já viram aqueles que
apresentaram esta doutrina incorreta afirmar que aquele que usar
``Shaatnez'' (i.e„ vestes feitas com lã e linho) estará sujeito a dois
açoita­mentos porque isso está expressamente proibido por dois preceitos
negativos? Eu nunca os vi defender tal opinião, ao contrário, eles
achariam estranho se outra pessoa o fizesse. Todavia, eles não acham
estranha sua própria opinião quando dizem que o caso de uma criatura que
rasteja sobre a terra ou de uma criatura rastejante alada é punível com
dois açoitamentos: um pela proibição da criatura comida, e outro por
causa da proibição: ``Não tomeis abomináveis vossas almas''! Isso não
passaria desapercebido nem mesmo a um surdo-mudo. •

Voltarei agora ao assunto que comecei a explicar.


Se acontecesse que um inseto nascesse numa determinada semente ou fruta
e saísse dela, aquele que o comesse estaria sujeito ao açoitamento uma
vez, mesmo que esse inseto nunca chegasse a tocar o chão, porque há uma
proi­bição específica com relação a ele, como explicamos ao tratar do
preceito prece­dente. Se ele caísse no chão e aí se movesse, aquele que
o comesse estaria sujeito a dois açoitamentos, um por causa de "De todo
o animal rastejante que se move sobre a terra, não os comereis porque
abominação são eles" (Levítico 11:42), e um por causa do "E não façais
impuras vossas almas com todo animal rastejante que se arrasta sobre a
terra" (Ibid., 44). Se, além disso, o inseto fosse do tipo que se
feproduz, estaria sujeito a três açoitamentos: dois pelas razões acima
mencio­nadas e o terceiro por causa de "E todo o animal rastejante que
se move sobre a terra é abominação, não será comido" (Ibid., 41). Se
além de tudo isso o inseto fosse alado, estaria sujeito a um quarto
açoitamento por causa de • 'E todo animal rastejante alado, impuro será
para vós: não serão comidos" (Deuteronómio 14:19). E caso o inseto, além
de ser alado, ainda se arrastasse na água, como fazem muitas espécies,
ele sujeitaria a cinco açoitamentos, sendo o quinto por causa de uma
criatura que se arrasta na água, que está abrangida na proibição
expressa em Suas palavras "Não tomeis abomináveis vossas almas com
nenhum animal rastejante que se move e não vos façais impuros com eles e
não sejais impuros por eles". E se, finalmente, ainda por cima a
criatura gerada fosse também uma ave, sujeita­ria a seis açoitamentos,
por causa de "E isto abominareis das aves: não serão co­midas, porque
elas são uma abominação" (Levítico 11:13).

Você não deve ficar surpreso de que um pássaro nasça de uma fruta podre,
já que freqüentemente vemos pássaros maiores do que uma pequena noz que
se geraram de alimentos decompostos. Não deve achar estranho que uma
criatura seja ambos uma ave impura e um animal rastejante alado, pois
não é impossível que uma criatura possua ao mesmo tempo as qualidades e
caracte­rísticas tanto de uma ave como de um animal rastejante alado.
Dessa forma, você vai encontrar comentaristas anteriores contando entre
esses seis açoita­mentos um em virtude de uma criatura que é ao mesmo
tempo um peixe impu­ro e um animal que se arrasta na água. Isso é
correto e eu não vou contestá-lo porque é possível que uma criatura seja
ambos um peixe e um animal que se arrasta na água, assim como é possível
que uma criatura seja ambos uma ave e uma animal que se arrasta na água,
ou uma ave e um animal rastejante alado, de forma que haveria quatro
punições por comê-lo. A enguia é ambos uma ave, um animal rastejante
alado, e uma criatura que se arrasta tanto na terra como na água;
conseqüentemente ela sujeita a quatro açoitamentos. A formiga
men­cionada é um inseto alado, gerado de frutas podres, que não se
reproduz e acar­reta.um\textsuperscript{347} por ser
uma criatura rastejante que sai do alimento, um por ser uma criatura que
rasteja sobre a terra, um por ser uma criatura que se move sobre a
terra, um por ser uma criatura rastejante alada e um por ser uma
criatura que se arrasta na água. A vespa, que também nasce de matéria em
decomposição é, além disso, uma ave e uma criatura rastejante alada. .

Não é impossível que a vespa, a formiga e outros tipos de criaturas
voadoras e rastejantes nasçam de materiais ou frutos em decomposição,
mas as massas, que desconhecem as ciências naturais, não pensam assim.
Como eles vêem que a maioria dos seres animados são gerados através da
união de um macho e de uma fêmea, eles imaginam que isso deve ser assim
com relação a todos os seres vivos.

PRECEITOS NEGATIVOS

Lembre e compreenda este assunto porque ele e .uma
palavra dita apropriadamente".

Eu expliquei a vocês os princípios através dos quais você pode deci­dir,
depois de analisá-los, que por comer um inseto a pessoa está sujeita a
'um determinado número de açoitamentos, enquanto que por comer um outro
ela está sujeita a um número menor de açoitamentos.

Pelos versículos citados ficará claro para você que aquele que co­mer um
inseto vivo inteiro de qualquer tamanho\textsuperscript{34}H e não
devemos perguntar se o inseto era do tamanho de uma oliva. Mesmo se
alguém comer um mosqui­to, ele estará sujeito a três açoitamentos: um
por ser uma criatura que rasteja sobre a terra, um por ser uma criatura
que se move sobre a terra, e um por ser uma criatura rastejante alada.

Também nos dizem: "Aquele que prender suas fezes peca contra 'Não
tomareis abomináveis Vossas almas'. Aquele que beber num copo de chifre
de cirurgião no qual ele. recebe o sangue peda contra 'Não tornareis
abomináveis vossas almas' " Isto se aplica também a comer imundícies ou
coisas repulsivas ou tomar líquidos pelos quais a maioria das pessoas
tenha aversão; todas essas coisas são proibidas. Entretanto, nào se está
sujeito ao açoitamento por essas coisas, uma vez que o significado
literal do texto se refere apenas aos seres ras­tejantes. Mas elas
acarretam um ``macat mardut''.

Assim, ficou claro, através de toda esta discussão. que o versículo 'Não
tomareis abomináveis vossas almas" é a única fonte de onde deduzimos a
proibição de comer criaturas que se arrastam na água, já que não há
nenhum precdto negativo especifico que Ne refira a isso. a não ser este
Isto deve ser entendido)

\section{Não comer ``nebelá''}

Por esta proibição somos proibidos de comer um animal que mor­reu por
si. Esta proibição está expressa em Suas palavras "Não comereis ne­nhum
animal que morreu por si (nebelá) (Deuteronômio 14:21).

Comer a quantidade correspondente ao tamanho de uma oliva de
.nebelá" é punido com o açoitamento.

\section{Não comer ``terefá''}

Por esta proibição somos proibidos de comer " terefá• • . Ela está
ex­pressa em Suas palavras, ``Carne dilacerada no campo mio comereis''
(Êxodo 22:30).

O significado literal do texto é o que está exposto na Mekhiltá: "As
Escrituras falam dos casos comuns e mencionam os lugares onde os
ani­mais são provavelmente dilacerados". Contudo, tradicionalmente o
versículo também é interpretado da seguinte forma: "Qualquer carne que
estiver no campo é `terefá', portanto não deveis comê-la". Isso
significa que qualquer carne que tenha sido levada para fora de seus
limites legais se torna ``terefá''; portanto, se a carne dos Sacrifícios
Mais Sagrados for levada para fora do Cam­po do Santuário, ou a carne
dos Sacrifícios Menos Sagrados para fora dos mu­ros de Jerusalém, ou a
carne do sacrifício de ``Pessah'' para tora da com

348. Estará sujeito a ser castigado.

panhia\textsuperscript{349}, ou se a cria puser sua pata dianteira para
fora\textsuperscript{35()}, como esá explica­do no quarto capítulo de
Hulin, em cada um destes casos a carne é chamada ``terefá'' e aquele que
comer o equivalente ao tamanho de uma oliva dela estará sujeito ao
açoitamento, de acordo com as Escrituras. Da mesma forma, carne
arrancada de um animal vivo é chamada ``terefá'' e aquele que a comer
estará sujeito ao açoitamento. A Guemará de Hulin diz: " 'Carne
dilacerada do campo não comereis' se refere à carne de um animal vivo e
também à carne de um animal que foi dilacerada por animais selvagens".

As proibições deste preceito e do precedente estão repetidas com relação
aos ``Cohanim'' em Suas palavras "Do animal que morre por si ou
dila­cerado por outros animais não comerá para não impurificar-se por
causa de­les" (Levítico 22:8). O motivo pelo qual a proibição está
repetida no caso deles é que, uma vez que as Escrituras lhes ordenam que
comam de um pássaro de Sacrifício de Pecado que foi\textsuperscript{351}
por ``meliká''\textsuperscript{352} --- um método que, se fosse usado para
abater carne comum certamente não seria válido, pois ele transfor­ma a
carne em ``nebelá'' --- poderia ocorrer-nos que eles podem comer, como
alimento comum, até mesmo ``meliká'' ou um que tenha sido degolado de
ma­neira inadequada; por isso as Escrituras explicam que eles continuam
a ser co­mo os israelitas no que se refere à advertência contra comer
``nebelá'' ou "tere­fá". Esta é a explicação dada pelos Sábios, que também
mencionam este versí­culo com relação a outra lei, que não é relevante
neste trabalho.

Mas o animal doméstico ou selvagem que comprovadamente tiver se tornado
``terefá'', de acordo com um dos métodos aceitos de interpretação, é
alimento proibido, mesmo que ele tenha sido abatido segundo os rituais;
e aquele que o abater segundo os rituais e comer sua carne será punido
com o açoitamento, de acordo com a lei Rabínica.

As coisas que transformam em ``terefá'' estão explicadas no terceiro
capítulo de Hulin. As normas deste e dos nove preceitos precedentes
estão ex­plicadas nesse mesmo capítulo, no último capítulo de Macot, e
no primeiro ca­pítulo de Bekhorot. .

\section{Não comer um membro de um animal vivo}

Por esta proibição somos proibidos de comer um membro de uma' criatura
viva, ou seja, cortar um membro de um animal vivo e comer o equi­valente
ao tamanho de uma oliva, na sua condição natura1\textsuperscript{353}. E
ainda que não haja mais do que uma porção ínfima de carne nele, aquele
que a comer será punido com o açoitamento. A proibição está expressa em
Suas palavras ``Não comerás enquanto a alma está junto à carne''
(Deuteronômio 12:23), a respei­to das quais o Sifrei diz: " 'Não comerás
enquanto a alma está junto à carne' se refere a um membro de uma
criatura viva". O versículo é interpretado da mesma maneira na Guemará
de Hulin, onde lemos também: "Aquele que


\begin{enumerate}
\def\labelenumi{\arabic{enumi}.}
\setcounter{enumi}{348}
\item
 
 O grupo que se reuniu para comer o Sacrifício de ' l'essah".
 
\item
 
 E depois a recolher novamente, durante o trabalho de parto. '
 
\item
 
 Que foi abatido.
 
\item
 
 Ver o preceito negativo 112.
 
\item
 
 I.e., junto com as veias e os tendões (embora em outras proibições as
 veias e os tendões não cheguem ao tamanho de uma oliva).
 
\end{enumerate}



comer um membro de uma criatura viva, e também carne de urna criatura
vi­va, é culpado duas vezes". A razão disso é que há duas proibições,
das quais a primeira é ``Não comerás enquanto a alma está junto à carne'',
que proíbe comer um membro, e a segunda é "Carne dilacerada no campo não
come­reis (Êxodo 22:30), que proíbe comer a carne de uma criatura viva,
como explicamos.

Esta proibição aparece novamente, sob outra forma, em Suas pala­vras a
Noé, proibindo-o de comer um membro de uma criatura viva: "Porém, a
carne com sua alma e seu sangue, não comereis" (Gênesis 9:4).

\section{Não comer ``guid hanashé''}

Por esta proibição somos proibidos de comer os tendões encolhi­dos. Ela
está expressa em Suas palavras "Por isso não comem os filhos de Israel o
tendão encolhido" (Gênesis 32:33). Todo aquele que comer o tendão todo,
ainda .que ele seja pequeno, ou o equivalente ao tamanho de uma oliva,
será punido com o açoitamento.


As normas deste preceito estão explicadas no sétimo capítulo de Hulin.


\section{Não comer sangue}

Por esta proibição somos proibidos de comer sangue. Ela está ex­pressa
em Suas palavras ``E sangue não comereis'' (Levítico 7:26). Ela aparece
várias vezes nas Escrituras\textsuperscript{354} e está declarado
expressamente que o castigo por sua contravenção voluntária é a
extinção: "Todo aquele que comer dele será banido (Ibid., 17:14). Aquele
que a infringir involuntariamente deverá oferecer um Sacrifício
Determinado de Pecado.


As normas deste preceito estão explicadas no quinto capítulo de


Queretot.

\section{Não comer gordura de um animal puro}

Por esta proibição somos proibidos de comer a gordura de um ani­mal
puro. Ela está expressa em Suas palavras "Todo sebo de boi, e de
carneiro, e de cabra, não comereis" (Levítico 7:23). Ela aparece mais
uma vez nas Escrituras\textsuperscript{355} e a pena de extinção está
explicitamente prescrita em caso de transgressão voluntária. Se alguém a
infringir involuntariamente ele deve ofe­recer um Sacrifício Determinado
de Pecado.


As normas deste preceito estão explicadas no sétimo capítulo de Hulin. •


\section{Não cozinhar carne no leite}

Por esta proibição somos proibidos de cozinhar carne no leite. Ela está
expressa em Suas palavras "Não cozinharás cabrito cora o leite de sua
mãe'

354. Levítico 3:17 e 17:14. 555. Levítico 3:17.

(Êxodo 23:19). Aquele que cozinhar carne no leite será punido com o
açoita­mento, mesmo que ele não a coma, como está explicado em vários
trechos do Talmud.

\section{Não comer carne cozida em leite}

Por esta proibição somos proibidos de comer carne em leite. Ela es­tá
expressa na repetiçào de Suas palavras "Não cozinharás cabrito com o
leite de sua mãe" (Êxodo 24:26), cujo objetivo é proibir-nos de comer. A
Guemará de Hulin diz: "Aquele que cozinhar carne no leite estará sujeito
ao açoitamen­to, e aquele que comer dela estará sujeito ao açoitamento."
. E a Guemará de Macot diz: "Aquele que, num dia de festa cozer o tendão
da coxa no leite e o comer estará sujeito a cinco açoitamentos: um por
comer o tendão, um por cozinhar\textsuperscript{356}, um por cozer carne
no leite, um por comer carne no leite, e um por acender o fogo". Diz
ainda: "Troque 'acender o fogo' por 'lenha que per­tence ao Santuário';
quanto à proibição necessária, ela está expressa no texto "Suas árvores
idolatradas queimareis no fogo... Não procedereis de modo se­melhante
para com o Eterno, vosso Deus" (Déuteronômio 12:3-4).

A Guemará de Hulin diz: "O Misericordioso expressa a proibição de comer
pelo termo 'cozinhar' porque assim como se fica sujeito ao açoitamento
por cozinhar, também se fica sujeito ao açoitamento por comer". E no
segun do capítulo de Pessahim lemos o seguinte, com relação à carne no
leite: "Co­mer não está mencionado especificamente para mostrar-nos que
podemos fi­car sujeitos ao açoitamento por causa de comida até mesmo se
não a usarmos da maneira convencional". Isto deve ser lembrado.

Neste ponto é conveniente que eu chame atenção para um princípio
importante, que não mencionei até agora.

Suas palavras ``Não cozinharás cabrito com leite de sua mãe'' apare­cem
três vezes na Torah\textsuperscript{357}, e de acordo com aqueles que
transcrevem a Tra­diçào, cada uma dessas proibições tem um objetivo
diferente. ``Uma'', dizem eles, "proíbe comê-la, outra proíbe tirar algum
proveito disso, e a outra proíbe de cozinhá-la".

Alguém poderia argumentar o seguinte: "Por que você conta a proi­bição
de comer e a proibição de cozinhar como dois preceitos, e não conta a
proibição de tirar proveito disso como um terceiro preceito?" Essa
pessoa deve ser informada que a proibição de tirar proveito disso não
pode ser con­tada propriamente como um preceito separado porque essa e a
proibição de comer são da mesma natureza, já que comer é uma forma de
tirar proveito. Toda a vez que ele disser, com relação a uma determinada
coisa, ``Isso não deve ser comido'', comer serve apenas como exemplo de
tirar proveito, e a intenção é de proibir-nos de tirar algum proveito da
coisa, seja comendo ou de qualquer outra forma. Isto está expresso pelos
Sábios da seguinte for­ma: "To ia a vez que as Escrituras disserein
'Isto não deve ser comido, Não deveis comer, Não deves comer',
compreende-se a proibição de comer e de tirar proveito, a menos que as
Escrituras declarem expressamente que não é assim, como no caso do
'nebelá', cujo uso Ele permite explicitamente ao


\begin{enumerate}
\def\labelenumi{\arabic{enumi}.}
\setcounter{enumi}{355}
\item
 
 Por cozinhar em dia de festa, sem necessidade.
 
\item
 
 Êxodo 23:19; Ibid., 34:19; Deuteronómio 14:21.
 
\end{enumerate}



dizer: "Ao peregrino incircunciso que está em tuas cidades o darás, e o
come­rá" (Deuteronómio 14:21).

De acordo com este princípio, não é correto contar-se a proibição de
comer e de tirar proveito dela como dois preceitos; e se contássemos
dois preceitos no caso da carne cozida com leite, deveríamos ter feito a
mesma coi­sa nos casos ``hametz'' "., de
``orlá''\textsuperscript{359}, e de "quil-ei ha
querem"\textsuperscript{360}, contan­do a proibição de tirar proveito
como um preceito independente em si, em ca­da um desses quatro casos.
Entretanto, como neles contamos apenas a proibi­ção de comer, porque ela
inclui a proibição de tirar proveito, como explica­mos, fazemos o mesmo
no caso da carne no leite.

Resta apenas uma pergunta. Pode ser perguntado por que as Escritu­ras
tiveram que mencionar a terceira proibição no caso de carne no leite, a
fim de proibir-nos de tirar proveito disso, como explicamos, se a
proibição de tirar proveito se deduz, como foi explicado, da proibição
de comer. A resposta é que, na realidade, as Escrituras não dizem, com
referência a carne no leite, "Não deveis \emph{comê-la",} o que
proibiria ambos de comer e tirar proveito dela; conse­qüentemente, era
necessário que houvesse um outro preceito para proibir que se tirasse
proveito.

Nós já mencionamos a razão pela qual o Misericordioso não escre­veu
``comer'' no caso da carne no leite, que é que toda vez que ``comer'' for
mencionado nào se é culpado a menos que se sinta prazer com isso. Se,
contu­do, alguém devesse abrir a boca e engolir comida proibida, ou
comè-la enquanto estivesse quente demais e como resultado queimasse a
garganta com ela, causando-lhe dor ao engolí-la, ou em qualquer caso
semelhante, ele estaria isento. As únicas exçessões são o caso da carne
no leite e o de "quil-ei ha querem", como explicaremos posteriormente;
nesses casos ele é culpado por comer, mes­mo que não sinta prazer com
isso. Você deve compreender e lembrar-se de to­dos esses princípios.


As normas deste preceito estão explicadas no oitavo capítulo de Hulin.


\section{Não comer a carne de um boi apedrejado}

Por esta proibição somos proibidos de comer a carne de um boi que tenha
sido apedrejado, mesmo que ele tenha sido abatido antes de ser
apedre­jado, porque uma vez que a sentença foi pronunciada ele passou a
ser alimento proibido, mesmo se ele tiver sido abatido de acordo com os
requisitos rituais. Esta proibição está expressa em Suas palavras,
enaltecido seja Ele, ``Não será comida a sua carne'' (Exodo 21:28); e a
Mekhiltá diz: "Se os proprietários de um boi que está sendo conduzido ao
apedrejamento o abaterem de acordo com os ritos, antes de seu
apedrejamento, sua carne será mesmo assim alimento proi­bido. Esse é o
significado de ``Não será comida da sua carne''.

Aquele que comer o equivalente ao tamanho de uma oliva de sua carne será
punido com o açoitamento.


\begin{enumerate}
\def\labelenumi{\arabic{enumi}.}
\setcounter{enumi}{357}
\item
 
 Alimento que contém fermento.
 
\item
 
 Nome que se dá ao fruto de uma árvore antes da mesma completar 3 anos.
 
\item
 
 Um vinhedo onde se colocou, juntamente com as sementes de uvas, outras
 espécies de se­mentes, como a de trigo ou as de vegetais.
 
\end{enumerate}

\section{Não comer pão feito com grãos da nova ceifa}

Por esta proibição somos proibidos de comer pão feito com grãos da nova
ceifa, antes do final do décimo sexto dia de Nissan. Ela está expressa
em Suas palavras, enaltecido seja Ele, "E pão, e farinha feita de grãos
das espigas verdes, torrada no forno, e grãos verdes de cereais não
comereis" (Le­vítico 23:14).

Aquele que comer o equivalente ao tamanho de uma oliva disso será punido
com o açoitamento.

\section{Não comer grãos da nova ceifa torrados}

Por esta proibição somos proibidos de comer grãos da nova ceifa
tor­rados, antes do final do décimo sexto dia de Nissan. Ela está
expressa em Suas palavras, enaltecido seja Ele, "E pão, e farinha feita
de \emph{grãos das espigas verdes, torrada} no forno, e grãos verdes de
cereais não comereis" (Levítico 23:14).

Aquele que comer o equivalente ao tamanho de uma oliva disso será punido
com o açoitamento.

\section{Não comer grãos verdes de cereais}

Por esta proibição somos proibidos de comer grãos verdes de ce­reais
antes do final do décimo sexto dia de Nissan. Ela está expressa em Suas
palavras, enaltecido seja Ele, "E pão, e farinha feita de grãos das
espigas verdes, torrada no forno, e \emph{grãos verdes de cereais} não
comereis" (Levítico 23:14).

Já nos referimos às palavras do Talmud: "Aquele que comer do pão e da
farinha feita de grãos de espigas verdes, torrada no forno, e dos grãos
ver­des de cereais será culpado por cada um deles separadamente" e
também já explicamos isto em detalhes no nono dos Fundamentos que
prefaciam este tra­balho, e aos quais você deve se reportar.

As normas da lei sobre a nova ceifa estão explicadas no sétimo capí­tulo
de Menahot, e em diversos trechos de Shebiit, Maasserot e Halá.

\section{Não comer ``orlá'''}

Por esta proibição somos proibidos de comer ``orlá''. Ela está ex­pressa
em Suas palavras "Por três anos vos será proibido; não se comerá"
(Le­vítico 19:23).

Aquele que comer o equivalente ao tamanho de uma oliva disso será punido
com o açoitamento.

As normas deste preceito estão explicadas nó Tratado Orlá.

A proibição contra comer .``orlá'' fora da Terra; de Israel está expres­sa
numa lei dada a Moisés no Sinai. O texto da Torah proíbe comê-los apenas
na Terra de Israel.


\section{Não comer "quil-ei ha querem"}

Por esta proibição somos proibidos de comer "quil-ei ha querem". Ela
está expressa em Suas palavras, enaltecido seja Ele, relativas a eles,
"Para que não se profane (pen tikdash) o produto com o que haja a mais
na semente que semeares" (Deuteronômio 22:9), sobre as quais a Tradição
diz: " 'Pen tid­kash --- pen tudak esh' (para que não seja consumido
pelo fogo)", ou seja, é proibido tirar-se algum proveito disso. Já nos
referimos ao princípio de que "To­da vez que as Escrituras dizem
`guarda-te (hishamer), ou 'para que não' (pen), ou 'não' (a1), há um
preceito negativo".

O segundo capítulo de Pessahim, depois de estabelecer que "Os ar­tigos
proibidos pela Torah não acarretam o açoitamento a não ser na sua
ma­neira habitual de consumo",- ou seja, que só se fica sujeito ao
castigo por se ter comido um alimento proibido se se sentiu prazer em
comê-lo. Diz ainda: "Abayé disse: Todos concordam que se deve impor
açoitamento no caso de um `quil-ei ha querem', mesmo que não de acordo
com seu uso habitual. Qual é a razão? Porque não se menciona 'comer'
neste caso", uma vez que as Escri­turas dizem apenas ``pen tikdash'' ---
(i.e.) .peri tukad esh (para que não seja consumido
pelo fogo).

As normas deste preceito estão explicadas no Tratado Quil-Aim. De acordo
com as Escrituras este preceito também só se aplica na Terra de Israel.

\section{Não beber ``yain nessech''}

Por esta proibição somos proibidos de beber ``yain nessech'' (i.e., vinho
de libação que foi usado para adoração de ídolos). Esta proibição não
está enunciada explicitamente nas Escrituras; mas elas dizem, a respeito
da ido­latria: "De cujos sacrifícios comiam a gordura e de cujas
libações bebiam o vi­nho" (Deuteronômio 32:38), e isso demonstra que a
proibição que se aplica a sacrifícios oferecidos a um ídolo se aplica da
mesma forma ao vinho de libação.

Você já está familiarizado com o princípio, freqüentemente mencio­nado
no Talmud, de que é proibido tirar proveito, sob pena do açoitamento.

Encontramos a prova de que o ``yain nessech'' é uma das proibições da
Torah, e que essa proibição deve ser contada entre os preceitos
negativos, na Guemará de Abodá Zará: "Rabi Yohanan e Rabi Shimeon ben
Lakish decla­raram: Com todas as coisas proibidas pela Torah, quer
consista a mistura da mesma variedade ou de variedades
diferentes\textsuperscript{361}, quando transmite um sabor, com exceção
de ``tebel'' e do ``yain nessech''. Nesses casos, com a mesma
variedade\textsuperscript{362}, pela menor
quantidade\textsuperscript{363}, mas com variedades diferentes,
quan­do." revelar um sabor". Isso é uma prova clara de
que a proibição do ``yain nessech'' está nas Escrituras.

Da mesma forma no Sifrei, numa descriçáo do erro de Israel em Shitim
ao. praticar a prostituição com as filhas de
Moab\textsuperscript{365}, lemos: "Entraram;


\begin{enumerate}
\def\labelenumi{\arabic{enumi}.}
\setcounter{enumi}{360}
\item
 
 A mistura será desqualificada para o consumo quando o elemento
 proibido transmitir um sabor.
 
\item
 
 A mistura está desqualificada.
 
\item
 
 Do elemento proibido.
 
\item
 
 Esse elemento proibido.
 
\item
 
 Ver Números 25:1.
 
\end{enumerate}

uma garrafa de vinho amonita estava junto dela, e naquela época o vinho
pagão ainda não havia sido proibido a Israel. Ela disse a ele: 'Você
gostaria de beber?' etc.". As palavras "naquela época o vinho pagão
ainda não havia sido proibido a Israel" prova acima de qualquer dúvida
que depois ele foi proibido.

As afirmações no Talmud de que a proibição de vinho pagão está entre as
Dezoito Regras prescritas e de que "O caso do 'yain nessech' é
diferen­te, visto que \emph{os Sábios} impuseram maior restrição a esse
respeito", se referem na realidade apenas ao vinho pagão, não ao
.yain nessech". que está proibido pelas escrituras. E
você conhece a máxima dos Rabinos: "Há ires tipos de vi­nho" etc.

As normas deste preceito estão explicadas nos últimos capítulos de Abodá
Zará.

\section{Não comer nem beber em excesso}

Por esta proibição somos proibidos de dedicar-nos excessivamente à
comida e à bebida enquanto formos jovens, de acordo com as condições
des­critas no caso de um filho inflexível e rebelde. Esta proibição está
expressa em Suas palavras, enaltecido seja Ele, "Não comereis sobre o
sangue" (Levítico 19:26).

A explicação disso é a seguinte. O filho inflexível e rebelde é um dos
que estão sujeitos à morte por condenação judicial, e a Torah determina
ex­pressamente que no seu caso a morte seja por
apedrejamento. Nós já expli­camos na Introdução a este
trabalho que cada vez que as Escrituras estabele­cem a pena de extinção
ou morte por condenação judicial há um preceito ne­gativo, exceto nos
casos dos sacrifícios de ``Pessah'' e da circuncisão, como
explicamos\textsuperscript{367}. Portanto. como a lei determina que o
glutão e beberrão está su­jeito ao apedrejamento, nas condições
mencionadas, sabemos que isso é algo que está totalmente proibido; e,
uma vez estabelecida a punição, resta-nos en­contrar a proibição, já que
é um princípio aceito que a Torah nunca prescreve um castigo sem ter
primeiro enunciado uma proibição\textsuperscript{368}. Na Guemará de
Sa­nhedrin lemos: "De que forma chegamos à proibição dirigida contra um
filho teimoso e rebelde? Pelo versículo 'Não comereis sobre o sangue' ".
Quer di­zer, não comereis de forma tal que isso ocasione derramamento de
sangue, e é assim que come o glutão e o beberrão, cujo castigo é a
morte. Se alguém co­mer dessa forma ruim e repreensível ele transgridirá
um preceito negativo e o fato desta proibição ser um ``lav shebikhlalut''
não tem conseqüências, como explicamos no Nono Fundamento, pois como o
castigo está formulado explici­tamente nas Escrituras, não importa se a
proibição está expressa numa lei ou num' ``lav shebikhlalut''. Nós já
explicamos isso várias vezes e já apresentamos exemplos disso antes.


As normas deste preceito estão explicadas no oitavo capítulo de


Sanhedrin.


\begin{enumerate}
\def\labelenumi{\arabic{enumi}.}
\setcounter{enumi}{365}
\item
 
 Ver Deuterdnomio 21:21.
 
\item
 
 Ver o preceito positivo 55.
 
\item
 
 Ver o preceito positivo 4.
 
\end{enumerate}



\section{Não comer durante um ``yom quipur''}

Por esta proibição somos proibidos de comer durante 'Yom
Qui­pui.". A Torah não contém nenhuma proibição
expressa a esse respeito, mas ela menciona o castigo --- a saber , que
aquele que comer nesse dia estará sujei­to à extinção --- em Suas
palavras "Porque toda alma que não se afligir neste mesmo dia, será
banida de seu povo" (Levítico 23:29). Conseqüentemente, sa­bemos que é
proibido comer durente ``Yom Quipur''.


No início de Queretot aquele que comer em ``Yom Quipur'' está incluído
entre aqueles que estão sujeitos à extinção, e está explicado que
há um\\
preceito negativo toda vez que se fica sujeito à extinção, exceto nos
casos do sacrifício
de ``Pessah'' e da circuncisão. Assim sendo, fica claro que comer
em "Yom\\
Quipur" é um preceito negativo; conseqüentemente aquele que o
transgredir\\
voluntariamente será punido com a extinção e aquele que o violar
involuntariamente
, com um Sacrifício Determinado de Pecado, de acordo com o que
está\\
explicado no início de Queretot e no Tratado Horayot: que esta obrigação
se aplica\\
apenas aos preceitos \emph{negativos,} uma vez que com relação a quem é
obrigado a\\
levar um Sacrifício Determinado de Pecado, Ele diz, enaltecido seja Ele,
"Por fazer
um dos preceitos do Eterno, daqueles que são de não fazer" (Levítico
4:27).\\
A Sifrá diz: " 'Porque toda a alma que não se afligir neste mesmo dia,\\
será banida do seu povo'. Isto deteripina o castigo por deixar de
afligir a alma,\\
mas não apresenta nenhuma proibição explícita com relação a afligir-se
nesse\\
dia. Não havia necessidade de que as Escrituras enunciassem o castigo
acarretado
pelo fato de trabalhar\textsuperscript{369}, porque isso se deduz
através de um `cal vahomer
'.'.. Se deixar de afligir a alma
em 'Yom Quipur' é algo que deva ser punido
, conclui-se que certamente a execução de um trabalho, que \emph{está}
determinada
com relação aos Festivais e Shabatot, deve ser punida. Então por que
está\\
expresso o castigo decorrente do fato de trabalhar? Para que se derive
dele a\\
\emph{proibição} relativa à aflição da alma: assim como o castigo por
trabalhar está precedido
por uma proibição, o castigo por afligir a alma também está
precedido\\
por uma proibição". Dessa forma explicamos o que havíamos prometido
explicar.\\
As normas deste preceito estão explicadas no final do Tratado Yoma.


\section{Não comer ``hametz'' durante ``pessah''}

Por esta proibição somos proibidos de comer ``hametz'' durante "Pes­sah".
Ela está expressa em Suas palavras, enaltecido seja Ele, "Não comereis
coisa levedada" (Êxodo 13:3).

A transgressão voluntária desta proibição será punida pela extinção, de
acordo com o que está claramente expresso em Suas palavras "Pois todo
aquele que comer coisa levedada, será banida aquele alma de Israel"
(Ibid., 12:15). Aquele que a transgredir involuntariamente é obrigado a
levar um Sa­crifício de Pecado.


As normas deste preceito estão explicadas no Tratado Pessahim.

\begin{enumerate}
\def\labelenumi{\arabic{enumi}.}
\setcounter{enumi}{368}
\item
 
 Levítico 23:30.
 
\item
 
 Isto é. .com toda razão".
 
\end{enumerate}

198 NÃO COMER NADA QUE CONTENHA ``HAMETZ'' DURANTE ``PESSAH''

Por esta proibição somos proibidos de comer qualquer coisa que con­tenha
uma mistura de ``hametz'', mesmo que não seja pão, como por exemplo
'cuta"\textsuperscript{371}, todo o tipo de condimento, e similares.
Esta proibição está expres­sa em Suas palavras "Nenhuma coisa levedada
contereis" (Êxodo 12:20), sobre as quais diz a Mekhiltá: "Incluindo
`cutá' da Babilônia, cerveja Central, e vina­gre Idumeano. Eu poderia
pensar que o castigo é causado por eles; por isso as Escrituras dizem:
'Todo aquele que comer levedura, será banida aquela al­ma' (Ibid., 19),
ou seja, por comida levedada, mas não por comida com uma mistura de
levedo. Por que então essas coisas foram mencionadas? Porque são uma
desobediência a um preceito negativo".

Está explicado no Tratado Pessahim que embora tais coisas sejam
proibidas e comê-las seja proibido, aquele que as comer não estará
sujeito ao açoitamento a menos que ele coma o equivalente ao tamanho de
uma oliva de ``hametz'' dentro do tempo em que se comeria meio
pão\textsuperscript{372}; de outra forma ele não será culpado.

\section{Não comer ``hametz'' depois da metade do decimo quarto de ``nissan''}

Por esta proibição somos proibidos de comer ``hametz'' depois da metade do
décimo quarto dia de Nissan. Ela está expressa em Suas palavras "Não
comerás nela levedo" (Deuteronômio 16:3), nas quais a expressão ``nela''
se refere ao cordeiro de ``Pessah'', que devemos abater no décimo quarto
dia ``ao anoitecer''. De acordo com isso Suas palavras São: "Não comereis
pão leveda­do a partir da hora do abate".

Na Guemará de. Pessahim lemos: "Como ficamos sabendo que aque­le que
come `hametz' a partir das seis horas transgride um preceito negativo?
Pelo versículo 'Não comereis nela levedo' ". No mesmo Tratado também
le­mos: "Todos concordam que o `hametz' é proibido pelas Escrituras a
partir das seis horas". Esse é o comentário que encontramos em todos os
textos corretos que foram lidos antes dos mais sábios
Talmudistas\textsuperscript{373}. Esse Tratado também dá a seguinte
razão para a proibição de comer ``hametz'' na sexta hora: "Os Sábios
estabeleceram uma proteção para uma proibição das Escrituras".

Aquele que transgredir ao comer ``hametz'' depois da metade do dia será
punido com o açoitamento.


As normas deste preceito estão explicadas no início de Pessahim.

\begin{enumerate}
\def\labelenumi{\arabic{enumi}.}
\setcounter{enumi}{370}
\item
 
 Urna conserva composta de leite azedo, crostas de pau e sal.
 
\item
 
 A duração do consumo de três ovos (Mishné Torah, cap., Hametz e
 Matzah. Halachá).
 
\item
 
 Os ``Gaoniin''.
 
\end{enumerate}


PRECEITOS NEGATIVOS 285

\section{Não pode ser visto ``hametz'' em nossas moradias durante ``pessah''}

Por esta proibição está proibido que se veja ``hametz'' em qualquer uma
das nossas moradias durante os sete dias\textsuperscript{374}. Esta
proibição está expressa em Suas palavras "Não será vista por.ti coisa
levedada, e não será visto contigo fermento, em todo o teu território"
(Êxodo 13:7).

Essas não são duas proibições, relativas a dois assuntos diferentes, mas
sim referem-se a um único tópico. O Talmud explica: "As Escrituras
ini­ciam com levedo e terminam com pão levedado para ensinar-nos que
levedo e pão levedado são a mesma coisa". Quer dizer, não há diferença
entre o leve­do e aquilo que é levedado.

Aquele que transgredir, guardando 'hametz" em seu poder, não es­tará
sujeito ao açoitamento a menos que ele tenha comprado ``hametz'' duran­te
``Pessah'' e tenha adquirido direito a ele, realizando dessa forma um ato
es­pecífico relacionado com ele. Pelas palavras da Tosseftá: "Aquele que
guardar `hametz' durante 'Pessah' e aquele que permitir que diversas
sementes cresçam.'., não estará
sujeito ao açoitamento".


\section{Não possuir ``hametz'' durante ``pessah''}

Por esta proibição somos proibidos de ter ``hametz'' em nossa pos­se,
ainda que ele não esteja visível ou mesmo que ele tenha sido deixado com
alguém. Esta proibição está expressa em Suas palavras "Sete dias
levedura não será encontrada em vossas casas" (Êxodo 12:19). Também por
esta proibição o castigo será o açoitamento, desde que haja um ato
envolvido, como mencio­namos, de acordo com os princípios estabelecidos
no Tratado Shebuot. Os sá­bios dizem explicitamente, em vários lugares:
"Transgride-se 'não será visto' e 'não será encontrado' ".

No início do Tratado Pessahim estão explicadas as normas desses dois
preceitos, bem como as coisas proibidas por Suas proibições "Não será
visto por ti fermento, em todo o teu território" (Ibid., 13:7) e
``Levedura não será encontrada em vossas casas''. Está explicado ali que
cada um desses preceitos negativos obtém alguma coisa a mais do outro, e
que aquele que permitir que ``hametz'' fique em seu poder durante ``Pessah''
transgride dois preceitos: ``não será visto'' e ``não será encontrado''.


\section{Um nazir não pode beber vinho}


Por esta proibição um Nazir fica proibido de beber vinho ou qual­quer
outro tipo de bebida forte da qual o suco da uva seja um componente
importante. Ela está expressa em Suas palavras, enaltecido seja Ele, "E
todo o remolho de beberagem de uvas não beberá" (Números 6:3).


\begin{enumerate}
\def\labelenumi{\arabic{enumi}.}
\setcounter{enumi}{373}
\item
 
 Os sete dias de Pessah.
 
\item
 
 Em seu vinhedo.
 
\end{enumerate}

Ele levou essa proibição tão longe a ponto de transcrevê-la e está
declarado que mesmo se o vinho ou a bebida feita com ele azedar, ele não
po­de bebê-la. Mas esta proibição, expressa em Suas palavras "Vinagre de
vinho novo e vinagre de vinho velho não beberá" (Ibid.), não é um
preceito indepen­dente. Se Ele tivesse dito: "Ele não beberá vinho nem
vinagre de vinho" have­ria dois preceitos diferentes; mas Suas palavras,
"Vinagre de vinho ... não be­berá" complementam a proibição do vinho.

A Guemará de Nazir explica que Suas palavras "mishrat'anab im" ("li­cor
de uvas") ``significam que o sabor é equivalente ã substância em si''. -

Uma prova de que elas são um único preceito é que se ele beber vi­nho
.e" vinagre de vinho ele não será punido por
açoitamento duas vezes; é esse o castigo aplicado se ele beber um
``rebiit''\textsuperscript{376} de vinho ou de vinagre.

\section{Um nazir não pode comer uvas frescas}

Por esta proibição um Nazir fica proibido de comer uvas frescas. Ela
está expressa em Suas palavras "E uvas frescas ... não comerá" (Números
6:3). Se ele comer o equivalente ao tamanho de uma oliva delas ele será
punido com o açoitamento.

\section{Um nazir não pode comer uvas secas}

Por esta proibição um Nazir fica proibido de comer uvas secas. Ela está
expressa em Suas palavras "E uvas ... secas não comerá" (Números 6:3).
Se ele comer o equivalente ao tamanho de uma oliva delas ele será punido
com o açoitamento.

\section{Um nazir não pode comer caroços de úvas}

Por esta proibição um Nazir fica proibido de comer caroços de uvas. Ela
está expressa em Suas palavras \emph{"Desde as grainhas} até a casca das
uvas não comerá" (Números 6:4). Se ele comer o equivalente ao tamanho de
uma oliva delas ele será punido com o açoitamento.

\section{Um nazir não pode comer bagaços de uvas}

Por esta proibição um Nazir fica proibido de comer bagaços de uvas. Ela
está expressa em Suas palavras "Desde as grainhas \emph{até a casca das
uvas} não comerá" (Números 6:4). Se ele comer o equivalente ao tamanho
de uma oliva delas ele será punido com o açoitamento.

A prova de que essas cinco proibições --- a do vinho, a de uvas
fres­cas, a de uvas secas, a de caroços e a de bagaços de uvas --- são
cada uma um
preceito individual é o fato de que o castigo por qualquer uma delas é a
puni­ção por açoitamento. A Mishná diz: "Há um castigo separado por
causa do vi­nho, das uvas, dos caroços das uvas e dos bagaços das uvas"
e na Guemará de Nazir está explicitamente dito: "Se ele comesse uvas
frescas, uvas secas, os ca­roços de uvas, e os bagaços de uvas, e se
espremesse um cacho de uvas e bebesse\textsuperscript{377}, ele estaria
sujeito ao açoitamento por cinco vezes". Na tentativa de provar que o
``Taná'' não mencionou todas as vezes, e que na realidade um Nazir poderia
estar sujeito a mais do que cinco açoitamentos, a Guemará diz que ele
omitiu o preèeito ``Não profanará a Sua palavra'' (Números 30:3); mas ela
não diz que ele omitiu o vinagre de vinho porque não se é culpado uma
vez por beber vinho e outra por beber vinagre de vinho. O vinagre só é
proibi­do porque ele é essencialmente vinho, como explicamos, e é como
se Ele ti­vesse dito que ao se tornar vinagre, o vinho não perde a
característica principal que faz com que ele seja proibido.

É importante que você saiba que todas as coisas proibidas a um Na­zir
podem ser postas juntas para perfazer um volume total equivalente ao
tama­nho de uma oliva, que acarreta a pena de açoitamento para aquele
que o co­me r

\section{Um nazir não pode fazer-se impuro pelos mortos}

Por esta proibição um Nazir fica proibido de fazer-se impuro pelos
mortos. Ela está expressa em Suas palavras "Por seu pai, por sua mãe,
por seu irmão, e por sua irmã, não se impurificará" (Números 6:7).
Aquele que se fizer impuro por qualquer pessoa morta, seja a impureza
uma daquelas pelas quais ele deve cortar seu cabelo ou não, ele será
punido com o açoitamento.

\section{Um nazir não pode fazer-se impuro entrando numa casa onde haja um
morto}

Por esta proibição um Nazir fica proibido de fazer-se impuro numa casa
com um morto. Ela está expressa em Suas palavras "Não se aproximará de
um morto" (Números 6:6), e a Guemará diz explicitamente que as
Escrituras fazem uma afirmação clara: " 'Não se impurificará' (Ibid.,7);
ao acrescentar 'Não \emph{se aproximará',} proíbe que se contamine e que
entre". Está explicado ali que se ele entrar numa casa com um morto
depois de ter-se tornado impuro, ele será punido com o açoitamento
apenas uma vez; mas se ele se tornar impuro e entrar na casa ao mesmo
tempo --- como, por exemplo, se ele entrar numa casa na qual houver um
moribundo e ficar ali até que o homem morra, de ma­neira que os fatos de
ter-se tornado impuro e o de ter entrado na casa onde há um morto
ocorram ao mesmo tempo --- ele estará sujeito ao açoitamento por duas
vezes. Contudo, se ele entrar numa casa onde haja um morto, ele se torna
impuro antes de sua entrada, como está explicado ali, de acordo com os
princípios estipulados em Oholoth.

\section{Um nazir não pode raspar a cabeça}

Por esta proibição um Nazir fica proibido de raspar a cabeça.. Ela es­tá
expressa em Suas palavras, enaltecido seja Ele, "Lâmina não passará por
sua cabeça" (Números 6:5).

Aquele que raspar a cabeça de um Nazir também será punido com o
açoitamento, porque quem barbeou e quem se deixou barbear sãoigualmen­te
culpados e a pena de açoitamento incide a partir do momento em que ele
tenha raspado um único fio de cabelo.

Todas as normas da lei dos Nazirim estão explicadas no Tratado de­dicado
especialmente a este assunto.

\section{Não ceifar toda a colheita}

Por esta proibição somos proibidos de ceifar a totalidade da colhei­ta;
deve-se deixar uma parte num canto do campo para os pobres. Esta
proibi­ção está expressa em Suas palavras, enaltecido seja Ele, "Não
acabarás de segar o canto de teu campo" (Levítico 19:9).

Este preceito negativo está justaposto a um preceito positivo por­que se
alguém o transgredir e segar toda a colheita ele deve dar a medida de
``peá'' da colheita ceifada aos pobres. Isto aparece em Suas palavras,
enaltecido seja Ele, ``Para o pobre e o imigrante os deixarás'' (Ibid.,
10). como explicamos ao tratar dos preceitos
positivos\textsuperscript{378}. A ``peá'' é obrigatória tanto no caso das
árvores como no caso dos campos.

De acordo com as Escrituras, este preceito só é obrigatório na Terra de
Israel. Suas normas estão explicadas no Tratado especialmente dedicado a
este assunto.

\section{Não recolher as espigas de cereais que caíram durante a colheita}

Por esta proibição somos proibidos de recolher as espigas de cereais que
caírem durante a colheita; elas devem ser deixadas para os pobres. Esta
proi­bição está expressa em Suas palavras, enaltecido seja Ele, "E as
espigas caídas no recolhimento de tua ceifa, não recolherás" (Levítico
19:9).

Este também está justaposto a um preceito positivo\textsuperscript{379},
como expli­camos no caso da ``peá''. Suas normas estão explicadas no
Tratado Peá.

\section{Não recolher todo o produto do vinhedo na época da vindima}

Por esta proibição somos proibidos de recolher todo o produto do vinhedo
na época da vindima. Esta proibição está expressa em Suas palavras,


378 Ver o preceito positivo 120.\\
379. Ver o preceito positivo 121.



enaltecido seja Ele, ``E tua vinha não rebuscarás'' (Levítico 19:10); os
cachos de uvas que não estiverem totalmente desenvolvidos devem ser
deixados para os pobres.

Isto não se aplica a outras árvores, ainda que elas sejam similares às
videiras, porque a proibição expressa em Suas palavras "Quando bateres a
tua oliveira, não tornarás a colher o que resta nos ramos" (Deuteronômio
24:20) significa que não devemos recolher uma oliva \emph{esquecida,} e
essa lei sobre a oli­va esquecida se aplica também às outras
árvores\textsuperscript{380}.

Este também está justaposto a um preceito positivo\textsuperscript{381},
e suas nor­mas estão explicadas no Tratado Peá.

\section{Não recolher os bagos das uvas que caírem durante a colheita}

Por esta proibição somos proibidos de recolher os bagos soltos que
caírem durante a vindima; eles devem ser deixados para os pobres. Esta
proibi­ção está expressa em Suas palavras "E bago de tua vinha não
recolherás" (Leví­tico 19:10).

Este também está justaposto a um preceito positivo\textsuperscript{382}
e suas nor­mas estão explicadas no Tratado Peá.

\section{Não voltar para buscar uma gavela esquecida}

Por esta proibição somos proibidos de ir buscar um feixe de espigas.
esquecido. Ela está expressa em Suas palavras "E esqueceres uma gavela,
no campo, não voltarás a tomá-la" (Deuteronômio 24:19). A obrigação do
que é esquecido se aplica tanto ao solo quanto à
árvore\textsuperscript{383}. Este preceito também está justaposto ao
preceito positivo expresso em Suas palavras, enaltecido seja Ele, "Para
o imigrante, o órfão, e a viúva será" (Ibid.)\textsuperscript{384} e se
alguém pecar e o re­colher, deverá devolvê-lo ao pobre. As normas deste
preceito estão explicadas no Tratado Peá.

Você deve saber que é um princípio aceito entre nós, no caso de um
preceito negativo que está acompanhado de uma ordem a uma ação positiva,
que se o transgressor executar a ordem positiva ele não será punido com
o açoitamento, mas se não a executar, ele o será. No caso da ``peá'', por
exemplo, se ele ceifar a colheita inteira não ficará sujeito ao
açoitamento na época da colheita, e poderá dar aos pobres espigas de
cereais. Da mesma for­ma, se ele debulhar o grão e o transformar em
farinha, e fizer massa com essa farinha, ele pode dar a medida de ``peá''
da massa. Mas se acontecer de o grão se perder ou se queimar
completamente --- e sobretudo se por sua livre iniciativa ele causar o
seu desaparecimento como por exemplo, comendo-o


\begin{enumerate}
\def\labelenumi{\arabic{enumi}.}
\setcounter{enumi}{379}
\item
 
 Mas não há menção quanto à lei das frutas não desenvolvidas em outras
 Arvores.
 
\item
 
 Ver o preceito positivo 123.
 
\item
 
 Ver o preceito positivo 124.
 
\item
 
 A proibição de recolher o que foi esquecido se aplica a todos os
 produtos do solo e das árvores.
 
\item
 
 Ver o preceito .positivo 122.
 
\end{enumerate}

todo --- ele estará sujeito ao açoitamento, uma vez que terá deixado de
cumprir um preceito positivo.

Vocês não devem interpretar mal as palavras da Guemará de Macot: "Nós só
temos este caso e um outro", onde ``um outro'' significa a ``peá'',
deduzindo, como vocês poderiam fazer, que essa regra se aplica
\emph{apenas à} ``peá''. Não é assim. Na realidade, ``um outro'' significa o
caso da ``peá'' e todos os casos semelhantes, já que os bagos caídos, as
espigas, o que foi es­quecido e os cachos de uvas não totalmente
desenvolvidos são cada um pre­ceito negativo acompanhado de uma ordem a
uma ação positiva; e em cada um desses casos, como no caso da ``peá'', há
possibilidades alternativas de "kiyemu ve lo
kiyemu"\textsuperscript{385} ou de "bitlo ve lo
bitlo"\textsuperscript{386}. Pois o texto através do qual ficamos
sabendo que o caso da ``peá'' está associado a uma ordem de urna ação
positiva é o de Suas palavras, enaltecido seja Ele, "Para o pobre e o
imigrante os deixarás" (Levítico 19:10), e essas palavras se aplicam à
``peá'', espigas, bagos soltos de uvas, e cachos de uvas não totalmente
desenvolvidos. Suas palavras são: "Não acabarás de segar o canto de teu
campo, e as espigas caídas no recolhimento de tua ceifa, não recolherás.
E tua vinha não rebusca­rás, e o bago de tua vinha não recolherás; para
o pobre e o imigrante os deixa­rás" (Ibid., 9-10). E Ele diz mais
adiante, sobre a gavela esquecida: "Não volta­rás a tomá-la; para o
imigrante, o órfão , e a viúva será". E como encontramos na Guemará que
a ``peá'' é um preceito negativo justaposto a um preceito positivo,
encontrado em Suas palavras "Para o pobre e o imigrante os deixa­rás",
segue-se que todas essas cinco proibições\textsuperscript{38-7} são
preceitos negativos jus­tapostos a preceitos positivos;
consequentemente, se alguém cumprir o pre­ceito positivo não será punido
com o açoitamento, como mencionamos, e se não lhe for possível
cumpri-1o, ele será punido com o açoitamento. Mas enquanto houver a
possibilidade de cumprí-lo, ele ficará isento do castigo mesmo que ele
ainda não o tenha cumprido, e nós simplesmente lhe ordena­remos que o
cumpra. Ele só passa a ficar sujeito ao açoitamento quando sou­bermos
que ele infringiu a proibição e que não resta nenhuma possibilidade de
cumprir o preceito positivo.

Você precisa saber e compreender isto.

\section{Não semear "quil-aim"}

Por esta proibição somos proibidos de semear
;quil-aim"\textsuperscript{388}. Ela es­tá expressa em Suas palavras,
enaltecido seja Ele, "Teu campo não semearás com diversas
sementes". (Levítico 19:19)

Semear "quil-ei zeraim" só é proibido na Terra de Israel e aquele que o
fizer estará sujeito, de acordo com as Escrituras, ao açoitamento. E
per­mitido fazê-lo fora da Terra de Israel.


As normas deste preceito estão explicadas no Tratado Quil-Aim.

\begin{enumerate}
\def\labelenumi{\arabic{enumi}.}
\setcounter{enumi}{384}
\item
 
 Se ainda houver uma possibilidade de cumprir o preceito positivo, ele
 não estará sujeito à punição, mas se não houver mais nenhuma
 possibilidade de cumprí-lo, ele estará sujeito.à punição.
 
\item
 
 Se ele próprio anulou a possibilidade de cumprir o preceito positiivo,
 ele estará sujeito à puni­ção, mas se não foi ele mesmo quem anulou
 essa possibilidade, então ele não ficará sujeito à punição.
 
\item
 
 As proibições estabelecidas nos preceitos negativos 210 a 214i
 
\item
 
 Ou seja, semear diversos tipos de sementes num só campo, tais 'como
 trigo com aveia. Tam­bém conhecido como .quil-ei
 zeraim".
 
\end{enumerate}



\section{Não semear grãos nem vegetais num vinhedo}

Por esta proibição estamos proibidos de semear grãos ou vegetais num
vinhedo. Esta forma de "quil-aim" é chamada "quil-ei ha querem", e a
proibição está expressa em Suas palavras, enaltecido seja Ele, "Não
semearás a tua vinha com diferentes espécies de sementes" (Deuterônomio
22:9). Sobre isso o Sifrei diz: " 'Não semearás a tua vinha': Por que
isto é necessário? Já não nos foi dito 'Teu campo não semearás com
diversas sementes' (Levítico 19:19), o que sem dúvida alguma, inclui
ambos o vinhedo e o campo?" A resposta é que isto é para ensinar-nos que
aquele que permite que grãos ou vegetais cres­çam em seu vinhedo
infringe dois preceitos negativos.

Vocês precisam saber que, de acordo com as Escrituras, "quil-ei ha
querem" só é proibido na Terra de Israel; aquele que semear na Terra de
Israel estará sujeito ao açoitamento, de acordo com as Escrituras, se
ele semear, de uma só vez, trigo e cevada misturados com caroços de uvas
A lei dos Rabinos proíbe semear fora da Terra de Israel também, e aquele
que semear trigo e ce­vada junto com caroços de uvas, com um único
movimento, fora da Terra de Israel, estará sujeito ao açoitamento, de
acordo com a lei Rabínica. Mas o en­xerto de árvores, cuja proibição
está incluída em Suas palavras "Teu campo não semearás com diversas
sementes" é punido com açoitamento em qualquer lugar.


As normas deste preceito estão explicadas no. Tratado Quil-Aim.


\section{Não cruzar animais de espécies diferentes}

Por esta proibição somos proibidos de cruzar animais de espécies
diferentes. Ela está expressa em Suas palavras "O teu animal não farás
juntar com outra espécie" (Levítico 19:19). A penalidade por isso é o
açoitamento, desde que se auxilie o cruzamento da forma como se coloca
um pincel num canudo. O Talmud diz explicitamente: "Em caso de adultério
eles\textsuperscript{389} devem tê-los visto em atitude de adultério;
mas com relação a espécies diversas, eles devem tê-lo visto auxiliando
tal como alguém que estivesse colocando um pin­cel num canudo". Só assim
se está sujeito ao açoitamento. • •


As normas deste preceito estão explicadas no oitavo capítulo do


Quil-Aim.

\section{Não trabalhar com duas espécies diferentes de animais juntos}

Por esta proibição somos proibidos de trabalhar com animais de duas
espécies diferentes juntos. Ela está expressa em Suas palavras "Não
lavrarás com boi e jumento juntamente" (Deuteronômio 22:10). 
Aquele que trabalhar ---
por exemplo, lavrar, debulhar, ou puxar --- com eles será punido com
açoitamen­to. A palavra ``juntos'' significa que não devemos juntá-los
para fazer qualquer espécie de trabalho.

De acordo com as Escrituras só se incorre na penalidade de açoi­tamento
se os dois animais de espécies diferentes forem um animal puro e um
impuro, como por exemplo, um boi e um jumento; se alguém arar ou puxar
com dois animais assim juntos, ou se os conduzir juntos, ele será pu
nido com o açoitamento. De acordo com a lei Rabínica, contudo,
incorre-se nessa penalidade se se trabalhar com animais de duas espécies
diferentes quaisquer.


As normas deste preceito estão explicadas no oitavo capítulo de


Quil-Aim.

\section{Não impedir um animal de comer do produto no meio do qual ele esteja trabalhando}

Por esta proibição somos proibidos de impedir um animal de comer do
produto no meio do qual ele esteja trabalhando. Assim, se um animal
esti­ver esmagando cereais, ou carregando palha no seu dorso de um lugar
para ou­tro, não devemos impedí-lo de comer do grão ou da palha. Esta
proibição está expressa em. Suas palavras, enaltecido seja Ele, "Não
amarrarás a boca do boi quando estiver debulhando" (Deuteronômio 25:4),
e está explicado que a proi­bição de amarrar a boca se aplica da mesma
forma ao boi e aos outros animais, embora as Escrituras mencionem apenas
o de uso comum. Quer esteja o ani­mal amassando cereais ou fazendo
qualquer outro trabalho, ele não deve ser impedido de comer do produto
no meio do qual ele esteja trabalhando; toda vez que alguém o impedir de
fazê-lo, nem que seja apenas dizendo ao animal para que não coma, ele
será punido com o açoitamento.


As normas deste preceito estão explicadas no sétimo capítulo de Ba


ha Metzia.

\section{Não cultivar o solo no sétimo ano}

Por esta proibição somos proibidos de cultivar\textsuperscript{390} o
solo no sétimo ano. Ela está expressa em Suas palavras, enaltecido seja
Ele, "E no sétimo ano, sábado de descanso será para a terra... teu campo
não semearás" (Levítico 25:4).

A contravenção a esta proibição será punida com o açoitamento. Suas
normas estão explicadas no Tratado Shebiit.


\section{Não podar árvores no sétimo ano}

Por esta proibição somos proibidos de cultivar árvores no sétimo ano.
Ela está expressa em Suas palavras, enaltecido seja Ele, "E tua vinha
não podarás" (Levítico 25:4). A penalidade pela contravenção desta
proibição tam­bém é o açoitamento.

A Sifrá diz: "Semear e podar já estão incluídos no preceito geral
.; então por que eles foram mencionados
especificamente? Para permitir que se faça uma analogia: como semear e
podar têm a característica específica de se­rem trabalhos comuns ao
campo e ao pomar, eu deduzo\textsuperscript{392} os tipos de traba­lho
que sejam comuns ao campo e ao pomar.


As normas deste preceito também estão explicadas no Tratado Shebiit.

\section{Não ceifar uma planta que nasceu por si só no sétimo ano da
maneira como se faz num ano comum}

Por esta proibição somos proibidos de ceifar qualquer coisa que cres­ça
por si só no sétimo ano da maneira como o faríamos num outro ano
qual­quer. O significado disto é que somos proibidos de cultivar o solo
ou árvores no ano de Shabat, como já mencionamos antes, mas podemos
comer, durante o sétimo ano, aquilo que crescer das sementes, que
tiverem caído no solo du­rante o sexto ano, conhecido como "produção
tardia" --- com a diferença de que ela deve ser colhida de maneira
diferente. Esta proibição está expressa em Suas palavras, enaltecido
seja Ele, "O que nascer por si mesmo, depois da ceifa, não segarás"
(Levítico 25:5). Isto não significa que a produção tardia não deva ser
colhida, pois Ele diz: "Serão os produtos do descanso da terra, livres
para comer, para vós e para todos, igualmente" (Levítico 25:6). O
significado é que a maneira de ceifar deve ser diferente da dos outros
anos: deve-se ceifar a pro­dução tardia como se ceifaria um produto que
não pertença a ninguém, sem precauções e sem preparativos, como
explicaremos.

\section{Não colher uma fruta que tenha crescido por si so no sétimo ano, da
mesma maneira como se faz num ano comum}

Por esta proibição somos proibidos de colher uma fruta que tenha
crescido no sétimo ano da maneira como a colheríamos num ano comum: de-


\begin{enumerate}
\def\labelenumi{\arabic{enumi}.}
\setcounter{enumi}{390}
\item
 
 Relativo ao sétimo ano.
 
\item
 
 Eu deduzo que a proibição se aplica \emph{apenas} aos tipos de
 trabalho, etc.
 
\end{enumerate}

vemos agir de modo diferente para indicar que ela 
não tem dono. Esta
proibi­ção está expressa em Suas palavras, enaltecido seja Ele, "As uvas
separadas para ti, da tua vinha, não colherás" (Levítico 25:5), que são
interpretadas da seguin­te forma: "Não deves colhê-las da maneira como
aqueles que as colhem". Daí as palavras dos Sábios: "Os figos do sétimo
ano não devem ser cortados com uma faca para figos, mas podem ser
cortados com uma faca comum; as uvas não devem ser esmagadas com uma
prensa de vinho, mas podem ser esmaga­das num tonel; e as olivas não
podem ser preparadas numa prensa de olivas nem com um triturador de
olivas, mas podem ser trituradas e colocadas em prensas pequenas".

As normas deste e do preceito precedente estão explicadas no Tra­tado
Shebiit.

\section{Não cultivar o solo no ano do jubileu}

Por esta proibição somos proibidos de cultivar o solo no Ano do
Ju­bileu. Ela está expressa em Suas palavras, relativas a esse ano, "Não
semeareis" (Levítico 25:11), que correspondem as Suas palavras,
relativas ao Ano de Sha­bat, ``Teu campo não semearás'' (Ibid., 4).

O cultivo tanto do solo como das árvores está proibido no Ano do Jubileu
assim como no Ano de Shabat e por essa razão Ele diz ``Não semeareis'',
uma expressão generalizada, que abrange ambos o solo e as árvores.

A contravenção a esta proibição também será punida com o açoitamento.

\section{Não ceifar a produção tardia do ano do jubileu da maneira como se
faz num ano comum}

Por esta proibição somos proibidos de ceifar a produção tardia do Ano do
Jubileu da maneira como a ceifamos num ano comum, como ex plicamos em
relação ao sétimo ano. Esta proibição está expressa em Suas pa lavras,
enaltecido seja Ele, ``E não segareis o que nascer por si mesmo''
(Leví­tico 25:11)

\section{Não colher frutas no ano do jubileu da maneira como' se faz num ano comum}

Por esta proibição somos proibidos de colher frutas no Ano do Jubileu da
maneira como as colhemos nos outros anos. Esta proibição está expressa
em Suas palavras "E não colhereis as uvas da vinha, separadas para vós"
(Levítico 25:11), como explicamos no caso do sétimo ano. A Sifrá diz: "
'E não segareis... e não colhereis as uvas da vinha': assim como isto
está determinado com relação ao sétimo ano, também está determinado com
rela­ção ao Jubileu"; quer dizer, ambos estão na mesma situação com
relação a todas essas proibições.


As normas do Ano de Shabat e do Ano do Jubileu não são obrigató­rias a
não ser na Terra de Israel.


\section{Não vender definitivamente nossas terras em Israel}


Por esta proibição somos proibidos de vender nossas terras na Ter ra de
Israel em caráter definitivo. Ela está expressa em Suas palavras "A
terra não será vendida em perpetuidade" (Levítico 25:23).

As normas deste preceito estão explicadas no final de Arakhin.

\section{Não vender as terras dos arredores dos levitas}

Por esta proibição somos proibidos de vender as terras dos arredo­res
que pertençam aos Levitas. Ela está expressa em Suas palavras "E o campo
do arrabalde de suas cidades não será vendido" (Levítico .25:34).

A Torah diz\textsuperscript{393}, como vocès sabem, que se deve dar
cidades aos Le­vitas, com arredores e campos, isto é, mil cúbitos de
arredores e dois mil cúbi­tos depois deles para campos e vinhedos, como
está explicado no Tratado So­tá. A proibição é dirigida aos Levitas, que
ficam proibidos de alterar essa de­marcação, transformar terreno urbano
em arredores e arredores em terreno ur­bano, ou campos em arredores, ou
arredores em campos. Esta proibição está contida em Suas palavras "Não
será vendido", que a Tradição interpreta como significando que isso não
deve ser alterado.

As normas deste preceito estão ,.•xplicadas no final. de Arakhin.

\section{Não abandonar os levitas}

Por esta proibição somos proibidos de abandonar os Levitas, de dei­xar
de dar-lhes suas porções completas ou de alegrar seus corações nos
Festi­vais. Ela está expressa em Suas palavras "Guarda-te de abandonar o
levita" (Deu­teronômio 12:19), sobre as quais diz o Sifrei: " 'Hishamer'
(guarda-te) é um pre­ceito negativo; `pen ta azov' (de abandonar) é um
preceito negativo.'

\section{Não cobrar as dívidas depois do ano de shabat}

Por esta proibição somos proibidos de cobrar as dívidas no Ano de
Shabat\textsuperscript{394}; elas devem ser perdoadas por
completo\textsuperscript{395}. Esta proibição está ex­pressa em Suas
palavras, enaltecido seja Ele, "Todo credor, que emprestou a seu
companheiro, o deixará; não reclamará a seu companheiro nem a seu
ir­mão" (Deuteronômio 15:2).

393 Em Núméros


\begin{enumerate}
\def\labelenumi{\arabic{enumi}.}
\setcounter{enumi}{393}
\item
 
 O sétimo ano
 
\item
 
 Ver o preceito positivo 141.
 
\end{enumerate}


De acordo com as Escrituras, este preceito é obrigatório apenas na Terra
de Israel nas ocasiões em que a exoneração da terra estiver em vigor
ali, ou seja, no Jubileu. Pela lei Rabínica, contudo, ele é obrigatório
em todos os lugares e para sempre, e não se permite que se peça o
pagamento de uma dívi­da depois do Ano de Shabat: a dívida .deve ser
cancelada.


As normas deste preceito estão explicadas no final do Tratado Shebiit.


\section{Não recusar um empréstimo que deva ser cancelado no ano de shabat}

Por esta.proibição somos proibidos de recusar um empréstimo por­que ele
será cancelado pelo Ano de Shabat\textsuperscript{396}. As Escrituras
proíbem tal tipo de relutância pelas palavras "Guarda-te que não haja
uma coisa perversa no teu coração, nem digas..." (Deuteronômio 15:9). A
esse respeito diz o Sifrei: " `1-lis­hamer' (guarda-te) é um preceito
negativo: `pen yi-yeh' (que não haja) também é um preceito negativo".
Quer dizer, a finalidade desses dois preceitos, coloca­dos um após o
outro, é dar maior ênfase.

232 NÃO DEIXAR DE FAZER CARIDADE A NOSSOS IRMÃOS NECESSITADOS

Por esta proibição somos proibidos de deixar de fazer caridade e de dar
assistência a nossos irmãos necessitados ao ficarmos cientes de sua
situação infeliz, sabendo que está erki nosso poder
ajudá-los\textsuperscript{397}. Esta proibição está ex­pressa em Suas
palavras, enaltecido seja Ele, "Não endurecerás teu coração, e não
fecharás tua mão a teu irmão o mendigo" (Deuteronômio 15:7). Isto nos
proíbe de agirmos de maneira avarenta e mesquinha a ponto de deixar de
dar aos necessitados.

\section{Não mandar embora de mãos vazias um servo hebreu}

Por esta proibição somos proibidos de mandar embora de mãos va­zias um
servo hebreu que nos serviu quando ele for liberto, ao final de seis
anos. Ao contrário, nós devemos oferecer-lhe presentes de nossa
propriedade\textsuperscript{398}. Es­ta proibição está expressa em Suas
palavras, enaltecido seja Ele. "Quando o dei­xares ir livre de ti, não o
deixarás ir vazio" (Deuteronômio 1 5:1 3).

As normas deste preceito, relativo aos presentes, está explicada no
primeiro capítulo do Tratado Kidushin.

396 O sétimi)..1 )


397 Ver o preceito positivo 195.\\
398. Ver o preceito positivo 196.


PRECEITOS NEGATIVOS .97

\section{Não cobrar uma dívida de alguém que se sabe que não pode pagar}

Por esta proibição ficamos proibidos de cobrar uma dívida de alguém que
sabemos que não pode pagá-la. Ela está expressa em Suas palavras,
enalte­cido seja Ele, ``Não serás para ele como credor'' (Êxodo 22:24).

Na Guemará de Baba Metzia lemos: 'De que forma sabemos que se alguém
emprestou um ``maneh''\textsuperscript{399} a seu vizinho e souber que ele
não tem na­da, ele nem deve passar diante dele? Pelas palavras das
Escrituras 'Não serás para ele como credor' " . E a Mekhiltá diz: " 'Não
serás para ele como credor': você não deve estar sempre diante dele".

Vocês devem saber que esta proibição se aplica também quanto a pedir o
pagamento dos juros de uma dívida e por essa razão a Mishná diz que
aquele que emprestar seu dinheiro a juros também estará infringindo Suas
pala­vras ``Não serás para ele como credor'', como explicarei adiante.

\section{Não emprestar a juros}

Por esta proibição somos proibidos de emprestar a juros. Ela está
expressa em Suas palavras "Teu dinheiro não lhe darás com lucro
(neshekh), e com usura (marbit) não lhe darás tua comida" (Levítico
25:37).

Esta repetição da proibição de uma única ofensa lhe dá maior força,
fazendo com que aquele que empresta a juros seja culpado de infringir
duas proibições. Elas não são duas ofensas, pois -` neshekh" e ``ribit''
são a mesma coisa. A Guemará de Baba Metzia diz: "Você não vai encontrar
'neshekh' sem `tarbit', nem `tarbit' sem `neshekh', e o único objetivo
das Escrituras ao men­cionar cada um deles separadamente
é\textsuperscript{4(}H. que se transgride duas
proibições". Também diz ali: "Nas Escrituras 'neshekh' e`tarbit' são
sinônimos"; e ainda: "Como está escrito 'Teu dinheiro não lhe darás com
``neshekh'' e com "mar­bit" não lhe darás tua comida', leia-se isso da
seguinte forma: 'Teu dinheiro não lhe darás com ``neshekh'' e com
``marbit'', e com ``neshekh'' e com "mar­bit" não lhe darás tua comida' "

Dessa forma, aquele que emprestar dinheiro ou provisões com ju­ros
transgredirá dois preceitos, além das outras proibições que também são
fei tas a quem empresta, para dar maior ênfase, pois esta proibição está
repetida sob outra forma em Suas palavras "Não tomarás dele lucro nem
usura" (Ibid., 36). Como está explicado na Guemará de Baba Metzia, esta
também é uma proi­bição imposta aos que emprestam, mas como explicamos
no Nono Fundamen­to, todas essas proibições são redundantes, pois elas
são repetições do preceito que proíbe emprestar com juros.


As normas deste preceito estão explicadas no quinto capítulo de Ba


ba Metzia.

399 Cem .shekalim''.

400. Para ensinar-nos que transgride. etc.

\section{Não tomar emprestado com juros}

Por esta proibição quem tomar emprestado também fica proibido de fazè-lo
com juros, porque se não houvesse uma proibição imposta também àquele
que pede emprestado, proibindo-o de fazê-lo com juros, poderíamos
imaginar que só peca aquele que empresta, porque ele prejudica alguém, e
que aquele que pede emprestado, submetendo-se a ser prejudicado, não
estaria co­metendo nenhum pecado. Este caso seria semelhante ao de um
prejuízo, onde aquele que prejudica peca, mas o prejudicado não. Por
essa razão se impõe- uma proibição também sobre quem pede emprestado, o
qual está proibido de fazê-lo com juros, de acordo com Suas palavras,
enaltecido seja Ele, "Lo tashikh le'ahikha" (Não pagarás a teu irmão
juros) (Deuteronômio 23:20), que a Tradi­ção explica como significando:
Não deixes que nenhum juro te seja cobrado! E a Guemará de Baba Metzia
diz explicitamente: "Quem pede emprestado trans­gride `lo tashikh' e
'Diante do cego não porás tropeço' (Levítico 19:14)", como explicaremos
ao falar deste último preceito.

\section{Não participar de um empréstimo a juros}

Por esta proibição somos proibidos de participar de uma transação entre
quem pede emprestado e quem empresta que envolva juros, seja como
fiador, como testemunha ou para registrar o contrato entre eles para o
paga­mento de juros com os quais eles tenham concordado. Ela está
expressa em Suas palavras, enaltecido seja Ele, "Não porás juros sobre
ele" (Êxodo 22:24)

A Guemará de Baba Metzia diz: "O fiador e a testemunha transgri­dem
apenas 'Não porás juros sobre ele' " e está explicado ali que quem fizer
o registro estará na mesma situação que a testemunha e o fiador. Também
está explicado ali que embora o preceito ``Não porás juros sobre ele'' se
aplique apenas a terceiros numa transação. ele inclui também quem
empresta, e que conseqüentemente aquele que empresta a juros transgride
seis preceitos nega­tivos: um, ``Não serás para ele como credor'' (Êxodo
22:24): dois, "Teu dinhei­ro não lhe darás com lucro" (Levítico 25:37);
três, ``Com usura não lhe darás tua comida'' (Ibid.); quatro, "Não tomarás
dele lucro, nem usura" (Ibid., 36); cinco, ``Não porás juros sobre ele'';
e seis, ``Diante do cego não porás tropeço'' (Ibid., 19:14).

Também lemos ali: "Os que transgridem preceitos negativos são Os
seguintes: quem empresta, quem pede emprestado, o fiador e a testemunha.
Os Sábios acrescentam: quem registrar a dívida também. Eles transgridem
'Teu dinheiro não lhe darás', 'Não tomarás dele', 'Não serás para ele
como credor', 'Não porás juros sobre ele', e 'Diante do cego não porás
tropeço--- . E na Gue­mará lemos: "Abayé disse: aquele
que empresta infringe tOdos; aquele que to­ma emprestado, 'Não pagarás a
teu irmão juros' e 'Diante do cego não porás tropeço'; o fiador é a
testemunha, apenas 'Não porás juros sobre ele'

No caso de violação deste preceito, se o juro era ribit ketsutsa" (ju ro
fixo), ele deve ser tomado de quem emprestou e devolvido a pessoa de
quem ele o cobrou.


\section{Não oprimir um empregado atrasando o pagamento de seus salários}

Por esta proibição somos proibidos de prejudicar um trabalhador
atra­sando o pagamento de seus salários. Ela está expressa em Suas
palavras "Não ficará a paga de um jornaleiro contigo até pela manhã"
(Levítico 19:13). Isto se refere, como mostram as palavras "até pela
manhã", a um trabalhador con­tratado para o dia, que pode pedir seus
soldos a qualquer momento da noite; mas um trabalhador contratado para a
noite, que pode pedir pagamento de seu soldo durante toda a noite e todo
o dia, deve ser pago antes do anoitecer, de acordo com Suas palavras "No
seu dia, lhe pagarás a sua diária, e isto o farás antes do pôr-do-sol"
(Deuteronômio 24:15). Como diz a Mishná: "Um traba­lhador contratado
para o dia pode receber a qualquer hora durante a noite; um contratado
para a noite pode receber a qualquer hora durante o dia".

Esses dois versículos não contém dois preceitos, e sim apenas um e o
objetivo das duas proibições é de complementar o enunciado da lei. A
par­tir das duas juntas ficamos sabendo qual é a hora de efetuar os
pagamentos.

As normas deste preceito estão explicadas no nono capítulo de Baba
Metzia, onde fica claro que é apenas no caso de um trabalhador israelita
contra­tado que aquele que atrasar o pagamento do soldo viola um
preceito negativo; no caso de um trabalhador não israelita, ele
transgride um preceito positivo, contido em Suas palavras "No seu dia,
lhe pagarás a sua diária" (Deuteronômio 24\textsubscript{:1}5)401

\section{Não tomar pela força um penhor de um devedor}

Por esta proibição somos proibidos de tomar pela força um penhor de um
devedor, a não ser por ordem de um juiz e através de seu emissário. Nós
próprios não devemos entrar na casa do devedor contra sua vontade e
pe­gar um penhor dele. Esta proibição está expressa em Suas palavras,
enaltecido seja Ele, "Não entrarás em sua casa para lhe tomar o seu
penhor" (Deuteronô­mio 24:10). Como a Mishná diz: "Se um homem empresta
a seu companheiro, ele pode pegar um penhor dele apenas através do
Tribunal e não deve entrar em sua casa para ir buscá-lo pois está
escrito 'Do lado de fora ficarás etc.' " (Ibid., 11).

Este preceito negativo está justaposto a um preceito positivo, que está
expresso em Suas palavras, enaltecido seja Ele, "Restituir-lhe-ás o
penhor" (Ibid., 13)\textsuperscript{402} e é assim que está explicado na
Guemará de Macot.

Mas vocês devem saber que se ele não o devolver, deixando de cum­prir o
preceito positivo relativo a isso, ele estará sujeito ao açoitamento e
deve­rá pagar pelo penhor, como está explicado no final de Macot.


As normas deste preceito estão explicadas no nono capítulo de Baba
Metzia.

• 401. Ver o preceito positivo 200. 402. Ver o preceito positivo 199.



\section{Não ficar com um penhor do qual seu proprietário precise}

Por esta proibição somos proibidos de ficar com um penhor quan­do seu
proprietário precisar dele; devemos devolver de dia o artigo usado
du­rante o dia, e a noite o artigo usado durante a noite, como diz a
Mishná: "Ele deve devolver um travesseiro à noite e um arado de dia". A
proibição a esse respeito está expressa em Suas palavras, enaltecido
seja Ele, ``Não passarás a noite com o seu penhor'' (Deuteronômio 24:12),
sobre as quais o Sifrei diz: " 'Não passarás a noite' enquanto seu
penhor estiver contigo" e deverás devolver-lhe aquilo sem o qual ele não
pode passar, por causa de sua pobreza, como Ele explicou nas palavras
"Pois só esta é a sua coberta, a túnica para a sua pele!" (Êxodo 22:26).

As normas deste preceito estão explicadas no nono capítulo de Baba
Metzia.

\section{Não pegar um penhor de uma viúva}

Por esta proibição somos proibidos de pegar um penhor de uma viú­va,
quer seja ela pobre ou rica. Ela está expressa em Suas palavras "Não
tomarás em penhor a roupa da viúva" (Deuteronômio 24:17). Como diz a
Mishná: "Não se deve pegar uma garantia de uma viúva, seja ela pobre ou
rica, pois está escri­to 'Não tomarás em penhor a roupa da viúva' ".

As normas deste preceito estão explicadas no nono capítulo de Baba
Metzia.

\section{Não pegar como penhor utensílios usados para a alimentação}

Por esta proibição somos proibidos de pegar como penhor utensí­lios que
são usados na preparação dos alimentos, tais como vasilhas para moer,
amassar ou cozinhar, apetrechos para o abate de gado, e todos os outros
obje­tos que estão na categoria de "utensílios necessários para a
preparação de ali­mentos". Esta proibição está expressa em Suas
palavras, enaltecido seja Ele, "Não lhe tomará em penhor nem a mó
abaixo, nem a mó de em cima, porque são coisas com as quais se elabora o
alimento do homem" (Deuteronômio 24:6), sobre as quais a Mishná diz:
"Isto significa não apenas a mó de baixo e a mó de cima, mas qualquer
coisa utilizada na preparação de comida para o consu­mo do homem, pois
está escrito: 'Porque são coisas com as quais se elabora o alimento do
homem' ".

Resta-nos explicar-lhes a afirmação "É culpado por causa de dois
uten­sílios, pois está dito que São lhe tomará em penhor nem a mó
abaixo, nem a mó de em cima' ". Vocês poderiam deduzir que aqui há dois
preceitos sepa­rados, e tal dedução seria ainda mais prontamente
confirmada pela afirmação deles de que "Ele é culpado por causa da mó de
baixo e por causa da mó de cima, separadamente". Mas o significado
dessas afirmações é o seguinte.

Aquele que pegar como penhor um utensílio com o qual se prepara o
alimento necessário infringe um preceito negativo, como foi explicado.
Aque­le que pegar como penhor vários utensílios, todos eles usados na
preparação do alimento necessário --- como por exemplo para moer ou para
assar, ou para amassar --- é culpado por cada um deles separadamente.
Este ponto não necessita de explicação, pois esse caso é semelhante ao
de pegar como pe­nhor as vestes da viúva de Reuben, as da viúva de
Shimon e as da viúva de Levi, quando se cometeria um pecado separado em
relação a cada uma das vestes.

Mas uma dúvida pode surgir quando se pegar como penhor dois uten­sílios
que realizam juntos uma única operação no preparo do alimento
necessá­rio, sendo um insuficiente sem o outro. Devemos dizer que, uma
vez que a co­mida só pode ser completamente processada com o uso dos
dois utensílios jun­tos, eles constituem um único instrumento e o
transgressor é culpado por cau­sa de um utensílio apenas? Ou devemos
dizer que, como há dois utensílios, ele é culpado por causa de cada um
deles separadamente? Foi-nos explicado que ele é culpado por causa de
dois utensílios, embora o trabalho seja executado pelos dois juntos,
como no caso da mó de baixo e da mó de cima, sendo que a moagem seria
impossível se faltasse qualquer uma das duas pedras, e que pe­gar as
duas mós como penhor seria como pegar a tina de amassar e a faca de
degolar, sendo que cada uma delas tem uma finalidade específica. Este é
o sig­nificado da afirmação de que ele é culpado por causa de dois
utensílios; ela não significa que haja dois preceitos separados.

Eis o que o Sifrei nos diz a respeito do assunto que acabo de expli­car
a vocês: "Assim como a mó de baixo e a mó de cima são diferenciadas como
sendo dois utensílios que executam juntos uma única operação, e se
incorre num castigo por cada um deles em separado, também no caso de
to­das as coisas compostas de dois utensílios usados juntos numa
operação incorre-se num castigo por cada um deles separadamente". O
significado e o pro­pósito desta afirmação são que, embora eles sejam
usados na realização de uma única operação, fica-se sujeito a um castigo
por cada um deles sepa­radamente.

Aquele que transgredir e tomar como penhor um utensílio desses deve
colocá-lo à disposição e devolvê-lo a quem o usa. E se ele tiver sido
per­dido ou queimado antes de sua restituição, ele será punido com o
açoita­mento. O mesmo se aplica àquele que pegar as roupas de uma viúva
como penhor.

As normas deste preceito estão explicadas no nono capítulo de Baba
Metzia.

\section{Não raptar um israelita}

Por esta proibição somos proibidos de raptar um israelita. Ela está
expressa em Suas palavras, nos Dez Mandamentos, ``Não furtarás'' (Êxodo
20:15), sobre as quais diz a Mekhiltá: " 'Não furtarás': esta é a
proibição de raptar". A Guemará de Sanhedrin diz: "De que forma
deduzimos a proibição de raptar? Rabi Yoshiá disse: De 'Não furtarás';
Rabi Yohanan disse: De 'Não se venderão como são vendidos os servos'
(Levít4co 25:42). Mas não há disputa entre eles; um Sábio determina a
proibição de roubar, e o outro Sábio a proibição de ven­der", já que não
se impõe o castigo a menos que o transgressor rapte e venda
e a penalidade por infringir essas duas proibições é o estrangulamento.
Suas palavras, enaltecido seja Ele, são "Aquele que roube um homem e o
vender, e for encontrado em sua mão, será certamente morto" (Êxodo 21:16).

As normas deste preceito estão explicadas no décimo primeiro capí­tulo de Sanhedrin.

\section{Não furtar dinheiro}

Por esta proibição somos proibidos de furtar dinheiro. Ela está
ex­pressa em Suas palavras ``Não furtareis'' (Levítico 19:11), sobre as
quais a Mek­hiltá diz: " 'Não furtareis': esta é a proibição, de furtar
dinheiro".

Aquele que infringir esta proibição deverá, de acordo com o que es­tá
prescrito nas Escrituras, fazer uma restituição dobrada, ou
quadruplicada ou quintuplicada\textsuperscript{403}, ou simplesmente
devolver o que furtou.

A Sifrá diz: "Pelas palavras, relativas ao furto, 'Pagará o dobro'
(Êxo­do 22:3), nós sabemos qual é a penalidade; mas como ficamos sabendo
da proi­bição? Pelas palavras das Escrituras 'Não furtareis', mesmo que
o objetivo seja aborrecer' ", ou seja, mesmo que o objetivo seja irritar
o proprietário e causar-lhe problemas e devolvê-lo a ele depois. " 'Não
furtareis' nem com o objetivo de pagar uma restituição quadruplicada ou
quintuplicada".

As normas deste preceito estão explicadas no sétimo capítulo de Ba­ba
Kamma.

\section{Não cometer um roubo}

Por esta proibição somos proibidos de cometer um roubo, isto é, de tomar
abertamente pela força e violência qualquer coisa a que não tenha­mos
direito. Ela está expressa em Suas palavras, enaltecido seja Ele, "Não
ex­torquirás" (Levítico 19:13). A Tradição também explica: " 'Não
extorquirás (tig­zol)': é como na expressão. 'Ele ``arrancou'' (vayigzol)
a lança das mãos dos Egípcios' "404.

Este preceito negativo está justaposto ao preceito positivo, expres­so
em Suas palavras ``Devolverá o que roubou'' (Levítico
5:23)\textsuperscript{405}; mas mes­mo que ele elimine o preceito
positivo, ele não será punido com o açoitamen­to, uma vez que um homem
não pode ser castigado com ambos o açoitamento
 a restituição, já que este é um preceito negativo sujeito a
 restituição.


Portanto, se o ladrão queimar o artigo roubado ou o jogar no mar, ele
deverá pagar o seu valor; e.se ele negar a culpa sob juramento ele deve
acres- • centar um quinto e deve levar um Sacrifício de Delito, como
está explicado no lugar
apropriado.".. Esta é a explicação
dada no final de Macot.

As normas deste preceito estão explicadas no nono e no décimo ca­pítulos
de Baba Kamma.


\begin{enumerate}
\def\labelenumi{\arabic{enumi}.}
\setcounter{enumi}{402}
\item
 
 Êxodo 21:37.
 
\item
 
 II Samuel 23:21.
 
\item
 
 Ver o preceito positivo 194.
 
\item
 
 Ver o preceito positivo 71.
 
\end{enumerate}



\section{Não alterar os limites das terras fraudulentamente}

Por esta proibição somos proibidos de alterar os limites das terras
fraudulentamente, isto é, deslocar as marcas entre nossas terras e as do
vizinho, de maneira a reivindicar como nossas as terras de outro. Esta
proibição está expressa em Suas palavras, enaltecido seja Ele, "Não
removerás o limite da he­rança, de teu companheiro" (Deuteronômio
19:14); a esse respeito o Sifrei diz: " 'Não removerás o limite da
herança de teu companheiro'; mas já não foi dito `Não extorquirás teu
próximo'? (Levítico 19:13). Então por que acrescentar 'Não removerás'?
Para ensinar-nos que aquele que usurpar as terras de seu vizinho
transgride dois preceitos. Eu poderia pensar que isto se aplica também
fora da Terra de Israel, por isso as Escrituras dizem: 'Na herança que
herdares'. Na Ter­ra de Israel desobedece-se dois preceitos negativos,
mas fora dela apenas um", que é o ``Não extorquirás teu próximo''. Fica
assim demonstrado que este pre­ceito negativo se aplica apenas à Terra
de Israel.

\section{Não usurpar nossas dívidas}

Por esta proibição somos proibidos de usurpar as dívidas que fizer­mos,
isto é, recusar-nos a pagá-las e ficar com seu valor. Ela está expressa
em Suas palavras, enaltecido seja Ele, "Não sonegarás teu próximo e não
extorqui­rás" (Levítico 19:13).

``Furtar'' é o ato de tirar os pertences de alguém por meios traiçoei­ros
e secretos, e está proibido por Suas palavra,« 'Não furtareis" (Ibid.,
19:11), como explicamos. A ``extorsão'' é o ato de tirar os pertences de
alguém por meio de força e violência, como fazem os ladrões nas
estradas, e está proibido por Suas palavras ``Não extorquirás''
(Ibid.,13). A ``sonegação'' ocorre quando você deve uma determinada
importância a alguém, quer dizer, você tem em seu poder e é responsável
por uma quantia de dinheiro que pertence a outra pessoa, e não a devolve
a ela, utilizando-se de força ou apenas de subterfúgios e trapaças. Tal
conduta também está proibida pelas Suas palavras, enaltecido seja Ele,
``Não sonegarás teu próximo'' que são explicadas da seguinte forma na
Sifrá: " 'Não sonegarás teu próximo' significa enganá-lo com relação a
di­nheiro, como por exemplo, reter o salário de um empregado" e todos os
atos similares. O salário de um empregado é dado como exemplo apenas
porque ele representa uma dívida claramente sua, embora ele não tenha
lhe dado ne­nhum dinheiro dele e você não tenha recebido dinheiro algum
dele; apesar dis­so, como você tem uma dívida certa para com ele, você
está proibido de recusar-lhe o dinheiro.

A proibição relativa a este assunto está repetida, e este caso
específi­co foi dado como exemplo em Suas palavras "Não defraudarás o
jornaleiro po­bre e necessitado" (Deuteronômio 24:14), que significam:
"Não oprimirás um empregado PORQUE ele é pobre e necessitado' como Ele
diz mais adiante: "E isto o farás antes do pôr-do-sol, porque é pobre"
(Ibid.,15).

O Sifrei diz: " 'Não defraudarás o jornaleiro pobre e necessitado'; já
não foi dito 'Não extorquirás'? É para ensinar-nos que aquele que
retiver o salá­rio de um empregado transgredirá São defraudarás', São
extorquirás' e 'Não ficará a paga de um jornaleiro contigo' (Ibid.) e
'No seu dia, lhe pagarás a sua diária' " 
(Ibid.)\textsuperscript{407}. E acrescenta, interpretando Suas
palavras "pobre e necessitado
": 'Eu puno mais rapidamente quando se trata dos pobres e dos
necessitados".


A lei é a mesma para aquele que engana como para aquele que extor­que;
Ele diz, enaltecido seja Ele, "E negar ao seu companheiro a coisa que
lhe foi entregue sob custódia, ou um empréstimo em dinheiro, ou roubou,
ou ex­torquiu o seu companheiro (etc)" (Levítico 5:21).

\section{Não negar nossas dívidas}

Por esta proibição somos proibidos de negar nossas dívidas ou aqui­lo
que tenha sido confiado a nós. Ela está expressa em Suas palavras "Não
en­ganareis" (Levítico 19:11) que se referem, como está explicado, a
transações de dinheiro.

A Sifrá diz: "Pelas palavras 'E negou, e jurou em falso (etc)" (Ibid.,
5:22), ficamos sabendo do castigo. De que forma ficamos sabendo da
proibi­ção? Pelas palavras das Escrituras ``Não enganareis''.

Vocês já sabem que aquele que negar algo que ele tiver em depósito está
desqualificado como testemunha, mesmo que ele não o tenha feito sob
juramento, porque ele terá transgredido Suas palavras, enaltecido seja
Ele, 'Não enganareis' .

As normas deste preceito estão explicadas em vários trechos do Tra­tado
Shebuot.

\section{Não jurar em falso ao negar uma dívida}

Por esta proibição somos proibidos de jurar em falso ao negar uma dívida
nossa. Ela está expressa em Suas palavras "E não mentireis cada um ao
seu companheiro" (Levítico 19:11). Por exemplo: aquele que negar
falsamente ter recebido algo em custódia transgride Seu preceito,
abençoado seja Ele, ``Não enganareis'' (Ibid.); se a isso ele acrescentar
um falso juramento, ele transgride o preceito ``Não mentireis''.

A Sifrá diz: " `Não mentireis cada um ao seu companheiro': qual é a
finalidade disto? Pelas palavras "E jurou em falso, (etc)" (Ibid., 5:22)
conhe­cemos o castigo; mas como ficamos sabendo da proibição? Pelas
palavras das Escrituras: São mentireis' ".

As normas deste preceito estão explicadas no quinto capítulo de
She­buot, onde fica claro que aquele que jurar um falso ao negar uma
dívida trans­gride dois preceitos negativos: "Não jurareis falso em Meu
nome" (Ibid., 19:12) e ``Não mentireis cada um ao seu companheiro''.

\section{Não enganar um ao outro em negócios}

Por esta proibição somos proibidos de enganar um ao outro em ne­gócios
de compra e venda. Ela está expressa em Suas palavras, enaltecido seja
Ele, "Quando fizerdes uma venda a vosso companheiro, ou comprardes da
mão

407. Ver o preceito positivo 200.


de vosso companheiro, não enganareis cada qual ao seu irmão" (Levítico
25:14), sobre as quais diz a Sifrá: " 'Não enganareis cada qual ao seu
irmão': enganar em termos de dinheiro".


As normas deste preceito estão explicadas no quarto capítulo de Ba-


ba Metzia.

\section{Não prejudicar um ao outro com palavras}

Por esta proibição somos proibidos de prejudicar um ao outro com
palavras, isto é, dizendo ao outro coisas que vão ferí-lo e humilhá-lo e
causar-lhe dor insuportável, tal como fazer uma pessoa lembrar dos erros
que come­teu na juventude, pelos quais ela já tenha se arrependido, e
dizer-lhe: "Agrade­ça aos Céus por tê-lo desviado de tal e tal
procedimento para conduzí-lo a seu atual estilo virtuoso de vida"; ou
como fazer referências cruéis a respeito de seus defeitos físicos
graves. Esta proibição está expressa em Suas palavras, enal­tecido seja
Ele, "Não enganareis cada um ao seu companheiro; e temerás a teu Deus"
(Levítico 25:17), a respeito das quais diz o Talmud: "Isto se refere a
en­ganar com palavras", e a Sifrá: "As palavras das Escrituras 'Não
enganareis cada um ao seu companheiro' se referem a enganar com
palavras. O que isto signifi­ca? Se um homem se corrigiu, não se lhe
deve dizer 'Lembre-se de seu antigo comportamento'. Se um homem estiver
sofrendo, se ele estiver doente ou ti­ver enterrado seu filho, não se
lhe deve dizer o que o companheiro de Yob disse: 'Não é teu temor a Deus
a tua confiança, e tua esperança a totalidade de teus caminhos? Recorda,
eu te suplico, quem pereceu, sendo inocente'. Se vir­mos arrieiros em
busca de grãos não devemos dizer-lhes 'Tal pessoa vende grãos' se
soubermos que ela nunca os vendeu. Também não se deve dizer ao
proprie­tário, quando não se tiver dinheiro, 'Quanto custa este artigo?'
". O Talmud diz ainda: "O prejuízo verbal é mais odioso do que o
prejuízo monetário por­que está escrito, a respeito do prejuízo verbal:
'Temerás a teu Deus' ".


As normas deste preceito estão explicadas no quarto capítulo de Ba-


ba Metzia.

\section{Não enganar um prosélito com palavras}

Por esta proibição somos proibidos de enganar um prosélito com palavras.
Ela está expressa em Suas palavras, enaltecido seja Ele. "Ao imigrante
não o fraudareis" (Êxodo 22:20), sobre as quais diz a Mekhiltá: '"Ao
imigrante não o fraudareis': com palavras".

A proibição relativa a este assunto está repetida em Suas palavras "Não
o enganareis" (Levítico 19:33), que a Sifrá explica assim: "Não lhe
diga: 'On­tem você adorava ídolos e agora você está sob a proteção da
``Shekhiná'' (Pre­sença Divina)' ".

\section{Não enganar um prosélito nos negócios}

Por esta proibição somos proibidos de enganar um prosélito e
prejudicá-lo numa compra e venda. Ela está expressa em Suas palavras
``Não o oprimireis'' (Êxodo 22:20), sobre as quais diz a Mekhiltá: 
" 'Não o
oprimireis' --- no que se refere a dinheiro".

Está explicado na Guemará de Baba Metzia que aquele que engana um
prosélito transgride ``Não enganareis cada um ao seu companheiro''
(Leví­tico 25:17) e ``Não o fraudareis'' (Êxodo 22:20). E aquele que o
engana em as­suntos de dinheiro transgride ``Não o oprimireis'', além da
proibição na qual ele está incluído juntamente com todos os israelitas,
ou seja, a de enganar em assuntos de dinheiro.

\section{Não entregar um escravo fugitivo}

Por esta proibição somos proibidos de entregar a seu senhor um es­cravo
que foi buscar refúgio na Terra de Israel. Mesmo que seu senhor seja um
israelita, como ele fugiu do exterior para a Terra de Israel ele não
deve ser en­tregue, e o senhor deve libertá-lo, recebendo dele uma nota
de reconhecimen­to de dívida pelo seu valor. Esta proibição está
expressa em Suas palavras, enal­tecido seja Ele, "Não entregarás ao seu
senhor o escravo" (Deuteronômio 23:16). E está explicado no quarto
capítulo de Guitin que as Escrituras se referem aqui a um escravo que
tenha fugido do exterior em busca de refúgio na Terra de Israel e que a
lei diz que ele deve escrever ao seu senhor uma nota de reconhe­cimento
de dívida por seu valor, e que o senhor deve redigir-lhe um Estatuto de
Liberdade. Em circunstância alguma ele retornará a ser seu escravo pois
ele foi residir na terra pura, que foi escolhida para as pessoas
enaltecidas.

As normas deste preceito estão explicadas ali.

\section{Não enganar um escravo fugitivo}

Por esta proibição somos proibidos de enganar o escravo que fugiu e veio
a nós. Ela está expressa em Suas palavras, enaltecido seja Ele, "Contigo
ficará no meio de ti... não o enganarás" (Deuteronômio 23:17), a
respeito das quais diz novamente a Sifrá: " 'Não o enganarás' se refere
a enganá-lo com pa­lavras". Pois assim como o Enaltecido acrescentou uma
proibição contra enga­nar um prosélito por causa de sua condição de
pessoa infeliz e. sem amigos, Ele acrescentou também uma terceira
proibição contra enganar um escravo, que é ainda mais infeliz e humilde
que o prosélito, para, que vocês não digam: Este escravo não se
incomodará se eu o enganar com palavras.

Está claro que o escravo referido aqui pelas Escrituras e o prosélito a
quem somos proibidos de enganar são pessoas que aceitaram a Torah como
sua lei, quer dizer, são Prosélitos Virtuosos.

\section{Não ser rude com crianças órfãs e com viúvas}

Por esta proibição somos proibidos de tratar de maneira rude as
crian­ças órfãs e as viúvas. Ela está expressa em Suas palavras,
enaltecido seja Ele, ``A nenhuma viúva ou órfão afligireis'' (Êxodo
22:21).

Esta proibição inclui o tratamento áspero por palavras ou atos. De­vemos
falar com eles de maneira muito gentil e amável, tratá-los tão bem
quan­to nos for possível, mostrar-lhes nossa boa vontade em relação a
eles e estabe­lecer para nós mesmos um padrão superior para lidar com
todos estes assun­tos. Todo aquele que violar qualquer uma dessas coisas
estará violando este preceito negativo, e o Enaltecido decretou
claramente seu castigo em Suas pa­lavras: "Acender-se-á a Minha ira, e
matar-vos-ei com a espada" (Ibid., 23).

\section{Não utilizar um servo hebreu para executar tarefas degradantes}

Por esta proibição somos proibidos de utilizar um escravo hebreu para
executar tarefas domésticas degradantes, como as que são executadas
pe­los escravos cananeus. Ela está expressa em Suas palavras, enaltecido
seja Ele, ``Não o farás servir com serviço de escravo'' (Levítico 25:39).
A Sifrá diz: "Ele não deve levar um `belinta' atrás de você, nem levar
suas coisas antes de você ao banho". Um ``belinta'' é uma pequena esteira
onde se senta ou se descansa quando se está cansado depois de ter feito
exercício físico e que o escravo leva para seu senhor. É proibido impor
qualquer tarefa de tal natureza a um escravo hebreu, que só deve ser
empregado em trabalhos como os que são executados por um trabalhador
contratado ou por um artesão, com o acordo de seu em­pregador. Isto está
estabelecido em Suas palavras "Como jornaleiro, como imi­grante, estará
contigo (Ibid., 40).

% NOta: 2 \textsubscript{58} 
\section{Não vender um servo hebreu em leilão}

Por esta proibição somos proibidos de vender um servo hebreu de maneira
como são vendidos os escravos, ou seja, pô-lo à venda em leilão no
mercado de escravos. Isso não deve ser feito em hipótese alguma; deve
ser fei­to em local fechado e em condições adequadas. A proibição
relativa a este as­sunto está expressa em Suas palavras, enaltecido seja
Ele, "Não se venderão co­mo são vendidos os servos" (Levítico 25:42),
sobre as quais diz a Sifrá: " 'Não se venderão como são vendidos os
servos': não se deve colocar uma platafor­ma e pô-lo na pedra de
leilão".

Este preceito negativo inclui sem dúvida alguma a proibição de rap­tar
um israelita porque se alguém o vender, o fará como faria comum escravo
cananeu, transgredindo assim Suas palavras "Não se venderão como são
vendi­dos os servos". Nós já nos referimos a esse assunto anteriormente
E a Torah deixa claro que quem o fizer será morto\textsuperscript{40}".

As normas deste preceito e as dos precedentes estão explicadas no
primeiro capítulo da Guemará de Kidushin.

\section{Não utilizar um servo hebreu para fazer um trabalho desnecessário}

Por esta proibição somos proibidos de utilizar um servo hebreu num
trabalho desnecessário, que é chamado de ``trabalho inclemente''. Esta
proibi­ção está expressa em Suas palavras, enaltecido seja Ele, "Não
dominarás sobre ele com rigor" (Levítico 25:43). Não devemos fazê-lo
trabalhar a não ser quan­do somos forçados a isso pela necessidade de
que um trabalho específico seja feito. A Sifrá comenta: " 'Não dominarás
sobre ele com rigor' significa que não lhe dirás 'Aqueça-me esta bebida'
se isso não for necessário", e coisas similares. O exemplo dado é a
menor e a mais simples tarefa, contudo nem mesmo isso é permitido a não
ser que seja necessário.

\section{Não permitir que se maltrate um servo hebreu}

Por esta proibição somos proibidos de permitir que um pagão que more em
nossa terra seja severo com um servo hebreu que se vendeu a ele. Esta
proibição está expressa em Suas palavras, enaltecido seja Ele, "Não
domi­nará sobre ele, com rigor, à tua vista" (Levítico 25:53). Não
devemos dizer que, uma vez que esse hebreu pecou contra si mesmo e se
vendeu a um pagão, nós o deixaremos sofrer as conseqüências de seu ato.
Devemos controlar o pagão e evitar que ele seja severo. A Sifrá diz: "
'Não dominará sobre ele com rigor, à tua vista': o preceito se aplica
apenas quando ele estiver à \emph{tua vista.} Quer dizer, não somos
obrigados a supervisioná-lo quando ele estiver em sua pró­pria casa para
verificar se ele está sendo severo ou não; mas toda vez que o virmos
fazendo isso, devemos proibí-lo de agir assim.


\section{Não vender uma serva hebréia}


Por esta proibição o proprietário de uma serva hebréia fica proibido de
vendê-la. Ela está expressa em Suas palavras, enaltecido seja Ele, "Não
a po­derá vender após ter-se servido dela" (Êxodo 21:8).

As normas deste preceito estão explicadas em sua totalidade no iní­cio
do Tratado Kidushin.

\section{Não privar uma serva hebréia que se desposou}

Por esta proibição o proprietário de uma serva hebréia que a tenha
desposado fica proibido de privá-la --- e por isso quero dizer diminuir
sua co­mida, roupas ou direitos conjugais --- de maneira tal a
causar-lhe dor e sofri­mento. Esta proibição está expressa em Suas
palavras, enaltecido seja Ele, "Sua manutenção, seu vestuário, e o seu
direito conjugal não lhe diminuirá" (Êxodo 2 1 : 1 O).

Este preceito negativo também se aplica a todo aquele que se casar com
uma israelita para que não a prive de nenhuma dessas três coisas a fim
de causar-lhe dor e infelicidade. Por Suas palavras, enaltecido seja Ele,
relativas a uma serva hebréia, proibindo-nos de diminuir sua comida,
suas roupas ou seus direitos conjugais, e pelo fato de acrescentar
"Trata-la-á como se tratam as fi­lhas" (Ibid., 9) ficamos sabendo que
``Como se tratam as filhas'' significa que não devemos diminuir sua
comida, roupas ou direitos conjugais. Isso está
%{}{esti­}pulado.na} 
Mekhiltá: "O que este
texto nos ensina com relação a 'como se tra­tam as filhas'? Ele serve
para esclarecer outro texto, mas na realidade ele é auto • explicativo".
Também está dito ali: " Sheerá. significa seu
alimento'; 'quessu­tá', sua roupa, no sentido literal; 'onatá', seus
direitos conjugais".

\section{Não vender uma prisioneira}

Por esta proibição somos proibidos de vender uma mulher formosa depois
de tê-la tomado, por ocasião da captura de uma cidade, como está
expli­cado no lugar apropriado. Esta proibição está expressa em Suas
palavras, enal­tecido seja Ele, "E se não a quiseres, a deixarás ir em
liberdade; e não a vende­rás por dinheiro" (Deuteronômio 21:14).


\section{Não escravizar uma prisioneira}


Por esta proibição somos proibidos de escravizar uma mulher for­mosa
depois de tê-la tomado; quer dizer, não se deve fazer dela sua escrava e
tratá-la como outras servas que fazem trabalhos vis. Esta proibição está
ex­pressa em Suas palavras, enaltecido seja Ele, "Não te servirás dela,
porque a afligiste" (Deuteronômio 21:14), sobre as quais diz o Sifrei: "
'Não te servirás dela' significa que não a 'utilizarás' ".

Ficou assim claro que estes dois preceitos negativos proíbem duas coisas
diferentes: vendê-la a outra pessoa e guardá-la, tratando-a como
escrava. Deve-se observar Seu preceito, enaltecido seja Ele, "A deixarás
ir em liberda­de" (Ibid.). Assim também está explicado o texto relativo
ao raptor: "E se ser­vir dele, e depois o vender" (Ibid., 24:7): "Ele
não incorre em nenhuma culpa a menos que o tenha sob seu próprio
controle e o ponha a seu serviço".


As normas relativas à mulher formosa estão explicadas no início de


Kidushin.

\section{Não planejar obter a propriedade de outrem}

Por esta proibição somos proibidos de ocupar nossas mentes com planos
para obter o que pertence a um de nossos irmãos. Ela está expressa em
Suas palavras, enaltecido seja Ele, "Não cobiçarás a casa de teu
próximo" (Êxo­do 20:17), sobre as quais diz a Mekhiltá:
--- Não cobiçarás': eu poderia pensar que se refere à
simples expressão de um desejo, por isso as Escrituras dizem `Não
cobiçarás a prata e o ouro que está sobre eles, nem os \emph{tomarás}
para tí' (Deuteronômio 7:25). Assim como naquele caso, neste também é
apenas quan­do se coloca o desejo em prática".

Ficou dessa forma claro que este preceito negativo nos proíbe de
planejar obtermos algo que cobicemos e que pertença aos nossos irmãos,
meti mo que o compremos e que paguemos por ele seu preço real. Qualquer
ato desse tipo é uma desobediência a ``Não cobiçarás''.

\section{Não cobiçar os pertences de outrem}

Por esta proibição somos proibidos de concentrar nossos pensamen­tos na
cobiça e no desejo de coisas que pertençam a outra pessoa, porque isso
levará a planejar obtê-las. As palavras a esse respeito são: "E não
desejarás a casa do teu próximo" (Deuteronômio 5:18).

Esses dois preceitos negativos não se referem ao mesmo assunto. O
primeiro, ``Não cobiçarás'' (Êxodo 20:17), proíbe obter efetivamente o que
per­tence a outra pessoa; o segundo nos proíbe até mesmo desejá-lo e
cobiçá-lo. A Mekhiltá diz: "Aqui está dito 'Não \emph{cobiçarás} a casa
do teu próximo' e, mais adiante, 'Não \emph{desejarás} a casa do teu
próximo'. Portanto, incorre-se na culpa por desejar apenas, bem como por
apenas cobiçar". Também diz: "Como sa­bemos que se alguém começar por
desejar, .ele acabará por cobiçar? Porque as Escrituras dizem: 'Não
cobiçarás .., e não desejarás' (Deuteronômio 5:18). Como sabemos que se
alguém começar por cobiçar, ele acabará por roubar usan­do de violência?
Porque as Escrituras dizem: 'Eles cobiçam campos, e se apo­deram
deles'\textsuperscript{409}.

A explicação disto é que se alguém vir um belo objeto que pertença a seu
irmão e se interessar por ele e passar a desejá-lo, ele estará
infringindo a proibição expressa em Suas palavras, enaltecido seja Ele,
``Não desejarás''. Então seu amor pelo objeto ficará cada vez mais forte
até que ele comece a arquitetar um plano para obtê-lo, e não cesse de
pedir e pressionar o proprietário para que o venda a ele ou o dê em
troca de algo melhor e mais valioso; se ele o conseguir, estará dessa
forma infringindo outro preceito, que é o "Não cobiça­rás", pois devido
à sua persistência e ardis ele obteve algo que o proprietário não
desejava vender. Dessa forma ele infringiu dois preceitos, como
explica­mos. Contudo, se o proprietário se recusar a vender ou trocar o
objeto, por sua grande estima por ele, e se aquele que o cobiçar, devido
a seu grande dese­jo de tê-lo, o tomar pela força e coação, ele também
transgredirá o preceito negativo ``Não extorquirás'' (Levítico 19:13).
Para compreender isto vocês de­veriam ler a história do rei Ah-Ab e
Nabot\textsuperscript{41}".

Agora deve estar clara para vocês a diferença entre ``Não desejarás'' e
``Não cobiçarás''.

\section{Um trabalhador contratado não pode comer das plantações em crescimento}

Por esta proibição um trabalhador contratado fica proibido de co­mer das
plantações em crescimento entre as quais ele esteja trabalhando. Ela
está expressa em Suas palavras, enaltecido seja Ele, "Foice não porás na
seara de teu companheiro" (Deuteronômio 23:26), sobre as quais diz o
Talmud: " 'Foi­ce': isto estende a lei a tudo o que necessita de foice
e' época da cortar com a foice", ou seja, na época da colheita não deves
colher para ti mesmo.


É sabido que este versículo se refere apenaS a um trabalhador con

\begin{enumerate}
\def\labelenumi{\arabic{enumi}.}
\setcounter{enumi}{408}
\item
 
 Micah \emph{2:2.}
 
\item
 
 Reis 1, cap. 21.
 
\end{enumerate}



tratado e que Suas palavras "Quando entrares (Ibid.) significam "quando

um trabalhador entrar", como o Targum o traduz: "Quando fores
contratado".

No sétimo capítulo de Baba Metzia lemos: "Eles podem comer de acordo com
a lei das Escrituras: daquilo que está no solo, e para o qual ele foi
contratado, depois que o trabalho estiver terminado".

As normas deste preceito estão explicadas nesse capítulo.

\section{Um trabalhador contratado não pode servir-se em demasia}

Por esta proibição um trabalhador contratado fica proibido de pe­gar
mais das plantações nas quais ele estiver trabalhando do que aquilo que
ele necessitar para sua refeição. Ela está expressa em Suas palavras,
enaltecido seja Ele, "Poderás comer uvas conforme teu desejo, até te
fartares, porém na tua bolsa não porás" (Deuteronômio 23:25).

As normas deste preceito estão explicadas no sétimo capítulo de Baba
Metzia, onde também está explicado o que lhe é e o que não lhe é
permitido comer, e que ele não pode comer sem violar a proibição "Na tua
bolsa não porás".

\section{Não ignorar uma propriedade perdida}

Por esta proibição somos proibidos de fechar nossos olhos a uma
propriedade perdida; devemos recolhê-la e devolvê-la a seu proprietário.
Esta proibição está expressa em Suas palavras, enaltecido seja Ele, "Não
farás co­mo se não os visses" (Deuteronômio 22:3). Nós já citamos o que
a Mekhiltá diz com relação à propriedade perdida: "Aprendemos assim que
se viola um preceito positivo e um preceito negativo". E a Guemará diz:
"Devolver uma propriedade perdida apoia-se sobre um preceito positivo e
sobre um preceito negativo".

Este assunto aparece novamente no Deuteronômio, onde há um pre­ceito
negativo separado em Suas palavras "Vendo o boi de teu irmão, ou o seu
cordeiro, extraviados, não farás como se não os visses" (Ibid., 1),
sobre as quais diz o Sifrei: " 'Vendo ...' é um preceito negativo"; e
diz ainda, mais adiante: "Quando encontrares' (Êxodo
23:4)\textsuperscript{-111} é um preceito positivo

As normas deste preceito estão explicadas no segundo capítulo de Baba
Metzia.

\section{Não abandonar uma pessoa sobrecarregada}

Por esta proibição somos proibidos de abandonar alguém que esteja
sobrecarregado e se atrase na estrada. Devemos ajudá-lo retirando dele
sua car ga até que ele possa arrumá-la e devemos ajudá-lo a colocá-la
nas costas ou so­bre seu animal, como está explicado nas normas deste
preceito. A proibição está expressa em Suas palavras "Não te recusarás a
ajudá-lo" (Êxodo 23:5), so-

4 1 1 Ver o pre;:ito positivo 204.

bre as quais diz a Mekhiltá: " 'Não te recusarás a ajudá-lo;
auxilia-lo-ás' nos en­sina que se viola ambos um preceito positivos " e
um preceito negativo".

Além disso há um preceito separado a esse respeito em Suas pala­vras no
Deuteronômio ``Vendo o jumento de teu irmão'' (Deuteronômio 22:4), sobre
as quais diz o Sifrei: " 'Vendo o jumento de teu irmão' é um preceito
negativo"; e diz mais adiante: " 'Quando vires o asno daquele que te
aborrece' é um preceito positivo".

As normas deste preceito também estão explicadas no segundo ca­pítulo de
Baba Metzia:

\section{Não trapacear na medida e nos pesos}

Por esta proibição somos proibidos de trapacear ao medir a terra ou de
usar medidas e pesos incorretos. Ela está expressa em Suas palavras "Não
fareis iniqüidade no juízo, nem na medida de comprimento, nem no peso, e
nem na medida de capacidades" (Levítico 19:35), que a tradição explica
como significando ``Não fareis iniqüidade ao medir''. E a Sifrá diz, ao
explicar este preceito negativo: ' 'Nào fareis iniqüidade no juízo': se
isto se\textsubscript{ç}refere ao profe­rir um julgamento, isso já foi
dito. Então porque diz aqui 'no juízo'? Para ensinar-nos que aquele que
mede é chamado de juiz' "

Também está dito ali: " 'Na medida de comprimento' se refere à me­dição
de terras", ou seja, a medição e o cálculo devem ser feitos de acordo
com as rigorosas leis da matemática, com precisão e com conhecimento dos
méto­dos corretos;.não devemos utilizar suposições sem base, com faz a
maioria dos funcionários.

``Peso'' inclui ambos pesos e balanças.

\section{Não manter pesos e medidas incorretos}

Por esta proibição somos proibidos de manter pesos è medidas in­corretos
em nossas casas, ainda que não os utilizemos para fins comerciais, Ela
está expressa em Suas palavras, enaltecido seja Ele, "Não terás no teu
bolso pe­sos diversos um grande e um pequeno" (Deuteronômio 25:13) e
isso se aplica também a diversas medidas. Como diz a Guemará de Baba
Batra: "Uma pessoa está proibida de manter em sua casa uma medida muito
pequena ou muito gran­de, ainda que seja para a coleta de urina".

Vocês não devem concluir, por Suas palavras "Não terás diversas
me­didas" (Ibid. 25:14) e ``Não terás diversos pesos'', que estes são dois
preceitos separados. O objetivo das duas proibições é completar as
normas do preceito para que elas cubram os dois tipos de medidas, a de
peso e a de tamanho. Por­tanto, é como se Ele tivesse dito: "Não terás
dois padrões de peso nem de me­dida", como explicamos com relação ao
preceito positivo\textsuperscript{413} Seu preceito "Não terás pesos
diversos ... Não terás diversas medidas" é uma simples proibi­ção que
inclui vários casos, todos regidos pela mesma lei, tal como acontece no


\begin{enumerate}
\def\labelenumi{\arabic{enumi}.}
\setcounter{enumi}{411}
\item
 
 Ver o preceito positivo 202.
 
\item
 
 Ver o preceito positivo 208.
 
\end{enumerate}



preceito "Não pagarás a teu irmão juro de dinheiro, nem'juro de comida,
nem juro de coisa alguma que se dá como juro (Deuteronômio 23:20).
Proibições repetidas da mesma coisa não devem ser contadas como
preceitos separados, como explicamos na Introdução, no Nono Fundamento.
Nós já demos um exem­plo disto no preceito negativo 200, que está
expresso em Suas palavras "Pãés ázimos serão comidos sete dias e não
será vista por ti coisa levedada" (Êxodo 13:7).

\section{Um juiz não pode cometer injustiças}

Por esta proibição um juiz fica proibido de cometer injustiças num
julgamento. Ela está expressa em Suas palavras "Não fareis injustiça no
juízo" (Levítico 19:15). O significado deste preceito é que não se deve
afastar-se dos princípios que a Torah estabeleceu com uma condenação ou
uma absolvição.

\section{Um juiz não pode aceitar presentes de uma das partes}

Por esta proibição um juiz fica proibido de aceitar um presente das
partes, mesmo que seja para que ele proceda a um julgamento justo. Ela
está expressa em Suas palavras ``E suborno não tomes'' (Êxodo 23:8). A
proibição relativa a este assunto está repetida em outro lugar. O Sifrei
diz: " 'Não tomarás suborno' (Deuteronômio 16:19) --- nem mesmo para
absolver o inocente e con­denar o culpado".


As normas deste preceito estão explicadas em vários trechos de


Sanhedrin.

\section{Um juiz não pode proteger uma das partes}

Por esta proibição um juiz fica proibido de proteger um dos litigan­tes
num julgamento. Mesmo que ele seja um homem de alta posição e de gran­de
distinção, o juiz não deve reverenciá-lo se ele aparecer diante dele
junta­mente com a outra parte, nem tratá-lo com deferência e respeito.
Esta proibi­ção está expressa em Suas palavras, enaltecido seja Ele,
``Nem honrarás as faces do poderoso'' (Levítico 19:15), a respeito das
quais diz a Sifrá: "Não dirás 'Este é um homem rico, de uma família
ilustre; como posso envergonhá-lo e teste­munhar seu embaraço?'
Certamente não o envergonhará\textsuperscript{414}; e é por essa razão
que as Escrituras dizem: 'Nem honrarás as faces do poderoso' ".

As normas deste preceito estão explicadas em vários trechos de
Sa­nhedrin e de Shebuot.

\section{Um juiz não pode acovardar-se com medo de pronunciar um julgamento justo}

Por esta proibição um juiz fica proibido de acovardar-se com medo de
pronunciar um julgamento justo contra um malfeitor inclemente e
perver­so. E seu dever pronunciar o julgamento sem pensar nos danos que
o malfeitor pode causar-lhe. Suas palavras, enaltecido seja Ele, são:
``Não temereis á homem algum'' (Deuteronômio 1:17), sobre as quais o
Sifrei diz: " 'Não temereis a ho­mem algum'; para que você não diga
'Tenho medo deste homem porque ele pode matar meu filho, ou queimar meu
trigo, ou destruir minha plantação'. as Escrituras dizem 'Não temereis a
homem algum'

\section{Um juiz não pode decidir em favor de um homem pobre por piedade}

Por esta proibição um juiz fica proibido de ter piedade de um ho­mem
pobre e distorcer um julgamento em seu favor por piedade. Ele deve
tra­tar os ricos e os pobres da mesma forma, e fazer com que se cumpra a
pena imposta. Esta proibição está expressa em Suas palavras "Ao pobre
não favore­cerás em sua briga" (Êxodo 23:3).

O preceito negativo relativo a este assunto se encontra novamente em
Suas palavras ``Não favorecerás as faces do mendigo'' (Levítico 19:15), a
res­peito das quais diz a Sifrá: "Para que você não diga: 'Este é um
homem pobre é como eu e este homem rico somos obrigados a sustentá-lo,
vou sentenciar em seu favor e assim permitir-lhe viver sem perder o
respeito por si mesmo', as Escrituras dizem: Ao pobre não favorecerás em
sua briga' ".

\section{Um juiz não pode distorcer um julgamento contra uma pessoa de má reputação}

Por esta proibição um juiz fica proibido de distorcer um julgamento em
detrimento de uma das partes que ele saiba ser um pecador perverso. O
Enaltecido nos proíbe de punir tal homem distorcendo seu julgamento
através de Suas palavras, enaltecido seja Ele, "Não perverterás o
julgamento de teu in­digente em sua causa" (Êxodo 23:6). A esse respeito
a Mekhiltá diz: "Para que você não diga, num caso entre um homem mau e
um homem honesto, 'Como este homem é mau vou distorcer o julgamento
contra ele' as Escrituras dizem: 'Não perverterás o julgamento de teu
\emph{indigente} em sua causa' --- significando 'indigente' no que se
refere a boas ações", isto é, mesmo que ele seja pobre de boas ações
você não deve distorcer o julgamento contra ele.

\section{Um juiz não pode ter piedade de alguém que matou um homem}

Por esta proibição um juiz fica proibido de apiedar-se de alguém que
matou um homem ou lhe causou a perda de um membro, ao estabelecer a
pe­na. Ele não pode dizer: "Este é um pobre homem que cortou a mão do
outro ou o cegou de um olho sem querer" e assim ter compaixão dele e ser
indulgen­te ao avaliar a importância dos prejuízos. Esta proibição está
expressa em Suas palavras "Alma por alma, olho por olho" (Deuteronômio
19:21). Este preceito negativo aparece novamente em Suas palavras "Não o
olharás com piedade e vingarás o sangue inocente de Israel" (Ibid.,13).

\section{Um juiz não pode distorcer a justiça por prosélitos ou órfaos}

Por esta proibição um juiz fica proibido de distorcer a justiça devi­do
a prosélitos ou órfãos. Ela está expressa em Suas palavras "Não
perverterás o juízo do imigrante e do órfão" (Deuteronômio 24:17).

Já lhes foi explicado que aquele que distorcer a justiça por causa de um
israelita transgride o preceito negativo expresso em Suas palavras "Não
fa­reis injustiça no juízo" (Levítico 19:15); mas aquele que distorcer a
justiça por causa de um prosélito transgride dois preceitos negativos,
como diz o Sifrei: " 'Não perverterás o juízo do imigrante' nos ensina
que aquele que distorcer a justiça devido a um prosélito violará dois
preceitos negativos". E se o proséli­to for órfão, ele transgredirá
três.

\section{Um juiz não pode ouvir uma das partes na ausência da outra}

Por esta proibição um juiz fica proibido de ouvir os argumentos de uma
das partes quando a outra não estiver presente. Ela está expressa em
Suas palavras ``Não dês ouvido à maledicência'' (Êxodo 23:1). Já que na
maioria dos casos o que um dos litigantes alega enquanto o outro não
está presente é incor­reto, o juiz fica proibido de ouví-lo para que ele
não tenha uma visão inexata e falsa do caso. A Mekhiltá diz: " 'Não dês
ouvido à maledicência' proíbe um juiz de ouvir uma das partes até que a
outra também esteja presente e proíbe uma parte de apresentar seu caso
ao juiz até que a outra também esteja presen­te". E para proibir tal
conduta que Ele diz: "Da palavra falsa afasta-te" (Ibid.,7), como está
explicado no quarto capítulo de Shebuot.

De acordo com o Talmud, este preceito negativo também proíbe ca­luniar,
ou ouvir uma calúnia, ou prestar um falso testemunho, como está
expli­cado. em Macot.

\section{Um tribunal não pode condenar por maioria de um num caso capital}

Por esta proibição um tribunal fica proibido de condenar por maio­ria de
um. O significado disto é que se houver uma divisão na opinião dos
juízes, ficando alguns a favor da pena de morte e outros não, e se
houver maioria de apenas um em favor da condenação, não se permite que
se mate o pecador, pois o Eterno proibiu o Tribunal de fazê-lo a menos
que a maioria favorável à conde­nação seja de dois. Esta proibição está
expressa em Suas palavras "Não concor­ras com a maioria para que alguém
seja condenado" (Êxodo 23:2); quer dizer, ao sentenciar a pena de morte
você não deve fazê-lo por causa de uma maioria casual. Este é o
significado da expressão restritiva "para que alguém seja conde­nado".
Como a Mekhiltá diz, "Se onze forem a favor da absolvição e doze a
fa­vor da condenação, eu poderia pensar que o veredito é de culpa, por
isso as Escrituras dizem: 'Não concorras com a maioria para que alguém
seja condena­do' ". Também está dito ali: "Um veredito de absolvição
precisa da maioria de um, mas um veredito de condenação precisa de uma
maioria de dois".


As normas deste preceito estão explicadas no quarto capítulo de Sa-


nhedrin.

\section{Um juiz não pode confiar na opinião de outro juiz}

Por esta proibição um juiz fica proibido de confiar na opinião de um
outro juiz ao condenar o culpado ou absolver o inocente sem que ele
próprio tenha examinado o assunto baseado em sua própria investigação e
deduções dos princípios da lei. Esta proibição está expressa em Suas
palavras "Não decla­res numa causa de forma a desviar-te (de acordo com
a opinião da maioria)" (Êxodo 23:2), cujo significado é: no caso de uma
polêmica, seu objetivo não deve ser apenas adotar uma opinião, seguir a
opinião da maioria ou dos juízes superiores e abolir seu próprio ponto
de vista sobre o assunto. A Mekhiltá diz: " São declares ...'. Não
digas, quando forem contadós\textsuperscript{415}, 'Basta que eu siga
tal pessoa'; dê a sua própria opinião. Poder-se-ia pensar que a mesma
lei se apli­ca a casos não econômicos, por isso as Escrituras dizem: 'de
maneira a desviar-te, de acordo com a opinião da maioria' ".

Deste preceito também se deduz a proibição que impede um juiz de
argumentar em favor de uma condenação quando ele próprio já tiver se
decla­rado em favor da absolvição, proibição essa expressa em Suas
palavras, enalte­cido seja Ele, "Não declares numa causa de forma a
desviar-te' , isto é, não te desvies para mudar o veredito para a
condenação.

Da mesma forma, um caso capital não pode ser aberto pela conde­nação
porque as Escrituras dizem: "Não declares numa causa de forma a
desviar-te". E esse mesmo texto também é a base das regras de que um
veredito de condenação pode ser revertido, mas não um veredito de
absolvição, e que não devem começar pelo mais
velho\textsuperscript{416}, como foi deixado claro no quarto capítu­lo
de Sanhedrin, onde estão explicadas as normas deste preceito.


\begin{enumerate}
\def\labelenumi{\arabic{enumi}.}
\setcounter{enumi}{414}
\item
 
 Quando os votos forem contados.
 
\item
 
 A leitura do veredito não deve começar pelo juiz mais velho.
 
\end{enumerate}


\section{Não designar um juiz inculto}

Por esta proibição fica proibido ao Grande Tribunal e ao Exilarca
designar, por causa de suas outras qualidades, um juiz que não seja
versado na sabedoria da Torah. Eles ficam proibidos de fazer isso e a o
fazer nomeações para a Torah\textsuperscript{417}, eles devem basear-se
apenas na sabedoria que o homem tem da Torah, em seus conhecimentos
sobre seus preceitos e proibições e em sua conduta absolutamente
irrepreensível. O preceito que proíbe que uma pessoa seja designada em
virtude de outras qualidades está expressa em Suas palavras, enaltecido
seja Ele, ``Não conheçais faces no juízo'' (Deuteronômio 1:17), so­bre as
quais o Sifrei diz: " São conheçais faces no juízo' se refere a alguém
cuja função seja designar juízes". Quer dizer, a proibição é dirigida
apenas àquele que tem o direito de nomear juízes para os israelitas e o
proíbe de fazê-lo basea­do em qualquer uma das razões que mencionamos. O
Sifrei acrescenta: "Você não deve dizer: 'Vou nomear tal pessoa porque
ele é simpático, ou porque é rico, ou porque é meu parente, ou porque
ele me emprestou dinheiro, ou por­que ele conhece muitos idiomas'. O
resultado será que ele absolverá o culpado e condenará o inocente não
porque ele seja mau, mas porque lhe falta a sabe­doria. É por essa razão
que as Escrituras dizem: 'Não conheçais faces no juízo' ".

\section{Não prestar um falso testemunho}

Por esta proibição somos proibidos de prestar falsos testemunhos. Ela
está expressa em Suas palavras "Não darás falso testemunho (ed shaker)
con­tra teu próximo" (Êxodo 20:16), e aparece novamente sob outra forma,
nas pa­lavras ``Não darás falso testemunho contra o teu próximo''
(Deuteronômio 5:17). Contra aquele que transgredir este preceito
negativo as Escrituras decretam que "Fareis a eles como pensavam fazer a
seu irmão" (Ibid., 19:19). E a Mekhiltá diz: " 'Não darás falso
testemunho' é uma advertência contra testemunhas que tencionem causar
danos".

A desobediência a este preceito também acarreta o açoitamento, co­mo foi
deixado claro no início do Tratado Macot, onde estão explicadas as
nor­mas deste preceito.

\section{Um juiz não pode aceitar o testemunho de um homem mau}

Por esta proibição um juiz fica proibido de aceitar o testemunho de um
homem mau e agir levando seu testemunho em consideração. Ela está
expressa em Suas palavras, enaltecido seja Ele, "Não acompanhes o mau
para servir de fal­so testemunho" (Êxodo 23:1), que a Tradição explica
assim: "Não deixes que o mau testemunhe e não deixes que o injusto
testemunhe; dessa forma os injustos e os ladrões ficam excluídos de ser
testemunhas", de acordo com o versículo "Não se levantarão testemunhos
injustos contra alguém (etc)" (Deuteronômio 19:16).


As normas deste preceito estão explicadas no terceiro capítulo de


Sanhedrin.

417. Ao nomear as pessoas que devem apresentar decisões em questões da
Torah.

\section{Um juiz não pode aceitar o testemunho de um parente de uma das partes}

Por esta proibição um juiz fica proibido de admitir o testemunho de
parentes, seja a favor ou contra ele. Ela está expressa em Suas
palavras, enal­tecido seja Ele, "Não se fará morrer os pais pelos
filhos, nem os filhos pelos pais" (Deuteronômio 24:16), cuja explicação
Tradicional se encontra no Sifrei: "Pais não deverão morrer pelo
testemunho de seus filhos nem os filhos pelo testemunho de seus pais".

A mesma lei se aplica a casos envolvendo reivindicações de dinhei­ro,
mas as Escrituras a estabeleceram, de uma forma hiperbólica,
referindo-se a casos capitais, para ensinar-nos que não devemos
reciocinar da seguinte for­ma: "Como esta acusação envolve a pena de
morte, não devemos duvidar da veracidade do testemunho de um parente, e
sim devemos agir de acordo cóm ele, já que o caso envolve a morte de seu
parente, não deixando lugar para sus­peita". E por essa razão que as
Escrituras destacam como exemplo o laço mais forte e mais profundo de
afeição que é o amor de um pai por seu filho e de um filho por seu pai;
os Sábios dizem que estamos proibidos de aceitar até mes­mo o testemunho
de um pai contra seu filho, mesmo que isso o condene à morte, e que isso
é um decreto das Escrituras para o qual não há razão, seja ela qual for.
Vocês devem compreender isso.


As normas deste preceito estão explicadas no terceiro capítulo de


Sanhedrin.

\section{Não condenar baseado no depoimento de uma única testemunha}

Por esta proibição somos proibidos de infligir uma punição ou uma multa
baseados no depoimento de uma única testemunha, mesmo que esta seja
digna de toda a confiança. Ela está expressa em Suas palavras,
enaltecido seja Ele, "Não valerá uma testemunha contra um homem por
qualquer delito ou por qualquer pecado" (Deuteronômio 19:15), a respeito
das quais o Sifrei diz: "Não valerá por qualquer delito, mas valerá por
um juramento".

As normas deste preceito estão explicadas em diversos trechos de
Yebamot, Quetubot, Sota, Guitin e Kidushin, e em vários trechos da Ordem
Nezikin.

\section{Não matar um ser humano}

Por esta proibição somos proibidos de matar\textsubscript{7}nos uns aos
outros. Ela está expressa em Suas palavras ``Não matarás'' (Êxodo 20:13),
e todo aquele que violar este preceito negativo será decapitado. O
Enaltecido diz: "Do meu altar o tirarás, para que morra" (Ibid., 21:14).

As normas deste preceito estão explicadas no nono capítulo de Ma-cot
Sanhedrin, e no segundo capítulo de Marcot.


\section{Não punir com a pena capital baseando-se em provas circunstanciais}

Por esta proibição somos proibidos de executar uma sentença baseando-se
numa forte suspeita, mesmo que ela seja quase conclusiva. Assim, se um
homem perseguir seu inimigo com a intenção de matá-lo e o perseguido se
refugiar numa casa, seguido pelo perseguidor, e se ao entrarmos depois
de­les encontrarmos o homem perseguido dando seu último suspiro e seu
inimi­go, o perseguidor, junto a ele com uma faca na mão, estando ambos
ensanguen­tados, o perseguidor não deve ser condenado à morte pelo
Tribunal, no de­sempenho da justiça já que não há testemunhas para depor
que eles viram o assassinato ser cometido. A Verdadeira Torah proíbe que
se condene um ho­mem à morte através de Suas palavras, enaltecido seja
Ele, "O inocente e o jus­to não mates, pois não justificarei ao mau"
(Êxodo 23:7).

A Mekhiltá diz: "Suponha que eles vejam alguém perseguindo seu
companheiro com a intenção de matá-lo e que eles lhe façam uma
advertência legal dizendo: O homem é um israelita, um filho da Aliança;
se você o matar você será morto; e que depois disso eles o percam de
vista e mais tarde encon­trem o outro dando seu último suspiro, e sangue
pingando da espada na mão do perseguidor, eu poderia pensar que ele
deveria ser declarado culpado. Por isso as Escrituras dizem: 'O inocente
e o justo não mates' ".

Não deixem que isto os confunda \textsubscript{!}nem pensem que a lei é
injusta. Entre os acontecimentos que estão entre os limites da
possibilidade, alguns são muito prováveis e outros altamente
improváveis, e outros ainda estão entre es­ses dois. Os limites da
possibilidade são muito amplos. Se a Torah nos tivesse permitido decidir
casos capitais com base numa forte probabilidade, Sue pode parecer
absolutamente convincente, como no caso do exemplo dado, no caso
seguinte estaríamos decidindo baseados numa probabilidade ligeiramente
me­nor, e assim por diante gradualmente, até que estivéssemos julgando
casos ca­pitais e condenando pessoas à morte baseados em suposições
injustificáveis, de acordo com os caprichos do juiz. Por isso o
Enaltecido fechou essa porta, por assim dizer, ordenando que
nenhum'castigo seja aplicado a menos que ha­ja testemunhas que aleguem
saber de fato o que aconteceu, sem uma dúvida qualquer, e que não haja
outra explicação possível. Se não sentenciarmos, ain­da que com base
numa suspeita muito forte, o pior que pode acontecer é que o pecador
será absolvido; mas se punirmos baseados na força de suspeitas e
suposições pode ser que um dia enviemos à morte uma pessoa inocente. E é
melhor e mais satisfatório absolver mil pessoas culpadas do que enviar à
morte uma única pessoa inocente, uma única vez.

Da mesma forma, se duas testemunhas declararem que um homem cometeu duas
transgressões, cujo castigo por cada uma delas é a morte, e se cada uma
das duas testemunhas o tiver visto cometendo apenas uma das
trans­gressões --- por exemplo, se uma testemunha declarar que o acusado
trabalhou no Shabat, e que ele o avisou para que não o fizesse, e a
outra declarar que o acusado adorou ídolos, e que ele o avisou para que
não o fizesse --- o acusado não deverá ser apedrejado. ``Suponha'', diz a
Mekhiltá, "que uma testemunha declare que uma determinada pessoa adora o
sol, e outra que ela adora a lua;

eu poderia penar que eles devam ser-reunidos\textsuperscript{418}; por
isso as Escrituras dizem: 'O inocente e o justo não mates' ".

\section{Uma testemunha não pode atuar como advogado}

Por esta proibição uma testemunha fica proibida de atuar como ad­vogado
num caso no qual ela preste depoimento. Mesmo que ele seja culto e bem
informado, ele não deve atuar como testemunha, juiz e advogado, e sim
depor com relação ao que ele viu, ficando em silêncio enquanto os juízes
fize­rem uso de seu depoimento como eles julgarem apropriado. A
testemunha está proibida de dizer o que quer que seja em acréscimo ao
seu depoimento. Esta proibição, que se aplica apenas a casos capitais,
está expressa em Suas palavras, enaltecido seja Ele, "Uma testemunha não
deporá contra alguém para que mor­ra" (Números 35:30), e novamente em
Suas palavras "Não será morto por de­poimento de uma testemunha"
(Deuteronômio 17:6). Quer dizer, ele não será morto pela argumentação
das testemunhas.

Na Guemará de Sanhedrin lemos: " 'Uma testemunha não deporá con­tra
alguém' seja por sua absolvição ou pela sua condenação", e o seguinte
moti­vo é dado: ``Ele fica parecendo uma testemunha interessada'';
\textbf{.`..} E apenas em casos
capitais que está proibido advogar em favor da absolvição ou da
condenação.

\section{Não matar um assassino sem julgamento}

Por esta proibição somos proibidos de matar quem cometeu um cri­me, a
quem vimos fazer algo que se pune com a morte, antes que ele seja levado
a julgamento. Ele deve ser levado a julgamento e devem ser apresentadas
provas contra ele ao Tribunal; nós só podemos depor contra ele, e o
Tribunal o senten­ciará por qualquer ofensa que ele tenha cometido. A
proibição está expressa em Suas palavras, enaltecido seja Ele, "Não
morrerá o homicida antes de ser apresen­tado diante da congregação para
o julgamento" (Números 35:12). Sobre isso a Mek­hiltá diz: "Eu poderia
pensar que ele pode matá-lo, se ele tiver cometido assassi­nato ou
fornicação, por isso as Escrituras dizem: 'Antes de ser apresentado
diante da congregação' ". Ainda que aqueles que o viram cometer o
assassinato sejam membros do Grande Tribunal, eles se tornam todos
testemunhas que prestarão depoimento diante de outro Tribunal, e esse
outro Tribunal é quem o condenará à morte. E a Mekhiltá diz: "Suponha
que 'uma congregação' veja um homem co­meter um assassinato; eu poderia
pensar que eles podem matá-lo antes que ele seja condenado corretamente
por um Tribunal. Por isso as Escrituras dizem: 'Não morrerá o homicida
antes de ser apresentado diante da congregação' ".

\section{Não poupar a vida de um perseguidor}

Por esta proibição somos proibidos de poupar a vida de um perse­guidor.
A explicação disto é a seguinte. O preceito precedente, que proíbe a
teste-


\begin{enumerate}
\def\labelenumi{\arabic{enumi}.}
\setcounter{enumi}{417}
\item
 
 Que esses dois testemunhos devam ser reunidos e que o acusado dava ser
 declarado culpado.
 
\item
 
 Como se ele tivesse interesse próprio em sua testemunha. •
 
\end{enumerate}



munha de matar o criminoso antes que ele seja condenado pelo Tribunal,
se apli­ca apenas ao caso de alguém que já tenha executado o ato pelo
qual ficou sujeito à morte; mas enquanto ele ainda estiver tentando
executá-lo, é chamado de "per­seguidor", e temos a obrigação de detê-lo
e impedí-lo de levar a cabo suas más intenções. Se ele se recusar
obstinadamente, devemos lutar com ele; e se for pos­sível detê-lo de
fazer o que ele tenciona privando-o de um de seus membros ---como por
exemplo, cortando sua mão ou seu pé, ou cegando-o --- muito bem; mas se
for impossível detê-lo a não ser que se lhe tire a vida, ele deve ser
morto antes que possa executar a ação. A proibição que nos proíbe de
poupar um per­seguidor e de recusar-nos a matá-lo está expressa em Suas
palavras "Cortar-lhe-ás a mão, o teu olho não terá piedade dela"
(Deuteronômio 25:12)\textsuperscript{420}, sobre as quais o Sifrei diz:
" 'Cortar-lhe-ás a mão' nos ensina que você deve salvá-lo com a mão do
atacante. Como sabemos que, se a mão não salvar a vítima, devemos
salvá-la tirando a vida do atacante? Porque as Escrituras dizem: 'Teu
olho não terá piedade' ". Também está dito ali: " `E lhe pegar pelas
\emph{suas vergonhas'} (Ibid.): assim como as vergonhas estão
especificadas aqui porque elas envolvem risco de vida, é assim acarretam
'Cortar-lhe-ás a mão', da mesma forma todas as vezes que a vida for
posta em perigo este mesmo princípio deve ser aplicado".

O princípio que nós estabelecemos, de que um perseguidor deve ser morto,
não se aplica a todos os que viriam a ser malfeitores e sim apenas
àquele que persegue outra pessoa com a intenção de matá-la, mesmo que
ele seja um menor, ou com a intenção de tomá-la de uma das maneiras
proibidas, o que obviamente inclui uma agressão por parte de um varão. O
Enaltecido diz ``A moça desposada gritou e não houve quem a salvasse''
(Deuteronômio 22:27), de onde se conclui que se houver um salvador, ele
deve salvá-la de qualquer forma possível; e Ele compara o caso de alguém
que persegue uma donzela, ao de quem persegue seu companheiro com
intenção de matá-lo, através de Suas palavras "Porque como no caso do
homem que se levanta contra o seu companheiro, e o mata, assim também é
este caso" (Ibid., 26).


As normas deste preceito estão explicadas no oitavo capítulo de


Sanhedrin.

\section{Não punir uma pessoa por um pecado cometido sob coação}

Por esta proibição somos proibidos de punir uma pessoa por um pe­cado
cometido sob coação, pois ela terá agido sob pressão. Esta proibição
está expressa em Suas palavras, enaltecido seja Ele, "Mas à moça não
farás nada" (Deuteronômio 22:26). Em Sanhedrin lemos: "O Misericordioso
isenta de pu­nição aquele que peca sob coação, pois está dito: 'Mas à
moça não farás nada' ".

\section{Não aceitar um resgate de alguém que tenha 
cometido um assassinato deliberadamente}

Por esta proibição somos proibidos de aceitar resgate de alguém que
tenha cometido um assassinato deliberadamente. Tal pessoa deve ser morta
em

420. Ver o preceito positivo 247.

todos os casos. A proibição está expressa em Suas palavras, enaltecido
seja Ele, "Não aceitarás resgate pela vida do homicida, que é condenado
a morrer" (Nú­meros 35:31).

As normas deste preceito estão explicadas em Macot.

\section{Não aceitar um resgate de alguém que tenha cometido 
um assassinato involuntariamente}

Por esta proibição somos proibidos de aceitar resgate por alguém que
tenha cometido um assassinato involuntariamente de forma a livrá-lo do
exílio\textsuperscript{421}. Ele deve ser banido, em todos os casos. A
proibição está expressa em Suas palavras, enaltecido seja Ele, "E não
aceitarás resgate por aquele que fugiu para a cidade de refúgio"
(Números 35:32).


As normas deste preceito estão explicadas na Guemará de Macot.

\section{Não se descuidar de salvar um israelita em perigo de vida}

Por esta proibição somos proibidos de descuidar-nos de salvar a vi­da de
um israelita que virmos correndo risco de vida ou de destruição e a quem
estiver em nosso poder salvar, como por exemplo se uma pessoa estiver se
afo­gando e se formos bons nadadores e pudermos salvá-la, ou se um pagão
estiver tentando matar alguém, e estivermos em condições de impedir sua
tentativa ou de salvar a pessoa ameaçada. Num caso assim somos proibidos
de manter-nos à parte e recusar-nos de ir em seu socorro por Suas
palavras "Não sejas in­diferente quando está em perigo o teu próximo"
(Levítico 19:16).

Os Sábios dizem que esta proibição abrange também o caso de al­guém que
negar provas, pois ele verá o dinheiro de seu amigo perder-se, estan­do
em posição de restituí-lo se contar a verdade. As Escrituras se referem
nova­mente a este assunto: "Se não o denunciar, levará seu pecado"
(Ibid., 5:1)\textsuperscript{422}.

A Sifd diz: "De que forma sabemos que se você tem conhecimento de alguma
prova favorável a ele você não deve omití-la? Pelo texto: São sejas
indiferente quando está em perigo o teu próximo'. E de que' forma
sabemos que se você vir alguém afogando-se ou sendo atacado por ladrões
ou por um animal selvagem, você tem a obrigação de salvá-lo? Pelo texto:
São sejas indi­ferente quando está em perigo o teu próximo'. E de que
forma sabemos que se alguém persegue seu vizinho com a intenção de
matá-lo você tem a obriga­ção de salvá-lo mesmo que isso custe uma vida?
Pelo texto: 'Não sejas indife­rente quando está em perigo o teu próximo'
".


As normas deste preceito estão explicadas no Tratado Sanhedrin.

\begin{enumerate}
\def\labelenumi{\arabic{enumi}.}
\setcounter{enumi}{420}
\item
 
 Ver o preceito positivo 182.
 
\item
 
 Ver o preceito positivo 178.
 
\end{enumerate}



\section{Não deixar obstáculos em propriedades públicas ou privadas}

Por esta proibição somos proibidos de deixar obstáculos ou empe­cilhos
em propriedades públicas ou privadas para que não causem acidentes
fatais. Ela está expressa em Suas palavras, enaltecido seja Ele, "Para
que não ponhas culpa de sangue em tua casa" (Deuteronómio 22:8). Sobre
isso o Sifrei diz: " 'Farás um parapeito' (Ibid.) e um preceito
positivo\textsuperscript{423}; 'Para que não po­nhas culpa de sangue em
tua casa' é um preceito negativo".

As normas deste preceito estão explicadas no primeiro capítulo de
Shekalim no Talmud de Jerusalém, e em vários trechos em Nezikin.

\section{Não dar um conselho enganoso}

Por esta proibição somos proibidos de dar um conselho enganador.
Portanto, se alguém pedir seu conselho sobre um assunto que ele não
com­preenda muito bem você está proibido de enganá-lo ou
desencaminhá-lo; você deve dar o que você considera ser a orientação
correta. A proibição está ex­pressa em Suas palavras, enaltecido seja
Ele, "Diante do cego não porás trope­ço" (Levítico 19:14), sobre as
quais a. Sifrá diz: "Se alguém é 'cego' de alguma maneira, e lhe pedir
um conselho, não lhe dê um conselho que não lhe seja apropriado".

De acordo com os Sábios, este preceito negativo também se aplica a
ajudar ou levar alguém a cometer uma transgressão, porque fazer isso é
ajudar e incitar a cometer um delito um homem cuja paixão o tenha
privado de sua capacidade de raciocinar e o tenha cegado, ou
apresentar-lhe oportunidades para o pecado. É nesse sentido que os
Sábios dizem, com relação a uma transa­ção envolvendo um empréstimo com
juros, que tanto quem empresta como quem pede emprestado transgridem
``Diante do cego não porás tropeço'', uma vez que um ajuda o outro a
completar a transgressão. Há muitos casos desse tipo, nos quais os
Sábios dizem que se transgride o ``Diante do cego não porás tropeço''.
Contudo, o significado literal do versículo é o que explicamos acima.

\section{Não infligir castigo corporal excessivo}

Por esta proibição um juiz fica proibido de infligir a um malfeitor um
castigo corporal tão severo que lhe cause dano permanente. Isto deve ser
explicado da seguinte forma: a quantidade máxima de açoites que pode ser
im­posta a um homem sujeito ao açoitamento foi fixada pela Tradição em
trinta e nove, mas nenhum homem pode ser submetido a um castigo corporal
até que seja feita uma estimativa do número de açoites que ele pode
suportar, levando-se em consideração sua idade, temperamento e físico.
Se ele puder su­portar o castigo pleno, ele será aplicado; se não, ele
deverá receber tantos

423. Ver o preceito 'positivo 184.

açoites quantos for capaz de suportar, com um mínimo de ires. Isto se
baseia nas palavras do Enaltecido "Com o número de açoites segundo a sua
culpa" (Deuteronômio 25:2). O castigo total é quarenta açoites menos um,
e o precei­to proíbe que se exceda nem que seja por um a quantidade que
o ofensor pode suportar, de acordo com a estimativa do juiz. A proibição
está expressa em Suas palavras "Com o número de açoites segundo a sua
culpa. Quarenta açoites lhe fará dar, não irá além" (Ibid., 2-3).

O Sifrei diz: "Se exceder o limite, ele violará um preceito negativo.
Isso me foi dito apenas com relação aos quarenta. De que forma fico
sabendo que isso se aplica a qualquer quantidade que ele possa suportar,
de acordo com a estimativa do Tribunal? Pelas palavras das Escrituras
'Não irá além, com re­ceio que (pen) suceda que indo além...' "
(Ibid.,3).

Este preceito negativo também proíbe bater num israelita, seja ele quem
for. Se já somos proibidos de bater num pecador, quanto mais numa ou­tra
pessoa! Os Sábios, a paz esteja com eles, também nos proíbem de ameaçar
de bater numa pessoa, mesmo que não o façamos realmente: "Aquele que
sim­plesmente levantar a mão contra seu vizinho com a intenção de bater
nele é chamado de 'malfeitor' (rasha), como está dito: 'E diz ao mau
(la-rasha): Por que feres a teu próximo?' " (Exodo 2:13).

\section{Não bisbilhotar}

Por esta proibição somos proibidos de bisbilhotar. Ela está expressa em
Suas palavras, enaltecido seja Ele. "Não andarás com mexericos (rachil)
en­tre o teu povo" (Levítico 19:16), sobre as quais o Sifrei diz: "Não
se deve ser indulgente com um e severo com outro". Outra interpretação
é: "Não deves ser como um vendedor ambulante, que carrega sua mercadoria
de um lugar pa­ra outro".

Este preceito negativo também proíbe a difamação\textsuperscript{424}.

\section{Não odiar uns aos outros}

Por esta proibição somos proibidos de odiar uns aos outros. Ela está
expressa em Suas palavras ``Não odiarás a teu irmão em teu coração''
(Levítico 19:17), sobre as quais a Sifrá diz: "Eu falo apenas de rancor
no coração\textsuperscript{..}. Con tudo, se alguém revelar seu ódio e
disser à pessoa que odeia que ele é seu inimi­go, ele não violará este
preceito negativo, mas transgredirá "Não te vingarás e nem guardarás
ódio" (Ibid., 18), e também violará o preceito positivo expres­so em
Suas palavras ``Amarás o teu próximo como a ti mesmo''
(Ibid.)\textsuperscript{425}. Mas o ódio no coração é o pecado mais
grave de todos.

\section{Não envergonhar ninguém}

Por esta proibição somos proibidos de envergonhar alguém, o que é
chamado de "embranquecer a face de seu companheiro`. ou seja. envergo-


\begin{enumerate}
\def\labelenumi{\arabic{enumi}.}
\setcounter{enumi}{423}
\item
 
 Ver o preceito positivo 219.
 
\item
 
 Ver o preceito positivo 206.
 
\end{enumerate}




\textgreater5

nhá-lo em público. A proibição relativa a este assunto está ::xpressa cm
Suas palavras "Repreenderás a teu companheiro e nao levarás sobre ti
pecado'• (Le vítico 19:17)\textsuperscript{426}'

A Sifrá diz: "De que forma ficamos sabendo que mesmo que alguém tenha
repreendido um homem quatro ou cinco vezes ele deve continuar a faze-lo?
Pelas palavras das Escrituras `Repreenderás'. Eu poderia pensar que é
assim mesmo quando sua repreensão transforma seu semblante, por isso as
Escritu­ras dizem: 'E não levarás sobre ti pecado' ". O sentido literal
do versículo, con­tudo, é que somos proibidos de guardar qualquer
pensamento sobre seu peca, do ou de recordá-lo.

\section{Não se vingar um do outro}

Por esta proibição somos proibidos de vingar-nos um do outro, ou seja,
se alguém nos tiver feito algum mal, não devemos insistir em perseguí-lo
até que tenhamos retribuído sua maldade ou que o tenhamos ferido como
ele nos feriu. O Eterno proíbe essas coisas com as palavras "Nao te
vingarás" (Le­vítico 19:18).

A Sifrá diz: Até onde vai o poder de vingança? Se A disser a 'B': -
'Empreste-me sua foice' e ele recusar, e se no dia seguinte '13 disser a
'A': `Empreste-me sua machadinha', e ele responder: 'Não a emprestarei a
vocè, da mesma forma que vocé se recusou a emprestar-me sua foice',
contra esse tipo de conduta está dito: 'Não te vingarás' ". Podemos
fazer deduções por analo­gia, a partir deste exemplo, para todos os
outros casos.

\section{Não guardar rancor}

Por esta proibição somos proibidos de guardar rancor --- isto é,
guar­dar na lembrança um mal que alguém nos tenha feito e lembrar disso
contra ele

mesmo que não nos vinguemos. Esta proibição está expressa em Suas
pala­vras, enaltecido seja Ele, ``Não te vingarás e nem guardarás ódio''
(Levítico 19:18).

- A Sifrá diz: "Até onde vai o poder do rancor? Se 'A' disser a 'B':
EmpreSte-me sua fole,: e ele recusar e se no dia seguinte 'B' disser a
'A': 'Em­preste-me sua machadinha", e ele responder: 'Aqui está ela; eu
não sou corno vocé, que não quis emprestar-me sua foice', contra esse
tipo de conduta está di­to: 'Nem guardarás ódio' ".

\section{Não pegar o ninho todo de um pássaro}

Por esta proibição somos proibidos, quando estivermos caçando, de pegar
o ninho todo de um pássaro, com a mãe e Os filhotes. Ela está expressa
em Suas palavras, enaltecido seja Ele, "Não tomarás a mãe estando com Os
fi­lhos" (Deuteronômio 22:6).

Este é um preceito negativo justaposto a um preceito positivo, a sa­ber,
``Deixarás ir livremente a mãe'' (Ibid.,7)\textsuperscript{4 \emph{2-}} Se
não se tobedecer o preceito


\begin{enumerate}
\def\labelenumi{\arabic{enumi}.}
\setcounter{enumi}{425}
\item
 
 Ver o preceito positivo 205.
 
\item
 
 Ver o preceito) positivo 148.
 
\end{enumerate}


positivo relacionado, deixando a mãe sair, e caso a mãe morra antes de
ser li­berta, ele será punido com o açoitamento.

As normas deste preceito estão explicadas no final de Hulin.

\section{Não raspar a tinha}

Por esta proibição somos proibidos de raspar o cabelo em volta da tinha.
Ela está expressa em Suas palavras ``O lugar da tinha não se raspará''
(Le­vítico 13:33).

Nas palavras da Sifrá: "De que forma ficamos sabendo que aquele que
tirar sinais de impureza de sua tinha viola um preceito negativo? Pelas
pala­vras das Escrituras 'O lugar da tinha não se raspará' ".

308 NÃO CORTAR OU • CAUTERIZAR MARCAS DE LEPRA


Por esta proibição somos proibidos de cortar ou de cauterizar mar-


,

cas de lepra de maneira a modificar sua aparência\textsuperscript{428}.
Esta proibição está ex­pressa em Suas palavras "Guarda-te da chaga da
lepra" (Deuteronômio 24:8), sobre as quais diz o Sifrei: " 'Guarda-te da
chaga da lepra' é um preceito negati­vo", e a Mishná diz: "O homem que
tira as marcas da impureza ou cauteriza a lepra transgride um preceito
positivo" e está sujeito ao açoitamento, como explicamos no lugar
apropriado.

309 NÃO LAVRAR UM VALE NO QUAL TENHA SIDO REALIZADO O RITUAL DE "EGLÁ
ARUFÁ"

Por 'esta proibição somos proibidos de lavrar ou cultivar um vale
vir­gem no qual se tenha destroncado o pescoço de uma
vaca\textsuperscript{429}. Ela está expres­sa em Suas palavras "Que não
se lavra nem se semeia" (Deuteronômio 21:4). A contravenção a esta
proibição é punida com o açoitamento.

A Guemará de Macot, ao enumerar as transgressões puníveis com o
açoitamento, diz: "Por que não incluir também aquele que semeia num
'vale virgem', já que a proibição necessária está expressa nas palavras
'Que não se lavra nem se semeia'?" Dessa forma, fica claro que este é
apenas um preceito negativo, e que sua desobediência é punida com o
açoitamento.

As normas deste preceito estão explicadas no final de Sotá.

\section{Não deixar viver um feiticeiro}

Por esta proibição somos proibidos de permitir que um feiticeiro vi­va.
Ela está expressa em Suas palavras ``Feiticeira não deixarás viver''
(Êxodo


\begin{enumerate}
\def\labelenumi{\arabic{enumi}.}
\setcounter{enumi}{427}
\item
 
 Ver os preceitos positivos 101 a 103.
 
\item
 
 Ver «preceito positivo 181. •
 
\end{enumerate}



22:17). Permití-lo é quebrar um preceito negativo, e não somente um
preceito positivo\textsuperscript{430}, como no caso de perdoar um
malfeitor que esteja sujeito à morte por sentença
judicia1\textsuperscript{431}.

\section{Não levar um recém-casado para longe de sua casa}

Por esta proibição somos proibidos de levar um recém-casado para longe
de sua casa durante um ano para fazer qualquer tipo de serviço, seja
mili­tar ou civil. Ao contrário, devemos liberá-lo, durante um ano
inteiro, de todos os deveres que possam afastá-lo de
casa\textsuperscript{432}. A proibição está expressa em Suas palavras,
enaltecido seja Ele, "Nem lhe será imposto carga alguma; livre estará
para cuidar de sua casa" (Deuteronômio 24:5).

Na Guemará de Sotá lemos: " 'Não servirá o exército' (Ibid.): eu
po­deria pensar que ele não sairá com o exército, mas que preparará
armas e for­necerá água e comida. Por isso as Escrituras dizem: 'Nem lhe
será imposto car­ga alguma'. A \emph{ele} não será imposta carga alguma,
mas você pode impô-la a ou­tros. Contudo, se está escrito 'Nem lhe será
imposto carga alguma', qual é a finalidade de 'Não servirá o exército'?
Para que a transgressão da lei envolva duas proibições".


Nós já explicamos no Nono Fundamento que nem toda transgressão


que nos torna culpados por duas proibições envolve dois preceitos.


Vocês devem saber que o próprio esposo está proibido de deixar


sua casa, ou seja, de sair numa viagem, durante um ano inteiro.


As normas deste preceito estão explicadas no oitavo capítulo de Sotá.


312 NÃO DISCORDAR DAS

AUTORIDADES TRADICIONAIS

Por esta proibição somos proibidos de discordar dos guardiães
au­torizados da Tradição, a paz esteja com eles, ou de afastar-nos do
que quer que eles ordenem em assuntos da Torah. Ela está expressa em
Suas palavras ``Não te desviarás da sentença que te anunciarem''
(Deuteronômio 17:11), sobre as quais o Sifrei diz: " 'Não te
desviarás...' é um preceito negativo..

Aquele que infringir este preceito negativo é chamado de "o velho
rebelde" e está sujeito à morte por estrangulamento, pelas condições
estabele­cidas pela Tradição, que estão expostas no final de Sanhedrin,
onde as normas deste preceito estão explicadas. •

313 NÃO FAZER ACRÉSCIMOS À LEI ESCRITA OU ORAL

Por esta proibição somos proibidos de fazer acréscimos à lei escrita ou
oral. Ela está expressa em Suas palavras, enaltecido seja Ele, "Não
acrescen­tareis... a isso nada" (Deuteronômio 13:1). Os Sábios dizem
freqüentemente:

430 Ver o preceito positivo 229.


\begin{enumerate}
\def\labelenumi{\arabic{enumi}.}
\setcounter{enumi}{430}
\item
 
 Nesse caso transgride se também um preceito positivo.
 
\item
 
 Ver o preceito positivo 214.
 
\end{enumerate}


.Ele transgride a lei 'Nao acrescentareis a isso nada'
", ou "Você transgrediu a lei 'Não acrescentareis a isso nada' ".

\section{Não fazer diminuições na lei escrita ou oral}

Por esta proibição somos proibidos de fazer diminuições na lei es­crita
ou oral. Ela está expressa em Suas palavras "Nem diminuireis a isso
nada" (Deuteronômio 13:1). Os Sábios dizem freqüentemente: "Ele
transgride a lei . 'Nem diminuireis a isso nada' ", ou "Você transgrediu
a lei 'Não diminuireis a isso nada' ".

\section{Não maldizer um juiz}

Por esta proibição somos proibidos de maldizer um juiz. Ela está ex
pressa em Suas palavras "Aos juízes (Elohim) não maldigas" (Exodo 22:27)
A contravenção a esta proibição será punida com o açoitamento.

316 NÃO MALDIZER UM CHEFE

Por esta proibição somos proibidos de maldizer um chefe. Ela está
expressa em Suas palavras "Ao chefe (Nassi) de teu povo não maldigas"
(Èxodo

Pela palavra --- Nassi" as Escrituras querem dizer rui
que governa. como em Suas palavras, enaltecido seja Ele, 'Quando o
\emph{príncipe da nação} pe car" (Levítico 4:22). Contudo, os Sábios
usam as palavras apenas com relação ao Chefe da Academia dos Setenta
Anciões (isto é, do Grande Sanhedrin). As­sim, em todo o Talmud e na
Mishná eles falam de: "Os Nessiim \emph{e} os Juízes Prin­cipais", "O
Nassi e o Juiz Principal". Eles também dizem: 'Se um Nassi per­doar sua
honra. seu perdão será aceito, mas se um rei perdoar sua honra, ele não
será aceito".

Vocês devem saber-que esue preceito negativo se refere ao Nassi as­sim
como ao rei, pois seu objetivo é advertir-nos contra maldizer qualquer
um que tenha uma posição de autoridade suprema, seja na esfera da
autoridade go­vernamental ou na da Torah, como Cabeça da Academia. É
isso o que se de­preende das normas deste preceito.

A contravenção a esta proibição é punida com o açoitamento.

\section{Não maldizer um israelita}

Por esta proibição somos proibidos de maldizer qualquer israelita. Ela
está expressa em Suas palavras ``Não amaldiçoarás ao surdo'' (Levítico
19:14).

Explicarei agora o significado do. termo ``heresh'' (surdo).

Quando uma pessoa é movida pelo desejo de vingar-se de alguém que o
enganou, causando-lhe um dano do tipo que ele acreditít ter sofrido, ele
não ficará satisfeito enquanto não tiver devolvido o mal des;;a forma, e
seus sentimentos só serão mitigados e sua mente só abandonará essa idéia
quando ele tiver se vingado. Algumas vezes o desejo de vingança de um
homem se sa­tisfaz meramente maldizendo e insultando porque ele sabe
quanto dano e ver-


gonha isso causará ao seu inimigo Mas -algumas vezes O assunto será mais
sério e ele não se contentará enquanto não tiver arruinado completamente
o outro, e assim se satisfará ao pensar na dor que a perda de seus bens
causou a seu inimigo. Em outros casos o assunto será ainda mais sério, e
ele não se satisfará até que tenha surrado seu inimigo ou o tenha
ferido. Ou pode ser. mais sério ainda, e seu desejo de
vingança não ficará satisfeito enquanto ele não chegar ao extremo de
tirar a vida de seu inimigo e destruir toda sua existência. Por outro
lado, algumas vezes, devido à leveza da ofensa, o desejo de vingança não
será grande, de forma que ele se sentirá aliviado praguejando e
maldizendo enraivecido, mesmo se o outro não o ouvisse, se estivesse
pre­sente. É sabido que pessoas geniosas e coléricas encontram alívio
dessa forma com relação a ofensas triviais, ainda que o ofensor
desconheça sua raiva e não ouça suas explosões.

No entanto poderíamos supor que a Torah nos proíbe de maldizer um
israelita pela vergonha e a dor que a imprecação lhe causaria ao
ouví-la, mas que não há pecado em maldizer um surdo, pois como ele não
pode ouvir, não há de se sentir ofendido. Por essa razão Ele nos diz que
não se deve maldi­zer. proibindo de faze lo no caso do surdo, já que a
Torah se refere não apenas ao ofendido, mas também ao ofensor, a quem se
diz que não deve ser vingati­vo nem genioso. É dessa maneira que ficamos
sabendo que os guardiães da Tra­dição deduzem a proibição de maldizer um
israelita das palavras das Escrituras ``Não amaldiçoarás ao surdo''.

A Sifrá diz: "Está dito que não devo amaldiçoar o surdo; de que for­ma
fico sabendo que não se deve amaldiçoar ninguém? Pelas palavras das
Es­crituras 'De teu povo não maldigas' (Êxodo \emph{22:27).} Isto é para
excluir os mor­tos, os quais, embora sendo como os
surdos\textsuperscript{433}, se diferenciam deles por não estarem mais
vivos' .

A Mekhiltá diz: " 'Não amaldiçoarás ao surdo': as Escrituras mencio­nam
o mais infeliz dos seres humanos".

Ao dizer que se pune com o açoitamento, queremos dizer apenas
se\textsuperscript{43-}' com o Nome Divino. Aquele que amaldiçoar a si
mesmo também será pu­nido com o açoitamento.

Ficou, dessa forma, claro para vocês que aquele que amaldiçoar seu
companheiro com o Nome Divino viola o preceito .Não
amaldiçoarás ao surdo"; aquele que amaldiçoar um juiz é culpado duas
vezes; e àquele que amaldiçoar um chefe é culpado três vezes. A Mekhiltá
diz: .Quando as Escri­. turas dizem: 'Ao chefe de teu
povo' (Ibid.), eu interpreto isso como significan­do ambos um chefe e um
juiz. Então por que elas dizem 'Aos juízes não maldi­gas'? Para deixar
claro que se é culpado ao maldizer qualquer um dos dois". Dessa forma, a
Mekhiltá diz: "É possível, através de um simples pronuncia­mento,
tornar-se culpado quatro vezes. O filho de um chefe que amaldiçoa seu
pai é culpado quatro vezes: por amaldiçoar seu pai, por amaldiçoar um
juiz, por amaldiçoar um chefe, e por amaldiçoar um israelita (incluído
em 'teu povo')''.

E assim demos a explicação prometida acima.


As normas deste preceito estão explicadas np quarto capítulo de


Shebuoth.

\section{Não amaldiçoar os pais}


Por esta proibição cada um de nós está proibido de amaldiçoar seus pais.


A Torah enuncia claramente a punição nas palavras "Aquele que mal­disser
a seu pai ou a sua mãe, será certamente morto" (Êxodo 21:17), e esta é
uma das ofensas que se pune com o apedrejamento. Mesmo aquele que
deli­beradamente amaldiçoar com o Nome Divino o pai ou a mãe que não
estive­rem mais vivos será apedrejado. Esta proibição, contudo, não está
claramente expressa nas Escrituras, pois Ele não diz: "Não amaldiçoarás
teu pai". Mas, co­mo foi explicado antes, há uma proibição contra
amaldiçoar um israelita, e ela inclui nosso pai entre os outros.

A Mekhiltá diz: " 'Aquele que maldisser a seu pai ou a sua mãe, será
certamente morto': ouvimos o castigo por fazê-lo, mas onde encontramos
essa proibição? Nas palavras das Escrituras 'Aos juízes (Elohim) não
maldigas'. (Êxo­do 22:27). Se seu pai for um juiz, ele está incluído em
`Elohim.; se ele for um chefe, ele está incluído em
`Nassi' (Ibid.); e se ele for uma pessoa comum, ele está incluído em
``Não amaldiçoarás ao surdo'' (Levítico 19:14). E você pode estabelecer
uma regra geral com base no que há de comum aos três, a saber, que eles
são 'de teu povo' (Ibid.) e portanto você está proibido de
amaldiçoá-los. Assim também com relação a seu pai. Ele pertence a 'teu
povo' e portanto você está proibido de amaldiçoá-lo".

A Sifrá diz: " 'O homem que amaldiçoar a seu pai, e a sua mãe'
(Leví­tico 20:9): ouvimos o castigo por isso, mas não ouvimos a
proibição, por isso as Escrituras dizem: 'Aos juízes não maldigas' "
sendo este texto igual ao da Mekhiltá, citado acima.


As normas deste preceito estão explicadas no sétimo capítulo de


Sanhedrin.

\section{Não ferir seus pais}


Por esta proibição somos proibidos de ferir nossos pais.


Também neste caso não há uma proibição expressa nas Escrituras, mas o
castigo está mencionado em Suas palavras "Aquele que ferir a seu pai ou
a sua mãe, será certamente morto" (Êxodo 21:15), e deduzimos a proibição
contra ferir nosso pai através do método usado com relação a amaldiçoar
nos­so pai, o qual explicamos no preceito negativo 300, ou seja, que
nosso pai está incluído no preceito e nos proíbe de ferir qualquer
israelita.

A Mekhiltá diz: " 'Aquele que ferir a seu pai ou a sua mãe': ouvimos o
castigo, mas não ouvimos a proibição, por isso as Escrituras dizem:
'Quaren­ta açoites lhe fará dar, não irá além' (Deuteronômio 25:3), e
pelo método de kal vahomer'\textsuperscript{435} raciocinamos da
seguinte forma: se no caso de alguém em quem temos a obrigação de bater
somos proibidos de fazê-lo\textsuperscript{436}, conclui-se que no caso
dos pais, em quem temos o dever de não bater, somos totalmente
proi­bidos de fazê-lo".


Aquele que transgredir este preceito negativo --- ou seja, que delibe-

\begin{enumerate}
\def\labelenumi{\arabic{enumi}.}
\setcounter{enumi}{434}
\item
 
 Literalmente, ``com mais razão'': argumento do menor ao maior.
 
\item
 
 Somos proibidos de dar Mais do que um determinado número de açoites.
 
\end{enumerate}


PRECEITOS NEGATIVOS 331

radamente ferir seu pai ou sua mãe, fazendo com que eles sangrem ---
está su­jeito à morte por estrangulamento.


As normas deste preceito estão explicadas no final de Sanhedrin.


\section{Não trabalhar no shabat}

Por esta proibição somos proibidos de fazer qualquer
trabalho\textsuperscript{437} no Shabat. Ela está expressa em Suas
palavras ``Não farás nenhuma obra'' (Êxodo 20:10). As Escrituras
prescrevem expressamente a pena de extinção pela deso­bediência deste
preceito negativo, se ela não chegar ao conhecimento do Tri­bunal; mas
se houver o depoimento de testemunhas, o castigo será a morte por
apedrejamento. Isto se aplica à transgressão voluntária; aquele que
pecar invo­luntariamente deve oferecer um Sacrifício Determinado de
Pecado\textsuperscript{438}.

As normas deste preceito estão explicadas no Tratado Shabat.

\section{Não viajar no shabat}

Por esta proibição somos proibidos de viajar no Shabat. Ela está
ex­pressa em Suas palavras ``Não saia ninguém de seu lugar no sétimo dia''
(Êxodo 16:29). A Tradição fixa o limite além do qual é proibido ir em
dois mil cúbitos além dos limites da cidade; nem um cúbito a mais. É
permitido ir até dois mil cúbitos em qualquer direção. A Mekhiltá diz: "
'Não saia ninguém de seu lu­gar': isto é, além de dois mil cúbitos".

A Guemará de Erubin diz: "A pena de açoitamento por transgressão da lei
de `erub' de limites está prescrita por lei das Escrituras". As normas
deste preceito estão explicadas nesse Tratado.


\section{Não castigar durante o shabat}


Por esta proibição somos proibidos de aplicar o castigo de um mal­feitor
no Shabat. Ela está expressa em Suas palavras, enaltecido seja Ele, "Não
acendereis fogo em todas as vossas habitações, no dia de sábado" (Exodo
35:3), cujo significado é: ``Não queime o réu que deve ser queimado''; e a
mesma lei se aplica a todas as outras formas de execução. A Mekhiltá
diz: " São acende­reis fogo'. Acender fogo, que está incluído entre os
tipos de trabalho proibidos no Shabat, foi especialmente destacado para
ensinar-nos que assim como as leis do Shabat não podem ser transgredidas
nem mesmo no caso especificamente mencionado de execução pelo fogo, elas
também não podem ser transgredidas no caso de qualquer uma das formas de
execução judicial".

No Talmud lemos que acender o fogo está destacado porque é um preceito
negativo\textsuperscript{439}; mas o parecer aceito é que isso está
destacado porque a execução de cada tipo diferente de trabalho acarreta
uma penalidade separada, como está explicado ali.


\begin{enumerate}
\def\labelenumi{\arabic{enumi}.}
\setcounter{enumi}{436}
\item
 
 Inclui 39 tipos de trabalho proibidos no Shabat. Ver Mishné Torá
 Milchot Shabat, 7? Capítu­lo, 1? Lei.
 
\item
 
 Ver o preceito positivo 69.
 
\item
 
 E a punição por sua transgressão é apenas o açoitamento e não a
 extinção e a pena capital, como no caso dos outros trabalhos.
 
\end{enumerate}


Na Guemará de Jerusalém lemos: " 'Em todas as vossas habitações': Rabi
Ilai, em nome de Rabi lanai, comenta:. 'Em todas as vossas habitações':
desta forma ficamos sabendo que os Tribunais não devem julgar no
Shabat".

\section{Não trabalhar no primeiro dia de ``pessah''}

Por esta proibição somos proibidos de trabalhar no primeiro dia de
``Pessah''. Ela está expressa em Suas palavras "Nenhum trabalho será feito
ne­les.' (Êxodo 12:16)\textsuperscript{44}°.

\section{Não trabalhar no sétimo dia de ``pessah''}

Por esta proibição somos proibidos de trabalhar no sétimo dia de
``Pessah''. Ela está expressa em Suas palavras "Nenhum trabalho será feito
ne­les" (Êxodo 12:16)\textsuperscript{441}, ou seja, no primeiro e no
sétimo dia.

\section{Não trabalhar em ``atzeret''}

Por esta proibição somos proibidos de trabalhar em ``Atzeret'', isto é,
``Shabuot''. Ela está expressa em Suas palavras "Nenhum trabalho servil
fa­reis" (Levítico 23:21).

\section{Não trabalhar em ``rosh hashaná''}

Por esta proibição somos proibidos de trabalhar no dia de "Rosh
Has­haná". Ela está expressa em Suas palavras, relativas a esse dia,
"Nenhum traba­lho servil fareis" (Levítico 23:25).

\section{Não trabalhar no primeiro dia de ``sucot''}

Por esta proibição somos proibidos de trabalhar no primeiro dia de
``Sucot''. Ela está expressa em Suas palavras, relativas a esse dia,
"Nenhum tra­balho servil fareis" (Levítico 23:35).

\section{Não trabalhar em ``shemini atzeret''}

Por esta proibição somos proibidos de trabalhar em "Shemini Atze­ret".
Ela está expressa em Suas palavras, relativas a esse dia, "Nenhum
trabalho servil fareis" (Levítico 23:36).


\begin{enumerate}
\def\labelenumi{\arabic{enumi}.}
\setcounter{enumi}{439}
\item
 
 Ver também Levítico 23:7.
 
\item
 
 Ver também Levítico 23:8.
 
\end{enumerate}


PRECEITOS NEGATIVOS

Vocês devem saber que todo aquele que fizer qualquer tipo de tra balho
em qualquer desses dias estará sujeito ao açoitamento, a menos que o
trabalho seja relativo ao preparo do alimento necessário, como dizem as
Escri­.turas, referindo-se a um deles, "Salvo o que é
para comer para toda alma, isto só será feito para vós" (Êxodo 12:16). O
mesmo se aplica em relação ao resto dos festivais.

As normas deste preceito estão explicadas no Tratado Betzá.


\section{Não trabalhar em ``yom quipur''}


..•Por esta proibição somos proibidos de trabalhar em
``Yom Quipur''. Ela está expressa em Suas palavras, relativas a esse dia,
"Nenhum trabalho fa­reis" (Levítico 23:31).

Aquele que transgredir voluntariamente este preceito negativo esta­rá
sujeito à extinção, como determinam as Escrituras; aquele que pecar
invo­luntariamente deverá oferecer um Sacrifício Determinado de Pecado.

As normas deste preceito estão explicadas nos Tratados Betzá, Me-guita,
e outros.

\section{Não cometer incesto com sua mãe}

Por esta proibição um homem fica proibido de cometer incesto com sua
mãe. Ela está expressa em Suas palavras "Tua mãe é ela, não descobrirás
sua nudez" (Levítico 18:7).

A contravenção a esta proibição será punida com a extinção. Se
tes­temunhas depuserem contra o transgressor ele será apedrejado, caso
tenha pe­cado deliberadamente; se o tiver feito involuntariamente, ele
deverá oferecer um Sacrifício Determinado de Pecado.

331 NÃO COMETER INCESTO COM A ESPOSA DE SEU PAI

Por esta proibição um homem fica proibido de cometer incesto com a
esposa de seu pai. Ela está expressa em Suas palavras "Nudez da mulher
de teu pai não descobrirás" (Levítico 18:8).

A contravenção a esta proibição será punida com a extinção. Se
tes­temunhas depuserem contra o transgressor, ele será morto por
apedrejamento se tiver pecado deliberadamente, mas se o tiver feito
involuntariamente, ele de­verá oferecer um Sacrifício Determinado de
Pecado.

Ficou dessa forma claro para vocês que o homem que cometer in­cesto com
sua mãe será culpado uma vez por ser ela sua ``mãe'' e outra vez por ser
ela a ``esposa de seu pai'', quer seja durante a vida de seu pai ou depois
de sua morte, como está explicado em Sanhedrin.

\section{Não cometer incesto com sua irmã}

Por esta proibição um homem fica proibido de cometer incesto com sua
irmã. Ela está expressa em Suas palavras "Nudez de tua irmã, filha de
teu pai ou filha de tua mãe... não descobrirás sua nudez" (Levítico
18:9).

Aquele que deliberadamente transgredir este preceito está sujeito à
extinção; se o violar involuntariamente, ele deverá oferecer um
Sacrifício De­terminado de Pecado.

\section{Não cometer incesto com a filha da esposa de seu pai se ela for sua irmã}

Por esta proibição um homem fica proibido de cometer incesto com a filha
da esposa de seu pai, se ela for sua irmã. Ela está expressa em Suas
pala­vras, enaltecido seja Ele, "Nudez da filha da mulher de teu pai,
gerada de teu pai, tua irmã ela é, não descobrirás sua nudez" (Levítico
18:11).

O objetivo desta proibição é fazer da filha da esposa do pai uma
re­lação proibida separada, para que aquele que cometer incesto com sua
irmã pôr parte de pai, se a mãe dela for casada com o pai dele, seja
culpado duas vezes: porque ela é sua irmã e porque ela é filha da esposa
de seu pai, assim como um homem que comete incesto com sua mãe é culpado
duas vezes: porque ela é sua mãe e porque ela é a esposa de seu pai,
como explicamos. Isso aparece no seguinte trecho do segundo capítulo de
Yebamot: "Nossos Sábios nos ensi­naram: Um homem que comete incesto com
sua irmã, que também é a filha da esposa de seu pai, é culpado ambos por
que ela é sua irmã e é a filha da esposa de seu pai. Rabi Yossi ben
Yehudá disse que ele é culpado apenas por ser ela sua irmã mas não por
ser a filha da mulher de seu pai. Por que motivo ele disse isso?
Observem, diriam eles, que está escrito 'Nudez de tua irmã, filha de teu
pai etc' (Ibid., 9); qual é a finalidade das palavras 'Nudez da filha da
mu­lher de teu pai, gerada de teu pai etc'? É para declarar que ele é
culpado por ser ela ambas sua irmã e filha da mulher de seu pai".

Aquele que violar esta proibição cometendo incesto com uma irmã cuja mãe
esteja casada com seu pai está sujeito à extinção, mas apenas se seu
pecado tiver sido deliberado. Se ele tiver pecado involuntariamente, ele
deve­rá oferecer um Sacrifício Determinado de Pecado.

\section{Não cometer incesto com a filha de seu filho}

Por esta proibição um homem fica proibido de cometer incesto com a filha
de seu filho. Ela está expressa em Suas palavras "Nudez da filha de teu
filho... não descobrirás" (Levítico 18:10).


\section{Não cometer incesto com a filha de sua filha}

Por esta proibição um homem fica proibido de cometer incesto com a filha
de sua filha. Ela. está expressa em Suas palavras "Ou filha de tua
filha, não descobrirás sua nudez, porque tua nudez são 'elas" (Levítico
18:10).

\section{Não cometer incesto com sua filha}

Por esta proibição um homem fica proibido de cometer incesto com sua
própria filha.

Esta proibição não está explicitamente enunciada na Torah; as
Escri­turas não dizem ``Não descobrirás a nudez de tua filha''. O motivo
dessa omis­são é que a proibição é evidente, pois uma vez que o incesto
com a filha de um filho ou com a filha de uma filha é proibido, é óbvio
que o incesto com uma filha é proibido.

A Guemará de Yebamot diz: "Chegou-se ao princípio que fundamenta a
proibição com a filha por interpretàção, pois Rabá disse: 'Rabi Isaac
ben Abd­mei me disse que ficamos sabendo da existência da lei pelo fato
de que ``hená'' (elas) e ``zimá'' (maldade) aparecem ambos em dois trechos
relacionados' ". Quer dizer, ao proibir o incesto com a filha de um
filho e com a filha de uma filha, as Escrituras dizem: "Pois tua nudez
são elas (hená)" (Levítico 18:10); e ao proi­bir de chegar-se a uma
mulher e a sua filha, ou a uma mulher e à filha de seu filho, ou a uma
mulher e à filha de sua filha, está dito: "Elas (hená) são parentes
próximas; mau pensamento (zimá) é" (Ibid., 17). Assim como na proibição
de chegar-se a uma mulher e à filha do filho dela ou à filha da filha
dela, também está proibido chegar-se à filha dela, de forma que a
proibição de chegar-se à filha de um filho ou à filha de uma filha
inclui a proibição de chegar-se a sua própria filha. Além disso, as
Escrituras dizem, com relação ao castigo: "E um homem que tomar uma
mulher e a sua mãe, obra de pensamento mau (zimá) é; no fogo queimarão a
ele e a elas" (Ibid., 20:14); e chegar-se a uma mulher e à filha do
filho dela ou à filha da filha dela também é punido pelo fogo, por­que a
palavra ``hená'' foi usada nos dois casos.

Na Guemará de Queretot lemos: "Nunca trate um Suezerá shavá'
levianamente pois, observe, o preceito da filha é um dos mais
importantes da Torah, e ele nos foi ensinado pelas Escrituras apenas
através de um `guezerá shavá', isto é, através do aparecimento da
palavra 'hená' em dois trechos rela­cionados, e da palavra 'zimá' em
dois outros". Observem que a expressão é ``ele nos foi ensinado'' e não
"nós o aprendemos" porque todas essas coisas nos foram transmitidas pela
Tradição através do Emissário\textsuperscript{442} e elas constituem a
explicação tradicional, como explicamos na Introdução à nossa obra, o
"Co­mentário sobre a Mishná". A razão pela qual as Escrituras não o
mencionam explicitamente é apenas porque ele pode ser deduzido através
de um ``guezerá shavá''. É isso o que o Talmud quer dizer com as palavras:
"Ele nos foi ensina­do pelas Escrithras apenas através de um Suezerá
shavá' "; ea referência à filha como "um dos mais importantes preceitos
da Torah" é o bastante.


Por todo o exposto ficou claro que o incesto com uma filha, com


442. Do emissário do Eterno. ou seja\textsubscript{;} Moisés.

a filha de uma filha, ou com a filha de um filho e punida com o fogo. Se
nin­guém tiver conhecimento da transgressão ou se nào houver eviçlèncias
concre­tas contra o transgressor, ele estará sujeito à extinção, caso o
tenha feito volun­tariamente. Aquele que cometer alguma dessas
transgressões involuntariamen­te deverá oferecer um Sacrifício
Determinado de Pecado.

\section{Não chegar-se a uma mulher e a sua filha}

Por esta proibição fica proibido chegar-se a uma mulher e a 'sua fi­lha.
Ela está expfessa em Suas palavras "Nudez de uma mulher e de sua filha
não descobrirás" (Levítico 18:17).

Aquele que violar esta proibição --- sendo uma das mulheres sua es­posa
--- está sujeito a morrer no fogo, se a prova contra ele for
evidenciada. Ele está sujeito à extinção se a transgressão for
voluntária. mas se pecou invo­luntariamente ele deverá oferecer um
Sacrifício Determinado de Pecado.

\section{Não chegar-se a uma mulher e à filha do filho dela}

Por esta proibição fica proibido de chegar-se a uma mulher e à filha do
filho dela. Ela está expressa em Suas palavras, abençoado seja Ele, "É à
filha de seu filho" (Levítico 18:17). Também neste caso o transgressor
será queima­do e punido com a extinção, se tiver pecado deliberadamente,
e oferecerá um Sacrifício Deterrriinado de Pecado se o tiver feito
involuntariamente.

\section{Não chegar-se a uma mulher e à filha da filha dela}

Por esta proibição fica proibido de chegar-se a uma mulher e à filha da
filha dela. Ela está expressa em Suas palavras, enaltecido seja Ele, "E
à filha de sua filha" (Levítico 18:17). O transgressor será punido com a
extinção e mor­rerá queimado se tiver pecado deliberadamente; sé tiver
pecado involuntaria­mente ele deverá oferecer um Sacrifício Determinado
de Pecado.

\section{Não cometer incesto com a irmã de seu pai}

Por esta proibição um homem fica proibido de cometer incesto com a irmã
de seu pai. Ela está expressa em Suas palavras "Nudez da irmã de teu pai
não descobrirás" (Levítico 18:12). O transgressor está sujeito à
extinção, se tiver pecado deliberadamente, e se o tiver feito
involuntariamente, ele deve­rá oferecer um Sacrifício Determinado de
Pecado.

\section{Não cometer incesto com a irmã de sua mãe}

Por esta proibição um homem fica proibido de cometer incesto com a Irma
de sua mãe. Ela está expressa em Suas palavras, enaltecido seja Ele, "Nu
dez da irmã de tua mãe não descobrirás.' (Levítico
18:13). Quem transgredir este preceito deliberadamente está sujeito à
extinção, e quem pecar involunta­riamente deverá oferecer um Sacrifício
Determinado de Pecado.

\section{Não chegar-se à esposa do irmão de seu pai}

Por esta proibição um homem fica proibido de chegar-se à esposa do irmão
de seu pai. Ela está expressa em Suas palavras "Não te chegarás a sua
mulher; ela é tua tia" (Levítico 18:14). O transgressor deste preceito
está sujei­to à extinção se tiver pecado voluntariamente; se o tiver
feito involuntariamen­te, ele deverá oferecer um Sacrifício Determinado
de Pecado.

\section{Não chegar-se à esposa de seu filho}

Por esta proibição um homem fica proibido de chegar-se a esposa de seu
filho. Ela está expressa em Suas palavras "Nudez de tua nora não
descobrirás" (Levítico 18:15). O transgressor será punido com o
apedrejamento; mas se a pro­va contra ele não tiver sido evidenciada, ou
se ninguém souber da transgressão, a pena é a extinção se ele tiver
pecado voluntariamente; se ele o tiver feito invo­luntariamente, ele
deverá oferecer um Sacrifício Determinado de Pecado.

\section{Não chegar-se à esposa de seu irmão}

Por esta proibição um homem fica proibido de chegar-se á esposa de seu
irmão. Ela está expressa em Suas palavras "Nudez da mulher de teu ir­mão
não descobrirás" (Levítico 18:16). O transgressor deste preceito está
sujei­to à extinção se tiver pecado voluntariamente; se o tiver feito
involuntariamen­te, ele deverá oferecer um Sacrifício Determinado de
Pecado

\section{Não chegar-se à irmã de sua esposa enquanto esta última for viva}

Por esta proibição um homem fica proibido de chegar-se à irmã de sua
esposa enquanto esta viver. Ela está expressa em Suas palavras,
enaltecido seja Ele, ``E a mulher com sua irmã não tomarás'' (Levítico
18:18). Aquele que transgredir este preceito voluntariamente está
sujeito à extinção; aquele que pe­car involuntariamente deverá oferecer
um Sacrifício Determinado de Pecado.

\section{Não unir-se a uma mulher menstruada}

Por esta proibição um homem fica proibido de unir-se a uma mu­lher
menstruada durante o período de sua impureza, ou seja, durante os sete
dias completos. Ela está expressa em Suas palavras "E a mulher na impureza de
sua menstruação não te chegarás" (Levítico 18:19); e enquanto ela não
tiver feito uma imersão\textsuperscript{443} depois de completados os
sete dias, ela será considerada menstruada.

O transgressor voluntário deste preceito está sujeito à extinção; aquele
que pecar involuntariamente deverá oferecer um Sacrifício Determinado de
Pecado.

\section{Não chegar-se à esposa de outro homem}

Por esta proibição um homem fica proibido de chegar-se à esposa de um
outro homem. Ela está expressa em Suas palavras "E com a mulher de teu
companheiro não te deitarás para dar sêmen" (Levítico 18:20).

A punição pela violação deste preceito varia de acordo com as
cir­cunstâncias. Se a mulher for noiva\textsuperscript{444} ambos ficam
sujeitos ao apedrejamen­to, como determinam as
Escrituras\textsuperscript{445}. Se ela for a filha de um ``Cohen'', ela
deverá morrer queimada e o homem estrangulado. Se ela for a filha de um
is­raelita, ambos estão sujeitos à morte por estrangulamento. Tudo isso
se aplica se a prova for evidenciada, caso contrário o homem fica
sujeito à extinção. Tam­bém neste caso tudo isso se aplica se o pecado
tiver sido cometido voluntaria­mente pelo homem, mas se ele o cometeu
involuntariamente, ele deverá ofere­cer um Sacrifício Determinado de
Pecado.

A proibição desta transgressão aparece em outro lugar, nos Dez
Man­damentos, em Suas palavras ``Não adulterarás'' (Êxodo 20:14), que
significam que não se deve chegar-se à esposa de outro homem. Nas
palavras da Mekhiltá: 'Por que foi dito 'Não adulterarás'? Porque nas
palavras `Certamente serão mor­tos, o adúltero e a adúltera' (Levítico
20:10) ouvimos a penalidade, mas não ouvimos a proibição. Por essa razão
as Escrituras dizem: `Não adulterarás' ". Da mesma forma, a Sifrá diz: "
'O homem que cometer adultério com a mulher de outro homem, que
adulterar com a mulher de seu próximo' (Ibid.): ouvi­mos aqui a
penalidade, mas não ouvimos a proibição. Por essa razão as Escritu­ras
dizem: 'Não adulterarás' a ambos o homem e a mulher". Eles não
encontra­ram a proibição nas palavras "E com a mulher de teu companheiro
não te dei­tarás para dar sêmen" porque essa proibição não inclui ambos
o adúltero e a adúltera, mas é dirigida apenas ao adúltero. Da mesma
forma, no que se refere às relações proibidas em geral, eles tiveram que
aplicar a proibição também à mulher, e por isso lemos na Sifrá: "
'Nenhum \emph{de vós} se chegará... para desco­brir a sua nudez' (Ibid.,
18:6) proíbe ambos o homem através da mulher e a mulher através do
homem".

A Guemará de Sanhedrin diz: "Todos estão incluídos nos termos `adúltero'
e 'adúltera' mas as Escrituras excluem a filha de um 'Cohen', ensi­nando
que ela deve ser queimada, e a moça noiva, ensinando que ela deve ser
apedrejada".


Nós explicamos este assunto na Introdução ao presente trabalho.

\begin{enumerate}
\def\labelenumi{\arabic{enumi}.}
\setcounter{enumi}{442}
\item
 
 Num banho ritual.
 
\item
 
 A mulher noiva, ou prometida legalmente, está no estágio preliminar ao
 casamento, que acar­reta todas 'as conseqüências legais deste.
 
\item
 
 Vide Deuteronômio 22:23-24.
 
\end{enumerate}



\section{Os homens não podem deitar-se com animais}

Por esta proibição um homem fica proibido de se deitar com um ani­mal,
macho ou fêmea. Ela está expressa em Suas palavras "E com qualquer
ani­mal não te deitarás" (Levítico 18:23). O transgressor voluntário
está sujeito à morte por apedrejamento, e se não for apedrejado, à
extinção Se ele pecou involuntariamente, deverá oferecer um Sacrifício
Determinado de Pecado.

\section{As mulheres não podem deitar-se com animais}

Por esta proibição as mulheres ficam proibidas de se deitarem éom
animais. Ela está expressa em Suas palavras, enaltecido seja Ele, ''Nem
a mulher se porá diante de um animal para se juntar com ele" (Levítico
18:23). Este é um preceito independente, não incluído no precedente, uma
vez que as Escri­turas, ao proibir os homens de se deitar com animais,
não impõem a mesma proibição às mulheres, na ausência de um preceito
negativo específico a elas. Assim, no princípio de Queretot lemos: "Há
trinta e seis ofensas pelas quais a Torah prescreve a extinção", e a
enumeração delas que se segue inclui as co­metidas por um homem que se
deita com um animal e por uma mulher que se deita com um animal, embora
sejam enumeradas apenas categorias gerais, como explicamos em nosso
``Comentário''. Dessa forma fica claro que esta proi­bição é um preceito
independente, e deve ser incluída na lista dos preceitos negativos.

Aquela que violar este preceito voluntariamente está sujeita ao
ape­drejamento; se a prova não for evidenciada, ela está sujeita à
extinção se tiver pecado voluntariamente. Se tiver pecado
involuntariamente, deverá oferecer um Sacrifício Determinado de Pecado.

\section{Um homem não pode chegar-se a outro homem}

Por esta proibição um homem fica proibido de chegar-se a um va­rão. Ela
está expressa em Suas palavras "E com um homem não te deitarás co­mo se
fosse uma mulher" (Levítico 18:22), e aparece também em outro lugar, em
Suas palavras "Nem haverá destinado à pederastia dentre os filhos de
Is­rael" (Deuteronômio 23:18). Este preceito negativo está repetido para
dar maior força e não para dirigir a proibição à vítima. As palavras das
Escrituras ``E com um homem não te deitarás'' estipulam a advertência às
duas partes.

A Guemará de Sanhedrin diz que é Rabi Ishmael que considera "Nem haverá
destinado à pederastia dentre os filhos de Israel" como a proibição
diri­gida à vítima. Conseqüentemente, "Aquele que cometer pederastia
ativamen­te e que também permitir que o ofendam dessa forma, de maneira
negligente, está sujeito, de acordo com o ponto de vista de Rabi
Ishmael, a duas penalida­des; mas Rabi Akiba diz que isso é
desnecessário porque "E com um homem não te deitarás (lo tishcab) como
se fosse mulher" pode ser lido como "Não serás deitado (lo tishacheb)".
Portanto, aquele que cometer pederastia e também permitir que
o ofendam dessa forma, de maneira negligente, estará
sujeito a uma penalidade apenas, pois ``lo tishcab'' (Não te deitarás) e
``lo tishacheb'' (Não serás deitado) são um único preceito, e na opinião
de R. Akiba o objetivo de ``Nem haverá destinado à pederastia'' é apenas
para reforçar ó preceito, da mesma forma que a fim de reforçar "Não
adulterarás" (Êxodo 20:14) que é, como já explicamos, a proibição da
mulher de outro homem, temos "E com a mulher de teu companheiro não te
deitarás para dar sêmen" (Levítico 18:20).


Há muitos casos deste tipo, como explicamos no Nono Fundamento.


O transgressor deste preceito será punido com o apedrejamehto. Se ele
não for apedrejado, estará sujeito à extinção se o pecado tiver sido
voluntá­rio; se ele o tiver cometido involuntariamente, ele deverá
oferecer um Sacrifí­cio Determinado de Pecado.

\section{Um homem não pode chegar-se a seu pai}

Por esta proibição um homem fica proibido de chegar-se a seu pai. Elz
está expressa em Suas palavras "A nudez de teu pai... não descobrirás"
(Le­vítico 18:7). Também neste caso o transgressor está sujeito ao
apedrejamento. Assim, um homem que se chegar a seu pai será culpado duas
vezes: por ser um varão e por ser seu pai.

A Guemará de Sanhedrin explica que Suas palavras "A nudez de teu pai...
não descobrirás" se referem na realidade ao pai. A isso se objetou o
se­guinte: "Mas não sabemos disso através do versículo 'E com homem não
te dei­tarás como se fosse mulher'? (Ibid 22)", ao que se respondeu:
"Isso nos ensina que se incorre numa penalidade dupla. Rabi Yehudá disse
que um pagão que cometer pederastia com seu pai incorre numa penalidade
dupla". Para explicar isso eles dizem ali: "Essas palavras de Rabi
Yehudá se referem supostamente a um judeu\textsuperscript{246},
inconscientemente, e\textsuperscript{247} um sacrifício; e o termo
'pagão' é um eufemismo". Quer dizer, aquele que inconscientemente se
chegar a seu pai de­verá oferecer dois Sacrifícios de Pecado, da mesma
forma que aquele que in­conscientemente se chegar a duas mulheres
proibidas; mas se ele não for seu pai, ele deverá oferecer apenas um
Sacrifício de Pecado.

\section{Um homem não pode chegar-se ao irmão de seu pai}

Por esta proibição um homem fica proibido de chegar-se ao irmão de seu
pai. Ela está expressa em Suas palavras "Nudez do irmão de teu pai não
descobrirás" (Levítico 18:14). Portanto, aquele que involuntariamente se
che­gar ao irmão de seu pai deverá oferecer dois Sacrifícios de Pecado,
como expli­camos no caso de seu pai. Na Guemará de Sanhedrin lemos:
"Todos concor­dam que aquele que cometer pederastia com seu tio paterno
incorre numa pe­nalidade dupla, pois as Escrituras dizem: 'Nudez do
irmão de teu pai não descobrirás' ".


Vocès devem saber que toda vez que eu usar a expressão "a prova


446 Que cometer a ofensa.

447 Que devera oferecer.


for evidenciada.' eu quero dizer que deve haver duas
ou mais testemunhas qua­lificadas que tenham feito uma advertência, e
que tenham apresentado a prova diante de um Tribunal qualificado de
vinte e três juízes, e que isto se aplica apenas enquanto estiverem em
vigor as leis da pena capital.

Está claro que, no caso de todos os delitos carnais mencionados, as
Escrituras estabelecem explicitamente a penalidade de extinção porque
após enumerá-los elas dizem: "Aquele que fizer alguma destas
abominações, serão banidas as almas que o fizerem" (Levítico 18:29). Da
mesma forma, toda vez que afirmamos que uma ofensa desse tipo acarreta a
morte por sentença judi­cial, essa penalidade também está prescrita nas
Escrituras. Mas quanto ao modo de execução, que em alguns casos está
determinada que seja pelo apedrejamento, em outros pelo estrangulamento,
e em outros pelo fogo, a autoridade é, em alguns casos a Tradição, e em
outros as Escrituras.

As normas das leis relativas a todos esses delitos carnais estão
expli­cadas nos Tratados Sanhedrin e Queretot, e em vários trechos de
Yebamot, Que­tubot e Kidushin.

Está explicado no início de Queretot que no caso de qualquer peca­do
cuja penalidade é a morte se cometido voluntariamente e a oferta de
Sacrifí­cio Determinado de Pecado se cometido involuntariamente, a
pessoa de cuja culpa se tem dúvidas deverá oferecer um Sacrifício
Suspensivo de Delito\textsuperscript{448}. O termo "Sacrifício
Determinado de Pecado" significa que o sacrifício deve ser sempre de
rebanho, isto é, uma ovelha ou uma cabra.

Se vocês observarem todos os preceitos negativos e examinarem os
castigos relativos a cada um deles vocês perceberão que no caso de cada
peca­do pelo qual a pena é a extinção, se cometido voluntariamente, e a
oferta de um Sacrifício de Pecado, se cometido involuntariamente, o
sacrifício referido é um Sacrifício Determinado de Pecado, com a exceção
de dois pecados, pelos quais a pena é a extinçào, se cometidos
voluntariamente, e um Sacrifício de Maior ou Menor Valor, se cometidos
involuntariamente, ao invés de um Sacri­fício Determinado de
Pecado\textsuperscript{449}. Esses dois pecados são a profanação do
San­tuário e a profanação de suas Santidades. Por "profanação do
Santuário" eu quero dizer a entrada de uma pessoa impura no Campo do
Santuário ("az-hara:') e por ``profanação de suas Santidades'' eu quero
dizer que uma pessoa que te­nha se tornado impura coma a carne das
ofertas consagradas.

Da mesma forma, ficará claro para vocês que pela violação de qual­quer
preceito negativo, cuja penalidade é a extinção se cometida
voluntaria­mente, fica-se obrigado a oferecer um Sacrifício de Pecado se
ela tiver sido co­metida involuntariamente, exceto no caso da blasfêmia,
a qual acarreta a extin­ção se cometida voluntariamente mas não obriga a
um Sacrifício de Pecado se cometida involuntariamente.

Ficará também claro para vocês que todos aqueles que estiverem su­jeitos
à morte por uma das quatro formas de execução judicial estarão sujeitos
à extinção se o Tribunal não os executar ou não souber de suas
transgressões, exceto em dez casos, nos quais as pessoas ficam sujeitas
à morte por sentença judicial, mas não à extinção, a saber: quem desviar
outrem\textsuperscript{45()} do caminho certo, quem desencaminhar outras
pessoas do caminho certo 'S I, num falso profeta\textsuperscript{452},


\begin{enumerate}
\def\labelenumi{\arabic{enumi}.}
\setcounter{enumi}{447}
\item
 
 Ver O preceito) positivo .0.
 
\item
 
 Ver o preceito) positivo 72.
 
\item
 
 Ver o preceito negativo 15.
 
\item
 
 Ver o preceito negativo 16.
 
\item
 
 Ver o preceito negativo \emph{2'.}
 
\end{enumerate}

quem profetizar em nome de um ídolo 
". um ancião que
menosprezar a deci­são do Supremo Tribunal\textsuperscript{454}, um
filho teimoso e rebelde \textsuperscript{55}, quem raptar um
israelita\textsuperscript{45}". um assassino"., quem
bater em seu pai ou em sua mãe\textsuperscript{458} e quem amaldiçoar
seu pai. ou sua mãe\textsuperscript{459}. Em cada um desses casos, se a
prova for evi­denciada, ele será morto; mas se o Tribunal nao souber de
sua transgressão ou não tiver podido condená-lo à morte, ele se expôs à
morte mas não corre o perigo de extinção.


Vocês devem conhecer e lembrar-se desses princípios.


\section{Não ter intimidades com uma parenta}

Por esta proibição somos proibidos de procurar prazer no contato com
qualquer parenta que esteja na categoria das mulheres proibidas, mesmo
que seja apenas através de abraços, beijos e coisas assim. A proibição
de tal con­duta está expressa em Suas palavras, enaltecido seja Ele,
"Nenhum de vós se chegará a aquele que lhe é próximo por carne, para
descobrir a sua nudez" (Le­vítico 18:6) que tem o mesmo significado que
se Ele tivesse dito: "Vocês não devem se aproximar deles de forma tal
que possa conduzir a uma relação proi­bida". Sobre isso a Sifrá diz: "
'Se chegará... para descobrir a sua nudez' proíbe apenas o fato de
chegar-se a ela; de que forma fico sabendo que é proibido
'aproximar-se'? Pelas palavras das Escrituras: 'E à mulher na impureza
de sua menstruação \emph{não te chegarás'} (Ibid., 19). Contudo,
isto.proíbe apenas aproximar-se de uma mulher menstruada e chegar-se a
ela; de que forma fico sabendo que as duas proibições se aplicam a todas
as mulheres das categorias proibidas? Pe­las palavras \emph{'Se chegará}
a aquele que lhe é próximo por carne' ".

A Sifrá diz também: " 'Serão banidas as almas que o fizerem' (Ibid.,
29). Por que foi dito isso? Porque pelas palavras 'Se aproximar' eu
poderia pen­sar que se fica sujeito à extinção pela simples
'aproximação'. Por isso as Escri­turas dizem: 'Que \emph{°fizerem'} mas
não as almas que simplesmente se aproxima­rem delas".

A proibição de tal conduta indecente aparece novamente nas pala­vras
``Para não fazer nenhum dos costumes abomináveis'' (Ibid., 30). Mas o
ver­sículo contendo as duas proibições, a saber, "Segundo as obras da
terra do Egi­to, na qual estivestes, não fareis, e segundo as obras da
terra de Canaã, à qual Eu vos levo, não fareis" (Ibid., 3) nos proíbe
não apenas a prática dos "costu­mes abomináveis", mas proíbe também os
atos abomináveis específicos que Ele expõe nos versículos seguintes.
Portanto estas duas proibições são de ex­tensão global e cobrem todas as
categorias proibidas, bem como nos proíbem de fazer "Segundo as obras da
terra do Egito ... e segundo as obras da terra de Canaã", que englobam
todos os seus costumes, tanto os de libertinagem, como os de
agricultura, os de criação de gado ç os de vida social. Depois Ele


\begin{enumerate}
\def\labelenumi{\arabic{enumi}.}
\setcounter{enumi}{452}
\item
 
 Ver o preceito negativo 26.
 
\item
 
 Ver o preceito negativo 312.
 
\item
 
 Ver o preceito negativo 195.
 
\item
 
 Ver o preceito negativo 243.
 
\item
 
 Ver o preceito negativo 289.
 
\item
 
 Ver o preceito negativo 319.
 
\item
 
 Ver o preceito. negativo 318.
 
\end{enumerate}



passa a explicar que essas ``obras'' que Ele proíbe de praticar são
relações ilíci­tas com este e aquele, como fica claro pelas palavras que
concluem o relato: "Porque todas estas abominações fizeram os homens da
terra que estavam an­tes de vós" (Ibid., 27).

A Sifrá diz: "Eu poderia pensar que não devemos construir nossas casas
nem plantar vinhedos da maneira como o fazem as outras nações; por isso
as Escrituras dizem: 'Não andeis segundo os seus costumes' (Ibid. 3),
que significam que o decreto se aplica apenas aos costumes que eles e
seus ances­trais prescreveram por lei". Diz também: "O que costumavam
eles fazer? Um homem se casava com um homem, uma mulher se casava com
uma mulher, e uma mulher se casava com dois homens". Portanto, fica
claro que esses dois preceitos negativos --- a saber, "Segundo as obras
da terra do Egito ... não fa­reis e segundo as obras da terra de Canaã
... não fareis" --- proíbem em termos gerais todas as relações ilícitas
e são seguidos por proibições específicas relati­vas a cada uma das
relações proibidas individualmente\textsuperscript{460}.

Nós mesmo já explicamos todas as normas deste preceito em nosso
``Comentário sobre a Mishná'', no sétimo capítulo de Sanhedrin, onde
explica­mos que são punidas pelo açoitamento.

Também é importante para vocês saber que o fruto de uma relação pela
qual se está sujeito à extinção é chamado ``um bastardo''. O Eterno
cha­mou tal fruto de ``bastardo'', e quer o pecado tenha sido cometido
voluntária ou involuntariamente, seu fruto será ``um bastardo''. A única
exceção é o fruto da relação com uma mulher menstruada; nesse caso esse
fruto não será um bas­tardo, mas será chamado de "filho de uma mulher
menstruada". Isto está ex­plicado no quarto capítulo de Yebamot.

\section{Um ``mamzer'' não pode chegar-se a uma israelita}

Por esta proibição um bastardo fica proibido de chegar-se a uma
is­raelita. Ela esta expressa em Suas palavras, enaltecido seja Ele,
"Não entrará bas­tardo na congregação do Eterno" (Deuteronômio 23:3).

A contravenção a esta proibição será punida com o açoitamento. As normas
deste preceito estão explicadas no oitavo capítulo de Ye­bamot e no
final de Kidushin.

\section{Não chegar-se a uma mulher antes do casamento}

Por esta proibição um homem fica proibido de chegar-se a uma mu­lher sem
estar devidamente casado com ela. Ela está expressa em Suas palavras,
enaltecido seja Ele, "Não haverá mulher destinada à prostituição dentre
as fi­lhas de Israel" (Deuteronômio 23:18), e aparece sob outra forma em
Suas palà­vras, enaltecido seja Ele, "Não profanarás a tua filha para
fazê-la prostituta" (Le­vítico 19:29), sobre as quais a Sifrá diz: "
'Não profanarás a tua filha para fazê-la prostituta' se refere a alguém
que entregue sua filha solteira à luxúria ou a uma mulher que se
entregue à luxúria".

460. Ver o preceito positivo 38.

Deixem que eu explique porque Ele repete este preceito dessa forma e o
que a repetição acrescenta. Nós já havíamos recebido Sua
lei,.enaltecido seja Ele, segundo a qual um hómem que
seduzir ou forçar uma moça não está sujeito a nenhum castigo a não ser a
pagar uma multa em dinheiro e a casar-se com a moça, como determinam as
Escrituras\textsuperscript{461}. De acordo com isso poderíamos pen­sar
que, uma vez que envolve apenas uma penalidade em dinheiro, este caso é
como qualquer outro que envolve dinheiro e que assim como uma pessoa é
livre de dar seu dinheiro a quem ele quiser ou de liberar uma outra
pessoa de uma importância devida a ele, assim também seria permitido que
alguém deixas­se que um homem se chegasse a sua filha solteira e
renunciasse ao pagamento em dinheiro, uma vez que isso --- ou seja, os
cinquenta siclos de prata --- perten­ce ao pai da moça por direito; ou
que alguém desse sua filha a um homem em troca de uma soma em dinheiro.
Qualquer idéia desse tipo está impedida pela proibição "Não profanarás a
tua filha para fazê-la prostituta" porque o dinheiro está prescrito
apenas nos casos de sedução ou força, e fazer tal coisa de mútuo acordo
é totalmente ilegal. A razão para isso está ém Suas palavras "Para que a
terra não seja entregue à prostituição e que não se encha a terra de
pensamentos maus" (Levítico 19:29). Sedução e força são raros mas se tal
conduta fosse per­mitida e legalizada, a impudicícia se tornaria comum.
Esta é uma bela e adequa­da explicação do versículo em questão, e está
em harmonia com os ensinamen­to de nossos Sábios e com as prescrições da
Torah.

A contravenção a esta proibição relativa a moças solteiras será puni­da
com o açoitamento.


As normas deste preceito estão explicadas em Quetubot e Kidushin.


\section{Não tornar a casar-se com a esposa de quem se divorciou,
depois que ela tenha se casado novamente}

Por esta proibição um homem fica proibido de tornar a casar-se com a
esposa de quem ele se divorciou se ela tiver estado casada com outro.
Ela está expressa em Suas palavras, enaltecido seja Ele, "Não poderá seu
primeiro marido, que a despediu, tornar a tomá-la, para que seja sua
mulher, depois que foi contaminada" (beuteronômio 24:4).

A contravenção a esta proibição será punida com o açoitamento. As normas
deste preceito estão explicadas em várias passagens em

Yebamot.

\section{Não chegar-se a uma mulher sujeita ao casamento levirato}

Por esta proibição outros homens ficam proibidos de chegar-se a uma
viúva sujeita ao casamento levirato\textsuperscript{462}. Ela está
expressa em Suas palavras "A mulher do defunto não se casará com homem
estranho de fora" (Deuteronô­mio 25:5).


\begin{enumerate}
\def\labelenumi{\arabic{enumi}.}
\setcounter{enumi}{460}
\item
 
 Vide Deuteronômio 22:28-29 e *..xodo 22:15-16.
 
\item
 
 Ver o preceito positivo 216.
 
\end{enumerate}



A contravenção a esta proibição será punida com o açoitamento do homem e
da mulher.

As normas deste preceito estão explicadas em Yebamot.

\section{Não se divorciar da mulher que se violentou e com a qual se foi
obrigado a casar}

Por esta proibição um homem fica proibido de divorciar-se da mu­lher que
ele violentou. Ela está expressa em Suas palavras "Ela lhe será por
mu­lher ... e não a poderá despedir por todos os seus dias"
(Deuteronômio 22:29).

Este preceito negativo está precedido pelo preceito positivo "Ela lhe
será por mulher"\textsuperscript{463}, e isso está exposto na Guemará de
Macot, que prossegue assim: "Um violentador israelita que tiver se
divorciado de sua mulher, poderá casar-se novamente com ela sem ficar
sujeito ao açoitamento; mas se ele for um 'Cohen', ele será açoitado e
não poderá casar-se novamente com ela".

Vocês devem saber que se um israelita se divorciar de uma mulher com
quem foi obrigado a casar-se, e se ela morrer antes que ele torne a se
casar com ela, ou se ela se casar com outro, ele será punido com o
açoitamento pois não terá cumprido o preceito positivo em questão. Isto
está de acordo com o principio aceito de que "Se ele tiver
cumprido\textsuperscript{464}, mas se ele não o tiver
cumprido\textsuperscript{465}".

As normas deste preceito estão explicadas no terceiro e quarto
capí­tulos de Quetubot.

\section{Não se divorciar de uma mulher depois de tê-la caluniado}

Por esta proibição um homem fica proibido de divorciar-se de sua mulher
depois de tê-la caluniado. Ela está expressa em Suas palavras (que são
usadas neste caso também) "Não a poderá despedir, por todos os seus
dias" (Deuteronômio 22:19).

Este preceito negativo também está precedido pelo positivo que es­tá em
Suas palavras ``E lhe será por mulher'' (Ibid.)\textsuperscript{466}.
Portanto, se ele se di­vorciar dela estará sujeito ao açoitamento, assim
como quem for culpado de forçar uma mulher, de acordo com o que está
explicado no final do Tratado Macot.

Ali, e no terceiro e no quarto capítulos de Quetubot, as normas des­te
preceito estão explicadas.


\begin{enumerate}
\def\labelenumi{\arabic{enumi}.}
\setcounter{enumi}{462}
\item
 
 Ver o preceito positivo 218.
 
\item
 
 O preceito positivo, ele estará isento.
 
\item
 
 Embora ele não o tenha invalidado por decisào própria, ele é
 considerado culpada.
 
\item
 
 Ver o preceito positivo 219.
 
\end{enumerate}



\section{Um homem incapaz de procriar não pode casar-se com uma israelita}

Por esta proibição um homem que tiver sofrido um acidente que o tenha
privado do poder da procriação está proibido de casar-se com uma mu­lher
israelita. Esta proibição está expressa em Suas palavras, enaltecido
seja Ele, • 'Não entrará aquele que tem os testículos trilhados e aquele
cujo derrame de sêmen é deficiente, na congregação do Eterno '
(Deuteronômio 22:2), Se um homem assim chegar-se a uma mulher israelita
depois do noivado\textsuperscript{46}", ele será punido com o
açoitamento.


As normas deste preceito estão explicadas no nono capítulo de


Yebamot.

\section{Não castrar}

Por esta proibição somos proibidos de castrar um macho de qual­quer que
seja a espécie, homem ou animal. Ela esta expressa nas últimas pala­vras
do versículo "De testículos machucados, ou moídos, ou desprendidos, ou
cortados, não oferecereis ao Eterno, nem fareis estas coisas na terra"
(Levítico 22:24) que a Tradição explica como significando: ,"Nem fareis
isso a vós mesmos"

.

A contravenção a esta proibição, ou seja, a castração de qualquer
es­pécie, é punida com o açoitamento.

No capítulo ``Shemona Sheratsim'' lemos: "Como sabemos que a cas­tração de
um homem é proibida? Através do versículo 'Nem fareis estas coisas na
terra', portanto não fareis isso a vós mesmos. Mesmo se alguém castrar
de­pois que um outro já o tenha feito, ele será culpado. Rabi Hiya ben
Abun diz, em nome do Rabi Yohanan: Todos concordam que se alguém
preparar leveda­do depois que alguém já o tenha preparado levedado ele é
culpado, porque está dito: 'Não será cozido levedado' (Levítico :10) e
'Será preparada com fer­mento (Ibid., 2:11). Se alguém castrar depois
que outro já tenha castrado ele será culpado porque está dito: `De
testículos machucados, ou moídos, ou des­prendidos, ou cortados ...' Se
alguém é culpado por cortá-los, imaginem o quanto não o será por
despedaçá-los! Isso é para ensinar-nos que se alguém cortá-los depois de
outro despedaçá-los, ele será culpado."

As normas deste preceito estão explicadas em várias passagens em Shabat
e Yebamot.

\section{Não nomear um rei que não seja israelita de nascimento}

Por esta proibição somos proibidos de nomear um rei sobre nós que não
seja israelita de nascimento, mesmo que ele seja um Prosélito Justo. Ela
es­tá expressa em Suas palavras, enaltecido seja Ele, "Não poderás pôr
sobre ti um homem estranho, que não seja teu irmão" (Deuteronômio
17:15), sobre

467. O noivado, como preliminar ao casamento, contém todas as
conseqüencias legais deste
as quais o Sifrei diz: " 'Não poderás pôr sobre ti um homem estranho' é
um preceito negativo".

Da mesma forma, com referência a todas as outras nomeações, se­jam elas
religiosas ou governamentais, não podemos nomear sobre nós mes­mos um
homem prosélito, a menos que sua mãe seja uma israelita; isto é
decor­rente de Suas palavras, enaltecido seja Ele, "Poderás, certamente,
por sobre ti o rei ... dentre teus irmãos, porás rei sobre ti' '
(Ibid.), que o Talmud interpreta como significando: "Todas as nomeações
que fizerdes deverão ser apenas 'dentre teus irmãos' -468.

No que se refere à realeza, vocês já sabem, pelas sagradas escrituras
dos profetas, que Davi foi considerado digno do título pois o Talmud diz
clara­mente: "A coroa da realeza, Davi mereceu e recebeu", assim como
todos os de sua linhagem até o fim de todas as gerações. Para aquele que
crê na Torah de Moisés, nosso mestre, não pode haver rei que não seja
descendente de Davi atra­vés de Salomão apenas; e alguém que não seja
dessa linhagem nobre é conside­rado um ``estrangeiro'', no que se refere à
realeza, assim como todo aquele que não for descendente de Aarão é
considerado um ' estranho", no que se refere a oficiar no Santuário.
Isto está claro, e não há dúvida alguma a esse respeito.

As normas deste preceito estão explicadas em várias passagens de
Yebamot, Sanhedrin, Sotá e Nidá. •

\section{Um rei não pode possuir muitos cavalos}

Por esta proibição o rei fica proibido de possuir muitos cavalos. Ela
está expressa em Suas palavras, enaltecido seja Ele, "Não multiplicará
para si cavalos" (Deuteronômio 1 7: 16).

O limite permitido é que ele não deve ter cavalos correndo diante dele,
nem possuir um único cavalo a não ser aquele que ele monta. Ele pode
manter cavalos em seus estábulos para que seu exército monte em tempos
de guerra, mas só lhe é permitido ter um animal para seu uso privado.


As normas deste preceito estão explicadas no segundo capítulo de


Sanhedrin.

\section{Um rei não pode ter muitas esposas}

Por esta proibição um rei fica proibido de ter muitas esposas. Ela es­tá
expressa em Suas palavras, enaltecido seja Ele, "Não multiplicará para
si mu­lheres" (Deuteronômio 17:17). O limite permitido é que ele nào
pode ter mais do que dezoito esposas, por matrimônio devidamente
contraído.

Os estatutos deste preceito já foram explicados no segundo capítulo de
Sanhedrin.

\section{Um rei não pode acumular grande fortuna pessoal}

Por esta proibição o rei fica proibido de acumular uma grande fortu­na
para si próprio. Ela está expressa em Suas palavras, enaltecido seja
Ele, "Pra­ta e ouro não multiplicará muito para si" (Deuteronômio 1 7:1
7). O limite per­mitido é que .ele não deve ir além do
que é estritamente necessário para a ma­nutenção de seu exército e de
seus criados pessoais. Contudo, é-lhe permitido acumular riquezas para
as necessidades de todo o povo de Israel.

O Enaltecido explica nas Escrituras a razão desses três preceitos, a
saber, ``Não multiplicará para si cavalos'', "Não multiplicará para si
mulheres"
 ``Prata e ouro não multiplicará muito para Si''\textsuperscript{469}; e
 o conhecimento dessas razões levou a sua desobediência, como no caso
 notório de Salomão, a paz es­teja com ele, apesar da superioridade de
 seu conhecimento e sabedoria, e de ser ele "o amado do
 Eterno"\footnote{Sam. 12:25.}.


Nossos Sábios aprenderam com isso que se os homens soubessem as razões
para todos os preceitos, eles encontrariam meios de desobedecê-los. Pois
se um homem tão perfeito supôs erroneamente que sua atitude não o
leva­ria de modo algum a uma transgressão, as massas, providas de mentes
simples, seriam mais facilmente levadas a desobedecê-los, argumentando o
seguinte: Ele proibiu isto e ordenou aquilo apenas por tal e tal razão,
por isso vamos evitar o pecado cuidadosamente para impedir aquilo que
este preceito estabeleceu, mas não seremos minunciosos com relação ao
preceito em si; e isto destruiria a própria base da Religião. Por essa
razão o enaltecido não expôs as razões, mas não há um único preceito que
não tenha uma razão e uma causa, remotas ou imediatas. A maioria dessas
causas e razões, contudo, está acima da inteligência
 da compreensão das massas; contudo, o Profeta testemunha com relação a
 todos\footnote{A todos os preceitos.}: "Os preceitos do Eterno são justos e
 alegram o coração; o manda­mento do Eterno é puro e ilumina os
 olhos"\textsuperscript{Salmos 19:9.}.

Eu imploro a ajuda do Eterno para cumprir tudo o que ele ordenou
para abster-me de tudo o que Ele proibiu.

Aqui termina o que tencionamos incluir neste trabalho.


\chapter{Glossário}


%% Letra pequena
  \begingroup\footnotesize
\begin{itemize}

\item[\textbf{Abayé}] Comentarista famoso da Mish­ná de. origem Babilônica,
do século IV. \textbf{Abodá Zará} Adoração de ídolos. \textbf{Agudot} 
Aglomerações de pessoas. \textbf{Ah-Ab} Rei que matou o sábio Nabot
para apossar-se de sua propriedade. \textbf{Aharé Mot} Trecho de
leitura sema­nal do 3? livro do Pentateuco chama­do Levítico.

\item[\textbf{Akh}] Mas.

\item[\textbf{Al}] Negação.

\item[\textbf{Amá}] Cúbito. Medida equivalente a 48 centímetros.

\item[\textbf{Amalec}] Neto de Esaú, que herdou de seu avô o ódio mortal que
este tinha por seu irmão gêmeo, Jacob, e por to­dos os seus
descendentes.

\item[\textbf{Amon}] Irmão de Moab, frutos ambos da união de Lot (sobrinho do
patriar­ca Abraham) com suas duas filhas, os três únicos sobreviventes
de Sodoma e Gomorra. Pensando ter sido o mun­do destruído e não haver
mais nenhum homem sobre a terra, as filhas de Lot embebedaram o próprio.
pai para se unirem a ele. Dessa união a filha mais
velha concebeu um filho, a quem deu o nome de Moab, e a mais nova teve
outro filho, que chamou de Amon. Ao sair do Egito e atravessar o deserto
pa­ra atingir a Terra Prometida, o povo ju­deu precisava passar por
muitos po­voados. Amon e Moab, proibiram ter­minantemente a passagem dos
judeus por suas cidades, e chegaram a contra­tar o maior mago da terra,
Bil-An, para amaldiçoar o povo de Israel (vide Nú­meros, capítulo 22).
Diante de tal ati­tude, proibiu-se aos judeus que se ca­sassem com os
integrantes daqueles povos.

\item[\textbf{Amonita}] Povo descendente de Amon.

\item[\textbf{Amorita ou Emorita}] Um dos povos que vivia na terra de Canaã.

\item[\textbf{Arakhin}] Capítulo de um dos tratados do Talmud.

\item[\textbf{Ashera}] Deusa da fertilidade entre o povo de Canaã.

\item[\textbf{Atzeret}] Nome dado ao sétimo dia de Pessah (Páscoa) e ao
oitavo dia de Su­cot (Festa das Cabanas).

\item[\textbf{Avon}] Pecado.

\item[\textbf{Az-Hará}] Alertar-se.

\item[\textbf{Azariah}] Um dos três sábios que san­tificaram o nome de Deus
na época de Nabucodonosor penetrando na caldei­ra de fogo e saindo
totalmente ilesos. \textbf{Azanechá} Cinto bélico no qual se pendura
todo armamento do soldado em guerra.

\item[\textbf{Bali}]  Nome dado a um ídolo.

\item[\textbf{Baba Batra}] Capítulo de um dos tra­tados do Talmud.

\item[\textbf{Baba Kamma}] Capítulo de um dos tratados do Talmud.

\item[\textbf{Baba Metzia}] Capítulo de um dos tra­tados do Talmud.

\item[\textbf{Bakar}] Gado.

\item[\textbf{Bekhorot}] Primogênitos. Também nome de um Capítulo de um dos
trata­dos do Talmud.

\item[\textbf{Belinta}] Esteira.

\item[\textbf{Berakhot}] Bênçãos. Também nome de um capítulo de um dos
tratadós do Talmud.

\item[\textbf{Beit Hashoebá}] Local onde ha: via uma fonte
cuja água era usada na 2? noite da festa de Sucot (Cabanas) na época do
Templo Sagrado.

\item[\textbf{Betzá}] Capítulo de um dos tratados do Talmud.

\item[\textbf{Bicurim}] Primícias (de Pentecostes). \textbf{Bitlo ve lo Bitlo} Não cumpriu e cumpriu.

\item[\textbf{Bitrumath}]Contribuição espontânea entregue ao sacerdote na época do
Templo.

\item[\textbf{Caleb}] Representante chefe da tribo de Judá. Cada uma das 12
tribos tinha o seu representante chefe.

\item[\textbf{Casher}] O que é permitido em matéria de alimento, segundo a
religião judaica. \textbf{Cohanim} Plural de Cohen. \textbf{Cohanim
Guedolim} Plural de Co­hen Gadol.

\item[\textbf{Cohen}] Descendente da família sacer­dotal.

\item[\textbf{Cohen Gadol}] Sumo sacerdote no Templo Sagrado.

\item[\textbf{Col}] Cada.

\item[\textbf{Cutá}] Mistura de farinha fermentada com leite.

\item[\textbf{Demai}] Capítulo de um dos tratados do Talmud.

\item[\textbf{Derash}] Explicação com comentário. \textbf{Doresh el hametim} Evocando os mortos a fim de descobrir o futuro. \textbf{Ebal} Nome
de um monte sobre o qual metade das tribos recebeu uma parte das leis da
Torah (1? metade). \textbf{Ed shaker} Falso testemunho.

\item[\textbf{Ed shav} Falso testemunho. \textbf{Edumeu (ou Adomi)}] O nome
Edom, que significa vermelho, é atri­buído a Esaú, descendente do
patriar­ca Isaac, e Adomi ou Adomen é a ma­neira como são conhecidos os
seus descendentes.

\item[\textbf{Eduyoth}] Capítulo de um dos trata­dos do Talmud.

\item[\textbf{Efa}] Medida equivalente a 36,44 litros cúbicos.

\item[\textbf{Efod}] Espécie de manto curto usado pelo Sumo Sacerdote • na
época do Templo Sagrado.

\item[\textbf{Eglá Arufá} Bezerra degolada. \textbf{Elazar}] Filho de
Aharon, o Sumo Sa­cerdote.

\item[\textbf{Elohim}] Nome divino.

\item[\textbf{Erub}] Confundir fronteiras (no caso do sábado).

\item[\textbf{Erubin}] Capítulo de um dos tratados do Talmud.

\item[\textbf{Etrog}] Tipo de fruta cítrica que faz parte das 4 espécies
usadas na festa de Sucot.

\item[\textbf{Exilarca (Rosh Galut)}] Descenden­te da casa de Davi que era
representan­te geral e diretamente responsável por toda a coletividade
judaica numa de­terminada cidade ou num determina­do país do exílio
(como a Pérsia e a Ba­bilônia).

\item[\textbf{Guedidá}] Tatuagem.

\item[\textbf{Guemará}] Seria a continuação da Mishná com a qual se forma o
Talmud. \textbf{Guer toshab} Habitante estranho que não é nativo do
próprio local.

\item[\textbf{Guerizim}] Nome de um monte sobre
o qual metade das tribos recebeu uma parte das leis da Torah (2?
metade). \textbf{Guezerá} Sentença decretada. \textbf{Guezerá shavá}
Sentença decreta­da (porém comparada a outra seme­lhante).

\item[\textbf{Guid hanashé}] Tendão encolhido so­bre a junção da coxa.

\item[\textbf{Guitin}] Capítulo de um dos tratados do Talmud.

\item[\textbf{Habdalá}] Separação, cerimônia de despedida do sábado.

\item[\textbf{Haguigá}] Festividade.

\item[\textbf{Halá}] Nome dado ao pão usado no sá­bado e nos dias de festa,
de formato di­ferente do comum.

\item[\textbf{Halachá} Preceito rabínico. \textbf{Halachot Guedolot}]
Grandes regras. Nome de uma obra talmúdica. \textbf{Halalá} Nome que
se dava a uma mu­lher viúva, separada do marido, ou prostituta, que
estavam terminante­mente proibidas de desposar o Sumo Sacerdote ou um
simples Sacerdote, durante a existência do Templo Sa­grado.

\item[\textbf{Halel}] Oração de louvor que se reci­ta especialmente nas
festas e nos pri­meiros dias de cada mês.

\item[\textbf{Halitzá}] Recusa de uma mulher sem filhos que acabou de
enviuvar a unir-se ao irmão solteiro de seu marido. \textbf{Hametz}
Que contém fermento. \textbf{Hanayá} Pouso.

\item[\textbf{Hananiah}] Um dos três sábios que santificaram o nome de Deus
na épo­ca de Nabucodonosor penetrando na caldeira de fogo e saindo
totalmente ilesos.

\item[\textbf{Hanucá}] Festa de luzes que se cele­bra na noite de 25 de
Kislev, comemo­rando a vitória dos macabeus, cujo símbolo é o candelabro
de oito braços. \textbf{Haran} Local em que nasceu o patriar­ca
Abrahão.

\item[\textbf{Hassid}] Caridoso.

\item[\textbf{Helek}] Parte (divisão).

\item[\textbf{Hen}] Graça.

\item[\textbf{Hená}] As mesmas.

\item[\textbf{Heresh}] Surdo.

\item[\textbf{Hezekiel}] Profeta.

\item[\textbf{Hilchot}] Plural de Halachá.

\item[\textbf{Hilchot Rambam}] Preceitos Maimo­nídicos.

\item[\textbf{Hin} Medida para líquidos. \textbf{Hishamer}] Alertar-se.

\item[\textbf{Hober}] Feiticeiro.

\item[\textbf{Hober haber}] Feiticeiro praticante (de grau um pouco mais
alto). \textbf{Horayot} Capítulo de um dos trata­dos do Talmud.

\item[\textbf{Hulin}] Ser profano, ou seja, prome­ter cumprir uma promessa e
não fazê-lo. Também nome de capítulo de um dos tratados do Talmud.

\item[\textbf{Issi (ben Yiehudá)}] Grande comen­tarista talmúdico.

\item[\textbf{Issur bevad ehad}] Proibição de uma só vez.

\item[\textbf{Issur colei}] Proibição geral.

\item[\textbf{Issur mossif}] Proibição a acrescen­tar.

\item[\textbf{Itamar}] Filho de Aharão, o Sumo Sa­cerdote.

\item[\textbf{Iyar} O segundo mês do ano judaico. \textbf{Kadashim}]
Tratado completo do Tal­mud.

\item[\textbf{Kal vahomer} Com toda razão. \textbf{Kalam (os mestres do)}]
Os eruditos. \textbf{Kedoshim} Trecho de leitura sema­nal do 3? livro
do Pentateuco, chama­do Levítico.

\item[\textbf{Kehat}] Filho de Levy (neto do patriar­ca Jacob).

\item[\textbf{Kenaz (Otniel ben Kenaz)}] Grande sábio da época de Josué, em
1272 an­tes da era comum.

\item[\textbf{Kidush}] Santificação. Benção que é pronunciada sobre um copo
de vinho no sábado ou em dia festivo. \textbf{Kidushin} Capítulo de um
dos trata­dos do Talmud.

\item[\textbf{Kinim}] Capítulo de um dos tratados do Talmud.

\item[\textbf{Kiriat Sefer}] Metrópole de livros (no caso, comparando Otniel
ben Kenaz que, de tão sábio, dominava até uma metrópole de livros).

\item[\textbf{Kiyemu ve lo kiyemu}] Cumpriram e não cumpriram.

\item[\textbf{Koptim}] Povo egípcio que costuma­va tatuar o próprio corpo.

\item[\textbf{Korah}] Primo em primeiro grau de nosso mestre Moisés.

\item[\textbf{Kossem}] Pessoa que adivinha o futu­ro por meio de magia.

\item[\textbf{Kossem Kessamim}] Pessoa que pra­tica bruxaria.

\item[\textbf{Lav shebikhlalut} O não total. \textbf{Levi}] Filho de Jacob,
o patriarca. \textbf{Levita} Da tribo de Levi.

\item[\textbf{Ló}] Não.

\item[\textbf{Ló tehonem} Não ter piedade. \textbf{Ló tefaer}] Não adornar.

\item[\textbf{Ló tishacheb}] Não deitará (emprego do verbo no futuro).

\item[\textbf{Ló tishcab}] Não deitará (emprego do verbo no imperativo).

\item[\textbf{Ló titgodedu} Não se tatuarão. \textbf{Log}] Medida líquida
equivalente a 506 cm. ou 0,23 kg.

\item[\textbf{Lulav}] Uma das quatro espécies de palmeira usadas na festa de
Sucot (Festa das Cabanas).

\item[\textbf{Maasser Sheni}] Segundo dízimo (dízi­mo do dízimo) dado no
Templo Sagrado. \textbf{Maasserot} Dízimos.

\item[\textbf{Macat mardut} Bater rebeldemente. \textbf{Macot}] Capítulo de
um dos tratados do Talmud.

\item[\textbf{Madiá}] Aquele que incentiva a prati­car o mal.

\item[\textbf{Makhshirin}] Capítulo de um dos tra­tados do Talmud.

\item[\textbf{Mamzer}] Bastardo.

\item[\textbf{Maneh}] Nome popular de antiga moe­da usada na época do Talmud.
\item[\textbf{Marbit}] Usura.

\item[\textbf{Mashkin}] Capítulo de um dos trata­dos do Talmud.

\item[\textbf{Mashuah Mil-Hama (ou Meshuah Mil-Hama)}] O comandante chefe de
uma batalha, que também tem, a res­ponsabilidade de preparar sua equipe
psicologicamente e de recusar os que não estiverem preparados para a
luta. \textbf{Mashukh} Alguém que teve seu pre­púcio puxado para a
frente a fim de cancelar o sinal do pacto de Abraham. \textbf{Matzah}
Espécie de pão sem fermen­to, usado apenas na Páscoa, em lugar do pão
comum.

\item[\textbf{Matzebá}] Lápide que se coloca sobre o túmulo.

\item[\textbf{Meguilá}] Relato de um acontecimen­to verídico.


\item[\textbf{Meilá}] Desfalque.

\item[\textbf{Mekhashef}] Feiticeiro.

\item[\textbf{Mekhiltá}] Compêndio de regras rabí­nicas relativas ao Êxodo, o
segundo li­vro do Pentateuco (termo aramaico). \textbf{Meliká} - A
maneira como se devia dego­lar a ave, na época do Templo Sagrado.
\item[\textbf{Menahesh}] Adivinho.

\item[\textbf{Menahot}] Capítulo de um dos trata­dos do Talmud.

\item[\textbf{Meonen}] Feiticeiro.

\item[\textbf{Meribá}] Briga.

\item[\textbf{Merkulis}] Deus dos negócios dos ro­manos.

\item[\textbf{Meshichá}] Recibo ou comprovante de uma transação comercial.

\item[\textbf{Messit}] Incitante.

\item[\textbf{Mezuzá}] Prece protetora que se co­loca nos umbrais das portas,
do lado direito.

\item[\textbf{Midot}] Capítulo de um dos tratados do Talmud.

\item[\textbf{Midrashot ou Midrashim (plural de Midrash)}] Palestras,
conferências. Também nome da compilação de co­mentários bíblicos feitos
durante pales­tras dos mestres a seus alunos.

\item[\textbf{Mikvá}] Reservatório de água para ba­nho ritual.

\item[\textbf{Mikvaot}] Plural de Mikvá.

\item[\textbf{Mishael}] Um dos três sábios que san­tificaram o nome de Deus
na época de Nabucodonosor penetrando na caldei­ra de fogo e saindo
totalmente ilesos. \textbf{Mishná} Primeira parte do Talmud.
\item[\textbf{Mishpatim}] Processos.

\item[\textbf{Mishrat anabim}] Líquido derivado da uva (vinho).

\item[\textbf{Moab}] Irmão de Amon, filho de Lot e sobrinho do patriarca
Abraham (vi­de ``Amon'').

\item[\textbf{Moabita} Povo descendente de Moab. \textbf{Moed Catan}]
Pequena festa ou come­moração. '

\item[\textbf{Molekh}] Ídolo do povo chamado Amon que costumava adorá-lo por
meio do fogo, ofertando-lhe os pró­prios filhos.

\item[\textbf{Monte Moriá (Har Hamoriá)}] Local onde o patriarca Isaac foi
levado ao al­tar do sacrifício, pelo próprio pai.

\item[\textbf{Nabot}] Sábio da época de Reis que foi morto pelo rei Ah-Ab que
queria apos­sar-se de sua propriedade.

\item[\textbf{Nassi}] Presidente.

\item[\textbf{Nazir}] Asceta.

\item[\textbf{Nazirim}] Plural de Nazir.

\item[\textbf{Nebelá}] Impureza.

\item[\textbf{Nedarim}] Capítulo de um dos trata­dos do Talmud.

\item[\textbf{Negaim}] Capítulo de um dos tratados do Talmud.

\item[\textbf{Neshekh}] Usura.

\item[\textbf{Nessiim}] Presidentes.

\item[\textbf{Nezikin}] Tratado completo do Tal­mud.

\item[\textbf{Nidá}] Capítulo de um dos tratados do Talmud.

\item[\textbf{Nikebu}] Insultaram.

\item[\textbf{Nissan}] Primeiro mês do ano judai­co.

\item[\textbf{Noahid} Descendente de Noé. \textbf{Nokeb}] Insultar.

\item[\textbf{Notar}] Sobras.

\item[\textbf{Ob}] Feitiçaria na qual se evoca os mor­tos para fazer-lhes
perguntas e saber o futuro.

\item[\textbf{Ohalot}] Capítulo de um dos tratados do Talmud.

\item[\textbf{Okatzin}] Capítulo de um dos tratados do Talmud.

\item[\textbf{Olelot}] Pequenos cachos (de uva) em formação.

\item[\textbf{Omer}] Espécie de medida de cevada nova, recém-colhida, a qual
era ofere­cida no Templo no 2? dia de Páscoa. \textbf{Oná} Período

\item[\textbf{Onen}] Entristecido.

\item[\textbf{Onatá}] Período amoroso.

\item[\textbf{Onkelos} Grande comentarista Bíblico. \textbf{Orlá}] Nome que
se dá ao fruto de uma árvore antes que ela complete 3 anos.
\item[\textbf{Otniel (ben Kenaz)}] Grande sábio da época de Josué em 1272
antes da era comum.

\item[\textbf{Pará}] Capítulo de um dos tratados do Talmud.

\item[\textbf{Parasanga ou Parsá}] Medida métri­ca equivalente a 3.840
metros. \textbf{Patriarca Jacob} O terceiro patriar­ca Jacob.

\item[\textbf{Peá}] Sobras abandonadas nos campos para os pobres.

\item[\textbf{Pen}] Para que.

\item[\textbf{Pen tikdash} Para que te santifiques. \textbf{Pen tukad esh}]
Para que a chama fi­que acesa (no Templo Sagrado). \textbf{Pen yi-yeh}
Para que seja.

\item[\textbf{Peor} Idolo do povo moabita. \textbf{Perutá}] Nome da moeda
israelense anterior à atual.

\item[\textbf{Pessahin}] Capítulo de um dos trata­dos do Talmud.

\item[\textbf{Pessah} Páscoa, em hebraico. \textbf{Pigul}] Nome dado ao
sacrifício que era ofertado sem total intenção, duran­te a época do
Templo.

\item[\textbf{Pinhas}] Neto de Aharão, o Sumo Sa­cerdote.

\item[\textbf{Pitom}] Nome de um antigo feiticeiro egípcio.

\item[\textbf{Portão de Nicanor - Shaar Nikanor}] - Portão do Templo sagrado
oferecido por Nicanor, um dos homens mais ri­cos do Egito no último
século antes da. era comum, razão pela qual tem seu próprio nome.

\item[\textbf{Quelim}] Capítulo de um dos tratados do Talmud.

\item[\textbf{Quemosh} Nome de um ídolo. \textbf{Queretot}] Capítulo de um
dos trata­dos do Talmud.

\item[\textbf{Quessutá} A vestimenta dela. \textbf{Quetubot}] Capítulo de
um dos trata­dos do Talmud.

\item[\textbf{Quil-aim}] A mistura, proibida pela Torah, de duas espécies
distintas, co­mo por exemplo, lã e linho, cavalo e mula etc. Também nome
de capítulo de um dos tratados do Talmud.

\item[\textbf{Quil-ei ha querem}] Mistura do vinhedo.

\item[\textbf{Quil-ei zeraim}] Mistura de planta­ções, proibida pela Torah,
como por exemplo de bananeira com macieira etc.

\item[\textbf{Quipurim}] Capítulo de um dos trata­dos do Talmud.

\item[\textbf{Rabá}] Membro do 4? ciclo dos amo­raitas da Babilônia, de
meados do 4? no século.

\item[\textbf{Raban Shimeon ben Gamliel}] Mem­bro do 1? ciclo dos tanaitas,
do fim do 1? século.

\item[\textbf{Rabi Abin} ou \textbf{Rabi Ilai}] Membro do

3? ciclo dos amoraitas de Jerusalém, do início do 4? século.

\item[\textbf{Rabi Akiba}] Membro do 3? ciclo dos tanaitas, em meados do 2?
século. \textbf{Rabi Dossá} Membro .do 1? ciclo dos tanaitas, no 1?
século.

\item[\textbf{Rabi Eliezer}] Membro do 4? ciclo dos tanaitas, no fim do 2?
século. \textbf{Rabi Eliezer benJacob} Membro do
ciclo dos tanaitas, no 1? século. \textbf{Rabi Hananya ben Akabya}
Mem­bro do 4? ciclo dos tanaitas do fim do
século.

\item[\textbf{Rabi Haniná}] Membro do 1? ciclo dos emoraitas (geração de
eruditos que vieram após os tanaitas), no 3? século. \textbf{Rabi Hisdá} 
Membro do 2? ciclo dos amoraitas da Babilônia, no fim do 3? século.

\item[\textbf{Rabi Hiya ben Abun}] Membro do 3? ciclo dos amoritas de
Jerusalém, no princípio do 4? século.

\item[\textbf{Rabi lanai} Membro do \textbf{1 ?}] ciclo dos\\
amoraitas de Jerusalém, em meados doséculo.

\item[\textbf{Rabi Ilai}] Vide Rabi Abin.

\item[\textbf{Rabi Isaac ben Abdimei}] Grande co­mentarista talmúdico do
início do 5? século.

\item[\textbf{Rabi Ishmael}] Membro do 3? ciclo dos tanaitas, em meados do 2?
século. \textbf{Rabi Meir} Tanaita de grande gabari­to do 2? século. É
muito conhecido, inclusive nos dias de hoje, pelos seus milagres.

\item[\textbf{Rabi Nathan}] Membro do 4? ciclo dos tanaitas, do fim do 2?
século. \textbf{Rabi Shimeon ben Gamliel} Mem­bro do 1? ciclo dos
tanaitas, do fim do 1? século.

\item[\textbf{Rabi Shimeon ben Lakish} ou \textbf{Resh Lakish}] Membro do 2?
ciclo dos amo­raitas de Jerusalém, do fim do 3? sé­culo.

\item[\textbf{Rabi Yehoshuá ben Hananya}] -Membro do 2? ciclo dos tanaitas, do
início do 2? século.

\item[\textbf{Rabi Yehudá}] Compilou a Mishná no fim do 2? século.

\item[\textbf{Rabi Yehudá ben Betera}] Membro do 1? ciclo dos tanaitas, do
fim do 1? século.

\item[\textbf{Rabi Yohanan}] Membro do 2? ciclo dos amoraitas de Jerusalém,
do fim do 3?. século.

\item[\textbf{Rabi Yohana ben Gudgoda}] Mem­bro do 2? ciclo dos tanaitas, do
início do 2? século.

\item[\textbf{Rabi Yossi ben Hanina}] Membro do
ciclo dos emoraitas, do fim do 3? século.

\item[\textbf{Rabi Yossi ben Yehudá}] Membro\\
do 5? ciclo dos tanaitas, do início do
século.

\item[\textbf{Rabi Yossi Hagalili}] Membro do 3 ? ci­clo dos tanaitas, em
meados do 2 ? século. \textbf{Rabi Yoshiá} Membro do 4? ciclo dos
tanaitas, do fim do 2? século.

\item[\textbf{Rabiná} Rabino (do aramaico). \textbf{Rachil}] Caluniador.

\item[\textbf{Rashá}] Ímpio.

\item[\textbf{Rav}] Rabino.

\item[\textbf{Rebiit}] A 4 parte de um cálice de vinho.

\item[\textbf{Resh Lakish}] Vide Rabi Shimeon ben Lakish.

\item[\textbf{Ribit}] Usura.

\item[\textbf{Ribit Ketsutsa} Usura reduzida. \textbf{Rosh Hashaná}] Festa
do ano novo ju­daico.

\item[\textbf{Sanhedrin}] Capítulo de um dos tra­tados do Talmud.

\item[\textbf{Sefer Hamitzvot}] Livro dos precei­tos.

\item[\textbf{Selaim (Plural de ``sela'')}] Espécie de moeda antiga.

\item[\textbf{Seret}] Tatuar-se.

\item[\textbf{Seritá}] Tatuagem.

\item[\textbf{Shaatnez}] A mistura de lã com linho, que é proibida pela
Torah. \textbf{Shabatot} Plural de Shabat. Shabuot - Festa de
Pentecostes ou também feAa do recebimento da To­rah. É também o nome de
um capítu­lo de um dos tratados do Talmud. \textbf{Shebiit} Não
trabalhar a sua terra no 7? ano. Também nome de um capítu­lo de um dos
tratados do Talmud. \textbf{Shebuat bitui} Jurar cumprir e não
cumprir.

\item[\textbf{Shebuat shav} Jurar em vão. \textbf{Shebuat sheker}] - Jurar
pela mentira. \textbf{Shebuot} Capítulo de um dos trata­dos do Talmud.

\item[\textbf{Sheerá}] O sustento.

\item[\textbf{Shehitá}] Ato de abater a ave ou o ani­mal segundo os preceitos
da Torah. \textbf{Shekalim} Capítulo de um dos trata­dos do Talmud.

\item[\textbf{Shekel}] Moeda de prata.

\item[\textbf{Shekhiná}] Divindade.

\item[\textbf{Shemá}] Principal oração da religião judaica.

\item[\textbf{Shemini Atzeret}] Oitavo dia de Su­cot (Festa das Cabanas).

\item[\textbf{Shemoná Sheratsim}] Capítulo de um dos tratados do Talmud.
\item[\textbf{Sheniyot}] Preceitos rabínicos talrnú­dicos.

\item[\textbf{Shitim}] Local onde o povo judeu acampou quando saiu do Egito e
atual fronteira jordaniana.

\item[\textbf{Shoel Adam (Mehaberó)}] Capítulo de um dos tratados do Talmud.
\item[\textbf{Shoel ob}] Consultar-se com um feiti­ceiro.

\item[\textbf{Shofar}] Cometa feita de chifre de car­neiro que costuma ser
tocada no ano novo judaico.

\item[\textbf{Sido}] Tipo de moeda antiga.

\item[\textbf{Sidrá}] Porção semanal do Pentateu­co lida aos sábados.

\item[\textbf{Sidrá tsáv}] Uma das porções sema­nais que começa o nome Tsáv.

\item[\textbf{Sifrá}] Obra antiga que comenta pre­ceitos rabínicos relativos
ao 3? livro do Pentateuco, o Levítico, escrita por Ra­bi Yehudá Ilai Z.
L. no 2? século. \textbf{Sifrei} Estilo exclusivo literário pelo qual
foi transmitida a Torah sagrada ao nosso mestre Moisés.

\item[\textbf{Sotá}] Capítulo de um dos tratados do Talmud.

\item[\textbf{Sucá}] Cabana coberta com ramos. Também nome de um capítulo de um dos
tratados do Talmud.

\item[\textbf{Sucot}] Festa das 'Cabanas.

\item[\textbf{Taalé}] Fará subir.

\item[\textbf{Taaniot}] Parágrafo de um dos trata­dos do Talmud.

\item[\textbf{Taanit}] Capítulo de um dos tratados do Talmud.

\item[\textbf{Taassé}] Fará.

\item[\textbf{Talmud}] Obra composta pela Mish­ná e pela Guemará.

\item[\textbf{Talmud Torah.}] Estudo da Torah.


\item[\textbf{Tamid}] Sempre. É também o nome de um dos tratados da Guemará que fala
sobre a proibição de se plantar árvo­res no Templo para embelezá-lo.

\item[\textbf{Taná (ou Tanaita)}] Palavra que vem do aramaico e que significa
professor. O Talmud emprega esse termo para os doutores da lei que se
empenharam de corpo e alma para que a Mishná fosse escrita e
posteriormente impressa.

\item[\textbf{Taná Kamá}] Palavra aramaica que sig­nifica o primeiro tanaita,
ou seja, o pri­meiro que elaborou a primeira lei de uma determinada
parte da Mishná.

\item[\textbf{Tanaim}] Tanaita ou Taná.

\item[\textbf{Tarbit}] Usura.

\item[\textbf{Targum}] Tradução explicativa, com comentários.

\item[\textbf{Tazria}] Trecho de leitura semanal do 3? livro do Pentateuco, o
Levítico. \textbf{Tebel} Palavra aramaica, que signifi­ca algo
impróprio para ser ingerido. \textbf{Tebul Yom} Banho ritual diário.
\item[\textbf{Tefilin}] Filactérios usados diariamen­te nas preces matinais.

\item[\textbf{Teharot}] Capítulo de um dos tratados do Talmud.

\item[\textbf{Tehorot}] Tratado completo do Tal­mud.

\item[\textbf{Telussin}] Espécie de jóia usada anti­gamente pelos soldados.

\item[\textbf{Templo Monte (Har Habait)}] O lo­cal onde foi construído o
Templo. \textbf{Temurá} Capítulo de um dos tratados do Talmud.

\item[\textbf{Tenahashu}] Praticar a bruxaria. \textbf{Teonenu- A prática de
prever o fu­turo por meio de bruxaria. Terefá} Impróprio para ser
ingerido. \textbf{Terra} Terra de Israel (Erets Israel).
\item[\textbf{Terumá}] Contribuição oferecida pe­lo povo aos sacerdotes do
Templo. \textbf{Terumot} Plural de Terumá. Também o nome de um
capítulo de um tratado do Talmud.

\item[\textbf{Tigzol}] Saquear.

\item[\textbf{Tishri} 7? mês do calendário judeu. \textbf{Torah}]
Pentateuco.

\item[\textbf{Tossafot}] Suplementos de comentá­rios do Talmud que surgiram
bem de­pois dos comentários, daí o nome de suplementos.

Tosseftá - Palavra aramaica que signifi­ca suplementos da Mishná
elaborados pelos tanaitas, de onde o nome Tosseftá. \textbf{Toshab
vesachir} Habitante estranho (não circuncisado) contratado (novo).
\item[\textbf{Tsedaká}] Caridade.

\item[\textbf{Tsitsit}] Espécie de franjas do chale usado nas preces.

\item[\textbf{Tson}] Pequeno gado.

\item[\textbf{Tzav}] Trecho de leitura semanal do 3? livro do Pentateuco, o
Levítico, que tem como título o próprio nome ``Tzav''.

\item[\textbf{Tzion}] Israel.

\item[\textbf{Ushmartem} E vocês observarão. \textbf{Uziah}] Rei de
Jerusalém entre 645 e 707 antes da era comum que penetrou no local
sagrado do Templo, onde era proibido entrar. Como castigo ele se tornou
leproso até o fim da vida. \textbf{Vayehi Bayom Hashemini} Trecho de
leitura semanal do 3? livro do Pen­tateuco, o Levítico, que começa com
essas palavras.

\item[\textbf{Vayikrá}] Trecho de leitura semanal do 3? livro do Pentateuco,
o Levítico. \textbf{Vayigzol} E extorquiu. \textbf{Vayitgodedu}
Tatuaram-se. \textbf{Veshameru} E observarão. \textbf{Yadayim}
Capítulo de um dos trata­dos do Talmud.

\item[\textbf{Yahel}] Profanar.

\item[\textbf{Yain nessech}] Vinho impróprio pa­ra ser consumido em ritual
religioso ju­daico.

\item[\textbf{Yarimu}] Refere-se à separação dos donativos.

\item[\textbf{Yebamot}] Capítulo de um dos trata­dos do Talmud.

\item[\textbf{Yehudá} Uma das doze tribos. \textbf{Yideoni}] Que pratica
bruxaria. \textbf{Yidoa} Este nome é dado a um deter­minado osso
existente nas aves com o qual se praticava a bruxaria.

\item[\textbf{Yom Quipur} Dia do perdão. \textbf{Yom Tob}] Capítulo de um
dos trata­dos do Talmud.

\item[\textbf{Yoma}] Capítulo de um dos tratados do Talmud.

\item[\textbf{Zab}] Pessoa doente que tem propen­são a expelir o próprio
sêmen sem au­to-controle.

\item[\textbf{Zaba}] Feminino de Zab.

\item[\textbf{Zabim}] Capítulo de um dos tratados do Talmud.

\item[\textbf{Zar}] Estranho que não seja descen­dente da família de Aarão, o
Sumo Sa­cerdote.

\item[\textbf{Zebahim}] Capítulo de um dos trata­dos do Talmud.

\item[\textbf{Zeraim}] Tratado completo do Talmud. Zimá - Depravação.

\item[\textbf{Zimri}] Representante chefe da tribo de Shimeon.

\item[\textbf{Zoná}] Prostituta.

\item[\textbf{Zot Tih-yé}] Trecho da leitura sema­nal do 3? livro do
Pentateuco, o Leví­tico, que começa com esse título.
\end{itemize}
\endgroup % fim da fonte pequena
