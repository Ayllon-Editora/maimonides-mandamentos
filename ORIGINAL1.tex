MAIMÔNIDES

Os 613 Mandamentos


\textbf{(TARIAG HA-MITZVOTH)}

MO.SHÉ BEN MAIMON

\textbf{HA RAMBAM}

\begin{quote}
2 2 EDIÇÃO

TRADUÇÃO • BIOGRAFIA
\end{quote}

\textbf{GIUSEPPE NAHAISSI}

NOVA STELLA

\textbf{Dados de Catalogação, na Publicação (CIP) Internacional\\
(Câmara Brasileira do Livro, SP, Brasil)}

\begin{quote}
Maimônides, 1135-1204.

Os 613 Mandamentos : Tariag Ha-Mitzvoth / Moshé ben Maimon ; tradução,
biografia Giuseppe Nahaïssi. --- São Paulo : Nova Stella, 1990.

1. Judaísmo - Doutrinas 2. Maimônides, 1135-1204 3. Seiscentos e treze
Mandamentos I. Título. II. Título: Tariag Ha-Mitzvoth.
\end{quote}

CDD-296.092

-296

90-0763 -296.172

\begin{quote}
\textbf{Índices para catálogo} sistemático:
\end{quote}

\begin{enumerate}
\def\labelenumi{\arabic{enumi}.}
\item
  \begin{quote}
  Judaísmo 296
  \end{quote}
\item
  \begin{quote}
  Mandamentos : 613 : Teologia social judaica 296.172
  \end{quote}
\item
  \begin{quote}
  Teólogos judeus : Biografia e obra 296.092
  \end{quote}
\end{enumerate}

MAIMÔNIDES\\
OS 613 MANDAMENTOS

Título original\\
TARIAG HA-MITZVOTH

Tradução\\
GIUSEPPE NAHAÏSSI

Produção e Capa\\
LUCIANO GUIMARÃES

Revis.o\\
CRISTIANE REGINA BARBIERI\\
CRISTINE BAENA FONTELLES\\
NICOLE WEXLER

Composição, Paginação e Filmes\\
HELVÉTICA EDITORIAL LTDA.

1? edição: 1990\\
3000 exemplares

2 edição: 1990\\
2000 exemplares

Copyright de tradução\\
GIUSEPPE NAHAÏSSI

NOVA STELLA EDITORIAL\\
R. Antonio de Souza Noschese, 289\\
05324 --- São Paulo --- SP\\
Tel.: 268.4214 --- FAX: 268.0987

DO REI DAVID

\emph{"Ajuda-me a trilhar os caminhos de Teus Man­damentos, pois é neles
que eu encontro deleite"}

DA TORÁ

\emph{Neste dia o Eterno, teu Deus, te ordena Que cumpras estes
estatutos e leis. Deverás cumpri-los diligentemente}

\emph{Com todo o teu coração e com toda a tua alma.}

\begin{quote}
\emph{Com relação ao Eterno, confessaste, hoje, que Ele é teu Deus}
\end{quote}

\begin{itemize}
\item
  \emph{que andarás em Seus caminhos}
\item
  \emph{observarás Seus estatutos, preceitos e leis,}
\item
  \emph{que atenderás a Suas determinações.}
\item
  \begin{quote}
  \emph{o Eterno confessou hoje, a teu respeito, que fazes parte de Seu
  povo,}
  \end{quote}
\end{itemize}

\emph{Como Ele te havia prometido,}

\begin{quote}
\emph{Portanto, deves observar todos os Seus preceitos, Para que possas
ser um povo consagrado ao Eterno, teu Deus ...}
\end{quote}

\begin{itemize}
\item
  \begin{quote}
  \emph{agora, Israel, o que o Eterno, teu Deus, pede de ti}
  \end{quote}
\end{itemize}

\emph{A não ser que temas o Eterno, teu Deus,}

\emph{Que andes em todos os caminhos e que O ames}

\begin{itemize}
\item
  \emph{que sirvas o Eterno, teu Deus, com todo o}
\end{itemize}

\begin{quote}
\emph{teu coração e com toda a tua alma;}
\end{quote}

\emph{Que observes, para o teu bem, os Mandamentos}

\begin{quote}
\emph{do Eterno}
\end{quote}

Agradeço

Ao Rabino SHMUEL ZAIONTZ, ROSH YECHIVA, TOMCHEI TMIMIM LUBAVITCH de Nova
York pela revisão e aconselhamento na preparação desta tradução.

Ao Professor ELIE SOU''CAR pela grande ajuda na preparação do Glossário
e seus comentários.

A Sra MARY VANSTREELS pela ajuda e colabora­ção em ordenar esta obra,
sem a qual a mesma se­ria absolutamente impossível.

Dedico estra tradução da obra do Rambam, Rabi Moshé ben Maimon, o Grande
Maimônides, à minha mulher Sarah e aos meus filhos Moshé,
Nathan\textsuperscript{.} e Carmelah na esperança de que continuem a
trilhar o caminho da verdade.

\textbf{ÍNDICE GERAL}

PREFÁCIO 11

MAIMÔNIDES VIDA E OBRA 13 OS 14 FUNDAMENTOS 37 OS PRECEITOS POSITIVOS 75
OS PRECEITOS NEGATIVOS 183 GLOSSÁRIO 349

\begin{quote}
PREFÁCIO
\end{quote}

"De Moisés a Moisés não houve outro igual a Moisés". Esta frase
singu­lar está gravada na pedra tumular de Maimônides, ou Moisés filho
de Maimon, na cidade de Tiberíades não longe das margens do mar da
Galiléia, onde o grande mestre está sepultado. Ela foi escrita pelos
seus discípulos querendo dizer que desde Moi­sés filho de Amram --- o
maior legislador Hebreu, autor por inspiração divina dos dez mandamentos
e da lei, a Torah --- até Moisés filho de Maimon --- o seu maior
intérprete e autor do \emph{Misbneb Torah ---} não houve outro que
pudesse ser compara­do em grandeza e sabedoria ao primeiro Moisés.

Quando Moisés foi incubido por Deus de empreender a gigantesca tare­fa
de resgatar os filhos de Israel da escravidão do Egito e levá-los à
terra prometida, lá em Canaã, muito mais que libertar escravos era
necessário torná-los aptos a guiar seus próprios destinos como
indivíduos e como nação para os séculos a virem. Era necessário uní-los
sob um estatuto que iria reger seus modos de vida, criar costu­mes e
tradições nacionais, dar a eles um conjunto de leis que deveria servir
tanto na guerra como na paz e conter instruções para a conquista, para o
tratamento a ser dispensado aos inimigos, aos cativos de guerra, aos
estrangeiros e peregrinos e aos aliados da paz e da guerra, conter o
estatuto e divisão da terra, o direito civil de paz, a obrigatoriedade
do ensino, as leis do casamento, as uniões lícitas e ilícitas, as leis
do divórcio e da herança, as instruções de higiene e de saúde e o regime
alimentar. Era também necessário afirmar este povo num código moral e
ético apoia­do num único Deus universal, intangível, justo e onipotente;
um código que dis­serta sobre o bem e o mal, a sua relatividade e seus
impactos sobre a vida humana, sobre a virtude e a promiscuidade, a
verdade e a mentira, a pureza e as impurezas. Exige também a luta
constante contra os inimigos da moral estabelecida, personifi­cados
pelos filhos de Amalec que cultuam o caos e, por outro lado, insiste que
os homens se aproveitem corretamente das dádivas divinas, de serem
férteis, de cres­cerem e multiplicarem-se. Neste código são determinados
os deveres e direitos dos cidadãos, dos sacerdotes, dos reis e dos
juízes. O amor ao próximo só é superado pelo amor a Deus, provedor de
todas as coisas.

Esta obra monumental intitulada "Torah" ou "a Lei" é a primeira
cons­tituição escrita e distribuída a um povo para lhe servir de
estatuto e guia. Escrita sobre pergaminho e dividida em 5 capítulos ou
livros (Gênesis, Êxodo, Levíticos, Deuteronômio, Números), contém 613
artigos de lei mais conhecidos como os 613 preceitos ou 613 mandamentos.
Estes preceitos são divididos em duas grandes sec­ções: os preceitos
positivos ou "Farás" e os negativos ou "Não farás". São 248 os preceitos
positivos e 365 os negativos, pois usará as 248 partes que compõem o seu
corpo para fazer os seus deveres para com Deus e seu próximo e se
recusará a fazer o mal os 365 dias do ano. Os dez mandamentos resumem os
613 preceitos.

Esta obra é de tal profundidade que não somente guardou o povo de Israel
por 3500 anos como a mais velha nação do mundo mas influenciou a
huma­nidade com sua essência. Os "dez mandamentos" são aceitos e
respeitados em to­dos os tribunais do planeta como a lei magna. Duas das
maiores religiões se inspirá­ram na moral deste código, o Cristianismo e
o Islamismo, que juntas contam hoje com mais de 2 bilhões de fiéis. Os
cinco livros de Moisés são os cinco primeiros livros da Bíblia,
considerada sagrada e divina tanto por cristãos, muçulmanos como judeus
e base de sua ética e suas orações.

Esta lei também chamada de "Lei de Moisés ou lei Mosaica", aplicada na
disciplina do povo hebreu durante os 40 anos no deserto de Sinai, foi a
lei dos juízes de Israel, dos profetas; teve sua corte instalada durante
o período dos reis

\begin{quote}
PREFÁCIO

"De Moisés a Moisés não houve outro igual a Moisés". Esta frase
singu­lar está gravada na pedra tumular de Maimônides, ou Moisés filho
de Maimon, na cidade de Tiberíades não longe das margens do mar da
Galiléia, onde o grande mestre está sepultado. Ela foi escrita pelos
seus discípulos querendo dizer que desde Moi­sés filho de Amram --- o
maior legislador Hebreu, autor por inspiração divina dos dez mandamentos
e da lei, a Torah --- até Moisés filho de Maimon --- o seu maior
intérprete e autor do \emph{Mishneh Torah ---} não houve outro que
pudesse ser compara­do em grandeza e sabedoria ao primeiro Moisés.

Quando Moisés foi incubido por Deus de empreender a gigantesca tare­fa
de resgatar os filhos de Israel da escravidão do Egito e levá-los à
terra prometida, lá em Canaã, muito mais que libertar escravos era
necessário torná-los aptos a guiar seus próprios destinos como
indivíduos e como nação para os séculos a virem. Era necessário uní-los
sob um estatuto que iria reger seus modos de vida, criar costu­mes e
tradições nacionais, dar a eles um conjunto de leis que deveria servir
tanto na guerra como na paz e conter instruções para a conquista, para o
tratamento a ser dispensado aos inimigos, aos cativos de guerra, aos
estrangeiros e peregrinos e aos aliados da paz e da guerra, conter o
estatuto e divisão da terra, o direito civil de paz, a obrigatoriedade
do ensino, as leis do casamento, as uniões lícitas e ilícitas, as leis
do divórcio e da herança, as instruções de higiene e de saúde e o regime
alimentar. Era também necessário afirmar este povo num código moral e
ético apoia­do num único Deus universal, intangível, justo e onipotente;
um código que dis­serta sobre o bem e o mal, a sua relatividade e seus
impactos sobre a vida humana, sobre a virtude e a promiscuidade, a
verdade e a mentira, a pureza e as impurezas. Exige também a luta
constante contra os inimigos da moral estabelecida, personifi­cados
pelos filhos de Amalec que cultuam o caos e, por outro lado, insiste que
os homens se aproveitem corretamente das dádivas divinas, de serem
férteis, de cres­cerem e multiplicarem-se. Neste código são determinados
os deveres e direitos dos cidadãos, dos sacerdotes, dos reis e dos
juízes. O amor ao próximo só é superado pelo amor a Deus, provedor de
todas as coisas.

Esta obra monumental intitulada "Torah" ou "a Lei" é a primeira
cons­tituição escrita e distribuída a um povo para lhe servir de
estatuto e guia. Escrita sobre pergaminho e dividida em 5 capítulos ou
livros (Gênesis, Êxodo, Levíticos, Deuteronômio, Números), contém 613
artigos de lei mais conhecidos como os 613 preceitos ou 613 mandamentos.
Estes preceitos são divididos em duas grandes sec­ções: os preceitos
positivos ou "Farás" e os negativos ou "Não farás". São 248 os preceitos
positivos e 365 os negativos, pois usará as 248 partes que compõem o seu
corpo para fazer os seus deveres para com Deus e seu próximo e se
recusará a fazer o mal os 365 dias do ano. Os dez mandamentos resumem os
613 preceitos.

Esta obra é de tal profundidade que não somente guardou o povo de Israel
por 3500 anos como a mais velha nação do mundo mas influenciou a
huma­nidade com sua essência. Os "dez mandamentos" são aceitos e
respeitados em to­dos os tribunais do planeta como a lei magna. Duas das
maiores religiões se inspirá­ram na moral deste código, o Cristianismo e
o Islamismo, que juntas contam hoje com mais de 2 bilhões de fiéis. Os
cinco livros de Moisés são os cinco primeiros livros da Bíblia,
considerada sagrada e divina tanto por cristãos, muçulmanos como judeus
e base de sua ética e suas orações.

Esta lei também chamada de "Lei de Moisés ou lei Mosaica", aplicada na
disciplina do povo hebreu durante os 40 anos no deserto de Sinai, foi a
lei dos juízes de Israel, dos profetas; teve sua corte instalada durante
o período dos reis

12 PREFÁCIO

e sua autoridade excedia a do próprio
\href{http://monarca.Com}{{monarca. Com}} a destruição do primeiro
Tem­plo e de Jerusalém houve uma enorme necessidade de se preservar a
lei e seus valo­res e começou a era dos estatutos acadêmicos e a redação
do resto dos livros que . compõem a Bíblia \href{http://hoje.Com}{{hoje.
Com}} a volta a Sion, 70 anos mais tarde, reestabeleceu-se a Suprema
Corte ou Sanhedrim, mais conhecida como "Sinedrio", onde os juízes
decidiam todas as \href{http://questões.Com}{{questões. Com}} a invasão
helenística, judeus e gregos se influen­ciaram mutuamente; a dialética
da filosofia grega tornou-se presente nas academias israelitas e a moral
judaica invadiu os \href{http://gregos.Com}{{gregos. Com}} a dispersão
criada pela invasão romana, era imperativo salvar a lei. Rabbi Yohanan,
filho de Zaccai, vendo desmo­ronar o Templo e a Suprema Corte, conseguiu
escapar do desastre e montou uma academia em Yavne, Israel, formando 72
Mestres ou Rabbis.

O Talmud ou os estatutos da jurisprudência da lei Mosaica foi a coluna
mestre da sustentação da lei no Exílio. As cortes rabínicas espalhadas
pela comuni­dade na Diáspora resolviam as questões entre os judeus e,
desta forma, os rabinos mantiveram a comunidade, o ensino, a moralidade
e a fé.

Em mais de 2500 anos de jurisprudência em todos os estilos, os mestres
da lei Mosaica, os Rabbi, compuseram uma obra monumental que os judeus
con­vencionaram chamar de lei oral, pois ela foi proferida verbalmente
pelos juízes e acadêmicos e, posteriormente, redigida pelos escribas. A
lei oral explica a lei e con­tém os relatos e as opiniões de centenas de
mestres. Maimônides decidiu colocar num só trabalho o sumo da lei oral e
escreveu o \emph{Mishizeh Torah} ou a Torah pela segunda vez. Para
resumir decidiu escrever este trabalho em 2 volumes com curtas
explicações sobre cada preceito e referências de onde encontrar mais
sobre o as­sunto na literatura talmúdica.

Esta é a tradução do livro dos preceitos de Maimônides, intitulado em
hebraico \emph{Sefer Ha-Mitzvoth.} Recomendamos a sua leitura com muita
atenção e me­ditação e, apesar de muitos dos preceitos não caberem mais
nos dias de hoje, pelo menos de maneira literal --- como os serviços dos
sacerdotes no templo, ou dos cas­tigos a serem aplicados ---, no bojo de
cada preceito há' uma lição de vida como; no preceito de como retirar
cinzas do santuário em que está embutido o respeito ao passado: a
juventude foi o fogo de ontem e a velhice tem que ser respeitada,
cui­dada e levada em segurança. Quanto aos castigos por açoitamento, o
exílio ou até a pena de morte, hoje nos servem para dimensionar a
gravidade do crime cometido.

A maneira mais prática de estudar a lei Mosaica é começando pelo livro
dos preceitos de Maimônides e isso foi exatamente o que eu fiz quando,
no ano de 1985, festejava-se os 850 anos do seu nascimento e os Rabinos
do mundo inteiro e especialmente o Rabbi Menahem Meyer Schneersohn,
Rabino chefe do movimento "chabad", recomendaram o estudo da obra do
grande Mestre.

No estudo dos textos em hebraico, tendo ao lado a tradução em inglês, eu
tomava notas em português para minha memória. As notas se acumularam e
al­guns amigos sabendo do meu trabalho me recomendaram sua publicação.
Refinei o texto e parti para a tradução. literal. A fidelidade ao
espírito do texto foi funda­mental e muitas vezes achei conveniente
sacrificar o vernáculo a favor do assunto e com a ajuda de Deus e de
todos os que colaboraram comigo nesta tarefa apresen­tamos este
trabalho.
\end{quote}

\emph{Giuseppe Nahaïssi}

\begin{quote}
\emph{29 de ADAR 5750\\
26 de março de 1990\\
São Paulo --- Brasil}

\textbf{MAIMÔNIDES}

\textbf{VIDA E OBRA}

No ano de 1166, aos 31 anos, desembarca em Alexandria, no Egito, Moisés,
filho de Rabi Maimon, o homem que viria a ser conhecido como o Moi­sés
do Egito e respeitado pelo mundo todo como uma das mais relevantes
figu­ras do pensamento judeu.

"De Moisés a Moisés não houve outro igual a Moisés": é assim que os
estudiosos costumam se referir a esse grande sábio cujo legado foi
decisivo para a manutenção da fé e da união do povo judeu no século XII.
Sua glória se extendeu aos círculos não judeus, e nos meios cultos de
Bagdad ele passou a ser considerado como um dos mais eminentes homens da
época. Maimôni­des foi o responsável, entre outros feitos, pela
subordinação do valor moral ao valor teórico, e pela análise
contemplativa abstrata como objetivo final, ao invés do julgamento
concreto dos atos, se bem que a introdução da inteligên­cia no espírito
religioso já houvesse sido feita na época tanaica e que o valor
religioso da compreensão talmúdica já fosse conhecido pelo povo há muito
tem­po. A superioridade da contemplação sobre o rito e a moral constitui
o pilar central de seu pensamento e embora o Talmud ensine que não são
as pesquisas e sim o fato o que importa, ele insiste nas pesquisas
porque tem a profunda convicção de que o amor de Deus é tanto maior
quanto mais desenvolvida e aperfeiçoada for nossa inteligência.

Talmudista, codificador da Torah, filósofo, místico, matemático, mé­dico
e dono de um talento literário ímpar, ele iria transformar a comunidade
judaica do Egito, trazer um nova ordem para os judeus do mundo e viria a
ser o único pensador da Idade Média cujas teorias exerceram influência
significati­va sobre os pensadores tanto cristãos quanto muçulmanos e
judeus de sua épo­ca. Sua obra foi, aliás, freqüentemente citada por
filósofos como Tomás de Aqui-no, Alberto, o Grande, Roger Bacon, Inácio
de Loyola, Alexandre de Halle, Ní­colas de Coves, Leibniz Barouch de
Espinoza e muitos outros.

Homem de personalidade densa e complexa, Maimônides estabele­ceu para si
mesmo uma conduta estrita e complicada, mas soube simplificar o que
desejava transmitir de forma tal que seus leitores pudessem
compreendê-lo facilmente. Fanático pela brevidade, Maimônides se
preocupa sempre em cons-

14 MAIMÔNIDES

truir parágrafos claros, sem nenhum interesse em engrandecer seus
pensamen­tos nem glorificá-los com uma retórica exagerada. São suas
estas palavras: "Se me fosse possível resumir o Talmud inteiro numa
frase, eu não quereria fazê-lo em duas". Enquanto algumas de suas obras
são muito eruditas, outras são escri­tas de maneira muito fácil e são de
compreensão extremamente simples. Quan­do interpelado sobre o por quê
dessa diferença entre uma obra e outra, ele res­pondeu: "O pão e o leite
são para as crianças, e a carne e o vinho são para os adultos". Fiel a
essa filosofia, Maimônides conduz seu aluno, fazendo-o crescer em suas
mãos, e levando-o a passar por vários estágios de "pão e leite"
primeiro, para que ele possa chegar a compreender e a apreciar "a carne
e o vinho" da metafísica, a ciência superior que lhe abriu os caminhos
na sua busca da verdade.

Ele acredita que todos os homens devotos, sem exceção, que vivam de
acordo com a virtude e que sigam os mandamentos bíblicos e mantenham
sempre boa conduta, serão recompensados com o mundo futuro,
independen­temente de seu credo ou religião. Respeita e tem íntimos
amigos no mundo islâmico, e costuma afirmar que a doutrina cristã não
tem nenhuma contradi­ção com o judaísmo, pois ela também reconhece a
força e a necessidade dos mandamentos e da moral bíblica, e que seus
adeptos, se quiserem aprofundar-se no estudo contemplativo dos textos,
descobrirão a verdade.

Além de ter sido considerado o maior talmudista do século, esse ho­mem,
que se torna o médico da corte do Egito, servindo ao grão-vizir de
Sala­din, Al Fadil, e depois o sultão Al Afdal, gozava da :eputação de
ser o melhor médico de seu tempo. Os relatos sobre seu saber e sua
competência se esten­dem de tal forma que chegam ao conhecimento do rei
Ricardo Coração de Leão, da Inglaterra, e este o convida para ser seu
médico particular. Maimônides, no entanto, prefere permanecer no Egito,
pois lá ele acumula também o cargo de Naguid, e pode utilizar-se de sua
digna posição para proteger a comunidade ju­daica através do mundo
islâmico. O Naguid era o líder e o porta-voz dos judeus egípcios,
nomeado pelo sultão, e que representava a autoridade moral e políti­ca
de todas as comunidades judias no país dos Fatimitas. Ele era escolhido
den­tre a comunidade rabínica, mas tinha também direito de justiça sobre
os caraí­tas e os samaritanos.

Mas seu brilhantismo e seu sucesso não são fortuitos. Na sua juven­tude,
Maimônides aprende astronomia com o filho do célebre astrônomo Ibn
Aphla, de Sevilha, estuda o Almagesto, o tratado astronômico de
Ptolomeu, as proposições de Algebra, o tratado das secções cônicas, a
geometria, a mecâni­ca, a medicina, tratados astrológicos, bem como
livros teológicos de outras re­ligiões, para adquirir um conhecimento
geral das religiões de seu tempo. Apro­funda-se ainda nas doutrinas
filosóficas de Aristóteles, de Filo, de Afrodisias, de Themistius, de
Alfarabi, de Gazali, do Gaon Saadia, de Bachija, de Rabi Ye­huda Halevi,
de Rabi Abraham bar Chiha e de Rabi Abraham Ibn Esra, mas ba­seia suas
explicações metafísicas mais profundas no pensamento aristotélico.

Já aos 16 anos Maimônides escreve uma introdução à lógica, e aos 23 uma
dissertação matemática e astronômica, tratando das questões principais
da determinação do calendário judaico. Pelos lugares por onde passa
durante o êxodo em que vive durante 20 anos, ele estuda atentamente a
flora dos países e se interessa por suas arquiteturas. No Egito ele
estuda os usos e as particulari­dades da língua, os hábitos e a moral
dos judeus egípcios, e chega a redigir um comentário sobre essas
observações.

O RAMBAM, sigla de Rabi Moisés ben Maimon, ou simplesmente Mai­mônides,
do grego "filho de Maimon", nasceu em Córdoba, na Espanha, em 1135,
filho do Dayan ou Juiz Rabínico Rabi Maimon, descendente de uma lon-

MAIMÔNIDES --- VIDA E OBRA\_ 15

ga linhagem de Dayanim ou juízes, remontando a Rabi Yehuda Hanassi, o
autor da Mishná, sábio que havia atingido a perfeição moral e
intèlectual, e que, por sua vez, era descendente direto da casa real de
Davi.

Tendo ficado sem a mãe ao nascer, Maimônides se revela uma crian­ça que
desde cedo se habitua a entregar-se a meditações profundas sobre a vida
e a morte e a confiar-se sozinho a Deus. Para isso, ele se refugia na
sinagoga durante a semana, na parte reservada às mulheres, para meditar,
onde tem cer­teza de que ninguém virá interrompê-lo.

No ano de 1148, quando o jovem Moisés completa 13 anos, os Al­mohades,
liderados por Abd-el-Mumin, invadem a cidade de Córdoba. Esses
Almohades, ou "confessores da unidade", eram uma tribo berbere que
con­quistara o poder na Espánha e no Marrocos, após 20 anos de lutas
sangrentas. Abd-el-Mumin era o sucessor de Ibn Toumert, jovem e ardoroso
muçulmano que vivia no sudoeste do que atualmente é o Marrocos e que,
insatisfeito com os ensinamentos teológicos básicos que ali lhe haviam
sido ministrados, deci­diu aprofundar-se no assunto, indo para isso às
faculdades de Córdoba, de Me­ca e de Bagdad, onde entrou em contato com
os ensinamentos de Gazali. De­pois de ter aprendido a ciência teológica
oriental, Ibn Toumert voltou para sua região natal e; declarando-se
descendente de Maomé, liderou uma guerra santa contra as altas esferas
do governo as quais, segundo ele, eram as responsáveis --- entre outras
tantas coisas inadmissíveis --- pelo relaxamento religioso, pelo luxo e
pela decadência moral da corte e da alta sociedade, pela representação
material de Deus --- o que era uma blasfêmia ---, e pelo "politeísmo",
que ele atribuía aos antigos fiéis da África do Norte, os quais
afirmavam, tal como os cristãos, a pluralidade do Ser divino. A
revolução teológica, aliada ao desejo de conquista, levou a um sucesso
sem precedentes, e o reino dos Almohades se estendeu da Síria ao oceano
Atlântico. Eles destruíam as igrejas e as sinago­gas, e aos povos que
não aceitavam converter-se a "verdadeira religião islâmi­ca", propagada
por eles, restava a opção entre a imigração ou a morte.

O jugo dos Almohades se fazia sentir na mesma época em que as Cru­zadas
partiam da França e da Alemanha, a fim de conquistar a Terra Santa e
apossar-se do túmulo de Cristo. Intolerantes, os Cruzados arrasavam, na
sua mar­cha para Jerusalém, tudo o que encontravam de não cristão,
massacrando em seu caminho as populações judias indefesas. Mais uma vez
a história se repetia e os judeus se defrontavam com uma nova e grave
crise de identidade: dobrar-se aos conquistadores islâmicos ou à
barbárie dos cruzados em marcha:\textsuperscript{.} O fana­tismo
religioso imperava tanto no levante quanto no ocidente e continuar
pro­fessando o credo judaico representava um risco de vida. Talvez esse
tenha sido um dos momentos mais difíceis e trágicos da história da
sobrevivência do ju­daísmo; eram necessárias grandes forças para
sustentar a fé, e um dos persona­gens mais importantes dessa época foi
Maimônides, pois a clareza de suas idéias e de seus escritos mantiveram
acesas no povo judeu as chamas da crença e da liberdade da ciência e do
conhecimento; para enfrentar esse período singular e sinistro.

Preferindo o êxodo ao massacre ou à renúncia de sua fé, a família do
Rabi Maimon sai de Córdoba quando os Almohades lá chegam. Depois de 10
anos de vida errante, nos quais passam por diversas cidades do sul da
Espa­nha, eles chegam a Fez, capital do Marrocos --- a África do Norte
sempre fora o local de asilo para os judeus que fugiam das perseguições
religiosas na Espa­nha. Munido de coragem ímpar, Rabi Maimon opta por
Fez, onde os "confes­sores da unidade" haviam instalado sua corte,
porque tem a esperança de ser introduzido ao líder deles, o califa
Abd-el-Mumin, homem que gozava da repu-

16 MAIMÔNIDES

tação de interessar-se pelas coisas do espírito e que procurava
cercar-se de sá­bios, a fim de expor-lhe o pensamento judeu relativo a
Deus e tentar assim ob­ter uma mudança na política do governo em relação
aos judeus.

Muitos judeus, no entanto, para escapar a morte ou ao abandono do lar,
optavam pela conversão aparente a doutrina dos "confessores da
unida­de." Essa conversão, que os obrigava a uma vida dupla vergonhosa e
sem dig­nidade, era algo suportável apenas enquanto eles contassem com a
providência divina e enquanto essa situação de sofrimentos tivesse algum
sentido compreen­sível. Contudo, esse conflito se tornava um sofrimento
intolerável quando o sustentáculo moral da fé começava a desmoronar,
abalando sua confiança em Deus e em si próprios. A crença na unidade
absoluta de Deus, apregoada pelos Almohades, parecia ao povo mais
simples, idêntica à doutrina judaica e eles co­meçavam a acreditar que a
missão do povo eleito tinha chegado a seu fim e a se perguntar se o
profeta Maomé não era realmente superior a Moisés.

Extremamente preocupado com essa situação, Rabi Maimon decide escrever,
em 1159, uma carta em árabe que ele envia às comunidades judias da
África do Norte, recordando-lhes a infalibilidade divina, a existência
de uma aliança permanente entre Deus e Israel, a superioridade de Moisés
e o profun­do significado da prece. Nessa carta ele diz: "Um rei, ao
demitir um de seus funcionários, tem o hábito de nomear imediatamente um
outro, a fim de transmitir-lhe o cargo e as funções do primeiro. Um
marido que repudia sua mulher, geralmente coloca uma outra em seu lugar,
e lhe dá os adornos e a ca­ma da primeira. O sinal da mudança consiste
em dar ao sucessor os direitos e as honras do predecessor. Onde está,
afinal, o povo a quem o Eterno apare­ceu, ao qual Ele deu uma Torah e
sobre o qual Ele espalhou sinais de Sua bene­volência, semelhantes
aqueles com os quais Ele favoreceu os judeus?" Rabi Mai­mon refere-se
aqui às maravilhas do Êxodo, quando Deus libertou o povo de Israel da
opressão faraônica, às 10 pragas impostas ao inimigo, à abertura do mar
para a passagem do povo eleito, ao milagre do maná para sua alimentação
diária durante 40 anos, à presença de Suas colunas de fogo para guiá-los
à terra prometida e à entrega de Seus mandamentos a viva voz, desde o
cume do Si­nai, não através de um emissário, nem de um intermediário,
nem sequer de um anjo, mas através d'Ele próprio, em Sua glória. A carta
segue assim: "Enquanto nenhum outro povo puder mostrar sinais de
clemência e de benevolência simi­lares, só se pode considerar como
falatório o abandono de Israel em favor de um outro povo. Ainda que
vivamos incessantemente na angústia, ainda que pe­la manhã desejemos a
chegada da noite, e a noite a chegada da manhã, ainda assim devemos
pensar na seguinte profecia: 'Deus não esquecerá a aliança que ele fez
com teus pais' ".

... Deus não quer destruir, mas purificar Israel. Devemos conside­rar
nossa aflição atual como um ensinamento, como uma prova. Como acredi­tar
na ira do Eterno, no repúdio de Israel? A missão de Moisés, de nosso
incom­parável mestre, prova a eleição de Israel. ... O sucesso material
não prova o valor de uma nação. A preferência de Deus por Moisés e por
Israel, preferência confirmada em várias ocasiões pela benevolência
divina, garante a efetivação das promessas do Senhor, mas não se pode
saber quando ocorrerá sua realiza­ção, trazidas pelo arrependimento e
pela oração. ..."

Seguindo o exemplo de seu pai, aos 24 anos Maimônides decide sair da
vida de estudo e de trabalho solitários que levava até então para
redigir um tratado sobre um julgamento feito por um rabino que condenara
como traido­res do judaísmo os judeus que se convertiam em aparência à
doutrina dos Al­mohades. Maimônides considera que professar a fé
islâmica para continuar vi-

vo não é apostasia, baseado no fato de que outros judeus haviam tido
atitudes similares anteriormente, sem que por isso tenham provocado a
ira do Senhor, e que o mais importante é a sobrevivência do povo de
Israel. Nessa sua primei­ra obra, publicada em Fez, ele diz: "Se as
colunas do mundo, Moisés, Elias, Isaías, e até mesmo um anjo, foram
punidos porque ousaram elevar a voz contra Is­rael, então quanto não
deve ser censurado um homem suficientemente auda­cioso para dizer que
nas comunidades judias há malfeitores, pagãos, homens indignos de
prestar testemunho a Deus, ateus! ... Então esse rabino estrangeiro e de
pouca reflexão não sabia que os que se convertem pela força não pecam
por negligência? ... O Senhor não os abandonará; Ele não os rejeitará:
Ele nun­ca menosprezou a miséria dos infelizes."

Orientando-se pelo provérbio "Não se consegue nada sem penar", o
objetivo a que Maimônides se fixa é o de "compreender" Deus, até onde
isso for possível ao homem. Para tanto ele julga que deve iniciar pela
lógica, segui­da das ciências matemáticas, das naturais, e por fim da
metafísica, numa pro­gressão do concreto para o abstrato. Assim, ele se
dedica ao estudo de várias ciências para exercitar seu espírito e suas
capacidades intelectuais, a fim de dis­cernir a lógica demonstrativa dos
outros métodos de raciocínio. Ele se consa­gra com zelo ao estudo das
ciências gerais, mas apenas como elementos neces­sários à aquisição de
uma cultura global, e não por uma necessidade interna, pois esta ele
satisfaz através do estudo da Torah.

Já na sua adolescência, Maimônides procura compreender e aprofun­dar-se
nos mistérios proféticos e suas reflexões a esse respeito formam o ponto
culminante de toda sua vida intelectual. Ele tem a convicção de que a
sabedo­ria, a integridade e a modéstia são os atributos que preparam o
espírito do ho­mem para o advento da profecia. Acredita também que, por
mais profundo que possa parecer o saber acumulado por um homem, ele deve
colocar tudo nas mãos do Todo Poderoso, pois o conhecimento é um dom de
Deus. No entan­to, sua tendência especulativa o leva a buscar
incessantemente o sentido da exis­tência individual, já que a crença na
necessidade do pensamento é a idéia con­dutora de sua vida. Para ele o
pensamento é sagrado, e ele só consegue aceitar a crença através da
inteligência e do entendimento, afirmando que a inteligên­cia filosófica
é uma condição "sine qua non" para a imortalidade da alma e pa­ra a
participação no reino eterno.
\end{quote}

A solução, a resposta, não é o essencial para Maimônides. A discipli-\\
na e a dedicação são as qualidades fundamentais de sua inteligência, daí
ele rea-\\
firmar sempre que não deseja construir um sistema filosófico e sim
apenas faci-\\
litar o caminho para alcançar o conhecimento de Deus. Ele considera
como\\
fraqueza de espírito acomodar-se na crença tradicional toda vez que a
lógica\\
se inclina diante da religião, como por exemplo no caso das posições
dogmáti-\\
cas, e diz, a esse respeito, o seguinte: "... se alguma coisa não tem um
motivo\\
compreensível e se ela não traz nenhum benefício nem evita nenhum mal,
por\\
que diríamos daquele de quem ela é o objeto de crença ou a regra de
conduta,\\
que ele é sábio e inteligente, e que ele ocupa uma posição elevada? Que
haveria\\
de surpreendente para os povos nisso? ... Diríamos que ... o homem é
mais per-\\
feito que seu criador, pois o homem falaria e agiria visando um
determinado\\
objetivo, enquanto que Deus, ao invés de agir dessa forma, nos ordenaria
...\\
a fazer o que não tem nenhuma utilidade para nós e nos proibiria ações
que\\
não nos trazem nenhum prejuízo." Ele conhece os limites da razão, mas
consi-\\
dera como imperativo viver sob o império dela, pois para ele a
inteligência não\\
é um local para descarregar suas dúvidas, mas já faz parte do reino de
Deus.\\
Em 1158, durante sua fuga através da Espanha, ele inicia a redação

\begin{quote}
18 MAIMÔNIDES

de seu \emph{Comentário sobre a Mishná,} obra que leva 7 anos para ser
concluída e na qual, à guisa de prefácio ao décimo capítulo do tratado
\emph{"Sanhedrin" ,} ele faz uma exposição da tradição e da doutrina do
judaísmo. Redigida para pro­porcionar uma resposta às dificuldades e às
necessidades do povo judeu na épo­ca, e com o intuito de preservar a
unidade de seu povo, que ameaçava desmo­ronar diante de tantas provações
e de tantos conflitos, essa introdução levou Maimônides a sacrificar
seus princípios liberais e a propor um quadro quase dogmático que
representa, sob seu próprio ponto de vista, o verdadeiro credo do
judaísmo, e que pode ser resumido da seguinte forma:
\end{quote}

\begin{enumerate}
\def\labelenumi{\arabic{enumi}.}
\item
  \begin{quote}
  Eu acredito plenamente que o Criador, que o Seu nome seja bendito, é o
  Criador e Guia de todos os seres, que Ele e apenas Ele, criou, cria e
  criará todas as coisas.
  \end{quote}
\item
  \begin{quote}
  Eu acredito plenamente que o Criador, que o Seu nome seja bendito, é
  um e único e que não existe nada mais único do que Ele; que apenas Ele
  é nos­so Deus, era, é e será.
  \end{quote}
\item
  \begin{quote}
  Eu acredito plenamente que o Criador, que o Seu nome seja bendito, é
  eté­reo; que Ele não tem nenhuma propriedade antropomórfica; que nada
  é parecido com Ele.
  \end{quote}
\item
  \begin{quote}
  Eu acredito plenamente que o Criador, que o Seu nome seja bendito, é
  pri­meiro e último.
  \end{quote}
\item
  \begin{quote}
  Eu acredito plenamente que o Criador, que o Seu nome seja bendito, é o
  único a quem é apropriado rezarmos e que não é apropriado rezar a mais
  ninguém.
  \end{quote}
\item
  \begin{quote}
  Eu acredito plenamente que todas as palavras dos profetas são
  verdadeiras.
  \end{quote}
\item
  \begin{quote}
  Eu acredito plenamente que a profecia de Moisés, nosso mestre, que
  esteja em paz, foi verdadeira, que foi ele o pai de todos os profetas,
  daqueles que o precederam como daqueles que o seguiram (no sentido de
  ter sido o maior deles).
  \end{quote}
\item
  \begin{quote}
  Eu acredito plenamente que a totalidade da Torah que está em nossas
  mãos foi dada a Moisés, nosso mestre, que descanse em paz.
  \end{quote}
\item
  \begin{quote}
  Eu acredito plenamente que esta Torah não será modificada e que não
  ha­verá outra Torah dada pelo Criador, bendito seja Seu nome.
  \end{quote}
\item
  \begin{quote}
  Eu acredito plenamente que o Criador, bendito seja Seu nome, conhece
  todas as ações e todos os pensamentos de todos os seres humanos, como
  está escrito: \textsuperscript{-}É Ele que amolda o coração de todos,
  Ele que capta todas as suas ações" (Salmos 33:15).
  \end{quote}
\item
  \begin{quote}
  Eu acredito plenamente que o Criador, bendito seja o Seu nome,
  recom­pensa aqueles que observam Seus mandamentos, e pune aqueles que
  os transgridem.
  \end{quote}
\item
  \begin{quote}
  Eu acredito plenamente na vinda do Messias, ainda que possa tardar, no
  entanto espero a cada dia pela sua vinda.
  \end{quote}
\item
  \begin{quote}
  Eu acredito plenamente que haverá ressurreição dos mortos no momento
  em que assim o desejar nosso Criador, bendito seja Seu nome, exaltada
  seja a Sua recordação para todo o sempre.
  \end{quote}
\end{enumerate}

\begin{quote}
Incorporados depois à liturgia de várias populações judias, esses
prin­cípios foram recebidos com grande alegria pelas comunidades
carentes, trans­formando-se em hinos de glorificação a Deus. O mais
famoso deles é o Yigdal, de autor desconhecido, com força poética sem
igual e totalmente baseado nas palavras do grande mestre. Este hino é
cantado até os dias de hoje em todas as sinagogas.

Para elaborar seu \emph{Comentário sobre a Mishná,} Maimônides se
inspi­ra, tanto no pensamento como na forma, na própria Mishná, redigida
por

\textbf{MAIMÔNIDES --- VIDA E OBRA 19}

seu antepassado, o rabino Yehuda Hanassi. A Mishná, escrita no início do
sécu­lo III, é a obra que condensa as explicações e os resultados dos
estudos e das pesquisas intelectuais que haviam sido feitos até então em
torno da Sagrada Es­critura. Esse trabalho separa o conteúdo da doutrina
daquilo que está direta­mente ligado ao texto da Bíblia, tal como havia
sido transmitido pela tradição, e o reduz a regras e a decisões.

De acordo com seu autor, o \emph{Comentário} deveria trazer novidades e
melhorias aos \href{http://estudos.Com}{{estudos. Com}} o passar do
tempo a Guemará, que é um comen­tário elaborado sobre a Mishná, havia
suplantado o estudo desta; grandes sá­bios e profundos conhecedores da
Guemará ignoravam a Mishná, e Maimôni­des os recriminava por isso. Os
objetivos de Maimônides ao redigir o seu \emph{Co­mentário} eram,
portanto o de restabelecer a Mishná na sua posição preponde­rante e o de
fazer um resumo dos debates qué estão nessa obra, de modo a ter-se uma
referência rápida e fácil sobre todas as questões da Lei e de modo a
servir aos debutantes como uma preparação para a dialética superior.

Publicado em 1168, o \emph{Comentário} foi concebido da seguinte for­ma:
as introduções sistemáticas, escritas livremente, se distinguem
totalmente das explicações curtas dos textos da Mishná. A amplidão e a
profundidade da sabedoria do autor aparecem aí de maneira mais forte e
mais clara do que nas partes explicativas, necessariamente limitadas
pelo próprio texto da Mishná. Mai­mônides coloca, nesse seu trabalho, a
seguinte advertência: "Leia várias vezes meu livro e reflita
atentamente. Se sua imaginação lhe fizer crer após a primeira leitura ou
mesmo após a décima que você o compreendeu, então ela o enga­nou. Pois
você não deve fazer a leitura deste livro de maneira rápida: eu não o
escrevi simplesmente, como é o caso às vezes; ele é o fruto de muitas
pesqui­sas e reflexões".

A publicação dessa obra, no entanto, não provoca aparentemente nenhuma
polêmica; Maimônides não possuía as condições habitualmente ne­cessárias
para ser reconhecido como uma autoridade: passar pela escola, tornar-se
professor ou Gaon, ou fazer parte dos trabalhos efetuados numa academia
re­presentativa. Mas ele não queria dever nada a uma posição nem a uma
dignidade.

Com a morte de Abd-el-Mumin em 1163, o qual havia sido de certa maneira
tolerante com a comunidade judia de Fez, seus sucessores retomam as
perseguições cruentas aos judeus, e a família de Maimônides emigra então
para Ceuta, cidade situada à beira-mar, na extremidade norte de
Marrocos, e que naquela época ocupava um lugar preponderante no mundo
das artes e das ciências. Mas os Almohades disputavam acirradamente o
governo de Ceuta e em meio a tumultuados golpes e contra-golpes
políticos, em 1165, Rabi Mai­mon decide partir novamente, desta vez com
destino à Terra Santa.

A Terra Santa era na época o ponto mais cobiçado do ocidente nas.
disputas religiosas. As Cruzadas haviam conquistado o país e quando a
família de Maimônides chega lá o governador é o franco Amaury, homem
ambicioso e ávido de poder e. de riquezas. Contudo Nureddin, governador
muçulmano da região do Tigre, decide partir para a Guerra Santa contra
os cristãos das cru­zadas e faz uma aliança com o Egito e a Síria para
cercar a Terra Santa. É nesse momento que Maimônides se dirige para lá.

Maimônides desembarca na florescente cidade de São João de Acre em 16 de
maio de 1165, em cujas ruas se ouviam todos os idiomas do oriente e do
ocidente, e é asilado pelo Rabi Jafet, que presidia a vida das 200
famílias judias da cidade

Na Terra Santa as antigas tradições judaicas se haviam perpetuado, pois
tinham sido transmitidas de maneira ininterrupta dentro do país. Maimô-

20 MAIMÔNIDES

nides descobre ali que a ordem das quatro partes do Pentateuco nos
Teffilin, tal como ele a havia aprendido em Córdoba, difere da ordem
estabelecida de acordo com a opinião de conhecidos Gaonin, que eram os
chefes das grandes academias da Babilônia, e de acordo com antigos
textos do Talmud, e decide então corrigir a ordem de seus Teffilin. Esse
fato é um acontecimento impor­tante, sobretudo por tratar-se ele de um
homem que dedica sua existência à pesquisa, à explicação e à
representação da lei judaica, pois é uma demonstra­ção de grande
humildade e de aceitação do conhecimento daqueles sábios no que se
refere à tradição.

Em outubro de 1165 Maimônides se dirige à Jerusalém para rezar dian­te
do Muro das Lamentações, e de lá ele vai para Hebron, a fim de rezar
sobre o túmulo dos patriarcas.

Mas não tardaria muito para que a depravação que ele observava nos
imigrantes estrangeiros que chegavam à Terra Santa o convencesse a
partir de lá, já que ele acreditava que "É inato no homem curvar-se, com
relação aos seus hábitos e atos, aos costumes dos países e dos amigos ou
companheiros que ele encontra ali". Todo homem, diz ele, deve procurar
assimilar os hábitos e a con­duta dos sábios. Para isso, é preciso que
ele faça tudo que estiver ao seu alcan­ce para viver junto aos justos e
afastar-se dos maus, a fim de não correr o risco de se integrar a eles e
de se sentir inclinado a agir como eles: "Se acontece de vivermos num
lugar onde os habitantes não seguem o caminho correto, é pre­ciso
imigrar para um lugar onde os habitantes sejam devotos e tenham bons
costumes."

Diante disso, ele parte com direção ao Egito onde, ao contrário do que
ocorria nos outros países muçulmanos, os judeus podiam contar com a
to­lerância religiosa dos califas Fatimitas. Dentre os 50.000 habitantes
do país na­quela época havia 3.000 famílias judias que viviam em paz e
gozavam de com­pleta liberdade civil e religiosa. Chegando em Alexandria
em 1166, a família de Maimônides se depara com uma cidade internacional,
que embora não mais fosse a capital do Egito nem a segunda cidade do
mundo, continuava sendo grande e bela, "cidade do comércio de todos os
povos", como descrevia o co­merciante Benjamim de Tudele, para onde se
dirigiam comerciantes tanto da Europa cristã, como do sul da Arábia, da
África do Norte e das Índias, e onde cada nação possuía seu próprio
armazém.

Mas Rabi Maimon pouco desfruta dessa merecida paz, depois de tan­tos
anos de peregrinação, pois vem a falecer poucos meses depois de terem
chegado ao Egito. Davi, seu filho mais moço, assume então a manutenção
ma­terial da família, dedicando-se ao comércio de pedras preciosas, e
liberando as­sim Maimônides dessa preocupação para que ele pudesse
continuar seus estudos.

Se as comunidades judias da Espanha e do Marrocos estavam amea­çadas de
extinção pela fé implacável dos Almohades, e as da Terra Santa pelas
cruzadas e pelos maus costumes, o conforto material e a comodidade em
que viviam os judeus do Egito também representavam uma ameaça, só que
neste caso para a vida intelectual, como soe acontecer toda vez que a
opulência se instala na existência do homem. Eles ali negligenciavam á
observação das leis religiosas, ignoravam os sábios e a falta de
conhecimentos se generalizava, pro­vocando a decadência religiosa, como
se podia observar pelo desenvolvimen­to e prosperidade da doutrina
caraíta no país.

Os caraítas constituíam uma seita judia separatista, que desprezava a
tradição oral legada pelas instituições rabínicas, e que se guiava ao pé
da letra pela Torah. Ao contrário do que ocorria nos outros países, onde
essa seita esta­va em vias de extinção, seu distanciamento da maioria da
comunidade judia

MAIMÔNIDES --- VIDA E OBRA 21

fazia com que ela prosperasse no Egito, pois lá encontrava um ambiente
favo­rável, junto aos Maometanos, que acreditavam estarem os caraítas
mais próxi­mos do Islam do que os judeus seguidores das leis talmúdicas.
Vários judeus se juntaram aos caraítas e foram recompensados com favores
políticos, já que essa seita gozava da confiança dos Fatimitas xiitas. A
influência deles se fazia sentir, e aos poucos desapareciam na
comunidade os ritos tradicionais, fazen­do com que os próprios rabinos
se sentissem impotentes com relação ao pro­gresso dessa assimilação.

Como estimasse, por tudo isso, que os caraítas eram inadequados para
executar os deveres religiosos dos judeus tradicionais, Maimônides
come­çou a propôr uma cisão de cultos a fim de eliminá-los
definitivamente da vida religiosa judia. Isso provocou a cólera dos
caraítas a tal ponto que ele se viu forçado a partir de Alexandria.

Portanto, por volta de 1168, Maimônides parte para Fostat, onde se erige
atualmente a antiga cidade do Cairo. Dois anos depois de sua chegada,
ele já ocupa ali um rabinato e trabalha intensamente na ajuda aos
necessitados. Sabe-se, por exemplo, que em 1169 ele envia diversas
cartas circulares às co­munidades egípcias a fim de conseguir o dinheiro
para o resgate dos judeus que haviam sido aprisionados por Amaury,
governador franco de Jerusalém, para evitar que eles fossem vendidos
como escravos.

Depois de ter tentado afastar o perigo que os caraítas representavam
para a vida do judaísmo autêntico, Maimônides se lança na reforma dos
costu­mes no que se refere às preces feitas na sinagoga. Ele percebera
que enquanto o "hazan" fazia em voz alta, na sinagoga, a oração
silenciosa da comunidade, as pessoas conversavam ao invés de escutar com
recolhimento a oração. Con­siderando isso um desrespeito a Deus, ele
ordenou que se desse início a essa oração primeiramente em voz alta, e
que ela fosse seguida depois por todos em silêncio e com recolhimento,
em vez do que se fazia até então. Essa melho­ria, que ele ousa impor a
despeito da ordem da oração talmúdica, encontra o apoio e o
reconhecimento dos\_ sábios contemporâneos e é aceita no Egito.

Incansável na sua busca da perfeição, ele deseja também unificar os
ritos dos dois grupos em que estavam divididos os judeus do Egito, ou
seja, os Babilonianos e os Jerusalemitas. Os Babilonianos dividiam a
Torah de ma­neira que ela pudesse ser lida completamente num ano
enquanto que os Jerusa­lemitas utilizavam um ciclo de três anos. Cada um
desses grupos tinha sua sina­goga e não tinha outros ritos em comum a
não ser o Simhat Torah e o Sche­vuoth. Essa oposição de usos dentro da
comunidade judia chocava Maimôni­des que, como judeu espanhol, estava
habituado a uma liturgia uniforme, fun­dada na ordem das orações de
Amram. A diversificação dos ritos pareceu-lhe imprópria, e embora ela se
apoiasse na tradição local, Maimônides sentiu que a lógica da lei e do
pensamento deveria substituir a comodidade dos hábitos. Contudo, ele se
defrontou aqui com a oposição de Zuta, o Naguid da época, homem
ambicioso que aproveitou a ocasião para afastar aquele estrangeiro
atre­vido e audacioso, dizendo que a reforma que Maimônides queria
instaurar no Egito devia ser considerada como um ato de hostilidade
contra o governo. Não resta então a Maimônides outra alternativa a não
ser abandonar temporariamente essa luta e afastar-se de Fostat por algum
tempo, tempo esse que ele aproveita para iniciar sua obra magna, o
\emph{Mishneh Torah.}

A morte de seu irmão Davi no naufrágio de um navio, por volta de 1171,
representa um grande golpe para Maimônides. Além de minar
irremedia­velmente sua saúde, essa grande dor desencadeia uma crise
decisiva em sua al­ma e acarreta uma mudança profunda em seu pensamento
e na sua visão do

22 MAIMÔNIDES

mundo. Junto com o irmão desaparece também toda a fortuna da família, e
co­mo estimasse que nenhum sábio devesse viver às custas da comunidade
para poder prosseguir seus estudos, "pois nem na Torah, nem nos livros
posterio­res dos sábios não encontra alguma coisa que apóie essa tese"
--- como ele pró­prio diz ---, Maimônides decide então tornar-se médico
e ganhar dessa forma seu sustento. Ele dava, dessa forma, o exemplo e
recomendava também a to­dos os estudiosos sábios que ganhassem seu pão
graças a seu trabalho e não às custas da religião. Médico devoto,
escreve um juramento para todos os mé­dicos, judeus ou não, no qual
reafirma o dever que eles têm de curar, e faz uma oração para que Deus
lhes preste assistência e intervenha por eles. Essa oração diz o
seguinte:

"Oh Deus! O Senhor formou o corpo do homem com uma bonda­de infinita!
--- O Senhor uniu nele inúmeras forças que trabalham incessante­mente
como tantos instrumentos a fim de preservar em seu todo esta casa
ma­ravilhosa, contendo uma alma imortal, e essas forças atuam com toda a
ordem, concordância, e harmonia imaginável. Mas se fraqueza ou paixão
violenta per­turbam esta harmonia, estas forças agem uma contra a outra
e o corpo volta ao pó de onde ele veio. O Senhor então envia ao homem
seus mensageiros, as doenças, que anunciam a aproximação do perigo e
pede que se prepare para vencê-las. A Eterna Providência me apontou para
cuidar da vida e da saúde de Suas criaturas. Que o amor a minha arte me
deixe agir em todos os momentos, que a avareza e a mesquinhez, tanto
como a sede da glória ou da reputação, não tomem conta dos meus
pensamentos, porque sendo inimigos da verdade e da filantropia poderiam
me decepcionar e me fazer esquecer da minha meta de fazer o bem aos Seus
filhos. Me enriqueça com força de alma e mente, para que ambas estejam
prontas a servir ao rico e ao pobre, ao bom e ao mau, amigo e inimigo e
que jamais enxergue o paciente senão como um ser igual, adoecido.

"Se médicos mais cultos que eu desejam me aconselhar, me inspirar
confiança e obediência, aceito de bom grado, pois o estudo da ciência é
mara­vilhoso. Um só ser não pode enxergar tudo. Que eu seja moderado em
tudo, exceto na sabedoria da ciência; que eu seja insaciável até o ponto
certo; permi­ta-me ter sempre a força e a oportunidade de corrigir
minhas aquisições, sem­pre estendendo meu domínio, porque a sabedoria
não tem limite e o espírito do homem também se estende infinitamente,
para que diariamente se enrique­ça com novos conhecimentos. Hoje ele
pode descobrir os erros de ontem, e amanhã ele pode obter nova
iluminação sobre o que ele deu certeza hoje.

"Deus, o Senhor me apontou para cuidar da vida e da morte de Suas
criaturas; eis-me pronto para minha vocação."

Por volta de 1172, comovido e preocupado com a situação dos ju­deus que
viviam no Yemen, acossados que estavam pelos "confessores da uni­dade"
no oeste, pelos xiitas no leste, e desnorteados depois do anúncio que
havia sido feito, por um pobre e ingênuo lunático, da chegada do Messias
para breve, Maimônides lhes envia três cartas alentadoras, que ele
redige em árabe para que pudessem ser compreendidas por todos. Numa
delas ele diz o seguin­te: "Devemos ficar satisfeitos por sofrer todos
esses infortúnios, essas perse­guições, esse exílio, a perda de nossos
bens e as injúrias de todos, pois todas essas misérias são uma honra que
Deus nos concede". Ele diz ainda que todo o mal que se sofria era um
sacrifício que se levava ao altar e lhes recorda que Deus havia
prometido que nenhuma opressão duraria muito tempo e que seu povo nunca
seria destruído. Afirma também que o que eles estavam passando naquele
momento não era um sofrimento, e sim um mal preliminar que anun­ciava o
reino do verdadeiro Messias. Maimônides acredita que a inveja é a ver-

MAIMÔNIDES --- VIDA E OBRA 23

dadeira motivação que leva os outros povos a perseguir e a oprimir
constante­mente os judeus: não podendo atacar-se ao Todo Poderoso por
ter Ele escolhi­do o povo de Israel como herdeiro e guardião de Seu
estatuto e de Sua doutri­na, eles se lançam contra o povo em si, numa
Guerra Santa que dura desde os tempos de Amalec.

Uma vez concluído seu \emph{Comentário sobre a Mishná,} Maimônides
concebe a \emph{Mishná Torah,} uma obra que o ocuparia de 1170 a 1180, e
que de­veria guiar os leitores através das obscuridades e das
imprecisões do Talmud, no qual está contida a vida interior e exterior
do judaísmo.

O Talmud (tanto o da Babilônia quanto o de Jerusalém) se desenvol­veu
durante séculos como o comentário da Mishná. Uma enorme quantidade de
opiniões e de novos conhecimentos foram expressos ali e fixados como
con­tinuação do texto da Mishná. Esse trabalho se terminou no século 5,
mas ficou rapidamente demonstrado que o povo era incapaz de compreender
esse alto ensinamento e é por isso que apenas um pequeno número de
pessoas se consa­grava ao estudo do Talmud. As perseguições e os
sofrimentos que vinham cas­tigando os judeus anos a fio os obrigavam a
deixar os estudos religiosos cada vez mais em segundo plano e a dar
prioridade à preservação das próprias vidas; a sapiência dos sábios e o
raciocínio dos filósofos se perdiam, as explicações sobre o Talmud que
os Gaonim davam, e que eles julgavam estar ao alcance de todos,
começavam a não mais ser compreendidas, assim como os próprios textos do
Talmud, da Sifrá, dos Sifris e da Toseftá, pois a compreensão dessas
obras exige uma grande inteligência, uma alma preparada e extensos e
aprofun­dados estudos.

Maimônides constata que o povo em si não tinha a sua disposição um
código onde pudesse encontrar regras seguras, sem mistura de
controvér­sias e de opiniões. Assim sendo, ele deseja expor em sua obra,
numa linguagem clara e breve, o que é proibido e o que é permitido, o
que é puro e o que é impuro, bem como tudo o que se refere às questões
da Torah, tudo isso para que a lei possa ser conhecida por todos, sem
deixar dúvidas. Ele quer que a Lei esteja, em palavras claras, na boca
de todos os homens e faz sua exposição de maneira direta e didática,
buscando dar um esclarecimento simples e satisfa­tório a questões que,
de outra forma, poderiam ser interpretadas erroneamente pelo povo. É
assim, por exemplo, que ele explica que o preceito que nos orde­na temer
a Deus implica, na realidade, não no medo do Eterno, mas sim no respeito
ao nome de Deus, na reserva em utilizá-lo, no cuidado para não come­ter
uma blasfêmia, glorificando-o e louvando-o cada vez que ele for
pronuncia­do. Portanto, o temor ao Eterno significa a obrigação de
santificar Seu nome e de estar sempre alerta para não profaná-lo, o que
envolve, entre outras coi­sas, o dever de se deixar matar antes de
renegar o Senhor em benefício de um deus pagão ou antes de entregar aos
gentios um de seus irmãos israelitas para que ele seja morto ou
desonrado. Santifica também o nome do Senhor aquele que "se afasta de
toda transgressão ou observa os preceitos sem ser levado a fazer isso
por alguma consideração de ordem profana, terror, temor, ou busca de
reconhecimento", ao passo que profana o santo Nome "todo aquele que
transgride espontaneamente e na ausência de qualquer tipo de pressão,
por des­dém e com o intuito de escandalizar, nem que seja apenas um dos
preceitos anunciado pela Lei". Assim, com este que pode ser considerado
um código me­tódico de referência para o dia a dia, Maimônides tem a
convicção de poder levar seu leitor a descobrir, passo a passo, qual é a
atitude correta e qual é o caminho a ser seguido para alcançar a
perfeição do corpo e da alma que o Se­nhor espera de nós.

24 MAIMÔNIDES

A grande dificuldade para a execução de sua obra estava principal­mente
no fato de que não havia nenhuma preparação anterior que pudesse
aju­dá-lo em seu trabalho pois, de acordo com suas palavras, "A
sapiência dos sá­bios de nossa época consiste em julgar a verdade de uma
sentença não de acor­do com seu conteúdo, e sim de acordo com sua
conformidade com a sentença de um predecessor, sem examiná-la."
Maimônides decide então executar sua obra como uma codificação, como um
resumo sistemático, e não como um co­mentário, que era o que o Talmud
fazia com relação a Mishná. Decide também não citar as opiniões
discutidas e refutadas, mas sim fornecer apenas as deci­sões que têm
força de lei. Ele deseja expor todas as doutrinas da Mishná e do Talmud,
sem dar o nome do autor de cada uma delas após cada citação, mas dizendo
apenas que todas as frases da Torah que constituem a lei oral haviam
sido transmitidas por este e aquele, desde Ezra e desde nosso mestre
Moisés.

Seguindo fielmente a tradição judaica, Maimônides se limita estrita e
logicamente ao Talmud, tomando por lei o que ele encontra explicitamente
decidido. Como na maioria dos casos as questões estão dissertadas, sem
apre­sentar uma recomendação firme e final, ele próprio as resolve, e é
exatamente isso o que há de mais importante em seu trabalho, não a
compilação em si. Os princípios e os métodos que ele aplica, com uma
lógica minuciosa, em suas de­cisões pessoais, são de uma elevação de
pensamento que prepara um terreno novo para todos os séculos seguintes.
Ele distingue no Talmud os elementos halachicos obrigatórios e os
elementos hagádicos, que não o são. Isso lhe per­mite ter uma
independência de opinião completa em relação às decisões dos sábios
talmúdicos que não eram de origem religiosa, sempre que ele não pôde
prová-las cientificamente. O equilíbrio resultante da independência e da
fideli­dade de seu espírito original está repleto da mais legítima e
autêntica autorida­de e é a obra-prima de sua atitude intelectual.

Sua objetividade científica faz com que ele atribua a mesma impor­tância
a todas as matérias da Halacha, sem levar em conta sua relação com a
atualidade. Ele fala sobre todos os preceitos, inclusive aqueles que não
estão mais em aplicação desde a destruição do Templo, e lastima que
ninguém mais se interesse em pesquisar ou conhecer essas leis, pois
assim elas terminam por ser esquecidas.

Maimônides tem consciência de escrever um livro definitivo: "Nin­guém
terá necessidade de ajuda para conhecer a lei judaica, se ele tiver
minha obra que forma uma coletânea completa de todas as instituições,
usos e decre­tos, desde Moisés até o fim da redação do Talmud, incluindo
as explicações posteriores dos Gaonim", e é por isso que ele intitula
seu código de "Mishneh Torah", que significa "repetição da lei". Isso
representa uma revolução e uma reforma no ensino da religião: quem
estudasse primeiro as Escrituras e depois o Código de Maimônides
conheceria toda a doutrina da tradição oral e poderia, em tese, se
abster de pesquisar em outras obras. Através da utilização de seu código
ganha-se o tempo que outrora se dedicaria ao estudo do Talmud, tem­po
esse que podia ser consagrado aos estudos filosóficos. Maimônides deseja
despertar o interesse dos estudantes pela posição filosófica do problema
para depois dirigir seus pensamentos para a metafísica, pois da mesma
forma que ele também prefere a análise contemplativa ao julgamento das
ações, ele prefe­re o estudo da metafísica, "raízes da doutrina", ao
estudo da dialética do Tal­mud, "ramificações da doutrina", embora ele
considere indispensável as par­tes dialéticas do Talmud e considere que
as partes hagádicas são a fonte da ciência filosófica.
\end{quote}

Ele abre mão da discussão, que é o que acontece no Talmud, em

\begin{quote}
MAIMÔNIDES --- VIDA E OBRA\_ 25

favor da decisão. Na ciência talmúdica as pesquisas analíticas não são
levadas em consideração; sua pedagogia inculca "a doutrina pela
doutrina" e seu estu­do conduz a uma teoria, e não a uma decisão
essencial para a prática. É esse aspecto do ensino e dos métodos de
pensamento que Maimônides pretende reformular.

Aquela era a primeira vez na história que um homem ousava reco­lher numa
só obra a totalidade que constitui a ciência hebraica. As tentativas
feitas nesse sentido até então haviam fracassado, devido a imensidade de
mate­rial disperso existente. O caráter de sua codificação consiste em
citar a idéia em vez do acontecimento, a lei em vez do caso, a coisa em
vez das pessoas, trocando a história pela especulação, a situação
concreta pela abstração.

Maimônides utiliza um método novo para a forma do livro. Ele de­via
repartir um grande número de prescrições, de leis e de decisões
especiais nos 613 "compartimentos" de seu livro, correspondentes cada um
deles a um dos preceitos derivados da Torah. Resolve, então, redigir 613
parágrafos, orde­nando-os em 83 seções e distribuindo-os em 14 livros.
Para que sua grande obra pudesse ser lida com maior clareza e para
preservar a unidade do conjunto, Mai­mônides decide escrever um prefácio
onde apresentaria um resumo dos "Tar­yag Mitzvoth", ou dos 613 preceitos
divinos mas, ao estudar as enumerações já existentes desses preceitos,
chega a conclusão de que nunca se haviam esta­belecido normas claras
para se chegar à classificação do que era e do que não era um preceito,
e de que muitas vezes o conteúdo havia cedido lugar à forma. Sendo os
"613 preceitos" o fio condutor da vida e da crença do povo judeu, já que
eles representam os desejos do Eterno expressos na Torah, era de
supre­ma importância que o povo soubesse com clareza e com precisão
quais eram esses desejos para que pudesse cumpri-los.

Assim sendo, Maimônides inicia a redação de seu \emph{Livro dos
Precei­tos Divinos} ("Sefer Ha-Mitzvoth") com o intuito de esclarecer e
ordenar os 613 preceitos, e de dirimir as dúvidas dos leigos. Esse
livro, escrito em linguagem precisa e acessível, propeorcionava também
aos jovens da época um caminho mais fácil para adquirir algum
conhecimento a respeito dos preceitos não mais em aplicação e das coisas
do Templo e do Santuário, pois Maimônides estava preocupado com o fato
de que não só eles ignoravam esses aspectos da Tradi­ção, como também
não demonstravam nem ao menos alguma curiosidade ou interesse pelas
coisas do passado.

Maimônides divide o livro em 2 partes. Na primeira ele estabelece os 14
Fundamentos ou os princípios lógicos que utiliza para determinar o que é
e o que não é um preceito. Na segunda ele apresenta de maneira detalhada
os 613 preceitos, divididos em 248 positivos e 365 negativos. Ao longo
de to­do o livro Maimônides se empenha em ir esclarecendo seu
raciocínio, cita fon­tes, argumentos, a literatura rabínica, sem no
entanto deixar de apresentar sem­pre uma conclusão clara e decisiva,
para que não restem dúvidas ao leitor quanto ao que deve e ao que não
deve ser feito.

"A grosso modo", os princípios são apresentados agrupados de acor­do com
os assuntos neles tratados. Assim, vemos agrupadas as obrigações ---bem
como as proibições --- referentes ao homem para com Deus, para com seus
semelhantes, para com sua família, às relativas ao Templo, às impurezas,
às festas religiosas, ao cultivo da terra, à justiça, ao estado etc.

\emph{O Livro dos Preceitos Divinos} serviu de base para muitas outras
obras, inclusive para o \emph{"Livro do Ensino"} (Sefer Ha-Chinuch), a
grande obra do tal­mudista espanhol.do século XIII Aaron Ha-Levi, que
trata especificamente des­ses preceitos.

26 MAIMÔNIDES

Uma vez concluído esse livro, em 1170, Maimônides pode então vol­tar a
dedicar-se de corpo e alma ao \emph{Mishneh Torah,} obra que foi
recopiada por escribas profissionais e que se espalhou pelo mundo
inteiro, conquistando sá­bios, estudiosos, rabinos e juízes. Várias
comunidades o adotaram como códi­go. O Talmud era uma obra vasta e
complexa; o \emph{Mishneh Torah} era escrito de maneira ordenada e
clara, com uma linguagem fácil, onde os leitores encontra­vam a verdade
e aprendiam o que a doutrina da moral encerrava de mais pro­fundo.
Maimônides deixou o \emph{Mishneh Torah} como seu grande legado
organi­zador que iria exercer uma influência definitiva na vida de seu
povo. Há muito tempo a voz de um homem não tinha tal influência sobre os
judeus.

Contudo, em breve começam a aparecer os opositores de Maimôni­des, que
colocam em questão sua autoridade como legislador. Eles censuram a obra
de Maimônides porque ele expõe as regras legais sem citar a origem, sem
dar o nome do autor, sem provas nem peças de apoio.. Até então só se
reconhe­ciam como obrigatórias as decisões dos Gaonim pois elas se
apoiavam não ape­nas nas suas qualidades pessoais, mas também na
autoridade das academias, e os judeus reconheciam a instituição e não a
pessoa. Pesa ainda contra ele o fato de ter escrito, na introdução de
sua obra, que depois de ter estudado seu Códi­go se poderia renunciar ao
estudo da literatura pós-bíblica. Isso foi interpreta­do como uma
tendência a afastar o Talmud das escolas, o que representava uma
profanação e um perigo. Maimônides deixa passar a ocasião de contestar a
tem­po essa falsa interpretação de sua doutrina e acaba por começar a
ser conside­rado por alguns como herege!

Depois de ter mantido uma correspondência preliminar com o grande
mestre, na qual solicitava ser aceito como discípulo e poder estudar com
ele, em 1185 chega a Alexandria o jovem Yossef Ibn Yehuda. Maimônides,
que an­tes nunca se havia interessado pelo ensino público e imediato, e
cuja necessi­dade de ensinar estava voltada mais para a redação do que
para a exposição oral, decide no entanto transmitir seus ensinamentos a
esse rapaz que viera de Ceuta, e que acaba por suscitar a simpatia do
mestre. A decisão de concordar em transmitir seus ensinamentos a um
aluno se prendeu principalmente ao fato de que o número dos que se
interessavam pela ciência e pela filosofia no Egito era muito reduzido.
Há muito ele esperava por uma pessoa a quem pudesse trans­mitir seus
ensinamentos e descobertas, e a afinidade e a afeição que ele acaba
desenvolvendo por Ibn Yehuda é tão grande que ele passa a chamá-lo de
"meu filho".

Esse jovem, para quem Maimônides era o representante de uma ciência mais
elevada, se interessava sobretudo pela questão da interpretação das
Escrituras e desejava iniciar-se aos mistérios da ciência superior.
Embora tendo durado pouco menos de dois anos, o convívio com Ibn Yehuda
foi muito esti­mulante para Maimônides, que procura fazer com que esse
jovem entusiasta e impaciente por atingir a "ciência interior" seguisse
o caminho do estudo me­tódico e lento. Ele lhe diz, numa carta: "E
quando você fez comigo seus estu­dos de lógica, eu depositava minha
esperança em você, e o julgava digno de revelar-lhe os mistérios dos
livros proféticos, para que você compreendesse aqui­lo que os homens
perfeitos devem compreender."

Contudo, antes que Maimônides chegasse a introduzir seu discípulo aos
assuntos metafísicos, este se muda para Alep, por motivos ignorados. Mas
não parte sem antes obter de seu mestre a promessa de que ele redigiria
um tratado no qual responderia as questões que o preocupavam e cuja
resposta ti­nha ido buscar junto a ele.
\end{quote}

E é assim que, para cumprir a promessa feita ao seu único discípulo

\begin{quote}
MAIMÔNIDES --- VIDA E OBRA 27

Maimônides redige, de 1187 a 1190, e não sem grande hesitação, a sua
última obra, o \emph{Guia dos Perplexos,} um trabalho que deveria
"explicar os pontos obs­curos da lei e manifestar o verdadeiro sentido
das alegorias, que estão acima da inteligência comum". Esse é um
trabalho difícil para ele, pois trata-se de des­vendar os mistérios que
lhe parecem invioláveis, e se algumas considerações incidentais em suas
obras anteriores já haviam provocado grandes oposições, o que não
poderia resultar da apresentação completa de suas concepções
filo­sóficas? Ele chega, no entanto, à conclusão de que, apesar de tudo,
é importan­te que o livro seja escrito, e explica nele porque tomou essa
decisão: "... eu sou o homem que, vendo-se apertado numa arena estreita
e não encontrando a forma de ensinar uma verdade bem demonstrada, a não
ser de uma maneira que convenha a um só homem notável e que desagrade a
dez mil ignorantes, prefere falar para essa única pessoa, sem prestar
atenção à reprovação da gran­de multidão, e espera tirar esse único
homem notável da confusão em que ele caiu e mostrar-lhe o caminho para
sair de sua perplexidade, a fim de que ele se torne perfeito e de que
obtenha o repouso.' Num outro trecho dessa mes­ma obra ele diz também
que "A verdade não se torna mais verdadeira pelo fato de todo mundo
acreditar nela, tampouco pelo fato de todo mundo discordar dela".

Ele estava preocupado com o desinteresse dos judeus pela filosofia. A
maioria deles sentia uma separação entre a fé e o saber, entre o
conteúdo da revelação e as doutrinas filosóficas. Para Maimônides, no
entanto, a Hagada é também uma das fontes da ciência filosófica: tudo o
que ali se encontra em forma de parábolas coincide com os ensinamentos
da filosofia \href{http://abstrata.Com}{{abstrata. Com}} o \emph{Guia
dos Perplexos} Maimônides torna possível o acesso da razão àqueles
as­pectos da Torah que não estão ao alcance da capacidade humana. Ele
deseja guiar "o homem religioso, no qual a verdade de nossa lei está
estabelecida na alma e se tornou um objetivo de crença, que é perfeito
na sua religião e nos seus costumes, que estudou as ciências dos
filósofos e conhece os diversos as­suntos delas, e que foi atraído e
guiado pela razão humana, para fazê-lo entrar em seu domínio". Sua
obra-prima filosófica se destina, portanto, apenas aos sábios e aos
estudipsos, pois os leigos e os ignorantes jamais poderiam com­preender
as revelações nela contidas. Na introdução desse livro ele diz: "Meu
pensamento vai guiá-los no caminho do verdadeiro e vai torná-lo mais
fácil. Venham, caminhem pela sua senda, vocês que vagam no campo da
religião! O impuro e o ignorante não passarão por ele; ele será chamado
de caminho secreto."

Essa obra visava, em primeiro lugar, proteger os estudiosos das
co­munidades judias da sedução que as filosofias árabe e grega exerciam
no século XII, e a grande originalidade de Maimônides nesse trabalho foi
a de estabelecer um diálogo entre o mosaísmo e a filosofia, ao invés de
se limitar a utilizar-se de seus conhecimentos filosóficos para fazer a
apologia do judaísmo. Ele não renuncia a nenhuma das tradições do
pensamento judeu, nem tampouco ali­menta a ilusão de poder "conciliar" a
verdade bíblica e a verdade filosófica. Ao invés disso, ele confronta as
duas tradições, de maneira a sobrepô-las. As­sim, ele se outorga a
missão de guiar os estudiosos para o conhecimento meta­físico o qual,
segundo ele, é uma possessão original do judaísmo que havia sido perdida
durante o exílio, e é essa perda que torna o exílio tão trágico. Ele tem
a convicção de que o renascimento da compreensão mais elevada, obtida
gra­ças a introdução da filosofia nos estudos religiosos, é o fato
libertador que con­duzirá ao acontecimento messiânico, teoria essa que,
aliás, acredita-se ser ele o primeiro a introduzir naqueles tempos de
exílio.

28 MAIMÔNIDES

Seu \emph{Guía dos Perplexos} se espalha extraordinariamente na
literatura mundial e seu sistema é de grande importância para Nicolau de
Coues, Leibniz e Spinoza. A negação da tese da eternidade do mundo em
favor de sua perpe­tuidade, a modificação da doutrina de Aristóteles
sobre Deus, sua profetologia independente, sua explicação do significado
dos preceitos divinos e seu méto­do de explicação da Bíblia entram no
sistema de pensamento dos escolásticos cristãos como Tomás de Aquino.
Alexandre de Halle, Alberto, o Grande e Iná­cio de Loyola também
aceitaram suas teorias como elementos de seus sistemas.

Embora ciente de que essa obra não esclareceria todas as dúvidas,
Maimônides espera que ela afaste as principais. A aspiração a conhecer
Deus, necessidade de seu pensamento presente desde sua juventude, faz
com que ele se lance à procura de um sistema metafísico. Ele adota a
razão como método de meditação, esperando chegar ao conhecimento de Deus
pela eliminação de negativas e das imperfeições. A própria existência do
indivíduo se torna para Maimônides o ponto de partida para o
conhecimento e para a compreensão de Deus. Ele conclui que "Não há na
realidade no ser nada além de Deus e to­das as suas obras; estas são
tudo o que o ser encerra fora d'Ele. Não há nenhu­ma outra forma de
conceber Deus a não ser por Suas obras; são elas que indi­cam Sua
existência e o que se deve crer a Seu respeito".

Uma questão que preocupa muito Maimônides é como ascender à profecia, o
mais alto grau de perfeição possível a um homem.,Ele faz no seu
\emph{Guia} a exposição de seu ponto de vista a esse respeito, e diz que
ela é um dom que Deus concede, quando Ele quer, a alguns poucos homens
que Ele escolhe dentre aqueles que possuem os três requisitos básicos: a
perfeição do pensa­mento, atingida através da especulação e da
concentração da inteligência no estudo e na busca do conhecimento de
Deus; a perfeição da imaginação, que foge ao controle do homem e que
depende de um cérebro perfeitamente for­mado; e, por último, a perfeição
do caráter e da moral, que se alcança libertando-se o espírito do desejo
de se obter satisfações terrenas e erradicando a ambição pelo domínio e
pelo poder. Essas condições são indispensáveis porque a reve­lação
profética é uma emanação de Deus que se manifesta, a um homem justo,
primeiramente por intermédio do "intelecto ativo", em relação a
faculdade de pensar, e depois em relação a imaginação.

Se essa emanação divina atingir unicamente a faculdade racional do
homem, seja porque sua faculdade imaginativa não foi perfeitamente
formada, seja porque a emanação foi insuficiente para alcançar a
faculdade imaginativa desse homem, ele pertencerá à categoria dos sábios
que dedicam suas vidas ao estudo e às pesquisas. Se ela agir apenas
sobre a faculdade imaginativa, seja por­que a faculdade racional é
primitiva ou porque ela foi pouco exercitada, o que teremos é um homem
que pertence ao grupo onde se classificam os legislado­res, os
adivinhos, os ocultistas, os videntes e os feiticeiros. Essas pessoas
têm sonhos e visões semelhantes aos dos profetas, e isso faz com que
eles acredi­tem ser um deles. De acordo com as palavras de Maimônides,
elas pensam "que adquiriram ciências sem ter estudado, e provocam uma
grande confusão nas coisas importantes e especulativas, misturando, de
uma maneira surpreenden­te, as coisas verdadeiras e as quimeras. Tudo
isso ocorre porque sua faculdade imaginativa é forte, enquanto a
faculdade racional é fraca e não obteve absolu­tamente nada". Mas se
essa emanação divina se espalhar pela faculdade racio­nal e daí passar à
faculdade imaginativa, tendo ambas atingido a perfeição que explicamos
anteriormente, esse homem estará na categoria dos verdadeiros
\end{quote}

profetas. E como recebe o homem uma profecia? À exceção de Moisés, a
quem

\begin{quote}
MAIMÔNIDES --- VIDA E OBRA 29

Deus falou diretamente, os profetas recebem essas revelações através de
um mensageiro de Deus, um anjo, que lhes fala num sonho ou numa visão,
situa­ção esta em que seus sentidos ficam paralisados e a emanação
divina se espalha por sua faculdade racional, passando daí para a
faculdade imaginativa, fazendo com que ela se aperfeiçoe e entre em
funcionamento. Maimônides escreve que "Às vezes a revelação começa por
uma visão profética; depois essa agitação e essa forte emoção
decorrentes da ação perfeita da imaginação, vão aumentan­do e então
ocorre a revelação".

Convencido, portanto, de que "o vigor humano do espírito não é
suficiente para se alcançar o conhecimento de todas as coisas",
Maimônides crê que a profecia é a única capaz de decifrar todas as
verdades para a limitada inteligência humana, e de conduzir à
compreensão da criação do mundo e da origem das espécies. Para ele o
sábio está no cume da hierarquia humana, uma vez que este se prepara
para receber as revelações de Deus: "pois apenas aque­le que chegou a
perfeição especulativa pode obter a seguir outros conhecimen­tos, quando
o intelecto divino se extravasa sobre ele." O verdadeiro profeta é
aquele que prevê os acontecimentos, com absoluta verdade, a médio e a
lon­go prazo, e cujas previsões de coisas boas sempre se realizam.
Contudo, caso o profeta tenha previsto um mal que não venha a ocorrer,
isso não significa que ele não é mais um profeta, mas sim que Deus, na
sua infinita bondade, teve misericórdia de seu povo e mudou o mal em
bem.

Para ilustrar seu pensamento a esse respeito ele conta no seu
\emph{Guia} uma parábola através da qual afirma que o estudo da
literatura talmúdica, fun­dada na tradição e sagrada para todos os
judeus, não é a chave que abre a porta do "palácio" (onde vive o
"soberano", ou seja, Deus), mas que filosofar a esse respeito,
aprofundando-se cada vez mais nesse estudo, é o caminho que abre essa
porta e que conduz a Deus. Sem dúvida, essa nova ordem das potências
espirituais é extremamente audaciosa, sobretudo vinda de um homem que
con­sagra o melhor de sua existência ao estudo das tradições talmúdicas,
e só pode ser legítima se corresponder a um autêntico desejo dele de
estabelecer uma no­va ordem das coisas. Essa nova ordem é o coroamento
da obra de Maimônides.

Quando escreveu seu \emph{Comentário sobre a Misbná,} ele pensava ain­da
que tudo o que existe no universo só existe em função dos homens.
Contu­do, essa concepção antropocentrista vai se dissipando ao longo dos
anos para dar lugar à crença de que o homem em si não pode ser imaginado
como o últi­mo objetivo do desenvolvimento universal, de que a espécie
humana é bem pouco com relação ao mundo superior das esferas e dos
astros, e de que não há nada que indique que o mundo só existe por causa
do homem: "... se a Ter­ra inteira não passa de um ponto imperceptível
com relação à esfera das estre­las, qual será a relação da espécie
humana com o conjunto das coisas criadas? E como então alguém dentre nós
poderia imaginar que elas existem em seu fa­vor e por sua causa, e que
elas devem servir-lhe como instrumentos?"

Essa nova concepção do homem como mero componente de um cosmo ordenado e
perfeito abre um caminho que o leva a encontrar a explica­ção filosófica
do mal, explicação essa que ele renunciara a descobrir aos 25 anos,
dando-se como justificativa o fato de que os sábios que o haviam
precedido também não o haviam conseguido. Agora, vários anos após a
morte de seu ir­mão David, a impetuosidade de sua juventude desaparece,
deixando lugar a uma calma e ponderação maiores, e essa maturidade de
espírito lhe traz console e paz de espírito. Maimônides destrói o mal
filosoficamente e conclui que sua ocorrência não passa de exceções
esporádicas, quando se leva em conta a har­monia de toda a criação
divina: "Todo ignorante imagina que o universo in-

30 MAIMÔNIDES

teiro só existe em favor de sua pessoa, como se não houvesse nele nenhum
outro ser a não ser ele próprio. Portanto, se o que lhe acontece é
contrário aos seus desejos, ele julga que todo o ser é o mal; mas se o
homem considerasse e concebesse o universo, e se soubesse quão pequeno é
o lugar que ele aí ocu­pa, a verdade lhe apareceria claramente".

Maimônides passa a definir o mal como uma simples privação, co­mo algo
que aparece em decorrência da ausência de uma ação. Ora, como uma ação
só pode resultar em algo que venha a existir e não numa privação, o mal
não pode ser resultante de uma ação direta se essa ação não existe, e
portanto esse mal também não existe. Caso assim não fosse, como diz
Maimônides, "se alguém produz uma matéria incapaz de receber determinada
capacidade, poder-se-ia dizer que ele fez tal privação; da mesma forma,
se alguém tivesse sido ca­paz de salvar uma pessoa da morte, mas tivesse
se abstido de fazê-lo e não a tivesse salvo, poder-se-ia dizer dele que
ele a matou". Ele conclui que o bom é o ser, aquilo que existe, e o ruim
é o não ser: "No homem, por exemplo, a morte é um mal e é sua não
existência; da mesma forma, sua doença, sua po­breza, sua ignorância são
males em relação a ele e são privações de capacidade ... A destruição
nada mais é do que a privação da forma".

Ele classifica os males em três tipos: o primeiro é o que recai sobre
determinados indivíduos, devido a nossa própria natureza humana, e que é
o responsável pelas doenças e deformações congênitas ou resultante de
altera­ções ocorridas na natureza, tal como os terremotos, conforme ele
próprio exem­plifica. Esses males são resultantes naturais da própria
imperfeição da matéria da qual o homem é feito, pois "a coisa mais
eminentemente perfeita que possa formar-se a partir do sangue e do
esperma é a espécie humana, com sua bem conhecida natureza de ser vivo,
racional e mortal". Mas levando-se em conta o total da humanidade em
todos os tempos, tais tipos de males não passam de meras exceções. A
segunda categoria abrange os males que os homens se infli­gem uns aos
outros, tal como a tirania e a barbárie provocados, entre outras causas,
pelas paixões e pelas divergências de opiniões e de crenças; segundo
Maimônides: "Esses grandes males ... são todos decorrentes de uma
privação, pois todos eles resultam da ignorância, ou seja, da privação
da ciência ... por­que o conhecimento da verdade faz cessar a inimizade
e o ódio e impede que os homens se façam mal uns aos outros ...". E há,
finalmente, os classificados no terceiro tipo, que são os mais
freqüentes e que constituem os males que acontecem aos seres humanos
como decorrência de seus abusos dos prazeres mundanos, tal como a
bebida, a comida e a luxúria. Ele escreve que "a maioria dos males que
atingem os indivíduos provêm deles próprios, quero dizer, dos indivíduos
humanos, que são imperfeitos. Se sofremos é por causa dos males que nós
mesmos nos infligimos espontaneamente, mas que atribuímos a Deus ..."
Este último tipo de mal é o mais nocivo, já que além de atingir o corpo
ele pre­judica também a alma, seja porque, sendo uma força corporal, ela
é influencia­da diretamente pelas alterações ocorridas no corpo, ou
porque a alma acaba por se familiarizar com o supérfluo e a se habituar
a ele, levando o homem a desenvolver uma ambição sem termo que o faz
buscar riquezas e grandezas não essenciais e totalmente desnecessárias.

Maimônides explica também no \emph{Guia dos Perplexos} que toda vez que
Deus deseja manifestar sua vontade, seja para desencadear algum
aconteci­mento, seja para intervir no desenvolvimento dos fatos, Ele o
faz por meio dos anjos, as "Inteligências separadas" que Ele cria e das
quais Ele se serve para reger o universo. Assim sendo, os anjos são
todas as forças propulsoras e todas as faculdades, tal como a força
formadora, que existe no esperma e que dá for-

MAIMÔNIDES --- VIDA E OBRA 31

mas ao feto, e como a faculdade que faz com que um animal aja de uma
deter­minada maneira, num dado momento, de acordo com os desígnios de
Deus, assim como se vê na citação que Maimônides faz das Escrituras:
"Meu Deus en­viou Seu anjo e fechou a goela dos leões, que não me
fizeram nenhum mal" (Dan. 6:22).

As Escrituras dizem também que os anjos formam um terço do uni­verso.
Isso significa que eles são uma das três coisas criadas por Deus e
existen­tes fora d'Ele, a saber: as Inteligências separadas (ou anjos),
os corpos das esfe­ras celestes, e a matéria que se encontra abaixo das
esferas celestes, e da qual é feito tudo o que existe no nosso mundo. As
Inteligências separadas, elemen­tos incorpóreos criados por Deus são os
intermediários entre Deus e todos os corpos celestes e fonte de
extravasamento de benefícios e de luz para os cor­pos das esferas
celestes. Quanto a estas esferas, Maimônides diz que elas são entidades
que possuem uma alma, no sentido de que elas foram postas em mo­vimento
pelas Inteligências criadas por Deus e que elas desenvolvem o desejo de
mover-se eternamente em círculos, o ponto mais alto da perfeição que um
corpo possa alcançar, já que esse é o movimento perpétuo.

De acordo com Maimônides, as esferas celestes são em número de nove: uma
que engloba tudo, uma onde estão as estrelas fixas, e uma esfera para
cada um dos sete planetas existentes (de acordo com os conhecimentos da
época). Cada uma dessas esferas também emana benefícios e forças que
re­gem a matéria de nosso mundo, "o mundo do nascimento e da corrupção",
como diz Maimônides. Segundo os filósofos, temos a evidência da
influência que a lua exerce sobre as águas e da que os raios do sol
exercem sobre o ele­mento do fogo e do calor. Baseado nisso, ele chega à
conclusão de que cada esfera "pode possuir um dos quatro elementos de
tal forma que tal esfera seja o princípio de força de tal elemento em
particular, ao qual, graças a seu próprio movimento, ela dê o movimento
do nascimento. Assim, portanto, a esfera da lua seria o que move a água;
a esfera do sol o que move o fogo; a esfera dos outros planetas, o que
move o ar (e seus movimentos múltiplos, sua desigual­dade, seu recuo,
sua retidão e sua estação produzem as diversas configurações do ar, sua
variação e sua rápida contração e dilatação); e, finalmente, a esfera
das estrelas fixas seria o que move a terra, e é talvez por isso que
esta última se move com dificuldade, por receber a impressão e a
mistura, porque as estre­las fixas têm o movimento lento".

Ele conclui também que é como se todas essas forças constituíssem a
força de um só corpo, já que o universo todo é um único indivíduo. Tudo
o que nele existe é elaborado não a partir de um ato concreto e
particular, mas sim a partir do que Maimônides chama de "extravasamento
divino", a fonte inesgotável de bondade, de criação e de continuidade do
universo, que se es­palha como numa cascata, derramando-se primeiro
sobre os anjos, que extra­vasam seus benefícios sobre as esferas
celestes as quais, por sua vez, os extrava­sam sobre os corpos
perecíveis. De acordo com ele, "... tal como foi demons­trada a
incorporeidade do Criador, e tal como foi estabelecido que o universo é
obra Sua e que Ele é sua causa eficiente, ... foi dito que o mundo vem
do extravasamento de Deus e que Deus extravasou sobre ele tudo o que
nele ocorre. Da mesma forma, foi dito que Deus extravasou sua ciência
sobre os profetas. Tudo isso significa que essas ações são a obra de um
ser incorpóreo, e a ação de um ser assim é chamada de extravasamento".

Com a mesma clareza que Maimônides utiliza para explicar que o
"ex­travasamento" divino é o ponto de origem e de renovação de todas as
coisas, corpóreas ou incorpóreas, ele escreve também um capítulo no qual
faz a de-

32 MAIMÔNIDES

claração, espantosa a princípio, de que Deus não tem nenhum poder sobre
o impossível. "O impossível", explica ele, "tem uma natureza estável e
constan­te que não é obra de um agente e que não varia sob condição
alguma; é por isso que não se pode atribuir a Deus nenhum poder a esse
respeito". Ele escla­rece, a seguir, que já que as coisas impossíveis
não são obra de um agente e que já que a existência delas é
inadmissível, o fato de que Deus não tenha po­der sobre elas não pode
significar, conseqüentemente, nenhum tipo de fraque­za por parte d'Ele.
As divergências que possam surgir entre os pensadores a es­se respeito
limitam-se à classificação do que eles consideram como possível e como
impossível. Assim, todos concordam em identificar como impossível a
união de contrários no mesmo instante e sobre o mesmo assunto, a
transfor­mação da substância em acidente e do acidente em substância, a
existência de uma substância corporal sem acidente, e o fato de que Deus
crie um ser seme­lhante a si próprio ou que Ele se corporifique, ou
ainda que se transforme. As divergências aparecem quando se trata de
saber, por exemplo, se é possível pro­duzir algo que possua um corpo,
sem para isso servir-se de uma matéria pré-exis­tente. "No entanto", diz
ele, "está claro que, segundo todas as opiniões e to­dos os sistemas, há
coisas impossíveis cuja existência é inadmissível e a respei­to das
quais não se pode atribuir poder a Deus".

Ainda no seu \emph{Guia dos Perplexos} Maimônides faz uma revelação
sur­preendente com relação às bases da idolatria. De acordo com ele, a
idolatria em todos os tempos teve sua origem na personificação dos
astros, feita pelos Sabianos, que acreditavam que os astros eram a
divindade e que o sol era o deus supremo. O próprio Abraão foi educado
dentro dessa religião, e sua opo­sição a ela lhe valeu a prisão, o
confisco de seus bens, e o exílio da Síria. Os Sabianos adoravam os 7
planetas e os 12 signos do Zodíaco, e diziam ainda que Adão era o
apóstolo da Lua, e que Noé foi encarcerado porque não aprovava o culto
dos ídolos e porque se dedicava ao culto de Deus. Assim, eles "ergue­ram
estátuas aos planetas, estátuas de ouro ao sol e estátuas de prata à
lua, e distribuíram os metais e os climas pelos planetas, dizendo que
tal planeta era o deus de tal clima. Eles construíram templos nos quais
colocaram estátuas e afirmaram que as forças dos planetas se derramavam
sobre essas estátuas, de tal forma que elas falavam, compreendiam,
pensavam, inspiravam os homens e lhes davam a conhecer o que lhes era
útil". Eles acreditavam ainda que se uma árvore fosse plantada em nome
de um planeta e consagrada a ele, de acor­do com determinados ritos e
cuidados, a força espiritual desse planeta passava para essa árvore ---
como por exemplo no caso da Ashera e do Baal ---, inspira­va os homens e
lhes falava durante seu sono, dando assim origem aos augúrios, à
feitiçaria, às previsões, à mágica etc.

Por não ser uma ciência completa, a idolatria leva à dúvida e à
su­perstição, transformando aqueles que nela acreditam em vítimas da
ignorância, e sujeitando-os a situações de miséria e de destruição,
tornando-se necessário que o ser humano saiba diferenciar entre os
verdadeiros profetas de Deus e os outros. Foi, portanto, para afastar
esses cultos dos hábitos dos homens que Deus se preocupou em estabelecer
os preceitos relativos à interdição da idolatria, pois "para
aproximar-se do verdadeiro Deus e para se obter a sua benevolência não
se precisa de todas essas práticas penosas, mas... basta amá-Lo e
temê-Lo, duas coisas .que são o verdadeiro objetivo do culto divino".
Assim sendo, não faz nenhum sentido que haja pessoas que imaginem que
seus destinos possam ser regidos por esses astros, a quem se dedicavam
templos e oferendas. Além do mais, se os astros se encarregassem de
dirigir a vida das pessoas, estabelecendo-lhes um destino que variaria
de acordo com a posição deles no momento do

MAIMÔNIDES --- VIDA E OBRA 33

nascimento de cada uma delas, e se elas nada pudessem fazer para
intervir ou mudar isso, a Lei que Deus nos deu não teria nenhuma razão
de ser: tudo já estaria traçado e determinado, de forma definitiva e
inexorável, independente­mente da atitude e do comportamento de cada um,
quer vivesse ele no cami­nho dos justos e dos bons ou na delinqüência e
na depravação. "Nesse caso", escreve ele, "toda recompensa e todo
castigo seriam injustiças manifestas que não poderiam ser permitidas nem
entre nós, nem por parte de Deus com rela­ção a nós".

E no entanto, fica evidente que a lei divina tem um objetivo bem claro:
"Cada qual dos 613 preceitos serve para inculcar as atitudes corretas ou
para eliminar algumas concepções errôneas, para estabelecer uma
legislação justa ou para eliminar injustiças, para nos imbuir de
virtudes exemplares ou para nos dissuadir de inclinações nocivas". O
conjunto dos preceitos está, portanto, li­gado a três coisas: às
opiniões, à moral e à prática dos deveres sociais, e visa fazer com que
o homem possa alcançar a perfeição do corpo e a da alma. A perfeição da
alma, objetivo máximo e superior da existência, através da qual o homem
atinge a "permanência perpétua", a comunhão com Deus, só pode ser, no
entanto, alcançada numa segunda etapa, depois que o bem-estar do cor­po,
embora segundo em importância, tenha sido atingido, já que "é impossível
que o homem, atormentado por uma dor, pela fome, a sede, o calor ou o
frio, compreenda as idéias que se deseja fazê-lo compreender."

A concepção da virtude maimonidiana, situada no termo médio,
equi­distante do excesso e da escassez, e que ele expunha no tratado de
ética que escrevera na juventude, vinha de Aristóteles. Mas ele agora
faz uma exceção a essa média ideal com relação à humildade, sobre a qual
ele diz: "É preciso atingir seu ponto culminante e exercê-la no seu mais
alto grau. Pois na Escritu­ra, toda vez que se fala da grandeza de Deus,
fala-se também de Sua humildade. E Deus louvou a humildade de Moisés,
que possuía todas as qualidades morais e intelectuais e que era o mestre
da doutrina, da ciência e da profecia".

A filologia é uma preocupação constante para Maimônides, em toda a sua
obra. Ele estuda os textos a fundo e compara vários manuscritos do
Talmud antes de tomar uma decisão, pois muitas vezes eles diferem em
pontos essen­ciais. Ele chega até mesmo a conseguir uma cópia do Talmud
do século 7, es­crita em pergaminho, e parece que sua leitura é de
extrema importância para a interpretação dessa obra.

Ele percebe que várias questões religiosas só podiam receber uma solução
com a ajuda da ciência geral, e é o caráter religioso de sua obra que
lhe dá unidade. O equilíbrio de sua alma transparece em seu estilo e ao
longo de suas milhares de frases, cada uma das partes de seus livros se
integra no to­do, cada linha guarda sua medida, cada palavra está em
harmonia com o resto do livro, tanto no valor como na forma.

Ao mesmo tempo em que escreve seu \emph{Guia} ele trabalha num trata­do
de medicina que viria a ser muito difundido, e no qual ele expõe as
teorias fisiológicas, anatômicas, terapêuticas e higiênicas da medicina.
Em 1186, a pe­dido do sultão Al Malik al Muzaffar, Maimônides redige o
\emph{Livro dos Segredos,} onde enumera vários remédios conhecidos
graças aos mais profundos segre­dos da medicina e que fazem parte da
literatura médica e do Talmud, além de algumas fórmulas resultantes de
suas próprias pesquisas. escreve ainda, a pedi­do de Al Fadil,
grão-vizir de Saladin, um tratado conciso de primeiros socorros para os
casos de envenenamento, que adquire uma grande autoridade nos meios
competentes e que é freqüentemente citado pelos médicos durante toda a
Ida­de Média.

34 MAIMÔNIDES

No ano de 1187 Al Fadil, nomeia Maimônides médico da corte. Co­mo
conseqüência desse emprego e da consideração que o cerca ele recebe
pouco depois o título de Naguid. A alta posição política que ele adquire
na qualidade de chefe das comunidades judias, e o respeito que ele havia
adquirido graças a sua personalidade, permitem-lhe defender os judeus
nas diversas regiões do reino. Um de seus primeiros decretos, ao assumir
esse posto, é o de proclamar que nas cidades egípcias apenas os juízes
expressamente qualificados podem promulgar casamentos e divórcios. Esse
decreto visa uma supervisão geral dos registros do estado civil e uma
proteção à mulher, para evitar que os emigran­tes, já casados em seus
países de origem, tornem a casar-se no Egito. Seu novo cargo vai
ajudá-lo a combater a influência dos caraítas de forma decisiva e a
obter sucesso na reforma dos ritos, que ele havia iniciado anos antes.
Ele consi­dera, no entanto, que o dever de guiar os judeus é muito mais
uma carga exte­nuante do que um benefício, pois ele tem consciência de
quanto é fácil ser ca­luniado e mal interpretado.

Sua situação como médico da corte o obriga a passar o dia inteiro no
Cairo, mas essa grande fama não lhe traz nenhum prazer: ele se sente
depri­mido, e o cansaço físico provocado por seu trabalho na corte
durante o dia e como médico e como Naguid para os que o procuram em sua
casa à noite e nos fins-de-semana, o impedem de encontrar tempo para
continuar dedican­do-se aos estudos. Esse cansaço acaba por provocar uma
doença que o deixa de cama durante todo um ano e da qual ele nunca se
recupera totalmente, pois seu organismo fica debilitado e minado.

Após o \emph{Guia dos Perplexos} ele não escreve mais nenhuma obra de
caráter religioso, a não ser alguns conselhos e cartas e uma dissertação
para ex­plicar os três princípios fundamentais da metafísica da época: a
existência de Deus, as relações da origem do mundo com relação a Deus, e
a eternidade ou a criação do mundo. Para curar os doentes ele renuncia
ao seu projeto de es­crever um livro sobre a Hagada, que deveria colocar
em evidência a filosofia do judaísmo, e assim legitimar suas próprias
teorias. Renunciando até mesmo a traduzir para o hebraico os livros que
havia escrito em árabe, a acabar seus comentários sobre o Talmud,
iniciados em sua juventude, e a redigir o livro de referências de que
dependia o futuro de seu \emph{Mishneh Torah,} obra que tan­tas
controvérsias havia provocado após sua publicação, ele consagra seus
últi­mos anos de vida inteiramente à medicina, e é nomeado, em 1198,
médico par­ticular do sultão Al Afdal, sucessor de Al Aziz no trono de
Damasco e da Síria, que reinou depois da morte de Saladin, em 1195.
Portanto, no ponto culmi­nante de sua vida, Maimônides se dirige ao povo
é essa e sua última transforma­ção: ele passa da contemplação à•prática,
da metafísica à medicina. Renuncian­do ao isolamento, Maimônides
consegue agora falar com as outras pessoas sem por isso deixar de pensar
em Deus e de ter seu coração aberto a Ele. Ele não mais sente a
necessidade de pensar nas coisas sagradas para se sentir próximo de
Deus. A superioridade da profecia sobre a filosofia torna-se-lhe clara:
"a pro­fecia não é demonstrada por nenhuma prova, toda tentativa de
exame científi­co deve ser posta de lado. Seria como querer colocar
todas as coisas do mundo num pequeno vaso", escreve ele em seus últimos
anos de vida.

Mas o "corpus" de sua doutrina logo começa a ser sacrificado por aqueles
que, não se contentando em ler seus livros, decidem criticá-los, sem no
entanto ter compreendido corretamente seu significado,. A questão
relativa à redenção dos corpos, por exemplo, foi mal compreendida e os
yemenitas en­viam ao Gaon de Bagdad uma carta onde se queixam de que a
ressurreição dos mortos, tal como era concebida pelo povo, havia sido
negada na obra de

MAIMÔNIDES --- VIDA E OBRA 35

Maimônides. Seus adversários conseguem o apoio do poderoso Gaon Samuel
ben Ali. A notícia de que o Gaon, representante legítimo da tradição
judaica, havia tomado partido contra Maimônides e de que o havia
condenado e prova­do claramente que ele negava a redenção, espalha-se
pelo mundo judeu. Para desfazer esse mal-entendido Maimônides se vê
obrigado a redigir um \emph{"Tratado sobre a Ressurreição",} onde ele
diz: "... eu julguei incorreto estudar apenas os galhos da doutrina e
negligenciar as raízes. E por isso que discuti os princí­pios
fundamentais da fé... Mas como a compreensão das provas dessas
doutri­nas fundamentais pressupõe o conhecimento de muitas ciências, eu
apenas apre­sentei os meios de prová-las, sem citá-los... Eu expressei a
opinião que deve­mos imaginar o mundo futuro sem relação alguma com a
redenção, mas ressal­tei expressamente que a reencarnação dos mortos é
um pilar fundamental da doutrina religiosa. ... Mas dizer que afirmei
que a alma não retorna jamais a seu corpo é uma difamação, porque essa
negação constituiria a negação dos pró­prios milagres e equivaleria ao
repúdio da religião." Uma vez mais, Maimôni­des se vê forçado a fazer
aqui uma conceção dogmática, mas o faz provavel­mente apoiado em estudos
metafísicos que lhe permitem conciliar seus conhe­cimentos místicos com
o que foi estabelecido pela tradição.

Contudo, se Maimônides encontra oposição e controvérsia por par­te de
alguns, ele é admirado e respeitado por muitos outros, como pelos
estu­diosos que vivem nos países cristãos, onde não há nem Gaon nem
Exilarca, descendente do rei Davi no exílio, onde a autoridade das
escolas é fraca em relação a da academia de Bagdad, e onde se fazem
verdadeiras pesquisas. Esse é o caso da região de Provença, na França,
onde florescem a ciência talmúdica e a especulação filosófica, e onde as
obras de Maimônides produzem uma sen­sação sem precedentes na história
dos judeus junto a um grupo de sábios res­peitados, chamados de "Sábios
de Lunel". Em 1195 chega a Fostat um tratado redigido em hebraico por
esses sábios, cujo porta-voz se chama Jonathan Co­hen, no qual se
cantavam as glórias de Maimônides e se fazia uma declaração de admiração
e fidelidade àquele que eles consideram como o maior mestre que havia
aparecido desde a conclusão do Talmud.

Por volta de 1201 chegam da região do Midi da França cartas com várias
assinaturas, pedindo a Maimônides para traduzir ele próprio para o
he­braico o seu \emph{"Guia dos Perplexos",} mas ele se vê forçado a
recusar, pois não tem mais tempo nem sequer para concluir os livros que
havia iniciado. A tradu­ção é então entregue a Samuel Ibn Tibbon, que
trabalha nela com afinco, pois deseja submetê-la a Maimônides para
verificação e revisão, porém infelizmente ele não consegue terminá-la em
tempo.

Maimônides tem a firme convicção de que a imortalidade é a vida eterna
do espírito que tem conhecimentos, mas "as almas que sobrevivem após a
morte não são a mesma coisa que a alma que nasce com o homem no mo­mento
de seu nascimento; pois aquela que nasce ao mesmo tempo que ele é apenas
algo em potencial e uma disposição, enquanto que a coisa que fica em
separado depois da morte é aquilo em que ela se transformou". Portanto,
a imor­talidade depende da quantidade de conhecimentos adquiridos. A
morte é acon­tecimento mais importante para o sábio que adquiriu o
conhecimento de Deus, pois a compreensão da inteligência se fortifica no
momento em que ela se sepa­ra do corpo, já que "Há um limite para a
compreensão humana: enquanto a alma estiver no corpo ela não poderá
compreender o sobrenatural... A matéria é um grande véu". Uma vez
atingida a separação, essa inteligência fica para sem­pre nesse estado
de plenitude, gozando continuamente dessa que é a verdadei­ra e a grande
felicidade.

36 MAIMÔNIDES

Essa importante separação chega para Maimônides na noite
de\textsuperscript{.} 13 de dezembro de 1204. Ele é kvado, de acordo com
seu desejo, para Tiberíades na Terra Santa, e ali é enterrado num bosque
próximo ao local onde descan­sam os grandes talmudistas dos séculos II,
III e IV, e onde seu antepassado, o rabino Yehuda Hanassi, havia estado
por diversas vezes.

Nas paredes de mármore que circundam sua tumba, vários seguido­res,
estudiosos e amigos deixaram gravados seus testemunhos de admiração e de
respeito pelo grande sábio, um dos quais diz o seguinte:

"Aqui jaz um homem --- e, no entanto, ele não foi um homem; Se fostes um
homem, então foram os seres celestes que te criaram".

\textbf{OS 14}
\end{quote}

\textbf{FUNDAMENTOS}

\begin{quote}
Começarei agora a mencionar os Fundamentos --- em número de qua­torze
--- que nos guiarão na enumeração dos preceitos. Começarei dizendo que a
soma total dos preceitos que nos foram ordenados por Deus, conforme
cons­ta no Rolo da Torah, é de seiscentos e treze. Deles, duzentos e
quarenta e oito correspondem ao número de membros do corpo humano e são
preceitos posi­tivos, e trezentos e sessenta e cinco correspondem aos
dias de um ano solar e são preceitos negativos. Este número está
mencionado no texto do Talmud, no final do Tratado Macot, onde está
dito: "Seiscentos e treze preceitos foram ditos a Moisés no Sinai:
trezentos e sessenta e cinco correspondendo aos dias de um ano solar e
duzentos e quarenta e oito correspondendo aos membros do corpo humano".
A título de "derash"\textsuperscript{1} eles disseram também, com
relação ao fato de que os preceitos positivos correspondem ao número de
membros, que é como se cada membro dissesse à pessoa: "Cumpra um
preceito comi­go"; e com relação ao fato de que os preceitos negativos
correspondem ao nú­mero de dias no ano solar, eles disseram que é como
se cada dia dissesse à pes­soa "Não cometa uma transgressão hoje". O
fato de que eles constituem o nú­mero dos preceitos não passou
desapercebido a nenhum dos que se empenha­ram na enumeração dos
preceitos; mas no processo da enumeração em si eles contaram assuntos
que são produtos de imaginação sem base, como será expli­cado neste
trabalho. Isto deveu-se ao fato de que eles desconheciam estes qua­torze
Fundamentos, os quais passarei a explicar.

O Primeiro \textbf{Fundamento:} Não se deve incluir nesta enumeração os
preceitos adicionais de autoria rabínica.

\textbf{O Segundo Fundamento:} Não devemos incluir nesta enumeração o
que se po­de deduzir das Escrituras por meio de qualquer um dos treze
princípios exegé­ticos, pelos quais se interpreta a Torah, ou por meio
de inclusão.

38 MAIMÔNIDES

\textbf{O Terceiro Fundamento:} Não devemos incluir nesta enumeração
preceitos que não sejam obrigatórios por todos os tempos.

\textbf{O Quarto Fundamento:} Não se deve incluir nesta enumeração
ordens que abran­jam todos os preceitos da Torah.
\end{quote}

\begin{itemize}
\item
  \begin{quote}
  \textbf{Quinto Fundamento:} A razão dada para um preceito não deve ser
  contada como um preceito separado.
  \end{quote}
\end{itemize}

\begin{quote}
\textbf{O Sexto Fundamento:} Quando um preceito contém tanto uma ordem
positiva quanto uma negativa, cada uma das partes deve ser contada
separadamente, uma entre os preceitos positivos e a outra entre os
negativos.
\end{quote}

\begin{itemize}
\item
  \begin{quote}
  \textbf{Sétimo Fundamento:} Não se deve contar as leis detalhadas de
  um preceito.
  \end{quote}
\item
  \begin{quote}
  \textbf{Oitavo Fundamento:} Não se deve incluir entre os preceitos
  negativos uma declaração negativa que exclua um caso particular de um
  determinado assunto.
  \end{quote}
\item
  \begin{quote}
  \textbf{Nono Fundamento:} Esta enumeração não deve ser baseada no
  número de vezes que um determinado preceito negativo ou positivo está
  repetido nas Es­crituras, mas sim na Natureza da ação proibida ou
  ordenada.
  \end{quote}
\end{itemize}

\begin{quote}
\textbf{O Décimo Fundamento:} Não se deve contar os atos estipulados
como prelimi­nares ao cumprimento do preceito.

\textbf{O Décimo Primeiro Fundamento:} Não se deve contar separadamente
os diver­sos elementos que compõem um só preceito.

\textbf{O Décimo Segundo Fundamento:} Não se deve contar separadamente
as etapas sucessivas na execução de um preceito.

\textbf{O Décimo Terceiro Fundamento:} Quando um determinado preceito
tiver que ser cumprido por vários dias não se deve contar um preceito
por cada dia.
\end{quote}

\begin{itemize}
\item
  \begin{quote}
  \textbf{Décimo Quarto Fundamento:} De que forma os tipos de castigo
  devem ser contados como preceitos positivos.
  \end{quote}
\end{itemize}

\begin{quote}
E agora voltarei a explicar cada um dos fundamentos e trazer provas
sobre eles, se Deus quiser.

OS 14 FUNDAMENTOS 39

\textbf{O PRIMEIRO}

\textbf{FUNDAMENTO}

NÃO SE DEVE INCLUIR NESTA ENUMERAÇÃO OS PRECEITOS ADICIONAIS DE AUTORIA
RABÍNICA

Vocês devem saber que não deveria ter sido necessário comentar este
assunto, pois ele está perfeitamente claro. Se o texto do Talmud diz que
"Seiscen­tos e treze preceitos foram ditos a Moisés no Sinai", como
seria possível dizer que algo que vem dos Rabinos se inclui nesta
enumeração? Mas fomos forçados a comentá-lo, pois outros já se enganaram
a este respeito e contaram --- como pre­ceito --- a luz de "Hanucá" e a
leitura do Rolo (de Esther) entre os preceitos positi­vos. Também o
recitar de cem bençãos diariamente, o consolo aos que estão de luto, a
visita aos doentes, o enterro dos mortos, o vestir os que estão
despidos, o cálculo das estações, e os dezoito dias nos quais
completamos o "Halel".

De fato, deve-se olhar com espanto para quem ouve as palavras "Fo­ram
ditos a Moisés no Sinai" e ainda assim conta a leitura do "Halel", com a
qual Davi exaltou ao Eterno, enaltecido seja Ele, como se ela tivesse
sido orde­nada a Moisés e ainda conta a luz de "Hanucá", que os Sábios
estabeleceram na época do Segundo Templo, e a leitura do Rolo (de
Esther)! Contudo, não creio que haja alguém que pudesse imaginar ou
ousasse supor que foi dito a Moisés no Sinai que ele deveria nos ordenar
acender a luz de "Hanucá", se no final de nossa soberania um determinado
acontecimento relacionado com os gregos ocorresse entre nós.

Parece-me, no entanto, que o que os levou a errar e a enganar-se a esse
respeito é que, ao recitar a bênção, dizemos: "que nos santificasteis
atra­vés de Vossos preceitos e nos ordenasteis com relação à leitura do
Rolo", ou "a acender a luz `Hanucá\textsuperscript{---}, ou "a completar
o Além disso, o Talmud pergunta: "Onde nos foi ordenado"? E eles
responderam: "Em Suas palavras `Não te desviarás' (Deuteronômio 17:11)".

Mas se foi essa a razão pela qual contaram dessa maneira, eles tam­bém
deveriam ter contado tudo o que foi imposto pelos Mestres, pois já havia
sido ordenado a nosso mestre Moisés, no Sinai, que nos obrigasse a
executar tudo o que os Sábios nos ordenaram ou proibiram, por Suas
palavras: "Confor­me o mandamento da lei que te ensinarem, e conforme o
juízo que te disse­rem, farás" (Deuteronômio 17:11); também mais adiante
Ele nos proibiu de desobedecê-los em tudo o que decretassem e
estabelecessem, ao dizer: "Não te desviarás da sentença que te
anunciarem, nem para a direita nem para a es­querda" (Ibid.). Mas se
fosse correto contar entre os 613 preceitos tudo o que está revestido de
autoridade rabínica --- levando-se em consideração que se in­clui em
palavras, enaltecido seja Ele, "Não te desviarás da sentença" e
"Con­forme o m. damento da lei que te ensinarem, farás" --- por que
foram estes enfa zados? Assim como contaram a luz de "Hanucá" e a
leitura do Rolo (de Est r) les deveriam ter contado a lavagem das mãos e
o preceito de "er i b"\textsuperscript{2}, po' em relação a isso
recitamos as bênçãos "que nos santificasteis

40 MAIMÔNIDES

através de Vossos preceitos e nos ordenasteis com relação à lavagem das
mãos" ou "com relação ao •receito de 'erub\textsuperscript{---}, da
mesma forma que recitamos a bên­ção "com relaç à le ura do Rolo" ou "com
relação a acender a luz de 'Hanu­cá"'. E tudo iss i vem
\textbf{\textsuperscript{4}e} lei rabínica! Eles dizem explicitamente:
"As primeiras águas são prec= tosa. •mo? Abayé disse: É preceito
obedecer as palavras dos Sábios". Isto é e ante ao que dizem com relação
à leitura do Rolo (de Es­ther) e à luz de anucá': "Onde nos foi ordenado
isso? Nas palavras 'Não te desviarás"'. Também foi deixado claro que
tudo aquilo que ordenaram os pro­fetas --- a paz esteja com eles --- que
vieram depois de Moisés, nosso mestre, também é de autoridade rabínica.
Assim, eles dizem expressamente: "Quando Salomão estabeleceu o 'erubin'
e as mãos\textsuperscript{3}, uma voz divina apareceu e disse: `Meu
filho, seja sábio e alegre Meu coração\textsuperscript{---}. Em outros
trechos eles explica­ram que o 'erubin' é lei rabínica e que as
mãos\textsuperscript{3} é decreto dos Escribas. Assim, foi explicado que
tudo o que foi ordenado depois de Moisés, nosso mestre, é chamado de
"lei rabínica".

Eu lhes expliquei isso a fim de que não pensem que pelo fato de ter sido
ordenada pelos profetas, a leitura do Rolo (de Esther) deve ser
considera­da "lei das Escrituras", pois o "erubin" é chamado de "lei
rabínica", embora tenha sido ordenado por Salomão, o filho de Davi, e
seu Tribunal.

Foi isso o que confundiu alguém, que por esse motivo contou vestir os
despidos, pois ele encontrou em Isaías o seguinte: "Quando vires um
despi­do, o cobrirás". Mas ele não sabia que isto está incluído em Suas
palavras "E lhe emprestarás o suficiente para o que lhe faltar"
(Deuteronômio 15:8). Está sem dúvida alguma claro que o significado
deste preceito é que devemos ali­mentar os famintos, vestir os despidos,
dar um colchão e um cobertor a quem não os tiver, ajudar a casar-se
aquele que não tiver meios para fazê-lo e provi­denciar urna montaria
para quem não a tiver, pois sabe-se, pelo texto do Tal­mud, que tudo
isso está incluído em Suas palavras globais "O suficiente para o que lhe
faltar".

Parece, contudo, que na opinião dessas pessoas a linguagem do Tal­mud
foi composta "com lábios vacilantes e com língua estranha", caso
contrá­rio eles não teriam contado a leitura do Rolo (de Esther) e
similares como pre­ceitos ditos a Moisés no Sinai!
\end{quote}

A Guemará de Sheb ot diz. "Eu só tenho conhecimento dos precei-

\begin{quote}
tos que foram ordenados no S que forma fico sabendo dos que foram

destinados a ser estabelecidos ovos artigos, tais como a leitura do Rolo

(de Esther)? Pelas palavras das s uras: 'Os judeus cumpriram e
assumiram' (Esther 9:27) --- eles cumpriram aquilo que já haviam
assumido". Isso significa que eles aceitaram crer em todos os preceitos
que os profetas e os Sábios orde­nassem depois.

Eu estou surpreso. Por que eles contaram preceitos positivos de
au­toridade rabínica, como os que mencionamos, mas não contaram também
pre­ceitos negativos de autoridade rabínica? Assim como eles contaram
entre os pre­ceitos positivos a lei de "Hanuc '" itura do Rolo (de
Esther), as cem bên­çãos e o "Halel", eles deveriam gualm. te ter
contado entre os preceitos ne­gativos os vinte casos de "She o vinte
preceitos negativos. Pois as-
\end{quote}

\begin{enumerate}
\def\labelenumi{\arabic{enumi}.}
\setcounter{enumi}{2}
\item
  \begin{quote}
  A lavagem das mãos antes da refeição .osição às "águas finais", que
  são a lavagem das\\
  mãos após a refeição).
  \end{quote}
\item
  \begin{quote}
  Dos que o povo de "Israel" recebeu no monte Sinai.
  \end{quote}
\item
  \begin{quote}
  Incesto de segundo grau, que foram proibidos por autoridade rabínica.
  \end{quote}
\end{enumerate}

\begin{quote}
OS 14 FUNDAMENTOS 41

sim como toda e qualquer relação proibida pela Torah constitui um
preceito negativo de lei das Escrituras, assim também cada relação
proibida pelos Mes­tres constitui um preceito negativo de autoridade
rabínica. Eis exatamente o que os Sábios dizem: "As
Sheniyoe\textsuperscript{5} foram estabelecidas pelos Escribas". E es­tá
também explicado no Talmud que as palavras do Mishná, "proibida por um
preceito" se referem às "Sheniyot". A esse respeito eles comentaram:
"Que proi­bição por um preceito? O preceito que ordena obedecer as
palavras dos Sá­bios". Da mesma forma, eles deveriam ter contado nesta
enumeração "a irmã de uma mulher (na qual foi feita a)
`Halitzá\textsuperscript{---}, que é proibido por decreto dos Escribas!

Resumindo, se devêssemos contar todos os preceitos positivos e
ne­gativos impostos pelos Rabinos, o número chegaria a muitos milhares.

Isto está, sem dúvida alguma, claro. Tudo o que for de autoridade
rabínica não deve ser contado na soma total dos 613 preceitos, uma vez
que esses estão todos baseados em versículos da Torah, não havendo entre
eles na­da que seja de autoridade rabínica, como explicamos. Mas o fato
de contar al­guns preceitos de autoridade rabínica e deixar
arbitrariamente outros de fora não pode ser aceito em hipótese alguma,
seja quem for seu autor.

Explicamos, assim, o teor deste fundamento e sua prova, para que ninguém
venha a ter nem sombra de dúvida a este respeito.

\textbf{O SEGUNDO}

\textbf{FUNDAMENTO}

NÃO DEVEMOS INCLUIR NESTA ENUMERAÇÃO O QUE SE PODE DEDUZIR DAS
ESCRITURAS POR MEIO DE QUALQUER UM DOS TREZE PRINCÍPIOS EXEGÉTICOS,
PELOS QUAIS SE INTERPRETA A TORAH,

OU POR MEIO DE INCLUSÃO

Nós já explicamos no início de nosso \emph{Comentário sobre a Mishná}
que a maioria das leis da Torah são deduzidas por meio dos treze
princípios exegéticos pelos quais se interpreta a Torah, e que uma lei
deduzida dessa for­ma está algumas vezes sujeita a uma diferença de
opiniões. Também explica­mos ali que há leis que são interpretações
tradicionais recebidas de nosso mes­tre Moisés, e portanto não sujeitas
a diferenças de opinião, mas que foram pro­vadas por meio de algum
desses treze princípios. Na realidade, a sabedoria das Escrituras é tal
que é possível encontrar nelas um indício ou uma semelhança que nos
conduza à interpretação recebida.

Assim sendo, conclui-se que nem toda lei deduzida pelos Sábios por meio
de um dos treze princípios pode ser declarada como tendo sido dita a
Moisés no Sinai; da mesma forma, não devemos concluir que pelo fato de
os Sábios do Talmud encontrarem respaldo para uma determinada lei num
dos treze princípios, ela será de autoridade rabínica, visto que é
possível que uma

42 MAIMÔNIDES

determinada lei seja uma interpretação recebida.

Resumindo: toda lei que não estiver explicitamente enunciada na To­rah,
mas que tenha sido deduzida do Talmud por meio de um dos treze
princí­pios, deve ser contada se aqueles que receberam a Tradição
declararem expli­citamente que "ela pertence ao corpo da Torah" ou que
"ela é lei da Torah". Mas se eles não disserem ou explicarem claramente
que é assim, então ela é uma lei de autoridade rabínica, uma vez que não
há nenhum versículo que a indique diretamente.

Também com relação a este fundamento um outro se enganou e con­tou o
temor aos Sábios entre os preceitos positivos.

Quer-me parecer que ele o fez por causa das palavras de Rabi Akiba: "
'Ao Eterno, teu Deus, temerás' (Deuteronômio 6:13) inclui os sábios". E
eles pensaram que tudo o que se deduz pela Inclusão é semelhante aquilo
que lhe deu origem.

Mas se fosse correto o que eles pensaram, então por que eles não
contaram também o dever de honrar o padrasto e a madrasta e o irmão mais
velho, além dos pais --- a quem temos o dever de honrar conforme o que
foi determinado através do princípio de Inclusão? (Como os sábios
disseram, " 'A teu pai' (Êxodo 20:21) inclui teu irmão mais velho assim
como teu padrasto. 'E a tua mãe' inclui tua madrasta"). Isso é análogo
ao que os Sábios disseram: " 'Ao eterno, teu Deus, temerás' inclui os
sábios". Então por que eles conta­ram este último e não o anterior?

O fato de não terem o conhecimento necessário já os levou a come­ter um
erro maior do que esse: ao encontrar uma interpretação de algum
versí­culo que exigisse a execução ou a proibição de um determinado ato
--- obriga­ções que são sem dúvida impostas por lei rabínica --- eles as
contaram entre os preceitos, ainda que o significado literal do
versículo não indicasse de ma­neira alguma aquelas obrigações. Isso
contraria o princípio que eles, abençoa­da seja sua memória, nos
ensinaram: "Um versículo da Torah não perde nunca seu sentido literal".
Também contraria o processo de raciocínio encontrado em todo o Talmud, e
que está demonstrado pelo fato de que quando os Sábios falam dè um
versículo, que dá origem a tópicos derivados por meio de inter­pretação
e de várias provas, eles perguntam: "Mas de que trata o versículo em
si?"

Mas eles baseados em comparações sem fundamento, contam entre os
preceitos positivos a visita aos doentes, o consolo aos que estão de
luto, e o enterro dos mortos, tudo isso por causa da seguinte
interpretação, mencio­nada com relação às Suas palavras, enaltecido seja
Ele, "E fa-lo-ás saber o cami­nho por onde andarão, e a obra que farão"
(Êxodo 18:20): " 'O caminho' se refere aos atos de bondade; 'andarão' se
refere a visitar os doentes; 'por onde' se refere a enterrar os mortos;
'e a obra' se refere às leis; 'que farão' se refere ao que ultrapassa o
estritamente requerido pela lei". Eles pensaram que cada uma das
obrigações mencionadas constitui um preceito em si, mas eles não sa­biam
que todas essas obrigações, bem como outras semelhantes, estão
incluí­das nos termos de um dos preceitos explicitamente enunciado na
Torah, que é o que está expresso em Suas palavras, enaltecido seja Ele,
"Amarás o teu pró­ximo como a ti mesmo" (Levítico 19:18). De maneira
semelhante eles conta­ram o cálculo das estações do ano como um
preceito, baseados na seguinte in­terpretação dada pelos Sábios ao
versículo "Porque isto é a vossa sabedoria e o vosso entendimento à
vista dos povos" (Deuteronômio 4:6): "Qual é a sabe­doria e qual o
entendimento que está à vista dos povos? Devo dizer que é o cálculo das
estações e das constelações".
\end{quote}

Se eles tivessem contado questões ainda mais claras do que esta, e

\begin{quote}
OS 14 FUNDAMENTOS 43

portanto mais fáceis de supor que é correto enumerá-las --- ou seja, as
leis de­duzidas através de um dos treze princípios de interpretação da
Torah --- o nú­mero de preceitos atingiria vários milhares! Talvez vocês
possam pensar que nós desistimos de contá-las porque elas não são
suficientemente claras, ou por­que haja dúvidas quanto ao fato de
determinada lei deduzida através daquele princípio estar correta ou não,
mas a razão não é essa. O motivo pelo qual não as contamos é que tudo o
que se deduzir dessa forma são ramificações das raí­zes que foram
explicitamente declaradas a Moisés no Sinai e que constituem os 613
preceitos.

Mesmo que tenha sido o próprio Moisés quem as deduziu, elas não devem
ser contadas. A prova de tudo isto é que eles disseram na Guemará de
Temurá: "Mil e setecentas deduções de menor a maior., analogias de
frases e aspectos.especiais nos decretos dos Escribas foram esquecidos
durante os dias de luto por Moisés. Contudo Ataniel, filho de Kenaz, os
reconstituiu com seu raciocínio, como foi dito: 'E Caleb disse: Aquele
que ferir a "Kiriat Sefer" e a tomar... E Ataniel, o filho de Kenaz a
tomou' ". E esse número já era tão gran­de, quantas então não seriam as
leis originais aprendidas através de Moisés! Pois é inconcebível que se
tenha esquecido tudo o que foi aprendido. Sendo assim não há dúvida de
que as leis aprendidas por dedução de menor a maior, ou por algum dos
outros princípios, chegavam a vários milhares e que todas elas já eram
conhecidas nos dias de Moisés, nosso mestre, posto que foram esqueci­das
nos dias de luto.

Assim, foi-lhes explicado que mesmo no tempo de Moisés já se fala­va de
"aspectos especiais nos decretos dos Escribas", pois tudo o que eles não
ouviram explicitamente no Sinai é considerado como "decreto dos
Escribas". Da mesma forma, foi explicado que mesmo na época de Moisés, a
paz esteja com ele, não se contava entre os 613 preceitos ditos a ele no
Sinai nenhuma lei deduzida através dos treze princípios, e que nós
certamente não devemos contar o que tenha sido deduzido num período
posterior. Em vez disso, deve­mos contar o que constitui uma
interpretação formulada em seu nome, desde que os guardiães da Tradição
nos digam claramente que um ato específico nos foi proibido e que essa
proibição é lei da Torah, ou que eles nos digam que "ela é parte da
própria Torah". Nesse caso a contaremos, pois a aprendemos pela Tradição
e não por um dos treze princípios. Em tais casos a referência feita a um
dos treze princípios ou a apresentação de provas por meio deles serve
apenas para demonstrar a sabedoria contida na Torah, como explicamos no
Co­mentário sobre a Mishná.

\textbf{O TERCEIRO}

\textbf{FUNDAMENTO}

\textbf{NÃO DEVEMOS INCLUIR NESTA ENUMERAÇÃO PRECEITOS QUE NÃO SEJAM
OBRIGATÓRIOS POR TODOS OS TEMPOS}

Você deve saber que as palavras "Os 613 preceitos foram declara­dos a
Moisés no Sinai" ensinam que este número constitui a quantidade de pre-

44 \textbf{MAIMÔNIDES}

ceitos obrigatórios por todos os tempos, isso porque os que assim não
forem não têm relação específica com o Sinai, quer tenham eles sido ali
proclamados ou não. A expressão "no Sinai" significa apenas a Revelação
essencial da Torah, que ocorreu no Sinai. Isto está expresso em suas
palavras, enaltecido seja Ele, "Sobe a Mim, ao monte, e fica ali; e
dar-te-ei..." (Êxodo 24:12). E eles disseram expressamente: "Onde nas
Escrituras está dito que os 613 preceitos foram de­clarados a Moisés no
Sinai? No versículo 'E a Lei que nos ordenou Moisés, heran­ça e...'
(Deuteronômio 33:4). Quer dizer, ele nos ordenou a soma das
letras-nú­meros TORAH, que totaliza 611. Eles ouviram o 'Eu sou o
Eterno, teu Deus' (Êxo­do 20:2) e o 'Não terás outros deuses diante de
Mim' (Ibid., 3) do próprio Todo Poderoso". Com mais esses dois preceitos
se completa o número 613.

O propósito desta rubrica é demonstrar que a Palavra que nos foi
ordenada por Moisés e que nós ouvimos apenas dele é a soma das
letras-núme­ros da palavra TORAH, e que é isso o que Ele declarou ser a
"Herança para a congregação de Jacob." (Deuteronômio 33:4). Um preceito
que não seja obri­gatório por todos os tempos não é "uma herança" para
nós, pois "uma heran­ça" é apenas aquilo que perdura para sempre, assim
como foi dito: "Por todos os dias que os céus estiverem sobre a terra"
(Ibid., 11:21). Assim também a afir­mação deles, de que é como se cada
um dos membros de uma pessoa lhe orde­nasse cumprir um preceito e cada
dia do ano o aconselhasse a não cometer uma transgressão, é uma prova de
que este número nunca vai diminuir. Mas se os preceitos que não são
obrigatórios para sempre devessem ser incluídos nesta enumeração, esse
número diminuiria toda vez que um determinado preceito, ao alcançar seu
objetivo, tivesse sido completamente cumprido, e assim aquela afirmação
teria sido correta apenas durante um determinado momento.

Uma vez mais errou o outro com relação a este fundamento, e con­tou ---
ao se ver pressionado --- "E não entrarão para ver, quando cobrirem os
objetos da santidade" (Números 4:20) e "Não farão mais o trabalho de
carre­gamento" (Ibid., 8:25), relativo aos levitas. Mas esses preceitos
eram obrigató­rios apenas no deserto, e não para sempre. Embora digam:
"Há uma insinuação contra roubar um vaso sagrado em 'E não entrarás para
ver"', o termo "insi­nuação" já é prova suficiente para indicar que este
não é o significado literal do versículo; tampouco se inclui esta
transgressão entre as passíveis de morte pelas mãos dos Céus, como foi
explicado na Tosseftá e em Sanhedrin.

De fato, surpreendo-me com quem contou essas proibições. Por que eles
não contaram o versículo relacionado com o maná, "Ninguém deixe so­brar
dele até a manhã" (Êxodo 16:19) assim como o versículo "Não molestes a
Moab, e não faças a ele guerra" (Deuteronômio 2:9), e o versículo
relativo a Amon, "Não os molestes, e não combatas com eles" (Ibid., 19)?
Eles também deveriam ter contado entre os preceitos positivos os
versículos "Faze para ti uma serpente abrasadora e põe-na sobre uma
haste" (Números 21:8) e "Toma um vaso, põe nele a quantia de um 'omer'
de maná\textsuperscript{-} (Êxodo 16:33) da mesma forma como contaram "a
oferenda de elevação do tributo" e a dedicação do altar. E também
deveriam ter contado "Estejam prontos para o terceiro dia" (Êxodo
19:15), bem como "Tampouco o rebanho, o gado, aparecerão" (Ibid., 34:3),
"Que não transpassem o termo para subir ao Eterno" (Ibid., 19:24), e
muitos outros versículos semelhantes.

Nenhum ser racional duvidará que todos esses preceitos --- positi­vos e
negativos --- foram de fato ditos a Moisés no Sinai, só que eles foram
apli­cáveis durante um determinado período de tempo e não são
obrigatórios para sempre, e por isso não devem ser incluídos.
\end{quote}

De acordo com este fundamento não devemos contar nem as Bên-

\begin{quote}
OS 14 FUNDAMENTOS 45

çãos e as Maldições que lhes foram ordenadas no Gerizim e no Ebal, nem a
edi­ficação do altar que nos foi ordenado construir quando entrássemos
na terra de Canaã, porque esses eram todos preceitos aplicáveis a um
determinado pe­ríodo de tempo.

Tampouco devemos contar o preceito positivo de que se desejásse­mos
comer carne de algum animal só poderíamos fazê-lo depois de levá-la
co­mo oferenda de pazes, porque isso foi um decreto especificamente
aplicável no deserto, como aparece em Suas palavras "Os trarão ao
Eterno" (Levítico 17:5), sobre as quais a Sifrá comenta: " 'E os trarão'
constitui um preceito positivo" --- mas obrigatório apenas no deserto,
pois Ele explicou no Deuteronômio a permissão perene de comer uma
refeição de carne ao dizer: "Com todo o dese­jo de tua alma poderás
comer carne" (Deuteronômio 12:20).

Se fosse necessário contar tudo o que está nesta categoria, todos os
preceitos or na. os a Moisés desde o dia em que ele se tornou profeta
até o dia de sua orte incluindo tudo o que lhe foi ordenado no Egito,
durante a Consagr ão\textsuperscript{6}, e outros preceitos além desses,
todos contidos na Torah, al­guns posit v o s utros negativos ---
teríamos mais de 300 preceitos, além dos que são vigentes por todos os
tempos. Mas como é impossível contá-los todos, fatalmente não se deve
contar nenhum, e não fazer como fizeram outros, usan­do apenas alguns
deles para assim completar o número que não conseguiram atingir.

Isto é o que desejávamos alcançar com este Fundamento.

\textbf{O QUARTO}

\textbf{FUNDAMENTO}

NÃO SE DEVE INCLUIR NESTA ENUMERAÇÃO ORDENS QUE ABRANJAM TODOS OS
PRECEITOS DA TORAH

Há preceitos positivos e negativos na Torah que não se referem a uma
obrigação concreta, mas incluem todos os preceitos, como se o Eterno,
enaltecido seja Ele, estivesse dizendo: "Faça tudo o que eu lhe ordenei
e cuide-se para não fazer todas as coisas que eu lhe proibi", ou "Não se
rebele contra algo que eu lhe ordenei". Tal ordem não deve ser contada
como um preceito separado já que ela não se refere a uma obrigação
específica, o que faria dela um preceito positivo, nem adverte contra um
ato determinado, o que a trans­formaria num preceito negativo.

Assim por exemplo é o que Ele disse: "E de tudo o que vos disse,
guardá-lo-eis" (Êxodo 23:13); "Meus estatutos guardareis" (Levítico
19:19); "Os Meus juízos cumprireis" (Ibid., 18:4); "E guardardes Minha
aliança" (Êxodo 19:5); "E guardareis o Meu mandado" (Levítico 18:30), e
muitas outras afirmações aná­logas.

Já se enganaram com relação a este Fundamento, contando "Santos sereis"
(Ibid., 19:2) como um preceito positivo, sem saber que os versículos

46 MAIMÔNIDES

"Santos sereis" e "Santificar-vos-ei e sereis santos" (Ibid., 11:44) são
ordens para que cumpramos a totalidade da Torah, como se Ele dissesse:
"Seja santo fazendo tudo o que Eu lhe ordeno e afaste-se de tudo o que
lhe proibi de fazer".

A Sifrá diz: " 'Santos sereis' --- fique distante". Ou seja, fiquem
lon­ge das abominações contra as quais Eu os adverti.

Na Mekhiltá disseram: " 'Homens de santidade sereis para Mim' (Êxo­do
22:30). Issi, o filho de Yehudá diz: a cada novo preceito que o Santo,
enalte­cido seja Ele, impõe a Israel, Ele lhes acrescenta mais
santidade". Quer dizer, esta não é uma obrigação independente, mas está
relacionada aos preceitos que lhes foram ordenados ali, pois todo aquele
que cumprir aquela obrigação será chamado de "santo".

Dessa forma não há diferença entre Suas palavras "Santos sereis" e
"Cumpre Meus preceitos". Assim como não diríamos que esta intimação
glo­bal constitui um preceito positivo a ser acrescentado a todos os
outros precei­tos, assim também não podemos dizer que Suas palavras
"Santos sereis" e ou­tras expressões semelhantes constituem preceitos
separados, uma vez que não há nelas nada de específico além do que já
sabemos.

O Sifrei diz: " 'E sejais santos' (Números 15:40) --- isto se refere à
santidade dos preceitos".

Assim, ficou claro aquilo a que nos propusemos.

A partir deste Fundamento segue-se também que Suas palavras "E tirareis
o entupimento de vosso coração" (Deuteronômio 10:16) significam: Sede
humilde e ouví todos os preceitos que Ele mencionou anteriormente. Da
mes­ma forma, o versículo "E vossa cerviz não endurecereis mais" (Ibid.)
significa: Não vos rebeleis, aceitando tudo o que vos ordenei e não o
desobedeçais.

\textbf{O QUINTO}

\textbf{FUNDAMENTO}

A RAZÃO DADA PARA UM PRECEITO NÃO DEVE SER CONTADA COMO UM PRECEITO
SEPARADO

Ocasionalmente encontramos razões para os preceitos em forma de
preceitos negativos que poderíamos pensar ser apropriados contar como
pre­ceitos independentes. Assim, por exemplo, são Suas palavras "Não
poderá seu primeiro marido, que a despediu, tornar a tomá-la, para que
seja sua mulher... e não farás condenar a terra" (Deuteronômio 24:4),
onde as palavras "E não farás condenar a terra" são as razões da
proibição que as antecede, como se Ele tivesse dito: "Se fizeres assim,
aumentará a corrupção na terra".

Outro exemplo são Suas palavras "Não profanarás a tua filha para fazê-la
prostituta, para que a terra não seja entregue à prostituição" (Levítico
19:29), onde as palavras "Para que a terra não seja entregue à
prostituição" cons­tituem o motivo, como se Ele tivesse dito: "A razão
desta proibição é para que a terra não seja entregue à prostituição".

O mesmo ocorre no versículo "E não vos façais impuros com eles e não
sejais impuros por eles" (Ibid., 11:43). Depois de ter mencionado a
proi-

OS 14 FUNDAMENTOS 47

bição contra comer determinadas coisas, Ele deu a razão para isso
dizendo: "Não se tornem impuros comendo-os", dando a entender que a
transgressão desta proibição causa a impurificação da alma.

O Sifrei diz expressamente, com relação a Suas palavras, enaltecido seja
Ele, sobre a proibição de pedir um resgate pela vida de um assassino: "O
versículo 'E não contaminarás a terra' (Números 35:34) nos ensina que o
derra­mamento de sangue impurifica a terra". Portanto, foi explicado que
esta or­dem negativa constitui uma razão para a proibição anterior e não
uma declara­ção independente.
\end{quote}

Da mesma forma dizem, com relação ao seguinte versículo "E do\\
santuário não sairá e não profanará" (Levítico 21:12): "Mas se sair, ele
profanará".\\
Também com relação a este Fundamento errou um outro, incluindo\\
essas ordens, sem compreendê-las. Todavia, se se perguntasse a alguém
que as\\
incluiu qual é a obrigação específica que essa ordem estabeleceu ele
ficaria con-\\
fuso e não teria resposta. E isso anula a alegação de que elas possam
ser contadas.

\begin{quote}
\textbf{O SEXTO}

\textbf{FUNDAMENTO}

QUANDO UM PRECEITO CONTÉM TANTO UMA ORDEM POSITIVA QUANTO UMA NEGATIVA,
CADA UMA DAS PARTES DEVE SER CONTADA SEPARADAMENTE, UMA ENTRE OS
PRECEITOS POSITIVOS E A OUTRA ENTRE OS NEGATIVOS

Você deve saber que um assunto pode ser regulamentado por meio de ambos
um preceito positivo e um negativo, de uma destas 3 maneiras:
\end{quote}

\begin{enumerate}
\def\labelenumi{\arabic{enumi}.}
\item
  \begin{quote}
  Se o cumprimento de uma determinada obrigação acarretar um preceito
  positivo e a sua transgressão um preceito negativo, como por exem­plo
  no Shabat, nos festivais e no Ano Sabático, quando se alguém fizer
  certos trabalhos estará violando um preceito negativo e se descansar
  estará cumprin­do um preceito positivo, como será explicado
  posteriormente. Da mesma for­ma, o jejum em "Yom Quipur" constitui um
  preceito positivo e comer nesse dia é um preceito negativo.
  \end{quote}
\item
  \begin{quote}
  Se houver um preceito negativo precedido por um preceito posi­tivo,
  tal como se vê em Suas palavras com relação a quem seduziu ou
  maldisse: "E lhe será por mulher" (Deuteronômio 22:19), que constituem
  um preceito positivo, e "Não a poderá despedir por todos os seus dias"
  (Ibid., 29), que cons­tituem um preceito negativo.
  \end{quote}
\item
  \begin{quote}
  Se houver um preceito negativo enunciado primeiro, e depois
  jus­taposto a um preceito positivo, como aparece, por exemplo, em Suas
  palavras "Não tomarás a mãe estando com os filhos" (Ibid., 6), que são
  seguidas de "Dei­xarás ir livremente a mãe" (Ibid., 7).
  \end{quote}
\end{enumerate}

\begin{quote}
Em cada um destes casos devemos contar a ordem positiva entre os
preceitos positivos e a negativa entre os preceitos negativos, pois os
Sábios fa-

48 MAIMÔNIDES

lam explicitamente em cada caso do preceito positivo e do preceito
negativo. Assim, eles dizem muitas vezes: "O positivo ou o negativo,
relativos a isto". Isto é perfeitamente compreensível, posto que o
significado do positivo é dife­rente do negativo e portanto eles são
duas obrigações diferentes: num Ele nos dá uma ordem e no outro Ele nos
faz uma advertência.

Eu não me recordo no momento de alguém que tenha se enganado com relação
a este Fundamento.

\textbf{0 SÉTIMO}

\textbf{FUNDAMENTO}

NÃO SE DEVE CONTAR AS LEIS DETALHADAS DE UM P ECEITO

Saiba que cada preceito está expresso n s E crituras e a partir des base
seguem-se muitas obrigações e advertências •m r lação às leis que reg m
esse preceito. Este é um exemplo disso: a "Halit • á"\textsuperscript{7}
o casamento levir o\textsuperscript{8} são dois preceitos positivos. Não
há nenhuma c o tro érsia quanto a isso. Mas quando estudamos as leis
desses dois preceitos p ■ it os e o que deve ser cum­prido de acordo com
os postulados da lei, verifica-se que algumas mulheres realizarão a
"Halitzá" e não o casamento levirato e que outras realizarão o
casa­mento levirato e não a "Halitzá", enquanto outras ainda farão ou um
ou outro, e outras não farão nem um nem outro. O mesmo se aplica aos
homens, ou seja, aos cunhados. Alguns se submetem à "Halitzá" mas não
realizam o casamento levirato, outros se casam, mas não se submetem à
"Halitzá"; outros ainda não fazem nenhum dos dois, e outros podem fazer
ou um ou outro: contrair o casa­mento levirato ou submeter-se à
"Halitzá". Da mesma forma, verifica-se que algumas das cunhadas devem
fazer a "Halitzá" e outras devem casar-se com um dos cunhados; algumas
farão a "Halitzá" a todos; a algumas mulheres era permitido que ela se
casasse com o homem que se tornou seu marido, mas não com os irmãos
dele, enquanto que no caso de outra mulher ela poderia ter-se casado com
um dos irmãos de seu marido, mas estava proibida de casar-se com o homem
que de fato se tornou seu marido; em outros casos, tanto o marido como
os irmãos dele eram homens proibidos para ela e, em outro caso ainda,
era-lhe possível casar-se tanto com o homem que se tornou seu marido
como com os irmãos dele.

Se fôssemos considerar cada lei dessas como um preceito indepen­dente,
só as leis do Tratado Yebamot somariam mais de duzentos preceitos!
Contudo, nenhuma delas é por si só um preceito positivo ou um negativo;
ao contrário, devemos dizer que em determinadas circunstâncias uma
cunhada deve fazer a "Halitzá" ou o casamento levirato, e que em outras
ela está proibida de casar-se com um determinado cunhado, ou então que
tanto a "Halitzá" quan­to o casamento levirato são impossíveis no seu
caso. E assim deve ser necessa­riamente com relação a cada um dos
preceitos.
\end{quote}

\begin{enumerate}
\def\labelenumi{\arabic{enumi}.}
\setcounter{enumi}{6}
\item
  \begin{quote}
  Ver o preceito positivo 217.
  \end{quote}
\item
  \begin{quote}
  Ver o preceito positivo 216.
  \end{quote}
\end{enumerate}

\begin{quote}
OS 14 FUNDAMENTOS 49

Sendo assim --- e isto é um assunto acima de qualquer discussão
---conclui-se que ainda que as leis de um preceito estejam
explicitamente enun­ciadas na Torah elas não devem ser contadas. O mero
fato de as Escrituras te­rem explicado as leis ou as condições de um
determinado preceito não signifi­ca que devamos contar cada condição e
cada detalhe da lei como um preceito individual.

Mas muitos já se enganaram a este respeito, contando tudo o que
encontraram nas Escrituras, sem refletir sobre a substância do preceito,
suas leis e condições. A título de exemplo podemos citar que as
Escrituras, no livro de "Vayikrá" (Levítico), obrigam uma pessoa que
tenha tornado impuros o San­tuário, suas ofertas consagradas e outras
coisas ali mencionadas, a levar um Sa­crifício de Pecado. Isto
constitui, sem dúvida, um preceito positivo. Logo de­pois as Escrituras
explicam as leis relativas a essa oferenda, dizendo que ela de­ve
constar de uma ovelha ou uma cabra, e que se lhe não tiver posses
suficien­tes para isso deverá levar duas rolas ou dois pombinhos, e se
ele também não puder se permitir isso, deverá levar a décima parte de
uma "efá" de farinha. Isto constitui um Sacrifício de Maior ou Menor
Valor. É óbvio que todas estas leis são apenas uma explicação sobre qual
é o sacrifício imposto, e que de for­ma alguma elas devem ser contadas
como três preceitos --- o de oferecer o ani­mal, o de oferecer as aves,
e o de oferecer a décima parte de uma "efá" pois elas não são três
ordens e sim apenas um único preceito, a saber, que um transgressor deve
oferecer um sacrifício por pecado e que esse sacrifício deve ser isto ou
aquilo, dependendo de seus recursos.

O mesmo princípio se aplica no caso de um sacrifício por um erro
cometido com relação aos preceitos. Assim, as Escrituras explicaram no
livro "Vayikrá" que aquele que transgride involuntariamente um dos
preceitos do Eterno deve levar um Sacrifício de Pecado, e que isto
constitui um preceito positivo se o erro for um dos que acarretam a
extinção se cometido voluntaria­mente, se houver algum ato relacionado
com ele e se ele acarretar a transgres­são de um preceito negativo, como
explicamos no Comentário a Horayot e Que­retot. Em seguida, a Escritura
detalhou as leis relativas a esse sacrifício, cando a isso vários
versículos e dizendo que se a pessoa que cometeu o p for alguém do povo,
ela deve levar uma ovelha ou uma cabra; se for o ele deve levar um bode;
e se for o "Cohen Gadol", ele deve levar um se o erro cometido for
especificamente com relação à idolatria, o transg --- seja ele o líder,
alguém do povo ou o "Cohen Gadol" --- deve oferece a cabra. Mas o fato
de oferecer diferentes tipos de animais não altera a natureza do
sacrifício em si, que é a oferta a ser levada por um pecado não
intencional, e não o transforma em vários sacrifícios, de modo a dar
origem a vários precei­tos. Se assim fosse, deveríamos da mesma forma
contar Suas palavras "uma ove­lha" ou "uma cabra" como dois preceitos
separados, e "duas rolas" ou "duas pombinhas" como outros dois
preceitos. Obviamente isto não estaria correto, pois o que constitui o
preceito positivo é a obrigação de levar um sacrifício; e o fato de que
uma pessoa ofereça uma cabra como sacrifício, e a outra um bode, é
apenas uma condição dessa oferenda, mas nem toda condição de um preceito
deve ser considerada como um preceito independente.

Compreenda bem este ponto, pois um erro acerca disto pode ser disfarçado
e só será percebido por alguém dotado de compreensão.
\end{quote}

Também entra nesta categoria o que Ele disse, enaltecido seja Ele,

\begin{quote}
50 MA MÔNIDES

em complemento à lei do castigo de uma moça compromet cometer

adultério: que se uma moça comprometida cometer adultéri tigo será

o apedrejamento, e que se ela for a filha de um "Cohen", seu o será ser

\href{http://queimada.Com}{{queimada. Com}} relação a este assunto,
todos os que consultei se enganaram pois eles contaram um preceito
separado para a mulher casada, um para a moça comprometida, e um para a
filha de um "Cohen". Mas isso não está correto, como explicarei a
seguir.

Suas palavras, enaltecido seja Ele, "Não adulterarás" (Êxodo 20:14), que
constituem um preceito negativo, são explicadas pela Tradição como
sen­do uma advertência às mulheres casadas. Essa advertência é seguida
por um versículo que declara que quem violar esta proibição está sujeito
à pena de morte. Isso está expresso em Suas palavras "Certamente serão
mortos, o adúltero e a adúltera" (Levítico 20:10). A seguir, as
Escrituras completam a lei desta puni­ção estabelecendo condições,
dizendo que o versículo "Certamente serão mor­tos, o adúltero e a
adúltera" está sujeito às seguintes condições: se a mulher casada for
filha de um "Cohen", seu castigo será ser queimada; se ela for uma moça
virgem comprometida, seu castigo será ser apedrejada; e se ela não for
mais virgem e não for a filha de um "Cohen", seu castigo será o
estrangulamen­to. As leis detalhadas relativas ao tipo de morte a ser
aplicado não transformam esse único preceito em vários pois, apesar de
todos esses detalhes, não nos afas­tamos da proibição básica imposta à
mulher casada.

Os Sábios dizem explicitamente em Sanhedrin: "Estavam todos in­cluídos
nos termos 'adúltero e adúltera' (Ibid.); as escrituras apenas
especifica­ram que a filha de um israelita está sujeita a ser apedrejada
e que a filha de um "Cohen" está sujeita a ser queimada". Com essa
afirmação eles pretenderam dizer o seguinte: a proibição imposta à
mulher casada através de Suas palavras "Certamente serão mortos, o
adúltero e a adúltera" inclui a todos; as Escrituras apenas
estabeleceram uma diferença quanto ao tipo de morte, impondo à quei­ma a
algumas e o apedrejamento a outras.

Se devêssemos contar as leis detalhadas de um preceito por estarem elas
mencionadas nas Escrituras, então deveríamos contar muitos preceitos ao
invés de um só na lei que determina que um homicida não intencional deve
exilar-se numa cidade de refúgio, posto que as Escrituras mencionaram
especi­ficamente os detalhes das leis relativas a esse preceito.
Deveríamos, então, con­tar da seguinte forma: "E se com instrumento de
ferro ferir" (Números 35:16) --- um preceito; "E se com uma pedra que
cabe na mão" (Ibid., 17) --- o segun­do preceito; "Ou se com instrumento
de madeira, com o qual se pode matar, ferir alguém" (Ibid., 18) --- o
terceiro preceito; "O vingador do sangue, matará o homicida" (Ibid., 19)
--- o quarto preceito; "E se com ódio empurrar alguém" (Ibid., 20) --- o
quinto preceito; "Ou jogar alguma coisa sobre ele, de embosca­da"
(Ibid.) --- o sexto preceito; "Ou por inimizade o ferir com a mão"
(Ibid.,
\end{quote}

\begin{enumerate}
\def\labelenumi{\arabic{enumi})}
\setcounter{enumi}{20}
\item
  \begin{quote}
  --- o sétimo preceito; "E se por acaso, sem inimizade o empurrou"
  (Ibid.,
  \end{quote}
\item
  \begin{quote}
  --- o oitavo preceito; "Ou jogou sobre ele algum instrumento sem ser
  de emboscada" (Ibid.) --- o nono preceito; "Ou não o vendo... alguma
  pedra que possa causar-lhe a morte" (Ibid., 23) --- o décimo preceito;
  "Jogou sobre ele... e este morrer, não sendo ele seu inimigo" (Ibid.)
  --- o décimo primeiro precei­to; "E salvará a congregação ao homicida"
  (Ibid., 25) --- o décimo segundo pre­ceito; "E a congregação o fará
  voltar à sua cidade de refúgio" (Ibid.) --- o déci­mo terceiro
  preceito; "E ficará nela até morrer o 'Cohén Gadol"' (Ibid.) --- o
  \end{quote}
\end{enumerate}

\begin{quote}
10. Através do noivado legal, que acarreta as responsabilidades legais
do casamento.

OS 14 FUNDAMENTOS 51

décimo quarto preceito; "E se sair o matador fora" (Ibid., 26) --- o
décimo quinto preceito; "E depois da morte do 'Cohen Gadol' voltará o
homicida" (Ibid., 28) --- o décimo sexto preceito.

Se devêssemos fazer isso em todo e cada preceito, o número de pre­ceitos
ultrapassaria os dois mil! É óbvio, contudo, que isso não seria
racional, já que todas essas leis são apenas detalhes de um preceito, e
que o preceito a ser contado é a lei do homicídio, ou seja, que devemos
lidar com o assunto em questão de acordo com a lei estabelecida nesses
versículos. Na realidade, o Eterno se referiu a eles como "mishpatim"
(leis) e não como "mitzvot" (pre­ceitos). Ele disse: "Então julgará a
congregação entre quem feriu e o vingador do sangue, segundo estas leis"
(Ibid., 24).

O autor do "Halachot Guedolot" já se preocupou com relação a es­te
assunto e prestou atenção a ele, mas quando se deparou com complicações
ele começou a contar seções, enumerando "a seção de herança", "a seção
de promessas e juramentos", "a seção do difamador" e muitas outras
assim. En­tretanto, este conceito não lhe ficou totalmente claro e nem
foi completamen­te compreendido por ele, e assim ele colocou nessas
seções, sem notar, assun­tos que já havia enumerado anteriormente.
Assim, devido ao fato de desconhe­cer este Fundamento, ele contou onze
preceitos com relação à lepra, sem per­ceber que nesse caso há apenas um
preceito e que tudo o que está mencionado nas Escrituras não é mais do
que a enumeração detalhada de suas leis e condições.

O significado disto é o seguinte. Foi-nos ordenado que uma pessoa, ao
tornar-se impura por causa da lepra, deve cumprir todas as obrigações
im­postas aos impuros, ou seja, afastar-se do Santuário e de suas
ofertas consagra­das, e sair do campo da Presença Divina. Contudo, como
ainda não sabemos qual tipo de lepra impurifica uma pessoa e qual não,
as Escrituras começaram, conseqüentemente, a explicar e a detalhar a
lei: se for assim, ele estará puro; se for de outra forma, ele estará
impuro; e se ela for de uma determinada ma­neira, ele deverá afastar-se
por um certo período de tempo.

Os Sábios dizem explicitamente: "Para declará-las puras ou impuras"
(Levítico 13:59) --- assim como é obrigatório declará-la pura, também é
obriga­tório declará-la impura. Assim, é óbvio que o preceito consiste
apenas em declará-lo "impuro'.' ou "puro", mas os detalhes que tornam a
pessoa impura ou pura não devem ser contados, uma vez que eles nada mais
são do que as condições e os detalhes da lei. Isso é o mesmo que dizer
que não se deve ofere­cer um animal defeituoso como sacrifício, o que
seguramente constitui um pre­ceito negativo; resta-nos saber apenas o
que é considerado como defeito. Mas devemos contar todo defeito como um
preceito independente? Se assim fosse, o número chegaria a cerca de 70!
Portanto, assim como não contamos os defei­tos --- qual deles é
considerado como defeito e qual não --- mas sim apenas a advertência que
nos previne para não oferecer um animal defeituoso, assim tam­bém não
devemos contar as manifestações da lepra --- qual é impura e qual é pura
--- mas sim contar apenas que a lepra é impura, sendo todo o resto uma
explicação sobre de que consiste a lepra.

É segundo esse método que devemos contar como um único pre­ceito cada um
dos (treze) tipos de impureza, e não contar os detalhes das leis e
condições de um tipo específico de impureza, como será esclarecido em
nos­sa enumeração.

Compreenda este Fundamento, pois ele é uma "coluna central" de apoio
neste assunto.

52 MAIMÔNIDES

\textbf{O OITAVO}

\textbf{FUNDAMENTO}

NÃO SE DEVE INCLUIR ENTRE OS

PRECEITOS NEGATIVOS UMA DECLARAÇÃO NEGATIVA QUE EXCLUA

UM CASO PARTICULAR DE UM DETERMINADO ASSUNTO

Você deve saber que uma proibição é uma das duas partes de uma ordem.
Quer dizer, pode-se ordenar a uma pessoa que faça algo ou que não o
faça, como por exemplo, quando você mandar alguém comer alguma coisa e
lhe disser: "Coma", ou quando mandar que ele se abstenha de comer e lhe
disser: "Não coma". No idioma árabe, porém, não há uma palavra que
abranja esses dois significados. Os mestres do Kalam dizem, na teoria da
lógica, algo assim: "Em árabe o preceito e a advertência não têm uma
palavra em comum para expressá-los, e por isso fomos obrigados a
referir-nos a ambos com o mes­mo termo, que é o preceito". Assim fica
explicado que uma advertência e um preceito são a mesma coisa. A palavra
usada em árabe no sentido de advertên­cia é "LA" (não imperativo). Este
aspecto da comunicação --- a saber, ordenar que uma pessoa faça ou deixe
de fazer alguma coisa --- é encontrado sem dúvi­da alguma em todos os
idiomas. Está, portanto, claro que tanto o preceito posi­tivo quanto o
negativo englobam ordens absolutas: a de fazer determinadas coi­sas e a
de prevenir-nos para não fazer outras. O termo que define as coisas que
somos ordenados a fazer é "preceito positivo" e o que define as coisas
que so­mos avisados para não fazer é "preceito negativo", e o nome que
abrange os dois juntos no idioma hebreu é "Guezerá" (decreto). Dessa
forma os Sábios se referem a todos os preceitos, positivos e negativos,
como sendo "os decre­tos do Rei".

Contudo, uma simples declaração negativa que exclua alguma coisa de um
determinado assunto é algo diferente, pois não há uma ordem com rela­ção
a ela. Esse é o caso, por exemplo, quando se diz: "Aquela pessoa não
co­meu ontem, e aquela outra não bebeu o vinho, e Zaid não é o pai de
Ornar", e outras declarações semelhantes. Todas elas são meras negações
e não têm a menor semelhança com uma ordem.

A palavra usada em árabe para negar uma determinada coisa é quase sempre
"MA", mas algumas vezes eles usam as palavras "LA" ou "LAISA' . Por
outro lado, os hebreus negam a maioria das vezes com a mesma palavra
"LO", também usada para fazer uma advertência. Eles negam também com a
palavra "AYIN", bem como com as formas que essa palavra adquire quando
se lhe acres­centam pronomes como sufixos, tais como "EINO" (ele não é),
"EINAM" (eles não são), "EINCHEM" (você não é), e outras.

Encontram-se negações em hebreu com a palavra "LO" em versícu­los tais
como: "E não (VE'LO) se levantou mais, em Israel, profeta algum como
Moisés" (Deuteronômio 34:10); "Deus não (LO) é homem, para que minta"
(Nú­meros 23:19); "Não (LO) haverá desgraça duas vezes" (Nahum 1:9); "E
não (VE'LO) ficou homem com ele" (Gênesis 45:1); "E ele não (VE'LO) se
levantou nem foi em sua direção" (Esther 5:9); e em muitos outros casos
como esses. Negações com a palavra "EIN" aparecem, por exemplo, nos
versículos "E ho-

OS 14 FUNDAMENTOS 53

mem não (AYIN) existia" (Gênesis 2:5); "Mas os mortos não (EINAM) sabem
nada" (Ecc. 9:5), e muitos outros versículos além destes.

Ficou, assim, clara a diferença entre uma negação e uma advertên­cia. A
advertência tem um caráter de obrigatoriedade e é, na realidade, o verbo
em sua forma de ordem. Quer dizer, da mesma forma que uma ordem, uma
advertência aparece sempre no futuro; assim como é inconcebível num
idioma que uma ordem seja dada no tempo passado, o mesmo ocorre com uma
adver­tência; assim como é impossível introduzir uma ordem numa sentença
que te­nha um relato ou uma narração --- pois uma sentença assim deve
ter um sujeito e um objeto, enquanto que uma ordem constitui por si só
uma expressão com­pleta, como foi explicado nos livros que tratam desse
assunto --- também não se pode introduzir uma advertência numa narração.
Mas nada disto se aplica a uma negação. Uma declaração negativa pode
entrar numa narração e pode referir-se ao passado, futuro ou presente.
Tudo isso fica evidente quando se reflete a respeito.

Sendo assim, não devemos em hipótese alguma contar entre os pre­ceitos
negativos as declarações que forem apenas negações. Naturalmente, isto é
evidente por si só, sem necessidade de provas, a não ser o que foi
menciona­do com relação ao esclarecimento do conteúdo de certas
expressões a fim de permitir a distinção entre a advertência e a
negação.

Um outro, contudo, não estava a par disto e conseqüentemente con­tou
"Não sairá como saem os escravos" (Êxodo 21:7), sem perceber que esta
era uma declaração negativa, e não uma proibição.

Vou explicar este assunto. O Eterno decretou que se um senhor fe­rir seu
servo ou serva cananeus, causando-lhes a perda de um de seus órgãos
externos, eles deverão ser libertados. A partir disso poderíamos pensar
que es­sa lei se aplica com toda certeza à serva hebréia, e que caso o
seu senhor lhe cause a perda de um de seus principais órgãos externos,
ela deverá ser liberta­da; é por isso que Ele a excluiu dessa lei,
dizendo: "Não sairá como saem os escravos", que é como se Ele estivesse
dizendo que não há obrigação de liber­tá-la caso ele lhe tenha causado a
perda de um de seus órgãos. Dessa forma, o versículo nega a aplicação de
uma determinada lei com relação a ela, mas não é uma proibição. Os
transmissores da Tradição explicaram isto, dizendo o se­guinte na
Mekhiltá: 'Não sairá como saem os escravos' significa que ela não será
libertada por causa da perda de um dos órgãos principais, como acontece
com os escravos cananeus". Foi, assim, explicado que esta é apenas a
negação de aplicar a ela uma determinada lei, mas que não constitui uma
advertência.
\end{quote}

Basicamente também não há diferença entre Suas palavras "Não sai-

\begin{quote}
rá como saem os e vos" e "Não procurará o 'Cohen' o pelo louro; impuro

é ele" (Levítico 1 Este versículo constitui igualmente uma negação abso-

luta, e não um•ncia. Quer dizer, Ele nos ensina que ao apresentar um\\
determinado si olamento se torna desnecessário e o "Cohen" não de-

ve hesitar em o impuro.

Da m sma forma Suas palavras "Ele não morrerá, pois ela não era
libertada" (Levíti e 9:20) não constituem uma advertência, mas sim uma
ne­gação que significa, na realidade, que ele não está sujeito à pena de
morte por­que sua liberdade não estava completa. Não se deve traduzir
esse versículo co­mo "Elès não deverão ser mortos" pois assim ele se
transformaria numa adver­tência. Suas palavras "Ele não morrerá, pois
ela não era libertada" são seme-

54 MAIMÔNIDES

lhantes ao que Ele disse: "A moça não tem pecado de morte" (Deuteronômio
22:26). Assim como ele negou a ela a pena de morte porque ela estava sob
coa­ção, Ele também negou a ele a pena de morte por causa da escravidão
da moça, como se tivesse dito que ele não está sujeito à pena de morte
porque ela não era livre.

Um outro caso de negação é o que está em Suas palavras "Para que não
seja como Korah e como sua congregação" (Números 17:5). Os Sábios
ex­plicaram que este versículo é uma negação e interpretam seu
significado dizen­do o seguinte: o Eterno declara que o castigo imposto
a todo aquele que vier e disputar o sacerdócio, reclamando-o para si
próprio não será o mesmo que se abateu sobre Korah e sua congregação ---
a saber, ser tragado pela terra e devorado pelo fogo --- e sim será
"Conforme tinha falado o Eterno, por inter­médio de Moisés", isto é, a
lepra. Isto está expresso em Suas palavras "Leva, por favor, a tua mão
ao teu peito... E a tirou e eis que sua mão estava leprosa como a neve"
(Êxodo 4:6). Eles oferecem como prova disto o que aconteceu com Uziah,
rei de Yehudá. Embora encontremos uma opinião diferente dos Sá­bios na
Guemará de Sanhedrin, afirmando que "Todo aquele que estimula a
discórdia viola um preceito negativo, pois está dito: 'Para que não seja
como Korah e como sua congregação\textsuperscript{---}, isso é apenas o
significado moral, mas não é o significado literal do versículo. Na
realidade, a advertência quanto a esse conceito está em outro preceito,
como explicarei no lugar apropriado.

Não há nenhuma regra precisa para se fazer a distinção entre urna
declaração negativa e uma advertência, a não ser o sentido da
declaração. Não há urna palavra específica que diferencie a negação da
advertência, pois ambas são expressas em hebreu pela mesma palavra:
"LO". Assim, é importante que aquele que analisa o assunto profundamente
pese com cuidado em sua mente o significado das palavras e então ele
perceberá facilmente qual expressão ne­gativa constitui uma negação e
qual uma advertência, como já explicamos anteriormente.

Os Sábios, a paz esteja com eles, comentaram este assunto com
refe­rência a uma controvérsia que surgiu entre eles devido a uma de r
inada ex­pressão negativa, para saber se ela era uma simples negação o
urna dvertên­cia. Isso ocorreu por causa do que está expresso em Suas p
vras, e altecido seja Ele, acerca do pássaro de Sacrifício de Pecado: "E
dest • ncará s cabeça pela nuca, porém não o separará" (Levítico 5:8).
"Nosso T ná"I\textsuperscript{2}, q e fala na Mishná, é de opinião que é
uma advertência, e portanto d "Se ele o separar completamente ele o
invalidará". Conseqüentemente, esta •claraç\textsuperscript{-}o negativa
constitui um preceito, pois se ele o separar, ele o invalida, a • orno
quem oferece levedura ou mel. Por outro lado, Rabi Elazar, o filho do
Rabi Shimon, é de opinião que este versículo não é urna advertência, e
sim uma negativa, e que as palavras "Não o separará" significam que não
é necessário separar a ca­beça e que será suficiente se ele cortar
apenas uma parte dela; dessa maneira, na sua opinião, o sacrifício
também será válido se ele a separar completamente. Os Sábios dizem o
seguinte, na Guemará de Zebahim: "Rabi Elazar, o filho de Rabi Shimon,
costumava dizer: Ouvi dizer que eles separam completamente o pássaro de
oferta de pecado, mas Suas palavras 'não o separará significam que ele
não precisa separá-lo". A respeito dessas palavras os Sábios perguntaram
o seguinte: "Então você também diria que no caso de um Poço, sobre o
qual está

OS 14 FUNDAMENTOS 55

dito: 'E não o cobrir' (Êxodo 21:33), isso significa que ele não precisa
cobri-lo?" A resposta a isso foi: "O versículo diz que 'O dono do poço
pagará' (Ibid. 34). Isso deixa claro que ele deve cobri-lo".

Foi, assim, explicado que os Sábios apresentam provas sobre se se trata
de uma negação ou de uma açlvertência a partir do sentido da própria
declaração.

Também foi explicado que Suas palavras "Não o separará" consti­tuem um
preceito negativo, como está evidenciado na Mishná. E, finalmente, isso
também deixará claro que Suas palavras com relação ao pássaro de
holo­causto, "E o rasgará pelas asas, mas não o dividirá" (Levítico
1:17) não devem ser contadas, pois todos os Sábios concordam que mesmo
que ele o divida com­pletamente, o sacrifício ainda será válido. Isso
porque, como no caso do ani­mal que se oferece como holocausto Ele diz:
"E o cortará em seus pedaços" (Ibid., 12), poder-se-ia pensar que a
mesma regra se aplica ao pássaro de holo­causto, por isso Ele diz que
não é necessário dividi-lo, e sim apenas rasgá-lo, e caso ele o divida
completamente, o sacrifício ainda será válido, como será explicado no
devido lugar.

Um outro caso de um versículo que pertence à categoria de declara­ções
negativas é o versículo "Toda a consagração para pagar o resgate da
ava­liação da pessoa condenada à morte, não poderá ser feita" (Ibid.,
27:29). Uma vez que você saiba qual é o significado dessa afirmação,
ficará claro para você que se trata de uma negação, e não de uma
advertência. E o seguinte. As Escri­turas estipularam que um determinado
pagamento seja feito por "avaliações", de acordo com a idade da pessoa
avaliada e dependendo se ela for homem ou
\href{http://mulher.Com}{{mulher. Com}} relação a este conceito não faz
diferença se alguém disser: "Mi­nha avaliação cabe a mim" ou "A
avaliação de tal pessoa cabe a mim", pois nesses casos vemos o que ela é
e qual a sua idade, e ela pagará de acordo com isso. Mas se a pessoa
avaliada for uma que ficou sujeita à pena de morte pelo Tribunal e foi
julgada culpada, e se alguém então disser: "A avaliação dessa pes­soa
cabe a mim", ela não precisa pagar nada, pois a partir do final do
julgamen­to ela é considerada como morta, e não se avalia os mortos.
Portanto, este é o sentido de Suas palavras ao dizer: "Não poderá ser
feito" (Ibid.): não é preci­so pagar por essa pessoa o resgate que
normalmente aquele que fez o voto de avaliação deveria pagar. Esta é uma
das leis e estatutos sobre as avaliações que foi mencionada nas
Escrituras, mas não é uma advertência.

A Mishná diz: "Não se fazem votos nem avaliações com relação a um
moribundo ou a alguém que foi condenado à morte". O Talmud explica que
isto se aplica a alguém condenado à morte por um tribunal israelita. A
Mekhiltá também diz: "As pessoas sujeitas à morte pelo tribunal não têm
resgate, pois as Escrituras dizem: 'Toda a consagração para pagar o
resgate da avaliação da pessoa condenada à morte, não poderá ser feita".
Avalie a exatidão e a profun­didade das palavras dos Sábios, pois quando
dizem "não tem resgate" --- e não "não deverão ser resgatadas" --- eles
deixam claro que esta afirmação é uma declaração negativa e não uma
advertência.

Esse mesmo assunto está explicado pelos Sábios na Sifrá, na seção de
Avaliações, onde dizem: "De que forma sabemos, se uma pessoa disser com
relação a um condenado a morte pelo Tribunal: 'Sua avaliação cabe a
mim', que suas palavras não têm efeito?" --- significando que ele não é
obrigado a pagar nada? "Pelas palavras das Escrituras: 'Ele não poderá
ser resgatado"'.

Este assunto foi tão perfeitamente explicado que na minha opinião até
mesmo uma pessoa obtusa não terá mais dúvidas a este respeito.
\end{quote}

E já que estamos falando deste assunto, você deve saber que há quatro

\begin{quote}
56 MAIMÔNIDES

palavras na Torah --- a saber, "Hishamer" (guarda-te de), "pen" (para
que não), "aI" (não faça), e "lo" (que não haja) --- que são usadas para
estabelecer uma ad­vertência, e tudo o que tiver sido advertido através
de uma dessas palavras se cha­ma preceito negativo. Os Sábios dizem
claramente: "Toda vez que aparece 'guar­da-te de', 'para que não', 'não
faça' e 'que não haja' há um preceito negativo".

Resta-nos explicar o seguinte ponto para que possamos completar o
propósito desta seção. Toda vez que aparece na Torah que somos obrigados
a proclamar que não fizemos um determinado ato, para assim isentarmo-nos
de toda a responsabilidade quanto a ele, esse ato específico deve ser
contado entre os preceitos negativos, ainda que a proibição que aparece
nele seja ape­nas uma negação e não uma advertência. Pois se Ele nos
obriga a isentarmo-nos dizendo: "Eu não fiz isto ou aquilo", fatalmente
concluímos que estamos sen­do advertidos para não fazer essas coisas.
Tal é o caso, por exemplo, em que a Torah nos obriga a dizer: "Não comi
do segundo dízimo no primeiro dia de luto, e não comi dele em estado de
impureza, e não o troquei para fazer o se­pultamento de um morto"
(Deuteronômio 26:14). Através dessas palavras fica óbvio que fomos
advertidos para não realizar nenhum desses atos, como será explicado no
devido lugar, quando falarmos desses preceitos.

\textbf{O NONO}

\textbf{FUNDAMENTO}

\textbf{ESTA ENUMERAÇÃO NÃO DEVE SER BASEADA NO NUMERO DE VEZES QUE UM
DETERMINADO PRECEITO}

\textbf{NEGATIVO OU POSITIVO ESTÁ REPETIDO NAS ESCRITURAS, MAS}

\textbf{SIM NA NATUREZA DA AÇÃO PROIBIDA OU ORDENADA}

Você deve saber que todas as obrigações e advertências da Torah se
referem a quatro coisas: as opiniões, os atos, os traços de caráter e o
que se diz. Assim, a Torah nos ordena a acatar certas opiniões, tais
como acreditar na unidade, amar a Deus e temê-lo, enaltecido seja Ele,
ou nos adverte para que não acreditemos em certas' opiniões, tais como
acreditar em e atribuir divinda­de a outro que não Ele. Da mesma forma
ela nos manda realizar certos atos, tais como oferecer os sacrifícios e
construir o Santuário ou nos previne quanto a certas ações, tais como a
advertência para não oferecer sacrifícios a outros que não Ele,
enaltecido seja Ele, ou curvar-se diante de outros que não Ele. As­sim
também ela ordenou que rios conduzamos de acordo com determinados traços
de caráter, tais como a benevolência, a misericórdia, a piedade e o
amor, conforme está no versículo "E amarás o teu próximo como a ti
mesmo" (Leví­tico 19:18), ou nos adverte com relação a determinados
outros traços de cará­ter, tais como guardar rancor, recompensar o mal
ou vingar-se, e outros mais, como explicarei. A Torah nos ordena recitar
certas palavras, tais como expres­sar a nossa gratidão a Ele, orar a
Ele, confessar os pecados e outros assuntos,

OS 14 FUNDAMENTOS 57

similares, como expl,icarei, ou nos previne quanto a dizer certas
coisas, tais co­mo a advertência contra pronunciar um falso juramento,
fazer intrigas, falar mal dos outros, amaldiçoar, e outros além destes.

Quando estes aspectos forem compreendidos ficará claro que é a na­tureza
do assunto ordenado ou proibido --- quer se refira ele a um ato, a algo
que se disse, a uma opinião ou a um traço de caráter --- que deve ser
contado, e que não devemos levar em consideração o número de vezes que
determina­da ordem ou advertência está repetida --- dependendo se se
trata de um precei­to positivo ou de um negativo ---, pois a finalidade
de todas as repetições é dar maior ênfase, realçando, com a repetição da
lei, o assunto proibido ou ordena­do. Apenas quando você encontrar uma
declaração dos Sábios relativa à divi­são dos assuntos, e quando tiver
sido explicado pelos Intérpretes que cada um desses preceitos negativos
ou positivos contém um assunto específico, não abrangido pelo outro, só
assim deve-se contá-los, mesmo que à primeira vista possa parecer que
eles tratam de um mesmo assunto, pois o objetivo das repe­tições não
será enfatizar e sim dar instruções a respeito de aspectos adicionais.
Somente quando não tivermos escolha, e não encontrarmos apoio nas
palavras dos Intérpretes, 'os guardiães da Tradição, dizendo que o
versículo foi repetido para acrescentar alguma instrução, diremos
necessariamente que ele foi repeti­do para dar maior realce. Mas se
encontrarmos uma Tradição de que uma certa ordem ou proibição se refere
a um determinado assunto, e que a repetição des­sa declaração acrescenta
alguma coisa, será certamente verdadeiro e correto afir­mar que o
versículo foi repetido para ensinar-nos algum princípio novo, e nes­se
caso cada versículo deverá ser contado separadamente. Mas onde nada de
novo tiver sido acrescentado, o objetivo da repetição será dar maior
ênfase, informar-nos que a transgressão daquela lei é muito grave --- já
que as Escritu­ras nos advertem várias vezes a esse respeito ---,
completar a lei de um determi­nado preceito, ou deduzir a partir dela
certa lei para outro preceito, como ex­plica o Talmud ao dizer: "Ela
está repetida a fim de constituir a base para uma analogia ou para
ensinar-nos através da dedução de frases semelhantes".

Verificamos que os Sábios, a paz esteja com eles, comentam este pon­to
no segundo capítulo da Guemará Pessahim ao discutir uma determinada
proi­bição que parece estar repetida porque já havia sido deduzida de
outro precei­to, e que por isso querem aplicá-la como uma instrução
adicional. Eles dizem, como discussão e consideração: "Rabina disse a
Rav Ashei: Talvez seja porque a pessoa transgrediu dois preceitos
negativos". Em outras palavras, por que pro­curar a solução tentando
aplicar esta proibição a algo diferente daquilo que con­cluímos com a
proibição original? Talvez ela tenha sido repetida com referên­cia ao
mesmo assunto, de modo que aquele que a desobedecer será culpado por
transgredir duas proibições. A resposta a isso foi: "Disse-lhe: Sempre
que há uma possibilidade de interpretar o versículo nós o fazemos e não
fazemos dele uma proibição adicional". Portanto, foi deixado claro que
uma proibição que não estabelece alguma instrução nova é chamada de
"adicional", ou seja, é repetitiva. Assim, embora os Sábios falem de
"culpado por haver transgredi­do dois preceitos negativos", fica claro,
por toda esta explicação, que se trata apenas de uma proibição adicional
e que por essa razão ela não deve ser conta­da. Dessa forma foi deixado
claro que, a enumeração dos preceitos não deve ser baseada no número de
vezes que um determinado preceito negativo ou po­sitivo foi repetido.

Sabe-se que o preceito que nos ordena descansar no Shabat está
men­cionado doze vezes na Torah. Acaso alguém que enumera os preceitos
diria "Entre os preceitos positivos está descansar no Shabat, que
consiste de 12 pre-

58 MAIMÔNIDES

ceitos"? Uma pessoa sensata diria que a proibição de comer sangue
consiste de sete preceitos? Ninguém vai se enganar com relação ao fato
de que descan­sar no Shabat é apenas um dos preceitos positivos e de que
a advertência quan­to a comer sangue é apenas um dos preceitos
negativos.

Você deve saber que mesmo que se encontre uma frase dos Sábios dizendo
que aquele que comete uma certa transgressão viola dessa forma um certo
número de proibições, ou que aquele que deixa de fazer um certo ato
viola por causa disso um certo número de obrigações, não se deve
concluir que devemos contar separadamente cada uma dessas proibições ou
obrigações, pois a natureza da ação é uma só, e não várias. O fato de
dizerem que a pessoa transgride tantos preceitos positivos ou negativos
só se deve à repetição da or­dem ou da advertência feita com relação
àquele preceito em particular, pois ele violou aquele número de
advertências ou ordens. Apenas quando os Sábios disserem que "ele deve
ser açoitado duas vezes" ou que "ele deve ser açoitado três vezes", aí
então cada advertência deve ser contada separadamente, já que ninguém é
açoitado duas vezes por violar um único preceito, como explicarei de
acordo com o que se sabe pelos textos do Talmud --- em Macot, Hulin, e
outros trechos. Contudo, administram-se dois açoitamentos por dois
preceitos, isto é, por dois assuntos independentes para os quais haja
advertências separadas.

Portanto essa é a diferença entre eles dizerem "ele violou tantos e
tantos" e "ele deve ser açoitado duas ou três vezes".

Encontramos a prova de tudo o que dissemos nas palavras dos Sá­bios
"Aquele que não tiver `tsitsit' em suas vestes violará cinco preceitos
posi­tivos", porque a ordem relativa a eles aparece cinco vezes: "Que
façam para eles 'tsitsit'... e porão sobre os `tsitsit' ... E será para
vós por 'tsitsit\textsuperscript{---} (Números 15:38-39). " 'Tsitsie
farás ra ti e os porás nos quatro cantos de tua vestimen­ta"
(Deuteronômio 22:1 tudo, encontramos uma declaração explícita dos Sábios
com relação ao receit dos "tsitsit" dizendo que se trata de um único
preceito, como explic eii\textsuperscript{3} q ando tratar dele.

Similar a isso disseram: "Quem não coloca 'Tefilin' vio-

la oito preceitos positi o ", ordem relativa a eles --- isto é, o "Tefi-

lin" da cabeça e o do b aço oito vezes. Também disseram: "O 'Co-

hen' que não subir a platafo la três preceitos positivos" porque a or-

dem relativa a isto está rep ti vezes. Mas ninguém que enumerasse os

preceitos pensaria que a Bênção do "Cohen" constitui três preceitos e
que o "Tefilin" constitui oito.

Sendo assim, conclui-se que não devemos contar "enganar um pro­sélito"
como três preceitos devido à reiteração da proibição e às palavras dos
Sábios na Guemará de Metzia: "Aquele que enganar um prosélito violará
três proibições e aquele que o oprimir violará três proibições". Ao
contrário, deve­mos contar apenas os dois preceitos segbintes: "E ao
peregrino não o frauda­reis e não o oprimireis" (Êxodo 22:20), sendo os
outros repetições destas proi­bições. Não há dúvidas q to a isto.

Os Sábios diz\textsuperscript{-} .licitamente na Guemará de Metzia: "Por
que a

Torah adverte em trinta lugares para que não se trate mal o prosélito?

Porque seu temperamen mau". Alguém pensaria em incluir esses trinta

e seis preceitos nos seis s e treze preceitos? É totalmente
inconcebível.
\end{quote}

\begin{enumerate}
\def\labelenumi{\arabic{enumi}.}
\setcounter{enumi}{12}
\item
  \begin{quote}
  Ver o preceito positivo 14.
  \end{quote}
\item
  \begin{quote}
  A plataforma onde os "Cohanim" fazem a bênção do povo.
  \end{quote}
\item
  \begin{quote}
  Significando seu instinto do mal.
  \end{quote}
\end{enumerate}

\begin{quote}
OS 14 FUNDAMENTOS 59

Dessa forma, foi explicado e esclarecido que nem todo preceito ne­gativo
ou positivo encontrado na Torah deve ser contado, pois pode ser que ele
seja apenas uma repetição; somente deve ser contado o conceito da ação
ordenada ou proibida. Apenas um professor --- um dos transmissores do
Co­mentário, a paz esteja com eles --- pode instruir-nos quanto a se um
determina­do preceito positivo ou negativo reaparece a fim de
estabelecer alguma instru­ção adicional ou não.

Também não se deixe confundir por uma proibição que aparece sob
diferentes formas, tal como: "E tua vinha não rebuscarás" (Levítico
19:10), "E esqueceres uma gavela no campo, não voltarás a tomá-la"
(Deuteronômio 24:19), e "Quando bateres a tua oliveira (lo tefo'er), não
tornarás a colher o que resta nos ramos" (Ibid.,20). Na realidade, elas
não são duas proibições, e sim uma única advertência relativa a um único
assunto, a saber, que não se deve voltar para buscar o cereal ou as
frutas esquecidas durante a colheita e por isso Ele mencionou dois
exemplos: as uvas e as olivas. O significado da palavra "Lo te­fo'er" é:
não corte o que você esqueceu no fim dos galhos, isto é, os ramos.

Explicarei agora o que deve ser acrescentado a este fundamento. É o
seguinte. O que dissemos, a saber, que devemos contar os conceitos que
nos foram ordenados ou proibidos, está condicionado ao fato de que para
cada um desses conceitos haja um preceito negativo específico ou uma
prova de que os mestres da Tradição separam um conceito do outro,
resultando cada um nu­ma advertência. Mas se uma proibição inclui muitos
assuntos, então contamos apenas essa proibição e não cada um dos vários
conceitos incluídos nela. Essa é a proibição global (Lav shebikhlalut),
cuja violação não acarreta açoitamento, como vamos explicar a seguir.

Ao comentar Suas palavras, enaltecido seja Ele, "Não comereis so­bre o
sangue" (Levítico 19:26), os Sábios dizem: "De que forma sabemos que é
proibido comer da carne de um animal antes que a vida o tenha abandonado
por completo? Pelo v sl lo: 'Não comereis sobr sangue'. Outra
interpre­tação: 'Não comereisr.obre o sangue' --- não com a c• ne
enquanto o sangue ainda estiver na tigel•• Dossá diz: 'De qu form
sabemos que não de­vemos dar alimentos efeição de Confort\textbf{o
\textsuperscript{17}} por alguém que foi execu­tado judicialmente?' 'e •
ersículo 'Não comera s sob o sangue'. Rabi Akiba diz: 'De que forma
sabemos que um Sanhedrin - alizou uma execução não deve er nesse dia?'
Pelo versículo 'Não come eis sobre o sangue'. Rabi Yossi, o fi • de abi
Haniná, disse: 'De que forma deduzimos a advertência com rela­çã•o filhe
impertinente e rebelde?' Pelo versículo São comereis sobre o san­gu
\textbf{"\textsuperscript{18}.} As im, pois, temos cinco assuntos
sujeitos todos eles a uma advertên­cia *ncluí s todos nessa
\href{http://proibição.Com}{{proibição. Com}} relação a eles os Sábios
dizem expli­citamente, a Guemará de Sanhedrin: "Nenhum deles acarreta o
açoitamento pois tra se de uma proibição global e não se aplica o
açoitamento por uma proibição global". Mais adiante eles explicam que
uma proibição global é aque­la que dá origem a duas ou três proibições.
Portanto fica, claro que não deve­mos contar cada proibição incluída
nesse preceito negativo como sendo um preceito separado, mas sim como um
único preceito negativo que abrange to­das elas.
\end{quote}

\begin{enumerate}
\def\labelenumi{\arabic{enumi}.}
\setcounter{enumi}{15}
\item
  \begin{quote}
  Ou seja, antes do sangue ser aspergido sobre o altar.
  \end{quote}
\item
  \begin{quote}
  Dada ao enlutado após o funeral.
  \end{quote}
\item
  \begin{quote}
  Que quer dizer "Não comereis a comida que acarreta a pena de morte".
  \end{quote}
\end{enumerate}

\begin{quote}
60 MAIMÔNIDES

Semelhante à proibição "Não comereis sobre o sangue" são Suas pa­lavras
"E diante do cego não porás tropeço" (Levítico 19:14) porque elas
tam­bém incluem muitas proibições, como vamos explicar. Da mesma forma
Suas palavras "Não dês ouvidos à maledicência" (Êxodo 23:1) incluem
muitos con­ceitos, como explicaremos. Este é o primeiro dos dois tipos
de proibições globais.

O segundo tipo consiste de um preceito negativo que proíbe várias coisas
juntas e acrescidas umas as outras, tal como quando Ele diz: "Não faça
isto e aquilo". Por sua vez, este tipo se divide em duas partes, e de
acordo com a explicação do Talmud uma delas acarreta o açoitamento por
cada um dos con­ceitos, e a outra causa um único açoitamento, por ser
uma proibição global. Segundo os Sábios, devemos contar como preceitos
separados cada uma das proibições que ocasionam um açoitamento por cada
conceito, e devemos con­tar como um único preceito as proibições que
ocasionam o açoitamento uma única vez, de acordo com o que estabelecemos
neste Fundamento --- que em circunstância alguma se é açoitado duas
vezes por um único preceito. Por ou­tro lado, onde eles estabeleceram
claramente que se está sujeito ao açoitamen­to por cada um dos assuntos
ligados e relacionados entre si, de maneira que aquele que fizer todos
ao mesmo tempo estará sujeito a vários açoitamentos, concluímos com
certeza que eles constituem vários preceitos, a serem conta­dos
separadamente.

Mencionarei agora vários exemplos das duas partes desse segundo tipo. É
possível até que mencione todos os preceitos negativos desta categoria
para que o assunto fique completamente elucidado.

Contamos a proibição expressa em Suas palavras, enaltecido seja Ele,
relativa ao cordeiro Pascal "Não comais dela mal passada no fogo nem
cozida na água" (Êxodo 12:19) como um preceito, e não "Não comais dela
mal passa­da no fogo" como um e "Não comais dela cozida na água" como
outro precei­to, pois Ele não expressou especificamente uma declaração
proibitiva em cada assunto dizendo "Não comais dela mal passada nem
tampouco a comais cozida na água". Em vez disso, Ele expressou uma
proibição que inclui dois assuntos ligados e relacionados entre si.

No capítulo de Pessahim os Sábios dizem: "Abayé disse que se ele a comeu
mal passada, ele deve ser açoitado duas vezes; se a comeu cozida na
água, duas vezes; e se a comeu mal passada e cozida na água, três
vezes". Isso se deve ao fato de que ele acredita que se deve ser
açoitado por desobedecer uma proibição global; portanto, quando alguém a
comer mal passada, estará transgredindo duas proibições: a que diz "Não
comais dela mal passada no fo­go" e a que se conclui por dedução, que é
como se Ele tivesse dito: "Coma-a apenas grelhada", ao passo que ele a
comeu de outra forma. E se ele a comeu mal passada e cozida na água, ele
será açoitado três vezes, de acordo com Aba­yé: uma por comê-la mal
passada, outra por comê-la cozida na água, e uma ter­ceira por tê-la
comido sem ser grelhada.

Continuando este debate, disseram ali: "Mas Rabá diz que não se fi­ca
sujeito ao açoitamento por uma proibição global. Há quem diga que se
fica sujeito a pelo menos um açoitamento", isto é, se a comer mal
passada e cozida na água, será açoitado uma vez. "Outros dizem que não
se fica sujeito a açoita­mento algum, pois esta não é tão específica
quanto a proibição de colocar uma focinheira". Este é o preceito "Não
amarrarás a boca ao boi quando estiver de­bulhando" (Deuteronômio 25:4),
que consiste de uma proibição advertindo con­tra fazer uma única coisa,
enquanto que a outra proibição adverte com relação a duas coisas --- mal
passada e cozida em água --- e conseqüentemente não su­jeita o
transgressor a açoitamento algum.

OS 14 FUNDAMENTOS 61

Você já está familiarizado com o que está explicado na Guemará de
Sanhedrin: "Não se fica sujeito ao açoitamento por uma proibição
global". Con­seqüentemente, as palavras de Abayé são rejeitadas, sendo a
opi 1. correta a que diz que se fica sujeito a apenas um açoitamento,
quer se enha c mido mal passada ou cozida na água, ou mal passada e
cozida na á: a Por essa ,azão devemos contar Suas palavras, enaltecido
seja Ele, "Não com\textsuperscript{,} i dela mal pq ssa­da no fogo nem
cozida na água" como um preceito apena

Ali disseram também os Sábios: "Abayé disse qu se \textsuperscript{19}
co \textsuperscript{-}sse a casca da uva, ele seria açoitado duas vezes;
se comesse caro • s de u , duas vezes; se comesse cascas e caroços de
uva, três vezes. Mas R. que não se fica sujeito ao açoitamento por causa
de uma proibição global", fazendo alu­são a Suas palavras "De tudo o que
sai da videira" (Números 6:4) pelo que, na opinião de Abayé, fica-se
sujeito ao açoitamento.

Da mesma forma eles dizem, no quarto capítulo de Menahot: "Aba­yé disse
que aquele que oferecer lêvedo e mel sobre o altar deverá ser açoitado
uma vez pelo lêvedo, uma vez pelo mel, uma vez por ter misturado o
lêvedo e uma vez por ter misturado o mel" (12). Ou seja, a palavra "col"
inclui duas coisas: que essas coisas não devem ser oferecidas separadas
nem misturadas, seja em que quantidade for. Como você já sabe, tudo isso
está de acordo com a teoria essencial de Abayé de que se está sujeito ao
açoitamento por uma proi­bição global. E os Sábios prosseguem, dizendo:
"Mas Rabá diz que não se fica sujeito a castigo por causa de uma
proibição global. Alguns dizem que se está sujeito a pelo menos um
açoitamento, e outros dizem que não se está sujeito a nenhum, uma vez
que ela não é tão específica como a proibição de colocar uma
focinheira".

Então, como foi explicado que os versículos "Não comais dela mal passada
no fogo nem cozida na água" e "Não fareis queimar fermento algum ou mel"
(Levítico 2:11) constituem um preceito cada um, assim também conta­remos
cada um dos seguintes versículos como um só preceito: "Não entrará
nenhum amonita e nem moabita" (Deuteronômio 23:4); "A nenhuma viúva ou
órfão afligireis" (Êxodo 22:21); "Não perverterás o juízo do peregrino e
do ór­fão" (Deuteronômio 24:17,); "Sua manutenção, seu vestuário, e seu
direito con­jugal não lhe diminuirá" (Exodo 21:10); cada uma dessas
proibições é idêntica às proibições mencionadas: "Não comais dela mal
passada no fogo nem cozida na água" e "Não fareis queimar fermento algum
ou mel". Não há diferença en­tre elas.

Do mesmo modo, Suas palavras "Não trarás salário de rameira nem preço de
um cão" (Deuteronômio 23:19) constituem um preceito negativo. E esse
também é o caso de Suas palavras "Vinho e bebida forte não bebereis...
quando entrardes à tenda da revelação... e para ensinar" (Levítico
10:9-11). Quer dizer, num único preceito Ele advertiu para que não se
entre no Santuário nem se dê instruções sobre a Torah em estado de
embriaguez. Esta é uma das duas partes do segundo tipo de proibição
global.

A segunda parte trata precisamente do mesmo tipo de expressão que a
primeira, exceto que aqui a instrução da Tradição é de que cada assunto
liga­do e acrescido acarreta um açoitamento em separado e que caso eles
sejam to­dos transgredidos, mesmo que seja todos de uma só vez, fica-se
sujeito ao açoi­tamento por cada um deles. É em casos como estes que
devemos contar cada conceito como uma proibição separada.

62 MAIMÔNIDES

Esse é o caso de Suas palavras "Não te será permitido comer em tuas
cidades o dízimo de teus cereais, e de teu mosto, e de teu azeite"
(Deuteronô­mio 12:17), com relação às quais os Sábios dizem na Guemará
Queretot: "Se al­guém comer do dízimo dos cereais, do mosto e do azeite
ele será culpado por cada um deles separadamente". A esse respeito eles
perguntaram: "Mas é-se açoi­tado por uma proibição global?" E a resposta
foi: "O texto é repetitivo. Veja: a Torah já disse: 'Comerás diante do
Eterno, teu Deus,... o dízimo de teu grão, teu mosto e teu
azeite'(Ibid., 14:23). Por que ela determina 'Não te será permiti­do
comer em tuas cidades'? E se você disser que é para estabelecer uma
proibi­ção, então que a Torah diga: 'Não te será permitido comê-los'. P
• • ue ela enun­cia outra vez todos detalhadamente? Só pode ser para
estabele -los e separado.

Está explicado ali que se é culpado por cada uma eparad ente tam­bém no
caso de Suas palavras, enaltecido seja Ele, "E pão, e arinha fei rs de
grãos de espigas verdes, torrada no forno, e grãos verdes de cer
is\textsuperscript{20} não omereis" (Levítico 23:14). Os Sábios dizem:
"Aquele que come pã•rin de grãos de espiga verdes e grãos verdes de
cereais é culpado por cad um deles separada­mente. Mas fica-se sujeito
ao açoitamento por uma proibi ão global? O texto é repetitivo. Que o
Misericordioso escreva uma e as outras serão deduzidas de­la". Depois de
uma discussão a respeito foi explicado que não havia necessida­de de que
Ele mencionasse a "farinha de grãos de espigas verdes", e que isso foi
mencionado para estabelecer uma separação: para sujeitar-nos ao
açoitamento no caso dessa farinha separadamente. O Talmud continua o
debate perguntan­do: Talvez se fique sujeito ao açoitamento por causa da
farinha de grãos, uma vez que ela foi mencionada com esse objetivo, mas
será que se fica sujeito a um único açoitamento por comer pão e farinha
de grãos verdes? A resposta foi: "Por que motivo o Misericordioso
escreveu 'farinha de grãos verdes' no meio? Para ensinar-nos que o pão é
como a farinha de grãos, e que esta é como os grãos verdes", de maneira
que se é culpado por cada um individualmente.

O mesmo direi com relação a Sua declaração, enaltecido seja Ele, "Nãó se
achará entre ti quem faça passar seu filho ou sua filha pelo fogo, nem
agoureiro, nem prognosticador, nem adivinho, nem feiticeiro, nem
encanta­dor, nem necromante ou Yideonita, nem quem consulte os mortos"
(Deutero­nômio 18:10-11); cada uma das nove coisas enumeradas são
contadas como um preceito individual e nenhuma delas pertence à primeira
parte do segundo ti­po. Prova disso é que Suas palavras, enaltecido seja
Ele, "Nem prognosticador, nem adivinho" estão no meio da frase, pois já
foi explicado que em Suas pala­vras "Não augurareis e não
prognosticareis' (Levítico 19:26) cada uma dessas proibições constitui
um preceito em si. Assim como o prognosticador e o adi­vinho ---
mencionados no meio --- são casos separados, também todos os ou­tros
casos mencionados antes e depois são semelhantes a eles, tal como
expli­cam os Sábios no caso do "pão, da farinha de grãos e dos grãos
verdes".

Outros se enganaram a respeito désta questão, seja porque suas men­tes
não compreenderam em absoluto estes assuntos, ou então porque eles se
esqueceram e se desviaram do rumo correto. Assim, eles contaram Suas
pala­vras, enaltecido seja Ele, com respeito aos "Cohanim", "Mulher
prostituta ou profana não tomarão, nem mulher divorciada de seu marido
não tomarão" (Ibid., 21:7) como um único preceito, embora já estivesse
explicado na Guemará de Kidushin que se fica sujeito a castigo por cada
uma dessas desqualificações, mes­mo que todas elas digam respeito a uma
única mulher, como explicaremos no local apropriado. De fato, poderíamos
encontrar uma desculpa por contar uma

20. Da nova colheita, até que se traga a oferenda do "omer".

OS 14 FUNDAMENTOS 63

prostituta e uma mulher profana como um preceito, porque tendo
compreen­dido alguns detalhes da proibição global ele considerou Suas
palavras, enalteci­do seja Ele, "Mulher prostituta ou profana não
tomarão" semelhante a "Não comais dela mal passada no fogo nem cozida na
água" e não percebeu que a primeira proibição estabelece uma separação e
a segunda não. Também não di­ferenciou o versículo "E pão, e farinha
feita de grãos de espigas verdes, torra­das no forno, e grãos verdes de
cereais não comereis" de "Sua manutenção, seu vestuário e seu direito
conjugal não lhe diminuirá". Contudo, não vou cri­ticá-lo em casos como
estes. Mas não há desculpa por ter contado uma mulher divorciada junto
com uma prostituta e uma profana, incluindo-as todas num só preceito,
pois a mulher divorciada constitui claramente uma proibição sepa­rada,
como Ele disse, enaltecido seja Ele: "Nem mulher divorciada de seu
mari­do não tomarão".

Portanto, deixamos claro este grande Fundamento --- isto é, a proi­bição
global --- e explicamos as dúvidas relativas a ele. Também esclarecemos
em que casos ele estabelece uma separação e em que casos há apenas uma
proi­bição global, sujeitando-nos ao castigo apenas uma vez.
Esclarecemos ainda que quando há uma separação deve-se contar todos como
preceitos separados e que quando não há separação deve-se contar um só
preceito. Tenha sempre todo este Fundamento diante de si pois ele é um
guia da maior importância para a enumeração correta dos princípios.

\textbf{O DÉCIMO}

\textbf{FUNDAMENTO}

NÃO SE DEVE CONTAR OS ATOS ESTIPULADOS COMO PRELIMINARES AO CUMPRIMENTO
DO PRECEITO

Ocasionalmente a Torah menciona ordens que não constituem pre­ceitos em
si, mas apenas preliminares para o cumprimento de um preceito, co­mo se
Ele estivesse descrevendo a maneira como o preceito deve ser executa­do.
Um exemplo disso é o versículo "E tomarás a flor da farinha de trigo"
(Le­vítico 24:5). Não seria correto contar o fato de tomar a farinha
como um pre­ceito, e o cozimento dela como outro; o que deve ser contado
é apenas o que Ele diz: "E porás sobre a mesa o pão da proposição diante
de Mim, continua­mente" (Êxodo 25:30), pois o preceito consiste apenas
em que haja sempre pão • diante do Eterno. Depois Ele descreve como deve
ser esse pão e a partir de que ele deve ser feito, dizendo que ele deve
ser feito com flor de farinha e que consiste de doze pães.

Do mesmo modo não devemos contar Suas palavras "Que te tragam azeite de
oliveira puro" (Ibid., 27:20), e sim apenas "para acender a lamparina
contínua" (Ibid.), que é o preceito de manter as lamparinas acesas, como
foi explicado no Tamid.

Assim também não devemos contar Suas palavras "Toma para ti
es­peciarias" (Êxodo 30:34), mas sim a queima diária de incenso, como
dizem as Escrituras: "Pela manhã, quando limpar as lamparinas, o
queimará. E ao acen-

64 MAIMÔNIDES

der Aarão os fogos..." (Ibid.,78). É este versículo que constitui o
preceito a ser contado, ao passo que Suas palavras "Toma para ti
especiarias" são apenas uma preparação, explicando como o preceito deve
ser realizado e de que especia­rias deve ser feito esse incenso.

Da mesma forma não devemos contar Suas palavras "Toma para ti
especiarias principais" (Ibid., 23). O que deve ser contado é a ordem
que nos obriga a ungir os "Cohanim Gadol", os reis e os utensílios
sagrados com o Óleo de Unção descrito.

Você deve julgar todos os casos semelhantes a esses baseando-se neste
critério para que você não aumente a enumeração com tópicos que não
fazem parte dela. Este é nosso objetivo com este Fundamento, e isso está
perfeitamente claro. Porém nós o mencionamos e o comentamos porque
também com rela­ção a este assunto muitos se enganaram, contando um
preceito e seus atos pre­, liminares como dois preceitos, como ficará
claro para quem observar a enume­ração das seções feitas por Shimon
Kayará, bendita seja sua memória, e por seus seguidores.

O DÉCIMO

PRIMEIRO

FUNDAMENTO

NÃO SE DEVE CONTAR SEPARADAMENTE OS DIVERSOS ELEMENTOS QUE COMPÕEM UM SÓ
PRECEITO

Ocasionalmente um preceito pode consistir de muitas partes, como é o
caso do preceito do ramo de palma ("lulav" e "etrog"), que compreende
quatro tipos. Nesse caso não devemos dizer que "o fruto das árvores
formo­sas" e "os galhos das árvores frondosas", e "o salgueiro do
regato" e "os ra­mos das palmeiras" constituem cada um um preceito
separado; ao contrário, eles são todos partes de um preceito, pois Ele
ordenou juntá-los e o preceito consiste em segurá-los na mão, todos
juntos, num dia determinado.

Um caso exatamente igual: não devemos contar Suas palavras que dizem que
o leproso deve purificar-se com dois pássaros, um pau de cedro,
carmezin, hissopo, água corrente e uma vasilha de barro como sendo seis
pre­ceitos; o que constitui o preceito é a purificação do leproso
através de todos os elementos estabelecidos --- os já citados e mais a
raspagem --- sendo que os diferentes elementos nos são impostos para
fazer essa purificação, que é fei­ta de tal e tal maneira.

A mesma lei se aplica com relação aos sinais de reconhecimento que nos
foi ordenado fazer no leproso quando ele estiver em estado de impureza,
a fim de que nos mantenhamos afastados dele, como está dito: "Suas
vestes serão rasgadas e seu cabelo não será cortado, e com seu bigode se
cobrirá; e impuro! impuro! clamará" (Levítico 13:43). Cada um desses
atos não constitui um preceito em si; ao invés disso, é o preceito que
consiste no conjunto deles,

OS 14 FUNDAMENTOS 65

isto é, que devemos fazer com que o leproso possa ser identificado para
que possamos nos manter afastados dele e que essa identificação se
compõe disto e daquilo, tal como nos foi ordenado alegrar-nos diante do
Eterno no primeiro dia dos Tabernáculos, sendo que Ele explicou que essa
alegria consiste em le­var determinados objetos.

Há um aspecto difícil de ser compreendido neste Fundamento e a razão
disso é o que explicarei a seguir: Toda vez que os Sábios disserem, com
relação a um determinado preceito, que um de seus elementos prejudica a
vali­dade de outro, é óbvio que ele constitui um preceito, como por
exemplo as quatro variedades usadas no caso do ramo de palma e o incenso
levado junto com o pão da proposição sobre o qual os Sábios disseram:
"As fileiras e os pra­tos comprometem um a validade do outro"; nesses
casos fica claro que eles constituem um único preceito. Da mesma forma,
toda vez que ficar claro que o objetivo desejado não será obtido através
de um dos elementos, também é óbvio que é o conjunto deles que deve ser
contado. Tal é, por exemplo, o caso explicado sobre o reconhecimento do
leproso, pois se apenas suas roupas fo­rem rasgadas mas se ele não tiver
deixado crescer seu cabelo, não tiver coberto o lábio superior e não
gritar "impuro, impuro", ele não terá efetivado nada; ele só será
identificável quando fizer tudo. Assim também sua purificação não será
alcançada até que ele se utilize de todas as coisas mencionadas: os
pássa­ros, o pau de cedro, o hissopo, o carmezin, e a raspagem. Só assim
sua purifica­ção será alcançada.

Contudo, o aspecto difíc' rge quando os Sábios dizem, com rela­ção aos
elementos, que "eles nã comprometem a validade uns dos outro ". Num
primeiro raciocínio você oncluirá que como cada um desses vário:
ele­mentos não precisa do outro, ada um les deve ser contado como u
pre­ceito independente. Esse é o so, por emplo, de sua afirmação: " 1
zu1\textsuperscript{21} não prejudica a validade do br.
0\textsuperscript{22}, e o branco não compromete o az 1' . Isto pode
levá-lo a concluir que o r. co e azul devem ser contados como dois
preceitos, se não fosse pela afirm xplícita que encontramos na Mekh a de
Rabi Ishmael de que "Poderíamos pensar que estes são dois preceitos ---o
do azul e o do branco ---; por isso as Escrituras afirmam 'E será para
vós por "tsitsit"' (Números 15:39), mostrando assim que se trata de um
preceito e não de dois".

Assim, foi explicado que mesmo que os elementos não prejudiquem a
validade uns dos outros, eles às vezes são um único preceito, desde que
o significado deles seja um só. Tal é o caso dos "tsitsit", onde o
objetivo é "Para que vos lembreis" (Números 15:40) e portanto o que deve
ser contado é o con­junto de coisas que vai fazer com que nos lembremos.

Assim sendo, quando formos enumerar os preceitos, resta-nos não prestar
atenção quanto a se cada elemento compromete a validade do outro ou não,
e sim fixarmo-nos em seu conceito para saber se trata-se de um ou de
vários, tal como explicamos no nono dos Fundamentos que estamos tentando
elucidar.
\end{quote}

\begin{enumerate}
\def\labelenumi{\arabic{enumi}.}
\setcounter{enumi}{20}
\item
  \begin{quote}
  O cordão azul do "tsitsit".
  \end{quote}
\item
  \begin{quote}
  O cordão branco do "tsitsit".
  \end{quote}
\end{enumerate}

\begin{quote}
66 MAIMÔNIDES

\textbf{O DÉCIMO}

\textbf{SEGUNDO}

\textbf{FUNDAMENTO}

NÃO SE DEVE CONTAR SEPARADAMENTE AS ETAPAS

SUCESSIVAS NA EXECUÇÃO DE UM PROCESSO

Às vezes somos ordenados a executar uma determinada ação, e lo­go em
seguida a Torah começa a explicar como essa ação deve ser executada,
elucidando a expressão que usou e definindo o que está incluído nela. Em
casos assim não devemos contar cada ordem contida na explicação como um
preceito individual. Por exemplo: Suas palavras "E me farão um
santuário" (Êxodo 25:8) constituem um dos preceitos positivos, que é o
de que devemos construir uma casa para a qual devemos nos dirigir, para
onde devemos ir a fim de oferecer os sacrifícios e onde as assembléias
terão lugar, durante os festivais. Depois disso Ele começa a descrever
seus detalhes e como eles devem ser executados; esses atos específicos
--- cada um deles precedido pela expressão "E me farão" --- não devem
ser contados como preceitos separados.

O mesmo acóntece com relação aos sacrifícios mencionados no "Va­yikrá",
onde um preceito consiste de todo o ritual descrito em cada um dos
vários tipos de sacrifícios. Um exemplo disso é o holocausto, cujo
ritual que nos foi ordenado é o seguinte: que ele seja degolado, que sua
pele seja retirada, que seja cortado em pedaços, que seu sangue seja
derramado de tal e tal manei­ra, é que sua gordura seja queimada,
seguida da queima de toda sua carne, jun­tamente com uma determinada
medida de flor de farinha misturada com óleo e com uma certa quantia de
vinho --- que são as libações --- e que o couro seja dado ao "Cohen" que
estiver oficiando. A totalidade deste ritual constitui um preceito
positivo, que é a lei do holocausto, sendo que a Torah nos obriga a
executar cada holocausto dessa maneira.
\end{quote}

Um caso semelhante é o do ritual completo do Sacrifício de Pecado:

\begin{itemize}
\item
  \begin{quote}
  degolamento, a retirada do couro, a queima do que deve ser queimado, a
  lavagem das vasilhas nas quais deve ser derramado parte do sangue, e a
  lava­gem ou a quebra das vasilhas nas quais foi cozida a carne. Tudo
  isso é a lei do Sacrifício de Pecado e constitui um único preceito.
  \end{quote}
\end{itemize}

\begin{quote}
Da mesma forma, a lei do Sacrifício de Delito constitui um só pre­ceito,
assim como a "lei do Sacrifício de Oferta de Paz", oferecido como
sacri­fício de graças, com ou sem pão, quando o "Cohen" pega o peito e a
coxa e os levanta, sendo que tudo isso é o ritual do "Sacrifício de
Oferta de Pazes"
\end{quote}

\begin{itemize}
\item
  \begin{quote}
  constitui um só preceito.
  \end{quote}
\end{itemize}

\begin{quote}
Esses compõem a totalidade dos sacrifícios cuja obrigação cabe ao
indivíduo e à congregação, com a exceção do Sacrifício de Delito que é
sempre uma oferta individual, como explicamos na introdução à Ordem de
Kadashim.
\end{quote}

\begin{itemize}
\item
  \begin{quote}
  que constitui o preceito positivo é o procedimento nos vários rituais
  e não se deve contar cada detalhe desses rituais como um preceito
  separado, a não
  \end{quote}
\end{itemize}

\begin{quote}
OS 14 FUNDAMENTOS 67

ser que eles sejam ordens que abranjam todos os diversos tipos de
sacrifícios e que não sejam específicos para apenas um deles, excluindo
os outros; tais or­dens devem ser contadas como preceitos em separado,
já que elas não são me­ros detalhes do ritual de algum sacrifício
específico. Esse é o caso, por exem­plo, de Sua advertência para não
trazer um sacrifício defeituoso, ou de sua or­dem para que seja
perfeito, ou de que não ofereçamos um animal que não te­nha atingido a
idade de ser aceito, conforme consta em Suas palavras "Do oita­vo dia em
diante..." (Levítico 22:27); ou de que ofereçamos sal com todos os
sacrifícios, como Ele disse: "Toda tua oferta de oblação temperarás com
sal" (Ibid., 2:13); ou de que não deixemos faltar o sal num sacrifício,
como Ele dis­se: "Não deixarás faltar o sal" (Ibid.); ou de que sejam
comidas as partes que devem ser comidas. Cada um desses casos constitui
um preceito independen­te, pois eles não são meros detalhes no ritual de
um sacrifício específico; ao contrário, eles são ordens que abrangem
todos os sacrifícios, como explicare­mos em nossa enumeração.

Está claro que o que o "Cohen" toma como sua parte do sacrifício é
apenas um dos detalhes do preceito, tal como mencionamos com relação ao
couro da oferta de Holocausto. Esse também é o caso da primeira tosquia,
on­de a essência do preceito consiste em separar a primeira lã do
carneiro e dá-la ao "Cohen", assim como no caso do primeiro dízimo, que
devemos separar e dar ao Levita.

Outros se enganaram a este respeito, e contaram os vinte e quatro tipos
de presentes ao "Cohen" como vinte e quatro preceitos, depois de ter
contado alguns preceitos nos quais alguns desses presentes eram meros
deta­lhes, tal como explicamos com relação ao couro da oferenda de
Holocausto e ao peito e coxa do Sacrifício de Paz.

Além disso, devido ao fato de que eles não conheciam este Funda­mento,
não se aperceberam dele, nem lhe prestaram atenção, resultando que eles
contaram, como preceitos separados, verter (óleo nos utensílios),
embe­ber (com óleo), cortar em pedaços, salgar, levar junto ao altar,
levantar, retirar um punhado e queimar, sem saber que todos esses atos
são detalhes do ritual da oblação. Ou seja, depois de ter-nos ordenado
oferecer a oblação de trigo, Ele começou a explicar a que se refere esse
nome --- o da "lei da oblação de trigo" --- e disse que se tratava de
flor de farinha, ou pão assado de uma certa forma --- sobre uma chapa,
numa panela a vapor ou no forno ---; depois de­ve-se embebê-la numa
determinada quantia de óleo, separá-la em pedaços e por nela sal e
incenso e deve-se levá-la junto ao altar e erguê-la, tomar um punhado
dela e queimá-la, de acordo com o procedimento que explicamos e
elucidamos no lugar apropriado: o Tratado Menahot. Todos esses são
detalhes do ritual que, quando executado de acordo com todo este
procedimento, é chamado de obla­ção. Assim sendo, o preceito é o
seguinte: somos obrigados a que o ritual do sacrifício do pão \%I da
flor de farinha esteja de acordo com o procedimento assim definido. No
caso do preceito da oblação a oferenda consiste do seguin­te: verter,
embeber, separar em pedaços, salgar, levantar, levar junto ao altar,
pegar um punhado e queimá-la, tal como Ele disse no caso do preceito
único da "Halitzá": "E lhe descalçará o sapato do pé, e cuspirá no chão,
diante dele, e responderá dizendo..." (Deuteronômio 25:9), onde não
contamos o fato de tirar o sapato, o de cuspir e o de proferir as
palavras como preceitos indepen­dentes, uma vez que eles estão incluídos
no ritual de "Halitzá", que constitui um preceito. Da mesma forma que
nesse caso, também não devemos contar separadamente as seguintes ordens:
"E deitarás sobre ela azeite" (Levítico 2:6), "E porás sobre ela
incenso" (Ibid., 15), "Temperarás com sal" (Ibid., 13), "E

68 MAIMÔNIDES

movimentará o sacerdote" (Ibid., 23:20), e a aproximará, "E tirará" dali
um pu­nho cheio... e o fará queimar" (Ibid., 2:2).

Isto só passará desapercebido a quem compreender os assuntos
su­perficialmente, sem examiná-los e avaliá-los em sua mente, tal como
dizem os Sábios de abençoada memória: "Ele disse isso por ser
precipitado". Quer di­zer, ele disse isso sem refletir a respeito,
baseado apenas no primeiro pensa­mento que lhe veio à mente.

Assim, este Fundamento nos explicou as leis de todos os sacrifícios e a
maneira como elas devem ser contadas para que não haja nenhum engano nem
confusão, como as explicaremos em nossa enumeração, com a ajuda do Todo
Poderoso.

\textbf{O DÉCIMO}

\textbf{TERCEIRO}

\textbf{FUNDAMENTO}

QUANDO UM DETERMINADO PRECEITO TIVER QUE SER CUMPRIDO POR VÁRIOS DIAS
NÃO SE DEVE CONTAR UM PRECEITO POR CADA DIA

rios durante o transcor­se período é contínuo, eito se sucede dia após
utras vezes ele corres-o de exemplo, se devês­itui um preceito, isto
sig­nificaria que fomos ordenados a levar um sacrifício adicional toda
vez que hou­ver lua nova. Se alguém perguntasse por que não contamos o
sacrifício adicional de cada lua nova como um preceito em si, diríamos
que se assim fosse, você tam­bém deveria contar o Holocausto de cada dia
como um preceito em si, assim como contar queimar o incenso e manter as
lamparinas acesas, os quais são obri­gatórios todos os dias do ano,
também como preceitos individuais. Mas como contamos apenas o conceito
do que nos foi ordenado, sem levar em considera­ção o fator tempo com
relação a sua execução, nós contamos o sacrifício adicio­nal da lua nova
como um único preceito, assim como o sacrifício adicional do Shabat e de
cada um dos cinco festivais, mesmo que eles sejam obrigatórios por
vários dias seguidos. Pois assim como Ele diz: "E vos alegreis diante do
Eterno, vosso Deus, por sete dias" (Levítico 23:40), Ele diz também
"Sete dias oferece­reis ofertas queimadas ao Eterno" (Ibid., 36); assim
como o preceito do ramo de palma é um só, também é um só o preceito do
sacrifício adicional de Páscoa. A mesma regra se aplica aos sacrifícios
adicionais de cada uma das estações.
\end{quote}

\begin{enumerate}
\def\labelenumi{\arabic{enumi}.}
\setcounter{enumi}{22}
\item
  \begin{quote}
  Sentar na "sucá" durante sete dias.
  \end{quote}
\item
  \begin{quote}
  O preceito do "lulav".
  \end{quote}
\end{enumerate}

\begin{quote}
OS 14 FUNDAMENTOS 69

Com base neste Fundamento também ficará claro que o sacrifício de festa
é um único preceito, embora ele seja obrigatório nas três estações,
co­mo também o é o de comparecer e o de alegrar-se. Ninguém deve
enganar-se nem pensar de maneira diferente.

Contudo, alguns cometeram um erro extremamente sério e estranho com
relação a este Fundamento: eles contaram todos os sacrifícios adicionais
--- o do Shabat, o das luas novas e o das festas --- como um único
preceito! Se assim fosse, eles deveriam ter contado o descanso em todos
os festivais co­mo um preceito, mas não o fizeram. Mas o Eterno sabe e é
testemunha de que eles não devem ser criticados por isso, pois, de uma
maneira geral, eles não seguiram uma teoria ao fazer suas enumerações;
ao contrário, "Eles subiram até o céu, eles desceram às profundezas"
(Ps. 107:26). A verdade é o que eu lhes mencionei: que cada sacrifício
adicional constitui um preceito indepen­dente, assim como o descanso em
cada um dos festivais constitui um preceito diferente. Esta é a teoria
correta.

O DÉCIMO

QUARTO

FUNDAMENTE)

DE QUE FORMA OS TIPOS DE CASTIGO DEVEM SER CONTADOS

COMO PRECEITOS POSITIVOS

Você deve saber que todos os preceitos, positivos e negativos, estão
primeiramente divididos em duas partes, de acordo com o propósito deste
Fun­damento. A primeira parte é aquela em que a Escritura não
especificou castigo algum, mas apenas estabeleceu uma ordem ou uma
proibição, e não sujeitou o transgressor a qualquer castigo nem lhe
designou um castigo determinado por transgredir aquela ordem ou
proibição específica. A segunda parte é a que estabelece a recompensa e
o castigo.

Entre os preceitos nos quais Ele explica o castigo estão os preceitos
que nos ordenam a apedrejar os transgressores de determinados preceitos,
a queimá-los, a executá-los com a espada como foi indicado na explicação
da Tra­dição, a estrangulá-los, e a açoitá-los com uma correia. A
determinados trans­gressores Ele impôs a extinção, isto é, o
transgressor que morrer em estado de pecado não terá um lugar no Mundo
que Há de Vir, conforme explicamos no capítulo "Helek" ; a outros Ele
impôs apenas a morte, isto é, Ele fará com que morram por seu pecado e
sua morte lhes trará a absolvição.

Os Sábios explicam no início de Macot que no caso de uma proibi­ção cujo
castigo é a extinção ou apenas a morte pela Mão dos Céus --- se se
concluir que o transgressor pecou premeditadamente diante de testemunhas
e apesar das advertências --- o transgressor está sujeito ao
açoitamento, mesmo que seu castigo principal consista em que seu
julgamento caberá ao Céu. Há também preceitos nos quais Ele nos ordenou
castigar os transgressores de cer­tos preceitos apenas com seu dinheiro,
não com seu corpo, tal como determi-

70 MAIMÔNIDES

nou a um assaltante que ele acrescentasse um quinto adicional e a uM
ladrão que pagasse o dobro do que roubou. E também há preceitos em que
Ele nos ordenou que os violadores levem um sacrifício por seu pecado
para serem as­sim perdoados.

As aplicações de todas essas formas de castigo constituem preceitos
positivos, pois fomos ordenados a matar um, a açoitar outro, a apedrejar
aque­le outro, ou a levar um sacrifício pelo que fizemos. Quanto a
inclui-los na enu­meração, devemos contar as quatro penas de morte
impostas pelo Tribunal co­mo quatro preceitos positivos. Tal é, na
realidade, a expressão da Mishná: "Es­te é o preceito dos que devem ser
apedrejados". Eles dizem também: "De que maneira deve ser o preceito de
queimar?", "De que maneira deve ser o precei­to de estrangular?", "De
que maneira deve ser o preceito de decapitar?".

Os Sábios também dizem que Suas palavras, enaltecido seja Ele, "Não
acendereis fogo" ( Êxodo 35:3), são uma advertência para que não se
aplique castigos no Shabat. Ou seja, este versículo nos proíbe de
executar um preceito que nos ordene queimar alguém, pois a expressão "Em
todas as vossas habita­ções" (Ibid.) significa que não se deve acender
fogo no Tribunal, mesmo que isso seja um preceito positivo. Portanto os
Sábios dizem: "Acender um fogo, que está incluído nas categorias de
tarefas proibidas no Shabat, está destacado para ensinar-nos que assim
como as leis do Shabat não podem ser desconside­radas no que se refere
ao tipo de execução especificamente mencionado, elas também não podem
ser desconsideradas no caso de outros tipos de execução judicial". Isto
está claro e ninguém terá dúvidas a respeito. Da mesma forma, devemos
contar o açoitamento com uma correia como um preceito individual.

Entretanto, não se deve fazer o que fizeram outros sem meditar, ou seja,
contar cada castigo em particular como um preceito separado dizendo, a
título de exemplo, que a ordem de apedrejar aquele que profanar o Shabat
é um preceito, que o apedrejamento daquele que pratica o "Ob" é um
segun­do preceito, e que o apedrejamento daquele que adora ídolos é um
terceiro preceito, resultando que o número de preceitos corresponderá ao
número de pessoas sujeitas às quatro penas de morte impostas pelo
Tribunal. Se assim fos­se, nós fatalmente teríamos que contar cada
açoitamento em separado, fazendo com que o açoitamento de quem come
"nebelá" seja um preceito individual, o de quem come carne de porco um
segundo preceito, o de quem come carne cozida no leite um terceiro
preceito, o de quem usa "shaatnez" um quarto pre­ceito, resultando assim
que teríamos tantos preceitos positivos quanto o núme­ro de preceitos
negativos que acarretam o açoitamento. Dessa maneira (inú­mero de
preceitos positivos aumentaria e chegaria com certeza a mais de
qua­trocentos! Ao invés disso, assim como não contamos separadamente
todos os que estão sujeitos ao açoitamento, e sim apenas o tipo de
castigo --- a saber, o açoitamento com uma correia ---, devemos contar
também nas penas de morte apenas as formas de execução, que são pelo
fogo, por apedrejamento, por es­trangulamento e por decapitação. Da
mesma forma, não devemos contar sepa­radamente todos os que estão
sujeitos a oferecer um sacrifício, dizendo que o sacrifício de pecado
daquele que viola o Shabat sem querer é preceito, e que o Sacrifício de
Pecado de quem adora ídolos sem querer é preceito; em vez disso devemos
contar apenas o tipo de sacrifício, tal como fizemos nos tipos de pena
de morte.

Você já sabe que o tipo de sacrifício que sé é obrigado a oferecer varia
de acordo com o tipo de pecado que se cometeu. Há pecados pelos quais se
oferece o Sacrifício de Pecado, ou o Sacrifício Suspensivo de Delito, ou
o Sacrifício Incondicional de Delito, ou o Sacrifício de Maior ou Menor
Valor.

OS 14 FUNDAMENTOS 71

É por essa razão que não devemos contar o Sacrifício de Pecado
juntamente com o Sacrifício de Delito; ao invés disso, contaremos a
obrigação do Sacrifí­cio de Pecado, a do Sacrifício Suspensivo de
Delito, a do Sacrifício Incondicio­nal de Delito e a do Sacrifício de
Maior ou Menor Valor como preceitos separa­dos, sendo que as obrigações
são de responsabilidade da pessoa que deve ofe­recer aquele tipo
específico de sacrifício. Não voltaremos nossa atenção para os vários
pecados pelos quais se é obrigado a oferecer um sacrifício específico,
da mesma forma que contamos o açoitamento como um só preceito e
descon­sideramos os vários pecados que acarretam esse castigo. Da mesma
forma, a Escritura dedicou um seção especial a cada tipo.

Outros já fizeram tanta confusão com relação a este Fundamento que se
torna desnecessário refutá-los, nem seria fácil fazê-lo, tamanha é a
desordem que eles implantaram a este respeito.

De fato, devemos espantar-nos e surpreender-nos com uma pessoa que conta
um a um, entre os preceitos negativos, todos aqueles que estão sujei­tos
a alguma das penas de morte aplicadas pelo Tribunal, bem como os que
estão sujeitos à extinção e à morte, além de contar também os atos
proibidos cuja violação implica numa daquelas formas de morte! Foi isso
o que fez o au­tor do Halachot Guedolot. Ele contou "quem profanar o
Shabat" entre aqueles que estão sujeitos à morte por apedrejamento, e
depois contou "Não farás ne­nhuma obra" (Êxodo 20:10). Devemos de fato
concluir que eles pensaram ini­cialmente que a execução judicial
constitui um preceito negativo em si. Mas se assim fosse, como poderiam
eles conter o castigo e a proibição específica pela qual se aplica o
castigo?

Ainda mais surpreendente é o fato de que eles contaram entre os
pre­ceitos negativos os que estão sujeitos à extinção, bem como os que
estão sujei­tos à morte pela Mão dos Céus, os quais não envolvem
execução! Deve ser por­que eles imaginam que ficar sujeito à extinção e
que a aplicação do castigo cons­tituem a natureza desse preceito
específico. De fato, foi assim que o autor do "Sefer Hamitzvot" explicou
isso ao resumir o conteúdo do primeiro capítulo nas seguintes palavras:
"Entre estes há trinta e dois conceitos sobre os quais Ele nos informa
que Ele --- abençoado e enaltecido seja ---, e não nós, o fará
seguramente cumprir." "Entre estes" significa entre os conceitos
mencionados naquele capítulo. Os "trinta e dois conceitos" compreendem,
de acordo com sua enumeração, vinte e três casos sujeitos apenas à
extinção e nove sujeitos à morte pela Mão dos Céus, e procede a sua
\href{http://enumeração.Com}{{enumeração. Com}} a palavra "segu­ramente"
ele quer dizer que o Eterno, enaltecido seja Ele, assegurou que Ele
aplicará a extinção a este e a morte ao outro. Não há dúvidas de que
esse ho­mem se separou completamente da idéia de que \emph{todos} os
seiscentos e treze pre­ceitos são obrigação \emph{nossa;} em vez disso,
alguns seriam nossa obrigação e ou­tros obrigação d'Ele, enaltecido seja
Ele, tal como ele afirmou claramente: "Ele os fará cumprir, e não nós".
Deus sabe e é testemunha de que na minha opi­nião tudo isto é uma
confusão absoluta e não há necessidade alguma de falar a respeito pois a
invalidez de suas palavras é óbvia. O motivo deste erro é que ao contar
os castigos como preceitos eles se confundiram ,e algumas vezes
con­taram os castigos e também as ações que os acarretam, estabelecendo
tudo co­mo preceito negativo, sem refletir a respeito.

Contudo, a forma correta de enumeração é como eu mencionei: ca­da
\emph{tipo} de castigo constitui um preceito positivo. De acordo com
isso, a lei de restituição referente a um ladrão é um preceito positivo,
a saber, que somos ordenados a impor-lhe uma determinada quantia. Assim
também são as seguin­tes leis: o quinto adicional, a obrigação do
Sacrifício de Pecado, o Sacrifício

72 MAIMÔNIDES

Incondicional de Delito, o Sacrifício Suspensivo de Delito e o
Sacrifício de Maior ou Menor Valor. Da mesma forma, cada um dos
seguintes castigos: apedreja­mento, queima, decapitação, estrangulamento
e enforcamento constitui um pre­ceito individual que se aplica a todo
aquele que ficar sujeito a eles, tal como o açoitamento com a correia
constitui um só preceito que se aplica a todo aquele que ficar sujeito a
esse castigo. Isso é o que desejávamos expor neste Funda­mento, e com
ele completamos os Fundamentos, cuja introdução ajudará na­quilo em que
nos empenhamos.

É importante acrescentar a seguinte introdução. Todo pecado, pelo qual a
penalidade é a execução judicial ou a extinção, é necessariamente um
preceito negativo, a não ser o sacrifício de Páscoa e a circuncisão, que
acarre­tam a extinção mesmo sendo preceitos positivos, como explicamos
no início do Tratado Queretot. Não temos nenhum outro preceito positivo
a não ser es­ses por cuja transgressão se fique sujeito à extinção --- e
mais ainda, à execução judicial. Sendo assim, toda vez que a Torah
disser que aquele que cometer um determinado ato deverá ser morto ou
estará sujeito à extinção saberemos per­feitamente que esse ato
específico está proibido e constitui um preceito negativo.

As vezes as Escrituras apresentam uma proibição sem expor o casti­go,
embora tanto o castigo como a advertência estejam claros. Esse, por
exem­plo, é o caso da profanação do Shabat e da adoração de ídolos, com
relação aos quais Ele disse: "Não farás nenhuma obra" (Êxodo 20:10) e
"Nem os servi­rás" (Ibid., 5) e depois disso Ele declara que aquele que
trabalha e aquele que serve estão sujeitos ao apedrejamento.

Algumas vezes as Escrituras não mencionam claramente a proibição de um
determinado ato mas declaram apenas a punição, omitindo a advertên­cia.
Contudo, como é regra para nós que "não há nenhum castigo estipulado na
Torah sem que uma advertência o tenha precedido" deve haver de alguma
forma uma advertêcia com relação ao ato que nos sujeita ao castigo .
Isto é o que os Sábios sempre dizem: "Ouvimos o castigo, mas não ouvimos
a proibi­ção. Por isso a Torah diz isto e aquilo". E se a advertência
não estiver explicita­mente enunciada nas Escrituras, eles a deduzem
através de um dos Princípios. Esse é, por éxemplo, o caso do que os
Sábios dizem com relação às proibições de amaldiçoar ou de bater no pai,
as quais não estão de modo algum explicita­mente mencionadas nas
Escrituras, pois em lugar algum está dito "Não amaldi­çoe seu pai" nem
"Não bata em seu pai"; em vez disso Ele declarou que aquele que bater ou
amaldiçoar está .sujeito à morte. Á partir dessas palavras deduzi­mos
que estes atos são proibidos e que os Sábios extraíram delas, através de
um dos princípios exegéticos, as advertências referentes a esses atos,
assim co­mo fizeram em casos semelhantes, em outros lugares.

Este método de dedução de uma advertência não contradiz de for­ma alguma
o que os Sábios frequentemente dizem: "Uma lei derivada por ana­logia
não é considerada uma lei específica pela qual se possa ser castigado",
nem o princípio de que "Acaso se pode castigar pela violação de uma lei
cuja advertência se deduziu por analogia?" O objetivo da declaração "Uma
lei deri­vada por analogia não é considerada uma lei específicg pela
qual se possa ser castigado" é apenas de proibir-nos de derivar por
analogia um assunto com re­lação ao qual não haja proibição
\emph{alguma} mencionada; mas se encontrarmos a punição contra fazer um
determinado ato claramentç exposta na Torah, sabe­remos com certeza que
esse ato está proibido e que fomos advertidos para não fazê-lo. E apenas
para estar de conformidade com a regra de que "Nenhum cas­tigo está
estabelecido na Torah a menos que uma advertência proibitiva o te­nha
precedido" que fazemos deduções a partir de um dos princípios, quando

OS 14 FUNDAMENTOS 73
\end{quote}

Ele tiver se referido a essa advertência. E uma vez encontrada a
advertência con-\\
tra esse ato, o violador que o fizer fica sujeito à extinção ou à
execução judicial.

\begin{quote}
Assim, fique ciente desta introdução e recorde-se dela e de todos os
Fundamentos precedentes juntamente com tudo o que tencionamos mencio­nar
a seguir.

Agora começarei a mencionar todos os preceitos, um a um, expli­cando-os
apenas a fim de elucidar o título do preceito, como prometemos no início
de nosso estudo, pois esse é o objetivo deste trabalho. Entretanto,
creio que é aconselhável acrescentar o seguinte ao nosso objetivo.
Quando me refe­rir a um preceito, positivo ou negativo, que acarreta
algum castigo, menciona­rei o castigo dizendo "Aquele que o violar está
sujeito à morte, ou à extinção, ou a oferecer determinado sacrifício, ou
ao açoitamento, ou a uma das penas de morte impostas pelo Tribunal, ou a
pagamento." E todas as vezes que ne­nhum castigo for mencionado você
deverá saber que se for com relação a um dos preceitos negativos a regra
a ser aplicada é, como dizem os Sábios: "Como um homem que viola o
preceito do Rei" e não cabe a nós puní-lo. Mas com relação a todos os
preceitos positivos, quando sua execu - • ainda for aplicá-

vel, devemos açoitar com uma correia aquele que se ri' fazê-lo até que

ele morra ou cumpra, ou até que passe o momento d ão, pois aquele

que violar o preceito positivo de viver num Taberná•deve ser açoi­tado
por seu pecado depois dos Tabernáculos. Saiba

Além disso, quando eu mencionar os preceitos, positivos ou negati­vos,
que não são obrigatórios para as mulheres, direi: "E este não é
obrigatório para as mulheres". É sabido que as mulheres não estão
qualificadas para julgar nem testemunhar, nem oferecer elas próprias os
sacrifícios, nem tomarem par­te numa guerra opcional. Consequentemente,
não será necessário que eu diga: "E este não é obrigatório para as
mulheres" com respeito a todos os preceitos relativos ao Tribunal, a
testemunhos ou ao Serviço, pois isto seria apenas re­dundante e
desnecessário. •

Também quando eu mencinar os preceito, positivos ou negativos, que são
obrigatórios apenas na terra de Israel ou enquanto existir o Templo,
direi: "E este é obrigatório apenas na terra de Israel, ou enquanto o
Templo existir."

É sabido que todos os sacrifícios eram levados apenas ao Templo, e que
tal ritual está proibido fora do Campo. Da mesma forma, as leis de
puni­ção capital são impostas apenas durante a existência do Templo. A
Mekhiltá diz: "De que forma sabemos que a execução judicial só pode ter
lugar enquanto o Templo existir? Pelo que dizem as Escrituras: "Do meu
altar o tirarás, para que morra" (Êxodo 21:14). Também é sabido que a
profecia e o Reino ficarão desaparecidos de nosso meio até o momento em
que desistamos dos pecados a que nos habittiamos, quando então Deus nos
perdoará e terá misericórdia de nós, de acordo com o que Ele nos
prometeu, e nos restituirá, como disse com relação ao retorno da
profecia: "E virá depois, verterei Meu espírito sobre toda carne, e
vossos filhos e filhas profetizarão" (Joel 3:1); e com relação ao
retorno do reino e do poder Ele disse: "Nesse dia levantarei o
Tabernáculo de Davi que caiu, fecharei suas brechas, levantarei suas
ruínas e o edificarei como nos ve­lhos dias" (Amos 9:11). E também é
sabido que a guerra e a conquista da terra só será feita com um rei, sob
o comando do Grande Sanhedrin e do "Cohen Gadol", tal como Ele disse: "E
diante de Elazar, o 'Cohen"' (Números 27:21).

25. Sentar na "sucá" durante sete dias.

74 MAIMÔNIDES

Como todos estes assuntos são do conhecimento da maioria das pes­soas,
toda vez que houver um preceito positivo ou negativo relativo a
sacrifí­cios, rituais, a penas de morte impostas pelo Tribunal, pelo
Sanhedrin, ou pelo profeta e rei, ou à guerra obrigatória ou opcional,
não será necessário que eu diga que "Este preceito se aplica apenas
durante a existência do Templo", pois isto ficou claro, de acordo com o
que mencionamos. Contudo, nos casos em que possa surgir alguma dúvida ou
engano eu comentarei a respeito, se Deus quiser.

E agora começarei a mencionar todos os preceitos, com a ajuda do Todo
Poderoso.
\end{quote}

\textbf{PARTE I}

OS 248 PRECEITOS POSITIVOS

TEMA PRECEITOS

\begin{quote}
1 A CRENÇA EM DEUS (I a 19)

2 O SANTUÁRIO \emph{(21)} a 95)

3 A PURIFICAÇÃO (96 a 113)
\end{quote}

\begin{enumerate}
\def\labelenumi{\arabic{enumi}.}
\setcounter{enumi}{3}
\item
  \begin{quote}
  OS DÍZIMOS E AS DOAÇÕES (114 a 152)
  \end{quote}
\item
  \begin{quote}
  OS FESTIVAIS (1 S3 a 1\textsuperscript{-}1)
  \end{quote}
\item
  \begin{quote}
  A ÉTICA DO ESTAI)() (1\textsuperscript{-}1 a 19,-;)
  \end{quote}
\end{enumerate}

\begin{quote}
--- OS DEVERES PARA COM OS

SEMELHANTES (194 a 209)
\end{quote}

\begin{enumerate}
\def\labelenumi{\arabic{enumi}.}
\setcounter{enumi}{7}
\item
  \begin{quote}
  OS DEVERES PARA COM A FAMÍLIA (210 a \emph{223)}
  \end{quote}
\item
  \begin{quote}
  A APLICAÇÃO DAS LEIS CRIMINAIS (224 a 231)
  \end{quote}
\item
  \begin{quote}
  O DIREITO DA PROPRIEDADE \emph{(232} a 248)
  \end{quote}
\end{enumerate}

\begin{quote}
ÍNDICE

OS PRECEITOS POSITIVOS

1 Crer em Deus \emph{85}

2 A unidade de Deus \emph{85}

3 Amar a Deus \emph{86}

4 Temer a Deus \emph{86}

5 Servir a Deus \emph{87}

6 A aproximação de Deus \emph{87}

7 Jurar em nome de Deus \emph{88}

8 Trilhar os caminhos de Deus \emph{88}

9 Santificar o nome de Deus \emph{89}
\end{quote}

10 Ler o "Shemá" \emph{90}

\emph{11} O estudo da Torah \emph{90}

12 O "Tefilin" da cabeça \emph{91}

13 O "Tefilin" do braço \emph{91}

14 Os "Tsitsit" \emph{91}

15 A "Mezuzá" \emph{92}

16 A reunião do povo no Santuário durante a Festa dos Tabernáculos
\emph{92}

17 Um rei deve transcrever o Rolo da Torah \emph{92}

18 Obter um Rolo da Torah \emph{92}

19 Dar graças após as refeições 93

20 A construção do Santuário 93

21 Respeitar o Santuário \emph{94}

22 A guarda do Santuário \emph{95}

23 Os serviços dos Levitas no Santuário \emph{95}

24 As abluções dos "Cohanim" 96

25 A obrigação dos "Cohanim" de manter as lamparinas acesas \emph{96}

26 A obrigação dos "Cohanim" de abençoar os israelitas \emph{96}

27 O pão da proposição 96

28 A queima do incenso 97

29 O fogo perpétuo do altar 97

30 Remover as cinzas do altar 97

31 Retirar os impuros 97

32 Honrar os "Cohanim" \emph{98}

33 As vestes dos "Cohanim" \emph{98}

34 Os "Cohanim" devem carregar a Arca Sagrada 99

35 O óleo da unção \emph{100}

36 Os "Cohanim" devem oficiar em grupos, revezando-se no serviço
\emph{100}

37 Os "Cohanim" devem fazer-se impuros pelos parentes mortos \emph{101}

38 A obrigação do "Cohen Gadol" de casar-se apenas com uma virgem
\emph{102}

39 O holocausto diário \emph{102}

40 A oferta diária de alimento pelo "Cohen Gadol" \emph{102}

41 A oferta adicional do Shabat \emph{103}

42 A oferta adicional da lua nova \emph{103}

43 A oferta adicional de "Pessah" \emph{103}

44 A oblação da nova cevada \emph{103}

45 A oferta adicional de "Shabuot" \emph{104}

46 Levar dois pães em "Shabuot" \emph{104}

47 A oferta adicional do Ano Novo \emph{104}

80 MAIMÔNIDES

\begin{quote}
48 A oferta adicional do décimo dia de "Tishri" \emph{104}

49 O ofício de "Yom Quipur" \emph{104}

50 A oferta adicional da Festa dos Tabernáculos \emph{105}

51 A oferta adicional de "Shemini Atzeret" \emph{105}

52 As três peregrinações anuais \emph{105}

53 Comparecer diante do Eterno durante os Festivais \emph{106}

54 Alegrar-se nos Festivais \emph{106}

55 Abater a oferta de "Pessah" \emph{107}

56 Comer a oferta de "Pessah" \emph{107}

57 Abater a segunda oferta de "Pessah" \emph{108}

58 Comer a segunda oferta de "Pessah" \emph{109}

59 Tocar as cornetas no Santuário \emph{110}

60 Oferecer gado com idade mínima determinada \emph{110}

61 Oferecer apenas sacrifícios perfeitos \emph{110}

62 Levar sal com cada sacrifício \emph{111}

63 O Holocausto \emph{111}

64 O Sacrifício de Pecado \emph{111}

65 O Sacrifício de Delito \emph{111}

66 O Sacrifício de Paz \emph{111}

67 A Oblação \emph{112}

68. O sacrifício de um tribunal que cometeu um erro \emph{112}

69 O Sacrifício Estabelecido de Pecado \emph{112}

70 O Sacrifício Suspensivo de Delito \emph{113}

71 O Sacrifício Incondicional de Delito \emph{113}

72 O Sacrifício de Maior Valor ou de Menor Valor \emph{114}

73 Confessar \emph{114}

74 A oferenda levada por um "Zab" \emph{116}

75 A oferenda levada por uma "Zaba" \emph{116}

76 O sacrifício depois do parto \emph{117}

77 O sacrifício levado por um leproso \emph{117}

78 O Dízimo do Gado \emph{118}

79 Santificar o primogênito \emph{118}

80 Resgatar o primogênito \emph{119}

81 Resgatar o primogênito de um jumento \emph{119}

82 Quebrar a cerviz do primogênito de um jumento \emph{119}

83 Levar os sacrifícios devidos durante o primeiro Festival \emph{120}

84 Levar todas as ofertas apenas ao Santuário \emph{121}

85 Levar ao Santuário, desde fora da Terra de Israel, todos os sacrifí-

cios devidos. \emph{121}

86 Redimir Oferendas Defeituosas \emph{122}

87 A santidade de uma oferenda substituída \emph{122}

88 Os "Cohanim" devem comer os resíduos das oblações \emph{123}

89 Os "Cohanim" devem comer a carne dos Sacrifícios Consagra-

dos \emph{123}

90 Queimar Sacrifícios Consagrados que se tornaram impuros \emph{124}

91 Queimar as sobras dos Sacrifícios Consagrados \emph{124}

92 O Nazir deve deixar crescer seus cabelos \emph{125}

93 A obrigação do Nazir de consumar seu voto \emph{125}

94 Cumprir todos os compromissos orais \emph{126}

95 A revogação de promessas \emph{\_}

96 Tornar-se impuro com carcaças de animais \emph{127}
\end{quote}

PRECEITOS POSITIVOS 81

\begin{quote}
97 Tornar-se impuro através de carcaças de determinados animais ras-

tejantes \emph{128}

98 Tornar-se impuro através de comida e bebida \emph{128}

99 A mulher menstruada \emph{128}
\end{quote}

100 Depois do nascimento de uma criança \emph{•128}

\emph{101} O leproso \emph{128}

102 As roupas contaminadas pela lepra \emph{129}

103 A casa de um leproso \emph{129}

104 O "Zab" \emph{129}

105 O sêmen \emph{129}

106 A "Zaba" \emph{129}

107 A impureza de um cadáver.. \emph{129}

108 A lei da água de aspersão \emph{129}

109 Mergulhar no banho ritual \emph{130}

110 Purificar-se da lepra \emph{131}

\emph{111} O leproso deve raspar a cabeça \emph{131}

112 O leproso deve ser reconhecível 133

113 As cinzas da vaca vermelha \emph{133}

114 A avaliação de uma pessoa \emph{134}

115 A avaliação de animais \emph{134}

116 A avaliação de casas \emph{134}

117 A avaliação de campos \emph{134}

118 A restituição por sacrilégio \emph{134}

119 A colheita do quarto ano \emph{135}

120 "Peá" para os pobres \emph{135}

121 A respiga para os pobres \emph{135}

122 Deixar a gavela esquecida para os pobres \emph{136}

123 Deixar as sobras dos cachos de uva para os pobres \emph{136}

124 Deixar as uvas caídas para os pobres \emph{136}

125 Levar as primícias ao Santuário \emph{136}

126 A grande Oferta de Elevação 137

127 O primeiro dízimo \emph{137}

128 O segundo dízimo 137

129 O dízimo dos Levitas para os "Cohanim" ou a Oferta de Elevação
\emph{138}

\emph{130} O dízimo do homem pobre \emph{138}

131 A declaração do dízimo \emph{139}

\emph{132} A narração ao levar as primícias \emph{139}

133 A oferta de massa \emph{139}

134 Renunciar à produção de sua propriedade no ano de Shabat 139

135 O pousio da terra durante o ano de Shabat \emph{140}

136 Santificar o Ano do Jubileu (50 anos) \emph{140}

137 Fazer soar\textsuperscript{.}° "Shófar" no décimo dia de "Tishri" do
Ano do Jubileu \emph{141}

138 A devolução da terra no Ano do Jubileu \emph{141}

\emph{139} O resgate de propriedades dentro das muralhas da cidade
\emph{142}

140 Contar os anos até o Jubileu \emph{142}

141 Cancelar as dívidas no Ano de Shabat , \emph{143}

142 Cobrar as dívidas dos idólatras \emph{143}

143 A parte do "Cohen" de cada animal puro que abate \emph{143}

144 A primeira tosquia deve ser dada ao "Cohen' \emph{144}

145 As coisas consagradas \emph{144}

146 "Shehitá" \emph{144}

147 Cobrir o sangue de pássaros e animais abatidos \emph{144}

82 MAIMÔNIDES

148 Liberar a mãe quando se pegar seus filhotes \emph{145}

149 Procurar os sinais de pureza determinados no gado e nos animais
\emph{145}

150 Procurar os sinais de pureza determinados nos pássaros \emph{145}

151 Procurar os sinais de pureza determinados nos gafanhotos \emph{146}

152 Procurar os sinais de pureza determinados nos peixes \emph{146}

153 Determinar a lua nova \emph{146}

154 Descansar no Shabat \emph{148}

155 Proclamar a santidade do Shabat \emph{148}

156 Retirar o fermento \emph{149}

157 Narrar o êxodo do Egito \emph{149}

158 Comer pão ázimo na véspera do décimo quinto dia de Nissan \emph{149}

159 Descansar no primeiro dia de "Pessah" \emph{150}

160 Descansar no sétimo dia de "Pessah" \emph{150}

161 Contar o "omer" \emph{150}

162 Descansar no dia de "Shabuot" \emph{151}

163 Descansar no dia de "Rosh Hashaná" \emph{151}

164 Jejuar no dia de "Yom Quipur" \emph{151}

165 Descansar no dia de "Yom Quipur" \emph{152}

166 Descansar no primeiro dia de "Sucot" \emph{152}

167 Descansar no dia de "Shemini Atzeret" \emph{152}

168 Morar numa cabana durante os dias de "Sucot" \emph{153}

169 Pegar um "Lulav" no "Sucot" \emph{153}

170 Ouvir o "Shofar" no dia de "Rosh Hashaná" \emph{153}

171 Dar meio "Shekel" anualmente \emph{153}

172 Acatar o que dizem os profetas \emph{153}

173 Nomear um rei \emph{154}

174 Obedecer o Grande Tribunal \emph{154}

175 Aceitar a decisão da maioria \emph{155}

176 Nomear juízes e oficiais do tribunal \emph{155}

177 Tratar as partes com igualdade perante a lei \emph{156}

178 Testemunhar no tribunal \emph{157}

179 Investigar o depoimento das testemunhas \emph{157}

180 Condenar as testemunhas que prestarem falso testemunho \emph{157}

181 "Eglá Arufá" \emph{158}

182 Separar seis ,cidades de refúgio \emph{158}

183 Designar cidades para os Levitas \emph{158}

184 Eliminar o perigo de nossas moradias \emph{158}

185 Destruir todo tipo de idolatria na Terra de Israel \emph{159}

186 A lei da cidade apóstata \emph{159}

187 A guerra contra as Sete Nações Hereges \emph{159}

188 A extinção de Amalec \emph{160}

189 Recordar os atos nefastos de Amalec , \emph{161}

190 A lei da guerra não obrigatória \emph{161}

191 Nomear um "Cohen" para a guerra \emph{161}

192 Preparar um lugar separado do acampamento \emph{162}

193 Incluir uma estaca entre os utensílios de guerra \emph{162}

194 Um ladrão deve devolver o objeto roubado \emph{163}

195 "Tsedaká" \emph{163}

196 Bonificar o servo que recobrar sua liberdade \emph{163}

197 Emprestar dinheiro aos pobres \emph{164}

198 Cobrar juros do idólatra \emph{164}

199 Devolver o penhor ao proprietário necessitado \emph{164}

PRECEITOS POSITIVOS 83

200 Pagar os soldos no dia \emph{165}

201 Um empregado deve poder comer daquilo com que ele trabalha
\emph{165}

202 Descarregar um animal cansado \emph{166}

203 Ajudar o próximo a levantar sua carga \emph{166}

204 Devolver a seu dono o que ele tiver perdido \emph{166}

205 Repreender o pecador \emph{167}

206 Amar o próximo \emph{167}

207 Amar o prosélito \emph{167}

208 A lei dos pesos e medidas \emph{168}

209 Honrar os eruditos e os idosos \emph{168}

210 Honrar os pais .\emph{169}

211 Respeitar os pais \emph{170}

212 "Frutificar e multiplicar" \emph{170}

213 A lei da consagração pelo casamento \emph{170}

214 O marido deve dedicar-se a sua esposa durante um ano \emph{171}

215 A lei da circuncisão \emph{171}

216 A lei do Casamènto Levirato \emph{171}

217 "Halitzá" \emph{171}

218 Um violador deve casar-se com a moça que violentou \emph{172}

219 A lei sobre aquele que difama sua esposa \emph{172}

220 A lei sobre o sedutor \emph{172}

221 A lei sobre a mulher cativa \emph{172}

222 A lei do divórcio \emph{173}

223 A lei sobre uma mulher suspeita de adultério \emph{173}

224 Açoitar os transgressores de determinados preceitos \emph{173}

225 Alei do homicídio involuntário \emph{173}

226 Executar com a espada os transgressores de determinados preceitos
\emph{174}

227 Estrangular os transgressores de determinados preceitos \emph{174}

228 Queimar os transgressores de determinados preceitos \emph{174}

229 Apedrejar os transgressores de determinados preceitos \emph{174}

230 Pendurar os corpos de certos transgressores depois de executados
\emph{175}

231 A lei do enterro \emph{175}

232 A lei do servo hebreu \emph{175}

\emph{233} O casamento de uma serva hebréia com seu amo ou com o filho
dele \emph{175}

234 O resgate de uma serva hebréia \emph{176}

235 A lei sobre o escravo cananeu \emph{176}

236 A penalidadt por causar ferimentos \emph{176}

237 A lei sobre ferimentos causados por um boi \emph{177}

238 A lei sobre ferimentos causados por um poço \emph{177}

239 A lei sobre o roubo \emph{177}

240 A lei sobre os prejuízos causados por um animal \emph{177}

241 A lei sobre os prejuízos causados pelo fogo \emph{178}

242 A lei sobre o depositário não remunerado \emph{178}

243 A lei sobre o depositário remunerado \emph{178}

244 A lei sobre quem pede emprestado \emph{178}

245\textsuperscript{.} A lei de compra e venda \emph{178}

246 A lei sobre os litigantes \emph{179}

247 Salvar a vida do perseguido { ■} \emph{179}

248 A lei sobre as heranças \emph{180}

\begin{quote}
\textbf{1} CRER EM DEUS
\end{quote}

Por este preceito somos ordenados a crer em Deus, ou seja, a acredi­tar
que há um Agente Supremo que é o Criador de tudo o que existe. Ele está
expresso em Suas palavras, enaltecido seja Ele, "Eu sou o Eterno, teu
Deus, que te tirei da terra do Egito etc" (Êxodo 20:2).

No final do Tratado Macot está dito: "Seiscentos e treze preceitos

foram comunicados a Moisés no. Sinai, como diz o verso 'A Lei que orde-

nou \textsubscript{n t s}

\begin{quote}
Moiség'•\textsuperscript{.}(Deuteronômio 33:4)"; ou seja, ele nos
ordenou obe o
\end{quote}

preceitos quantos há na soma das letras-número TORAH. A isso que

as letras-números da palavra TORAH somam apenas seiscentos o res-

posta foi: "Os dois preceitos 'Eu sou o Eterno, teu Deus' e ão erás
utros\\
deuses diante de Mim' (Êxodo, 20:3) foram ouvidos do próprio Todo
\textsuperscript{-}roso."

Portanto, foi deixado claro que o versículo "Eu sou o Eterno, teu Deus"
é um dos 613 preceitos, e é o que nos ordena a crer em Deus, como
explicamos.

\textbf{2} A UNIDADE DE DEUS

Por este preceito somos ordenados a crer na Unidade de Deus, ou seja, a
acreditar que o Criador de todas as coisas existentes e Primeiro Agente
delas é Uno. Este preceito está expresso em Suas palavras, enaltecido
seja Ele, "Escuta, Israel! O Eterno é nosso Deus, o Eterno é Uno!"
(Deuteronômio 6:4)

Em quase todo Midrashot você vai encontrar que essas palavras
sig­nificam que devemos declarar a Unidade do Nome de Deus, ou a Unidade
de Deus, ou algo nesse sentido. A intenção dos Sábios era ensinar que
Deus nos tirou do Egito e nos cumulou de bondade apenas com a condição
de que acre­ditemos em Sua Unidade, e isso é nosso dever.

O preceito de crer na Unidade de Deus está mencionado em muitos lugares,
e os Sábios também o chamam de preceito de crer no Reino dos Céus,

186 MAIMÔNIDES

pois eles falam da obrigação "de tomar a nosso cargo a união com o Reino
dos Céus", ou seja, de declarar a Unidade de Deus e de crer n'Ele.

AMAR A DEUS

Por este preceito somos ordenados a amar ao Eterno, enaltecido se­ja
Ele, ou seja, a deter-nos e a meditar sobre Seus preceitos, Suas ordens,
e Seus trabalhos, de maneira a obter uma concepção d'Ele, e ao
concebê-Lo, alcançar o júbilo absoluto, e isto é o amor que nos foi
ordenado. Como diz o Sifrei: "Uma vez que dizemos 'E amarás ao Eterno,
teu Deus' (Deuteronômio 6:5), o estu­dioso perguntará: 'Como se deve
manifestar seu amor pelo Eterno?' As Escritu­ras dizem: 'E estarão estas
palavras que Eu te ordeno hoje, no teu coração' (Deu­teronômio 6:6),
porque é através disto que você aprenderá a conhecer Aquele cuja palavra
ordenou ao Universo sua existência".

Desta forma ficou claro que através deste ato de meditação você vai
alcançar a concepção de Deus e alcan o estado de júbilo no qual o amor a
Deus será uma conseqüência nece

\begin{quote}
Os Sábios dizem que este também inclui a obrigação de con-
\end{quote}

vocar todos os descendentes de Ad ra serví-Lo, louvado seja Ele, e ter

fé n'Ele. Porque da mesma forma q e exalta e glorifica alguém a quem

você ama, e convoca os outros hom ara amá-lo, se você ama o Eterno até

a concepção de Sua verdadeira Natu eza, que você já alcançou através do
co­nhecimento, sem dúvida convocará os tolos e os ignorantes a procurar
o co- . nhecimento da Verdade que você já encontrou.

Como diz o Sifrei: " 'E amarás ao Eterno, teu Deus': isso significa que
você deverá fazer com que Ele seja amado pelos homens, como o fez o seu
pai Abraham, como foi dito: 'E as almas que haviam adquirido em Haran' "

(Gênesis 12:5). Ou seja, s esma forma que Abraham, sendo um amante do

Eterno --- como a Tor unha, quando designado pelo Eterno como sen-

do: 'Meu amado Abra pela força de sua concepção de Deus e pelo

seu grande amor por El cone o cou a humanidade a crer, assim você deve
amá­Lo de forma tal a atrai anidade para Ele.

4 TEMER A DEUS

Por este preceito somos ordenados a crer no temor a Deus, enalteci­do
seja Ele, de maneira a não ficar acomodados e auto-confiantes, e sim a
espe­rar sempre Seu castigo.. Este preceito está expresso em Suas
palavras "Ao Eter­no, teu Deus, temerás" (Deuteronômio 6:13, 10:20).

A Guemará do Tratado Sanhedrin comenta da seguinte forma o ver­sículo
"Aquele que insultar (nokeb) o nomé real do Eterno certamente será
mor­to" (Levítico 24:16): "Talvez a palavra `nokeb' devesse significar
'declarar', já que èncontramos em outro lugar 'Estes homens que foram
declarados (nikebu) por nomes' (Números 1:17), provindo a advertência do
versículo 'Ao Eterno, teu Deus, temerás' ". Ou seja, o versículo "Aquele
que insultar o nome real do\textsuperscript{.} Eterno, etc" deve ser
entendido como significando aquele que simplesmente mencionar o Nome do
Eterno, sem louvá-Lo; e se você perguntar: "Que peca-

\begin{enumerate}
\def\labelenumi{\arabic{enumi}.}
\setcounter{enumi}{26}
\item
  \begin{quote}
  Os filhos de Adam no contexto da humanidade no seu todo.
  \end{quote}
\item
  \begin{quote}
  Isa. 41:8.
  \end{quote}
\end{enumerate}

\begin{quote}
PRECEITOS POSITIVOS 87

do há nisso?" responderemos que quem o fizer estará abandonando o temor
ao Eterno, porque faz parte do temor ao Eterno não pronunciar Seu Nome
em vão.

Os Sábios respondem a esta pergunta, e contestam a perspectiva
en­volvida nela, como segue: "Primeiro, para que constitua um insulto, o
Nome deve ser utilizado, e neste caso essa condição está ausente"; ou
seja, ele deve ser culpado de insultar o Nome em nome d'Ele, tal como
eles dizem: 'Deixe Yossi castigar Yossi' ".

"Além do mais, a advertência que você cita está na forma de um pre­ceito
positivo, e é um princípio aceito que tal tipo de advertência não é
váli­da". Quer dizer, sua teoria de que a proibição do mero
pronunciamento do Nome de Deus pode provir do versículo "Ao Eterno, teu
Deus, temerás" é inad­missível porque esse versículo é um preceito
positivo e uma proibição não po­de ser baseada num preceito positivo.

Assim, foi-lhe deixado claro que as palavras "Ao Eterno, teu Deus,
temerás" estipulam um preceito positivo.

\textbf{5} SERVIR A DEUS

Por este preceito somos ordenados a servir a Deus, enaltecido seja Ele.
Este preceito está repetido várias vezes nas Escrituras, como foi dito
em: "E servireis ao Eterno, vosso Deus" (Êxodo 23:25); "e a E e
servireis" (Deute-

ronômio 13:5); "e a Ele servirás" (Ibid., 6:13); "e ser ' (Ibid.,
11:13).

Embora este preceito seja da categoria dos tos gerais que es-

tão excluídos dos 613 preceitos pelo Quarto Fundarne
o\textsuperscript{29}, inda assim ele im­põe uma obrigação específica,
que é a Oração. O Sifrei diz. rví-Lo (Ibid. 11:13) significa Oração". Os
Sábios também dizem: " `Serví-Lo' significa estudar a Lei".

Na Mishná de Rabi Eliezer, filho de Rabi Yossi Ha-Galili, está dito: "De
que maneira ficamos sabendo que a Oração é obrigatória? Através do
ver­sículo 'Ao Eterno, teu Deus, temerás, e a Ele s virás' (Ibid.
6:13)". Os Sábios também dizem: " `Serví-Lo' através da Sua T ra e
`serví-Lo' em Seu Santuá­rio", o que significa que devemos aspirar ezar
no Templo ou voltados em sua direção, como disse claramente Salom
o\textsuperscript{3}°

6 A APROXIMAÇÃO DE DEUS

Por este preceito somos ordenados a juntarmo-nos e a associarmo-nos com
os homens sábios, a estar sempre em sua companhia, a unirmo-nos a eles e
a seguir seus caminhos através de toda forma possível de
companheiris­mo: comendo, bebendo .e a negócios, com a finalidade de
conseguir ser como eles, quanto as suas ações, e de acreditar nos
conceitos verdadeiros através de suas palavras. Este preceito está
expresso em Suas palavras, enaltecido seja Ele, "E d'Ele te aproximarás"
(Deuteronômio 10:20) que estão repetidas no versí­culo "E
aproximando-vos d'Ele" (Ibid. 11:22). O Sifrei diz: " 'E aproximando-vos
d'Ele' significa que devemos aproximar-nos dos homens sábios e de seus
discípulos".
\end{quote}

\begin{enumerate}
\def\labelenumi{\arabic{enumi}.}
\setcounter{enumi}{28}
\item
  \begin{quote}
  Ver o Quarto Fundamento.
  \end{quote}
\item
  \begin{quote}
  Reis 8:30.
  \end{quote}
\end{enumerate}

\begin{quote}
88 MAIMÔNIDES

Os Sábios também usam as palavras "E d'Ele te aproximarás" como prova de
que é nosso dever casarmo-nos com a filha de um homem sábio, dar nossa
própria filha em casamento a um discípulo de um homem sábio, benefi­ciar
os homens sábios e manter negócios com eles. Como está dito: "Existe a
possibilidade para um ser humano de aproximar-se da Presença Divina,
visto que está escrito 'Porque o Eterno, teu Deus, é um fogo
consumidor'? (Ibid. 4:24). Dessa forma devemos concluir, de acordo com
este versículo, que o casamen­to com a filha de um sábio deve ser
considerado como um meio de aproximar-se do Eterno".
\end{quote}

JURAR EM NOME DE DEUS

\begin{quote}
Por este preceito somos ordenados a jurar exclusivamente em Seu Nome,
enaltecido seja Ele, toda vez que nos for solicitado confirmar ou negar
alguma coisa sob juramento, porque fazendo isso nós O estaremos
exaltando, honrando e magnificando. Este preceito está expresso em Suas
palavras, enal­tecido seja Ele, "E pelo Seu nome jurarás" (Deuteronômio
6:13; 10:20), que os Sábios explicam assim: "a Torah diz 'Em Seu nome
jurarás' e novamente `Não jurarás em No d terno, teu Deus, em vão'
(Êxodo 20:7)". Da mesma forma que estamos roibi os de fazer um juramento
sem necessidade, e isto é um preceito nega vo\textsuperscript{31}, sim
também somos ordenados a fazer um juramen­to quando necessár {e} i o é
um preceito positivo.

É por essa razão que não é permitido jurar por nenhuma entidade criada,
tal como os anjos ou as estrelas, exceto quando o juramento é elíptico
como, por exemplo, quando se jura pela realidade do sol, significando
"pela realidade do Deus do sol". E assim que nossa nação jura pelo nome
de Moisés nosso Mestre --- honrado seja seu nome --- como se aquele que
jura estivesse dizendo "pelo Deus de Moisés" ou "por Aquele que mandou
Moisés". Mas quan­do aquele que jurar não tiver a intenção de dizer isso
dessa forma e jurar po um ente criado, acreditando que essa entidade
contém em si própria uma v r­dade tal que se possa jurar por ela, ele
estará cometendo um pecado ao col car alguma outra entidade em pé de
igualdade com o Nome dos Céus; a Tradi o\textsuperscript{32} diz, a esse
respeito: "Aquele que colocar o Nome dos Céus em pé de igual àde com
algum outro ente deverá ser erradicado da face da terra".

Este era o significado pretendido no versículo "Pelo Seu Nome ju a­ rás
apenas": em nome d'Ele você deve atribuir ma verdade que seja digna de
um juramento.

Está dito no início do Tratado Tem De que modo ficamos sa-

bendo que podemos nos comprometer por um ento a cumprir os precei-

tôs? Pelo versículo 'E pelo Seu Nome jurarás' ".

8 TRILHAR OS CAMINHOS DE DEUS

Por este preceito somos ordenados a assemelhar-nos a Deus, enalte­cido
seja Ele, o tanto quanto nos for possível. Este preceito está expresso
em Suas palavras "E andares por Seus caminhos" (Deuteronômio 28:9). Este
preceito
\end{quote}

\begin{enumerate}
\def\labelenumi{\arabic{enumi}.}
\setcounter{enumi}{30}
\item
  \begin{quote}
  Ver preceito negativo 62.
  \end{quote}
\item
  \begin{quote}
  Sucá 45:b.
  \end{quote}
\item
  \begin{quote}
  Temurá 3:B.
  \end{quote}
\end{enumerate}

\begin{quote}
PRECEITOS POSITIVOS 89

foi repetido várias vezes, e disse: "que andes em todos os Seus
caminhos" (Ibid. 11:22).

Com relação a este último versículo os Sábios comentam o seguinte:
"Assim como o Sagrado, enaltecido seja Ele, é chamado Misericordioso,
você também deve ser misericordioso; assim como Ele é chamado
Benevolente, vo­cê também deve ser benévolo; assim como Ele é chamado
Justo, você também deve ser justo; assim como Ele é chamado 'Hassid',
você também deve ser `hassid' ".

Este preceito já apareceu sob outra forma em Suas palavras "Após o
Eterno, vosso Deus, andareis" (Deuteronômio 13:5) que os Sábios explicam
como significando que devemos imitar as boas ações e os elevados
atributos pelos quais o Eterno, enaltecido seja Ele, é figurativamente
descrito, uma vez que Ele é de fato sublime de forma imensuravelmente
superior a toda essa descrição.
\end{quote}

SANTIFICAR O NOME DE DEUS

\begin{quote}
Por este preceito somos ordenados a santificar o Nome de Deus. Es­te
preceito está expresso em Suas palavras, "E serei santificado entre os
filhos de Israel" (Levítico 22:32). O conteúdo deste preceito é que
temos o dever de pregar esta verdadeira religião pelo mundo, sem temer
danos de qualquer es­pécie. Mesmo se um tirano tentar forçar-nos a
negá-Lo, não devemos obede­cer, e ao invés disso devemos preferir a
morte; e não devemos sequer enganar o tirano fazendo-o crer que O
negamos, embora em nossos corações continue­mos a crer n'Ele, enaltecido
seja Ele.

Este é o preceito relativo à Santificação do Nome que foí imposto a cada
um dos filhos de Israel: que devemos estar prontos a morrer nas mãos de
u' ti no por nosso amor a Ele, enaltecido seja Ele, e por nossa fé em
Sua Un de, \textless{} sim como fizeram Hananiah, Mishael e Azariah no
tempo do per-
\end{quote}

\begin{itemize}
\item
  \begin{quote}
  Na . ucodonosor, quando ele forçou o povo a prostrar-se diante do í
  olo\textsuperscript{34} e todos assim o fizeram, inclusive os
  israelitas, e não havia mais nin­g ém lá ara santificar o Nome dos
  Céus, pois estavam todos aterrorizados. Es-t foi tíma grande desgraça
  para Israel, pois este preceito foi desobedecido por todos eles.
  \end{quote}
\end{itemize}

Este preceito só se aplica em ocasiões como aquela, quando todo

\begin{itemize}
\item
  \begin{quote}
  mundo estava aterrorizado e era um dever declarar publicamente,
  naquela ocasião, a Sua Unidade. O Eterno já havia prometido, através
  de Isaías, que Is­rael não seria desgraçada por completo naquele
  momento difícil, pois entre eles surgiriam jovens sem medo da morte
  que derramariam seu sangue e proclama­riam a Fé, santificando o Nome
  publicamente, como ele nos ordenou através
  \end{quote}
\end{itemize}

\begin{quote}
de Moisés, nosso Mestre. Essa promessa está nas palavras "Agora Jac.
\textsuperscript{-}o fi-

cará envergonhado, nem ficará seu rosto pálido; quando ele vir que os,
\end{quote}

\begin{itemize}
\item
  \begin{quote}
  trabalho de Minhas mãos, no meio dele, santificam o Meu no A Sifrá
  diz: "Eu vos tirei da terra do Egito com a condiç que\\
  vocês santifiquem Meu nome publicamente".
  \end{quote}
\end{itemize}

\begin{quote}
Na Guemará do Tratado Sanhedrin está dito: "Um `Noachid' é obri­gado a
santificar Seu Nome, ou não? Ouçam isto: 'Aos `Noachidos' foi orde-

nado que obedecessem sete precei e a eles foi ordenado que santifi-

cassem Seu Nome, então seriam o

Assim, ficou claro que st é u e dos preceitos obrigatórios para Is­rael,
tendo os Sábios deduzido esté pr to das palavras "Eu serei santificado
entre os filhos de Israel". As leis detalhadas sobre este preceito estão
expostas no sétimo capítulo de Sanhedrin.

10 LER O "SHEMÁ"

Por este preceito somos ordenados a ler o Shemá diariamente, à noi­te e
pela manhã. Este preceito está expresso em Suas palavras, enaltecido
seja Ele, "E delas falarás sentado em tua casa, andando pelo caminho, e
ao deitar-te e ao levantar-te" (Deuteronômio 6:7).

As leis referentes a este preceito estão explicadas em Berakhot, on­de
está demonstrado que a leitura do "Shemá" foi ordenada pela Torah.

A Tosseftá diz: "Da mesma maneira que a Torah ordenou um horá­rio
determinado para a leitura do `Shemá', os Sábios determinaram um horário
para a Oração"; ou seja, os horários da Oração não são determinados pela
To­rah, mas o dever da oração em si é imposto pela Torah, como já
explicamos, e os Sábios determinaram os horários da Oração.

Os Sábios fixaram os horários da Oração de maneira a corresponder aos
horários em que os sacrifícios eram trazidos, nos tempos dos grandes
tem­plos de Jerusalém.

Este preceito não é obrigatório para as mulheres.

11 O ESTUDO DA TORAH

Por este preceito somos ordenados a ensinar e a estudar a sabedoria da
Torah, que é chamada de Talmud Torah. Este preceito está expresso em
Suas palavras "E as inculcarás a teus filhos" (Deuteron 6:7).

O Sifrei diz: " 'A teus filhos' significa .studantes: verificamos
\end{quote}

que em todo lugar os discípulos de um homemrão ch. mados de seus filhos,

\begin{quote}
como está dito em 'E os filhos dos profetas saí,: O Sifrei também diz,

no mesmo trecho: " 'E as inculcarás a teus filh significa que elas de-

vem fluir facilmente de sua boca, para que qua d pessoa faça 'uma per-
\end{quote}

gunta sobre elas, você não vacile em sua resposta, e esponda com
presteza".\\
Este preceito está repetido muitas vezes: "E ensina-las-eis"
(Deuteronô-\\
mio 11:19); "E para que aprendam" (Deuteronômio 31:12). A importância
deste pre-\\
ceito e a obrigação de cumprí-lo estão enfatizadas em várias passagens
do Talmud.

36. O termo "Noachidos" (descendentes de Noach) significa os
não-israelitas ou pagãos de todos\\
os tempos que, de acordo com a lei judaica, são obrigados a obedecer os
sete preceitos seguintes:

\begin{enumerate}
\def\labelenumi{(\arabic{enumi})}
\item
  \begin{quote}
  estabelecer tribunais de justiça;
  \end{quote}
\item
  \begin{quote}
  não praticar idolatria;
  \end{quote}
\item
  \begin{quote}
  não blasfemar;
  \end{quote}
\item
  \begin{quote}
  não comèter incesto;
  \end{quote}
\item
  \begin{quote}
  não matar;
  \end{quote}
\item
  \begin{quote}
  não roubar, e
  \end{quote}
\item
  \begin{quote}
  não comer carne retirada de animais enquanto vivos. Os. Noachidos que
  obser­vam estes 7 preceitos herdarão uma parte do Mundo Vindouro.
  \end{quote}
\end{enumerate}

\begin{quote}
37. Reis II, 2:3.

PRECEITOS POSITIVOS 91

As mulheres não são obrigadas a obedecer a este preceito, de acor­do com
Suas palavras "E ensina-las-eis a vossos filhos" (Ibid., 11:19) sobre as
quais os Sábios comentam: " Filhos', mas não filhas", como foi explicado
na Guemará de Kidushin.

12 O \textbf{"TEFILIN"} DA CABEÇA

Por este preceito nos é ordenado o uso do "Tefilin" da Cabeça. Este
preceito está expresso em Suas palavras "E serão por marca entre os teus
olhos" (Deuteronômio 6:8).

Este preceito está repetido quatro vezes (Êxodo 13:9; ibid., 16;
Deu­teronômio 6:8; ibid., 11:18).

\textbf{13 O "TEFILIN" DO BRAÇO}

Por este preceito nos é ordenado o uso do "Tefilin" do Braço. Este
preceito está expresso em Suas palavras, enaltecido seja Ele, "E os
atarás, como sinal na tua mão" (Deuteronômio 6:8). Também este preceito
está repetido qua­tro vezes (Êxodo 13:1; ibid., 16; Deuteronômio 6:8;
ibid., 11:18).

Na Guemará Menahot encontramos a prova de que o uso do "Tefi­lin" da
Cabeça e do "Tefilin" do Braço são dois preceitos, quando os Sábios se
mostram surpresos com a idéia de que o "Tefilin" dá Cabeça e o do Braço
não possam ser usados um sem o outro, e sim apenas os dois juntos. "Se",
di­zem eles, "alguém não puder cumprir dois preceitos, ele não deve
cumprir um?". Quer dizer, se alguém não puder cumprir os dois preceitos
ele não deve cum­prir nenhum deles? Não, ele deve cumprir aquele que ele
puder. Dessa maneira ele deve usar o "Tefilin" que possuir.

Assim ficou claro que os Sábios consideram o "Tefilin" do Braço e o
"Tefilin" da Cabeça como sendo dois preceitos.

Estes dois preceitos não são obrigatórios para as mulheres, pois quan­do
explicou sobre sua obrigatoriedade, Ele disse, enaltecido seja Ele,
"Para que esteja a Torah do Eterno em tua boca" (Êxodo 13:9) e as
mulheres não têm a obrigação de estudar a Torah. Esta é uma explicação
dada na Mekhiltá.

Todas as leis sobre estes dois preceitos estão explicadas no quarto
capítulo de Menahot.

1 4 \textbf{OS "TSITSIT"}

Por este preceito nos é ordenado o feitio dos "Tsitsit". Este precei­to
está expresso em Suas palavras, enaltecido seja Ele, "Façam para eles
`Tsitsie sobre as bordas de suas vestes, pelas suas gerações; e porão
sob Tsitsit' da borda, um cordão azul celeste" (Números 15:38).

Este não é contado como dois preceitos, embora s regra e tre nós que o
azu1\textsuperscript{38} não prejudica a validade do
branco\textsuperscript{38}, nem o b anco\textsuperscript{38} ' valida o
azu1\textsuperscript{38}. A razão disso aparece no Sifrei: "Poder-se-ia
pensar que s são dois preceitos --- o preceito do azul e o preceito do
branco; por esse motivo, a To­rah estabelece 'E será para vós por
"Tsitsit" ' (Ibid., 39), mostrando assim que se trata de um preceito e
não de dois".

Este preceito não é obrigatório para as mulheres, como está explicado no
Início de Kidushin. Todas as leis deste preceito estão explicadas em
Menahot.

92 MAIMÔNIDES

..(

Por este preceito nos é ordenado a confecça "Mezuzá". Este

preceito está expresso em Suas pai s, enaltecido seja , "E as escreverás

nos umbrais de tua casa e em teu ort es" (Deuteronômio 6:9). Este
preceito está repetido novamente na Tor h\textsuperscript{4}°.
\end{quote}

Todas as suas leis estãó : . \textbf{IP} nadas no terceiro capítulo de
Menahot.

\begin{quote}
16 A REUNIÃO DO POVO NO SANTUÁRIO DURANTE FESTA DOS TABERNÁCULO

Por este preceito somos ordenados a que odas as pessoas se reu­nam no
segundo dia dos Tabernáculos, depois do finalde cada sétimo ano, e a que
vários versículos de Deuteronômio sejam lidos para elas. Este preceito
está expresso em Suas palavras, enaltecido seja Ele, "Congrega o povo,
os ho­mens e as mulheres, e as crianças, etc." (Deuteronômio 31:12), as
quais são o preceito da Assembléia.

Em Kidushin está dito: "O cumprimento de todos os preceitos posi­tivos
que estão ligados a uma ocasião não é obrigatório para as mulheres". O
Talmud levanta a questão: "Não é a Assembléia um preceito positivo
ligado a uma ocasião e mesmo assim obrigatório para as mulheres?" A
conclusão a que se chegou ao final da discussão foi que "Não se deve
argumentar a partir de uma regra geral".

As normas deste preceito, a saber, como deve ser feita a leitura, quem
deve ler, e que trechos devem ser lidos estão explicadas no sétimo
capítulo do Tratado Sotá.

17 UM REI DEVE TRANSCREVER O ROLO DA TORAH,

Por este preceito somos ordenados a qu t o rei de nossa nação que ocupar
o trono real transcreva um Rolo da Tora \textsuperscript{42} ara si
mesmo, do qual ele não deve se separar. Este preceito está expresso Suas
palavras, enalteci­do seja Ele, "E quando se sentar sobre o trono de seu
reino, escreverá para si o traslado desta Lei" (Deuteronômio 17:18).

Todas as normas deste preceito estão explicadas no segundo capítu­lo de
Sanhedrin.

\textbf{18} OBTE UM ROLO DA TORAH
\end{quote}

Por este p ecei i. somos ordenados a que todo varão entre nós te-

\begin{quote}
nha um Rolo da Tor ara si próprio. Se ele o transcrever de próprio pu-

nho, isto será muito ciado e, de preferência, é assim que deve ser feito
\end{quote}

\begin{enumerate}
\def\labelenumi{\arabic{enumi}.}
\setcounter{enumi}{38}
\item
  \begin{quote}
  Ou seja, nos foi ordenado que fixemos a "Mezuzá". Vide Halachot
  Tefilin Cap. 5, lei 7.
  \end{quote}
\item
  \begin{quote}
  Deuteronômio 11:20.
  \end{quote}
\item
  \begin{quote}
  Festa de Sucot.
  \end{quote}
\item
  \begin{quote}
  O Sefer Torah (Pentateuco) que lemos na sinagoga, aos sábados.
  \end{quote}
\item
  \begin{quote}
  Sefer Torah
  \end{quote}
\end{enumerate}

\begin{quote}
PRECEITOS POSITIV 4 S 93

pois os Sábios di ele o transcrever de próprio punho, será conside-

rado pelas Escri u o se ele o tivesse recebido do Monte Sinai". Se ele

próprio não puder o, ele é obrigado a comprar um, ou a contratar alguém

que o transcreva por ele. Este preceito está expresso em Suas palavras,
enalte­cido seja Ele, "E agora escrevei para vós este cântico"
(Deuteronômio 31:19). Contudo, como não é permitido transc er alguns
trechos dela, as palavras "este cântico" devem necessariamen sig m car a
Torah completa, que inclui "este cântico".

A Guemará de Sanhedri d 45, ba diz: Ainda que seus pais lhe

tenham deixado um Rolo da Lei, ele d ve transcrever o seu próprio, como
está dito, 'E agora, escrevei para vós este cântico'. Abaye objetou:
Será que trans­crever um Rolo da Torah em seu próprio nome, porque ele
não deve desejar se apoiar nos seus pais, é apenas uma obrigação do Rei
e não do plebeu? A res­posta a isso foi: A regra é necessária apenas
para obrigar o Rei a transcrever dois Rolos, pois nos foi ensinado que
'Ele escreverá para si o traslado desta Lei' (Deu­teronômio 17:18)
significa que ele deve transcrever para si mesmo duas cópias". Ou seja,
a diferença entre o Rei e um plebeu é que cada homem deve transcre­ver
um Rolo da Lei, mas o Rei deve transcrever dois, como está explicado no
segundo capítulo de Sanhedrin.

As regras para a transcrição de um Rolo da Torah e as condições
re­lativas a isto estão explicadas no terceiro capítulo de Menahot, no
início de Ba­ba Batra e em Shabat.

19 DAR GRAÇAS APÓS AS REFEIÇÕES

Por este preceito somos ordenados a dar graças ao Eterno, enalteci­do
seja Ele, após cada refeição. Ele está expresso em Suas palavras,
enaltecido seja Ele, "E comerás e te fartarás, e louvarás ao Eterno, teu
Deus" (Deuteronô­mio 8:10).

A Tosseftá diz: "Ficamos sabendo que dar graças após uma refeição é um
preceito imposto pela Torah através do versículo 'E comerás e te
fartarás, e louvarás ao Eterno, teu Deus' ".

As normas deste preceito estão explicadas em vários trechos do Tra­tado
Berakhot.

20 A CONSTRUÇÃO DO SANTUÁRIO

Por este preceito somos ordenados a construir uma Casa para Seu serviço.
Lá deverão ser oferecidos sacrifícios e deverá arder o fogo perpétuo,
para lá serão feitas as peregrinações e lá terão lugar todos os anos as
festas e assembléias. Este preceito está expresso em Suas palavras,
enaltecido seja Ele, "E Me farão um santuário" (Êxodo 25:8).

O Sifrei diz: "Os israelitas foram ordenados a cumprir três preceitos ao
entrar na Terra: designar um rei para si próprios, construir o
Santuário, e destruir os descenden e alec' . Fica, dessa forma, claro
que a constru­ção do Santuário co tui u preceito em si.

Já explic e este preceito global inclui preceitos individuais
\end{quote}

\begin{enumerate}
\def\labelenumi{\arabic{enumi}.}
\setcounter{enumi}{43}
\item
  \begin{quote}
  Menahot 30:A
  \end{quote}
\item
  \begin{quote}
  Sanhedrin 21:B
  \end{quote}
\item
  \begin{quote}
  Décimo Segundo Fundamento.
  \end{quote}
\end{enumerate}

\begin{quote}
94 MAIMÔNIDES

e que o Castiçal, a Mesa, o Altar, e as outras coisas são todos partes
do Santuário e que tudo isso junto é chamado "o Santuário", embora haja
um preceito espe­cífico para cada uma das partes.

É verdade que Ele disse, com relação ao Altar: "um altar de terra vo­cê
deverá fazer para Mim" (Êxodo 20:21), de maneira que se poderia pensar
que este é um preceito independente, separado do de construir o
Santuário, mas o significado verdadeiro, neste caso é o que vou explicar
a vocês. O senti­do literal do versículo se refere aos tempos em que os
Altos Lugares nos eram permitidos e nós tínhamos autorização para fazer
um Altar de terra em qual­quer lugar, e oferecer sacrifício ne e os
Sábios já declararam que o objetivo do versículo era ordenar-nos const
ção de um Altar ligado à terra, que não fosse móvel, como era no des to
\textsuperscript{47}. sto foi dito por eles na Mekhiltá de Rabi Ishmael,
onde o versículo é in do assim: "Quando você entrar na Terra de Israel
você deverá erguer um altar para Mim ligado à terra". Sendo assim, o
preceito é um dos que são obrigatórios por todas as gerações, e
conclui-se que ele faz parte dos deveres do Templo, e quer dizer que o
altar a ser construí­do deve ser de pedra. Ao explicar as palavras "E se
você Me fizer um altar de pedra" a Mekhiltá diz: "Rabi Ishmael diz:
'Toda palavra 'se' na Torá implica permissão, exceto em três ocasiões,
uma das quais é 'E se você Me fizer um altar de pedra' " . Então os
Sábios disseram: "O versículo 'E se você Me fizer um altar de pedra'
estabelece uma obrigação. Você afirma que isto é uma obri­gação; e se
fôsse apenas uma permissão? A Torah nos diz 'Com pedras inteiras
edificarás o altar do Eterno, teu Deus' (Deuteronômio 27:6)".

As normas relacionadas à construção do Santuário, seu modelo e suas
divisões, a construção do Altar, e as leis relativas, estão explicadas
num Trata­do consagrado especialmente ao assunto, que é o Tratado Midot.
Da mesma forma, o modelo do Castiçal, da Mesa e do Altar de ouro, e suas
posições no Santuário estão explicados na Guemará Menahot e Yoma.

\textbf{21} RESPEITAR O SANTUÁRIO

Por este preceito somos ordenados a ter uma atitude de grande e profunda
admiração e de temor para com o Santuário, e a venerá-lo em nossos
corações com receio e temor, pois é esse o respeito ao Santuário que foi
im­posto por Suas palavras, enaltecido seja Ele, "E Meu santuário
temereis" (Leví­tico 19:30).

A definição desse respeito está na Sifrá: "O que significa respeito? Que
não se deve entrar no Monte do Templo com seu cajado, ou suas
sandá­lias, ou sua mochila, ou com poeira em seus pés, ou usar o templo
como passa­gem; e de forma alguma se deve cuspir lá dentro". Está
explicado em várias passagens do Talmud que não é permitido a ninguém
ficar sentado no Tribu­nal, com exceção dos Reis da Dinastia de David.
Tudo isso é decorrente de Suas Palavras, enaltecido seja Ele, "E Meu
Santuárió temereis", às quais temos a obri­gação de obedecer por todos
os tempos, até mesmo em, nossos dias, quando, sem virtude de nossos
muitos pecados, ele foi destruído.

A Sifrá diz: "De que modo deduzimos que não somente enquanto o Santuário
existia, mas também depois que ele deixou de existir, o respeito deve
continuar? A Torah diz: 'Meus "Shabatot" ' guardareis, e Meu Santuário
te-

47. Números 4:14; Êxodo 27:7.

PRECEITOS POSITIVOS 95

mereis' (Levítico 19:30 e 26:2), ou seja, assim como a observação do
Shabat é para sempre, da mesma forma o respeito pelo Santuário é para
sempre".

No mesmo trecho lemos: "Seu respeito não deve ser para com o San­tuário,
mas sim para com Ele que nos deu as ordens referentes ao Santuário".

\textbf{22} A GUARDA DO SANTUÁRIO

Por este preceito somos obrigados a manter a guarda ao Santuário e a
vigiá-lo todas as noites e durante toda a noite, e dessa forma honrá-lo,
exaltá-lo e glorificá-lo. Este preceito está expresso em Suas Palavras,
enaltecido seja Ele, a Aarão "E tu e teus filhos contigo, estareis
diante da tenda da assinação" (Nú­meros 18:2) ou seja, você deverá
manter a guarda no Santuário para sempre. Este preceito também é
encontrado sob outra forma: "E manterão o serviço da guarda da tenda da
assinação" (Ibid. 4).

E está escrito no Sifrei: " `Tu e teus filhos contigo, est diante

da tenda do testemunho': os 'Cohanim' dentro, e os Levitas fora". u s
ia, eles devem guardar e vigiar o Santuário e ao redor dele, revezando-
\textbf{48}

A Mekhiltá diz: " 'E manterão o serviço da guarda da te da da
assi­nação' nos dá apenas um preceito posifi • ► e que maneira sabem que
há um preceito negativo envolvido també ? Na orah está escrito: 'E
mantereis o serviço da guarda da santidade' " (Ib •
.5)\textsuperscript{49}. i ica, assim, claro que a guarda do Santuário
constitui um preceito pos tivo.

No mesmo lugar lemos: "Engran • - ce o Santuário que haja guardas nele,
porque um palácio que tem guardas é diferente de um palácio que não os
tem", e é sabido que palácio (Palturim) é um nome para o Santuário. O
signi­ficado disto é que se exalta e se glorifica o Santuário ao se
designar guardas pa­ra vigiá-lo.

Todas as normas deste preceito estão explicadas nos Tratados Ta­mid
(primeiro capítulo) e Midot.

23 OS SERVIÇOS DOS LEVITAS NO SANTUÁRIO

Por este preceito ordena-se que apenas os Levitas,realizem determi­nados
serviços específicos no Santuário, tal como fechar os portões e cantar
durante a oferta de sacrifícios. Este preceito está expresso em Suas
palavras "E servirão os Levitas no serviço da tenda de assinação"
(Números 18:23).

O Sifrei diz: "Eu poderia supor que ele pode querer escolher entre
executar o serviço ou não e por isso a Torah diz: 'E servirão os Levitas
no servi­ço', ou seja, eles devem fazê-lo". É, portanto, um dever que
tem que ser execu­tado querendo ou não.

A natureza do serviço dos Levitas está explicada em diversos trechos em
Tamid e em Midot, e no segundo capítulo de Arakhin está explicado que
apenas os Levitas devem cantar.

Este preceito aparece novamente sob outra forma: "Servir em nome do
Eterno, seu Deus, como todos os seus irmãos levitas" (Deuteronômio
18:7),
\end{quote}

\begin{enumerate}
\def\labelenumi{\arabic{enumi}.}
\setcounter{enumi}{47}
\item
  \begin{quote}
  O revezamento deve ser feito entre os membros de cada grupo apenas, e
  não entre os grupos, ou seja, a guarda do interior do Santuário está
  restrita aos Cohanim e a do exterior aos Levitas.
  \end{quote}
\item
  \begin{quote}
  Ver o preceito negativo 67.
  \end{quote}
\end{enumerate}

\begin{quote}
PRECEITOS POSITIVOS 97

vo a cada Shabat, juntamente com incenso, e que os "Cohanim" devem comer
do pão que havia sido colocado no Shabat anterior.

As normas deste preceito estão explicadas no décimo primeiro capí­tulo
de Menahot.

28 A QUEIMA DO INCENSO

Por este preceito os "Cohanim" são ordenados a colocar incenso
diariamente, duas vezes por dia, no Altar de Ouro. Ele está expresso em
Suas palavras, enaltecido seja Ele, "E Aarão fará queimar sobre ele,
incenso de espe­ciarias; pela manhã, quando limpar as lamparinas, o
queimará" (Êxodo 30:7).

As normas deste preceito e o procedimento a ser seguido na queima diária
do incenso estão explicadas no início de Queretot e em várias passagens
de Tamid.

29 O FOGO PERPÉTUO DO ALTAR

Por este preceito somos ordenados a manter o fogo aceso no Altar, todos
os dias, constantemente. Ele está expresso em Suas palavras "Fogo
con­tínuo estará aceso sobre o altar" (Levítico 6:6), o que só pode
significar que eles são obrigados a colocar lenha no fogo todos os dias,
sem falta, pela manhã e ao anoitecer, como está explicado no segundo
capítulo de Yoma e no Trata­do Tamid.

O Talmud diz claramente: "Embora o fogo venha dos Céus, é um dever
mantê-lo queimando por meios comuns".

As normas deste preceito, que é o preceito relativo à preparação diária
do Fogo sobre o Altar estão expostas no quarto capítulo de Yoma, e no
segun­do capítulo de Tamid.

30 REMOVER AS CINZAS DO ALTAR

Por este preceito os "Cohanim" são ordenados a remover as cinzas do
Altar diariamente. Isto é chamado a Retirada das Cinzas, e o preceito
está expresso em Suas palavras, enaltecido seja Ele, "E vestirá o
"Cohen" a sua tú­nica de linho ... e separará a cinza" (Levítico 6:3).

As normas deste preceito estão explicadas em várias passagens dos
Tratados Tamid e Quipurim.

31 RETIRAR OS IMPUROS

Por este preceito somos ordenados a expulsar do Templo as pessoas
impuras. Ele está expresso em Suas palavras, enaltecido seja Ele, "Que
enviem do acampamento todo o leproso, e todo aquele que padece de fluxo
e todo o impuro por ter tocado os ossos de cadáver de pessc■a" (Números
5:2).

A palavra "acampamento" aqui significa o Acampamento da Presen­ça
Divina, que, em gerações posteriores, foi o equivalente à Corte do
Santuá­rio, como explicamos no início da Ordem Teharot, em nosso
Comentário so­bre a Mishná. O Sifrei diz: " 'Que enviem do acampamento'
é uma advertência para que pessoas impuras não entrem no Santuário em
estado de impureza".
\end{quote}

Este preceito aparece também sob outra forma: "Se houver entre vós

\begin{quote}
98 MAIMÔNIDES

um homem que não estiver puro por causa de derramamento de sêmem, de
noite, sairá para fora do acampamento" (Deuteronômio 23:11). O
"acampamen­to" aqui deve ser entendido como o Acampamento da Presença
Divina, uma vez que o próprio preceito diz: "Fora do acampamento os
enviareis" (Núme­ros 5:3) e que na Guemará de Pessahim se lê: " `Sairá
para fora do acampamen­to' significa o Acampamento da Presença Divina".

A Mekhiltá diz: " 'Ordena aos filhos de Israel que enviem do
acam­pamento': este é um preceito positivo. De que maneira concluímos
que tam­bém 'há um preceito negativo envolvido? A Torah diz: 'Para que
não contami­nem os seus acampamentos"'.
\end{quote}

O Sifrei diz: " 'Sairá para fora do acampamento' é um preceito

\begin{quote}
positivo".

32 HONRAR OS "COHANIM"

Por este preceito somos ordenados a exaltar os descendentes de Aa­rão,
para demonstrar-lhes honra e respeito, e a conferir-lhes alto grau de
santi­dade e dignidade, mesmo contrariando suas próprias objeções. Tudo
isso é pa­ra a glória do Eterno, enaltecido seja Ele, já que Ele os
escolheu para Seu servi­ço e para as ofertas de Seus sacrifícios. Este
preceito está expresso em Suas pa­lavras "E santifica-lo-ás, porque o
sacrifício de teu Deus ele oferece, santo será para ti" (Levítico 21:8)
que os Sábios interpretam assim: " santifica-lo-ás', quer dizer, ele
será o primeiro em todos os assuntos sagrados como, por exemplo, na
leitura da Torah; ele deverá ter a prioridade para ler as Bençãos nas
refei­ções; e ele deverá ser o primeiro a receber uma porção justa".

A Sifrá também diz: " 'E santifica-lo-ás' --- mesmo contra sua
vonta­de"; ou seja, este é um preceito estabelecido para nós, e não
depende da von­tade do "Cohe "

Da a forma ele diz: " 'Santos serão para seu Deus' (Ibid., 6) ---

mesmo contra ontade. 'E serão santidade' (Ibid.) --- inclusive os que
tive-

rem um defe' rtanto não devemos argumentar: "Já que este 'Cohen' não

está capacita erecer o sacrifício de seu Deus, por que deveríamos
dar-lhe

prioridade e nstrar-lhe honra e respeito?" Porque Ele disse: "E serão
san-

tidade", significando toda essa honrada família, incluindo tanto os
perfeitos co­mo os defeituosos.

As cláusulas apropriadas nas quais está estabelecido que devemos
tratá-los desta forma estão explicadas em diversos trechos da Guemará de
Ma-cot, Hulin, Bekhorot, Shabat, e em outros trechos.
\end{quote}

AS VESTES DOS "COHANIM"

\begin{quote}
Por este preceito os "Cohanim" são ordenados a ornar-se com ves­tes de
especial esplendor e beleza antes de servir no Santuário. Ele está
expres-

so em Suas palavras, enaltecido seja Ele, "E farão vestidos antidade
para

Aarão, teu irmão, para o esplendor e para a beleza" (Êx. 4); "E a seus

filhos farás chegar, e os farás vestir as túnicas" (Ibid., 29:►são as
Vestes\\
dos "Cohanim": oito vestimentas para o "Cohen Gado uatro para o
\end{quote}

\begin{longtable}[]{@{}ll@{}}
\toprule
\endhead
\begin{minipage}[t]{0.47\columnwidth}\raggedright
\begin{enumerate}
\def\labelenumi{\arabic{enumi}.}
\setcounter{enumi}{50}
\item
  \begin{quote}
  \textbackslash ter os preceitos negativos 70 e 71.
  \end{quote}
\item
  \begin{quote}
  C4 Sumo Sacerdote.
  \end{quote}
\end{enumerate}\strut
\end{minipage} & \begin{minipage}[t]{0.47\columnwidth}\raggedright
\strut
\end{minipage}\tabularnewline
\bottomrule
\end{longtable}

\begin{quote}
PRECEITOS POSITIVOS 99

"Cohen" comum. Caso o "Cohen"celebre o ofício com menos ou com mais
vestimentas do que o número designado para aquele ofício específico, o
ofício fica invalidado, e ele fica sujeito à morte pela mão dos Céus ---
quero dizer, aquele que celebrar o ofício com menos vestes do que o
número designado. Na Guemará de Sanhedrin ele também está entre os que
podem ser mortos pela mão dos Céus. Isto não está explicitamente dito
nas Escrituras, mas é derivado do versículo "E lhes cingirás cintos ...
e será para eles o sacerdócio" (Êxodo 29:9), cuja interpretação é:
"Enquanto usam as vestimentas, eles estão revesti-

dos de seu sacerdócio; quando não usam suas vesti , não estão revesti-

dos de seu sacerdócio", e se transformam em leigi explicado a seguir

que um leigo que oficia está sujeito à pena de mo

A Sifrá diz: " 'E pôs sobre ele o peitoral ico 8:8): esta passa-
\end{quote}

gem nos ensina regras que se aplicam a essa ocasião es cífica e também
regras

\begin{quote}
que se aplicam permanentemente; regras para os diários, e também as

regras para o culto do Dia do Perdão. Todos os d via oficiar com trajes

dourados, mas no Dia do Perdão com vestes de 1 h. branco".

Pela seguinte passagem da Sifrá fica claro que a colocação dessas
ves­tes é um preceito positivo: "Como concluímos que Aarão não colocou
as ves­tes do 'Cohen' apenas para seu próprio enaltecimento, e sim como
aquele que obedece a ordem de seu Rei? Pelas palavras da Torah 'E fez
como ordenou o Eterno a Moisés' "; ou seja, embora essas vestes, com seu
ouro, ônix, jásper e outras pedras preciosas, fossem de beleza
inigualável, não seria o gozo desta beleza que o "Cohen" deveria tomar
em consideração, mas apenas o cumpri­mento do preceito que Deus impôs a
Moisés, a saber, que ele deveria sempre usar essas vestes no Santuário.

Todas as normas deste preceito estão explicadas no segundo capítu­lo do
Zebahim e em vários trechos de Quipurim e de Sucá.

\textbf{34 OS "COHANIM" DEVEM CARREGAR A ARCA SAGRADA}
\end{quote}

Por este preceito somos ordenados a que os "Cohanim" carreguem

\begin{quote}
a 0\textsuperscript{5} obre os ombros, quando desejarmos transportá-la
de um lugar para

o tr sse preceito está expresso em Suas palavras, enaltecido seja Ele,
"Por-
\end{quote}

q e o serviço da santidade estava sobre eles; eles o levavam aos ombros"
(Nú-\\
meros 7:9). Embora este preceito tenha sido imposto naquela época aos
Levi-\\
tas, isso foi apenas por causa do número limitado de "Cohanim", já que
Aarão

\begin{quote}
foi o primeiro. Na idade, o cumprimento deste preceito compete aos "Co-

hanim", e são e devem carregar, como fica claro no livro de Joshuá e

no livro de Sa quando Davi ordenou trazer a Arca pela segunda vez,

o livro das Crôni gistra: "Assim os "Cohanim" e os Levitas se
santificaram

para levantar a Arca do Eterno, Deus Israel. E os filhos dos Levitas
carrega-

ram a Arca de Deus com as barras s' • re eus ombros, como Moisés
ordenara, de acordo com a palavra do Eter
\end{quote}

\begin{enumerate}
\def\labelenumi{\arabic{enumi}.}
\setcounter{enumi}{52}
\item
  \begin{quote}
  Ver o preceito negativo 74.
  \end{quote}
\item
  \begin{quote}
  O "Cohen Gadol".
  \end{quote}
\item
  \begin{quote}
  Que continha as duas tábuas com os 10 Mandamentos.
  \end{quote}
\item
  \begin{quote}
  Josh. 3:14; II Sam. 15:25.
  \end{quote}
\item
  \begin{quote}
  I Cron. 15:14-15 e 5:16.
  \end{quote}
\end{enumerate}

\begin{quote}
100 MAIMÔNIDES

Da mesma forma, ao se referir à divisão dos "Cohanim" em 24 Gru­pos, o
livro de Crônicas diz: "Essas eram suas posições em seus serviços para
entrar na casa do Eterno, de acordo com as leis dadas a eles pela mão de
Aarão, seu pai, como o Eterno, Deus de Israel, lhe havia ordenado". Os
Sábios expli­cam este versículo como significando que é dever dos
"Cohanim" carregar a Arca nos ombros e que é isso o que o Eterno, Deus
de Israel, ordenou. O Sifrei diz: " `De acordo com as leis dadas a eles
.... como o Eterno, Deus de Israel, lhe havia ordenado': onde foi que
Ele lhe deu essa ordem? Em 'Porém aos fi­lhos de Kehat não deu; porque o
serviço da santidade estava sobre eles; eles o levavam aos ombros' ".

Fica assim claro que este é um dos preceitos.

35 O ÓLEO DA UNÇÃO

Por este preceito somos ordenados a ter óleo feito para nós de acor­do
com uma composição específica, pronto para a Unção de todo "Cohen Ga-

dol" que venha esignado, como Ele diz: "E o 'Cohen Gadol', entre seus

irmãos, sobre c a cab a for derramado o óleo da unção" (Levítico 21:10).
Com esse óleo tamb se de eria ungir alguns dos reis, como está explicado
nas nor­mas deste pre e 0\textsuperscript{58}.

O b lo e todos os seus vasos foram ungidos com este óleo,

mas os vasos n\textsuperscript{-}o serão ungidos com ele no futuro, pois
o Sifrei diz explicita­mente: "Com a unção destes," --- ou seja, dos
vasos do Tabernáculo --- "todos os vasos foram santificados para
sempre", como Ele disse, enaltecido seja Ele: "Oléo de unção de
santidade será este para Mim por vossas gerações" (Exodo 30:31).
\end{quote}

As cláusulas deste preceito estão explicadas no início de Queretot.

\begin{quote}
36 OS "COHANIM" DEVEM OFICIAR

EM GRUPOS, REVEZANDO-SE NO SERVIÇO

Por este preceito os "Cohanim" são or•enados a oficiar em grupos, sendo
que cada grupo deve oficiar durante u se i ana, e a não oficiar to

ao mesmo tempo, exceto durante os Festivais todos os grupos d. em\\
participar igualitariamente, e quando qualquer m\textsuperscript{59} p
esente pode ofere' er sa-

crifícios. Aparece nas Crônicas que Davi e Sam 1 os d vidiram em 24
os\textsuperscript{6}°,

e na Guemará Sucá está explicado que durante vais todos partici 7am

de forma igual.

O trecho das Escrituras no qual está expresso este preceito é •
se­guinte: "E quando vier o 'Cohen', o qual descende da tribo de Levi,
de alguma das tuas cidades, de todo o Israel, onde ele habita e vier com
todo o desejo de sua alma ao lugar que escolheu o Eterno; e servir em
nome do Eterno, seu Deus, como todos os seus irmãos Levitas, que servem
ali diante do Eterno, igual porção receberão todos" (Deuteronômio
18:6-8). O Sifrei diz: " 'E vier com to-
\end{quote}

\begin{enumerate}
\def\labelenumi{\arabic{enumi}.}
\setcounter{enumi}{57}
\item
  \begin{quote}
  Ver também o preceito negativo 84.
  \end{quote}
\item
  \begin{quote}
  Qualquer "Cohen".
  \end{quote}
\item
  \begin{quote}
  1 Cron. 24:4-18.
  \end{quote}
\end{enumerate}

\begin{quote}
PRECEITOS POSITIVOS 101

do o desejo de sua alma' poderia ser a qualquer momento, por isso nas
Escritu­ras está dito: 'De alguma de tuas cidades', ou seja, quando todo
o povo de Is­rael estiver reunido em uma cidade durante os Festivais.
Poder-se-ia pensar que todos os grupos participavam igualitariamente das
oferendas nos Festivais, mes­mo daquelas que não decorriam
especificamente dos Festivais, por isso a To­rah esclarece: 'Exceto a
parte dos patrimônios paternos' (Deuteronômio 18:8). O que significa o
patrimônio paterno? 'Celebre você durante sua semana, que eu celebrarei
durante minha semana' "; ou seja, eles concordaram com o reve­zamento
dos grupos, e com todo o arranjo do ofício em grupos, com um novo grupo
celebrando a cada semana. O Targum explica o versículo da seguinte
ma­neira: "Exceto o grupo daquela semana, pois assim o decretaram os
pais".
\end{quote}

As normas deste preceito estão explicadas no final da Guemará de

\begin{quote}
Sucá.

37 OS "COHANIM" DEVEM FAZER-SE IMPUROS PEL•PARENTES MORTOS

Por este preceito os "C r. ani ' são ordenados a fazer-se impuros por
seus parentes lembrados na T
{\href{http://ah61.ma}{ah}\textsuperscript{61}. ma} vez que as
Escrituras lhes proí­bem, por respeito, de fazer-se impu • s p i os
mortos, mas lhes permitem fazê-lo por parentes, poder-se-ia pensar qu •
"Cohen" pode escolher entre fazer-se impuro ou não. Ele lhes impôs uma
obrigação positiva ao dizer: "Por ela se fará impuro" (Levítico 21:3).

A Sifrá diz: " 'Por ela se fará impuro' é um preceito positivo. Se ele
não deseja se fazer impuro, deverá fazê-lo contra sua vontade. Isso
aconteceu com o "Cohen" Yossi cuja esposa morreu na véspera de "Pessah"
e ele se re­cusou a se fazer impuro por ela; os Sábios então se
utilizaram de força e o obri­garam a fazer-se impuro, contra sua própria
vontade".
\end{quote}

Neste preceito está baseado o dever do luto, ou seja, a obrigação de ovo
de Israel de ficar de luto pelos parentes, que são em número de para
confirmar essa obrigação que Ele declarou expressamente que no

\begin{quote}
c o "Cohen", o qual está normalmente proibido de fazer-se impuro, ele

d verá fazê-lo a qualquer custo, como todos os outros israelitas, de
forma que a lei de luto não seja julgada com leviandade.

Tem sido demonstrado que o luto do primeiro dia está prescrito pe­la lei
das Escrituras. Na interpretação eles disseram explicitamente na Guemará

de Mo tan que o luto não deve ser guardado durante um Festival: "Se o

luto antes do Festival, o preceito positivo abrangendo todo o povo

de Isr sobrepõe ao preceito que lhe foi imposto individualmente". Por-

tanto, aro que as Escrituras obrigam a guardar o luto, mas apenas no
pri-

meiro dia, pois os outros seis dias foram impostos pelos Rabinos; e que
até mes­mo um "Cohen" é obrigado a observar o luto no primeiro dia, e
fazer-se impu­ro por seus parentes. Compreenda isso.
\end{quote}

As regras detalhadas deste preceito estão expostas no Tratado Mas-

\begin{enumerate}
\def\labelenumi{\arabic{enumi}.}
\setcounter{enumi}{60}
\item
  \begin{quote}
  Lembrados ou enumerados na Torah. Levítico 21:2-3.
  \end{quote}
\item
  \begin{quote}
  Mãe, pai, filho, filha, irmão, irmã; o marido e a mulher só têm essa
  obrigação por ordem poste­rior contida no Talmud.
  \end{quote}
\item
  \begin{quote}
  De que os Festivais devem ser cheios de alegria.
  \end{quote}
\end{enumerate}

\begin{quote}
102 MAIMÔNIDES

hkin, em vários trechos de Berakhot, Quetubot, Yebamot e Abodá Zará, e
na Sifrá, na passagem que começa com "Fala aos `Cohanim' " (Levítico 2 1
: 1 ).

A obrigação do "Cohen" de fazer-se impuro por seus parentes não se
estende às mulheres porque o "Cohen", que está proibido de fazer-
impu­ro por outros que não sejam seus parentes, tem a obrigação de fazê
o p e r pa-

rentes, mas a mulher da família do "Cohen" não está proibida de im-

pura por qualquer pessoa morta, como vou explicar oportuname con-

seqüentemente não tem o dever nem a obrigação de fazê-lo. Ela d v ardar

o luto, mas quanto a fazer-se impura ou não, depende de sua vontade.
Com­preenda isso.

38 A OBRIGAÇÃO DO "COHEN GADOL" DE CASAR-SE APENAS

COM UMA VIRGEM

Por este prece o "Cohen Gadol" é ordenado a casar-se com uma virgem.
Está expresso S s palavras, enaltecido seja Ele, "E ele, mulher em sua
virgindade, toma " (Le ítico 21:13).
\end{quote}

Nas Escrit ras\textsuperscript{65} es á explicitamente dito: "Rabi Akiba
afirmava que\\
até mesmo o nasci ento de um filho contrário a este preceito positivo
seria\\
considerado um bastúdo" como exemplo de uma união meramente contrá-\\
ria a um preceito positive eles citam o caso de um "Cohen Gadol" que
tenha\\
um relacionamento com uma mulher que não seja virgem; porque é um
princí-\\
vado de um preceito positivo tem a\\
claro que este é um preceito positi-\\
`Cohen Gadol' é obrigado a casar-

\begin{quote}
39 O HOLOCAUSTO DIÁRIO

Por este preceito somos ordenados a oferecer no Santuário todos os dias
dois cordeiros que são chamados de Oferendas Contínuas. Ele está
ex­presso em Suas palavras, enaltecido seja Ele, "Dois para cada dia, em
holocaus­to contínuo" (Números 28:3).

As normas que regem este preceito, a ordem dos sacrifícios e os mé­todos
a serem seguidos estão explicados no segundo capítulo de Yoma e no
Tratado Tamid.

40 A OFERTA DIÁRIA DO ALIMENTO PELO "COHEN GADOL"

Por este preceito somos ordenados a que o "Cohen Gadol" ofereça todos os
dias uma oblação pela manhã e uma ao anoitecer, chamada de Bolo
\end{quote}

\begin{enumerate}
\def\labelenumi{\arabic{enumi}.}
\setcounter{enumi}{63}
\item
  \begin{quote}
  No preceito negativo 166.
  \end{quote}
\item
  \begin{quote}
  Quetubot 30:A.
  \end{quote}
\item
  \begin{quote}
  Horayot 11:B.
  \end{quote}
\end{enumerate}

\begin{quote}
PRECEITOS POSITIVOS 103

do "Cohen Gadol", conhecida também como oblação do "Cohen" ungido. Este
preceito está expresso em Suas palavras, enaltecido seja Ele, "Esta é a
oferta de Aarão e seus filhos" (Levítico 6:13).

As normas deste preceito, bem como a hora e a maneira de fazer a
oferenda estão expostas no sexto e nono capítulos de Menahot e em vários
tre­chos em Yoma e Tamid.

41 A OFERTA ADICIONAL DO SHABAT

Por este preceito somos ordenados a oferecer um sacrifício todos os
Shabatot, além do holocausto diário. Ele está expresso em Suas palavras,
enal­tecido seja Ele, "E no dia de Shabat, dois cordeiros de um ano de
idade etc" (Números 28:9).

A ordem dos sacrifícios está explicada no segundo capítulo de Yo­ma e em
Tamid.

42 A OFERTA ADICIONAL DA LUA NOVA

Por este preceito somos ordenados a oferecer um sacrifício a cada lua
nova, além do holocausto diário, sendo esta a Oferta Adicional da Lua
No­va. Este preceito está expresso em Suas palavras, enaltecido seja
Ele, "E nos princípios de vossos meses oferecereis em holocausto ao
Eterno" (Números 28:11).
\end{quote}

43 A OFERTA ADICIONAL DE "PESSAH"

\begin{quote}
Por este preceito somos ordenados a oferecer um sacrifício em cada um
dos sete dias de "Pessah", além do holocausto diário, sendo esta uma
Ofer­ta Adicional do Festival do Azimo. Este preceito está expresso em
Suas pala­vras, enaltecido seja Ele, "E oferecereis por sete dias oferta
queimada ao Eter­no" (Levítico 23:8).

44 A OBLAÇÃO DA NOVA CEVADA

Por este preceito somos ordenados a fazer a oblação da cevada no
sexagésimo dia de Nissan, juntamente com o holocausto de um carneiro de
um ano, sem defeito. Este preceito está expresso em Suas palavras,
enaltecido seja Ele, "Trareis ao 'Cohen' um 'omer' das primícias de
vossa ceifa" (Levítico 23:10).

Esta oblação é chamada "A Oferta das Primícias" e ela está mencio­nada
em Suas palavras, enaltecido seja Ele, "E se ofereceres oblação de
primí­cias" (Levítico 2:14). A Mekhiltá diz: "Todo `se' na Torah implica
numa opção, exceto em três casos, em que é usado com relação a uma
obrigação; um deles é este: 'E se ofereceres oblação de primícias'. Você
está certo de que isto é uma obrigação? Talvez seja apenas uma
permissão. E dizem: 'Assim oferecerás a obla­ção de tuas primícias etc.'
(Ibid.), mostrando que isso é uma obrigação e não uma permissão".

Todas as normas deste preceito estão explicadas na íntegra no déci­mo
capítulo de Menahot.

104 MAIMÔNIDES

45 A OFERTA ADICIONAL DE "SHABUOT"

Por este preceito somos ordenados a oferecer uma Oferta Adicional também
no quinquagésimo dia após a oferta do "Omer", que é no sexagésimo dia de
Nissan. Esta é a Oferta Adicional da Festa das Semanas mencionada no
livro de Números. Este preceito está expresso em Suas palavras "E no dia
das priiinícias, quando oferecerdes oblação nova ao Eterno... E
oferecereis holocaus­to, para ser aceito com agrado pelo Eterno"
(Números 28:26-27).

46 LEVAR DOIS PÃES EM "SHABUOT"

Por este preceito somos ordenados, como está prescrito, a levar ao
Santuário dois pães na Festa das Semanas, juntamente com os sacrifícios
obri­gatórios da Oferenda do Pão, e a oferecer sacrifícios, como
prescrito no livro de Levítico; e depois que esses pães forem movidos,
os "Cohanim" devem comê-los juntamente com os cordeiros das Ofertas de
Paz. Este preceito está expres­so em Suas palavras, enaltecido seja Ele,
"De vossas habitações trareis dois pães para serem movidos, de duas
décimas partes de uma `efa' " (Levítico 23:17).

Está explicado no quarto capítulo de Menahot que este sacrifício, que
era um complemento da Oferta do Pão, é uma oferta a parte e diferente da
Oferta Adicional do dia. Já fornecemos explicação suficiente a este
respeito em nosso comentário no Tratado Menahot.

Todas as normas deste preceito estão explicadas em Menahot nos capítulos
quatro, cinco, oito e onze.

47 A OFERTA ADICIONAL DO ANO NOVO

Por este preceito somos ordenados a oferecer uma Oferenda Adi­cional no
primeiro dia de "Tishri", que é a Oferta Adicional do Ano Novo. Este
preceito está expresso em Suas palavras, enaltecido seja Ele, "E no
sétimo mês, no primeiro dia do mês... e oferecereis como holocausto,
para ser aceito com agrado pelo Eterno" (Números 29:1-2).

48 A OFERTA ADICIONAL DO DÉCIMO DIA DE "TISHRI"

Por este preceito somos ordenados a oferecer uma Oferenda Adi­cional no
décimo dia de "Tishri". Ele está expresso em Suas palavras, enalteci­do
seja Ele, "E no décimo dia, deste sétimo mês... E oferecereis holocausto
ao Eterno, para ser aceito com agrado" (Números 29:7-8).

49 O OFÍCIO DE "YOM QUIPUR"

Por este preceito somos ordenados a celebrar o Ofício do Dia, ou seja,
todos os sacrifícios e profissões de fé ordenados pelas Escrituras para
o Dia do Perdão, para expiar todos os nossos pecados. Esta é a instrução
que está expressa na porção "Aharé Mot" (Levítico 16:1-34).

PRECEITOS POSITIVOS 105

A prova de que ela constitui, em sua totalidade, apenas um preceito se
encontra no final do quinto capítulo de Quipurim: "Com relação a cada
ce­lebração de `Yom Quipur', mencionada na ordem prescrita, se algum
ofício for celebrado fora da ordem estabelecida é como se nenhum deles
tivesse sido ce­lebrado."

Todas as normas deste preceito estão explicadas no Tratado dedica­do
exclusivamente a este assunto, que é o Tratado de Yoma.

50 A OFERTA ADICIONAL DA FESTA DOS TABERNÁCULOS

Por este preceito somos ordenados a oferecer uma Oferenda Adi­cional no
Festival dos Tabernáculos. Ele está expresso em Suas palavras,
enal­tecido seja Ele, "E oferecereis por holocausto, oferta queimada
para ser aceita com agrado pelo Eterno" (Números 29:13). Esta é a Oferta
Adicional dos Ta­bernáculos.

51 A OFERTA ADICIONAL DE "SHEMINI ATZERET"

Por este preceito so cional no oitavo dia da Festa oitavo Dia da
Assembléia sol

O que nos faz consi ta em separado, diferente das oferecidas diariamente
durante o Festival dos Ta­bernáculos, é o princípio aceito que o oitavo
Dia da Assembléia Solene é, por si só, um outro Festival. Os Sábios
dizem claramente: "Ele é um Festival separa­do, com ofertas separadas".
Isto prova que sua oferenda é específica, deixando assim o assunto
perfeitamente claro.

52 AS TRÊS PEREGRINAÇÕES ANUAIS

Por este preceito somos ordenados subir ao Santuário três vezes por ano.
Ele está expresso em Suas palavras, e o seja Ele, "Três vezes cele-

brarás, pa im, festas no ano" (Êxodo 2 Escrituras deixam claro que

"subir" ri ica ir até lá com uma oferen te preceito está repetido vá-\\
rias veze \textsuperscript{9}

„ o Sifrei • "Três preceitos em ser obedecidos num Festi-

val, a sabèr: estejar, recer diante do Eterno, e alegrar-se". A Guemará

de Haguigá também Três preceitos são impostos a Israel num Festival:

festejar, comparecer diante do Eterno e alegrar-se". "Festejar"
significa levar uma Oferta de Paz, o que não é obrigatório para as
mulheres.

As normas deste preceito estão explicadas no Tratado Haguigá.
\end{quote}

\begin{enumerate}
\def\labelenumi{\arabic{enumi}.}
\setcounter{enumi}{66}
\item
  \begin{quote}
  Números 29:35-38.
  \end{quote}
\item
  \begin{quote}
  Deuteronômio 16:16.
  \end{quote}
\item
  \begin{quote}
  Ibid., 15; Êxodo 34:23.
  \end{quote}
\item
  \begin{quote}
  Haguigá 6:B.
  \end{quote}
\end{enumerate}

\begin{quote}
106 MAIMÔNIDES

53 COMPARECER DIANTE D TERNO DURANTE OS FESTIVAIS

Por este preceito somos ordenados a compar cer\textsuperscript{71} du
ante os Fes­tivais. Está expresso em Suas palavras, enaltecido seja Ele,
"Três ezes ao ano, aparecerão todos os teus homens diante do Eterno, teu
Deu " euteronômio 16:16). O significado deste preceito é que todo homem
deve subir ao Santuá­rio com todos os seus filhos homens que possam
andar sozinhos e oferecer um holocausto quando subir. Este é chamado o
Holocausto do Comparecimento. Nós já nos referimos às palavras dos
Sábios: "Três preceitos são obrigatórios durante um festival: festejar,
comparecer e alegrar-se".

As normas deste preceito, ou seja, o Preceito do Comparecimento também
estão expostas no Tratado Haguigá. Este preceito também não é
obri­gatório para as mulheres.

54 ALEGRAR-SE NOS FESTIVAIS

Por este preceito somos ordenados a alegrar-nos nos Festivais. Está
expresso em Suas palavras, enaltecido seja Ele, "E alegrar-te-ás na tua
festa" (Deuteronômio 16:14). Este é o terceiro dos três preceitos
observados num festival.

A obrigação mais importante imposta por este preceito é a das Ofer­tas
de Paz obrigatórias. Essas são outras Ofertas de Paz do Festival
adicionais às ofertas de Haguigá e são chamadas no Talmud de "Ofertas de
Paz da Alegria".

Com relação a essas Ofertas de Paz nos é dito o seguinte: "As mulhe­res
têm a obrigação de tomar parte na alegria". Como está nas Escrituras: "E
sacrificarás ofertas de pazes e comerás ali; e te alegrarás diante do
Eterno, teu Deus" (Deuteronômio 27:7). As normas deste preceito também
estão expostas no Tratado Haguigá.

As palavras "E alegrar-te-ás na tua festa" incluem o preceito dos
Sá­bios de que devemos alegrar-nos de todas as maneiras possíveis, como
comen­do carne nos festivais, bebendo vinho, vestindo roupas novas,
distribuindo frutas e doces às crianças e mulheres, e alegrando-nos com
instrumentos musicais e dançando no Santuário especificamente, sendo que
essa será a Alegria de "Beit Hashoeba". Todos esses tipos de regozijo
estão compreendidos em Suas pala­vras "E alegrar-te-ás na tua festa".

Dentre as maneir egrar-se, a mais obrigatória é a de beber vi-

nho, porque ela está especi ligada à alegria.

Como diz a Gue Um homem tem o dever de fazer com que

seus filhos e sua família se ale rance um Festival... De que maneira?
Com

vinho".

Diz ainda, mais adian : "Foi-nos ensinado: Rabi Yehudá ben Bete­ra diz
que quando o Santuário existia não podia haver outro tipo de regozijo a
não ser o de comer carne, como está escrito: 'E sacrif' 's ofertas de
pa-

zes...'. Mas agora, que o Santuário não mais existe, só há ijo com
vinho,

como está dito 'O vinho faz alegre o coração do hom '. Diz também:
\end{quote}

\begin{enumerate}
\def\labelenumi{\arabic{enumi}.}
\setcounter{enumi}{70}
\item
  \begin{quote}
  Ao Santuário Sagrado.
  \end{quote}
\item
  \begin{quote}
  Pessahim 109:A.
  \end{quote}
\item
  \begin{quote}
  Salmos 104:15.
  \end{quote}
\end{enumerate}

\begin{quote}
PRECEITOS PO ITIVOS 107

"Os hom•maneira apropriada a eles, e as mulheres de maneira apro­priada
a

A rah nos obriga a incluir nesse regozijo os pobres, os necessita­dos e
os estranhos. Suas palavras, enaltecido seja Ele, são: "E alegrar-te-ás
dian­te do Eterno, teu Deus, tu... e o peregrino, e o órfão, e a viúva"
(Deuteronômio 16:11).
\end{quote}

ABATER A OFERTA DE "PESSAH"

\begin{quote}
Por este preceito somos ordenados a sacrificar o cordeiro de "Pes­sah"
no décimo quarto dia de Nissan. Está expresso em Suas palavras,
enalteci­do seja Ele, "E o degolará toda a assembléia da congregação de
Israel, pela tar­de" (Êxodo 12:6). Aquele que não cumprir este preceito
e deliberadamente ne­gligenciar a oferenda .deste sacrifício no momento
determinado está sujeito à extinção, seja homem ou mulher, uma vez que
está claramente expresso na Gue­mará de Pessahim que a primeira Oferta
de "Pessah" é obrigatória para as mu­lheres da mesma forma que para todo
homem do povo de Israel, e que essa oferta tem prioridade sobre o
Shabat, ou seja, que ela deve ser oferecida no décimo quarto dia de
Nissan, mesmo que esse dia seja Shabat.

A pena de extinção está prescrita em Suas palavras "E o homem que está
puro, e não estiver em viagem, e deixar de fazer o `Pessah', essa alma
será banida de seu povo" (Números 9:13).

Na enumeração dos preceitos --- negativos --- cuja transgressão in­corre
na penalidade de extinção, no início do Tratado Queretot, estão
incluí­dos os preceitos positivos de "Pessah" e da circuncisão, como foi
mencionado na introdução.
\end{quote}

As regras detalhadas deste preceito estão expostas no Tratado Pessahim.

\begin{quote}
56 COMER A OFERTA DE "PESSAH"

Por este preceito somos ordenados a comer a Oferta de "Pessah" na décima
quinta noite de Nissan, de acordo com as condições especificadas, ou
seja, ela deve ser grelhada, deve ser comida numa casa, e deve ser
comida com pão ázimo e ervas amargas. Ele está expresso em Suas
palavras, enaltecido seja Ele, "E comerão a carne nesta noite, grelhada
no fogo, e pães ázimos, com ervas amargas comerão" (Êxodo 12:8).

Se alguém perguntar: "Por que você conta comer a Oferta de "Pes­sah", o
pão ázimo e as ervas amargas como um único preceito, e não como três,
uma vez qu omer pão ázimo é um preceito, comer ervas amargas é ou­tro e
comer a me da Oferta de "Pessah" é outro?", eu direi que é verdade que
comer ão zimo é um preceito por si só, como explicarei posteriormen
e\textsuperscript{75}; d mesma forma, comer a carne da Oferta de
"Pessah" tam­bém é por si sou preceito, como já foi mencionado. Mas
comer ervas amar­gas está colocado como dependente de comer a Oferta de
"Pessah", e não de­ve ser contado como um preceito separado. Isto está
provado pelo fato de que se deve comer a carne da Oferta de "Pessah" em
cumprimento ao preceito, quer haja ervas amargas disponíveis oú não, mas
não se comem ervas amargas
\end{quote}

\begin{enumerate}
\def\labelenumi{\arabic{enumi}.}
\setcounter{enumi}{73}
\item
  \begin{quote}
  Os homens devem alegrar-se de maneira apropriada a eles.
  \end{quote}
\item
  \begin{quote}
  Ver o preceito positivo 158.
  \end{quote}
\end{enumerate}

\begin{quote}
108 MAIMÔNIDES

a não ser com a carne da Oferta de "Pessah", porque Ele diz, enaltecido
seja Ele, "Com pães ázimos e ervas amargas comerá o sacrifício" (Números
9:11). Ao comer ervas amargas sem carne não se cumpre nenhuma obrigação,
e não se pode dizer que se cumpre o preceito de comer ervas amargas. Nas
palavras de Mekhiltá: " 'Grelhada no fogo, e pães ázimos, com ervas
amargas comerão'; isso nos ensina que o preceito referente à Oferta de
"Pessah" estipula que ela deve ser comida grelhada, com pão ázimo e
ervas amargas", ou seja, o preceito consiste na totalidade destes três.

Também está escrito: "De que modo você conclui que na ausência do pão
ázimo e das ervas amargas a obrigação pode ser cumprida comendo-se
apenas a Oferta de "Pessah"? Pelas palavras: 'A comerão' ", isto é, a
carne ape­nas. Também se poderia pensar que, da mesma forma que na
ausência de pão ázimo e de ervas amargas, a obrigação pode ser cumprida
comendo-se a Oferta de "Pessah", assim também na falta da Oferta da
"Pessah" a obrigação pode ser cumprida comendo-se pão ázimo e ervas
amargas, argumentando-se que uma vez que comer a Oferta de "Pessah" é um
preceito positivo, e comer pão ázi­mo e ervas amargas também é um
preceito positivo, se na ausência de pão ázi­mo e ervas amargas a
obrigação pode ser cump comendo-se apenas a Ofer­ta de "Pessah", então
deveria concluir-se qu na lta da Oferta de "Pessah" a obrigação pode ser
cumprida comendo-se p o ázi o e ervas amargas. As Es­crituras dizem, a
esse respeito: 'A comerão "\textsuperscript{76}.

Está também escrito: " 'A comerão' dest s palavras deve-se concluir que
a Oferta de "Pessah" deve ser comida até sa edade, enquanto que o pão
ázimo e as ervas amargas devem ser comidos antes que o estado de
saciedade seja atingido", uma vez que a essência do preceito consiste em
comer a carne, como Ele disse: "Comerão a carne nesta noite", enquanto
que comer as ervas amargas é uma obrigação complementar a comer a carne,
como estas citações deixam claro para os que as compreendem.

Uma prova evidente da exatidão de nosso parecer está numa afirma­ção do
Talmud: "Comer ervas amargas hoje em dia é apenas uma regra
estabe­lecida pelos Rabinos" porque no que diz respeito à Torah, não há
obrigatorie­dade de comê-las sozinhas, mas elas devem ser comidas com a
carne da Oferta de "Pessah". Isto constitui prova clara e evidente de
que elas são acessórias ao preceito e não constituem um preceito
individual.
\end{quote}

As normas deste preceito também estão expostas no Tratado Pessahim.

\textbf{ABATER A SEGUNDA OFERTA DE "PESSAH"}

\begin{quote}
Por este preceito aquele que não tenha podido oferecer a primeira Oferta
de "Pessah" é ordenado a abater a Segunda Oferta de "Pessah". Este
preceito está expresso em Suas palavras, enaltecido seja Ele, "No
segundo mês, aos 14 dias do mês, pela tare • . celebrará" (Números
9:11).
\end{quote}

Aqui novamente s.f. \textbf{•} ia objetar e perguntar: "Por que você
con-

\begin{quote}
ta a segunda Oferta de "Pessa i ransgredindo dessa forma a regra que vo-

cê estabeleceu no Sétimo Fun. 1\textsuperscript{,} o de que uma lei
pértencente a um preceito
\end{quote}

\begin{enumerate}
\def\labelenumi{\arabic{enumi}.}
\setcounter{enumi}{75}
\item
  \begin{quote}
  Ou seja, "comerão" a Oferta de "Pessah".
  \end{quote}
\item
  \begin{quote}
  Como um preceito diferente.
  \end{quote}
\end{enumerate}

\begin{quote}
PRECEITOS POSITIVOS 109

não é por si só considerada como um preceito separado? A pessoa que
assim argumentar deve saber que os Sábios sustentaram diferentes
opiniões sobre a questão da Segunda Oferta de "Pessah" ser considerada
como uma continua­ção da primeira ou como um preceito ind endente e a
conclusão a que se che­gou foi que essa obrigação é um prec o di erente
e que, consequentemente, deve ser e . . •s erado em separado.
\end{quote}

A G k emará de Pessahim di 78: " a opinião de Rabi, fica-se sujeito

gunda. as Ra PA Nathan diz: fica-se sujeito à extinção por causa da
primeira,\\
à extin \textsuperscript{-}•\textsuperscript{79} po , causa da primeira
e fi a- sujeito à extinção por causa da se-

\begin{quote}
4\textsubscript{,1} se-

mas não • . segunda; e Rabi Hananyá ben Akabya diz: Não se fica sujeito
à extinção nem por causa da primeira, a menos que se deixe de levar a
segunda"

A Guemará passa a perguntar: "Em que eles diferem?". E respon : "Rabi
sustenta que o segundo é um Festival em separado, e Rabi Nathan sgst
n-ta que ele é apenas complementar ao primeiro". Isso explica nossa
afirma` 80

A Guemará diz ainda, no mesmo trecho: "De acordo com isa , se alguém
deliberadamente negligenciar ambas", ou seja, se alguém delibera mente
deixar de levar tanto a Primeira Oferta de "Pessah" quanto a Segunda,
"todos concordam que ele é culpado. Se ele negligenciar as duas
involuntaria­mente, todos concordam que ele não é culpado. Se ele
negligenciar a primeira intencionalmente e a segunda involuntariamente,
ele é culpado e está sujeito à extinção, de acordo com o Rabi e com Rabi
Nathan, é inocente e não está sujeito à punição, de acordo com Rabi
Hananyá ben Akabya. Da mesma forma, se seu erro for intencional no caso
da primeira, e ele trouxer a oferenda na se­gunda, ele é culpado, de
acordo com o Rabi", porque na sua opinião o segun­do "Pessah" não é mer
mente complementar ao primeiro, e a lei segue, em tod ses casos, o r do
Rabi.

Este prec é obrigatório para as mulheres porque está expli-

c' do\textsuperscript{81} ue a Segun•pcional para as mulheres.

As norma preceito estão explicadas na Guemará de Pessahim.

58 COMER A SEGUNDA OFERTA DE "PESSAH"

Por este preceito somos ordenados a comer a carne da segunda Oferta de
"Pessah" na noite do décimo quinto dia de Iyar juntamente com pão ázimo
e ervas amargas. Ele está expresso em Suas palavras, enaltecido seja
Ele, refe­rentes a esta também: "A comerão com pão ázimo e ervas
amargas" (Números 9: 1 1)
\end{quote}

As normas detalhadas deste preceito também estão expostas em Pes-

\begin{quote}
sahim.

É óbvio que este preceito não é obrigatório para as mulheres já que elas
não são obrigadas a fazer o abatimento, como explicamos, portanto não
resta dúvida de que elas não são obrigadas a comer esta oferenda.
\end{quote}

\begin{enumerate}
\def\labelenumi{\arabic{enumi}.}
\setcounter{enumi}{77}
\item
  \begin{quote}
  Pessahim 93:A.
  \end{quote}
\item
  \begin{quote}
  Punição à pessoa que não abater a oferta de "Pessah".
  \end{quote}
\item
  \begin{quote}
  De que os Sábios diferem quanto à classificação da Segunda Oferta de
  "Pessah".
  \end{quote}
\item
  \begin{quote}
  No Tratado Pessahim.
  \end{quote}
\item
  \begin{quote}
  A Segunda Oferta de "Pessah".
  \end{quote}
\end{enumerate}

\begin{quote}
110 MAIMÔNIDES
\end{quote}

TOCAR AS .CORNETAS NO . SANTUÁRIO

\begin{quote}
Por este preceito somos ordenados a fazer soar as cometas no San­tuário
ao oferecer qualquer um dos sacrifícios sazonais, e ele está expresso em
Suas palavras, enaltecido seja Ele, "E também no dia de vossa alegria,
nas vos­sas solenidades fixas, e nos princípios de vossos meses,
tocareis as cometas so­bre vossos holocaustos etc." (Números 10:10). Os
Sábios dizem explicitamen­te que este é o preceito das cometas.

As normas deste prèceito estão explicadas no Sifrei, em Rosh Hashaná e
Taaniot, uma vez que somos ordenados a tocar as cometas em épocas de
difi­culdades e infortúnios, quando clamamos pelo Eterno, enaltecido
seja Ele, de acor­do com Suas palavras "E quando estiverdes em guerra em
vossa terra, contra o adversário, que vos oprime, tocareis retinindo o
Shofar' " (Números 10:9).

60 OFERECER GADO COM IDADE MÍNIMA DETERMINADA

Por este preceito somos ordenados a que todo gado que trouxer­mos como
oferenda tenha 8 dias ou mais de idade, e não menos. Este é o pre­ceito
da oferenda cujo momento de ser aceita ainda não chegou por motivos
físicos e ele está expresso em Suas palavras, enaltecido seja Ele,
"Ficarão por sete dias atrás de sua mãe" (Levítico 22:27).

Este preceito também nos é dado de uma outra forma: "Sete dias estará
com sua mãe" (Êxodo 22:29). Isto se aplica a todas as oferendas de todos
os tipos, privadas e públicas.

Das palavras "E do oitavo dia em diante serão aceitos por sacrifício,
como oferta queimada ao Eterno" (Levítico 22:27) concluímos que antes
disso eles não seriam aceitáveis. Assim, está claramente proibido
oferecer um animal que não tenha atingido idade para ser aceito; mas
como este é um preceito ne­gativo derivado de um preceito positivo sua
transgressão não acarreta a pena de flagelo, e aquele que trouxer um que
não tenha alcançado a idade certa não será açoitado, como está explicado
no capítulo "Ele e seus filhotes", onde tam­bém se lê: "Desconsidere a
oferenda cuja época ainda não tenha chegado, pois a Escritura a justapôs
a um preceito positivo".

As normas deste preceito estão explicadas na Sifrá, no final do Tra­tado
Zebahim.

61 OFERECER APENAS SACRIFÍCIOS PERFEITOS

Por este preceito somos ordenados a oferecer ao Eterno apenas espé­cimes
perfeitos, sem os defeitos mencionados nas Escrituras, e livres de to as
imperfeições consideradas como defeitos pela Tradição. Este preceito est
presso em Suas palavras, enaltecido seja Ele, "Estes deverão ser sem
defe't ra que sejam aceitos" (Levítico 22:21), sobre as quais a Sifrá
diz: " 'Este rão ser sem defeitos para que sejam aceitos': este é um
preceito positi

83. Ver também preceitos negativos 91 a 96.

PRECEITOS POSITIVOS 111

As palavras "Ser-vos-ão eles sem defeito, igualmente as suas libações"
(Números 28:31) foram citadas como sendo a prova de que os vinhos das
liba­ções e seus óleos e farinha fina devem ser absolutamente perfeitos
e sem qual­quer tipo de defeito.
\end{quote}

As normas deste preceito estão explicadas no oitavo capítulo de

\begin{quote}
Menahot.

62 LEVAR SAL COM C SACRIFÍCIO

Por este preceito somos orden..os a oferecer sal com cada sacrifí­cio.
Ele está expresso em Suas palavras, enaltecido seja Ele, "E toda tua
oferta de oblação temperarás com sal" (Levítico
2:13).\textsuperscript{84}
\end{quote}

As normas deste preceito estão explicadas na Sifrá e em Menahot.

\begin{quote}
63 O HOLOCAUSTO

Por este preceito somos ordenados quanto ao procedimento a se­guir ao
oferecermos o Holocausto. Ou seja, todo o Holocausto, seja ele uma
oferta privada ou pública, deve ser oferecido de uma maneira
pré-estabelecida. Este preceito está expresso em Suas palavras no
Levítico, "Quando algum de vós oferecer sacrifício ao Eterno ... Se seu
sacrifício for holocausto de gado" etc (Levítico 1:2-3).

64 O SACRIFÍCIO DE PECADO

Por este preceito somos ordenados a oferecer o Sacrifício de Peca­do,
seja ele de que tipo for, da maneira especificada. Este preceito está
expres­so em Suas palavras, enaltecido seja Ele, "Esta é a lei do
sacrifício de pecado" etc. (Levítico 6:18)

No Levítico está explicado também de que forma o sacrifício deve ser
oferecido, que parte dele deve ser queimada e que parte deve ser comida.

65 O SACRIFÍCIO DE DELITO

Por este preceito somos ordenados a oferecer o Sacrifício de Delito de
uma determinada maneira. Este preceito está expresso em Suas palavras,
enal­tecido seja Ele, no Levítico: "E esta é a lei do sacrifício de
delito" etc. (Leví­tico 7:1).

As Escrituras explicam como este sacrifício deve ser oferecido, que
parte dele deve ser queimada e que parte deve ser comida.

66 O SACRIFÍCIO DE PAZ

Por este preceito somos ordenados a oferecer o Sacrifício de Paz da
forma especificada. Este preceito está expresso em Suas palavras "E se
sacrifí­cio de pazes é sua oferta" etc. (Levítico 3:1) e "É esta a lei
do sacrifício de pazes ... Se por ação de graças a oferecer" etc
(Levítico 7:11-12).

84. Ver também o preceito negativo 99.

112 MAIMÔNIDES

Esses quatro rituais --- o Holocausto, o Sacrifício de Pecado, o
Sacri­fício de Delito e o Sacrifício de Paz --- compõem todo o ritual
dos sacrifícios, uma vez que todas as ofertas de animais, sejam elas
trazidas por um indivíduo ou pela congregação, pertencem a uma dessas
quatro categorias, embora o sa­crifício de Delito seja sempre uma oferta
individual, como explicamos em di­versas ocasiões.

O Tratado Zebahim co • em normas destes quatro preceitos e as­suntos
relacionados a eles, e explica quai são as cerimônias obrigatórias, o
que não pode ser feito sem infringir a, lei \textsuperscript{85}, que
invalida um sacrifício e qual é o procedimento correto.

67 A OBLAÇÃO

Por este preceito somos ordenados a oferecer a Oblação de acordo com as
regras estipuladas para cada um de seus vários tipos. Este preceito está
expresso em Suas palavras "E quando uma alma oferecer uma oblação ao
Eter­no" etc. (Levítico 2:1); "E se tua oferta for oblação feita na
assadeira" etc (Ibid., 5); "E se oblação de panela é tua oferta," etc.
(Ibid., 7), que são complementa­das pelas seguintes palavras,
encontradas mais adiante: "E esta é a lei da obla­ção" (Ibid., 6:7).

As normas deste preceito, com seus vários aspectos, estão explica­das no
Tratado Menahot, que se dedica especificamente a este assunto.

68 O SACRIFÍCIO D UM TRIBUNAL
\end{quote}

QUE COMETE

\begin{quote}
Por este preceito o Tribunal so ele tenha tomado uma decisão e

Este preceito está express se a congregação de Israel pecar por da
assembléia" etc. (Levítico 4:13).

As normas e condições deste preceito estão todas explicadas no Tra­tado
Horayot e em vários trechos do Tratado Zebahim.

69 O SACRIFÍCIO DE PECADO
\end{quote}

tenha cometido involuntariamen­rdenado a oferecer um Sacrifício as,
enaltecido seja Ele, "E se uma . (Levítico 4:27).

\begin{quote}
Esta oferta é um Sacrifício elecido de Pecado, ou seja, um Sa-

crifício de Pecado que deve ser constituído de um animal.

Já explicamos que os pecados passíveis da penalidade de Sacrifício de
Pecado são os mesmos que acarretam a penalidade de extinção, caso sejam
\end{quote}

\begin{enumerate}
\def\labelenumi{\arabic{enumi}.}
\setcounter{enumi}{84}
\item
  \begin{quote}
  E se a lei for infringida isso implica em punição.
  \end{quote}
\item
  \begin{quote}
  Que tenha levado a congregação a cometer um pecado.
  \end{quote}
\item
  \begin{quote}
  Ver Mishné Torah Hilchot Shegagot, 1? capítulo, 4! Halachá.
  \end{quote}
\end{enumerate}

\begin{quote}
PRECEITOS POSITIVOS 113

cometidos deliberadamente, desde que o pecado implique na violação de um
preceito negativo e que haja algum ato relacionado a ele, como explicado
no início do Tratado Queretot.

As normas deste preceito estão explicadas nos Tratados Horayot e
Queretot, e em vários trechos de Shabat, Shabuot e Zebahim.

70 \textbf{O SACRIFÍCIO SUSPENSIVO DE DELITO}

Por este preceito somos ordenados a oferecer um sacrifício determi­nado
em caso de dúvida quanto a um dos pecados capitais que implicam na
penalidade de extinção, se cometidos voluntariamente, e em Sacrifício
estabe­lecido de Pecado, se cometidos involuntariamente. Este sacrifício
é chamado

de Sacrifício Suspensivo de Delito. Um exemplo de um caso de • . que im-

plicar* Sacrifício Suspensivo de Delito é o seguinte: su•nha que uma

pes a tenh diante de si dois pedaços de gordura, uma de r

\textbf{\textsuperscript{8}} e outra de coOção\textsuperscript{89}. e
come um dos dois pedaços e o outro é comido • o outra pes­soá, ou per d
ido. Uma dúvida surge então em sua mente quanto a se o pedaço de go a
que ele comeu foi o permitido ou o proibido. Nesse caso, ele deve
oferecer um sacrifício expiatório, pela dúvida que surgiu, chamado
Sacrifício Suspensivo de Delito. Se posteriormente ele se certificar que
o pedaço de gor­dura que comeu era o de rins, fica confirmado que pecou
involuntariamente e ele deverá então oferecer um Sacrifício Estabelecido
de Pecado.

O versículo referente a esta oferenda está no livro de Levítico: "E se
alguma alma pecar e fizer um dos preceitos do Eterno, daqueles que não
se devem fazer, e não souber, e for culpado, levará a sua iniqüidade. E
trará do rebanho, um carneiro sem defeito, no valor de dois siclos, por
sacrifício de de­lito, ao 'Cohen'; e expiará por ele, o 'Cohen', pelo
erro que cometeu sem sa­ber" (Levítico 5:17-18), ou seja, porque ele não
sabia se tinha pecado ou não. Isto é o que os Sábios chamam de Pecado
Cometido Involuntariamente.
\end{quote}

As normas deste preceito estão explicadas no Tratado Queretot.

\begin{quote}
\textbf{71 O SACRIFÍCIO INCONDICIONAL DE DELITO}

Por este preceito aquele que cometer certas transgressões é ordena­do a
oferecer um Sacrifício de Delito para obter o perdão. Essa oferenda é
cha­mada de Sacrifício Incondicional de Delito.

As transgressões que requerem este sacrifício são: sacrilégio, roubo,
manter uma ligação com uma escrava prometida em casamento e jurar em
falso no caso de algo entregue sob custódia. Aquele que cometer um
sacrilégio sem intenção, ou seja, que usufruir de uma "perutá" com algo
que pertença ao San­tuário, seja dos Objetos Consagrados do tesouro do
Santuário ou dos do Altar; aquele que roubar uma "perutá" ou mais de um
amigo e fizer juramento falso a respeito; e aquele que mantiver uma
ligação com uma escrava comprometi­da, seja sem intenção ou
voluntariamente, todos eles terão a obrigação de tra-
\end{quote}

\begin{enumerate}
\def\labelenumi{\arabic{enumi}.}
\setcounter{enumi}{87}
\item
  \begin{quote}
  A gordura dos rins não é "Casher".
  \end{quote}
\item
  \begin{quote}
  A gordura do coração é "Casher".
  \end{quote}
\end{enumerate}

\begin{quote}
114 MAIMÔNIDES

zer uma oferenda por seus pecados --- não um Sacrifício de Pecado, mas
sim um Sacrifício de Delito chamado Sacrifício Incondicional de Delito.

Com relação ao sacrilégio, Suas palavras, enaltecido seja Ele, são: "E
pec ro nas santidades do Eterno, trará por seu delito" etc. (Levítico
5:15).

E dis E negar ao seu companheiro a coisa que lhe

tódia.. rou em falso ... e como oferta de delito trará

um carneiro sem defeito" etc. (Ibid., 21-22, 25). E d se deitar com uma
mulher ... e ela for escrava despo•a

72 O SACRIFÍCIO DE MAIOR VALOR OU DE MENOR VALOR
\end{quote}

ferecer um Sacrifício de Maior sgressões.

\begin{quote}
rifício são: impurificar o San-jurar em falso em relação a mpuro por
alguma das fontes trodução à Ordem de Teharot, entrar sem querer no
Santuário, o que implicaria em impurificação do Santuá­rio; se alguém
sem querer comer carne sagrada, o que implicaria em impurifi­cação dos
Objetos Santificados do Santuário; se alguém pronunciar um jura­mento e
involuntariamente deixar de cumpri-lo; se alguém, jurar em falso em
relação a um testemunho, seja sem querer ou intencionalmente, em todos
es­ses casos ele terá que oferecer o que chamamos de um Sacrifício de
Maior ou Menor Valor.

Este preceito está expresso em Suas palavras, enaltecido seja Ele, "E
quando alguma alma pecar, sob juramento ...; se alguma alma tocar em
alguma coisa impura... e lhe for oculto que estava impuro e o souber
depois, e se tor­nar culpado; ou quando alguma alma jurar, pronunciando
com os lábios... e lhe for oculto e o souber depois que foi culpado de
uma dessas coisas ... trará como sacrifício ... E se as suas posses não
lhe permitirem trazer ..." etc. (Levíti­co 5:1-11).

Este sacrifício é chamado um Sacrifício de Maior ou Menor Valor por­que
ele não está especificado, variando de acordo com as posses do
transgres­sor que tem que oferecê-lo.

As normas deste preceito também estão explicadas nos Tratados Que­retot
e Shabuot.

73 CONFESSAR

Por este preceito somos ordenados a fazer a confissão oral dos peca­dos
que tivermos cometido contra o Eterno, enaltecido seja Ele, depois de
nos termos arrependido deles. Esta é a maneira de fazer a confissão:
`Oh, Deus,
\end{quote}

\begin{enumerate}
\def\labelenumi{\arabic{enumi}.}
\setcounter{enumi}{89}
\item
  \begin{quote}
  Com relação ao roubo e falso testemunho em caso de algo entregue sob
  custódia.
  \end{quote}
\item
  \begin{quote}
  Com relação a manter uma ligação com uma escrava noiva.
  \end{quote}
\item
  \begin{quote}
  Ou seja, alguém perjura que não pode testemunhar quando na realidade
  pode.
  \end{quote}
\end{enumerate}

\begin{quote}
PRECEITOS POSITIVOS 115

eu pequei, eu cometi injustiça, eu infringi ...". Deve-se elaborar a
confissão e pedir perdão com toda a eloqüência de que se for capaz.

Deve-se notar que mesmo no caso de pecados pelos quais se deve oferecer
um dos sacrifícios anteriormente especificados, a fim de obter-se o
per­dão prometido pelo Eterno, é preciso confessar-se no momento da
oferta. Isso se depreende de Suas palavras, enaltecido seja Ele, "Fala
aos filhos de Israel: Quando homem ou mulher fizer algum dos pecados do
homem ... E confes­sará os seus pecados que cometerá" (Números 5:6-7).
Comentando esse versí­culo, a Mekhiltá diz: "Uma vez que está escrito:
'Confessará aquilo em que pe­cou' (Levítico 5:5), deduzimos que ele deve
confessar 'junto com o pecado' que cometeu, ou seja, junto ao sacrifício
de pecado enquanto o mesmo ainda estiver vivo, e não depois de abatido.
O versículo não nos diz que o indivíduo deve confessar-se por qualquer
pecado a não ser pelo de entrar no Santuário em estado de impureza".
Isso é assim porque o versículo "Confessará aquilo em que pecou" aparece
no trecho "Vayikra" das Escrituras que se refere à im­purificação do
Santuário e dos Objetos Santificados, bem como às transgres­sões
mencionadas nele, como explicamos. É por isso que a Mekhiltá diz que
desse versículo se pode deduzir a obrigação de confessar-se apenas no
caso de se ter impurificado o Santuário. "De que modo você conclui sua
aplicação em caso de violação de qualquer um dos outros preceitos? Pelas
palavras das Escri­turas: 'Fala aos filhos de Israel ... E confessará'.
De que modo você conclui que a obrigação se aplica aos casos de pena de
extinção ou morte? Pelas palavras `Seus pecados', ou seja, todos os seus
pecados, estendendo a aplicação aos preceitos negativos. 'Que cometerá'
abrange as transgressões dos preceitos positivos.

A Mekhiltá diz, no mesmo trecho: " 'Algum dos pecados do homem': isto
significa transgressão dos preceitos relacionados com outros homens, co
mo roubo, assalto e maledicência. 'Por falsear em nome do Eterno': isso
inc
\end{quote}

ma': isso estende a ri­to à pena de mo e\textsuperscript{93}. estão para
ser c nde­Escrituras dizem:

atória para aquele que

sabe c, ue não cometeu nenhum delito e contra • qual foi prestado falso
testemunho".

Assim, fica claro que em todos os tipos de pecado, sejam eles gran­des
ou pequenos, inclusive os casos de transgressões dos preceitos
positivos, acarr a obrigação de confessar-se.

Como o preceito "Confessará" só está mencionado com relação à
\textsubscript{1}6 a rigaçã • de oferecer um sacrifício, poderia
ocorrer-nos a idéia de que a con­ssão não é uma obrigação em si, mas
apenas algo acessório ao sacrifício. Por sso\textsuperscript{95} fo m
obrigados a explicar isso na Mekhiltá da seguinte forma: "Poder­e-ia p
sar que a confissão é necessária quando se oferece um sacrifício; de

\begin{quote}
aneira sabemos que ela é necessária mesmo quando não há uma oferen­da?
Pelas palavras das Escrituras: 'Fala aos filhos de Israel: ... e
confessará'. Poder-se-ia ainda pensar que a confissão só é obrigatória
na Terra de Israel; de que modo se conclui que ela também é obrigatória
na Diásp9ra? Pelo versículo 'E
\end{quote}

\begin{enumerate}
\def\labelenumi{\arabic{enumi}.}
\setcounter{enumi}{92}
\item
  \begin{quote}
  Por sentença judicial.
  \end{quote}
\item
  \begin{quote}
  Ver preceito positivo 180.
  \end{quote}
\item
  \begin{quote}
  Os Sábios.
  \end{quote}
\end{enumerate}

\begin{quote}
116 MAIMOPÜDES

confessarão a sua iniquidade, e a iniquidade d ais' (Levítico
26:40)\textsuperscript{96}.

niel disse: 'A ti, ó Eterno, pertence a justiça'

Fica assim claro, por tudo o que dis • s, que a confissão é por si

só uma obrigação independente e é obrigatória ao transgressor por cada
peca­do que ele cometer, seja na Terra de Israel ou fora dela, quer ele
tenha ofereci­do um sacrifício ou não. Em todos os casos ele fica
obrigado a confessar, de acordo com Suas palavras, enaltecido seja Ele,
"E confessará os seus pecados".

A Sifrá diz também: " 'E manifestará' (Levítico 16:21): isso significa
confissão oral".
\end{quote}

As normas deste preceito estão explicadas no último capítulo de Qui-

\begin{quote}
purim.
\end{quote}

74 A OFERENDA ADA POR UM "ZAB"

\begin{quote}
Por este preceito um " ordenado a levar, ao se curar, um

sacrifício de "duas rolas ou dois po os, um por holocausto e o outro por

sacrifício de pecado". Este é o sacrifíc do "zab", cujo perdão não é
alcança­do até que ele o leve. Este preceito está expresso em Suas
palavras, enaltecido seja Ele, "E quando estiver limpo de seu fluxo quem
o tiver... E no oitavo dia tomará para si duas rolas" etc. (Levítico
15:13-14).

75 A OFERENDA VADA POR UMA "ZAB/V'

Por este preceito uma "z *a"\textsuperscript{99} é • rdenada a levar, ao
se curar, uma oferenda de "duas rolas ou dois po *nho '. Este é o
sacrifício da "zaba", cu­jo perdão não é alcançado até que ela • eve.

Talvez se pudesse objetar o seguinte: "Uma vez que o sacrifício de um
"zab" é o mesmo que o de uma "zaba" e que você leva em consideração
apenas o tipo de oferenda envolvido e não se preocupa com o tipo de
trans­gressor --- como acontece no caso do Sacrifício de Pecado, do
Sacrifício Incon­dicional de Delito, do Sacrifício Suspensivo de Delito
e do Sacrifício de Maior ou Menor Valor, onde você enumera cada um deles
como um preceito separa­do, sem se preocupar sobre os tipos de
transgressões pelas quais o sacrifício em questão pode ser requerido ---
por que você não adota o mesmo método neste caso e não deixa de lado o
fato de que são indivíduos diferentes, uma vez que deles se exige o
mesmo tipo de sacrifício?"

Quem usar essa argumentação deve saber que a oferenda do homem ou da
mulher que padecem de fluxo não é levada por causa do pecado, mas é
obrigatória em certas circunstâncias. Se a natureza do fluxo fosse
idêntica no homem e na mulher, como idêntico é o nome, um sendo chamado
de "zab" e a outra de "zaba", então seria apropriado contar-se os dois
preceitos como um só. Mas este não é realmente o caso, porque o que faz
com que o homem deva levar uma oferenda é a saída do sêmem, enquanto que
se a mulher tivesse algum tipo de emissão de sêmem, ela não seria uma
"zaba"; o que obriga a mu-
\end{quote}

\begin{enumerate}
\def\labelenumi{\arabic{enumi}.}
\setcounter{enumi}{95}
\item
  \begin{quote}
  Esse versículo se refere à época em que o povo está disperso.
  \end{quote}
\item
  \begin{quote}
  Daniel, 9:7. Daniel viveu fora de Israel.
  \end{quote}
\end{enumerate}

\begin{quote}
cerviz quebrada, sendo cada um deles um preceito diferente, como foi
exposto.
\end{quote}

\begin{enumerate}
\def\labelenumi{\arabic{enumi}.}
\setcounter{enumi}{97}
\item
  \begin{quote}
  Homem que padece de fluxo.
  \end{quote}
\item
  \begin{quote}
  Mulher que padece de fluxo após o período menstrual.
  \end{quote}
\end{enumerate}

\begin{quote}
PRECEITOS POSITIVOS 117

lher a levar uma oferenda é um fluxo de sangue, mas um fluxo de sangue
no homem não o obriga ao sacrifício. A palavra "fluxo" significa apenas
"fluir" mas o que flui não é necessariamente a mesma coisa. Os Sábios
dizem explicita­mente: "Um homem transmite a impureza através do sêmem,
e a mulher atra­vés do sangue".

A lei do "zab" e da "zaba" não é igual a lei do leproso e da leprosa.
Isso está claramente provado pelo que está dito em Queretot: "Há quatro
pes­soas cujo perdão fica pendente: o homem e a mulher com fluxo, a
mulher de­pois do parto, e o leproso". Você verifica que o "zab" e a
"zaba" são contados como dois, porque o fluxo do homem é diferente do
fluxo da mulher, enquan­to qug o leproso e a leprosa não são contados
separadamente.

O versículo referente à oferenda da "zaba" é: "E se limpar-se de seu
fluxo... no oitavo dia, tomará para si duas rolas, ou dois pombinhos"
etc. (Leví­tico 15:28-29).

76 O SACRIFÍCIO DEPOIS DO PARTO

Por este preceito a mulher que tiver dado à luz a uma criança é
orde­nada a levar uma oferenda, a saber, um cordeiro de menos de um ano
de idade como Holocausto e uma pombinha ou uma rola como Sacrifício de
Pecado. Se ela for pobre, ela pode levar duas rolas ou dois pombinhos:
um como Holo­causto e o outro como Sacrifício de Pecado. Ela faz parte
daqueles cujo perdão só é alcançado depois de ter levado o sacrifício,
de acordo com o que Ele disse, enaltecido seja Ele: "E ao cumprirem-se
os dias de sua purificação, pelo filho ou pela filha, trará um cordeiro
de idade de um ano, por holocausto, e um pom­binho e uma rola por
sacrifício de pecado...". (Levítico 12:6)

77 O SACRIFÍCIO LEVADO POR UM LEPROSO

ar de sua lepra, é ordenado usto, um Sacrifício de Peca-óleo. Se ele for
pobre, ele e e duas rolas ou dois pombi­nhos, um como Holocausto e o
outro como Sa rifício de Pecado. Ele é a quarta pessoa cujo perdão fica
pendente até que tenha levado seus sacrifícios. Este pre­ceito está
expresso em Suas palavras, enaltecido seja Ele, "E no oitavo dia to­mará
dois cordeiros sem defeito e uma ovelha da idade de um ano, sem
defei­to". (Levítico 14:10).

Pode-se perguntar: "Por que você não conta como um único pre­ceito a
obrigação imposta a todos aqueles cujo perdão fica pendente, uma vez que
a pendência do perdão é comum a todos eles? Se você fizesse isso, esse
seria um dos meios de purificar-se e você poderia dizer: 'Preceito tal é
o que obriga certas pessoas impuras, a saber, um `zab', uma `zaba', uma
mulher de­pois do parto e um leproso a levar um sacrifício para que sua
purificação possa ser considerada completa'. Assim como você conta a
obrigação de ser purifica­do por um 'mikvá' como um único preceito, sem
se preocupar com quem é o impuro ou de que tipo é a sua impureza, você
poderia, da mesma forma, ter

100. Medida líquida que corresponde a 506 cm\textsuperscript{3} ou 0,24
kg.

118 MAIMÔNIDES

contado o sacrifício daqueles cujo perdão fica pendente como um único
pre­ceito, sem preocupar-se com o tipo de impureza".

O Eterno sabe, e é minha testemunha, que isso seria perfeitamente válido
se o sacrifício obrigatório àqueles cujo perdão fica pendente fosse o
mes­mo em todos os casos, e nunca fosse alterado, como no caso de ser
purificado pela água, que é um tipo de purificação obrigatória igual
para to.. as pessoas impuras. Mas devido à diversidade de seus
sacrifícios somos orça.. s, como você vêem, a contar cada sacrifício
separadamente, porque ue co pleta a purificação num caso não é o mesmo
que em outro. Este a o 101 é igual ao da água de aspersão, ao da água do
"mikvá" e ao das quatr c. as las quais o leproso é purificado. Esses são
três preceitos separados, embora eles todos se refiram à purificação de
pessoas impuras, como explicarei depois.

As normas detalhadas referentes às quatro pessoas cujo perdão fica
pendente e às suas oferendas estão explicadas global e detalhadamente no
pri­meiro e no segundo capítulos de Queretot, nos segundos capítulos de
Arakhin e Nezikin, no oitavo capítulo de Nazir, no final de Negaim, no
Tratado Kinim e em diversos trechos do Talmud, mas quase todas elas
estão nos trechos referidos.

78 O DÍZIMO DO GADO

Por este preceito somos ordenados a separar o dízimo dos nossos animais
puros nascidos todo ano, oferecer sua banha e seu sangue e comer o resto
em Jerusalém. Este preceito está expresso em Suas palavras, enaltecido
seja Ele, "E todo o dízimo do gado e do rebanho, todo o que passar
debaixo da vara marcadora, o décimo será santidade ao Eterno". (Levítico
27:32). Esse é o Dízimo do Gado.
\end{quote}

As normas deste preceito estão explica • no último capítulo de Bek-

\begin{quote}
horot. Lá também está explicado que este prec obrigatório também fora

da Terra de Israel e depois da destruição do . Essa é a lei estabelecida

na Torah. Mas para que ele não fosse comid esmo que sem defeito ---

uma vez que não temos Santuário --- os Sábi denaram que este preceito

fosse obrigatório apenas quando o Templo estivesse de pé, e que quando o
Tem­plo fosse reconstruído ele seria obrigatório tanto na Terra quanto
fora dela.
\end{quote}

SANTIFICAR O PRIMOGÊNITO

\begin{quote}
•

Por este preceito somos ordenados a santificar os primogênitos, ou seja,
a separá-los e deixá-los de lado para aquilo que deverá ser feito com
eles no momento exato. Ele está expresso em Suas palavras, enaltecido
seja Ele, "Con­sagra para Mim todo primogênito... no homem e no animal."
(Êxodo 13:2). A Torah explica que "animal" aqui significa apenas o gado,
os carneiros e os ju­mentos. Este preceito está repetido com relação aos
primogênitos dos animais puros em Suas palavras, enaltecido seja Ele,
"Todo o primogênito que nascer do teu gado" etc. (Deuteronômio 15:19).

Esta lei do primogênito de animais puros estipula que ele deve ser dado
aos "Cohanim", que deverão oferecer sua banha e seu sangue e comer
\end{quote}

\begin{enumerate}
\def\labelenumi{\arabic{enumi}.}
\setcounter{enumi}{100}
\item
  \begin{quote}
  O caso das quatro pessoas cujo perdão fica pendente.
  \end{quote}
\item
  \begin{quote}
  Fora de Jerusalém.
  \end{quote}
\end{enumerate}

\begin{quote}
PRECEITOS POSITIVOS 119

o que restar de sua carne. As regras detalhadas deste preceito estão
largamente explicadas no Tratado Bekhorot.

No final do Tratado Halá está explicado que este preceito é obriga­tório
apenas na Terra de Israel. O Sifrei diz: "Poder-se-ia pensar que é
obrigató­rio levar seus primogênitos nascidos fora de Israel até a Terra
de Israel, por -isso as Escrituras dizem: 'E o comerás diante do Eterno,
teu Deus,... o dízimo de teu grão... e os primogênitos do teu gado e do
teu rebanho' ou seja, os primo­gênitos devem ser levados do mesmo lugar
de onde vem o dízimo do grão, e não do exterior, pois o grão não é
levado do exterior".

Fica, portanto, claro que este preceito só é obrigatório na Terra de
Israel. Contudo, um primogênito nascido no exterior, embora não precise
ser levado como sacrifício, deve ser comido somente se for defeituoso,
quer esteja o Templo erguido ou como está agora. Este preceito não é
obrigatório para os Levitas.

80 RESGATAR O PRIMOGÊNITO

Por este preceito somos ordenados a resgatar nossos filhos primogê­nitos
e dar o dinheiro do resgate ao "Cohen". Ele está expresso em Suas
palavras, enaltecido seja El , „
\end{quote}

• é . ogênito de teus filhos dar-me-ás" (Êxodo 22:28).

\begin{quote}
A ma ei de "d o primogênito é explicada da seguinte forma: o primogênito
é r sgatado d. "Cohen", que o tem por direito e é tirado dele por cinco
"selai "103\textsubscript{.} O \textsubscript{preceito} r

. ceito está expresso em Suas palavras, enaltecido seja Ele, "Porém
esgatarás g s primogênitos do homem" (Números 18:15), sen­do este o
precei o • o Res:ate do filho primogênito.

Esta o rigação não se aplica às mulheres; é um dever do pai em rela­ção
ao seu filho, como está explicado em Kidushin. Todas as normas deste
pre­ceito estão explicadas em Bekhorot.

81 RESGATAR O PRIMOGÊNITO DE UM JUMENTO
\end{quote}

Por este preceito somos ordenados a resgatar o primogênito de um apenas
com um cordeiro --- se não o resgatarmos pelo seu valor e a dar o
cordeiro ao "Cohen". Este preceito está expresso em Suas

\begin{quote}
enaltecido seja Ele, "E todo que abrir a matriz da jumenta, remi-lo-ás
por eiro" (Êxodo 34:20 e 13:13).

As normas deste preceito estão explicadas no Tratado Bekhorot. Es­te
preceito também não se aplica aos Levitas.

82 QUEBRAR A CERVIZ DO PRIMOGÊNITO DE UM JUMENTO

Por este preceito somos ordenados a quebrar a cerviz do primogê­nito de
uma jumenta, caso não se deseje resgatá-lo. Este preceito está expresso
\end{quote}

\begin{enumerate}
\def\labelenumi{\arabic{enumi}.}
\setcounter{enumi}{102}
\item
  \begin{quote}
  "Selaim" é o plural de "sela", uma espécie de moeda antiga.
  \end{quote}
\item
  \begin{quote}
  A pessoa pode resgatar o jumento com dinheiro e dá-lo ao "Cohen".
  \end{quote}
\end{enumerate}

\begin{quote}
120 MAIMÔNIDES

em Suas palavras, enaltecido seja Ele, "E se não o remires,
quebrar-lhe-ás a cer­viz" (Êxodo 13:13 e 34:20). As regras detalhadas
deste preceito também estão explicadas no Tratado Bekhorot.

Poderia ser-me feita a seguinte pergunta: "Por que você conta resga­tar
e quebrar sua cerviz como dois preceitos, e não como um, considerando a
quebra da cerviz como uma das regras detalhadas do preceito, de acordo
com o Sétimo Fundamento?".

O Eterno sabe, e é minha stem nha de que que pergunta teria razão se não
fosse pel prova e que eles são rados encontrada as guintes pala
\textsuperscript{rastos,} `O dever de r dever de quebr a ce z, e o dev r
d asamento levi at dever do `Halit ' "". u seja, da mesma forma que a vi
filhos tem direi o ao cas ento levirato ou ao "Halitzá", send sue o
casamen­to levirato é urh p'i o, como mencionado, e o "Halitzá" um
outro, assim também o primogênito de uma jumenta pode ser resgatado ou
ter sua cerviz quebrada, sendo cada um deles um preceito diferente, como
foi exposto.

83 LEVAR OS SACRIFÍCIOS DEVIDOS DURANTE O PRIMEIRO FESTIVAL
\end{quote}

Por este preceito somos ordenados a executar todos os deveres a nós\\
impostos na chegada do primeiro dos três Festivais de forma que com a
passa-\\
gem de qualquer um dos três Festivais cada um de nós tenha levado todos
os\\
sacrifícios devidos. Este preceito está expresso em Suas palavras,
enaltecido seja\\
Ele, "E lá ireis. E levareis ali vossos holocaustos" (Deuteronômio
12:5-6). O sig-\\
nificado deste preceito é que quando fordes lá durante cada um dos três
Festi-\\
vais, tereis a obrigação de levar todos os sacrifícios que vos tenham
sido impostos.\\
A esse respeito diz o Sifrei: " `E lá ireis. E levareis ali'; por que
foi\\
dito isso? Para estabelecer a.obrigação de levar os sacrifícios no
início do pri-\\
meiro Festival". Também diz, no mesmo trecho: "Não se transgride `não
de-\\
morarás em pagá-lo' (Deuteronômio 23:22) até que não tenham terminado os

\begin{quote}
três Fes --- os Festivais do ano todo". Ou seja, se os três Festivais
tiverem

passa o se tiver levado seu sacrifício, ter-se-á violado um preceito

negat as se apenas um Festival tiver passado, ter-se-á violado apenas

um p ositivo.

a Guemará de Rosh Hashaná lemos: "Rabi Meir diz: Assim que um festiv
terminar, ele terá transgredido o preceito 'Não demorarás em pagá-lo' ".
O Talmud pergunta: "Por que motivo Rabi Meir diz isso?" E a resposta é:
"Por­que está escrito 'E lá ireis. E levareis ali', o que significa que
você deve levá-los quando for lá". Mas os Sábios dizem que esse
versículo constitui apenas um preceito positivo.
\end{quote}

Assim, ficou claro que as palavras "E levareis ali" são um preceito

\begin{quote}
positivo, significando que se deve cumprir t as suas obrigações para com

o Eterno e que isso deve ser feito em cada F sti , u• m elas quaisquer

tipos de sacrifícios, ou que sejam donativ s\textsuperscript{108}, alor
coisas consagra-
\end{quote}

\begin{enumerate}
\def\labelenumi{\arabic{enumi}.}
\setcounter{enumi}{104}
\item
  \begin{quote}
  Bekhorot 13:A.
  \end{quote}
\item
  \begin{quote}
  Ver Preceitos Positivos 216 e 217.
  \end{quote}
\item
  \begin{quote}
  Ver o preceito negativo 155.
  \end{quote}
\item
  \begin{quote}
  Ver o preceito positivo 114.
  \end{quote}
\end{enumerate}

\begin{quote}
PRECEITOS POSITIVOS 121

da \textsuperscript{109}oias dedicadas ao Santuár.
\textsuperscript{0110} respigadu\textsuperscript{t} as
\textsuperscript{I I I} fixes esqueci 0s112 e "pe O cumprimento de to
esses tipos igações no prim\\
Festiva e houver é um preceito positivo, como está explicado na Guemará
de Rosh Hashaná.
\end{quote}

LEVAR TODAS AS OFERTAS APENAS AO SANTUÁRIO

\begin{quote}
Por este preceito somos ordenados a oferecer todos os sacrifícios apenas
no Santuário. Ele está expresso em Suas palavras, enaltecido seja Ele,
"Ali oferecerás os teus holocaustos, e ali farás tud• • • e Eu te
ordeno" (Deu­teronômio 12:14). Procurando alguma citação e dei asse
clara a proibição de levar qualquer tipo de sacrifício a outro luga ,
114 ncontraram Suas pala­vras, enaltecido seja Ele, "Guarda-te de
oferecer te s holocaustos em todo o lugar que vires'' (Deuteronômio
12:13).

O Sifrei diz: "Eu sei disso apenas com relação ao Holocausto. De que
forma fico sabendo que é assim para com todos os outros sacrifícios?
Pelas pa­lavras das Escrituras 'E ali farás tudo o que eu te ordeno'.
Aqui novamente eu poderia dizer que apenas no caso do Holocausto há
ambos um preceito positi-

vo e um preceito negativo; co ue o mesmo se aplica a todos os outros

sacrifícios? Pelas palavras das 'Ali farás tudo' ". Eu voltarei a este
as-

sunto mais tarde, nas proibiçõ

O fato de dizer que no o de um Holocausto estão envolvidos um preceito
positivo e um negativo significa que aquele que o levar a outro lugar
estará violando ambos: um preceito positivo e um preceito negativo,
sendo o preceito negativo "Guarda-te de ofereceres teus holocaustos" e o
positivo "Ali

oferecerás", porque os terá levado "ali". Os outros sacrifícios pode-

riam abranger apen ito positivo "E ali farás tudo o que eu te ordeno",

mas está explicado a também no caso deles se viola um preceito nega-

tivo, ai\textsuperscript{-} ► do ositiv á explicado no final do Tratado
Zebahim que todos

os sac fícios levados outro lugar envolvem um preceito positivo e um
negafi \textsubscript{0}115\textsubscript{,} = a penalidade de extinção.
Assim, por tudo o que expliquei, fica claro q e as alavras "Ali farás
tudo o que Eu te ordeno" são sem dúvida algu­ma um preceito positivo.

8 5 LEVAR AO SANTUÁRIO, DESDE FORA DA TERRA DE ISRAEL,

TODOS OS SACRIFÍCIOS DEVIDOS

Por este preceito somos ordenados a levar ao Santuário tudo o que
tenhamos a obrigação de oferecer --- quer seja um Sacrifício de Pecado,
um
\end{quote}

\begin{enumerate}
\def\labelenumi{\arabic{enumi}.}
\setcounter{enumi}{108}
\item
  \begin{quote}
  Ver o preceito positivo 145.
  \end{quote}
\item
  \begin{quote}
  Ver o preceito positivo 121.
  \end{quote}
\item
  \begin{quote}
  Ver o preceito positivo 121.
  \end{quote}
\item
  \begin{quote}
  Ver o preceito positivo 122.
  \end{quote}
\item
  \begin{quote}
  Ver o preceito positivo 120.
  \end{quote}
\item
  \begin{quote}
  Os Sábios
  \end{quote}
\item
  \begin{quote}
  Ver os preceitos negativos 89 e 90.
  \end{quote}
\item
  \begin{quote}
  No Sifrei.
  \end{quote}
\end{enumerate}

\begin{quote}
122 MAIMÔNIDES

Holocausto, um Sacrifício de Delito ou um Sacrifício de Paz --- ainda
que o mo­tivo responsável pelo sacrifício esteja fora da Terra de
Israel, ou seja, embora a obrigação tenha sido ocasionada fora da Terra
de Israel, somos ordenados a levar os sacrifícios ao Santuário, e temos
o dever de fazê-lo, não importa a que distância. Este preceito está
expresso em Suas palavras, enaltecido seja Ele, "De certo, tuas coisas
sagradas, e tuas ofertá de votos, tomarás e levarás ao lugar que o
Eterno escolher" (Deuteronômio 12:26), sobre as quais o Sifrei diz: "
'Tuas coisas sagradas' se refere apenas aos sacrifícios fora da Terra de
Israel. `Tomarás e levarás' nos ensina que devemos nos preocupar com o
transporte dos sacrifícios até o Santuário". Está explicado ali que isso
se aplica apenas nos casos de Sacrifício de Pecado, de Sacrifício de
Delito, de Holocausto e de Sacri­fício de Paz que a pessoa tem a
obrigação de oferecer.
\end{quote}

86 REDIMIR OFERENDAS DEFEITUOSAS

\begin{quote}
Por este preceito somos ordenados a redimir qualquer oferenda que tiver
se tornado defeituosa, liberando-a assim para uso normal e
permitindo-nos abatê-la e comê-la. Este preceito está expresso em Suas
palavras, enaltecido seja Ele, "Todavia, com todo o desejo de tua alma,
poderás degolar e comer carne, em todas as tuas cidades" (Deuteronômio
12:15), a respeito das quais o Sifrei diz: " 'Todavia, com todo o desejo
de tua alma, poderás degolar e co­mer carne, em todas as tuas cidades':
isto se refere apenas a oferendas defeituo­sas que tiverem sido
redimidas".

As regras detalhadas do preceito de redenção de oferendas estão
ex­plicadas no Tratado Bekhorot e Temurá, e em vários trechos de Hulin,
Arakhin e Meilá.

87 A SANTIDADE DE UMA OFERENDA SUBSTITUÍDA

Por este preceito somos ordenados a considerar como sagrado um animal
que for substituído por outro. Ele está expresso em Suas palavras,
enalte­cido seja Ele, "Tanto o que for trocado como aquele pelo qual
trocou, serão san­tidade" (Levítico 27:10). Está expressamente declarado
no início do Tratado Te­murá que Suas palavras, enaltecido seja Ele, "E
não trocará" (Ibid., 33) consti­tuem um preceito negativo justaposto a
um preceito positivo. Os Sábios pergun­tam: "Não é o caso da
substituição um exemplo de um preceito negativo justa­posto a um
preceito positivo?". No mesmo texto aparece um argumento suple­mentar
que sujeita aquele que fizer a substituição a ser açoitado, embora esse
seja u receito negativo justaposto a um preceito positivo: "Um preceito
posi­tivo ao ode se sobrepor a dois preceitos negativos." Ou seja, a
proibição de fawr a su stituição está expressa duas vezes, uma em "Não o
mudará" (Ibid., 170)\textsuperscript{117} outra em "E não o trocará"
(Ibid.), enquanto há apenas um preceito positiv que é "Tanto o que for
trocado como aquele pelo qual trocou, serão santidade" (Ibid., 27:10).
Dessa forma, fica ratificado o que desejávamos provar.

As normas deste preceito, ou seja, o que valida ou invalida a
substi­tuição, quais são suas regras, e como ela deve ser oferecida,
estão explicadas no Tratado Temurá.

PRECEITOS POSITIVOS 123

\textbf{88} OS "COHANIM" DEVEM COMER

OS RESÍDUOS DAS OBLAÇÕES

Por este preceito os "Cohanim" são ordenados a comer os resíduos das
Oblações. Ele está expresso em Suas palavras, enaltecido seja Ele, "E o
que ficar dela, comerão Aarão e seus filhos; comer-se-á sem fermento"
(Levítico 6:9). A Sifrá diz, a esse respeito: " 'Comer-se-á' é um
preceito positivo, da mesma forma que 'O\textsuperscript{-}irmão de seu
marido estará com ela e a tomará por mulher' (Deu­teronômio 25:5) também
é um preceito positivo. Ou seja, comer os resíduos das Oblações e
casar-se com a viúva de um irmão que morreu sem deixar filhos são dois
preceitos positivos e não meras opções.

As normas destes preceitos estão explicadas no Tratado Menahot. A Torah
determina que este preceito se aplique apenas aos homens, de acordo com
Suas palavras, enaltecido seja Ele, "Todo varão dos filhos de Aarão o
co­merá" (Levítico 6:11).

89 OS "COHANIM" DEVEM COMER A CARNE DOS SACRIFÍCIOS CONSAGRADOS

Por este preceito os "Cohanim" são ordenados a comer a carne dos
Sacrifícios Consagrados, a saber, o Sacrifício de Pecado e o Sacrifício
de Delito, que estão entre os Sacrifícios Mais Sagrados. Este preceito
está expresso em Suas palavras, enaltecido seja Ele, "E comerão das
coisas com que for feita a expia­ção" (Êxodo 29:33).

A Sifrá diz: "De que maneira saber que o fato de que comam os
Sa­crifícios Consagrados concede o perdão para toda Israel? Pelas
palavras da To­rah 'E o Eterno vô-lo deu para levardes a iniquidade da
congregação, a fim de perdoar por eles, diante do Eterno!' (Levítico
10:17). De que forma? O 'Cohen' o come, e Israel recebe o perdão."

Urna das condições deste preceito é que só se deve comê-los duran­te um
dia e uma noite até a meia-noite. Depois disso, fica proibido comer
de­les; ele é uma obrigação apenas durante o espaço de tempo
estabelecido.

Fica clarb que também este preceito se aplica apenas aos elementos
masculinos das famílias dos "Cohanim", e não às mulheres, pois as
mulheres não podem comer dos Sacrifícios Mais Sagrados a que se refere
este preceito. Os outros sacrifícios --- a saber, os Sacrifícios Menos
Sagrados--- podem ser co­midos no espaço de dois dias e uma noite,
exceto a Ação de Graças e o carnei­ro dos Nazirim os quais, emb jam
Sacrifícios Menos Sagrados, devem ser comidos em um dia e uma oite at a
meia-noite. Outrossim, as mulheres po­dem comer desses Sacrifí os Menu s
Sagrados.

O fato de com ostra t. bém é parte deste preceito, bem como o de comer a
Oferta de Eleào. Co tudo, comer os Sacrifícios Menos Sagrados e os
Sacrifícios de Elevação úúo é orno comer a carne dos Sacrifícios de
Peca­do e de Delito, pois o ato de comer a carne dos Sacrifícios de
Pecado e de Deli­to completa o perdão de quem ofereceu os sacrifícios,
como explicamos, e o ato de comer é ordenado explicitamente no caso
deles mas não no caso dos

118. O fato dos "Cohanim" comerem os Sacrifícios Menos Sagrados.

124 MAIMÔNIDES

Sacrifícios Menos Sagrados e dos Sacrifícios de Elevação.
Consequentemente, isso constitui apenas uma parte do preceito que se
aplica aos outros sacrifícios, e ao comê-los ele executa um preceito. O
Sifrei diz: " 'O serviço de vosso sa­cerdócio dei-o como dádiva a vós'
(Números 18:7) faz com que o ato de comer as Coisas Consagradas dentro
da Terra de Israel seja como o serviço no Santuá­rio: assim como ele
tinha que lavar suas mãos antes de iniciar o serviço no San­tuário, ele
também tinha que lavar as mãos antes de comer as Coisas Sagradas fora de
Jerusalém."
\end{quote}

As normas deste preceito estão explicadas em diversos trechos de

\begin{quote}
Zebahim.

90 QUEIMAR SACRIFÍCIOS CONSAGRADOS QUE SE TORNARAM IMPUROS

Por este preceito somos ordenados a queimar os Sacrifícios Consa­grados
que se tenham tornado impuros. Ele está expresso em Suas palavras,
enaltecido seja Ele, "E a carne Sagrada do sacrifício de pazes que tocar
em tudo o que for impuro, não será comida, no fogo será queimada"
(Levítico 7:19).

A Guemará de Shabat discute a questão de por que é proibido usar óleo de
Oferta de Elevação que tenha se tornado impuro para iluminação, em d'
\end{quote}

de Fest'

\begin{quote}
de Festival baseia-se num preceito positivo e num negativo, e um pr
positi o pode se sobrepor a ambos um preceito negativo e um positi

O significado deste trecho é que é proibido trabalhar num dl Festival e
aquele que o fizer estará infringindo um preceito positivo, tendo
anu­lado o preceito de que o dia de Festival "será para vós descanso
solene" (Leví­tico 23:24). Ele também estará infringindo um preceito
negativo, porque estará fazendo algo que lhe foi proibido, de acordo com
as palavras "Nenhuma obra será feita neles" (Êxodo 12:16), ou seja, nos
dias de Festival. Como a queima de Sacrifícios Consagrados que se
tornaram impuros é apenas um preceito po­sitivo, não é permitido
queimá-los num dia de Festival, de acordo com o prin­cípio já mencionado
de que um preceito positivo não pode se sobrepor a am­bos um preceito
negativo e um positivo. Também está dito: "Da mesma forma que é
obrigatório queimar Sacrifícios Consagrados que se tornaram impuros,
assim também é obrigatório queimar o óleo da Oferta de Elevação que
tiver se tornado impuro".

As normas deste preceito estão explicadas em Pessahim e no final de
Temurá.

91 QUEIMAR AS SOBRAS DOS

SACRIFÍCIOS CONSAGRADOS

Por este preceito somos ordenados a queimar as sobr. Sie está expresso
em Suas palavras, enaltecido seja Ele, "E o que ficar da ca do sacri-
\end{quote}

\begin{enumerate}
\def\labelenumi{\arabic{enumi}.}
\setcounter{enumi}{118}
\item
  \begin{quote}
  I.e., o preceito de queimar todos os sacrifícios que tenham se tornado
  impuros.
  \end{quote}
\item
  \begin{quote}
  Ver os preceitos negativos 323 a 329.
  \end{quote}
\item
  \begin{quote}
  Dos Sacrifícios Consagrados.
  \end{quote}
\end{enumerate}

\begin{quote}
PRECEITOS POSITIVOS 125

fício, no terceiro dia, no fogo será queimado" (Levítico 7:17). Com
relação a Suas palavras, enaltecido seja Ele, relativas ao cordeiro de
"Pessah", "E não fareis so­brar nada dele até a manhã; e a sobra dele,
pela manhã a queimareis no fogo" (Êxodo 12:10), a Mekhiltá diz: "O
objetivo das Escrituras é estipular um preceito positivo para o preceito
negativo". Em vários trechos de Pessahim e Macot, além de outros, está
explicitamente declarado que o preceito negativo relativo às so­bras
está justaposto a um preceito positivo, e dessa forma não há punição por
açoitamento. O preceito ositivo a que nos referimos está expresso nas
palavras já mencionadas, ou s 'E o que ficar da carne... no fogo
queimado".

A lei de R e a das Sobras são semelhan s, co o explicarei

nos preceitos negati , uma vez que a palavra "no usada como\\
recusa.

As normas deste preceito estão explicadas em Pessahim e no final de
Temurá.

92 O NAZIR DEVE DEIXAR CRESCER SEUS CABELOS

Por este preceito o Nazir é ordenado a deixar crescer seus cabelos. Ele
está expresso em Suas palavras, enaltecido seja Ele, "Sagrado será ele;
dei­xará crescer o cabelo de sua cabeça" (Números 6:5). A esse respeito
diz a Mek­hiltá: " 'Sagrado será o seu cabelo': ele deixará crescer seu
cabelo em sinal de santidade. 'Deixará crescer... cabeça' é um p
e\textsubscript{a}i

e positivo. De que forma fi-

camos sabendo que também há um preceito gtt o? Pelas palavras das E
turas: 'Lâmina não passará pela sua cabeç "124

Também está dito na Mekhiltá: 0 p eito positivo se aplica
o\textsuperscript{125} que esfregar terra ou aplicar produtos químicos";
ou seja, se ele aplicar pr

tos químicos em sua cabeça que provoquem a queda dos cabelos ele não
estará infringindo o preceito negativo, porque ele não terá colocado em
sua cabeça nada semelhante a uma lâmina, mas estará violando o preceito
positivo "Dei­xará crescer o cabelo de sua cabeça" ao não permitir que o
cabelo cresça. E, de acordo com nossos Fundamentos, um preceito negativo
derivado de um pre­ceito positivo é um preceito positivo.

As normas deste preceito estão explicadas no lugar apropriado, no
Tratado Nazir.
\end{quote}

A OBRIGAÇÃO DO NAZIR DE CONSUMAR SEU VOTO

\begin{quote}
Por este preceito o Nazir é ordenado a raspar sua cabeça e a levar seus
sacrifícios quando os dias de sua consagração estiverem terminados. A
Si­frá diz: "Há três so m que a raspagem da cabeça constitui um preceito
posi­tivo: para o N \textsubscript{t\}}r,\textsubscript{i}rr o leproso e
para os Levitas". Contudo, a raspagem da cabeça dos Lev as
\textsuperscript{6} s era obrigatória enquanto eles estavam no deserto e
dei-
\end{quote}

\begin{enumerate}
\def\labelenumi{\arabic{enumi}.}
\setcounter{enumi}{121}
\item
  \begin{quote}
  Ver os preceitos negativos 131 e 132.
  \end{quote}
\item
  \begin{quote}
  Sobras.
  \end{quote}
\item
  \begin{quote}
  Ver o preceito negativo 209.
  \end{quote}
\item
  \begin{quote}
  Se aplica ao Nazir.
  \end{quote}
\item
  \begin{quote}
  Números 8:7.
  \end{quote}
\end{enumerate}

\begin{quote}
126 MAIMÔNIDES

xou de sê-lo depois disso, enquanto que a raspagem da cabeça do leproso
e do Nazir será sempre obrigatória.

Está claro que o Nazir tem duas ocasiões para raspar sua cabeça: ele
deve fazê-lo se tiver se tornado impuro, como prescrito nas palavras "E
quan­do alguém morrer subitamente, junto a ele" etc. (Números 6:9), e em
estado de pureza, como prescrito nas palavras "No dia em que se
completarem os dias de seu Nazirado" (Ibid. 6:13).

Contudo, essas duas obrigações de raspar a cabeça não podem ser contadas
como dois preceitos diferentes, uma vez que a raspagem a ser feita por
causa de uma impureza é um dos detalhes do regulamento relativo ao
pre­ceito do voto do Nazir --- o preceito positivo de deixar crescer seu
cabelo em santidade, como já explicamos. Depois disto as Escrituras
especificam que se o Nazir se tornar impuro ele deve raspar sua cabeça e
trazer um sacrifício e de­pois deixar seu cabelo crescer novamente em
santidade durante o período de Nazirado a que ele se propôs; assim como
no caso do leproso, as duas obriga­ções de raspar a cabeça constituem um
único preceito, como explicarei no lu­gar apropriado.

Também explicarei mais tarde por que, no caso do Nazir, nós conta­mos
raspar a cabeça e trazer seus sacrifícios como um preceito, enquanto que
no caso do leproso nós os contamos como dois.

As normas deste preceito, ou seja, a raspagem da cabeça do Nazir, estão
explicadas no texto a esse respeito do Tratado Nazir.

94 CUMPRIR TODOS OS COMPROMISSOS ORAIS

Por este preceito somos ordenados a cumprir toda obrigação que
assumirmos através de palavras --- todo juramento, prA e a, oferenda ou
equi-

valente. Ele está expresso em Suas palavras, enaltec Ele, "O que sair de

teus lábios guardarás" (Deuteronômio 23:24). Emb nham analisado mi-

nuciosamente este versículo e tenham explicado cad avra separadamente,

o sentido global de tudo o que eles dizem é que este é um preceito
positivo que obriga o homem a cumprir todo co omisso que ele tenha
assumido, e que deixar de fazê-lo é transgredir um dito negativo. Isto
será explicado quando eu tratar dos preceitos negativo

O Sifrei diz: " 'O que sair dos t lábios' é um preceito positivo".

Você sabe, porém, que não se pode derivar nenhum preceito a partir das
sim­ples palavras "O que sair dos teus lábios", portanto o significado
delas deve ser o que mencionei como significado literal das Escrituras,
ou seja, que um homem tem a obrigação de cumprir o que seus lábios
tenham proferido. Este preceito também se encontra em outro trecho em
Suas palavras, enaltecido se­ja Ele, "Como tudo que saiu de sua boca,
assim fará" (Números 30:3).

As normas deste preceito estão explicadas em vários trechos de Sha­buot
e de Nedarim, no final de Menahot, e também no Tratado Kinim; ou seja,
foi deixado claro que devemos ser corretos no cumprimento de qualquer
obri­gação a que tenhamos nos comprometido, e de que forma podemos ser
absol­vidos se tivermos dúvidas quanto ao que nos tenhamos imposto.
\end{quote}

\begin{enumerate}
\def\labelenumi{\arabic{enumi}.}
\setcounter{enumi}{126}
\item
  \begin{quote}
  Os Sábios.
  \end{quote}
\item
  \begin{quote}
  Ver o preceito negativo 157.
  \end{quote}
\end{enumerate}

\begin{quote}
PRECEITOS POSITIVOS 127

95 A REVOGAÇÃO DE PROMESSAS

Por este preceito somos ordenados a aplicar as leis relativas à
revo­gação de promessas. Este preceito, contudo, não significa que
sejamos obriga­dos a revogar as promessas em todos os casos. Você deve
entender que a mes­ma coisa se aplica a todas as leis que eu enumerar:
um preceito não nos obriga necessariamente a fazer uma determinada
coisa, mas estipula como devemos tratar do assunto em o, de acordo com
as leis.

Evidentem tá explícito nas Escrituras que um marido e um pai

podem fazer a revogaç estão indicados os procedimentos para isso. A Tra-

dição também autoriz u ábio a liberar alguém de uma promessa ou de um\\
juramento, baseada em Suas palavras "Não profanar a palavra" (Nú ros\\
30:2), ou seja, "Fle não pode quebrar sua promessa aso. outros po•fa ê-

lo por ele". De uma maneira geral, não há base es ara isso
ele,\textsuperscript{131} i '-

zem: "As regras sobre liberar alguém de uma pro es airam n • na m

nada de concreto para apoiá-las", a não ser a verdadeira Tradiç .

As normas deste preceito estão explicadas no Tratado
\textbf{\textsuperscript{4}} trata es­pecificamente deste assunto, que é
o Tratado Nedarim.

96 TORNAR-SE IMPURO COM CARCAÇAS DE ANIMAIS

Por este preceito somos ordenados quanto à impureza da carcaça de um
animal, e ele inclui a impureza de uma carcaça e todas as normas a esse
respeito.

À guisa de prefácio mencionarei algo que você deve ter em mente com
relação a tudo que será dito a seguir, quanto aos vários tipos de
impureza. O fato de que contemos cada um dos vários tipos de impureza
como um pre­ceito positivo nã•significa que seja uma obrigação ou que
seja proibido tornar-se impuro de u ou de outra dessas maneiras, como se
isso fosse um preceito negativo. O e qu remos dizer é que quando a Torah
diz que quem tocar este ou aquele t o\textsuperscript{133} t. na-se
impuro, ou que este ou aquele objeto torna impuro de uma determina• a
forma aquele que o tocar, isto constitui um preceito posi­tivo; ou seja,
est ei que somos obrigados a seguir é um preceito que estabele­ce que
quem tocar determinadas coisas sob determinadas condições se tornará
impuro mas, caso seja sob condições diferentes, ele não se tornará
impuro. Na realidade, o fato de se tornar impuro é opcional: se um homem
quiser, ele se tornará impuro, e se não o quiser, não o fará.

A Sifrá diz: " 'No seu cadáver não tocareis' (Levítico 11:8):
podería­mos pensar que se uma pessoa tocar uma carcaça ele estará
sujeito ao açoita­mento; por isso as Escrituras dizem: 'E por estes vos
tornareis impuros' (Ibid., 24). Poderíamos pensar que se uma pessoa vê
uma carcaça ele deve tocá-la e assim tornar-se impuro. Por isso as
Escrituras dizem: `No seu cadáver não toca­reis'. Como fazer para
conciliar esses dois versículos? Devemos concluir que tocar uma carcaça
é opcional".
\end{quote}

\begin{enumerate}
\def\labelenumi{\arabic{enumi}.}
\setcounter{enumi}{128}
\item
  \begin{quote}
  Podem revogar as promessas da esposa e da filha, respectivamente.
  \end{quote}
\item
  \begin{quote}
  Na Torah escrita.
  \end{quote}
\item
  \begin{quote}
  Os Sábios.
  \end{quote}
\item
  \begin{quote}
  A Tradição oral.
  \end{quote}
\item
  \begin{quote}
  Tipo de impureza.
  \end{quote}
\end{enumerate}

128 MAIMÔNIDE

Portanto o conteúdo deste preceito é que todo aquele que toc
r\textsuperscript{134} se torna impuro, e ao se tornar impuro ele está
sujeito a todas as obrigaç "es impostas às pessoas impuras: ele deve
sair do Acampamento da Presença Divi­na; não deve comer nem tocar Coisas
Sagradas, e assim por diante. A essência deste preceito é que uma pessoa
se torna impura ao tocar um determinado ob­jeto ou, em determinadas
circunstâncias, ao estar próximo dele.

Tenha isso em mente com relação aos vários tipos de impureza.

\begin{quote}
97 TORNAR-SE IMPURO ATRAVÉS DAS CARCAÇAS DE DETERMINADOS ANIM •
RASTEJANTES

Por este ceiti somos ordenados quanto à impureza de oito tipos
\end{quote}

de criaturas rastejan s\textsuperscript{135}. ste preceito abrange a lei
da impureza de animais rastejantes e as regras" alhadas referentes a
ela.

\begin{quote}
98 TORNAR-SE IMPURO ATRAVÉS DE COMIDA E BEBIDA
\end{quote}

Por este preceito somos ordenados a lidar com a im re a da comi­da e da
bebida de acordo com as leis prescritas. Este preceitp incl i todas as
leis relativas à impureza de comida e bebida de todos os
tiOos\textsuperscript{136}

99 A MULHER MENSTRUADA

Por este preceito somos ordenados quanto à impureza a mulher menstruada.
Este preceito inclui todas as regras a esse
respeito\textsuperscript{137}.

100 DEPOIS DO NASCIMENTO

\begin{quote}
DE UMA CRIANÇA )
\end{quote}

Por este preceito somos ordenados quanto à impureza de uma/mu­lher
depois de um parto. Este preceito inclui todas as regras a esse
respeito\textsuperscript{138}.

101 O LEPROSO

Por este preceito somos ordenados quanto à impureza de um lepro­so. Este
preceito inclui toda a regulamentação referente à lepra: quais são casos
impuros e quais os puros, quais necessitam segregação e quais não, qúais
requerem, além da segregação, que a cabeça seja raspada, e quais não,
bem co­mo outros detalhes relativos as suas regras e à natureza de sua
impureza\textsuperscript{139}.

\begin{enumerate}
\def\labelenumi{\arabic{enumi}.}
\setcounter{enumi}{133}
\item
  \begin{quote}
  Que tocar qualquer tipo de impureza.
  \end{quote}
\item
  \begin{quote}
  Esse preceito está expresso em Levítico 11:29-30.
  \end{quote}
\item
  \begin{quote}
  Essas leis estão contidas em Levítico 11:34.
  \end{quote}
\item
  \begin{quote}
  Esse preceito se encontra em Levítico 15:19-24.
  \end{quote}
\item
  \begin{quote}
  Ele está expresso em Levítico 12:2-5.
  \end{quote}
\item
  \begin{quote}
  Todas as leis referentes a esse preceito estão contidas em Levítico
  13:1-59
  \end{quote}
\end{enumerate}

\begin{quote}
PRECEITOS POSITIVOS 129

\textbf{102} AS ROUPAS CONTAMINADAS PELA LEPRA

Por este preceito somos ordenados quanto à impureza de uma rou­pa
contaminada pela lepra. Ele inclui toda a regulamentação a esse
respeito: co­mo as roupas se tornam impuras e como elas causam impureza,
quais

ser separadas, rasgadas, queimadas, lavadas ou purificadas, bem como udo
o mais que está prescrito nas Escrituras e o que a Tradição diz a esse
respeito'"

\textbf{103} A CASA DE UM LEPROSO

Por este preceito somos ordenados quanto à impureza da casa de um
leproso. Este preceito inclui todas as regras a esse respeito: que elas
devem ser isola • as, • ais devem ter suas paredes parcialmente
demolidas, ou ser com­pletam me d\textsuperscript{-} olidas como elas se
tornam impuras e como elas causam impur zási41.

\textbf{104} O "ZAB"

Por este preceito somos ordenados qua à i ureza de um "zab".

Este preceito inclui toda a regulamentação relativa aos si tomas de um
"zab" e à maneira como ele torna outras pessoas
impu'ras\textsuperscript{142}.

\textbf{105} O SÊMEN

Por este preceito somos ordenados quanto impu za do sêmen. Este preceito
inclui toda a regulamentação a esse respeito "\textsuperscript{3}

\textbf{106} A "ZABA"

Por este preceito somos ordenados quanto à impu a de ma "za­ba". Ele
inclui as regras referentes aos sintomas que tornam ma mul er uma "zaba"
e à maneira pela qual ela torna outras pessoas impuras
\textsuperscript{144} a s ter-se tornado uma "zaba".

107 A IMPUREZA DE UM C \textsuperscript{-}VER

Por este preceito somos ordenados quanfo à imp reza de um cadá­ver. Este
preceito inclui todas as regras a esse respeito\textsuperscript{145}.

\textbf{108} A LEI DA ÁGUA DE ASPERSÃO
\end{quote}

Por este preceito somos ordenados quanto às regras referentes à água

\begin{enumerate}
\def\labelenumi{\arabic{enumi}.}
\setcounter{enumi}{139}
\item
  \begin{quote}
  As regras que regem este preceito estão em Levítico 13:47-59.
  \end{quote}
\item
  \begin{quote}
  Os detalhes deste preceito se encontram em Levítico 14:36-48.
  \end{quote}
\item
  \begin{quote}
  As regras que regem este preceito estão em Levítico 14:35-54 e
  15:1-12.
  \end{quote}
\item
  \begin{quote}
  As regras que regem este preceito estão em Levítico 15:16-18.
  \end{quote}
\item
  \begin{quote}
  As regras que regem este preceito estão em Levítico 15:25-30.
  \end{quote}
\item
  \begin{quote}
  As regras que regem este preceito estão em Números 19:14-16.
  \end{quote}
\end{enumerate}

\begin{quote}
130 MAIMÔNIDES

de aspersa \textsuperscript{146} a\textsubscript{/}qual, sob certas
circunstâncias, purifica e, sob outras, impuri­fica, como Sermoexplicado
no estudo detalhado deste preceito.

Você deve saber que os treze tipos de impurezas que foram enume­rados
--- a saber, a impureza de uma carcaça, de animais rastejantes, de
comi­das, de uma mulher menstruada, de uma mulher depois do parto, de um
lepro­so, de roupas contaminadas pela lepra, de uma casa contaminada
pela lepra, de um "zab", de uma "zaba", do sêmen, de um cadáver, e da
água de aspersão --- e a rificação para cada um deles que estão todos
explicados na Torah;

ve s er mbém quê 'Irá \textsubscript{\textbackslash{}}vários textos,
regulamentações e condições r a vos a c a um desses preceit expostos nos
trechos Vayehi Bayom Has , 1.147\\
Tazria\textsuperscript{148}, Zot Tih'ye\textsuperscript{149} e no trecho
Veyikemu Eilecha para Adoma\textsuperscript{15}°

Les­ses quatrojtextos está t o o que se refere aos vários tipos de
impure a. Mas todas as regras e regulamentos relativos a esses tipos de
impurezas estão conti­dos na Ordem de Teharot (Pureza). Três dos
Tratados dessa Ordem, a saber, Teharot, Makhshirin e Okatzin, contém as
impurezas referentes à comida, e tra­tam exclusivamente desse assunto, e
qualquer menção sobre outras impurezas nesses Tratados é meramente
acidental. Da mesma forma, o Nidá contém os regulamentos referentes à
mulher mestruada, à "zaba" e à mulher depois do parto, sendo que também
se encontram' algumas das regras referentes a esta úl­tima no Tratado
Queretot. O Tratado Negaim contém todos os regulamentos sobre a lepra
nos homens, roupas e casas; o Tratado Zabim contém os regula­mentos
referentes ao "zab", à "zaba" e ao sêmen; o Tratado Ohalot trata dos
cadáveres e o Tratado Pará traz as regras sobre a água de aspersão como
um agente de purificação ou de impurificação. Por outro lado, não há
Tratados es­pecíficos sobre a impureza de carcaças e de animais
rastejantes; os regulamen­tos sobre esses assuntos estão espalhados em
vários trechos da Ordem, sobre­tudo nos Tratados Quelim e Teharot. E
várias questões referentes a esses as­suntos são tratadas no Tratado
Eduyot. Nós próprios compusemos um comen­tário sobre a Ordem de Teharot
e não é necessário consultar qualquer outro livro além desse sobre
qualquer assunto relativo à pureza e impureza.

109 MERGULHAR NO BANHO RITUAL

Por este preceito somos ordenados a mergulhar nas águas de um ba­nho
ritual, e assim limpar-nos de todo tipo de impurezas que nos tentiam
afeta­do. Este preceito está expresso em Suas palavras, enaltecido seja
Ele, "E o ho­mem ... banhará em água toda a sua carne" (Levítico 15:16),
a respeito das quais a Tradição diz: "Água suficiente para cobrir todo o
seu corpo: isto é, a medida de um banho ritual" --- exceto no caso de
água corrente, para a qual não há medida prescrita, como está explicado
nos detalhes deste preceito.

Uma das cláusulas deste preceito estipula que apenas o "zab" deve
purificar-se em água corrente, de acordo com o que diz a Torah: "E
banhará sua carne nas águas vivas" (Ibid., 13).

Ao tratar da imersão como um preceito positivo, não queremos di­zer que
todas as pessoas impuras devam tomar banho de imersão, que todas
\end{quote}

\begin{enumerate}
\def\labelenumi{\arabic{enumi}.}
\setcounter{enumi}{145}
\item
  \begin{quote}
  Números XIX, 9-21.
  \end{quote}
\item
  \begin{quote}
  Levítico IX, 1-11; 47.
  \end{quote}
\item
  \begin{quote}
  Levítico XII, 1-13; 59.
  \end{quote}
\item
  \begin{quote}
  Levítico XIV, 1-15; 33.
  \end{quote}
\item
  \begin{quote}
  Números XIX, 1-22.
  \end{quote}
\end{enumerate}

\begin{quote}
PRECEITOS POSITIVOS 131

as pessoas que usam uma vestimenta devem colocar "Tsitsit" nela, ou que
to­dos os que tenham uma casa devem fazer um suporte; o que quero dizer
é ape­nas que pela lei da imersão aquele que desejar livrar-se de suas
impurezas não pode atingir seu objetivo a não ser pela imersão em água,
depois do que ele se tornará puro. •

A Sifrá diz: " 'E lavar-se-á nas águas' (Levítico 14:8); • • deríamos
jul­gar isto como sendo um decreto do Rei. Por isso as Escritur.: diz :
'E depois entrará no acampamento' (Ibid.), por causa de sua impura
a\textsuperscript{151}. \textsuperscript{i} sto leva ao princípio que eu
havia explicado, ou seja, que a lei de imersa• se plica apenas àquele
que quiser se purificar, e esta lei é um preceito. Não há, contudo,
ne­nhuma obrigação de banhar-se, e aquele que quiser permanecer impuro e
esti­ver disposto a privar-se de entrar no acampamento da Presença
Divina por al­gum tempo tem a liberdade de fazê-lo.

O Livro da Verdade deixa claro que todo aquele que estiver impuro e
fizer uma imersão se purificará, mas sua purificação não estará completa
até o pôr-do-sol; e de acordo com a interpretação tradicional, ele
deverá estar nu e todo o seu corpo deverá ficar em contato com a água.
Como diz o Talmud, " 'Toda sua carne', ou seja, não haver nada entre sua
carne e a água."

Fica assim claro que\textsuperscript{-} ceito da imersão inclui a
regulamenta­ção do banho ritual, a interposi o "Tebul yom". Ele está
explicado nos Tratados Mikvaot e Tebul Yom.

110 PURIFICAR-SE DA LEPRA

Por este preceito somos ordenados a que a purificação da lepra seja
realizada de acordo com as normas estabelecidas nas Escrituras, ou seja,
com pau de cedro, hissopo, carmezim, dois pássaros vivos e águas vivas,
e que eles sejam empregados como está determinado. O homem será
purificado por esse procedimento, como explicam as Escrituras.

Foi explicado, portanto, que de acordo com nossa Torah há três mé­todos
diferentes pelos quais pode ser realizada a purificação: um geral, e
dois aplicáveis cada um apenas a um tipo específico de impureza. O
método geral é pela água, a qual é indispensável para purificar-se; o
segundo é pela água de aspersão e aplica-se especificamente à impureza
adquirida através de um mor­to; o terceiro, que consiste de pau de
cedro, hissopo, carmezim, dois pássaros vivos e águas vivas, aplica-se
especificamente no caso da lepra.

Todas as normas deste preceito, ou seja, a primeira purificação da
lepra, estão explicadas no Tratado Negaim.

111 O LEPROSO DEVE RASPAR A CABEÇA

Por este preceito o leproso é ordenado a raspar a cabeça, e isto
cons­titui o segundo estágio de sua purificação, como está explicado no
final de Ne­gaim. Este preceito está expresso em Suas palavras,
enaltecido seja Ele, "E ao sétimo dia raspará todo o seu pelo" (Levítico
14:9). Já nos referimos anterior­mente às palavras dos Sábios: "Três
raspam suas cabeças, e para cada um deles
\end{quote}

\begin{enumerate}
\def\labelenumi{\arabic{enumi}.}
\setcounter{enumi}{150}
\item
  \begin{quote}
  Entrará no acampamento do qual havia sido excluído por causa de sua
  impureza.
  \end{quote}
\item
  \begin{quote}
  A interposição de alguma coisa entre o corpo e a água.
  \end{quote}
\end{enumerate}

\begin{quote}
\textbf{132 MAIMÔNIDES}

a raspagem constitui um preceito positivo: o Nazir, o leproso e o
Levita". As normas deste preceito estão explicadas no final de Negaim.

Aqui explicarei por que no caso do leproso contamos a raspagem e a
oferta dos sacrifícios determinados cada um como um preceito individual,
enquanto que no caso do Nazir contamos os dois juntos como um preceito.
É que no caso do leproso não há conexão entre o ato de raspar a cabeça e
o de levar seus sacrifícios, e o objetivo atingido pela raspagem é
diferente do al­cançado pelo oferecimento dos sacrifícios, porque sua
purificação depende da raspagem de sua cabeça. No sexto capítulo de
Nazir está dito: "Como um Nazir difere de um leproso? Sua purificação
depende dos dias, enquanto que a purifi­cação do leproso depende da
raspagem de seus cabelos". Tendo raspado a ca­beça, e tendo completado
sua segunda raspagem, o leproso se purifica e cessa de transmitir o tipo
de impureza que é transmitida pelos animais rastejantes, como está
explicado no final de Negaim; e seu perdão fica em suspenso até que ele
leve seus sacrifícios, da mesma forma que os ciutros cujo perdão fica em
suspenso, como está explicado ali.

Assim, a raspagem da cabeça torna o leproso puro a ponto de que ele
cesse de transmitir o mesmo tipo de impureza que é transmitida por um
animal rastejante, quer ele tenha ou não oferecido seus sacrifícios; e a
oferta dos sacrifícios complementa seu perdão, tal como nos outros casos
em que o perdão fica em suspenso --- a saber, o "zab", a "zaba", e a
mulher depois do parto. Nós já nos referimos às palavras dos Sábios: "Há
quatro pessoas cujo per­dão fica em suspenso" etc.

No caso de um Nazir, como está explicado ali, o perdão não fica
in­completo, e todo o procedimento estabelecido --- raspar a cabeça e
oferecer o sacrifício --- lhe permite beber vinho novamente. E um não é
suficiente sem o outro: a raspagem está ligada ao sacrifício, e o
sacrifício à raspagem, e os dois em conjunto atingem um objetivo único,
que é permitir-lhe as coisas que lhe eram proibidas nos seus dias de
Nazirado. No sexto capítulo de Nazir está dito: "Se ele raspou seu
cabelo depois de um dos sacrifícios e este foi considerado inválido, a
raspagem de seu cabelo também se torna inválida e seus sacrifícios não
contam. = 'm, ficou explicado que a raspagem é uma das condições da
oferenda, nda é uma das condições da raspagem.
\end{quote}

foi explicado na Tosseftá que um Nazir que tenha comple-

\begin{quote}
tado seus á proibido de raspar a cabeça, beber vinho e tornar-se im-

puro pel s s até que ele tenha completado todo o procedimento de ras-

pagem e estado de pureza, o qual, como está explicado no sexto capítulo
de Nazir, im s lica em que a raspagem seja feita diante da porta da
Tenda de Assina­ção, que le jogue se belos debaixo da caldeira e que ele
leve as oferendas, como está explica nas scrituras.

Vocês ontrar o que na maioria dos lugares os Sábios denominam

a oferta dos sacri ios\textsuperscript{154} d "raspagem", e eles dizem
explicitamente na Mish­ná: " 'Eu serei u Nazir e e comprometo a raspar a
cabeça etc': a intenção é de levar as ofer das do azir e oferecê-las por
si próprio. Assim foi explica­do que a raspagem um te o alternativo para
seu oferecimento de sacrifícios, e a razão disso é que parte deste
último, como explicamos, e é unicamente com a combinação destes dois que
o Nazirado se completa e que o Nazir pode beber vinho. Mas a raspagem
por impureza é apenas uma das leis do preceito, como explicamos
anteriormente.
\end{quote}

\begin{enumerate}
\def\labelenumi{\arabic{enumi}.}
\setcounter{enumi}{152}
\item
  \begin{quote}
  Seus dias de Nazirado.
  \end{quote}
\item
  \begin{quote}
  Dos Nazirim.
  \end{quote}
\end{enumerate}

\begin{quote}
PRECEITOS POSITIVOS 133

\textbf{112} O LEPROSO DEVE SER RECONHECÍVEL

Por este preceito somos ordenados á que o leproso seja tornado
re­conhecível, de forma que as pessoas possam se' manter afastadas dele.
Este pre­ceito está expresso em Suas palavras, enaltecido seja Ele, "E
do leproso que tem a chaga, suas vestes serão rasgadas e seu cabelo não
será cortado, e com seu bigode se cobrirá; e impuro! impuro! clamará"
(Levítico 13:45).

A prova de que este é um dos preceitos positivos é encontrada na Sifrá:
"Como foi dito, a respeito do 'Cóhen Gadol' que 'Seu cabelo não deixará
crescer e suas vestes não romperá' (Ibid., 21:10) poder-se-ia pensar que
isto se aplica mesmo se ele tiver contraído uma praga, e que o preceito
'Suas vestes serão rasgadas e seu cabelo não será cortado' se aplica a
todos menos ao 'Co­hen Gadol'. Por isso a Torah diz: 'Em quem estiver a
praga' etc.; mesmo se ele for um 'Cohen Gadol', suas roupas devem ser
rasgadas, o seu cabelo deve ficar solto, e ele deve deixar seu cabelo
crescer."

Está cl e um "Cohen Gadol" está proibido, através de um pre-

ceito negativo, • e rasga suas roupas ou de deixar crescer seu cabelo; e
é um princípio acei • entre n o s que toda vez que encontrarmos um
preceito positi­vo e um neg vo\textsuperscript{155}, se e udermos
cumprir os dois será ótimo; caso contrário o preceito p • sítivo se
brepõe ao negativo. Conseqüentemente, uma vez que está estabele id e e u
- m "Cohen Gadol" leproso deve deixar crescer seu ca­belo e rasgar suas
roupas, conclui-se que este é um preceito positivo.

Por tradição, outras pessoas impuras também são obrigadas a se tor­nar
reconhecíveis a fim de que os outros possam se manter afastados delas. A
Sifrá diz: "Como sabemos que isto também se aplica a alguém que tenha se
tor­nado impuro através de um cadáver ou de uma relação com uma mulher
mens­truada, e a todos os que possam transmitir impureza aos outros? As
Escrituras dizem: 'E impuro! impuro! clamará' (Ibid., 13:45)". Isto
significa que toda pes­soa impura deve anunciar sua impureza, ou seja,
deve se tornar reconhecível como pessoa impura, que transmite impureza
com seu contato, de forma a que as pessoas possam evitá-la.

Ficou claro que a obrigação de um leproso de se tornar reconhecí­vel não
se aplica às mulheres. "Um homem", diz o Talmud, "anda com o cabe­lo
solto e com as roupas rasgadas, mas uma mulher não anda com o cabelo
sol­to e com as roupas rasgadas", embora ela deva cobrir seu lábio
superior, e pro­clamar sua impureza como as outras pessoas impuras.

113 AS CINZAS DA VACA VERMELHA

Por este preceito somos ordenados a preparar a vaca vermelha de maneira
tal que suas cinzas possam ser utilizadas para fazer o que deve ser
feito a fim de remover a impureza de um cadáver, como disse o
Enaltecido: "E a congregação dos filhos de Israel a guardará por água
purificadora" (Números 19:9).

As normas deste preceito estão explicadas no Tratado que trata
es­pecificamente deste assunto, que é o Tratado Pará.

134 MAIMÔNIDES

114 A AVALIAÇÃO DE UMA PESSOA

Por este preceito somos ordenados quanto à lei da avaliação do ho­mem, a
qual estabelece que quando alguém diz "Eu prometo meu próprio va­lor" ou
"Eu prometo o valor de uma determinada pessoa", se for um homem, ele
deve pagar uma determinada importância, e se for uma mulher, ele deve
pagar urna determinada importância de acordo com a idade, como está
estabe­lecido nas Escrituras, e de acordo com as posses de que fez a
promessa. Este preceito está expresso em Suas palavras, enaltecido seja
Ele, "Quando alguém fizer um voto, para oferecer preço de almas...
segundo as posses daquele que fez o voto." (Levítico 27:2-8)
\end{quote}

As normas deste preceito estão explicadas no Tratado Arakhin.

\begin{quote}
115 A AVALIAÇÃO DE ANIMAIS

Por este preceito somos ordenados quanto à lei da avaliação dos ani­mais
impuros. Ele está expresso em Suas palavras "Fará apresentar o animal
dian­te do 'Cohen'. E o avaliará o 'Cohen' (Levítico 27:11-12).

As normas deste preceito estão explicadas em trechos dos Tratados Temurá
e Meilá.

116 A AVALIAÇÃO DE CASAS

Por este preceito somos ordenados quanto à avaliação das casas. Ele está
expresso em Suas palavras: "E quando alguém consagrar a sua casa para
ser santidade ao Eterno, o 'Cohen' a avaliará". (Levítico 27:14)

As normas deste preceito estão explicadas no Tratado Arakhin.

117 A AVALIAÇÃO DOS CAMPOS

Por este preceito somos ordenados quanto à avaliação dos campos. Ele
está expresso em Suas palavras, enaltecido seja Ele, "E se alguém
consagrar uma parte do campo de sua possessão ao Eterno" etc. (Levítico
27:16) ... "E se o campo de sua compra, que não é campo de sua herança,
consagrar ao Eterno" etc. (Ibid., 22). No caso de "campo de sua
possessão", "A tua avaliação será con­forme a semente necessária para
semeá-lo" (Ibid., 16); e no caso de "campo de sua compra", "O 'Cohen'
calculará para ele, a conta de sua avaliação" (Ibid., 23).

As normas deste preceito também estão explicadas na íntegra no 'fra­tado
Arakhin.

Que ninguém pense que esses quatro tipos de avaliação têm tanto em comum
que deveriam ser contados como um único preceito. Eles são qua­tro
preceitos separados, cada um com suas regras específicas, embora a
palavra "avaliação" seja comum a todos eles. Conseqüentemente não se
deve contar todos os tipos de avaliação como um único preceito, da mesma
maneira que não se contam todos os tipos de oferenda como um único
preceito. Isso se tor­na claro ao se estudar cuidadosamente o assunto.

118 A RESTITUIÇÃO POR SACRILÉGIO

PRECEITOS POSITIVOS 135

comeu acrescido de um quin 0\textsuperscript{156}. E te preceito está
expresso em Suas palavras, enaltecido seja Ele, "E pagará o da coisa
sagrada pela qual pecou, acrescen­tando a quinta parte" (Levítico 5:16)
e "E o homem que comer santidade por er­ro, acrescentará a quinta parte
do valor desta" (Ibid. 22:14). As normas deste pre­ceito estão
explicadas no Tratado Meilá, e também no Tratado Terumot.

119 A COLHEITA DO QUARTO ANO

Por este preceito somos ordenados a considerar todo o fruto da co­lheita
do quarto ano como sagrado. Este preceito está expresso em Suas
pala­vras "E no quarto ano, será todo o seu fruto santidade de louvores
ao Eterno" (Levítico 19:24). A lei prevê que o proprietário deve levá-lo
a Jerusalém e comê-lo lá, como no caso do Segundo Dízimo. Os "Cohanim"
não recebem nenhuma parte dele.

O Sifrei diz: " 'E as s tida es de todo o ho ele serão' (Núme-

ros : as Escrituras sep as Coisa g s deram aos 'Co-

ha ceto a Ação de raç Sacrifíci sacrifício de `Pes-

sa Dízimo do Gad i \textsuperscript{159}, o do Dízi ita do quarto ano

da as, os quais decl o • ertencerem aos prietarios."

As normas deste preceito estão explicadas integralmente no último
capítulo do Tratado Maasser Sheni.

120 "PEÁ" PARA OS POBRES

Por este preceito somos ordenados a deixar "peábiói
de\textsuperscript{-} ereais, fru­tas e similares. Ele está expresso em
Suas palavras, enaltecidos a Ele, "Para
\end{quote}

\begin{itemize}
\item
  \begin{quote}
  pobre e o imigrante os deixarás" (Levítico 19:10).
  \end{quote}
\end{itemize}

\begin{quote}
No Tratado Macot está explicado que "peá" envolve um preceito negativo
justaposto a um preceito positivo. O preceito negativo está exp
\end{quote}

em Suas palavras "Não acabarás de segar o canto de teu campo" (Ibid. 19.

\begin{itemize}
\item
  \begin{quote}
  o positivo está em Suas palavras "Para o pobre e o peregrino os deix
  As normas deste preceito estão explicadas no Tratado Peá. A Torah
  limita sua aplicação à Terra de Israel.
  \end{quote}
\end{itemize}

\begin{quote}
121 A RESPIGA PARA OS POBRES

Por este preceito somos ordenados a deixar respigas. Ele está expresso
em Stias palavras "Nem colhereis a respiga de vossa ceifa; para o pobres
e para
\end{quote}

\begin{itemize}
\item
  \begin{quote}
  imigrante os deixareis" (Levítico 23:22).
  \end{quote}
\end{itemize}

\begin{quote}
Este preceito também envolve um preceito negativo justaposto um preceito
positivo, como explicado no Tratado Macot com relação à "pe ".163
\end{quote}

\begin{enumerate}
\def\labelenumi{\arabic{enumi}.}
\setcounter{enumi}{155}
\item
  \begin{quote}
  Um quinto do valor daquilo que deve restituir.
  \end{quote}
\item
  \begin{quote}
  Ver preceito positivo 66.
  \end{quote}
\item
  \begin{quote}
  Ver preceito positivo 55 e 56.
  \end{quote}
\item
  \begin{quote}
  Ver preceito positivo 78.
  \end{quote}
\item
  \begin{quote}
  Ver preceito positivo 128.
  \end{quote}
\item
  \begin{quote}
  Sobras abandonadas nos campos para os pobres.
  \end{quote}
\item
  \begin{quote}
  Ver o preceito negativo 210.
  \end{quote}
\item
  \begin{quote}
  Ver o preceito negativo 211.
  \end{quote}
\end{enumerate}

\begin{quote}
136 MAIMÔNIDES

As normas deste preceito estão explicadas no Tratado Peá. A Torah limita
sua aplicação à Terra de Israel. •

122 DEIXAR • A GAVELA ESQUECIDA

PARA OS POBRES

Por este preceito somos ordenados a deixar a gavela esquecida. Este
preceito está expresso em Suas palavras, enaltecido seja Ele, "Quando
segares a messe no teu campo, e esqueceres uma gavela, no campo, não
voltarás a toma-la; para o imigrante, o órfão, e a viúva será"
(Deuteronômio 24:19). As palavras "Para o imigrante, o órfão, e a viúva
será" constituem o preceito positivo de

deixa vela esquecida, assim como as palavras "Os deixarás" (Levítico
19:10)

cons o preceito positivo, no caso das respigas e da "peá", como foi ex-
\end{quote}

licação deste preceito também está limitada pela Torah à Terra de

Isr

As normas deste preceito também estão explicadas no Tratado Peá.

\begin{quote}
123 DEIXAR AS SOBRAS DOS CACHOS DE UVA PARA OS POBRES

Por este preceito somos ordenados a deixar para os pobres os ca­chos de
uva que sobrarem no momento da colheita, chamados "olelot". Tam­bém a
respeito deles as Escrituras dizem: "Para o pobre e o imigrante os
deixa­rás". (Levítico 19:10).

As normas deste preceito estão explicadas no Tratado Peá, e a Torah
limita sua aplicação à Terra de Israel.

124 DEIXAR AS UVAS CAÍDAS PARA OS POBRES
\end{quote}

Por este preceito cair e ficar separado do c palavras "E o bago de t
deixarás." (Levítico 19:

As normas det e a Torah limita sua apli

mos ordenados a deixar para os pobres o que urante a vindima. Ele está
expresso em Suas não recolherás; para o pobre e o imigrante os

deste preceito estão explicadas no Tratado Peá, à Terra de Israel.

125 LEVAR AS PRIMÍCIAS\\
AO SANTUÁRIO

\begin{quote}
Por este preceito somos ordenados a separar as primícias e levá-las ao
Santuário. Ele está expresso em Suas palavras, enaltecido seja Ele, "O
princí­pio das primícias de tua terra trarás à casa do Eterno, teu
Deus." (Êxodo 23:19).
\end{quote}

\begin{enumerate}
\def\labelenumi{\arabic{enumi}.}
\setcounter{enumi}{163}
\item
  \begin{quote}
  Neste caso, bem como no caso dos preceitos 120 e 121, as leis
  rabínicas determinam que estas obrigações também devem ser cumpridas
  fora de Israel (Mishné Torah, Leis das Dádivas aos Po­bres, Cap. 1,
  Lei 14).
  \end{quote}
\item
  \begin{quote}
  Ver preceito negativo 213.
  \end{quote}
\end{enumerate}

\begin{quote}
PRECEITOS POSITIVOS 137

Está claro que este preceito só é obrigatório durante existe cia do
Santuário, e que se aplica apenas às "Sete Categorias" de prod
•s\textsuperscript{166} q crescem na Ter­ra de Israel, na Síria e na
Transjordânia.

As normas deste preceito estão explicadas atado Bicurim, on-
\end{quote}

de foi deixado claro que elas, ou seja, as primícias, são propriedades
do "Cohen".

\begin{quote}
126 A GRANDE OFERTA DE ELEVAÇÃO

Por este preceito somos ordenados a separar a grande Oferta de
Ele­vação. Ele está expresso em Suas palavras, enaltecido seja Ele, "As
primícias de teu grão... darás a ele" (Deuteronômio, 18:4). De acordo
com a Torah, ele é obrigatório apenas na Terra de Israel, e suas normas
estão explicadas no Tra­tado Terumot.

127 O PRIMEIRO DÍZIMO

Por este preceito somos ordenados a separar o dízimo do produto da
terra. Ele está expresso em Suas palavras, enaltecido seja Ele, "Porque
os dízimos dos filhos de Israel, que separarem ao Eterno em oferta"
(Números 18:24). As Escrituras explicam que este dízimo pertence aos
Levitas.

As normas deste preceito estão explicadas no Tratado Maasserot. Ele é
chamado de o Primeiro Dízimo, e a Torah só o torna obrigatório dentro da
Terra de Israel.

128 O SEGUNDO DÍZIMO

Por este preceito somos ordenados a separar o segundo dízimo. Ele está
expresso em Suas palavras, enaltecido seja Ele, "Certamente separarás o
dízimo de todo o produto das tuas sementes, que o campo produzir de ano
a ano." (Deuteronômio 14:22) sobre as quais diz o Sifrei: " 'Ano a ano':
isto nos ensina que os dízimos não devem ser deixados de um ano para o
outro. Contudo, as palavras das Escrituras se referem apenas ao segundo
dízimo; co­mo saber que elas devem ser aplicadas aos outros dízimos
também? Porque a Torah diz: 'Certamente separarás o dízimo etc' ".

A Torah expõe claramente que este dízimo deve ser levado a Jerusa­lém,
para lá ser comido pelo seu proprietário. Nós já nos referimos ao que os
Sábios dizem a este respeito.

As Escrituras dão as leis deste preceito em detalhes, dizendo que
quan­do é impossível levá-lo a Jerusalém devido à distância, ele deve
resgatá-lo e le­var seu valor em dinheiro ao Santuário e ali gastá-lo
exclusivamente com comi­da. Isso está estipulado em Suas palavras,
enaltecido seja Ele, "E se o caminho

te for comprido, de sorte que não o possas levar, por longe de ti," etc.

(Deuteronômio 14:24). Outra norma estabelecida pela que se ele o res-

gatar para si próprio, ele deverá acrescentar um qui
to\textsuperscript{167}. .to está determi-\\
nado em Suas palavras, enaltecido seja Ele, "E se quiser a remir o seu
dízi-
\end{quote}

\begin{enumerate}
\def\labelenumi{\arabic{enumi}.}
\setcounter{enumi}{165}
\item
  \begin{quote}
  Os sete tipos de produtos pelos quais a terra de Israel era famosa:
  trigo, cevada, uvas, figos, romãs, óleo de oliva, e mel de tâmara.
  \end{quote}
\item
  \begin{quote}
  Um quinto de seu valor.
  \end{quote}
\end{enumerate}

\begin{quote}
PRECEITOS POSITIVOS 137

Está claro que este preceito só é obrigatório durante existê cia do
Santuário, e que se aplica apenas às "Sete Categorias" de prod s" q
crescem na Ter-ra de Israel, na Síria e na Transjordânia.

As normas deste preceito estão explicadas atado Bicurim, on-
\end{quote}

de foi deixado claro que elas, ou seja, as primícias, são propriedades
do "Cohen".

\begin{quote}
126 A GRANDE OFERTA DE ELEVAÇÃO

Por este preceito somos ordenados a separar a grande Oferta de
Ele­vação. Ele está expresso em Suas palavras, enaltecido seja Ele, "As
primícias de teu grão... darás a ele" (Deuteronômio, 18:4). De acordo
com a Torah, ele é obrigatório apenas na Terra de Israel, e suas normas
estão explicadas no Tra­tado Terumot.

127 O PRIMEIRO DÍZIMO

Por este preceito somos ordenados a separar o dízimo do produto da
terra. Ele está expresso em Suas palavras, enaltecido seja Ele, "Porque
os dízimos dos filhos de Israel, que separarem ao Eterno em oferta"
(Números 18:24). As Escrituras explicam que este dízimo pertence aos
Levitas.

As normas deste preceito estão explicadas no Tratado Maasserot. Ele é
chamado de o Primeiro Dízimo, e a Torah só o torna obrigatório dentro da
Terra de Israel.

128 O SEGUNDO DÍZIMO

Por este preceito somos ordenados a separar o segundo dízimo. Ele está
expresso em Suas palavras, enaltecido seja Ele, "Certamente separarás o
dízimo de todo o produto das tuas sementes, que o campo produzir de ano
a ano." (Deuteronômio 14:22) sobre as quais diz o Sifrei: " 'Ano a ano':
isto nos ensina que os dízimos não devem ser deixados de um ano para o
outro. Contudo, as palavras das Escrituras se referem apenas ao segundo
dízimo; co­mo saber que elas devem ser aplicadas aos outros dízimos
também? Porque a Torah diz: 'Certamente separarás o dízimo etc' ".

A Torah expõe claramente que este dízimo deve ser levado a Jerusa­lém,
para lá ser comido pelo seu proprietário. Nós já nos referimos ao que os
Sábios dizem a este respeito.

As Escrituras dão as leis deste preceito em detalhes, dizendo que
quan­do é impossível levá-lo a Jerusalém devido à distância, ele deve
resgatá-lo e le­var seu valor em dinheiro ao Santuário e ali gastá-lo
exclusivamente com comi­da. Isso está estipulado em Suas palavras,
enaltecido seja Ele, "E se o caminho te for comprido, de sorte que não o
possas levar, por longe de ti," etc. (Deuteronômio 14:24). Outra norma
estabelecida pela orah ' que se ele o res­gatar para si próprio, ele
deverá acrescentar um qui to\textsuperscript{167} to está determi­nado
em Suas palavras, enaltecido seja Ele, "E se quiser pess a remir o seu
dízi-

{,L}
\end{quote}

\begin{enumerate}
\def\labelenumi{\arabic{enumi}.}
\setcounter{enumi}{165}
\item
  \begin{quote}
  Os sete tipos de produtos pelos quais a terra de Israel era famosa:
  trigo, cevada, uvas, figos, romãs, óleo de oliva, e mel de tâmara.
  \end{quote}
\item
  \begin{quote}
  Um quinto de seu valor.
  \end{quote}
\end{enumerate}

\begin{quote}
138 MAIMÔNIDES

mo, acrescentar-lhe-á a quinta de seu preço" (Levítico 27:31). Todas as

regras detalhadas deste prece o esto explicadas no Tratado Maasser
Sheni.

Da mesma forma e é obr atório, pela Torah, apenas com relação aos
produtos da Terra de Isr. 1, e dev ser comido apenas durante a
existência do Santuário. O Sifrei diz: co para o ato de comer os
primogênitos ao segundo dízimo: assim como os pr .gênitos podem ser
comidos apenas du­rante a existência do Santuário, Q se: ndo dízimo
também só pode ser comido durante a existência do Santuário".

129 O DÍZIMO DOS LEVITAS PARA OS "COHANIM" OU A OFERTA DE ELEVAÇÃO

Por este preceito os Levitas são ordenados a separar um dízimo do dízimo
que eles receberam de Israel, e a dá-lo aos "Cohanim". Este preceito
está expresso em Suas palavras, enaltecido seja Ele, "E aos Levitas
falarás e lhes dirás: Quando tomardes dos filhos de Israel o dízimo que
deles vos dei: por vossa herança, dele separareis, uma oferta para o
Eterno, o dízimo do dízimo" (Números 18:26). As Escrituras explicam que
este dízimo, que é chamado de Oferta de Elevação do Dízimo, deve ser
dado ao "Cohen' : "E dareis deles a oferta separada do Eterno, para
Aarão, o 'Cohen' " (Ibid., 28).

As Escrituras explicam que este dízimo deve ser retirado da melhor e
mais selecionada parte: "De todo o melhor delas, a parte consagrada que
lhe é consagrada" (Ibid., 29). As Escrituras ressaltam então que eles
cometem uma transgressão se não fizerem a seleção dentre a melhor parte:
"Não levareis so­bre vós por isso, pecado, quando separardes o melhor
dele" (Ibid., 32). Este é um preceito negativo de exclusão, como se ele
tivesse dito: "Não haverá pe­cado quando separardes do melhor". Daí
deduzimos que se separarmos do pior haverá pecado, e portanto este é um
preceito negativo derivado de um precei­to positivo, e portanto não está
contado entre os preceitos negativos: quer di­zer, o preceito de fazer a
seleção entre o que há de melhor implica que a esco­lha não deve ser
feita dentre a pior parte. O Sifrei diz: "De que modo você con­clui que
se fizerem a seleção de outra parte que não a melhor eles cometem um
pecado? Porque as Escrituras dizem: 'E não levareis sobre vós por isso,
pe­cado, quando separardes o melhor dele' ".

As normas deste preceito estão explicadas nos Tratados Terumot e
Maasserot, e em diversos lugares de Demai.

130 O DÍZIMO DO H M POBRE

Por este preceito somos orden dos a se rar o dízimo para os po­bres no
terceiro ano de cada ciclo de Sha i at\textsuperscript{169}, e • vamente
no terceiro ano depois de cada terceiro ano, ou seja, no se o ano de
cada ciclo de Shabat. Este preceito está expresso em Suas palavras, e
altecido seja Ele, "Ao fim de três anos tirarás todos os dízimos de teu
produt b " etc. (Deuteronômio 14:28).

A Torah também torna este preceito obrigatório não somente na Terra de
Israel. Suas normas estão explicadas nos Tratados Peá e Maasserot, e vá-
\end{quote}

\begin{enumerate}
\def\labelenumi{\arabic{enumi}.}
\setcounter{enumi}{167}
\item
  \begin{quote}
  A Torah.
  \end{quote}
\item
  \begin{quote}
  Deve-se deixar que a terra repouse a cada sete anos, formando assim o
  ciclo de Shabat.
  \end{quote}
\end{enumerate}

\begin{quote}
PRECEITOS POSITIVOS 139

rios assuntos ligados a ele estão espalhados em vários trechos dos
outros Trata­dos de Zeraim, e nos Tratados Makhshirin e .Yadayim.

131 A DECLARAÇÃO DO DÍZIMO

Por este preceito somos ordenados a declarar diante d'Ele, enalteci­do
seja Ele, que separamos os dízimos obrigatórios e as Ofertas de
Elevação, e a verbalmente declarar que estamos liberados de nossas
obrigações, assim co­mo nos exoneramos delas de fato. Este preceito,
chamado "a Declaração do Dízimo", está expresso em Suas palavras,
enaltecido seja Ele, "E dirás diante do Eterno, teu Deus: Tirei o que é
consagrado, de minha casa, e também o dei ao Levita, e ao imigrante, e
ao órfão, e à viúva" (Deuteronômio 26:13).

As normas deste preceito, a maneira de proceder a separação e o seu
significado estão explicadas no último capítulo de Maasser Sheni.

132 A NARRAÇÃO AO LEVAR AS PRIMÍCIAS

Por este preceito somos ordenados, ao levar as primícias, a narrar as
bondades que Deus, enaltecido seja Ele, nos concedeu, como Ele nos
liber­tou dos sofrimentos de nosso Patriarca Jacob, e da escravidão e
opressão dos Egípcios; a agradecer-Lhe por isso; e a implorar-Lhe para
perpetuar Suas bên­çãos. Este preceito está expresso em Suas palavras,
enaltecido seja Ele, "Falarás em voz alta e dirás diante do Eterno, teu
Deus: Arameu errante era meu pai" (Deuteronômio 26:5) e todo o resto
desta passagem. Este preceito é chamado de o Relato das Primícias. Suas
normas estão explicadas no Tratado Bicurim e no sétimo capítulo de Sotá.
Ele não é obrigatório para as mulheres.

133 A OFERTA DE MASSA

Por este preceito somos ordenados a separar uma Torta (Halá) de ca­da
massa, e dá-la ao "Cohen". Este preceito está expresso em Suas palavras,
enaltecido seja Ele, "Em primeiro lugar, separareis de vossas massas,
uma torta; como oferta da eira" (Números 15:20).

As normas deste preceito estão explicadas nos Tratados Halá e Orlá, e a
Torah nos obriga a ele somente na Terra de Israel.

134 RENUNCIAR À PRODUÇÃO

DE SUA PROPRIEDADE NO ANO DE SHABA

Por este preceito • mos ord nados a renunciar a toda a produção de
nossas terras no ano de S batl", e permitir que qualquer pessoa recolha
tudo o que cresce em nossos pos. E te preceito está expresso em Suas
pala­vras, enaltecido seja Ele, "E n - • • e deixa-la-ás de cultivar,
deixa-la-ás de adu­bar e limpar" (Êxodo 23:11).

170. O sétimo ano no ciclo de sete anos

140 MAIMÔNIDES

A Mekhiltá diz: "Se o vinhedo e o olival estão incluídos, porque eles
estão mencionados especificamente? Para servirem como analogia: assim
co­mo a obrigação é um preceito positivo específico, cuja violação
acarreta tam­bém a transgressão de um preceito negativo, a violação de
qualquer preceito positivo acarreta a infração de um preceito negativo".

O significado disto é o que explicarei a seguir. O preceito "No séti­mo
deixa-la-ás de cultivar, deixa-la-ás de adubar e limpar" abrange toda a
pro­dução da terra durante o Ano de Shabat: figos, uvas, azeitonas,
pêssegos, ro­mãs, trigo, cevada, e os outros. Conseqüentemente, é um
preceito positivo tra­tar todos esses tipos de produtos dentro dos
termos da lei de Shabat. Mas as Escrituras depois especificam: "Assim
farás com tua vinha e teu olival", embo­ra estes já estejam incluídos em
"todos" os produtos da terra. O preceito não se aplic ecificamente ao
vinhed • o olival, mas nos é ordenado por causa

da adv ia das Escrituras qua ecolher o produto do vinhedo, a qual

está n ras "As uvas sepa9 ti, da tua vinha, não colherás" (Levíti-

co 25 ssim como no cas hedo, em que é um preceito positivo

decla dono e não faz m preceito negativo, assim está clara-

mente sso que, no caso •e udo que crescer no sétimo ano, é um pre-

ceito positivo declará-lo sem dono eixar de fazê-lo é um preceito
negativo.

Portanto, o caso do olival é o mesmo que o do vinhedo no que se refere a
um preceito positivo e um negativo, e o caso do olival é o mesmo que o
dos outros produtos. Portanto ficou claro por tudo o que foi exposto que
a renúncia a toda a produção do sétimo ano é um preceito positivo.

As normas deste preceito estão explicadas no Tratado Shebiit e a To­rah
só o torna obrigatório para a produção da Terra de Israel.

13 5 O POUSIO DA TERRA DURANTE O ANO DE SHABAT

Por este preceito somos ordenados a deixar de cultivar a terra du­rante
o sétimo ano. Ele está expresso em Suas palavras, enaltecido seja Ele,
"Mes­mo no tempo de arar e ceifar descansarás" (Êxodo 34:21). Ele está
repetido vá­rias vezes, como em Suas palavras "E no sétimo ano, sábado
de descanso para a terra" (Levítico 25:4). Nós já mencionamos que, de
acordo com as palavras dos Sábios, benditas sejam suas memórias, a
palavra "descanso" (Shabaton) de­termina um preceito positivo. E Ele,
louvado seja, também diz "Descansará a terra, descanso em nome do
Eterno" (Ibid., 2).

As normas deste preceito estão explicadas no Tratado Shebiit e a To­rah
não o torna obrigatório a não ser na Terra de Israel.

136 SANTIFICAR O ANO DO JUBILEU (50 ANOS)

Por este preceito somos ordenados a santificar o quinquagésimo ano, ou
seja, a deixar de cultivar a terra durante esse ano, assim como no Ano
de Shabat. Este preceito está expresso em Suas palavras, enaltecido seja
Ele, "E santificareis o ano quinquagésimo" (Levítico 25:10), sobre as
quais se comen­ta: "A Lei do Sétimo Ano é a mesma da do Jubileu"; ou
séja, as Escrituras as colo-
\end{quote}

\begin{enumerate}
\def\labelenumi{\arabic{enumi}.}
\setcounter{enumi}{170}
\item
  \begin{quote}
  Ver o preceito negativo 223.
  \end{quote}
\item
  \begin{quote}
  Não fazê-lo é transgredir um preceito negativo.
  \end{quote}
\end{enumerate}

\begin{quote}
PRECEITOS POSITIVOS 141

cam em pé de igualdade com relação a relação ao negativo, como vou
explic

As leis do Jubileu e do Ano at são iguais no que se refere a

deixar de cultivar a terra e a declarar sem o tudo o que nela crescer.
Essas

duas leis estão compreendidas em Suas palavras "E santificareis o ano
quinqua­gésimo". As Escrituras explicam que sua santidade consiste em
não ter dono seus frutos e produtos, estando essa obrigação expressa em
Suas palavras "Por­que jubileu é ele; santidade será para vós; do campo
comereis seu produto" (Ibid. 25:12).

O Jubileu é observado apenas na Terra de Israel, e com a condição de que
cada tribo permaneça em seu próprio lugar, ou seja, que cada uma
per­maneça no seu território da Terra de Israel, e que não se misturem
umas com as outras.

os ordenados a fazer soar o "Shofar" no décimo proclamar por toda nossa
terra a liberdade dos escravos e a liberação, s ento, de todo escravo
hebreu nesse déèimo dia de "Tishri". Este preceito está expresso em Suas
palavras, enaltecido seja Ele, "E farás soar a voz do Shofar' aos dez
dias do sétimo mês; no dia das expiações fa­reis soar o Shofar' em toda
a vossa terra" (Levítico 25:9) e em Suas palavras "E proclamareis
liberdade em toda a terra, para todos os seus moradores" (Ibid., 10).

Tem sido explicado que o Jubileu é como Rosh Hashaná no que se refere a
fazer soar o "Shofar" e às Bençãos. As normas relativas a fazer soar o
"Shofar" em Rosh Hashaná estão explicadas no Tratado Rosh Hashaná.

É sabido que a intenção de fazer soar o "Shofar" no ano do Jubileu é
divulgar amplamente a libertação, e é parte da proclamação, como aparece
em Suas palavras "E proclamareis a liberdade em toda a terra". Sua
finalidade é diferente em Rosh Hashaná, quando se faz soar o "Shofar"
como "lembrança diante do Eterno", enquanto que no Jubileu é pela
liberação dos escravos, co­mo explicamos.
\end{quote}

\textbf{138 A DEVOLUÇÃO DA ERRA\\
NO ANO DO JU E}

\begin{quote}
Por este preceito somos ord ados a qu todas as terras compradas retornem
aos seus proprietários nesse no\textsuperscript{175}, e e elas sejam
entregues pe­. los compradores sem compensação m rietária. E e preceito
está expresso em Suas palavras, enaltecido seja Ele: "Em toda- rra de
vossa possessão, reden­ção concedereis à terra" (Levítico 2 5 : 2 4),
tendo ficado claro pelas Suas Pala­vras "Neste ano do jubileu, voltareis
cada um a sua possessão" (Ibid., 13) que o resgate deve ocorrer nesse
Ano.
\end{quote}

\begin{enumerate}
\def\labelenumi{\arabic{enumi}.}
\setcounter{enumi}{172}
\item
  \begin{quote}
  Ver os preceitos negativos 223 e 226.
  \end{quote}
\item
  \begin{quote}
  O ano do Jubileu
  \end{quote}
\item
  \begin{quote}
  O ano do Jubileu
  \end{quote}
\end{enumerate}

\begin{quote}
142 MAIMÔNIDES

As Escrituras explicam as regras detalhadas deste preceito, e deixam
claro os direitos do vendedor e do comprador caso ele queira resgatar
sua hé­rança antes do início do Ano do Jubileu. Também deixam claro que
esta lei es­pecífica se aplica apenas às terras localizadas fora das
muralhas de uma cidade, e que aldeias e casas construídas em campo
aberto estão sob a mesma lei, uma vez que não estão dentro das muralhas,
assim como pomares e jardins. Essas são as "casas das aldeias" a
respeito das quais as Escrituras dizem: "E as casas das aldeias que não
têm muro ao redor, como os campos da terra serão consi­deradas; redenção
haverá para elas e no Jubileu sairão do poder do compra­dor" (Ibid.,
31).

As normas deste preceito estão explicadas em Arakhin. Ele também só é
obrigatório na Terra de Israel, e apenas enquanto a lei do Jubileu
estiver em aplicação.

139 O RESGATE DE PROPRIEDADES DENTRO DAS MURALHAS DA CIDADE

Por este preceito somos ordenados a que o resgate das posses ven­didas
dentro das muralhas de uma cidade seja válido apenas por um ano
com­pleto e que depois de um ano elas se tornem propriedade do
comprador, e não sejam devolvidas no Jubileu. Este preceito está
expresso em Suas palavras, enal­tecido seja Ele, "E quando o homem
vender casa de moradia numa cidade mu­rada" (Levítico 25:29).

Este preceito é a "lei das casas numa cidade murada". Suas normas estão
explicadas no Tratado Arakhin, e ele é obrigatório apenas na Terra de
Israel.

140 CONTAR OS ANOS ATÉ. O JUBILEU

Por este preceito somos ordenados a contar os anos a partir do mo­mento
em que conquistamos a terra e nos tornamos seus proprietários, uma
contagem feita em ciclos de sete anos, até o Ano do Jubileu. Este
preceito, ou seja, a contagem dos anos dos ciclos Sabáticos, é de
responsabilidade do Tribu­nal, ou seja, do Grande Sanhedrin, o qual deve
contar os cinqüenta anos, ano a ano, assim como cada um de nós tem que
contar os dias do "omer". Este preceito está expresso em Suas palavras,
enaltecido seja Ele, "Contarás para ti sete semanas de anos" (Levítico
25:8).

A Sifrá diz: "Poder-se-ia pensar que se poderia contar os sete anos de
Shabat sucessivos e proclamar o Jubileu. Por isso a Torah diz 'Sete
vezes sete anos'. Esses são dois versículos e a lei só pode ser
compreendida através dos dois juntos". Quer dizer, a maneira como este
preceito deve ser executado só pode ser entendida através dos dois
versículos: o Sanhedrin tem que contar os anos e os ciclos de Shabat ao
mesmo tempo.

Uma vez que as Escrituras dizem que a lei só pode ser entendida atra­vés
dos dois versículos, conclui-se que, definitivamente, só há um preceito;
por­que se houvesse dois preceitos --- um para a contagem dos anos e
outro para a contagem dos ciclos sabáticos --- não haveria razão pata
dizer "A não ser atra­vés dos dois versículos juntos", porque dois
preceitos são sempre derivados cada um de um versículo, e a expressão "A
não ser através dos dois versículos juntos" só é usada com relação a um
único preceito, cujas normas só podem

PRECEITOS POSITIVOS 143

ser compreendidas por completo através de dois textos. Um exemplo disso
é o caso do primogênito, onde as Escrituras dizem "Todo o que abre a
matriz será para mim" (Exodo 34:19), ensinando-nos que todo primogênito,
macho ou fêmea, pertence ao Eterno; o versículo "Separarás... macho,
para o Eterno" (Ibid., 13:12) nos ensina que todos os machos, sejam
primogênitos ou não, per­tencem ao Eterno; e a partir desses dois
versículos concluímos o significado do preceito --- que ele se aplica
apenas ao primogênito macho --- como está explicado na Mekhiltá.

141 CANCELAR AS DÍVIDAS NO ANO DE SHABAT

Por este preceito somos ordenados a cancelar todas as nossas
recla­mações de dinheiro no Ano de Shabat. Ele está expresso em Suas
palavras, enal­tecido seja Ele, "O que tiveres em poder de teu irmão, o
deixarás" (Deuteronô­mio 15:3) e em Suas palavras "Este é o modo do ano
sabático: que todo o cre­dor, que emprestou a seu companheiro, o
deixará" (Ibid., 2).

A Tosseftá diz: "As Escrituras falam de dois tipos de desistência: a
desistência de terra e a desistência de dinheiro".

A Torah ordena a desistência de dinheiro apenas quando a lei refe­rente
à desistência de terras estiver em vigência, e nesse momento ela a torna
obrigatória em todo lugar.

As normas deste preceito estão explicadas no último capítulo do Tra­tado
Shebiit.

142 COBRAR AS DÍVIDAS DOS IDÓLATRAS

Por este preceito somos ordenados a cobrar as dívidas do id e a
pressioná-lo para que pague, da mesma forma que somos ordenad
misericordiosos para com o israelita e proibidos de exigir o pagamento
Ele está expresso em Suas palavras, enaltecido seja Ele, "Do estrangeir
tra reclamarás" (Deuteronômio, 15:3), a respeito das quais diz o Sifrei:
estrangeiro idólatra reclamarás' é um preceito positivo".

143 A PARTE DO "COHEN" DE CADA

ANIMAL PURO QUE SE ABATE

Por este preceito somos ordenados a dar ao "Cohen" o quarto dian­teiro,
as duas faces e o estômago de todo animal puro que abatermos. Este
pre­ceito está expresso em Suas palavras, enaltecido seja Ele, "E este
será o direito dos `Cohanim' sobre o povo: os que oferecerem sacrifício,
seja boi ou cordei­ro" (Deuteronômio 18:3).

As normas deste preceito estão explicadas no décimo capítulo de Hu­lin;
ele não é obrigatório para os Levitas.

144 MAIMÔNIDES

144 A PRIMEIRA TOSQUIA DEVE SER DADA AO "COHEN"

Por este preceito somos ordenados a separar a primeira tosquia e dá-la
ao "Cohen". Este preceito está expresso em Suas palavras, enaltecido
seja Ele, "A primícia da tosquia de tuas ovelhas, darás a ele"
(Deuteronômio, 18:4) e é obrigatório apenas na Terra de Israel. Suas
normas estão explicadas no décimo primeiro capítulo de Hulin.

145 AS COISAS CONSAGRADAS

Por este preceito somos ordenados quanto à lei das coisas consagra­das,
ou seja, aquele que consagrar alguma coisa que lhe pertence, dizendo:
"Seja isto consagrad\&', deve entregá-la ao "Cohen", a menos que ele
acrescente ex­plicitamente "a Deus", e nesse caso ele deve entregá-la
para ser guardada no Santuário, pois tudo o que for declarado
consagrado, em termos gerais, perten­ce ao "Cohen". Este preceito está
expresso em Suas palavras, enaltecido seja Ele, "No entanto, toda
consagração que uma pessoa fizer ao Eterno, de tudo o que lhe pertencer,
seja homem, ou animal etc." (Levítico 27:28).

Suas palavras "E será o campo, quando sair livre 'no Jubileu, santida­de
ao Eterno, como campo consagrado para o 'Cohen'; a possessão dele
per­tencerá aos 'Cohanim' " (Ibid., 21) mostram que todas as coisas
consagradas em termos gerais pertencem ao "Cohen".

As normas deste preceito estão explicadas no oitavo capítulo de
Arak­hin, e no início de Nedarim.

146 "SHEHITÁ"

Por este preceito somos ordenados a matar os animais de uma ma­neira
determinada, antes de comer sua carne, pois só assim ela será alimento
permitido. Este preceito está expresso em Suas palavras, enaltecido seja
Ele, "Poderás degolar do teu gado, e do teu rebanho, ... como te ordenei
(Deutero­nômio 12:21), a respeito das quais diz o Sifrei: " oderás
degolar': tal como as ofertas consagradas devem ser abatidas de a
m..\textsuperscript{-}leira determinada, assim também os animais
abatidos como alimento d vem se k abatidos dessa maneira. `Como te
ordenei' nos ensina que Moisés f i orderQdo quanto ao esôfago e à
traquéia, e quanto à maior parte de um del, s\textsuperscript{177} no
pássaros e à maior parte de ambos no gado".

Todas as normas e leis sobre este pre o estão explicadas no Trata­do que
lida especificamente com este assunto, que é o Tratado Hulin.

147 COBRIR O SANGUE DE PÁSSAROS E ANIMAIS ABATIDOS

Por este preceito somos ordenados a cobrir o sangue de um pássaro ou
animal depois de abatido. Este preceito está expresso em Suas palavras,
enal­tecido seja Ele, "Derramará o seu sangue e o cobrirá com pó."
(Levítico 17:13).

PRECEITOS POSITIVOS 145
\end{quote}

As normas deste preceito estão explicadas no sexto capítulo de Hulin.

\begin{quote}
148 LIBERAR A MÃE QUANDO SE PEGAR SEUS FILHOTES

Por este preceito somos ordenados a deixar partir do ninho. Ele está
expresso em Suas palavras, enaltecido seja Ele, "Mas deixarás ir
livremente a mãe, e os filhos poderás tomar para ti" (Deuteronômio
22:7).
\end{quote}

As normas deste preceito estão explicadas no último capítulo de

\begin{quote}
Hulin.

149 PROCURAR OS SINAIS DE PUREZA DETERMINADOS NO GADO E NOS ANIMAIS

Por este preceito somos ordenados a procurar certos sinais em ani­mais
domésticos e selvagens, ou seja, que eles ruminem o alimento e tenham o
casco totalmente fendido, pois isso faz deles alimento permitido.
Procurar esses sinais nos animais é um preceito positivo, expresso em
Suas palavras, enal­tecido seja Ele, "Estes são os animais que comereis"
(Levítico 11:2).

A Sifrá diz: " 'Esses comereis' (Ibid., 3): apenas 'esses' podem ser
co­midos, e não os animais impuros"; quer dizer, todo animal que tiver
esses si­nais é alimento permitido, e o animal que não os tiver é
proibido. Temos aqui um preceito negativo derivado de um preceito
positivo, que tem força de um preceito positivo, de acordo com o
princípio que explicamos. É por essa razão que depois dessa sentença a
Sifrá o4z: "Sei apenas que há um preceito positivo; de que modo eu
concluo que ta '.bé = preceito negativo? Porque o Tal-

mud diz: 'O camelô, que rumin casco fendido' " (Ibid., 4), como\\
explicarei nos preceitos negativ

Portanto, ficou claro avras "Esses comereis" são um pre-

ceito positivo, cujo significado , como e pliquei, que somos obrigados a
pro­curar esses sinais em todo animal, e doméstico ou selvagem, e só
então ele pode ser comido. O que o preceito prescreve é a observação
deste pro­cedimento.
\end{quote}

As normas deste preceito estão explicadas nos Tratados Bekhorot

\begin{quote}
e Hulin.

150 PROCURAR OS SINAIS DE PUREZA DETERMINADOS NOS PÁSSAROS

Por este preceito somos ordenados a procurar os sinais nos pássa­ros,
pois apenas alguns deles são alimento permitido. No caso dos pássaros,
os sinais não estão estipulados na Torah, mas foram obtidos através de
estudo. Quan­do examinamos todos os tipos declarados individualmente
proibidos, encon­tramos certos elementos comuns a todos eles e esses são
os sinais dos pássaros

178. Ver o preceito negativo 172.
\end{quote}

146

MAIMÔNIDES

\begin{quote}
impu preceito, de examinar os pássaros e determinar que um é pu-

ro e d impuro, é um preceito positivo.

Sifrei diz: " 'Toda ave pura, podereis comer' (Deuteronômio 14:11)este é
um preceito positivo. Ficou claro, portanto, o que nós assinala­mos
acima.

As normas deste preceito estão explicadas no Tratado Hulin.

151 PROCURAR OS SINAIS DE PUREZA DETERMINADOS NOS GAFANHOTOS

Por este preceito somos ordenados quanto aos sinais também nos
gafanhotos. Eles estão descritos na Torah com as seguintes palavras:
"Que tem pernas por cima dos pés" (Levítico 11:21).

A explicação que demos sobre o preceito anterior também é válida para
este. O versículo das Escrituras que se refere a ele é: "De todo o
réptil alado... deles comereis estes" (Ibid., 21-22).

As normas deste preceito estão explicadas no terceiro capítulo do
Tratado Hulin.
\end{quote}

152 PROCURAR OS SINAIS DE PUREZA\\
DETERMINADOS NOS PEIXES

\begin{quote}
Por este preceito somos ordenados quanto aos sinais nos peixes, que
estão expressos na Torah, em Suas palavras "Isto comereis de tudo o que
está nas águas " etc. (Levítico 11:9). A Guemará de Hulin diz
explicitamente: "Aquele que come um peixe impuro viola um preceito
positivo e um preceito negati­vo", já que de Suas palavras "Isto
comereis" eu concluo que outros peixes não devem ser comidos, e um
preceito negativo derivado de um preceito positivo tem força de preceito
positivo. Fica, portanto, claro que as palavras "Isto co­mereis" são um
preceito positivo. Isto significa, como eu disse, que somos or­denados a
decidir, baseados nesses sinais, que um peixe é alimento permitido e
outro não, como as Escrituras dizem claramente: "E fareis separação
entre o quadrúpede puro e o impuro" (Levítico 20:25).

A separação só pode ser feita através dos sinais, e portanto os sinais
de cada uma das quatro categorias --- animais domésticos e selvagens,
pássaros, gafanhotos e peixes --- constituem um preceito separado e
diferente. Já mostra­mos que cada um deles foi considerado como um
preceito positivo.

As normas deste preceito --- a saber, o preceito referente aos sinais
nos peixes --- também estão explicadas no terceiro capítulo do Tratado
Hulin.

15 3 DETERMINAR A LUA NOVA

Por este preceito o Enaltecido nos ordena quanto ao cálculo dos me­ses e
dos anos. Este é o "preceito da Santificação da Lua Nova" e ele está
ex­presso em Suas palavras, enaltecido seja Ele, "Este mês seja para
vós, princípio dos meses" (Êxodo 12:2). Para explicar isto está dito nos
escritos: "Este teste-

PRECEITOS POSITIVOS 147

munho será entregue a vocês"; ou seja, este preceito não é imposto a
todos, como é o caso do Shabat, da Criação, quando todas as pessoas são
obrigadas a contar seis dias, e a descansar no sétimo. Não cabe a cada
um, ao ver a lua nova, decidir que esse dia é o primeiro do mês, nem
fixar o primeiro dia do mês baseado em cálculos aprendidos, nem
intercalar um mês baseado numa primavera tardia ou em outras
considerações que mereçam ser levadas em con­ta. Esse dever nunca deve
ser cumprido por ninguém a não ser o Grande Tri­bunal e deve ser
executado na Terra de Israel e em nenhum outro lugar. Por­tanto, como
não existe nenhum Grande Tribunal hoje em dia, a observação cessou
atualmente entre nós porque não há nenhum Grande Tribunal, assim como
cessou a oferenda de sacrifícios porque o Templo não existe mais.

Neste ponto enganaram-se os descrentes, que aqui no Oriente são chamados
Caraítas; e até alguns Rabanitas não conseguiram perceber o signifi­cado
deste ponto e começaram a tatear na mais densa escuridão a esse
respeito. Vocês devem saber que não se permite que os cálculos que
fazemos hoje em dia, e pelos quais podemos determinar as luas novas e as
festividades, sejam feitos em algum outro lugar a não ser na Terra de
Israel; mas num caso de emer­gência, e na ausência dos Sábios da Terra
de Israel, foi permitido que um Tri-

bunal, que tenha sido habilitado na Ter Israel, intercale um mês no ano

e determine luas novas fora da Terra d orno o Talmud registra ter feito
\end{quote}

\begin{itemize}
\item
  \begin{quote}
  Rabi Akiba. Esse procedimento, co 1 udo, e tá repleto de grandes
  dificulda-
  \end{quote}
\end{itemize}

\begin{quote}
des, e é sabido que quase sempre hou erra de Israel, e que foram eles,

quando estiveram reunidos, que dete m as luas novas e intercalaram um

mês nos anos, de acordo com os m to corretos.

Um grande e fundamental s rincípio de nossa fé, que não pode ser
conhecido ou entendido corretamente a não ser através de reflexão
profunda, é que quando hoje em dia, estando fora da Teria de Israel,
calculamos pela ta­bela de ano bissexto que temos em mãos, e
determinamos que um dia é de Lua Nova e um outro é de Festival, nós o
fazemos não com base em nosso próprio cálculo, e sim porque o Grande
Tribunal da Terra de Israel designou esse dia como o primeiro do mês, ou
como um dia de Festival. Esse dia se tornou o primeiro dia do mês ou um
dia de Festival porque eles assim o decretaram, quer tenha sido sua
decisão baseada em cálculos ou na observação. Isso está de acordo com
nossa Tradição, que interpreta o versículo "Estas são as solenidades do
Eterno, as santas convocações que proclamareis no seu tempo determinado"
(Levítico 23:4) como significando:"Eu não conheço outras solenidades a
não ser essas", ou seja, aquelas que foram declaradas como sendo
"solenidades", ainda que isso tenha sido feito involuntariamente, ou sob
coação, ou por enga­no, como a Tradição nos diz. Hoje nós fazemos
cálculos apenas para saber que dia foi fixado pelos habitantes da Terra
de Israel, pois é por este método e não pela observação da lua nova que
eles determinam e estabelecem atualmente.
\end{quote}

\begin{itemize}
\item
  \begin{quote}
  na decisão deles que confiamos e não nos nossos cálculos, que nada
  mais são do que constatações. Isto deve ser bem compreendido.
  \end{quote}
\end{itemize}

\begin{quote}
Darei uma explicação adicional sobre este assunto. Suponhamos, por
exemplo, que os habitantes da Terra de Israel desaparecessem --- que
Deus nos livre de tal coisa, pois Ele nos prometeu que não tirará nem
apagará da terra
\end{quote}

\begin{itemize}
\item
  \begin{quote}
  remanescente da nação --- e que não houvesse mais Tribunal lá, e que
  fora da Terra não houvesse mais nenhum Tribunal que tivesse sido
  habilitado na Terra de Israel: nesse caso nossos cálculos não nos
  seriam de nenhuma utilidade
  \end{quote}
\end{itemize}

\begin{quote}
\textbf{180.} Quase sempre houve Sábios na Terra de Israel.

148 MAIMÔNIDES
\end{quote}

porque não dev novas, a não Tzion sairá a L diz a este respei e não
deixa dúvi

zer cálculos fora da Terra, nem intercalar ou fixar luas condições
mencionadas, como explicamos, pois "De alguém em sã consciência examinar
o que o Talmud ará a conclusão de que nossa interpretação está correta

\begin{quote}
Nas escrituras há indicações que estabelecem um funda to para

os princípios em que nos baseamos para reconhecer as luas no s anos

bissextos. Baseado num desses trechos, mais especificamente, " rás es-

te estatuto em seu prazo, de ano em ano" (Êxodo 13:10), foi d to nos

ensina que não devemos acrescentar nenhum mês a não ser n do ano

próxima às solenidades". Foi dito ainda: "De que modo conclui e só de-

vemos acrescentar um dia ao mês ou santificar a lua nova durante o dia?
Pelas palavras das Escrituras `mi-yamim yamima', a duração de um ano em
`dias' ". Sobre as palavras das Escrituras "Este mês seja para vós,
princípio dos meses" (Ibid. 12:2) foi dito: "Você calcula um ano pelos
meses mas não pelos dias", significando que o que se deve acrescentar é
um mês completo. Também foi dito, a respeito das palavras "Porém um mês"
(Números 11:20): "Você calcula um mês pelos dias, não pelas horas"; e
também foi explicado que o versículo "Estejas alerta desde antes que
chegue o mês da primavera" (Deuteronômio 16:1) nos ensina que em nossos
anos devemos levar em consideração as esta­ções do ano, e que portanto
eles devem ser anos solares.

Todas as normas deste preceito estão explicadas na íntegra no pri­meiro
capítulo de Sanhedrin, no Tratado Rosh Hashaná, e também em Berakhot.

154 DESCANSAR NO SHABAT

Por este preceito somos ordenados a descansar no Shabat. Ele está
expresso em Suas palavras "E no sétimo dia descansarás" (Êxodo 34:21), e
está repetido várias vezes; o Enaltecido nos diz que descansar de todo
trabalho é uma obrigação aplicável a nós, a nosso gado, e a nossos
empregados.

As normas deste preceito estão explicadas no Tratado Shabat e no Tratado
Yom Tob.

155 PROCLAMAR A SANTIDADE DO SHABAT

Por este preceito somos ordenados a recitar determinadas palavras no
início e no final do Shabat, mencionando a grandeza e a alta nobreza do
dia, e a diferença desse dia com relação aos dias da semana que o
precedem e os que o sucedem. Este preceito está expresso em Suas
palavras, enaltecido seja Ele, "Estejas lembrado do dia de sábado para
santificá-lo" (Êxodo 20:8): ou seja, comemorá-lo proclamando sua
santidade e sua grandeza. Este é o pre­ceito de "Kidush", santificação.
A Mekhiltá diz: " `Estejas lembrado do dia de sábado para santificá-lo':
santificá-lo com uma bênção". E os Sábios dizem ex­plicitamente:
"Recorda-o sobre o vinho", e também: "Santifica-o na sua chega­da e na
sua partida", referindo-se à "habdala", que faz parte de recordar o
Sha­bat como nos foi ordenado.
\end{quote}

\begin{enumerate}
\def\labelenumi{\arabic{enumi}.}
\setcounter{enumi}{180}
\item
  \begin{quote}
  Isa. 2:3.
  \end{quote}
\item
  \begin{quote}
  Foi dito na Mekhiltá.
  \end{quote}
\end{enumerate}

\begin{quote}
PRECEITOS POSITIVOS 149

As normas deste preceito estão explicadas no final de Pessahim e em
vários trechos de Berakhot e de Shabat.

15 6 RETIRAR O FERMENTO
\end{quote}

Por este preceito somos ordenados a remover o fermento de nossas\\
propriedades no décimo quarto dia de Nissan. Este é o preceito da
Retirada do\\
Fermento, e ele está expresso em Suas palavras, enaltecido seja Ele,
"Mas no\\
primeiro dia cessareis de ter fermento em vossas casas" (Êxodo 12:15).
Os Sá-\\
bios também o chamam de "a queima do pão" (bi'ur), ou seja, o ato de
quei-\\
mar o pão fermentado. A Guemará de Sanhedrin no Talmud de Jerusalém
diz:\\
"O pão fermentado envolve ambos um preceito positivo e um negativo: o
pre-\\
ceito positivo é queimá-lo --- 'Cessareis de ter fermento em vossas
casas' --- e\\
o negativo está em 'Levedura não será encontrada em vossas casas' "
(Ibid., 19).\\
As normas deste preceito estão explicadas no início de Pessahim.

\begin{quote}
157 NARRAR O ÊXODO DO EGITO

Por este preceito somos ordenados a narrar a história do Êxodo do Egito,
com toda a eloqüência de que formos capazes, na véspera do décimo quinto
dia de Nissan. Deverá ser elogiado aquele que discorrer sobre este tema,
contan­do a miséria que nos impuseram os Egípcios e os sofrimentos que
eles nos cau­saram, e sobre a maneira como o Eterno Se vingou deles,
agradecendo-Lhe, enal­tecido seja Ele, por todo o bem com que Ele nos
brindou; como foi dito, "To­dos aqueles que narram longamente a fuga do
Egito merecem ser louvados."

Nas Escrituras este preceito está expresso em Suas palavras, enalteci­do
seja Ele, "E anunciarás a teu filho naquele dia" (Êxodo 13:8). Sobre
isto foi comentado: " 'E anunciarás a teu filho': poder-se-ia pensar que
se deve contar a história do primeiro dia do mês em diante; por isso a
Torah diz 'Naquele dia'. As palavras 'Naquele dia' poderiam ser
interpretadas como significando duran­te o dia; por isso a Torah diz
'Por isto' --- uma expressão que não seria usada a não ser no momento em
que o pão ázimo e as ervas amargas estivessem dian­te de você".
Portanto, é uma obrigação contá-la somente depois do anoitecer.

A Mekhiltá diz: "Como foi dito que 'E será quando te perguntar teu filho
amanhã etc.' (Ibid., 14) poder-se-ia pensar que você deve narrar a
história a seu filho apenas se ele lhe perguntar. Por isso as Escrituras
dizem: 'Anunciarás a teu filho' --- mesmo que ele não pergunte.
Novamente poder-se-ia pensar que isto só se aplica àquele que tiver um
filho; de que forma concluímos que ele se aplica também a quem estiver
só ou entre pessoas estranhas? Pelas palavras das Escrituras: 'E disse
Moisés ao povo: Recordai este dia' (Ibid.,3)"; ou seja, Deus nos mandou
que recordassemos, tal como ele nos ordenou com as pala­vras "Estejas
lembrado do dia de sábado para santificá-lo" (Ibid., 20:8).

Você já está familiarizado com as palavras: "Mesmo se fôssemos to­dos
eruditos, homens com conhecimentos, versados na Lei, ainda assim seria
nossa obrigação narrar a fuga do Egito".
\end{quote}

As normas deste preceito estão explicadas no final de Pessahim.

\begin{quote}
150 COMER PÃO ÁZIMO NA VÉSPERA DO DÉCIMO QUINTO DIA DE NISSAN

Por este preceito somos ordenados a comer pão ázimo na véspera do décimo
quinto dia de Nissan, quer tenha o cordeiro de "Pessah" sido ofere-

150 MAIMÔNIDES

cido ou não. Ele está expresso em Suas palavras, enaltecido seja Ele,
"Pela noi­te, comereis pães ázimos" (Êxodo 12:18), sobre as quais
comentam: " 'Pela noite, comereis pães ázimos': as Escrituras apresentam
isto co ma obrigação."

Está explicado em Pessahim que comer pã•obrigatório na primeira noite
(da festividade), e opcional depois dis

As normas desse preceito estão explicadas atado Pessahim.

159 DESCANSAR NO PRIMEIRO DIA DE "PESSAH"

Por este preceito somos ordenados a descansar no primeiro dia de
"Pessah". Ele está expresso em Suas palavras, enaltecido seja Ele, "O
primeiro dia, de santa convocação será para vós" (Levítico 23:7).
Inicialmente, você de­ve saber que toda vez que o Eterno ordenou
\textsuperscript{-}uma santa convocação" isso é

interpretado como significando que o dia deve ser cado, o que quer di-

zer que nenhum tipo de trabalho deve ser feito nel ser o que se relacio-

na com comida, como está explicado nas Escritur

Nós já nos referimos ao fato de que foi r ito q e a palavra "Shaba­ton",
"descanso solene" constitui um preceito pos como se Ele tivesse dito
"descansa" ou "descansarás", que são maneiras de impor a abstenção do
trabalho. Todos os dias das "épocas determinadas", ou seja, dos
Festivais, são chamados de "Sábados do Eterno" (Levítico 23:38).

Está afirmado em vários trechos do Talmud que "há um preceito po­sitivo
e um preceito negativo com relação aos Festivais"; quer dizer, abster-se
do trabalho durante todos os Festivais é um preceito positivo, e
executar traba­lhos proibidos durante um Festival é um preceito
negativo. De acordo com is­so, todo aquele que fizer um trabalho
proibido estará violando ambos um pre­ceito positivo e um negativo.

As normas deste preceito, ou seja, o de descansar, estão explicadas no
Tratado Yom Tob.

160 DESCANSAR NO SÉTIMO DIA DE "PESSAH"

Por este preceito somos ordenados a descansar no sétimo dia de
"Pes­sah". Ele está expresso em Suas palavras, enaltecido seja Ele, "O
sétimo dia, de santa convocação é" (Levítico 23:8).

161 CONTAR O "OMER"

Por este preceito somos ordenados a contar o "omer". Ele está ex­presso
em Suas palavras, enaltecido seja Ele, "E contareis para vós desde o dia
seguinte ao primeiro dia festivo, desde o dia em que tiverdes trazido o
`omer' da movimentação; sete semanas completas serão" (Levítico 23:15).
\end{quote}

\begin{enumerate}
\def\labelenumi{\arabic{enumi}.}
\setcounter{enumi}{182}
\item
  \begin{quote}
  Não é obrigatório comer-se pão ázimo nos outros dias de "Pessah , mas
  é proibido comer-se pão levedado.
  \end{quote}
\item
  \begin{quote}
  Em Êxodo 12:16.
  \end{quote}
\end{enumerate}

\begin{quote}
PRECEITOS POSITIVOS 151

Você deve saber que da mesma forma que o Tribunal é obrigado a contar os
anos do Jubileu, ano por ano, e ciclo de Shabat por ciclo de Shabat,
como expliquei anteriormente, assim também nós somos obrigados a contar
os dias do "omer", dia por dia, e semana por semana, como está
determinado em Suas palavras "Contareis cinqüenta dias" (Levítico 23:16)
e "Sete semanas contarás para ti" (Deuteronômio 16:9).

Assim como contar os anos e os ciclos de Shabat constitui um único
preceito, também no caso do "omer" um só preceito determina a contagem.
Todos os meus antecessores o contaram corretamente como um preceito; e
você não deve se deixar confundir pelas palavras: "É obrigatório contar
os dias, e é obrigatório contar as semanas" e considerar que há dois
preceitos separados, porque quando o preceito tem várias partes, o
cumprimento de cada uma delas é um preceito. Contudo, haveria dois
preceitos se tivesse sido dito: "Contar os dias é um preceito, e contar
as semanas é um preceito". Isto não passará desapercebido a alguém que
costuma usar as palavras no seu sentido preciso. Se você disser "é
obrigatório fazer isto e aquilo", isto não significa necessaria­mente
que essa determinada ação seja um preceito separado.

Há uma prova clara disto no fato de que ao contar o "omer" nós
enumeramos as semanas todas as noites, dizendo "tantas semanas e tantos
dias", enquanto que se as semanas fossem um preceito separado,
deveríamos men­cionar o número de semanas apenas nas noites em que se
completa cada sema­na; e nesse caso haveria duas bênçãos: uma "Bendito
sejas, o Eterno ... que ... nos ordenastes contar os 'dias' do `omer' "
e outra "contar as 'semanas' do `omer' ". Mas não é esse o caso, e o
preceito determina a contagem do "omer", seus dias e suas semanas, de
acordo com Seu preceito.

Este preceito não é obrigatório para as mulheres.

162 DESCANSAR NO DIA DE "SHABUOT"

Por este preceito somos ordenados a descansar no dia de "Shabuot". Ele
está expresso em Suas palavras "E proclamareis nesse mesmo dia, e haverá
para vós convocação de santidade" (Levítico 23:21).

163 DESCANSAR NO DIA DE "ROSH HASHANÁÁ"

Por este preceito somos ordenados a descansar no primeiro dia de
"Thishri". Ele está expresso em Suas palavras "No sétimo mês, será para
vós descanso solene" (Levítico 23:24).

Nós já mencionamos que foi dito que a palavra "Shabaton", "des­canso
solene", constitui um preceito positivo.

164 JEJUAR NO DIA DE "YOM QUIPUR"

Por este preceito somos ordenados a jejuar no décimo dia do "This­hri".
Ele está expresso em Suas palavras, enaltecido\textsuperscript{.}seja
Ele, "Afligireis vossas almas " etc. (Levítico 16:29)„ que a Sifrá
interpreta desta forma: " 'Afligireis vos-

152 MAIMÔNIDES

sas almas': aflição com relação àquilo de que dependa a vida, ou seja,
abstinên­cia de comer e beber".

A tradição também proíbe lavar-se, untar-se, usar sapatos e manter
relações conjugais e diz que devemos cessar todas essas atividades
porque, co­mo foi dito, "Sábado solene é para vós, e afligireis vossas
almas" (Ibid., 31), o que equivale a dizer que é obrigatória a
abstinência tanto de trabalho de toda espécie como de alimentação e
cuidados com o corpo, e é isso que se usa a expressão "Shabat Shabaton"
--- "um Shabat de descanso solene".

A Sifrá diz: "De que forma concluímos que é proibido lavar-se, untar-se
e manter relações conjugais em Tom Quipur'? Pelas palavras da Sifrá:
'Sába­do solene'; ou seja, devemos abster-nos de todas essas coisas a
ponto que elas nos aflijam.

165 DESCANSAR NO DIA DE "YOM QUIPUR"

Por este preceito somos ordenados a descansar do trabalho de to­dos os
tipos nesse dia. Ele está expresso em Suas palavras " Shabat Shabaton' é
para vós" (Levítico 16:31).

Já explicamos várias vezes que foi dito que a expressão "Shabaton",
"descanso solene", constitui um preceito positivo.

166 DESCANSAR NO PRIMEIRO DIA DE "SUCOT"

Por este preceito somos ordenados a descansar no primeiro dia da Festa
dos Tabernáculos. Ele está expresso em Suas palavras "No primeiro dia
haverá santa convocação" (Levítico 23:35).

167 DESCANSAR NO DIA

DE "SHEMINI ATZERET"

Por este preceito somos ordenados a descansar no oitavo dia da Fes­ta
dos Tabernáculos. Ele está expresso em Suas palavras, enaltecido seja
Ele,

o o vo dia, haverá santa convocação para vós" (Levítico 23:36).
\end{quote}

Você deve saber que a mesma lei se aplica a cada um dos seis

\begin{quote}
as\textsuperscript{185}, d rante os quais somos obrigados a descansar, e
que nenhum deles está

eito a Igum tipo de restrição que não se aplique aos outros. Também
pode-

o parar comida em cada um deles. Portanto, as mesmas regras com rela­ção
ao "descanso" se aplicam a todos os Festivais. Todas as normas a esse
res­peito estão explicadas no Tratado Yom Tob.

Contudo, deve ser ressaltado que o descanso imposto em Shabat e em "Yom
Quipur" acarreta todas as abstinências e muitas outras mais, já que
nesses dois dias não podemos preparar comida. Há outras coisas que nos
são permitidas num dia de Festival e que nos são proibidas no Shabat,
embora não estejam rela­cionadas com a preparação de comida, como está
explicado no Tratado Yom Tob.

PRECEITOS POSITIVOS 153

168 MORAR NUMA CABANA DURANTE OS DIAS DE "SUCOT"

Por este preceito somos ordenados a morar numa cabana por sete dias,
durante toda a Festa. Ele está expresso em Suas palavras, enaltecido
seja Ele, "Nas cabanas habitareis por sete dias" (Levítico 23:42).

As normas deste preceito estão explicadas no Tratado Sucá. Ele não é
obrigatório para as mulheres.

169 PEGAR UM "LULAV" NO "SUCOT"

Por este preceito somos ordenados a pegar um ramo de palmeira e a
alegrar-nos com ele diante do Eterno durante sete dias. Este preceito
está ex­presso em Suas palavras, enaltecido seja Ele, "E tomareis para
vós, no primeiro dia, (o fruto da árvore formosa, palmas de palmeira, e
ramos de murta e de sal­gueiro de ribeiras, e vos alegrareis diante do
Eterno vosso Deus, por sete dias.)" (Levítico 23:40).

As normas deste preceito estão explicadas no terceiro capítulo do
Tratado Sucá. Lá está explicado que é apenas no Santuário que este
preceito é obrigatório por sete dias; nos outros lugares ele é
obrigatório apenas no pri­meiro dia, de acordo com a Torah. Ele não é
obrigatório para as mulheres.

170 OUVIR O "SHOFAR" NO DIA DE "ROSH HASHANÁ"

Por este preceito somos ordenados a ouvir o som do "Shofar" no primeiro
dia de "Tishri". Ele está expresso em Suas palavras "Dia de toque do
Shofar', será para vós" (Números 29:1).

As normas deste preceito estão explicadas no Tratado Rosh Hasha­ná. Ele
não é obrigatório para as mulheres.
\end{quote}

171 DAR MEIO "SHEKEL" ANUAL TE

\begin{quote}
or este preceito somos ordenados a dar meio "sheke os os

a o \textsuperscript{87}. El está expresso em Suas palavras, enaltecido
seja Ele: " a um

o sgate • e sua alma ao Eterno" (Êxodo 30:12), e "Isto dará" (Ib É aro •
e este preceito não é obrigatório para as mulheres, porque ras • izem:
"Cada um que passa para o número dos que são cont

As normas deste preceito estão explicadas no Tratado que 1 cificamente
com este assunto, a saber, o Tratado Shekalim. Lá está ex que o preceito
é obrigatório apenas durante a existência do Santuário.
\end{quote}

172 ACATAR O QUE DIZEM OS PROFETAS

\begin{quote}
Por este preceito somos ordenados a ouvir todo profeta, que a paz esteja
com eles, e a fazer o que quer que ele ordene, mesmo que isso seja con-
\end{quote}

\begin{enumerate}
\def\labelenumi{\arabic{enumi}.}
\setcounter{enumi}{185}
\item
  \begin{quote}
  Moeda de prata.
  \end{quote}
\item
  \begin{quote}
  Ao Santuário.
  \end{quote}
\item
  \begin{quote}
  E o senso militar não incluía mulheres.
  \end{quote}
\end{enumerate}

\begin{quote}
154 MAIMÔNIDES

trário a um ou mais preceitos, desde que isso seja temporário e que não
repre­sente uma adição ou subtração permanente, como explicamos na
introdução de nosso "Comentário sobre a Mishná". O versículo das
Escrituras pelo qual Ele nos impõe isto, enaltecido seja Ele, é: "A ele
ouvireis" (Deuteronômio 18:15), sobre o qual o Sifrei diz: " 'A ele
ouvireis': ainda que ele lhe diga para violar temporariamente um dos
preceitos impostos pela Torah, você deve atendê-lo". Todo aquele que
transgredir este preceito está sujeito à pena de morte pela mão dos
Céus, como estipulado em Suas palavras, enaltecido seja Ele, "E qualquer
homem que não ouvir as minhas palavras, que ele falar em Meu Nome, Eu
lhe pedirei contas" (Ibid., 19).

Está dito em Sanhedrin: "Três pessoas estão sujeitas à morte pela mão
dos Céus: aquele que desobedece um profeta, um profeta que desobedece
seu próprio preceito, e o que oculta sua profecia". Tudo isto se deduz
das palavras "E qualquer homem que não ouvir elo yishma') as Minhas
palavras etc.", lendo-se "lo yishma" também como "lo yishama", "não
obedecerá", aplicável ao pro­feta que desobedecer seu próprio preceito,
e como "lo yashmia", "não se fará ouvir", aplicável ao profeta que
ocultar sua profecia.
\end{quote}

As normas deste preceito estão explicadas no final de Sanhedrin.

\begin{quote}
17 3 NOMEAR UM REI\textsuperscript{-}

Por este preceito somos ordenados a nomear um rei sobre nós, um
Israelita, que unirá toda a nossa nação e' será nosso líder. Este
preceito está ex­presso em Suas palavras, enaltecido seja Ele, "Poderás,
certamente, pôr sobre ti o rei" (Deuteronômio 17:15).

Nós já nos referimos às palavras do Sifrei: "Três preceitos foram
im­postos aos Israelitas para quando eles chegassem à Terra de Israel:
nomear um rei para si mesmos, construir o Santuário, e aniquilar os
descendentes de Amalec".

O Sifrei diz ainda: " 'Poderás, certamente, pôr sobre ti o rei' é um dos
preceitos positivos", e explica que isso significa que ele deve ser
temido, e que nosso respeito por ele e estima pela sua grandeza e
supremacia devem

ser tão grandes que o coloquem num nível onra superior ao de todos os

profetas de sua geração. O Talmud diz ex ente: "O Rei tem prioridade

sobre o profeta"; e quando esse Rei der em que não for conflitante

com um preceito da Torah, nós deve er seu comando, e ele tem

o direito de matar com a espada todo a ele que o desobedecer. Nossos
ante-

passados aceitaram isso sobre si mesmo 'Todo aquele que se rebelar

contra teu comando... deverá ser mort a de todo aquele que se re-

belar contra a autoridade real, seja ele está entregue ao rei devida-

mente nomeado de acordo com a Torah.

As normas deste preceito estão explicadas no segundo capítulo de
Sanhedrin, no início de Queretot, e no sétimo capítulo de Sotá.

174 OBEDECER O GRANDE TRIBUNAL

Por este preceito somos ordenados a obedecer ao Grande Tribunal e a tudo
o que ele nos ordene com relação ao que é proibido e ao que é permi-

PRECEITOS POSITIVOS 155

tido. Quanto a isso não há diferença entre uma decisão baseada na
Tradição,uma a qual eles tenham chegado pela aplicação de uma das leis
de interpretação da Torah, e uma sobre a qual eles concordaram a fim de
delimitar alguma determi­nação da Lei, ou a fim de ir ao encontro de
alguma situação através de uma medida que lhes pareça correta e
calculada, e que reforce a Torah: em todos esses casos somos obrigados a
executar o que eles decidirem e a agir de acordo com suas ordens --- não
podemos desobedecê-los. Este preceito está expresso em Suas palavras,
enaltecido seja Ele, "Conforme o mandado da lei que te ensi­narem"
(Deuteronômio 17:11), a respeito das quais diz o Sifrei: " 'Conforme o
juízo que te disserem, farás' é um preceito positivo".
\end{quote}

As normas deste preceito estão explicadas no final de Sanhedrin.

\begin{quote}
175 ACEITAR A DECISÃO DA MAIORIA

Por este preceito somos ordenados a seguir a maioria caso haja uma
diferença de opinião entre os Sábios com relação a qualquer uma das leis
da Torah. Da mesma forma, se num litígio particular --- por exemplo, num
caso entre Reuben e Simeon --- surgir uma diferença de opiniões entre os
juízes da cidade quanto a ser Simeon ou Reuben o devedor, nós devemos
seguir a maio­ria. Este preceito está expresso em Suas palavras,
enaltecido seja Ele, "Inclina-te à maioria" (Êxodo 23:2). Foi dito
explicitamente: "A maioria é lei decisiva".

As normas e regulamentos deste preceito estão explicados em vá­rios
trechos de Sanhedrin.

176 NOMEAR JUÍZES E OFICIAIS DO TRIBUNAL

Por este preceito somos ordenados a nomear juízes que devem im­por o
cumprimento dos preceitos da Torah, forçar aqueles que se desgarraram a
voltar a trilhar o caminho da verdade, ordenar que se execute o que é
bom e que se evite o que é ruim e aplicar as penalidades sobre os
transgressores, a fim de que os preceitos e as proibições da Torah não
fiquem entregues à von­tade do indivíduo.

Uma das condições deste preceito é que esses juízes devem ser de graus
diferentes, da seguinte forma. Para cada cidade com um número
suficien­te de habitantes se nomeia 23 juízes --- para constituir o
Sanhedrin Menor ---que devem se reunir todos no portão da cidade. Em
Jerusalém deve ser nomea­do o Grande Tribunal, com setenta juízes, e
acima deles o Chefe da Assembléia, que também é chamado pelos Sábios de
"Nassi", e eles devem se reunir num local especificamente designado para
eles. Numa cidade cuja população seja nu­mericamente insuficiente para
ter um Sanhedrin Menor serão de qualquer for­ma nomeados três juízes que
deverão julgar os casos menores e enviar os casos importantes para o
Tribunal Superior.

Também deverão ser nomeados inspetores para visitar os mercados e
supervisionar a conduta das pessoas em suas transações, a fim de que
eles não cometam injustiças nem mesmo em assuntos insignificantes.

Este preceito está expresso em sua ordem, enaltecido seja Ele, "Juí­zes
e policiais, designarás para ti, em cada uma de tuas tribos"
(Deuteronômio 16:18). O Sifrei diz: "De que forma sabemos que devemos
nomear um tribunal para todo povo de Israel? Pelas palavras das
Escrituras: 'Juízes e policiais, desig­narás para ti'. De que forma
sabemos que devemos nomear um acima de to-

156 MAIMÔNIDES

dos os outros? Pelas palavras 'Designarás para ti'. De que forma sabemos
que deve ser nomeado um tribunal para cada tribo? Pelas palavras 'Em
cada uma das tuas portas'. Raban Shimeon ben Gamliel diz: " 'Em cada uma
de tuas por­tas; e julgarão': é obrigatório que cada tribo tenha seu
próprio tribunal porque a Torah diz: "e julgarão o povo" --- mesmo
contra sua vontade' ".

O preceito que nos ordena nomear 70 anciãos está repetido em Suas
palavras, enaltecido seja Ele, a Moisés, "Ajunta-me (li) setenta homens
dos an­ciãos de Israel " etc. (Números 11:16) e disseram que toda vez
que está dito para Mim (li), isso quer dizer que é para sempre.
Portanto, "E Me (li) servirão" (Êxodo 28:41) significa que este preceito
é obrigatório para sempre; não é um preceito temporário, mas sim
obrigatório de geração em geração.

Você deve saber que a nomeação de todos esses tribunais, a saber, o
Sanhedrin Maior e Menor, o Tribunal de Três, assim como as outras
nomea­ções, só podem ter lugar na Terra de Israel, ou não terão
validade. Mas os juízes ordenados na Terra de Israel podem julgar tanto
dentro como fora dela; e esse é o significado das palavras "O Sanhedrin
tem jurisdição dentro e fora da Ter­ra". Contudo, eles não podem julgar
casos de pena capital nem dentro nem fora da Terra a não ser durante a
existência do Santuário, como explicamos no início deste trabalho. A
respeito de Suas palavras, enaltecido seja Ele, --- relati­vas ao
homicídio acidental --- "E serão estes para vós por estatuto de
julgamen­to para as vossas gerações, em todas as vossas moradas"
(Números 35:29), o Sifrei diz: " 'Em todas as vossas moradas' significa
tanto na Terra como fora dela. Poderíamos pensar que as leis sobre as
Cidades de Refúgio também são obrigatórias fora da Terra; por isso a
Torah diz 'E serão estes etc.': estas leis, referentes aos Tribunais,
são obrigatórias tanto dentro como fora da Terra de Israel; as outras
referentes às Cidades de Refúgio, são obrigatórias apenas na Terra de
Israel". •
\end{quote}

Todas as normas deste preceito estão explicadas no Tratado Sanhedrin.

\begin{quote}
177 TRATAR AS PARTES COM IGUALDADE PERANTE A LEI

Por este preceito os juízes são ordenados a tratar com igualdade to­das
as partes, e a permitir que cada um diga o que tem a dizer, quer ele
fale longa ou brevemente. Este preceito está expresso em Suas palavras
"Com justi­ça julgarás o teu próximo" (Levítico 19:15), que a Sifrá
explica da seguinte for­ma: "E proibido permitir a uma pessoa que diga
tudo o que quiser e ordenar a outra que seja breve." Este é um dos
aspectos incluídos neste preceito.

Outro aspecto dele é que todo homem que for conhecedor da Lei é obrigado
a proceder a um julgamento se as partes tiverem começado a argüir diante
dele. Os Sábios dizem explicitamente: "De acordo com as palavras da
Torah, até mesmo uma única pessoa tem competência para julgar casos de
dívi­da, pois está dito: 'Com justiça julgarás o teu
próximo\textsuperscript{---}.

Outro aspecto ainda é que um homem é obrigado a julgar seu próxi­mo com
uma inclinação em seu favor, e a sempre interpretar seus atos e
pala­vras como sendo bons e caritativos.
\end{quote}

As intenções deste preceito estão explicadas em diversos trechos do

\begin{quote}
Talmud.

PRECEITOS POSITIVOS 157

178 TESTEMUNHAR NO TRIBUNAL

Por este preceito somos ordenados a dar ao Tribunal toda e qual­quer
prova que tivermos, quer ela arruíne a pessoa julgada ou salve sua vida
ou seu dinheiro. Somos obrigados a prestar testemunho sobre cada aspecto
e a dizer aos juízes o que vimos ou ouvimos. Os Sábios citam como prova
da obrigação de prestar testemunho as Suas palavras, enaltecido seja
Ele, "Sendo testemunha de um fato, por ter visto ou sabido" (Levítico
5:1). Aquele que vio­lar este preceito e ocultar provas comete pecado
grave, de acordo com Suas palavras, enaltecido seja Ele, "Se não o
denunciar, levará seu pecado" (Ibid.).

Este é o princípio geral. Contudo, se o testemunho que ele omitir é
relativo a dinheiro, e ele o omitir sob juramento, ele será obrigado a
oferecer um Sacrifício de Maior ou Menor Valor, como determinam as
Escrituras, de acor­do com as condições expostas em Shabuot.
\end{quote}

As normas deste preceito estão expostas em Sanhedrin e em Shabuot.

\begin{quote}
179 INVESTIGAR O DEPOIMENTO DAS TESTEMUNHAS

Por este preceito somos ordenados a investigar os depoimentos pres­tados
pelas testemunhas e examiná-los cuidadosamente antes de aplicar um
cas­tigo ou apresentar uma decisão. Temos que ter a máxima cautela para
não che­gar a uma conclusão mal ponderada e precipitada que venha a
prejudicar um inocente. Este preceito está expresso em Suas palavras,
enaltecido seja Ele, "E indagarás, e investigarás, e perguntarás bem; e
se for verdade, e se for certa a coisa" (Deuteronômio 13:15).

As normas deste preceito e suas subdivisões --- como devem ser
con­duzidas as indagações e averiguações, quão cautelosos devemos ser e
de que forma as evidências devem ser aceitas ou rejeitadas, com base nas
investiga­ções --- estão explicadas no Tratado Sanhedrin.

\textbf{180} CONDENAR AS TESTEMUNHAS QUE PRESTAREM FALSO TESTEMUNHO

Por este preceito somos ordenados a punir as testemunhas que pres­tarem
falso testemunho com a pena que elas pensaram que seria aplicada pelo
seu testemunho. Este preceito está expresso em Suas palavras, enaltecido
se* Ele, "Fareis a eles como pensavam fazer a seu irmão" (Deuteronômio 1
• 9). Esta é a Lei do Falso Testemunho: se seu testemunho foi calculado
par. . nde-

nar a uma perda monetária, devemos aplicar-lhes uma perda no mes alor;

se foi calculado para condenar à morte, elas deverão morrer daquela
ra\textsuperscript{190};

e se foi calculado para condenar ao açoitamento, elas deverão sofrer é
igo.

As normas deste preceito, as dúvidas que surgiram com r lação a e e, e a
maneira de provar que os testemunhos são falsos, e que estão, portanto,
sujeitos a esta lei, estão explicadas no Tratado Macot.

158 MAIMÔNIDES

181 " E G LÁ ARUFÁ"

Por este preceito somos ordenados a quebrar o pescoço de urna vaca se
encontrarmos num campo o corpo de um homem assassinado, e se não se
souber quem foi o assassino. Este preceito está expresso em Suas
pala­vras, enaltecido seja Ele,"Quando for achada uma pessoa
assassinada, caída no campo etc." (Deuteronômio 21:1). Esta é a Lei de
Quebrar o Pescoço de uma Vaca. Suas normas estão explicadas no último
capítulo do Tratado Sotá.

182 SEPARAR SEIS CIDADES DE REFÚGIO

Por este preceito somos ordenados a separar seis Cidades de Refú­gio que
estejam prontas para receber a quem matar uma pessoa involuntaria­mente,
e a construir estradas que levem a elas e a nivelar essas estradas, não
deixando nelas nada que possa atrapalhar o fugitivo em sua fuga. Este
preceito está expresso em Suas palavras, enaltecido seja Ele,
"Prepararás o caminho e dividirás em três partes a área de tua terra
etc." (Deuteronômio 19:3).

As normas deste preceito estão explicadas em Sanhedrin, Macot, She­kalim
e Sotá. Nós já citamos do Sifrei que as leis relativas às Cidades de
Refúgio são obrigatórias apenas na Terra de Israel.

183 DESIGNAR CIDADES PARA

O VITAS
\end{quote}

\begin{longtable}[]{@{}ll@{}}
\toprule
\endhead
\begin{minipage}[t]{0.47\columnwidth}\raggedright
\begin{quote}
P que habitem está express cidades par

E fúgio, oferec do Macot.
\end{quote}\strut
\end{minipage} & \begin{minipage}[t]{0.47\columnwidth}\raggedright
\begin{quote}
ste pre eito somos ordenados a dar aos Levitas cidades para ne
as\textsuperscript{191}, p • que eles não receberam nenhum pedaço da
Terra. Ele e Suas 'palavras, enaltecido seja Ele, "Que dêem aos
Levitas... habi ' (Números 35:2).

sas cidades dos Levitas também eram usadas como Cidades de Re­ndo asilo
sob condições especiais, como está explicado no Trata-
\end{quote}\strut
\end{minipage}\tabularnewline
\bottomrule
\end{longtable}

184 ELIMINAR O PERIGO DE\\
NOSSAS MORADIAS

\begin{quote}
Por este preceito somos ordenados a eliminar todos os obstáculos e
possibilidades de perigo dos lugares em que vivemos: ou seja, construir
mu­ros ou parapeitos nos telhados, poços, fossos e similares, para que
ninguém caia neles ou deles. Da mesma forma, toda estrutura perigosa
deve ser reconstruída ou consertada a fim de afastar todo tipo de
perigo. Este preceito está expresso em Suas palavras "Farás um parapeito
no teu telhado" (Deuteronômio 22:8),

PRECEITOS POSITIVOS 159

a respeito das quais diz o Sifrei: " 'Farás um parapeito no teu telhado'
é um preceito positivo".
\end{quote}

As normas deste preceito estão explicadas no Tratado Baba Kamma.

\begin{quote}
185 DESTRUIR TODO TIPO DE IDOLATRIA NA TERRA DE ISRAEL

Por este preceito somos ordenados a destruir todo tipo de idolatria e
seus templos por todas as maneiras possíveis de destruição e
aniquilação: que­brar, queimar, demolir e rasgar usando, para cada
objeto, o meio apropriado para que a destruição seja feita o mais
completa e rapidamente possível, pois a intenção é que não reste nem
traço dele. Isso está expresso em Suas palavras, enaltecido seja Ele,
"Certamente destruireis dos lugares" (Deuteronômio 12:2), em "Mas assim
fareis com elas: seus altares derrubareis etc." (Ibid., 7:5) e
nova­mente em "Porém seus altares derrubareis" (Êxodo 34:13).

A Guemará de Sanhedrin registra casualmente que a menção de um preceito
positivo relativo à idolatria provocou a seguinte pergunta: "Como pode
conceber um preceito positivo com relação à idolatria?" E o Rabi isdá
citou, como explicação: "Seus altares derrubareis".

O Sifrei diz: "De que modo se conclui que se cortar uma
eraI\textsuperscript{92}

e se ela tornar a crescer dez vezes, deve-se cortá-la novamente? Pel.. p
alavras da Torah 'Certamente, destruireis' ". Também está dito ali: " 'E
far is • arecer os seus nomes daquele lugar' (Deuteronômio 12:3): o
preceito de destrui ido­latria se aplica apenas na Terra de Israel".

8 6 A LEI DA CIDADE APÓSTATA

Por este preceito somos ordenados a matar todos os habitantes de uma
Cidade Apóstata e a queimá-la com tudo o que houver nela. Esta é a Lei
da Cidade Apóstata, e ela está expressa em Suas palavras, enaltecido
seja Ele, "E queimarás no fogo, a cidade e todo o seu despojo,
inteiramente" (Deuteronômio 13:17).
\end{quote}

As normas deste preceito estão explicadas no Tratado Sanhedrin.

\begin{quote}
8 7 A GUERRA CONTRA AS
\end{quote}

ES HEREGES

os ordenados a exterminar as Sete Nações que orque eles constituíram a
raiz e primeiro fun­eito está expresso em Suas palavras, enaltecido
Deuteronômio 20:17). Está explicado em vários textos que o objetivo
disso era evitar que imitássemos sua heresia. Há vários trechos nas
Escrituras que nos incitam e insistem veementemente para que os
exterminemos, e a guerra contra eles é obrigatória.

\begin{quote}
Poder-se-ia pensar que este preceito nãò é obrigatório para sempre, uma
vez que as sete nações há muito deixaram de existir, mas essa idéia só
seria concebida por alguém que não tivesse compreendido a diferença
entre os pre-
\end{quote}

\begin{enumerate}
\def\labelenumi{\arabic{enumi}.}
\setcounter{enumi}{191}
\item
  \begin{quote}
  Uma árvore ou bosque devotado à idolatria:
  \end{quote}
\item
  \begin{quote}
  I.e. os hiteus, os girgasheus, os emoreus, os cananeus, os periseus,
  os hiveus, os jebuseus, que eram os idólatras habitantes originais da
  Terra de Israel.
  \end{quote}
\end{enumerate}

\begin{quote}
160 MAIMÔNIDES

ceitos que são obrigatórios através das gerações e os que não o são. Não
se po­de dizer que não seja obrigatório para sempre um preceito que
tenha sido cum­prido por completo --- alcançando seu objetivo --- mas
cujo cumprimento não tenha sido ligado a um limite determinado de tempo,
porque ele será obrigató­rio para cada geração em que surja a
possibilidade de executá-lo. Se o Eterno destriiísse e exterminasse
completamente os Amalequitas --- e que isso ocorra em breve em nossos
dias, de acordo com Sua promessa, enaltecido seja Ele, "Pois extinguirei
totalmente a memória de Amalec" (Êxodo 17:14) --- podería­mos então
dizer que o preceito "Apagarás a memória de Amalec" (Deuteronô­mio
25:19) não seria mais obrigatório através das gerações? Não poderíamos;
o preceito é obrigatório através das gerações, e enquanto existirem
descenden­tes de Amalec eles deverão ser eliminados. Da mesma forma, no
caso das Sete Nações, sua destruição e exterminação é obrigatória, assim
como a guerra con­tra elas: temos o dever de exterminá-las e
persegui-las através de todas as gera­ções até que elas sejam destruídas
completamente. Assim fizemos até que sua destruição foi completada por
Davi, e seus remanescentes se dispersaram e se misturaram com outras
nações de tal forma que não restou mais nenhum traço deles. Mas embora
elas tenham desaparecido, isso não significa que o preceito de
exterminá-las não seja obrigatório para sempre --- assim como não
podemos dizer que a guerra contra Amalec não é obrigatória para sempre
--- mesmo de­pois delas terem sido extinguidas e destruídas. Não há
nenhuma indicação es-

pecífica de tempo ou lugar este preceito, como no caso dos preceitos

especialmente estipulado m cumpridos no deserto ou no Egito. Ao

contrário, ele está ligado em ele é imposto, e eles devem cumpri-lo

enquanto exista algum

De um mod ropriado entender e discernir a diferença

entre um preceito e a oc sia espeito da qual ele nos é ordenado. Um
precei-

to pode ser obrigatório para sempre ainda que as ocasiões deixem de
existir durante um determinado período; mas a falta de ocasião não faz
com que ele deixe de ser um preceito obrigatório através das gerações.
Um preceito deixará de ser obrigatório para sempre quando a situação for
inversa, ou seja, quando tiver sido obrigação nossa executar, sob certas
condições, um determinado ato ou uma determinada ordem que não seja mais
obrigação nossa atualmente, em­bora essas condições ainda persistam. Um
exemplo disso é o caso do Levita idoso que foi desqualificado para o
serviço no deserto e que hoje está qualifica­do entre nós, como está
explicado no lugar apropriado. Você deve entender este princípio e
segui-lo à risca.

188 A EXTINÇÃO DE AMALEC

Por este preceito somos ordenados a exterminar, dentre os descen­dentes
de Esaú, apenas a semente de Amalec, homens e mulheres, jovens e
ve­lhos. Este preceito está expresso em Suas palavras, enaltecido seja
Ele, "Apaga­rás a memória de Amalec" (Deuteronômio 25:19).

Nós já citamos do Sifrei: "Três preceitos foram impostos aos Israeli­tas
quando eles entraram na Terra de Israel: nomear um rei para si mesmos,
construir o Santuário, e aniquilar os descendentes de Amalec". A guerra
contra Amalec também é obrigatória.
\end{quote}

As normas deste preceito estão explicadas no oitavo capítulo de Sotá.

\begin{quote}
194. I.e., enquanto exista algum daqueles contra quem este preceito é
dirigido.

PRECEITOS POSITIVOS 161

109 RECORDAR \emph{OS} ATOS NEFASTOS DE AMALEC

Por este preceito somos ordenados a recordar o que Amalec nos fez quando
nos atacou sem ter sido provocado. Devemos falar nisso sempre e inci­tar
o povo a fazer guerra contra ele, e ordenar-lhe que o odeie, a fim de
que esse assunto não seja esquecido e que o ódio por ele não se
enfraqueça nem diminua com o passar do tempo. Este preceito está
expresso em Suas palavras, enaltecido seja Ele, "Recorda-te do que fez
Amalec" (Deuteronômio 25:17). A esse respeito diz o Sifrei: "
'Recorda-te do que te fez Amalec': com palavras; `Não te esquecerás'
(Ibid., 19): com o coração"; ou seja, você deve falar dessas coisas para
assegurar-se de que o ódio contra Amalec não seja afastado do cora­ção
dos homens. E a Sifrá diz: " 'Recorda-te do que te fez Amalec':
poder-se-ia pensar que isto significa 'Em teu coração'. Mas 'Não te
esquecerás' se refere

ao esquecimento do coração: então como se pode obedecer o prece'. e 're-

cordar'? Com a palavra". Veja como o profeta Samuel cumpriu e to:

primeiro ele recordou com palavras e depois deu ordens para qu fossem
r-tos, segundo Suas palavras "Eu recordo o que Amalec fez a I rae

190 A LEI DA GUERRA NÃO OB ATÓRIA

Por eito somos ordenados quanto a guerras não obrigató-

rias contra naç C'so entremos em guerra contra eles,.somos obrigados

a fazer um aco eles para poupar suas vidas se eles fizerem as pazes co-

nosco e nos ent eg em suas terras, e nesse caso eles deverão nos pagar
tribu­tos e ser nossos súditos. Este preceito está expresso em Suas
palavras, enalteci­do seja Ele, "Te será tributário ou te servirá"
(Deuteronômio 20:11). A esse res­peito diz o Sifrei: "Se eles disserem
'nós concordamos com os tributos mas re­cusamos a servidão', ou,
'concordamos com a servidão, mas nos recusamos a pagar os tributos', não
devemos concordar: eles devem aceitar as duas condi­ções". Isto
significa que eles devem pagar um tributo anual a ser determinado pelo
rei daquela ocasião, e obedecer suas ordens com temor e humildade, co­mo
convém aos súditos. Contudo, se eles não fizerem a paz conosco, somos
ordenados a matar toda a população masculina, jovens e velhos, e a tomar
tudo o que lhes pertence, inclusive suas mulheres. Este preceito está
expresso em Suas palavras, enaltecido seja Ele, "E se não fizer paz
contigo etc." (Ibid. 12); tudo isso está na lei da guerra não
obrigatória.

As normas deste preceito estão explicadas no oitavo capítulo de So­tá, e
no segundo capítulo de Sanhedrin.

191 NOMEAR UM "COHEN" PARA A GUERRA

Por este preceito somos ordenados a nomear um "Cohen" para fa­zer ao
povo o discurso referente à guerra, quando eles forem partir para a
luta,
\end{quote}

\begin{enumerate}
\def\labelenumi{\arabic{enumi}.}
\setcounter{enumi}{194}
\item
  \begin{quote}
  1 Sam. 15:2.
  \end{quote}
\item
  \begin{quote}
  Guerras contra outras nações além daquelas contra quem somos ordenados
  a guerrear.
  \end{quote}
\end{enumerate}

\begin{quote}
162 MAIMÔNIDES

e para mandar de volta todo homem que não estiver apto para a batalha,
seja porque ele tem o coração fraco ou porque seus pensamentos estão
ocupados com alguma outra coisa que possa impedi-lo de se concentrar na
luta, ou seja, com uma das três coisas especificadas nas Escrituras. Só
depois disso é que eles devem entrar em luta. Esse "Cohen" é chamado de
"Mashuah Mil-Hama" para a Guerra. Em seu discurso ele deve dizer o que
está escrito na Torah e acres­centar palavras que incitem as pessoas à
guerra, e que as induzam a entregar suas vidas pelo triunfo da Fé no
Eterno, e pela punição dos ímpios que arrui­nam a ordem social. Este
preceito está expresso em Suas palavras, enaltecido seja Ele, "E quando
vos aproximardes à luta, o 'Cohen' se chegar-1" (Deutero­nômio 20:2).

O "Cohen" então ordena que seja proclamado nas linhas de comba­te que
devem retornar às suas casas todos aqueles que forem fracos de coração e
os que tiverem construído uma casa e não tenham morado nela, ou que
tive­rem plantado um vinhedo e não tenham comido seus frutos, ou que
tenham prometido casamento a uma mulher e não a tenham desposado. Isso
está de acordo com as palavras das Escrituras "E falarão os policiais"
(Ibid., 5-8), sobre as quais a Guemará diz: "O 'Cohen' fala, e o
policial faz ouvir suas palavras".

Todo este procedimento --- o discurso do "Mashuah Mil-Hama" pa­ra a
guerra e sua proclamação entre as linhas de combate --- só é obrigatório
em caso de uma guerra não obrigatória, pois só a ela se aplica esta lei.
No caso de uma guerra obrigatória não há tal procedimento --- nem
discurso nem pro­clamação --- como se pode verificar no oitavo capítulo
de Sotá, onde as nor­mas deste preceito estão explicadas.

192 PREPARAR UM LUGAR SEPARADO DO ACAMPAMENTO
\end{quote}

or este preceito somos ordenados a que quando nossas tropas fo-

\begin{quote}
rem erra devemos preparar um local fora do acampamento ao qual

eles ra que eles não o façam indiscriminadamente em qualquer lu-

gar o tendas, como fazem as outras nações. Este preceito está expres-

so em S .alavras, e tecido seja Ele, "E um lugar (yad) terás para ti,
fora

do acampamento, e sai 's fora" (Deuteronômio 23:13), sobre as quais o
Si-frei diz: " `Yad' sig ica ap nas um lugar, como está dito 'E vede!
ele instalou um lugar (yad) par. si'\textsuperscript{198}"

193 INCL IR UMA ESTACA ENTRE OS UTENSÍLIOS DE GUERRA

Por este preceito somos ordenados a que cada homem do exército se muna
de um instrumento para cavar como parte de seus utensílios de guer­ra,
com a qual ele deverá cavar a terra e cobrir o excremento depois de ter
feito suas necessidades no local designado para esse fim, para que não
se veja ne­nhum traço dos excrementos no solo do acampamento, como Ele
ordenou no início do trecho que começa com as palavras "Quando te
acampares contra
\end{quote}

\begin{enumerate}
\def\labelenumi{\arabic{enumi}.}
\setcounter{enumi}{196}
\item
  \begin{quote}
  Para fazer suas necessidades.
  \end{quote}
\item
  \begin{quote}
  I Sam. 15:12.
  \end{quote}
\end{enumerate}

\begin{quote}
PRECEITOS POSITIVOS 163

os teus inimigos" (Deuteronômio 23:10). Este preceito está expresso em
Suas palavras, enaltecido seja Ele, "E uma estaca, terás para ti, entre
os objetos de teu uso (azenecha)" (Ibid., 14), sobre as quais o Sifrei
diz: " 'Azenecha' signifi­ca apenas o lugar de tuas armas".

194 UM LADRÃO DEVE DEVOLVER O OBJETO ROUBADO

Por este preceito somos ordenados a fazer devolver o objeto que alguém
tenha roubado, se el\textsuperscript{-} 'nda existir, acrescentando um
quinto de seu

valor, ou a reembolsar o se , caso ele tenha sofrido alguma alteração.
Es-

te preceito está expresso Su.. palavras, enaltecido seja Ele, "Devolverá
o

que roubou" (Levítico 5:

Está explicado no ado Macot que o preceito negativo referente

ao latrocínio é um precei • negativo justaposto a um preceito positivo.
"O Mi­sericordioso", diz, `ordenou `Não extorquirás' (Ibid., 19:13), e
'Devolverá o que roubou' ".

As normas deste preceito estão explicadas nos últimos capítulos de Baba
Kamma.

195 "TSEDAKÁ'

Por este preceito somos ordenados a dar "Tsed sustentar

os necessitados e a aliviar suas cargas. Este preceito está ex r várias
ma-

neiras em Suas palavras, como por exemplo "Abrirás tua mã teu irmão,

para teu pobre" (Deuteronômio 15:11), e também "Deterás sua decaída
mes­mo se ele é peregrino ou estrangeiro morador da terra e viverá
contigo" (Leví­tico 25:35), e ainda "E viverá teu irmão contigo" (Ibid.,
36). O significado de todas essas frases é o mesmo, ou seja, que devemos
ajudar os nossos pobres e sustentá-los de acordo com suas necessidades.

As normas deste preceito estão explicadas em vários lugares,
princi­palmente em Quetubot e em Baba Batra.

De acordo com a Tradição, mesmo o homem pobre que vive de "Tse­daká" tem
a obrigação de cumprir este preceito; quer dizer, ele deve dar
"Tse­daká", ainda que mínima, a alguém que seja mais pobre do que ele ou
tão po­bre quanto ele próprio.

196 BONIFICAR O SERVO QUE RECOBRAR SUA LIBERDADE

Por este preceito somos ordenados a bonificar um servo e a ajudá-lo
quando ele for libertado, a fim de que ele não saia de mãos vazias. Este
preceito está expresso em Suas palavras, enaltecido seja Ele,
"Carrega-lo-ás, fornecendo-lhe do teu rebanho, de tua eira, e do teu
depósito de vinho; e do que te aben-
\end{quote}

\begin{enumerate}
\def\labelenumi{\arabic{enumi}.}
\setcounter{enumi}{198}
\item
  \begin{quote}
  A lei que diz que se deve acrescentar um quinto do valor se apliéa se
  o ladrão tiver jurado em falso a esse respeito.
  \end{quote}
\item
  \begin{quote}
  Foi usado o termo Hebraico "Tsedaká" e não caridade, pois a raiz da
  palavra "Tsedaká" é "Tsedek", ou justiça, o que é mais forte do que
  fazer uma simples caridade. O preceito ordena fazer caridade no
  sentido de fazer justiça.
  \end{quote}
\end{enumerate}

\begin{quote}
164 MAIMÔNIDES

çoou o Eterno, teu Deús, lhe darás."(Deuteronômio 15:14).
\end{quote}

As normas deste preceito estão explicadas no primeiro capítulo de

\begin{quote}
Kidushin.
\end{quote}

\textbf{197 EMPRESTAR DINHEIRO AOS POBRES}

\begin{quote}
Por este preceito somos ordenados a emprestar ao homem pobre para
ajudá-lo a aliviar sua situação. Esta é uma obrigação maior e de maior
peso do que a "Tsedaká" porque o mendigo, cuja necessidade o obriga a
pedir es­mola abertamente, não sofre tão grande angústia quanto aquele
que nunca teve que fazer isso e que precise de ajuda para que sua
pobreza não seja descoberta. Este preceito está expresso em Suas
palavras, enaltecido seja Ele, "Se empresta­res dinheiro a Meu povo, ao
pobre que está contigo" (Êxodo 22:24).

A Mekhiltá diz: "Todo `se' na Torah implica uma opção, com exce­ção de
três deles, um dos quais está no versículo `Se emprestares dinheiro a
Meu povo' ". "Se emprestares dinheiro", dizem os Sábios, "envolve uma
obri­gação. Caso você questione isto, e sugira que esta seja uma simples
permissão, as Escrituras dizem mais adiante `Lhe emprestarás o
suficiente para o que lhe faltar' (Deuteronômio 15:8), o que é uma
obrigação, não simplesmente uma ques­tão de opção".

As normas deste preceito também estão explicadas em vários trechos de
Quetubot e Baba Batra.

\textbf{198 COBRAR JUROS DO IDÓLATRA}

Por este preceito somos ordenados a exigir juros do dinheiro que
emprestarmos a um idólatra, de maneira a não ajudá-lo, nem ser amável
com ele, mas ao contrário, prejudicá-lo, mesmo se lhe fizermos um
empréstimo, o

que nos é proibido fazer no caso de um filho de Isr te preceito está ex-

presso em Suas palavras, enaltecido seja Ele, "Do iro poderás cobrar

juros" (Deuteronômio 23. 1), e de acordo com a i ção tradicional este

é um preceito positivo nã uma questão de op o é o que diz o Si-

frei a respeito: " 'Do E range o poderás cobrar j preceito positivo;

`E a teu irmão não pag ás'\textsuperscript{202} um preceito negati o ste
preceito também

há certas condições est ulad s pelos Sábios que estão explicadas no
Tratado Baba Metzia.

\textbf{199 DEVOLVER O PENHOR AO}

\textbf{PROPRIETÁRIO NECESSITADO}

Por este preceito somos ordenados a devolver um penhor ao seu
proprietário israelita no momento em que ele o necessitar. Quando o
penhor for algo de que ele precise durante o dia --- como por exemplo,
as ferramentas de seu trabalho ou ocupação --- ele lhe deve ser
devolvido para uso durante o dia, e guardado em caução durante a noite;
se for algo de que ele precise à noite --- como por exemplo roupa de
cama ou roupas côm as quais ele durma
\end{quote}

\begin{enumerate}
\def\labelenumi{\arabic{enumi}.}
\setcounter{enumi}{200}
\item
  \begin{quote}
  Aparentemente o autor traduz este versículo como significando "Você
  emprestará com juros".
  \end{quote}
\item
  \begin{quote}
  Ver 0 preceito negativo 235.
  \end{quote}
\end{enumerate}

\begin{quote}
PRECEITOS POSITIVOS 165

--- ele deve ser devolvido para ser usado à noite, e guardado em caução
duran­te o dia. A Mekhiltá diz: " 'Até pôr-se o sol a devolverás' (Êxodo
22:25) se refe­re a uma vestimenta usada durante o dia, que deve ser
devolvida para o dia inteiro. De que forma concluímos que uma vestimenta
usada durante a noite deve ser devolvida para a noite inteira? Pelas
palavras das Escrituras: 'Restituir-lhe-ás o penhor ao por-do-sol'
(Deuteronômio 24:13). Assim, conclui-se que uma vestimenta de dia deve
ser guardada como penhor durante a noite e de­volvida para ser usada
durante o dia, e que uma vestime noturna deve ser guardada como penhor
durante o dia e devolvida pa ada a noite.
\end{quote}

Está explicado na Guemará de Macot que as pala ras "Não entra-

\begin{quote}
rás em sua casa para lhe tomar o seu penhor" (Ibid. tém um precei-

to negativo justaposto a um preceito positivo, estan o mo expresso em

Suas palavras "Restituir-lhe-ás". E o Sifrei diz: "As pala
estituir-lhe-ás' nos

ensinam que o que é usado durante o dia deve ser devolvido para o dia, e
o que é usado durante a noite deve ser devolvido para a noite: um
cobertor du­rante a noite, e um arado durante o dia".
\end{quote}

As normas deste preceito estão explicadas no nono capítulo de Baba

\begin{quote}
Metzia.

200 PAGAR OS SOLDOS NO DIA

Por este preceito somos ordenados a pagar a diária do trabalhador no
mesmo dia, e não adiar o pagamento para outro dia. Este preceito está
ex­presso em Suas palavras, enaltecido seja Ele, "No seu dia, lhe
pagarás a sua diá­ria" (Deuteronômio 24:15). 1 ac • do com as normas
deste preceito, um ope­rário que trabalha de dia p • - req erer seu
pagamento a qualquer momento da noite, e um trabalhado • turno a
qualquer momento do dia, como expli­carei nos preceitos negat
v.,\textsuperscript{204}.

As normas des e p eceit o estão explicadas no nono capítulo do Tra­tado
Baba Metzia, onde fica cl • • ue ele é obrigatório no caso dos
trabalhado­res diaristas, gentis ou Israelitas, e que é um preceito
positivo pagar no momento certo.

201 UM EMPREGADO DEVE

PODER COMER DAQUILO COM QUE ELE TRABALHA

Por este preceito somos ordenados a permitir que um trabalhador coma
durante seu trabalho daquilo com o qual ele está trabalhando, desde que
isso ainda esteja unido ao solo. Este preceito está expresso em Suas
palavras, enaltecido seja Ele, "Quando entrares na vinha de teu
companheiro, poderás comer uvas... quando entrares na seara de teu
companheiro, poderás colher espigas com a tua mão" (Deuteronômio
23:25-26). A Guemará de Baba Metzia explica que desses dois versículos
nós deduzimos que ele pode comer daquilo com que ele estiver trabalhando
e que ainda esteja ligado ao solo; e que ne­nhum dos dois versículos é
suficiente sem o outro, como no caso mencionado
\end{quote}

\begin{enumerate}
\def\labelenumi{\arabic{enumi}.}
\setcounter{enumi}{202}
\item
  \begin{quote}
  Ver o preceito negativo 239.
  \end{quote}
\item
  \begin{quote}
  Ver o preceito negativo 238.
  \end{quote}
\end{enumerate}

\begin{quote}
166 MAIMÔNIDES

anteriormente, a respeito do qual citamos que "estes são dois textos
diferentes e a lei só pode ser compreendida através dos dois juntos".
Nesse caso o precei­to positivo de que um trabalhador deve ter permissão
para comer o que ainda está unido ao solo se deriva de dois versículos,
e os Sábios dizem claramente: "Eles podem comer de acordo com a lei das
Escrituras etc".

As normas deste preceito estão explicadas no sétimo capítulo do Tra­tado
Baba Metzia.

202 DESCARREGAR UM ANIMAL CANSADO

Por este preceito somos ordenados a descarregar um animal que te­nha
sucumbido no campo sob o peso de sua carga. Este preceito está expresso
em Suas palavras, enaltecido seja Ele, "Quando vires o asno daquele que
te abor­rece, prostrado debaixo de sua carga, não te recusarás a
ajudá-lo; auxiliá-lo-ás" (Êxodo 23:5), a respeito das quais a Mekhiltá
diz: " 'Auxiliá-lo-ás' se refere a descarregar o peso". Também está
escrito ali: "As palavras 'Aux' - o 's' nos ensinam que se infringe
ambos um preceito positivo e um nega vo". •u seja, somos ordenados a
descarregar o animal e somos proibidos• deixá-1 pros­trado sob sua
carga, como explicaremos nos preceitos negat os\textsuperscript{205} ; e
\textless{} quele que o deixar caído estará infringindo um preceito
positivo e u negati o. Por­tanto foi explicado que as palavras
"Auxiliá-lo-ás" contém um p ceito ositivo.

As normas deste preceito estão explicadas no segu d. pítulo de Baba
Metzia.

203 AJUDAR O PRÓXIMO A LEVANTAR SUA CARGA

Por este preceito somos ordenados a carregar uma carga sobre o ani­mal
ou sobre o homem, sé ele estiver só, depois que ela tiver sido
descarregada por nós ou por outra pessoa. Assim como somos ordenados a
ajudar a descar­regar, somos ordenados a ajudar a carregar. Este
preceito está expresso em Suas palavras, enaltecido seja Ele, "Mas
ajudarás a levantá-los" (Deuteronômio 11:4), sobre as quais diz a
Mekhiltá: " 'Ajudarás a levantá-los' se refere a carregar".

As normas deste preceito estão explicadas no segundo capítulo de Baba
Metzia, onde foi deixado claro que a Torah,obriga tanto a carregar como
a descarregar.

204 DEVOLVER A SEU DONO O QUE ELE TIVER PERDIDO

Por este preceito somos ordenados a devolver a seu dono o que ele tiver
perdido. Este preceito está expresso em Suas palavras "Devolvê-lo-ás"
(Êxo­do 23:4) e "Mas os restituirás a teu irmão" (Deuteronômio 22:1). Os
Sábios di­zem explicitamente: "A devolução de um pertence perdido é um
preceito po­sitivo". Eles ainda dizem o seguinte, com respeito a um
pertence perdido: "So­mos ensinados que violamos um preceito positivo e
um negativo". Explicare-

205. ver o preceito negativo 270.

PRECEI • S POSITIVOS 167
\end{quote}

ao negativo referente a pertences perdidos no lugar apro-

\begin{quote}
pr'ádo\textsuperscript{2}ob.

A\textsubscript{)}5 normas deste preceito estão explicadas no segundo
capítulo de \textbf{Be}ba Metzia.

\textbf{205 REPREENDER O PECADOR}

Por este preceito somos ordenados a repreender quem estiver pe­cando ou
estiver inclinado a cometer um pecado, a fim de proibi-lo de agir des­sa
forma e a repreendê-lo. Um homem não pode dizer: "Eu não vou pecar; e se
outra pessoa pecar, isso é um assunto entre ele e Deus". Tal atitude é
contrá­ria à Torah. Somos ordenados a não pecar e não permitir que
ninguém de nos­sa nação o faça. Se alguém desejar pecar é dever de cada
um de nós repreendê-lo e evitar que ele o faça, ainda que não haja
evidência de que ele será castigado por isso. Este preceito está
expresso em Suas palavras, enaltecido seja Ele, 'Re­preenderás a teu
companheiro" (Levítico 19:17).

Está incluído neste preceito repreender aquele que nos tenha ofen­dido e
não lhe guardar rancor nem nutrir maus pensamentos a seu respeito. Somos
ordenados a repreendê-lo diretamente, para que nada perdure em nos­so
coração contra ele. A Sifrá diz: "Como sabemos que mesmo qúe já o
tenha­mos repreendido por quatro ou cinco vezes, ainda devemos tornar a
repreendê-lo? Porque a Torah diz `Repreenderás' --- ainda que mil vezes.
Poder-se-ia pen­sar que ao repreendê-lo se poderia humilhá-lo; o Talmud
diz 'E não levarás so­bre ti pecado' ".

Os Sábios explicam que este preceito é obrigatório para todos, de forma
que até um subalterno é obrigado a repreender um superior, e mesmo que
ele seja amaldiçoado e insultado ele não deve desistir nem deixar de
fazê-lo até que batam nele --- como disseram aqueles que transcreveram a
Tradição: "Até que ele seja esbofeteado".

As condições e regulamentos relativos a este preceito estão explica­das
em lugares dispersos do Talmud.

\textbf{206 AMAR O PRÓXIMO}

Por este preceito somos ordenados a amar uns aos outros da mesma forma
que amamos a nós mesmos, e que o amor e a compaixão de alguém por seu
irmão de fé deve ser igual ao amor e compaixão que ele tem por si mesmo
com relação a seu dinheiro, seu corpo, e a tudo o que ele possui e
deseja. Tudo aquilo que eu desejar para mim, devo desejar também para
ele; e tudo o que eu não quiser para mim nem para meus amigos também não
devo desejar para ele. Este preceito está expresso em Suas palavras,
enaltecido seja Ele, "Amarás o teu próximo como a ti mesmo" (Levítico
19:18).

\textbf{207 AMAR O PROSÉLITO}

Por este preceito somos ordenados a amar o prosélito. Ele está ex­presso
em Suas palavras, enaltecido seja Ele, "E amareis ao prosélito"
(Deute­ronômio 10:19). Embora esse prosélito esteja incluído em Israel
--- de tal forma

206. Ver o preceito negativo 269.

168 MAIMÔNIDES

que as palavras "Amarás o teu próximo como a ti mesmo" (Levítico 19:18)
se apliquem também a ele ---, o Eterno ordenou que lhe seja dedicado um
or maior porque ele se converteu à Fé, e acrescentou um
preceito\textsubscript{.} e\textsubscript{2:}

seci eu

favor, assim como fez no caso da advertência quanto a eng or-

dena: "E não enganareis cada um ao seu companheiro" (1 • ,
\textsubscript{5}
\end{quote}

de-

\begin{quote}
pois acrescenta: "Ao prosélito não fraudareis" (Êxodo 22
0\textsuperscript{208}. ção

dada na Guemará é que "Aquele que engana um proséli o culpa d duas

violações: 'Não enganareis cada um ao seu companheira' e•rosélito não
fraudareis'. Também nossa obrigação de amá-lo está em ambos 'Amarás o
teu próximo como a ti mesmo' e 'Amareis ao prosélito' " . Isto está
claro acima de qualquer dúvida, e não conheço ninguém que, ao enumerar
os preceitos, te­nha se enganado a este respeito.

Na maioria dos Midrashot está explicado que o Eterno nos ordenou com
relação aos prosélitos o que Ele nos ordenou com relação a Si mesmo ao
dizer "E amarás ao Eterno, teu Deus" (Deuteronômio 6:5) e "Amareis ao
prosélito".

208 A LEI DOS PESOS E MEDIDAS

Por este preceito somos ordenados a ter pesos, balanças e medidas
exatos, e a regulá-los com extrema precisão. Ele está expresso em Suas
palavras "Balanças justas, pesos justos, 'efá' justa e 'hin' justo
tereis para vós" (Levítico 19:36), sobre as quais diz a Sifrá: "
'Balanças justas' significa que as balanças deve ser absolutamente
precisas: 'pesos justos' significa que os pesos devem ser absolutamente
exatos; " 'efá" justa' significa que a 'efá' deve ser absoluta­mente
exata; e " 'hin" justo' significa que o 'hin' deve ser absolutamente
exa­to". Você sabe que a "efá" é uma medida para sólidos e o "hin" uma
medida para líquidos. A mesma regra se aplica a todos esses casos,
embora o tipo de medidas possa variar, porque o que é pesado ou medido é
simplesmente a quan­tidade de alguma coisa. Todo esse tipo de coisas ---
balanças, pesos, medidas para sólidos e medidas para líquidos --- são
chamadas medidas, e o preceito que nos obriga a regular cada uma delas
com precisão, de acordo com os pa­drões aprovados, é chamado de preceito
das Medidas.

A Sifrá diz: "Eu vos tirei da terra do Egito com a condição de que vocês
aceitem sobre si mesmos o preceito sobre as Medidas; pois aquele que
reconhece o preceito da Medidas reconhece através dele o Êxodo do Egito,
e aquele que o negar estará negando (também a autenticidade do) o Êxodo
do Egito".
\end{quote}

As normas deste preceito estão explicadas no quinto capítulo de Ba-

\begin{quote}
ba Batra.

209 HONRAR OS ERUDITOS E OS IDOSOS

Por este preceito somos ordenados a respeitar os eruditos e a
levantar-nos diante deles a fim de honrá-los. Ele está expresso em Suas
palavras, enalte­cido seja Ele, "Diante das cãs te levantarás e honrarás
as faces do velho" (Leví-
\end{quote}

\begin{enumerate}
\def\labelenumi{\arabic{enumi}.}
\setcounter{enumi}{206}
\item
  \begin{quote}
  Ver o preceito negativo 251.
  \end{quote}
\item
  \begin{quote}
  Ver o preceito negativo 252.
  \end{quote}
\end{enumerate}

\begin{quote}
PRECEITOS POSITIVOS 169

tico 19:32), sobre as quais a Sifrá diz: "Te levantarás e honrarás ---
levantar-se para demonstrar respeito". As normas deste preceito estão
explicadas no pri­meiro capítulo de Kidushin.

Você deve saber que embora este preceito para respeitar os Erudi­tos
seja um dever igual para todos, inclusive para um Erudito com relação a
outro de igual conhecimento --- como dizem os Sábios, "Os eruditos da
Babi­lônia estavam habituados a levantar-se uns diante dos outros" ---
ele é especial­mente e sobretudo obrigatório para um discípulo, pois ele
deve um respeito muito maior a seu mestre do que a qualquer outro
Erudito, assim como ele tem a obrigação de temê-lo, pois os Sábios
afirmam claramente que nosso dever pa­ra com nosso mestre é maior do que
nosso dever para com nosso pai, a quem as Escrituras nos obrigam a
honrar e a temer. E os Sábios dizem explicitamente: "Seu pai e seu
mestre --- seu mestre tem prioridade".

Os Sábios também deixam claro que um discípulo está proibido de
contestar seu mestre e por contestar quero dizer se opor a sua decisão,
re'

sua opinião e lecionar e instruir sem sua permissão. Ele também está
Óibi de brigar e discutir com ele, e de j negativamente, ou seja, atri u
r mot\\
vações ruins a seus atos ou palavas, p• s é possível que suas inten •
es' \textsuperscript{09} nã

sejam aquelas. No capítulo "Hel. s Sábios dizem: "Discordar e se

tre é como discordar da 'Shekhi mo está dito em 'Quando fizeram brigar o
povo contra o Eterno' (Núme os 26:9); brigar com seu mestre é como
brigar com a Shekhiná', como está dito em 'Estas são as águas de Meribá
porque bri­garam os filhos de Israel com o Eterno' (Ibid., 20:13 ue
ar-se de seu mestre

é como queixar-se da Shekhiná', como está dito e sobre nós vossas

queixas, senão sobre o Eterno' (Êxodo 16:8); atri eu mestre é como

atribuir à Shekhiná', como está dito em 'E falou o p ntra Deus e contra

Moisés' (Números 21:5)". Tudo isto está perfeitamente claro, pois embora
a dis­cussão de Korah, a briga dos filhos de Israel, e suas acusações e
suspeitas fos­sem na realidade dirigidas contra Moisés, que era o Mestre
de Israel, as Escritu­ras ós consideram como tendo sido dirigidas contra
Deus. E os Sábios dizem expressamente: "Que o temor a seu mestre seja
como o temor aos Cé

Tudo o que foi exposto foi deduzido do preceito das Escrit s de honrar
os Eruditos e os pais, e ficou claro, pela linguagem do Talmud, e212 não
é um preceito independente. Você deve compreender isto.

210 HONRAR OS PAIS

Por este preceito somos ordenados a honrar nossos pais. Ele está
ex­presso em Suas palavras, enaltecido seja Ele, "Honrarás a teu pai e a
tua mãe" (Êxodo 20:12). As normas deste preceito estão explicadas em
vários trechos do Talmud especialmente em Kidushin

A Sifrá diz: "O que signi hon r? Assegurar-lhes comida e bebi-

da, roupas e calor, e guiar seus pas os213.
\end{quote}

\begin{enumerate}
\def\labelenumi{\arabic{enumi}.}
\setcounter{enumi}{208}
\item
  \begin{quote}
  É possível que as intenções do mestre não sejam as que se pensa.
  \end{quote}
\item
  \begin{quote}
  Do Tratado Sanhedrin.
  \end{quote}
\item
  \begin{quote}
  Atribuir maldade a seu mestre.
  \end{quote}
\item
  \begin{quote}
  Que o temor por seu mestre.
  \end{quote}
\item
  \begin{quote}
  Quando os pais estiverem velhos e fracos.
  \end{quote}
\end{enumerate}

\begin{quote}
1\textsuperscript{-}( MAIMÔNIDES

211 RESPEITAR OS PAIS

Por este preceito somos ordenados a temer nossos pais, a olhá-los com o
respeito devido a alguém de quem se teme o castigo, como o Rei, e a
tratá-los como tratamos aqueles a quem tememos e receamos descontentar.
Ele está expresso em Suas palavras, exaltado seja Ele, "Cada um a sua
mãe e a seu pai temerá" (Levítico 19:3), sobre as quais diz a Sifrá: "O
que significa temer? Não se sentar em seu assento, nem falar em seu
lugar, nem contradizer suas palavras".
\end{quote}

As normas deste preceito estão explicadas em Kidushin.

\begin{quote}
212 "FRUTIFICAR E MULTIPLICAR"

Por este preceito somos ordenados a frutificar-nos e multiplicar-nos
para perpetuar a espécie. Esta é a lei da multiplicação; ela está
expressa em Suas palavras, enaltecido seja Ele, "Frutificai e
multiplicai" (Gênesis 1:28). O Talmud diz que por ocasião do casamento
com uma virgem o noivo está dispensado de recitar o "Shemá" porque ele
estará ocupado cumprindo um preceito.

As normas deste preceito estão explicadas no sexto capítulo de Ye­bamot.
Ele não se aplica às mulheres: o Talmud diz explicitamente "O dever de
frutificar e multiplicar-se recai sobre o homem, não sobre a mulher".

213 A LEI DA CONSAGRAÇÃO PELO CASAMENTO

Por este preceito somos ordenados a desposar uma mulher através de uma
cerimônia de compromisso: seja dando-lhe alguma coisa, ou entregando-lhe
uma certidão de consagração pelo casamento, ou por relação carnal. Este
é o preceito relativo à cerimônia da consagração pelo casamento. Está
dito o seguinte: "Quando um homem tomar uma mulher e se casar com ela,
etc." (Deu-

teronômio 24:1) nos ensina que se pode ter uma mulher através da ção
car-

nal. "E tendo ela saído da sua casa, poderá ir tornar-se mul ,2) nos

ensina que assim como sua partida se faz por um docu ent
\textsuperscript{14}, e a pode tornar-se esposa de um homem também por
um documen ndemos que uma mulher pode ser adquirida por dinheiro pelas
Sua • avras relativas à serva hebréia "Sem dar dinheiro" (Êxodo 21:11),
a respeito das quais diz o Talmud: "Este senhor não recebe dinheiro, mas
há outro senhor que recebe dinheiro, que é seu pai". Contudo, a
consagração pelo casamento ordena pela Torah é a da relação carnal, como
explicado em vários trechos dos Tratados Quetubot, Kidushin, e Nidá.

As normas deste preceito estão explicadas na íntegra no Tratado que lida
especificamente com este assunto, que é o Tratado Kidushin.

Os Sábios dizem especificamente que a consagração pelo casamento feita
através da relação carnal é ordenada pela Torah. Assim fica claro que a
ori-
\end{quote}

\begin{enumerate}
\def\labelenumi{\arabic{enumi}.}
\setcounter{enumi}{213}
\item
  \begin{quote}
  O documento que concede o divórcio.
  \end{quote}
\item
  \begin{quote}
  O documento que atesta a consagração pelo casamento.
  \end{quote}
\end{enumerate}

\begin{quote}
PRECEITOS POSITIVOS 171

gem do preceito relativo à cerimônia da consagração pelo casamento está
nas Escrituras.

214 O MARIDO DEVE DEDICAR-SE A SUA ESPOSA DURANTE UM ANO

Por este preceito o marido é ordenado a dedicar-se a sua esposa du­rante
um ano inteiro, durante o qual ele não deverá sair do país para viajar
ou ir à guerra, nem assumir qualquer outra obrigação desse tipo. Ele
deverá alegrar-se com sua esposa durante um ano inteiro a partir do dia
de seu casamento. Este preceito está expresso em Suas palavras,
enaltecido seja Ele, "Livre estará para cuidar de sua casa por um ano, e
alegrará a mulher que tomou" (Deutero­nômio 24:5).
\end{quote}

As normas deste preceito estão explicadas no oitavo capítulo de Sotá.

\begin{quote}
215 A LEI DA CIRCUNCISÃO

Por este preceito somos ordenados a circuncidar. Este preceito está
expresso em Suas palavras, enaltecido seja Ele, a Abraham "Será
circuncidado, entre vós, todo varão" (Gênesis 17:10). Torah decreta
explicitamente a ex­tinção para aquele que violar este pre• - to
ositivo, com as palavras, enalteci­do seja Ele, "E o varão incircunciso,
ue nã circuncidar a carne de seu prepú­cio, essa alma será cortada"
(Ibid. 4)216

As normas deste preceito tão plicadas no décimo nono capítulo de Shabat,
e no quarto capítulo de Yè..mot. A obrigação de circuncidar um filho
recai sobre o pai e não sobre a mãe, como está explicado em Kidushin.

21 6 A LEI DO CASAMENTO LEVIRATO

Por este preceito somos ordenados a que um cunhado tome por es­posa a
viúva de seu irmão quando este tiver morrido sem deixar descendentes.
Ele está expresso em Suas palavras, enaltecido seja Ele, "O irmão de seu
mari­do estará com ela" (Deuteronômio 25:5).
\end{quote}

As normas deste preceito estão explicadas no Tratado Yebamot.

\begin{quote}
217 "HALITZÁ'

Por este preceito a mulher de um irmão falecido fica ordenada a
exe­cutar "Halitzá" em seu cunhado, se ele não se casar com ela. Ele
está expresso em Suas palavras, enaltecido seja Ele, "E lhe descalçará o
sapato do pé" (Deute­ronômio 25:9).

As normas deste preceito estão explicadas no Tratado que lida
espe­cialmente com este assunto, que é o Tratado Yebamot.

216. Ou seja, todo Israelita incircunciso se torna culpado e está
sujeito à extinção, quando ele atin­ge a maioridade. O pai, contudo, não
está sujeito a essa penalidade por não ter incluído seu filho no pacto
de Abraham, embora ele assim esteja transgredindo o mesmo preceito
positivo.

172 MAIMÔNIDES

Você já está familiarizado com a regra estabelecida de que "O dever de
casar-se com a esposa de um irmão falecido tem prioridade sobre o dever
da `Halitzá' ". É por esse motivo que o Tratado é chamado "Yebamot",
embo­ra ele inclua ambas as leis do casamento levirato e as de
"Halitzá".

218 UM VIOLADOR DEVE CASAR-SE COM A MOÇA QUE VIOLENTOU

Por este preceito um homem é obrigado a casar-se com a moça que ele
tiver violentado. Ele está expresso em Suas palavras, enaltecido seja
Ele, "E ela lhe será por mulher, porquanto a afligiu, e não a poderá
despedir por todos os seus dia ' Deuteronômio 22:29). A Guemará de Macot
afirma que o precei­to negatiativo à violação, que é "E não a poderá
despedir por todos os

seus • a é um preceito negativo precedido por um positivo, e assim

disser mal isso é um preceito negativo precedido por um preceito po-

sitivo". nto, fica claro que as palavras "E ela lhe será por mulher"
consti-

tui um preceito positivo.

As normas deste preceito estão explicadas no terceiro e quarto
capí­tulos de Quetubot.

219 A LEI SOBRE AQUELE QUE DIFAMA SUA ESPOSA

Este preceito estabelece a lei relativa a um homem que difam Este
preceito ordena que ele seja açoitado e que fi vez que com relação a
isso foi dito "E lhe será por dir, por todos os seus dias" (Deuteronômio
22:19

A Guemará de Macot afirma que este é u ceito negativo prece-\\
dido por um positivo, assim como no caso do violentador.

As normas deste preceito estão explicadas no terceiro e quarto
capí­tulos de Quetubot.

220 A LEI SOBRE O SEDUTOR

Por este preceito somos ordenados quanto à lei sobre o sedutor. Ele está
expresso em Suas palavras, enaltecido seja Ele, "E quando enganar um
ho­mem a uma virgem etc." (Êxodo 22:15).

As normas deste preceito estão explicadas no terceiro e quarto
capí­tulos de Quetubot.

22 LEI SOBRE A MULHER CATIVA

P r este preceito somos ordenados quanto à lei de uma bela
\textsubscript{mu}i \textsubscript{r}221\textsubscript{.} El está
expresso em Suas palavras, enaltecido seja Ele, "E vires en-
\end{quote}

\begin{enumerate}
\def\labelenumi{\arabic{enumi}.}
\setcounter{enumi}{216}
\item
  \begin{quote}
  Ver o preceito negativ6358.
  \end{quote}
\item
  \begin{quote}
  Os Sábios.
  \end{quote}
\item
  \begin{quote}
  Difamar a moça com quem ele tenha se casado.
  \end{quote}
\item
  \begin{quote}
  Ver também o preceito negativo 359.
  \end{quote}
\item
  \begin{quote}
  Que tenha sido capturada durante uma guerra.
  \end{quote}
\end{enumerate}

\begin{quote}
PRECEITOS POSITIVOS 173

tre os cativos uma mulher formosa" (Deuteronômio 21:11).
\end{quote}

As normas deste preceito estão explicadas no início de Kidushin.

\begin{quote}
\textbf{222} A LEI DO DIVÓRCIO

Por este preceito somos ordenados a que, se desejarmos divorciar-nos de
uma mulher, o façamos unicamente através de um documento de divór­cio.
Este preceito está expresso em Suas palavras, enaltecido seja Ele,
"Escrever-lhe-á uma carta de divórcio" (Deuteronômio 24:1).

As normas deste preceito, que é a lei do divórcio, estão explicadas por
inteiro no Tratado que lida especificamente com este assunto, e que é o
Tratado Guitin.
\end{quote}

SUSPEITA DE ADULTÉRIO

Por este preceito somos ordenados quanto à lei da mulher suspeita

tido adultério. Ele está expresso em Suas palavras, enaltecido seja ho
em, quando se desviar sua mulher etc.' (Números 5:12). normas deste
preceito --- a maneira pela qual ela deve ser forçada levar seu
sacrifício, e as outras condições --- estão explicadas no lida
especificamente com este assunto, que é o Tratado Sotá.

\begin{quote}
224 AÇOITAR OS TRANSGRESSORES DE DETERMINADOS PRECEITOS

Por este preceito somos ordenados a açoitar com uma correia os
vio­ladores de determinados preceitos. Ele está expresso em Suas
palavras, enalte­cido seja Ele, "O juiz o fará deitar e o fará açoitar
na sua presença" (Deuteronô­mio 25:2). Quando lidarmos com os preceitos
negativos nós indicaremos quais são os preceitos cuja violação é punida
com o açoitamento.

As normas deste preceito estão explicadas no Tratado Macot.

225 A LEI DO HOMICÍDIO INVOLUNTÁRIO

Por este preceito somos ordenados a exilar de sua cidade um homi­cida
involuntário para uma cidade de refúgio. Ele está expresso em Suas
pala­vras, enaltecido seja Ele, "E ficará nela até morrer o 'Cohen
Gadol' " (Números 35:25). sobre as quais diz o Sifrei: " 'Ficará nela':
ele nunca poderá sair de lá porque a palavra 'nela' significa que lá ele
deverá viver, lá deverá morrer, e lá deverá ser enterrado".

As normas deste preceito estão explicadas no Tratado Macot.

174 MAIMÔNIDES

226 EXECUTAR COM A ESPADA OS TRANSGRESSORES DE

DETERMINADOS PRECEITOS

Por este preceito somos ordenados a executar com a espada os vio­ladores
de determinados preceitos. Ele está expresso em Suas palavras,
enalte­cido seja Ele, "Serão certamente vingados" (Êxodo 21:20). quando
tratarmos dos preceitos negativos mostraremos quais são os preceitos
cuja violação é pu­nida com a decapitação.

As normas deste preceito estão explicadas no sétimo capítulo do Tra­tado
Sanhedrin.

227 ESTRANGULAR OS TRANSGRESSORES DE

DETERMINADOS PRECEITOS

Por este preceito somos ordenados a estrangular os violadores de
determinados preceitos. Ele está expresso em Suas palavras, enaltecido
seja Ele, "Certamente serão mortos" (Levítico 20:10). Quando tratarmos
dos preceitos negativos mostraremos quais são os preceitos cuja violação
é punida com o estrangulamento.

As normas deste preceito estão explicadas no sétimo capítulo do Tra­tado
Sanhedrin.

228 QUEIMAR OS TRANSGRESSORES DE DETERMINADOS PRECEITOS

Por este preceito somos ordenados a queimar os violadores de
de­terminados preceitos. Ele está expresso em Suas palavras, enaltecido
seja Ele, "No fogo queimarão a ele e a ela" (Levítico 20:14). Quando
tratarmos dos pre­ceitos negativos mostaremos quais são os preceitos
cuja violação é punida com a morte pelo fogo.

As normas deste preceito estão explicadas no sétimo capítulo do Tra­tado
Sanhedrin.

229 APEDREJAR OS TRANSGRESSORES DE DETERMINADOS PRECEITOS

Por este preceito somos ordenados a apedrejar os violadores de
de­terminados preceitos. Ele está expresso em Suas palavras, enaltecido
seja Ele, "E os apedrejareis, e morrerão" (Deuteronômio 22:24). Quando
tratarmos dos preceitos negativos mostraremos quais são os preceitos
cuja violação é punida com a morte por apedrejamento.
\end{quote}

As normas deste preceito estão explicadas no sexto capítulo de

\begin{quote}
Sanhedrin.

PRECEITOS POSITIVOS 175

230 PENDURAR OS CORPOS DE

CERTOS TRANSGRESSORES DEPOIS DE EXECUTADOS

Por este preceito somos ordenados a pendurar certos transgressores
executados por ordem do Tribunal. Ele está e esso em Suas palavras,
enalte­cido seja Ele, "O pendurarás num madei • (De eronômio 21:22).
Quando tratarmos dos preceitos negativos mostr \textsuperscript{r}emos
qu s são os preceitos cuja vio­lação acarreta que o corpo seja pend
ado\textsuperscript{223}.
\end{quote}

As normas deste preceito e o expl adas no sexto capítulo de

\begin{quote}
Sanhedrin.

231 A LEI DO ENTERRO

Por este preceito somos ordenados a enterrar no dia da execução aqueles
que tiverem sido mortos por ordem do Tribunal. Ele está expresso em Suas
palavras, enaltecido seja Ele, "Certamente enterra-lo-ás no mesmo dia"
(Deuteronômio 21:23), sobre as quais o Sifrei diz: 'Certamente
enterra-lo-ás' é um preceito positivo.

A mesma lei é obrigatória com relação a todos os outros mortos: to­do
Israelita deve ser enterrado no dia de sua morte. É por essa razão que
quan­do não há ninguém para assistir ao enterro de um corpo ele é
chamado de Corpo de Obrigação Religiosa; ou seja, um corpo cujo enterro
é o dever de cada pessoa, de acordo com Suas palavras, enaltecido seja
Ele, "Certamente enterra-lo-ás".
\end{quote}

As normas deste preceito estão explicadas no sexto capítulo de

\begin{quote}
Sanhedrin.

232 A LEI DO SERVO HEBREU

Por este preceito somos ordenados quanto à lei do servo hebreu. Ele está
expresso em Suas palavras, enaltecido seja Ele, "Quando comprares um
escravo hebreu etc." (Êxodo 21:2).

As normas deste preceito estão claramente explicadas nos versícu­los da
Torah e todos os seus regulamentos estão no Tratado Kidushin.

2 3 3 O CASAMENTO DE UMA SERVA HEBRÉIA COM SEU AMO OU COM O FILHO DELE

Por este preceito o homem que comprar uma serva hebréia ou o fi­lho dele
são ordenados a casar-se com ela. Este é o preceito relativo aos
espon­sais. Os Sábios dizem explicitamente: "O dever dos esponsais tem
prioridade sobre o dever de resgate, porque o Enaltecido diz: 'Que não a
consagrou para si deve remi-la' (Êxodo 21:8)".
\end{quote}

Você deve estar ciente, de que as leis referentes ao servo e à serva

\begin{quote}
176 MAIMÔNIDES

hebreus só estarão em vigor durante a vigência da lei do Jubileu.

As normas deste preceito estão explicadas no primeiro capítulo do
Tratado Kidushin.

234 O RESGATE DE UMA SERVA HEBRÉIA

Por este preceito somos ordenados quanto ao resgate de uma serva
hebréia. Ele está expresso em Suas palavras, enaltecido seja Ele, "Deve
remi-la" (Êxodo 2 1 :8).

Há muitas normas, condições e regras referentes a este dever de
res­gate. Elas estão todas explicadas no Tratado Kidushin, onde a lei
sobre a serva hebréia está exposta na íntegra.

Explicando Suas palavras, enaltecido seja Ele, "E se não lhe fizer
es­tas três coisas" (Ibid., 11), a respeito da serva hebréia, a Mekhiltá
diz que seu dono deve casar-se com ela, ou casá-la com seu filho, ou
remi-la.

235 A LEI SOBRE O ESCRAVO CANANEJ

Por este preceito somos ordenados quanto à lei ,s\textsuperscript{l}obre
uru escravo cananeu; ela diz que ele deve ser escravo para sempre e
que\textsuperscript{/}não po adquirir sua liberdade a não ser por causa
de um dente ou um olho\textsuperscript{224}, ou por causa de qualquer
outro órgão do corpo que não torne a crescer, de ac rdo com a
interpretação tradicional. Este preceito está expresso em Suas palavras
"Perpe­tuamente vos fareis servir deles" (Levítico 25:46) e "E quando
ferir um homem o olho de seu escravo etc." (Êxodo 21:26).

A Guemará de Guitin diz: "Todo aquele que liberar seu escravo pa­gão
estará violando um preceito positivo pois está escrito 'Perpetuamente
vos fareis servir deles' " . Contudo, a Torah diz que ele poderá obter a
liberdade por causa de um dente ou de um olho.

As normas deste preceito estão explicadas por completo em Kidus­hin e em
Guitin.

236 A PENALIDADE POR CAUSAR FERIMENTOS

Por este preceito somos ordenados quanto à lei sobre alguém que fere seu
companheiro. Ele está expresso em Suas palavras, enaltecido seja Ele, "E
quando brigarem homens, e ferir um homem a seu próximo etc." (Exodo 21 :
18). Essas são chamadas leis sobre Penalidades todas elas têm sua base
nas Escrituras nas Suas palavras, enaltecido seja Ele, "Conforme ele
fez, assim lhe será feito" (Levítico 24:19), cujo significado é que um
homem deve pagar a im­portância equivalente ao dano que causou a seu
companheiro. A Tradição de­termina que mesmo que ele o tenha apenas
envergonhado, ele deve ser multa­do na importância equivalente.

Você deve saber que todas essas leis relativas a Penalidades se apli­cam
a ferimentos feitos por um homem a outro homem. Da mesma forma exis-

224. Se o dono do escravo lhe causar a perda de um dente ou de um olho.

PRECEITOS POSITIVOS 177

tem também leis relativas a ferimentos causados por um animal a uma
pessoa, e vice-versa. Apenas um Tribunal assentado na Terra de Israel
poderá julgar e pronunciar uma sentença referente a essas leis.

A regulamentação deste preceito está explicada no primeiro capítu­lo de
Baba Kamma.

237 A LEI SOBRE FERIMENTOS CAUSADOS POR UM BOI

Por este preceito somos ordenados quanto à lei do boi. Ele está
ex­presso em Suas palavras, enaltecido seja Ele, "Quando marrar um boi a
um ho­mem ou a uma mulher etc." (Êxodo 21:28), e "E quando ferir o boi
de um ho­mem ao boi de seu companheiro etc." (Ibid., 35).

A regulamentação desta lei está explicada nos primeiros seis capítu­los
de Baba Kamma.
\end{quote}

\begin{longtable}[]{@{}ll@{}}
\toprule
\endhead
\begin{minipage}[t]{0.47\columnwidth}\raggedright
\begin{quote}
238
\end{quote}

expresso\\
um poço

\begin{quote}
capítulos
\end{quote}\strut
\end{minipage} & \begin{minipage}[t]{0.47\columnwidth}\raggedright
\begin{quote}
A LEI SOBRE FERIMENTOS CAUSADOS POR UM POÇO

Por este preceito somos ordenados quanto à lei do poço. Ele está em Suas
palavras, enaltecido seja Ele, "E quando um homem abrir etc." (Êxodo
21:33).

A regulamentação deste preceito está explicada no terceiro e quarto de
Baba Kamma.
\end{quote}\strut
\end{minipage}\tabularnewline
\bottomrule
\end{longtable}

\begin{quote}
239 A LEI SOBRE O ROUBO
\end{quote}

Por este preceito somos ordenados quanto à lei do ladrã

\begin{quote}
cobrar dele u que podemo mos vendê-16\textsuperscript{227} que estão
s\textsubscript{i}

Todos os detalhes desta lei estão explicados no sétimo capítulo de Baba
Kamma, no oitavo capítulo de Sanhedrin, no terceiro capítulo de Baba
Metzia, e em alguns trechos de Quetubot, Kidushin, e Shabuot.

240 A LEI SOBRE OS PREJUÍZOS

CAUSADOS POR UM ANIMAL

Por este preceito somos ordenados quanto à lei sobre o animal que
destrói colheitas. Ele está expresso em Suas palavras, enaltecido seja
Ele, "Quan­do um homem fizer pastar num campo ou numa vinha etc." (xodo
22:4).

A regulamentação de toda esta lei está explicada no segundo e sexto
capítulos de Baba Kamma, e no quinto capítulo de Ghitin.
\end{quote}

\begin{enumerate}
\def\labelenumi{\arabic{enumi}.}
\setcounter{enumi}{224}
\item
  \begin{quote}
  Êxodo 21:37.
  \end{quote}
\item
  \begin{quote}
  Ibid., 22:1.
  \end{quote}
\item
  \begin{quote}
  Caso ele não possa fazer a restituição, como lhe é ordenado. Ibid.,
  22:2.
  \end{quote}
\end{enumerate}

\begin{quote}
1 MAIMÔNIDES

241 A LEI SOBRE OS PREJUÍZOS CAUSADOS PELO FOGO

Por este preceito somos ordenados quanto à lei do fogo. Ele está
ex­presso em Suas palavras, enaltecido seja Ele, "Quando houver fogo, e
pegar nos espinhos etc." (Êxodo 22:5).

A regulamentação desta lei está explicada no segundo e no sexto
ca­pítulos de Baba Kamma.

242 A LEI SOBRE O DEPOSITÁRIO NÃO REMUNERADO

Por este preceito somos ordenados quanto à lei de um depositário não
remunerado. Ele está expresso em Suas palavras, enaltecido seja Ele,
"Quando o homem der ao seu companheiro, dinheiro ou objetos para
guar­dar etc." (Êxodo 22:6).

Os detalhes desta lei estão explicados no nono capítulo de Baba Kam­ma,
no terceiro capítulo de Baba Metzia, e no oitavo capítulo de Shabuot.

243 A LEI SOBRE O DEPOSITÁRIO REMUNERADO

Por este preceito somos ordenados quanto à lei de um depositário
remunerado ou de um arrendador, sendo que uma só lei se aplica a ambos,
co­mo foi explicado pelos Sábios, os quais dizem que há três leis para
regulamen­tar quatro tipos de depositários. Este preceito está expresso
em Suas palavras, enaltecido seja Ele, "Quando der o homem a seu
companheiro, asno, boi, car­neiro etc." (Exodo 22:9).

Todos os detalhes desta lei estão explicados no sexto e no nono
ca­pítulo de Baba Kamma, no terceiro e no sexto capítulos de Baba
Metzia, e no oitavo capítulo de Shabuot.

244 A LEI SOBRE QUEM PEDE EMPRESTADO

Por este preceito somos ordenados quanto à lei sobre aquele que pede
emprestado. Ele está expresso em Suas palavras, enaltecido seja Ele, "E
quando um homem pedir emprestado de seu companheiro etc." (Êxodo 22:13).

A regulamentação desta lei está explicada no oitavo capítulo de Ba­ba
Metzia e no oitavo capítulo de Shabuot.

245 A LEI DE COMPRA E VENDA

Por este preceito somos ordenados quanto à lei de compra e venda; ou
seja, o procedimento a seguir pelo vendedor e pelo comprador ao efetuar
uma transação. Aprendemos este procedimento através de Suas palavras,
enal­tecido seja Ele, "E quando fizerdes uma venda a vosso companheiro,
ou com­prardes da mão de vosso companheiro etc." (Levítico 25:14), que
os Sábios in-

PRECEITOS POSITIVOS 179

terpretam como refer do-se a "uma mercadoria comprada de mão em mão, ou
seja, por `meshichá'\textsuperscript{228}

Foi demonstra que a aquisição por meio de dinheiro fica assegu­rada pela
lei das Escrituras e que a "meshichá" no caso de bens móveis é ape­nas
uma regulamentação dos Sábios, da mesma forma que entregá-la o vende­dor
ao comprador ou levantá-la o comprador. O Talmud diz explicitamente:
-"Assim como eles instituíram a `meshichá' para compradores, eles também
ins­tituíram a `meshichá' para depositários".

Assim, foi deixado claro que a norma da "meshichá" numa venda foi
instituída pelos Sábios, como está explicado no lugar apropriado; mas as
outras formas de procedimento através das quais são adquiridas terras e
outras coisas, a saber, documentos e delimitações, estão baseadas no
versículo.

Os detalhes dessa lei --- ou seja, o procedimento a seguir em cada caso,
ao efetuar uma venda --- estão eplicados no primeiro capítulo de
Kidus­hin, no quarto e oitavo capítulos de Baba Metzia, e no terceiro,
quarto, quinto, sexto e sétimo capítulos de Baba Batra.

246 A LEI SOBRE OS LITIGANTES

Por este preceito som

e o acusado. Ele está expresso m Su toda coisa de delito ... a respeito
da qu respeito a Mekhiltá diz: " 'É e\$te'\textsuperscript{229}

Esta lei inclui todos os caso vam confissões ou desmentidos.

Os detalhes desta lei estão explicados no terceiro capítulo de Baba
Kamma, no início e no oitavo capítulo de Baba Metzia, e no quinto, sexto
e sétimo capítulos de Shabuot; muitas perguntas também são encontradas
espa­lhadas em vários lugares do Talmud.

247 SALVAR A VIDA DO PERSEGUIDO

Por este preceito somos ordenados a salvar uma pessoa do persegui­dor
que tiver a intenção de matá-la, até mesmo tirando a vida do
perseguidor; ou seja, devemos matar o perseguidor se não pudermos salvar
o perseguido de nenhuma outra forma. Este preceito está expresso em Suas
palavras, enalte­cido seja Ele, "Cortar-lhe-ás a mão, o teu olho não
terá piedade dela (Deutero­nômio 25: 1 1-1 2). A esse respeito o Sifrei
diz: " 'Pelas suas vergonhas' (Ibid.): assim como o ato aqui
especificado, por envolver perigo de vida, justifica cortar-se a mão da
mulher, o mesmo princípio deve ser aplicado toda vez que houver risco de
vida. Contudo este versículo nos diz apenas que o homem deve ser salvo
cortando-se a mão da mulher. Como saber se no caso de um homem que não
possa ser salvo cortando-se a mão de alguém, devemos salvá-lo tirando
uma vida? Pelas palavras 'O teu olho não terá piedade' ".

Portanto, o significado deste preceito foi deixado claro, sendo que as
palavras "a mulher de um" são usadas apenas neste caso específico, e
sendo que o verdadeiro significado é que a vida do perseguido deve ser
salva a custa
\end{quote}

\begin{enumerate}
\def\labelenumi{\arabic{enumi}.}
\setcounter{enumi}{227}
\item
  \begin{quote}
  Recibo ou comprovante de uma transação comercial.
  \end{quote}
\item
  \begin{quote}
  O acusado.
  \end{quote}
\end{enumerate}

\begin{quote}
180 MAIMÔNIDES

dos membros do perseguidor, e que quando é impossível salvá-lo a não ser
ma­tando imediatamente o perseguidor, isso deve ser feito.
\end{quote}

As normas deste preceito estão explicadas no oitavo capítulo de

\begin{quote}
Sanhedrin.

248 A LEI SOBRE AS HERANÇAS

Por este preceito somos ordenados quanto à lei sobre as heranças. Ele
está expresso em Suas palavras, enaltecido seja Ele, "Quando um homem
morrer e não tiver filho etc." (Números 27:8).

Uma das normas desta lei é sem dúvida alguma que o filho nito herde o
dobro dos outros, pois esta é uma das leis das heranç

As normas deste preceito estão explicadas no oitavo e no los de Baba
Batra.
\end{quote}

COMENTÁRIOS FINAIS DE MAIMÔNIDES\\
SOBRE OS PRECEITOS POSITIVOS

\begin{quote}
Você deve saber que quando digo, a respeito de cada preceito, "suas
normas estão explicadas em tal-e-tal lugar" eu não estou querendo dizer
que o capítulo ou tratado mencionado contém todas as normas daquele
preceito, em seus mínimos detalhes. Eu estou apenas indicando o local
onde se encon­tram as principais regras e a maioria das normas daquele
preceito, embora haja muitas outras referências relativas a regras
espalhadas em outras partes do Tal­mud, que eu não menciono
especificamente.

Se você examinar todos os preceitos apresentados até agora verá que
alguns são obrigatórios a toda a congregação de Israel, de maneira
coletiva, e não a cada pessoa individualmente, como por exemplo a
construção do Tem­plo, a nomeação de um rei, e a exterminação da semente
de Amalec. Outros são obrigatórios ao indivíduo que realizou um
determinado ato, ou a quem acon­teceu alguma coisa, como por exemplo os
sacrifícios oferecidos por quem pe­cou sem querer, ou um "zab"; e é
possível que um homem não faça e nem lhe ocorra nenhuma dessas coisas em
toda sua vida. Há entre esses preceitos, como explicamos, determinadas
leis, como a do servo hebreu, e da serva he­bréia, a do servo cananeu, a
do depositário não remunerado, a de quem pede emprestado, e outras
mencionadas acima, que pode ser que nunca se apliquem a um determinado
homem, e que pode ser que ele nunca chegue a executar, em toda a sua
vida. Outros preceitos são obrigatórios apenas durante a existên­cia do
Templo, como por exemplo as ofertas dos festivais, o comparecimento
diante do Eterno, e a reunião do povo, que nós apresentamos uma a uma.
Ou­tros são obrigatórios apenas para quem tem bens, como por exemplo os
dízi­mos, os sacrifícios de elevação, os presentes prescritos para o
"Cohen", e par­te para os pobres, tais como as respigas, a gavela
esquecida, "peá", e os cachos de uva imperfeitos; e é possível a um
homem ficar isento deles se ele não tiver bens, e passar a vida toda sem
ser obrigado a realizar nenhum dos preceitos desse tipo. A caridade, no
entanto, não pertence a essa categoria, porque ela é uma obrigação até
mesmo para um homem pobre que vive ele próprio de caridade, como
explicamos. Outros preceitos, ainda, são definitivamente obri-

PRECEITOS POSITIVOS 181

gatórios a todos os homens, em todos os tempos, em qualquer lugar e em
quais­quer circunstâncias, como por exemplo os "tsitsit", os
filactérios, e o Shabat. A esses nós chamamos de preceitos
incondicionais, porque sua obrigatorieda­de recai sobre todo israelita
adulto, sempre, em qualquer lugar, e em quaisquer circunstâncias.

Se você refletir sobre os 248 preceitos positivos, descobrirá que os
preceitos "incondicionais" são 60, desde que a pessoa sobre quem eles
recaiam se encontre nas mesmas circunstâncias que a maioria das pessoas,
ou seja, que more numa casa da cidade, que se alimente como a maioria
das pessoas, ou seja, com pão e carne, que negocie com as outras
pessoas, que se case com uma mulher eprocrie.

Os 60 preceitos positivos, de acordo com a ordem de nossa enume­ração,
são: os preceitos positivos 1, 2, 3, 4, 5, 6, 7, 8 e 9. O 10? não é
obrigató­rio às mulheres, nem o 11? Os 12, 13, 14, 15 e 18 também não
são obrigatórios para as mulheres. Os 19 e 26 são obrigatórios apenas
aos varões "Cohanim". O 32, 54, 73, 94, 143, 146, 147, 149, 150, 152,
154, 155, 156, 157, 158, 159 e 160; o 161 não é obrigatório às mulheres.
O 162, 163, 164, 165, 166 e 167; os 168, 169 e 170 não são obrigatórios
às mulheres. O 172, 175, 184, 195, 197, 206, 207, 208, 209, 210 e 211; o
212 não é obrigatório às mulheres. O 213. Os 214 e 215 só são
obrigatórios aos varões.
\end{quote}

Uma mnemônica para o número de preceitos incondicionais é a se-

\begin{quote}
guinte: "Sessent s rainhas" (Cânticos 6:8), e o número dos não obrigató-

rios para as m ode ser lembrado pela expressão "Que o braço ("yad")

... está se fo ' (Deuteronômio 32:36): a palavra "nashim" (mulher)

perde seu " então o nú s que são obrigatórios às mulheres,

46, pode sei pelo versíc bém para você, pelo sangue ("be-

dam") de seu (Zacarias 9: seja, a palavra "bedam" indica o

número (46) dos preceitos que sã icionalmente obrigatórios às mulhe-\\
res e constituem seu pacto espe íf

Essas são as observações que achamos necessário registrar na enu­meração
dos preceitos positivos.
\end{quote}

\begin{enumerate}
\def\labelenumi{\arabic{enumi}.}
\setcounter{enumi}{230}
\item
  \begin{quote}
  O valor numérico das letras do termo hebraico usado, "yad", é 14.
  \end{quote}
\item
  \begin{quote}
  Isso significa que também ela será redimida pelo mérito do sangue do
  pacto (o preceito da circuncisão). O valor numérico das letras do
  termo hebraico usado aqui, "bedam", é 46.
  \end{quote}
\end{enumerate}

PARTE II

OS 365 PRECEITOS NEGATIVOS

\textbf{TEMA PRECEITOS}

\begin{enumerate}
\def\labelenumi{\arabic{enumi}.}
\item
  \begin{quote}
  A IDOLATRIA (1 a 59)
  \end{quote}
\item
  \begin{quote}
  OS DEVERES PARA COM DEUS (60 a 88)
  \end{quote}
\item
  \begin{quote}
  AS OFERENDAS (89 a 171)
  \end{quote}
\end{enumerate}

\begin{quote}
►. AS PROIBIÇÕES ALIMENTARES (172 a 209)

5. O CULTIVO DA TERRA (210 a 228)
\end{quote}

\begin{longtable}[]{@{}ll@{}}
\toprule
\endhead
\begin{minipage}[t]{0.47\columnwidth}\raggedright
\begin{quote}
6. OS DEVERES PARA COM OS SEMELHANTES, OS POBRES E OS EMPREGADOS

A AUTORIDADE DA CORTE DE JUSTIÇA

8. OS FESTIVAIS

9 AS LEIS DO CASAMENTO'
\end{quote}

10 A ÉTICA DOS GOVERNANTES\strut
\end{minipage} & \begin{minipage}[t]{0.47\columnwidth}\raggedright
(229 a 270)

\begin{quote}
(2\textsuperscript{-}1 a 319) \emph{(320} a 329) (330 a 361) (362 a 365)
\end{quote}\strut
\end{minipage}\tabularnewline
\bottomrule
\end{longtable}

\begin{quote}
ÍNDICE

OS PRECEITOS NEGATIVOS

1 Não crer e nem atribuir divindade a outro que não Ele \emph{195}

2 Não fazer imagens para adorá-las \emph{195}

3 Não fazer um ídolo para que outros o adorem \emph{195}

4 Não fazer figuras de seres humanos \emph{196}

5 Não se curvar diante de um ídolo \emph{196}

6 Não adorar ídolos \emph{197}

7 Não entregar parte de sua descendência a Molekh \emph{198}

8 Não praticar a feitiçaria do "ob" \emph{198}

9 Não praticar a feitiçaria do "yideoni" \emph{199}
\end{quote}

\emph{10} Não estudar as práticas da idolatria \emph{199}

\emph{11} Não erguer um pilar que as pessoas se reunirão para
reverenciar \emph{200}

12 Não esculpir pedras para prostar-se sobre elas \emph{200}

13 Não plantar árvores no Santuário \emph{201}

14 Não jurar por um ídolo \emph{201}

15 Não convocar pessoas para a idolatria \emph{202}

16 Não tentar persuadir um israelita a adorar ídolos \emph{202}

17 Não amar a pessoa que deseja seduzí-lo para a idolatria \emph{202}

18 Não diminuir nossa aversão pelo enganador \emph{203}

19 Não salvar a vida do enganador \emph{203}

20 Não defender um enganador \emph{203}

21 Não omitir uma evidência que seja desfavorável ao enganador
\emph{203}

22 Não tirar proveito de ornamentos que enfeitaram um ídolo \emph{204}

23 Não reconstruir uma cidade apóstata \emph{204}

24 Não tirar proveito dos pertences de uma cidade apóstata \emph{204}

25 Não aumentar nossa fortuna com qualquer coisa que provenha da

\begin{quote}
idolatria \emph{204}
\end{quote}

26 Não fazer profecias em nome de um ídolo \emph{205}

27 Não fazer falsas profecias \emph{205}

28 Não ouvir as profecias de quem profetiza em nome de um ídolo
\emph{206}

29 Não ter piedade de um falso profeta \emph{206}

30 Não adotar os hábitos e costumes dos descrentes \emph{206}

31 Não fazer adivinhações \emph{207}

32 Não orientar nossa conduta pelas estrelas \emph{208}

33 Não praticar a vidência \emph{209}

34 Não praticar feitiçaria \emph{209}

35 Não praticar a arte do encantador \emph{209}

36 Não consultar um necromante que use o "ob" \emph{210}

37 Não consultar um feiticeiro que se utilize do "yidoa" \emph{210}

38 Não tentar obter informações com os mortos \emph{210}

39 As mulheres não devem usar roupas ou adornos masculinos \emph{210}

40 Os homens não devem usar roupas ou adornos femininos \emph{211}

41 Não fazer marcas em nossos corpos \emph{211}

42 Não usar roupas de lã e linho \emph{211}

43 Não raspar os cabelos das têmporas \emph{211}

44 Não raspar a barba \emph{212}

45 Não fazer cortes em nossa carne \emph{213}

46 Não se fixar na terra do Egito \emph{213}

188 MAIMÔNIDES

\begin{quote}
47 Não aceitar opiniões contrarias as ensinadas na Torah \emph{214}

48 Não fazer uma aliança com as Sete Nações Idólatras de Canaà
\emph{214}

49 Nào poupar a vida de um homem das Sete Nações Idólatras \emph{215}

50 Não demonstrar compaixão para com os idólatras \emph{215}

51 Não permitir que idólatras residam em nossa terra \emph{215}

52 Não se unir pelo matrimónio a hereges \emph{216}

53 Não se unir pelo matrimônio a um homem Amonita ou Moabita \emph{216}

54 Não excluir os descendentes de Esaú \emph{217}

55 Não afastar os descendentes dos egípcios \emph{217}

56 Não oferecer a paz a Amon nem a Moab \emph{217}

57 Não destruir árvores frutíferas durante um assédio \emph{217}

58 Não temer os hereges em tempos de guerra \emph{218}

59 Não esquecer o que Amalec nos fez \emph{218}

60 Não blasfemar o grande Nome \emph{218}

61 Não violar um "shebuat bitui" \emph{219}

62 Não fazer um "shebuat shav" \emph{220}

63 Não profanar o nome de Deus \emph{221}

64 Não testar suas promessas e advertências \emph{222}

65 Não demolir casas de adoração ao Eterno \emph{222}

66 Não deixar o corpo de um criminoso pendurado durante toda a noi-

te após sua execução \emph{222}

67 Não interromper a vigilância do Santuário \emph{223}

68 O "Cohen Gadol" não deve entrar no Santuário em outras ocasiões além
das estabelecidas \emph{223}

\emph{69} Um "Cohen" com um defeito não deve entrar em nenhuma parte

do Santuário \emph{224}

70 Um "Cohen" com um defeito não deve ministrar no Santuário \emph{224}

71 Um "Cohen" com um defeito temporário não deve ministrar no San­tuário
\emph{224}
\end{quote}

72 Os Levitas e os "Cohanim" não devem realizar as tarefas uns dos
outros \emph{225}

\begin{quote}
73 Não entrar no Santuário nem pronunciar uma sentença sobre uma

lei da Torah estando intoxicado \emph{226}

74 Um "zar" não deve oficiar no Santuário \emph{226}

75 Um "Cohen" impuro não deve oficiar no Santuário \emph{227}

.76 Um "Cohen" que praticou um "Tebul yom" não deve oficiar no Santuário
\emph{227}
\end{quote}

77 Uma pessoa impura não pode entrar em nenhuma parte do Santuário
\emph{228}

78 Uma pessoa impura não pode entrar no acampamento dos Levitas
\emph{229}

79 Não construir um altar com pedras que tenham sido tocadas por ferro
\emph{229}

\begin{quote}
80 Não subir ao altar por degraus \emph{229}

81 Não apagar o fogo do altar \emph{230}

82 Não oferecer nenhum tipo de sacrifício sobre o altar de ouro
\emph{230}

83 Não fazer óleo igúal ao óleo de unção \emph{230}

84 Não ungir ninguém a não ser os "Cohanim Guedolim" e os reis com

o óleo de unção preparado por Moisés \emph{231}

85 Não fazer incenso igual ao usado no Santuário \emph{231}

86 Não retirar as varas das argolas da Arca \emph{231}

87 Não desprender o peitoral do "efod" \emph{231}

88 Não rasgar a orla do manto do "Cohen Gadol" \emph{232}

89 Não oferecer nenhum sacrifício fora do campo dó Santuário \emph{232}

90 Não degolar nenhum dos sacrifícios sagrados fora do campo do
San­tuário \emph{233}
\end{quote}

91 Não destinar animais defeituosos para serem oferecidos sobre o altar
\emph{234}

\begin{quote}
92 Não degolar animais defeituosos para oferecè-los como sacrifício
\emph{234}

93 Não aspergir o sangue de animais defeituosos sobre o altar \emph{235}

94 Não queimar as partes de sacrifício de um animal defeituoso sobre

o altar \emph{235}

95 Não sacrificar um animal com um defeito temporário \emph{236}

96 Não oferecer sacrifícios defeituosos de um gentio.... \emph{237}

97 Não fazer com que uma oferta se torne defeituosa \emph{.237}

98 Não oferecer fermento ou mel sobre o altar \emph{237}

99 Não oferecer um sacrifício sem sal \emph{23 7}
\end{quote}

\emph{100} Não oferecer no altar o salário de uma rameira ou o preço de
um cão \emph{238}

101 Não degolar a mãe e seu filhote no mesmo dia \emph{238}

102 Não colocar azeite de oliva sobre a oblação de um pecador \emph{238}

103 Não levar incenso junto com a oblação de um pecador \emph{238}

104 Não misturar azeite de oliva com a oblação de uma mulher suspeita

\begin{quote}
de adultério \emph{239}

105 Não colocar incenso sobre a oblação de uma mulher suspeita de
adul­tério \emph{239}
\end{quote}

106 Não trocar um animal que tenha sido consagrado como oferenda
\emph{239}

107 Não trocar uma oferenda sagrada por outra \emph{240}

108 Não resgatar o primogênito de um animal puro \emph{240}

109 Não vender o dízimo do gado ............. \emph{240}

110 Não vender uma propriedade consagrada \emph{240}

\emph{1 11} Não resgatar terra que tenha sido consagrada sem nenhuma
declara-

\begin{quote}
ção específica de finalidade \emph{241}
\end{quote}

112 Não cortar a cabeça do pássaro de um Sacrifício de Pecado durante

\begin{quote}
a "Meliká" \emph{241}
\end{quote}

113 Não fazer qualquer trabalho com um animal consagrado \emph{242}

114 Não tosquiar um animal consagrado \emph{242}

\emph{115} Não degolar o sacrifício de ''Pessah\textsuperscript{-}
enquanto tivermos pão leveda-

\begin{quote}
do em nosso poder \emph{242}
\end{quote}

116 Não deixar as partes do sacrifício da oferenda de "Pessah" de um

\begin{quote}
dia para o outro \emph{243}
\end{quote}

117 Não deixar ficar nenhuma parte da carne da oferenda de "Pessah"

\begin{quote}
até a manhã seguinte \emph{243}
\end{quote}

118 Não deixar sobrar carne do sacrifício do Festival do décimo quarto

\begin{quote}
do "Nissan" até o terceiro dia \emph{243}
\end{quote}

119 Mão .deixar sobrar carne do segundo sacrifício de "Pessah" até a ma-

\begin{quote}
nhã seguinte \emph{244}
\end{quote}

120 Não deixar sobrar carne do Sacríficio de Graças até a manhã seguinte
\emph{244}

121 Não quebrar nenhum osso do sacrifício de "Pessah" \emph{244}

122 Não quebrar nenhum osso do segundo sacrifício de "Pessah" \emph{245}

123 Não retirar o sacrifício de "Pessah" do lugar onde ele é comido
\emph{245}

124 Não cozer as sobras de uma oblação de cereal com levedo \emph{245}

125 Não comer o sacrifício de "Pessah" cozido nèm cru \emph{246}

126 Não permitir que um "guer toshab" coma do sacrifício de "Pessah"
\emph{246}

127 Uma pessoa incircuncisa não deve comer do sacrifício de "Pessah"
\emph{246}

128 Não permitir que um israelita apóstata coma do sacrifício de
"Pessah" \emph{246}

129 Uma pessoa impura não deve comer comida santificada \emph{247}

130 Não comer carne de sacrifícios consagrados que se tornaram impuros
\emph{247}

131 Não comer "notar" \emph{248}

132 Não comer "pigul" \emph{248}

190 MAIMÔNIDES

133 Um "zar" nao deve comer 'terumá" \emph{249}

\emph{134} Um servo ou um criado de um "Cohen" não deve comer "terumá"
\emph{250}

\emph{135} Um "Cohen': incircunciso não deve comer "terumá" \emph{250}

\emph{136} Um "Cohen" impuro não deve comer "terumá" \emph{251}

137 Uma "halalá" não deve comer alimento sagrado \emph{251}

138 Não comer a oblação de um "Cohen" \emph{252}

139 Não comer carne de sacrifícios de pecado cujo sangue tenha sido

\begin{quote}
levado para dentro do Santuário \emph{252}
\end{quote}

140 Não comer sacrifícios consagrados que tenham sido invalidados
\emph{252}

141 Não comer o segundo dízimo de cereais não remido fora de Jerusalém
\emph{253}

\begin{quote}
142 Não consumir o segundo dízimo de vinho não remido fora de Jeru­salém
\emph{253}
\end{quote}

\emph{143} Não consumir o segundo dízimo de azeite não remido fora de
Jerusalém \emph{253}

144 Não comer um primogênito sem defeito fora de Jerusalém \emph{254}

145 Não comer o Sacrifício de Pecado e o Sacrifício de Delito fora do

\begin{quote}
campo do Santuário \emph{254}
\end{quote}

146 Não comer a carne de um Holocausto \emph{255}

147 Não comer sacrifícios menos sagrados antes de aspergir seu sangue

\begin{quote}
sobre o altar \emph{255}
\end{quote}

148 Um "Cohen" não pode comer as primícias fora de Jerusalém \emph{256}

149 Um "zar" não pode comer os sacrifícios mais sagrados \emph{257}

150 Não comer o segundo dízimo impuro não remido, nem mesmo em

\begin{quote}
Jerusalém \emph{257}
\end{quote}

151 Não comer o segundo dízimo durante o período de luto \emph{257}

152 Não gastar o dinheiro do resgate do segundo dízimo a não ser com

\begin{quote}
comida e bebida \emph{258}
\end{quote}

153 Não comer "tebel" \emph{258}

154 Não alterar a ordem prescrita para separar o dízimo da Colheita
\emph{259}

155 Não adiar o pagamento de promessas \emph{260}

156 Não comparecer a um festival sem um sacrifício \emph{260}

157 Não deixar de cumprir uma obrigação oral, mesmo que não se tenha

\begin{quote}
feito um juramento \emph{260}
\end{quote}

158 Um "Cohen" não pode casar-se com uma "zoná" \emph{261}

159 Um "Cohen" não pode casar-se com uma "halalá" \emph{261}

160 Um "Cohen" não pode casar-se com uma mulher divorciada \emph{261}

161 Um "Cohen Gadol" não pode casar-se com uma viúva \emph{262}

162 Um "Cohen Gadol" não pode chegar-se a uma viúva \emph{263}

\emph{163} Um "Cohen" não pode entrar no Santuário com o cabelo solto
\emph{264}

\emph{164} Os "Cohanim" não podem usar vestes rasgadas ao entrar no
Santuário \emph{264}

\begin{quote}
\emph{165} Os "Cohanim" não podem sair do Santuário enquanto estiverem
ofi­ciando , \emph{265}
\end{quote}

166 Um "Cohen" comum não pode tornar-se impuro por nenhuma pes-

\begin{quote}
soa morta a não ser pelas que estão determinadas na Torah \emph{266}
\end{quote}

167 Um "Cohen Gadol" não deve ficar sob o mesmo teto que um cadáver
\emph{267}

168 Um "Cohen Gadol" não pode fazer-se impuro por nenhuma pessoa

\begin{quote}
morta \emph{267}
\end{quote}

\emph{169} Os Levitas não podem adquirir um pedaço da Terra de Israel
\emph{267}

\begin{quote}
170 Os Levitas não podem receber nenhuma parte da pilhagem da con­quista
da Terra de Israel \emph{268}
\end{quote}

171 Não arrancar nosso cabelo pelos mortos , \emph{269}

172 Não comer um animal impuro \emph{270}

173 Não comer um peixe impuro \emph{271}

PRECEITOS NEGATIVOS 191

174 Não comer nenhuma ave impura \emph{271}

175 Não comer nenhum inseto alado \emph{271}

176 Não comer nada que rasteje sobre a terra \emph{271}

\begin{quote}
\emph{177} Não comer nenhuma criatura rastejante que se reproduza em
maté- ' ria deteriorada \emph{271}
\end{quote}

178 Não comer criaturas vivas que se reproduzam em sementes ou frutas
\emph{272}

179 Não comer nenhuma espécie de criatura rastejante \emph{272}

180 Não comer "nebelá" \emph{275}

181 Não comer "terefá" \emph{275}

182 Não comer um membro de um animal vivo \emph{276}

183 Não comer "guid hanashé" \emph{277}

184 Não comer sangue \emph{277}

185 Não comer gordura de um animal puro \emph{277}

186 Não cozinhar carne no leite \emph{277}

187 Não comer carne cozida em leite \emph{278}

188 Não comer a carne de um boi apedrejado \emph{279}

189 Não comer pão feito com grãos da nova ceifa \emph{280}

190 Não comer grãos da nova ceifa torrados \emph{280}

191 Não comer grãos verdes de cereais \emph{280}

192 Não comer "orlá" \emph{280}

193 Não comer "quil-ei ha querem" \emph{281}

194 Não beber "yain nessech" \emph{281}

195 Não comer nem beber em excesso \emph{282}

196 Não comer durante um "Yom Quipur" \emph{283}

197 Não comer "hametz" durante "Pessah" \emph{283}

198 Não comer nada que contenha "hametz" durante "Pessah" \emph{284}

\begin{quote}
199 Não comer "hametz" depois da metade do décimo quarto de "Nissan"
\emph{284}
\end{quote}

200 Não pode ser visto "hametz" em nossas moradias durante "Pessah"
\emph{285}

201 Não possuir "hametz" durante "Pessah" \emph{285}

202 Um "Nazir" não pode beber vinho 285

203 Um "Nazir" não pode comer uvas frescas \emph{286}

204 Um "Nazir" não pode comer uvas secas \emph{286}

205 Um "Nazir" não pode comer caroços de uvas \emph{286}

206 Um "Nazir" não pode comer bagaços de uvas \emph{286}

207 Um "Nazir" não pode fazer-se impuro pelos mortos \emph{287}

208 Um "Nazir" não pode fazer-se impuro entrando numa casa onde ha-

\begin{quote}
ja um morto \emph{287}
\end{quote}

209 Um "Nazir" não pode raspar a cabeça \emph{288}

210 Não ceifar toda a colheita \emph{288}

211 Não recolher as espigas de cereais que caíram durante a colheita
\emph{288}

212 Não recolher todo o produto do vinhedo na época da vindima
\emph{288}

213 Não recolher os bagos das uvas que caírem durante a colheita
\emph{289}

214 Não voltar para buscar uma gavela esquecida \emph{289}

215 Não semear "quil-aim" \emph{290}

216 Não semear grãos nem vegetais num vinhedo \emph{291}

217 Não cruzar animais de espécies diferentes \emph{291}

218 Não trabalhar com duas espécies diferentes de animais juntos
\emph{291}

219 Não impedir um animal de comer do produto no meio do qual ele

\begin{quote}
esteja trabalhando \emph{292}
\end{quote}

220 Não cultivar o solo no sétimo ano \emph{292}

221 Não podar árvores no sétimo ano \emph{293}

192 MAIMÓNIDES

222 Não ceifar uma planta que nasceu por si só no sétimo ano da manei-

\begin{quote}
ra como se faz num ano comum \emph{293}
\end{quote}

223 Não colher uma fruta que tenha crescido por si só no sétimo ano,

\begin{quote}
da mesma maneira como se faz num ano comum \emph{293}
\end{quote}

224 Não cultivar o solo no Ano do Jubileu \emph{294}

22 5 Não ceifar a produção tardia do Ano do Jubileu da maneira como

\begin{quote}
se faz num ano comum \emph{294}
\end{quote}

226 Não colher frutas no Ano do Jubileu da maneira como se faz num

\begin{quote}
ano comum \emph{294}
\end{quote}

227 Não vender definitivamente nossas terras em Israel \emph{295}

228 Não vender as terras dos arredores dos Levitas \emph{295}

229 Não abandonar os Levitas \emph{295}

230 Não 'cobrar as dívidas depois do Ano de Shabat \emph{295}

231 Não recusar um empréstimo que deva ser cancelado no Ano de Shabat
\emph{296}

232 Não deixar de fazer caridade a nossos irmãos necessitados \emph{296}

233 Não mandar embora de mãos vazias um servo hebreu \emph{296}

234 Não cobrar uma dívida de alguém que se sabe que não pode pagar
\emph{297}

235 Não emprestar a juros \emph{297}

236 Não tomar emprestado com juros \emph{298}

237 Não participar de um empréstimo a juros \emph{298}

238 Não oprimir um empregado atrasando o pagamento de seus salários
\emph{299}

239 Não tomar pela força um penhor de um devedor \emph{'299}

240 Não ficar com um penhor do qual seu proprietário precise \emph{300}

241 Não pegar um penhor de uma viúva \emph{300}

242 Não pegar como penhor utensílios usados para a alimentação
\emph{300}

243 Não raptar um israelita \emph{301}

244 Não furtar dinheiro \emph{302}

245 Não cometer um roubo \emph{302}

246 Não alterar os limites das terras fraudulentamente \emph{303}

247 Não usurpar nossas dívidas \emph{303}

248 Não negar nossas dívidas \emph{304}

249 Não jurar em falso ao negar uma dívida \emph{304}

250 Não enganar um ao outro em negócios \emph{304}

251 Não prejudicar um ao outro com palavras \emph{305}

252 Não enganar um prosélito com palavras \emph{305}

253 Não enganar um prosélito nos negócios \emph{305}

254 Não entregar um escravo fugitivo \emph{306}

255 Não enganar um escravo fugitivo \emph{306}

256 Não ser rude com crianças órfãs e com viúvas \emph{306}

2 57 Não utilizar um servo hebreu para executar tarefas degradantes
\emph{307}

258 Não vender um servo hebreu em leilão \emph{307}

259 Não utilizar um servo hebreu para fazer um trabalho desnecessário
\emph{308}

260 Não permitir que se maltrate um servo hebreu \emph{308}

261 Não vender uma serva hebréia \emph{308}

262 Não privar uma serva hebréia que se desposou \emph{309}

263 Não vender uma prisioneira \emph{309}

264 Não escravizar uma prisioneira \emph{309}

265 Não planejar obter a propriedade de outrem \emph{309}

266 Não cobiçar os pertences de outrem \emph{310}

267 Um trabalhador contratado não pode comer das plantações em cres-

\begin{quote}
cimento \emph{310}
\end{quote}

268 Um trabalhador contratado não pode servir-se em demasia \emph{311}

PRECEITOS NEGATIVOS 193

269 Não ignorar uma propriedade perdida \emph{311}

270 Não abandonar uma pessoa sobrecarregada \emph{311}

271 Não trapacear nas medidas e nos pesos \emph{312}

272 Não manter pesos e medidas incorretos \emph{312}

273 Um juiz não pode cometer injustiças \emph{313}

274 Um juiz não pode aceitar presentes de uma das partes \emph{313}

275 Um juiz não pode proteger uma das partes \emph{313}

\begin{quote}
276 Um juiz não pode acovardar-se com medo de pronunciar um julga­mento
justo \emph{314}
\end{quote}

277 Um juiz não pode decidir em favor de um homem pobre por piedade
\emph{314}

278 Um juiz não pode distorcer um julgamento contra uma pessoa de

\begin{quote}
má reputação \emph{314}
\end{quote}

279 Um juiz não pode ter piedade de alguém que matou um homem \emph{315}

280 Um juiz não pode distorcer a justiça por prosélitos ou órfãos
\emph{315}

281 O juiz não pode ouvir uma das partes na ausência da outra \emph{315}

282 Um tribunal não pode condenar por maioria de um num caso capital 316

283 Um juiz não pode confiar na opinião de outro juiz \emph{316}

284 Não designar um juiz inculto \emph{317}

285 Não prestar um falso testemunho \emph{317}

286 Um juiz não pode aceitar o testemunho de um homem mau \emph{317}

287 Um juiz não pode aceitar o testemunho de um parente de uma dis
partes \emph{318}

288 Não condenar baseado no depoimento de uma única testemunha
\emph{318}

289 Não matar um ser humano \emph{318}

290 Não punir com a pena capital baseando-se em provas circunstanciais
\emph{319}

291 Uma testemunha não pode atuar como advogado \emph{320}

292 Não matar um assassino sem julgamento \emph{320}

293 Não poupar a vida de um perseguidor \emph{320}

294 Não punir uma pessoa por um pecado cometido sob coação \emph{321}

295 Não aceitar um resgate de alguém que tenha cometido um assassina-

\begin{quote}
to deliberadamente \emph{321}
\end{quote}

296 Não aceitar um resgate de alguém que tenha cometido um assassina-

\begin{quote}
to involuntariamente \emph{322}
\end{quote}

297 Não se descuidar de salvar um israelita em perigo de vida \emph{322}

298 Não deixar obstáculos em propriedades públicas ou privadas
\emph{323}

299 Não dar um conselho enganoso \emph{323}

300 Não infligir castigo corporal excessivo \emph{323}

3Ó1 Não bisbilhotar \emph{324}

302 Não odiar uns aos outros \emph{324}

303 Não envergonhar ninguém \emph{324}

304 Não se vingar um do outro \emph{325}

305 Não guardar rancor \emph{325}

306 Não pegar o ninho todo de um pássaro. \emph{325}

307 Não raspar a tinha \emph{326}

308 Não cortar ou cauterizar marcas de lepra \emph{326}

309 Não lavrar um vale no qual tenha sido realizado o ritual de "Eglá
Arufá" \emph{326}

310 Não deixar viver um feiticeiro \emph{326}

311 Não levar um recém-casado para longe de sua casa \emph{327}

312 Não discordar das autoridades tradicionais \emph{327}

313 Não fazer acréscimos à lei escrita ou oral \emph{327}

314 Não fazer diminuições na lei escrita ou oral \emph{328}

315 Não maldizer um juiz \emph{328}

316 Não maldizer um chefe \emph{328}

194 MAIMÔNIDES

317 Não maldizer um israelita \emph{328}

318 Não amaldiçoar os pais \emph{330}

319 Não ferir seus pais \emph{330}

320 Não trabalhar no Shabat \emph{331}

32 1 Não viajar no Shabat \emph{331}

322 Não castigar durante o Shabat \emph{331}

323 Não trabalhar no primeiro dia de "Pessah" \emph{332}

324 Não trabalhar no sétimo dia de "Pessah" \emph{332}

325 Não trabalhar em "Atzeret" \emph{332}

326 Não trabalhar em "Rosh Hashaná" \emph{332}

327 Não trabalhar no primeiro dia de "Sucot" \emph{332}

328 Não trabalhar em "Shemini Atzeret" \emph{332}

329 Não trabalhar em "Yom Quipur" \emph{333}

330 Não cometer incesto com sua mãe \emph{333}

331 Não cometer incesto com a esposa de seu pai \emph{333}

332 Não cometer incesto com sua irmã \emph{334}

333 Não cometer incesto com a filha da esposa de seu pai se ela for sua

\begin{quote}
irmã \emph{334}
\end{quote}

334 Não cometer incesto com a filha de seu filho \emph{334}

335 Não cometer incesto com a filha de sua filha \emph{334}

336 Não cometer incesto com sua filha \emph{335}

337 Não chegar-se a uma mulher e a sua filha 336

338 Não chegar-se a uma mulher e à filha do filho dela 336

339 Não chegar-se a uma mulher e à filha da filha dela 336

340 Não cometer incesto com a irmã de seu pai \emph{336}

341 Não cometer incesto com a irmã de sua mãe \emph{336}

342 Não chegar-se à esposa do irmão de seu pai \emph{337}

343 Não chegar-se à esposa de seu filho \emph{337}

344 Não chegar-se à esposa de seu irmão \emph{337}

345 Não chegar-se à irmã de sua esposa enquanto esta última for viva 337

346 Não unir-se a uma mulher menstruada \emph{337}

347 Não chegar-se à esposa de outro homem \emph{338}

348 Os homens não podem deitar-se com animais 339

349 As mulheres não podem deitar-se com animais \emph{339}

350 Um homem não pode chegar-se a outro homem 339

351 Um homem não pode chegar-se a seu pai \emph{340}

352 Um homem não pode chegar-se ao irmão de seu pai \emph{340}

353 Não ter intimidades com uma parenta \emph{342}

354 Um "mamzer" não pode chegar-se a uma israelita \emph{343}

355- Não chegar-se a uma mulher antes do casamento \emph{343}

356 Não tornar a casar-se com a esposa de quem se divorciou, depois

\begin{quote}
que ela tenha se casado novamente \emph{344}
\end{quote}

357 Não chegar-se a uma mulher sujeita ao casamento levirato 344

358 Não se divorciar da mulher que se violentou e com a qual se foi
obri-

\begin{quote}
gado a casar \emph{345}
\end{quote}

359 Não se divorciar de uma mulher depois de tê-la caluniado \emph{345}

360 Um homem incapaz de procriar não pode casar-se com uma israelita
\emph{346}

361 Não castrar \emph{346}

362 Não nomear um rei que não seja israelita de nascimento \emph{346}

363 Um rei não pode possuir muitos cavalos \emph{347}

\emph{364} Um rei não pode ter muitas esposas \emph{347}

365 Um rei não pode acumular grande fortuna pessoal \emph{348}

\begin{quote}
\textbf{1 NÃO CRER E NEM ATRIBUIR DIVINDADE A OUTRO QUE NÃO ELE}
\end{quote}

Por esta proibição somos proibidos de crer em ou atribuir divinda­de a
outro que não Ele, enaltecido seja. Ela está expressa em Suas palavras
---embora não se possa atribuir palavras a Seu Ser
transcendental\textsuperscript{233} --- "Não te­rás outros deuses diante
de Mim" (Êxodo 20:3).

Foi deixado claro no final de Macot que esta proibição é um dos 613
preceitos, pois está dito ali: "Seiscentos e treze preceitos foram dados
a Moisés no Sinai etc.". Nós já explicamos este assunto no primeiro
preceito positivo.

\begin{quote}
\textbf{2} NÃO FAZER IMAGENS PARA ADORÁ-LAS
\end{quote}

Por esta proibição somos proibidos de fazer imagens que venham a ser
adoradas, e não há diferença entre fazê-las nós mesmos ou mandar que
outros as façam. Esta proibição está expressa em Suas palavras,
enaltecido seja Ele, "Não farás para ti imagem de escultura etc." (Êxodo
20:4).

Todo aquele que transgredir este preceito negativo estará sujeito ao
açoitamento, seja por ter feito o ídolo ou por ter mandado que outra
pessoa o faça, mesmo que ele não o adore.

\begin{quote}
3 \textbf{NÃO FAZER UM ÍDOLO PARA QUE OUTROS O ADOREM}
\end{quote}

Por esta proibição somos proibidos de fazer dm ídolo, mesmo que seja
para que outros o adorem, e ainda que a pedido de um idólatra. Ela está

233. Já que Deus não é um corpo, e que portanto não se pode atribuir a
Ele a "fala" física.

196 MAIMÔNIDES

expressa em Suas palavras, enaltecido seja Ele, "Nem fareis vós ídolos
para vós mesmos" (Levítico 19:4), a respeito das quais diz a Sifrá: "
'Nem fareis vós': ainda que seja para os outros".

Também está dito ali: "Aquele que fizer um ídolo para uso próprio
transgride dois preceitos negativos". Ou seja, ele transgride a
proibição de fazê-lo ele próprio, ainda que para uso de outros, que está
contida neste terceiro pre­ceito, e também a proibição de adquirir um
ídolo e de guardá-lo, ainda que ou­tra pessoa o tenha feito para ele, e
que está contida no preceito precedente. Dessa forma ele está sujeito a
ser açoitado duas vezes.

As normas deste preceito e do precedente estão explicadas no Tra­tado
Abodá Zará.

\begin{quote}
4 NÃO FAZER FIGURAS DE SERES HUMANOS
\end{quote}

Por esta proibição somos proibidos de fazer figuras de seres huma­nos de
metal, pedra, madeira e similares, ainda que elas não sejam feitas para
serem adoradas. A finalidade disso é impedir-nos de fazer qualquer
imagem pa­ra que não pensemos, como fazem as massas, que elas possuem
poderes sobre­naturais. Esta proibição está expressa em Suas palavras,
enaltecido seja Ele, "Não fareis diante de Mim, deuses de prata nem
deuses de ouro para vós" (Exodo 20:23).

Para explicar esta proibição a Mekhiltá diz: " 'Não fareis ... deuses de
prata'; a fim de que você não diga: Vou fazê-los apenas como enfeites,
como outros fazem em vários outros países, as Escrituras dizem 'Não
fareis... para vós' ".

A transgressão deste preceito negativo é punida com o açoitamento.

As normas deste preceito --- que figuras nos é permitido ou proibi­do
confeccionar, e de que forma, e assim por diante --- estão explicadas no
ter­ceiro capítulo de Abodá Zará.

Está explicado em Sanhedrin que este preceito negativo --- ou seja, Suas
palavras "Não fareis diante de Mim, deuses de prata etc" --- também
abrange outros aspectos que vão além do limite destes preceitos; mas o
sentido literal deste versículo é o que expusemos, como está explicado
na Makhiltá.

NÃO SE CURVAR DIANTE DE UM ÍDOLO

Por esta proibição somos proibidos de curvar-nos diante de um ído­lo, e
está claro que o termo "ídolo" significa qualquer outro objeto de
adora­ção que não o Eterno. Esta proibição está expressa em Suas
palavras, enalteci­do seja Ele, "Não te prostarás diante deles, nem os
servirás" (Êxodo 20:5). A intenção não é de proibir unicamente o ato de
curvar-se, excluindo os outros; foi mencionada apenas uma forma de
adoração, que é a prostração, mas esta­mos da mesma forma proibidos de
oferecer sacrifícios e de queimar incenso diante de um ídolo; todo
aquele que fizer uma dessas coisas proibidas, isto é, que se curvar,
oferecer sacrifícios, oferecer uma libação ou queimar incenso estará
sujeito ao apedrejamento.

A Mekhiltá diz: " 'Aquele que sacrificar aos deuses, será morto' (Êxo­do
22:19). Ouvimos, dessa forma, a penalidade, mas não ouvimos a
advertên­cia. Por isso as Escrituras dizem: 'Não te prostarás diante
deles, nem os servirás'.

• PRECEITOS NEGATIVOS 197

\begin{quote}
Oferecer sacrifícios, que já está incluído\textsuperscript{234}, está
destacado aqui para ensinar-nos a seguinte lição: O sacrifício, que é
algo que se realiza em sinal de adoração a Deus, é um pecado, quer seja
ele normalmente adorado dessa forma quer não; portanto também no caso de
qualquer outro ato executado a serviço de Deus, é um pecado, quer seja
ele normalmente adorado dessa maneira, quer não". O significado dessas
palavras é que todo aquele que realizar diante de um ídolo qualquer um
desses quatro atos de adoração --- a saber, curvar-se, oferecer
sa­crifícios, queimar incenso e verter uma libação --- que nós temos a
obrigação de realizar a serviço de Deus --- está sujeito ao
apedrejamento, mesmo que o ídolo não seja normalmente adorado daquela
maneira. O que se quer dizer pela expressão "não seja normalmente
adorado" é o seguinte: embora alguém não tenha adorado o ídolo da
maneira pela qual ele é costumeiramente adorado, adorá-lo de uma das
formas mencionadas faz com que ele fique sujeito ao ape­drejamento, se
ele tiver pecado voluntariamente, e à extinção, se seu pecado não tiver
sido testemunhado ou se ele não tiver sido punido por isso. Contudo, se
o pecado foi cometido involuntariamente, o pecador deve oferecer um
Sa­crifício Determinado de Pecado. Isto também se aplica a quem deificar
um ob­jeto qualquer.

Esta proibição --- ou seja, a proibição de prestar homenagem a um ídolo
através de qualquer uma dessas quatro formas de adoração, mesmo que o
ídolo não seja normalmente adorado assim --- está repetida em Suas
palavras, enaltecido seja Ele, "E não oferecerão mais seus sacrifícios
aos `se-irim' " (Leví­tico 17:7), a respeito das quais a Sifrá diz: "
Se-irim' significa demônios".

Na Guemará de Pessahim está explicado que esta proibição se aplica em
especial ao caso de alguém que tenha abatido uma oferenda para um ídolo,
mesmo que ele não seja normalmente adorado dessa forma: "Como sabemos
que se alguém sacrificar um animal a `Merkulis' ele está sujeito a ser
punido? Por­que está escrito 'Não oferecerão mais seus sacrifícios aos
demônios'. Uma vez que há uma redundância no que se refere às formas
comuns de adoração, que aparece no versículo 'De que modo serviam estas
nações a seus deuses' (Deute­ronômio 12:30) deve-se considerar este como
referindo-se a formas não comuns de adoração". Dessa maneira, a
transgressão voluntária desta proibição é puni­da com ambas a extinção e
o apedrejamento, como explicado acima, e aquele que a transgredir
involuntariamente deve oferecer um sacrifício. As palavras das
Escrituras relativas a isto são: "Aquele que sacrificar aos deuses será
morto".
\end{quote}

As normas deste preceito estão explicadas no sétimo capítulo de

\begin{quote}
Sanhedrin.

6 NÃO ADORAR ÍDOLOS

Por esta proibição somos proibidos de adorar ídolós ainda que de outras
formas além das quatro especificadas antes, desde que aquele ídolo
es­pecífico seja normalmente adorado dessa maneira, como por exemplo,
evacuar para Peor ou jogar uma pedra para Merkulis. Esta proibição está
expressa em Suas palavras, enaltecido seja Ele, "Nem os servirás" (Êxodo
20:5), a respeito das quais a Mekhiltá diz: " 'Não te prostrarás diante
deles, nem os servirás': aqui há dois pecados separados e independentes
--- oferecer um sacrifício e prostrar-se". Dessa forma, aquele que
atirar uma pedra para Peor ou evacuar para Mer-

234. Na advertência "nem os servirás".
\end{quote}
