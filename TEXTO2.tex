\setcounter{secnumdepth}{-2}

\part{\textsc{os 365 preceitos negativos}}


%TEMA PRECEITOS
%
%1.A IDOLATRIA (1 a 59)
%
%2. OS DEVERES PARA COM DEUS (60 a 88)
%
%3. AS OFERENDAS (89 a 171)
%
%4. AS PROIBIÇÕES ALIMENTARES (172 a 209)
%
%5. O CULTIVO DA TERRA (210 a 228)
%
%6. OS DEVERES PARA COM OS SEMELHANTES, OS POBRES E
%
%OS EMPREGADOS (229 a 270)
%
%7. A AUTORIDADE DA CORTE
%
%DE JUSTIÇA (271 a 319)
%
%8. OS FESTIVAIS (320 a 329)
%
%9. AS LEIS DO CASAMENTO (330 a 361)
%
%10. A ÉTICA DOS GOVERNANTES (362 a 365)
%
%\begin{quote}
%ÍNDICE
%
%OS PRECEITOS NEGATIVOS
%
%1 Não crer nem atribuir divindade a outro que não Ele \emph{195}
%
%2 Não fazer imagens para adorá-las \emph{195}
%
%3 Não fazer um ídolo para que outros o adorem \emph{195}
%
%4 Não fazer figuras de seres humanos \emph{196}
%
%5 Não se curvar diante de um ídolo \emph{196}
%
%6 Não adorar ídolos \emph{197}
%
%7 Não entregar parte de sua descendência a Molekh \emph{198}
%
%8 Não praticar a feitiçaria do ``ob'' \emph{198}
%
%9 Não praticar a feitiçaria do ``yideoni'' \emph{199}
%\end{quote}
%
%10 Não estudar as práticas da idolatria \emph{199}
%
%11 Não erguer um pilar que as pessoas se reunirão para reverenciar
%\emph{200}
%
%12 Não esculpir pedras para prostar-se sobre elas \emph{200}
%
%13 Não plantar árvores no Santuário \emph{201}
%
%14 Não jurar por um ídolo \emph{201}
%
%15 Não convocar pessoas para a idolatria \emph{202}
%
%16 Não tentar persuadir um israelita a adorar ídolos \emph{202}
%
%17 Não amar a pessoa que deseja seduzi-lo para a idolatria \emph{202}
%
%18 Não diminuir nossa aversão pelo enganador \emph{203}
%
%19 Não salvar a vida do enganador \emph{203}
%
%20 Não defender um enganador \emph{203}
%
%21 Não omitir uma evidência que seja desfavorável ao enganador
%\emph{203}
%
%22 Não tirar proveito de ornamentos que enfeitaram um ídolo \emph{204}
%
%23 Não reconstruir uma cidade apóstata \emph{204}
%
%24 Não tirar proveito dos pertences de uma cidade apóstata \emph{204}
%
%25 Não aumentar nossa fortuna com qualquer coisa que provenha da
%
%\begin{quote}
%idolatria \emph{204}
%\end{quote}
%
%26 Não fazer profecias em nome de um ídolo \emph{205}
%
%27 Não fazer falsas profecias \emph{205}
%
%28 Não ouvir as profecias de quem profetiza em nome de um ídolo
%\emph{206}
%
%29 Não ter piedade de um falso profeta \emph{206}
%
%30 Não adotar os hábitos e costumes dos descrentes \emph{206}
%
%31 Não fazer adivinhações \emph{207}
%
%32 Não orientar nossa conduta pelas estrelas \emph{208}
%
%33 Não praticar a vidência \emph{209}
%
%34 Não praticar feitiçaria \emph{209}
%
%35 Não praticar a arte do encantador \emph{209}
%
%36 Não consultar um necromante que use o ``ob'' \emph{210}
%
%37 Não consultar um feiticeiro que se utilize do ``yidoa'' \emph{210}
%
%38 Não tentar obter informações com os mortos \emph{210}
%
%39 As mulheres não devem usar roupas ou adornos masculinos \emph{210}
%
%40 Os homens não devem usar roupas ou adornos femininos \emph{211}
%
%41 Não fazer marcas em nossos corpos \emph{211}
%
%42 Não usar roupas de lã e linho \emph{211}
%
%43 Não raspar os cabelos das têmporas \emph{211}
%
%44 Não raspar a barba \emph{212}
%
%45 Não fazer cortes em nossa carne \emph{213}
%
%46 Não se fixar na terra do Egito \emph{213}
%
%188 MAIMÔNIDES
%
%\begin{quote}
%47 Não aceitar opiniões contrárias às ensinadas na Torah \emph{214}
%
%48 Não fazer uma aliança com as Sete Nações Idólatras de Canaã
%\emph{214}
%
%49 Não poupar a vida de um homem das Sete Nações Idólatras \emph{215}
%
%50 Não demonstrar compaixão para com os idólatras \emph{215}
%
%51 Não permitir que idólatras residam em nossa terra \emph{215}
%
%52 Não se unir pelo matrimônio a hereges \emph{216}
%
%53 Não se unir pelo matrimônio a um homem Amonita ou Moabita \emph{216}
%
%54 Não excluir os descendentes de Esaú \emph{217}
%
%55 Não afastar os descendentes dos egípcios \emph{217}
%
%56 Não oferecer a paz a Amon nem a Moab \emph{217}
%
%57 Não destruir árvores frutíferas durante um assédio \emph{217}
%
%58 Não temer os hereges em tempos de guerra \emph{218}
%
%59 Não esquecer o que Amalec nos fez \emph{218}
%
%60 Não blasfemar o grande Nome \emph{218}
%
%61 Não violar um ``shebuat bitui'' \emph{219}
%
%62 Não fazer um ``shebuat shav'' \emph{220}
%
%63 Não profanar o nome de Deus \emph{221}
%
%64 Não testar suas promessas e advertências \emph{222}
%
%65 Não demolir casas de adoração ao Eterno \emph{222}
%
%66 Não deixar o corpo de um criminoso pendurado durante toda a noi
%
%te após sua execução \emph{222}
%
%67 Não interromper a vigilância do Santuário \emph{223}
%
%68 O ``Cohen Gadol'' não deve entrar no Santuário em outras ocasiões
%além das estabelecidas \emph{223}
%
%69 Um ``Cohen'' com um defeito não deve entrar em nenhuma parte
%
%do Santuário \emph{224}
%
%70 Um ``Cohen'' com um defeito não deve ministrar no Santuário
%\emph{224}
%
%71 Um ``Cohen'' com um defeito temporário não deve ministrar no
%Santuário \emph{224}
%\end{quote}
%
%72 Os Levitas e os ``Cohanim'' não devem realizar as tarefas uns dos
%outros \emph{225}
%
%\begin{quote}
%73 Não entrar no Santuário nem pronunciar uma sentença sobre uma
%
%lei da Torah estando intoxicado \emph{226}
%
%74 Um ``zar'' não deve oficiar no Santuário \emph{226}
%
%75 Um ``Cohen'' impuro não deve oficiar no Santuário \emph{227}
%
%76 Um ``Cohen'' que praticou um ``Tebul yom'' não deve oficiar no
%Santuário \emph{227}
%\end{quote}
%
%77 Uma pessoa impura não pode entrar em nenhuma parte do Santuário
%\emph{228}
%
%78 Uma pessoa impura não pode entrar no acampamento dos Levitas
%\emph{229}
%
%79 Não construir um altar com pedras que tenham sido tocadas por ferro
%\emph{229}
%
%\begin{quote}
%80 Não subir ao altar por degraus \emph{229}
%
%81 Não apagar o fogo do altar \emph{230}
%
%82 Não oferecer nenhum tipo de sacrifício sobre o altar de ouro
%\emph{230}
%
%83 Não fazer óleo igual ao óleo de unção \emph{230}
%
%84 Não ungir ninguém a não ser os ``Cohanim Guedolim'' e os reis com
%
%o óleo de unção preparado por Moisés \emph{231}
%
%85 Não fazer incenso igual ao usado no Santuário \emph{231}
%
%86 Não retirar as varas das argolas da Arca \emph{231}
%
%87 Não desprender o peitoral do ``efod'' \emph{231}
%
%88 Não rasgar a orla do manto do ``Cohen Gadol'' \emph{232}
%
%89 Não oferecer nenhum sacrifício fora do campo do Santuário \emph{232}
%
%90 Não degolar nenhum dos sacrifícios sagrados fora do campo do
%Santuário \emph{233}
%\end{quote}
%
%91 Não destinar animais defeituosos para serem oferecidos sobre o altar
%\emph{234}
%
%\begin{quote}
%92 Não degolar animais defeituosos para oferecê-los como sacrifício
%\emph{234}
%
%93 Não aspergir o sangue de animais defeituosos sobre o altar \emph{235}
%
%94 Não queimar as partes de sacrifício de um animal defeituoso sobre
%
%o altar \emph{235}
%
%95 Não sacrificar um animal com um defeito temporário \emph{236}
%
%96 Não oferecer sacrifícios defeituosos de um gentio \emph{237}
%
%97 Não fazer com que uma oferta se torne defeituosa \emph{.237}
%
%98 Não oferecer fermento ou mel sobre o altar \emph{237}
%
%99 Não oferecer um sacrifício sem sal \emph{23 7}
%\end{quote}
%
%100 Não oferecer no altar o salário de uma rameira ou o preço de um cão
%\emph{238}
%
%101 Não degolar a mãe e seu filhote no mesmo dia \emph{238}
%
%102 Não colocar azeite de oliva sobre a oblação de um pecador \emph{238}
%
%103 Não levar incenso junto com a oblação de um pecador \emph{238}
%
%104 Não misturar azeite de oliva com a oblação de uma mulher suspeita
%
%\begin{quote}
%de adultério \emph{239}
%\end{quote}
%
%105 Não colocar incenso sobre a oblação de uma mulher suspeita de
%adultério \emph{239}
%
%106 Não trocar um animal que tenha sido consagrado como oferenda
%\emph{239}
%
%107 Não trocar uma oferenda sagrada por outra \emph{240}
%
%108 Não resgatar o primogênito de um animal puro \emph{240}
%
%109 Não vender o dízimo do gado  \emph{240}
%
%110 Não vender uma propriedade consagrada \emph{240}
%
%\emph{1 11} Não resgatar terra que tenha sido consagrada sem nenhuma
%declara
%
%ção específica de finalidade \emph{241}
%
%112 Não cortar a cabeça do pássaro de um Sacrifício de Pecado durante
%
%a ``Meliká'' \emph{241}
%
%113 Não fazer qualquer trabalho com um animal consagrado \emph{242}
%
%114 Não tosquiar um animal consagrado \emph{242}
%
%115 Não degolar o sacrifício de ``Pessah'' enquanto tivermos pão leveda
%
%do em nosso poder \emph{242}
%
%116 Não deixar as partes do sacrifício da oferenda de ``Pessah'' de um
%
%\begin{quote}
%dia para o outro \emph{243}
%\end{quote}
%
%117 Não deixar ficar nenhuma parte da carne da oferenda de ``Pessah''
%
%\begin{quote}
%até a manhã seguinte \emph{243}
%\end{quote}
%
%118 Não deixar sobrar carne do sacrifício do Festival do décimo quarto
%
%\begin{quote}
%do ``Nissan'' até o terceiro dia \emph{243}
%\end{quote}
%
%119 Não deixar sobrar carne do segundo sacrifício de ``Pessah'' até a ma
%
%nhã seguinte \emph{244}
%
%120 Não deixar sobrar carne do Sacríficio de Graças até a manhã seguinte
%\emph{244}
%
%121 Não quebrar nenhum osso do sacrifício de ``Pessah'' \emph{244}
%
%122 Não quebrar nenhum osso do segundo sacrifício de ``Pessah''
%\emph{245}
%
%123 Não retirar o sacrifício de ``Pessah'' do lugar onde ele é comido
%\emph{245}
%
%124 Não cozer as sobras de uma oblação de cereal com levedo \emph{245}
%
%125 Não comer o sacrifício de ``Pessah'' cozido nem cru \emph{246}
%
%126 Não permitir que um ``guer toshab'' coma do sacrifício de ``Pessah''
%\emph{246}
%
%127 Uma pessoa incircuncisa não deve comer do sacrifício de ``Pessah''
%\emph{246}
%
%128 Não permitir que um israelita apóstata coma do sacrifício de
%``Pessah'' \emph{246}
%
%129 Uma pessoa impura não deve comer comida santificada \emph{247}
%
%130 Não comer carne de sacrifícios consagrados que se tornaram impuros
%\emph{247}
%
%131 Não comer ``notar'' \emph{248}
%
%132 Não comer ``pigul'' \emph{248}
%
%190 MAIMÔNIDES
%
%133 Um ``zar'' nao deve comer ``terumá'' \emph{249}
%
%134 Um servo ou um criado de um ``Cohen'' não deve comer ``terumá''
%\emph{250}
%
%135 Um ``Cohen'' incircunciso não deve comer ``terumá'' \emph{250}
%
%136 Um ``Cohen'' impuro não deve comer ``terumá'' \emph{251}
%
%137 Uma ``halalá'' não deve comer alimento sagrado \emph{251}
%
%138 Não comer a oblação de um ``Cohen'' \emph{252}
%
%139 Não comer carne de sacrifícios de pecado cujo sangue tenha sido
%
%\begin{quote}
%levado para dentro do Santuário \emph{252}
%\end{quote}
%
%140 Não comer sacrifícios consagrados que tenham sido invalidados
%\emph{252}
%
%141 Não comer o segundo dízimo de cereais não remido fora de Jerusalém
%\emph{253}
%
%142 Não consumir o segundo dízimo de vinho não remido fora de Jerusalém
%\emph{253}
%
%143 Não consumir o segundo dízimo de azeite não remido fora de Jerusalém
%\emph{253}
%
%144 Não comer um primogênito sem defeito fora de Jerusalém \emph{254}
%
%145 Não comer o Sacrifício de Pecado e o Sacrifício de Delito fora do
%
%\begin{quote}
%campo do Santuário \emph{254}
%\end{quote}
%
%146 Não comer a carne de um Holocausto \emph{255}
%
%147 Não comer sacrifícios menos sagrados antes de aspergir seu sangue
%
%\begin{quote}
%sobre o altar \emph{255}
%\end{quote}
%
%148 Um ``Cohen'' não pode comer as primícias fora de Jerusalém
%\emph{256}
%
%149 Um ``zar'' não pode comer os sacrifícios mais sagrados \emph{257}
%
%150 Não comer o segundo dízimo impuro não remido, nem mesmo em
%
%\begin{quote}
%Jerusalém \emph{257}
%\end{quote}
%
%151 Não comer o segundo dízimo durante o período de luto \emph{257}
%
%152 Não gastar o dinheiro do resgate do segundo dízimo a não ser com
%
%\begin{quote}
%comida e bebida \emph{258}
%\end{quote}
%
%153 Não comer ``tebel'' \emph{258}
%
%154 Não alterar a ordem prescrita para separar o dízimo da Colheita
%\emph{259}
%
%155 Não adiar o pagamento de promessas \emph{260}
%
%156 Não comparecer a um festival sem um sacrifício \emph{260}
%
%157 Não deixar de cumprir uma obrigação oral, mesmo que não se tenha
%
%\begin{quote}
%feito um juramento \emph{260}
%\end{quote}
%
%158 Um ``Cohen'' não pode casar-se com uma ``zoná'' \emph{261}
%
%159 Um ``Cohen'' não pode casar-se com uma ``halalá'' \emph{261}
%
%160 Um ``Cohen'' não pode casar-se com uma mulher divorciada \emph{261}
%
%161 Um ``Cohen Gadol'' não pode casar-se com uma viúva \emph{262}
%
%162 Um ``Cohen Gadol'' não pode chegar-se a uma viúva \emph{263}
%
%\emph{163} Um ``Cohen'' não pode entrar no Santuário com o cabelo solto
%\emph{264}
%
%\emph{164} Os ``Cohanim'' não podem usar vestes rasgadas ao entrar no
%Santuário \emph{264}
%
%\emph{165} Os ``Cohanim'' não podem sair do Santuário enquanto estiverem
%oficiando, \emph{265}
%
%166 Um ``Cohen'' comum não pode tornar-se impuro por nenhuma pes
%
%soa morta a não ser pelas que estão determinadas na Torah \emph{266}
%
%167 Um ``Cohen Gadol'' não deve ficar sob o mesmo teto que um cadáver
%\emph{267}
%
%168 Um ``Cohen Gadol'' não pode fazer-se impuro por nenhuma pessoa
%
%\begin{quote}
%morta \emph{267}
%\end{quote}
%
%169 Os Levitas não podem adquirir um pedaço da Terra de Israel
%\emph{267}
%
%170 Os Levitas não podem receber nenhuma parte da pilhagem da conquista
%da Terra de Israel \emph{268}
%
%171 Não arrancar nosso cabelo pelos mortos \emph{269}
%
%172 Não comer um animal impuro \emph{270}
%
%173 Não comer um peixe impuro \emph{271}
%
%PRECEITOS NEGATIVOS 191
%
%174 Não comer nenhuma ave impura \emph{271}
%
%175 Não comer nenhum inseto alado \emph{271}
%
%176 Não comer nada que rasteje sobre a terra \emph{271}
%
%177 Não comer nenhuma criatura rastejante que se reproduza em matéria
%deteriorada \emph{271}
%
%178 Não comer criaturas vivas que se reproduzam em sementes ou frutas
%\emph{272}
%
%179 Não comer nenhuma espécie de criatura rastejante \emph{272}
%
%180 Não comer ``nebelá'' \emph{275}
%
%181 Não comer ``terefá'' \emph{275}
%
%182 Não comer um membro de um animal vivo \emph{276}
%
%183 Não comer ``guid hanashé'' \emph{277}
%
%184 Não comer sangue \emph{277}
%
%185 Não comer gordura de um animal puro \emph{277}
%
%186 Não cozinhar carne no leite \emph{277}
%
%187 Não comer carne cozida em leite \emph{278}
%
%188 Não comer a carne de um boi apedrejado \emph{279}
%
%189 Não comer pão feito com grãos da nova ceifa \emph{280}
%
%190 Não comer grãos da nova ceifa torrados \emph{280}
%
%191 Não comer grãos verdes de cereais \emph{280}
%
%192 Não comer ``orlá'' \emph{280}
%
%193 Não comer ``quil-ei ha querem'' \emph{281}
%
%194 Não beber ``yain nessech'' \emph{281}
%
%195 Não comer nem beber em excesso \emph{282}
%
%196 Não comer durante um ``Yom Kipur'' \emph{283}
%
%197 Não comer ``hametz'' durante ``Pessah'' \emph{283}
%
%198 Não comer nada que contenha ``hametz'' durante ``Pessah'' \emph{284}
%
%199 Não comer ``hametz'' depois da metade do décimo quarto de ``Nissan''
%\emph{284}
%
%200 Não pode ser visto ``hametz'' em nossas moradias durante ``Pessah''
%\emph{285}
%
%201 Não possuir ``hametz'' durante ``Pessah'' \emph{285}
%
%202 Um ``Nazir'' não pode beber vinho 285
%
%203 Um ``Nazir'' não pode comer uvas frescas \emph{286}
%
%204 Um ``Nazir'' não pode comer uvas secas \emph{286}
%
%205 Um ``Nazir'' não pode comer caroços de uvas \emph{286}
%
%206 Um ``Nazir'' não pode comer bagaços de uvas \emph{286}
%
%207 Um ``Nazir'' não pode fazer-se impuro pelos mortos \emph{287}
%
%208 Um ``Nazir'' não pode fazer-se impuro entrando numa casa onde ha
%
%ja um morto \emph{287}
%
%209 Um ``Nazir'' não pode raspar a cabeça \emph{288}
%
%210 Não ceifar toda a colheita \emph{288}
%
%211 Não recolher as espigas de cereais que caíram durante a colheita
%\emph{288}
%
%212 Não recolher todo o produto do vinhedo na época da vindima
%\emph{288}
%
%213 Não recolher os bagos das uvas que caírem durante a colheita
%\emph{289}
%
%214 Não voltar para buscar uma gavela esquecida \emph{289}
%
%215 Não semear ``quil-aim'' \emph{290}
%
%216 Não semear grãos nem vegetais num vinhedo \emph{291}
%
%217 Não cruzar animais de espécies diferentes \emph{291}
%
%218 Não trabalhar com duas espécies diferentes de animais juntos
%\emph{291}
%
%219 Não impedir um animal de comer do produto no meio do qual ele
%
%\begin{quote}
%esteja trabalhando \emph{292}
%\end{quote}
%
%220 Não cultivar o solo no sétimo ano \emph{292}
%
%221 Não podar árvores no sétimo ano \emph{293}
%
%192 MAIMÓNIDES
%
%222 Não ceifar uma planta que nasceu por si só no sétimo ano da manei
%
%ra como se faz num ano comum \emph{293}
%
%223 Não colher uma fruta que tenha crescido por si só no sétimo ano,
%
%\begin{quote}
%da mesma maneira como se faz num ano comum \emph{293}
%\end{quote}
%
%224 Não cultivar o solo no Ano do Jubileu \emph{294}
%
%22 5 Não ceifar a produção tardia do Ano do Jubileu da maneira como
%
%\begin{quote}
%se faz num ano comum \emph{294}
%\end{quote}
%
%226 Não colher frutas no Ano do Jubileu da maneira como se faz num
%
%\begin{quote}
%ano comum \emph{294}
%\end{quote}
%
%227 Não vender definitivamente nossas terras em Israel \emph{295}
%
%228 Não vender as terras dos arredores dos Levitas \emph{295}
%
%229 Não abandonar os Levitas \emph{295}
%
%230 Não cobrar as dívidas depois do Ano de Shabat \emph{295}
%
%231 Não recusar um empréstimo que deva ser cancelado no Ano de Shabat
%\emph{296}
%
%232 Não deixar de fazer caridade a nossos irmãos necessitados \emph{296}
%
%233 Não mandar embora de mãos vazias um servo hebreu \emph{296}
%
%234 Não cobrar uma dívida de alguém que se sabe que não pode pagar
%\emph{297}
%
%235 Não emprestar a juros \emph{297}
%
%236 Não tomar emprestado com juros \emph{298}
%
%237 Não participar de um empréstimo a juros \emph{298}
%
%238 Não oprimir um empregado atrasando o pagamento de seus salários
%\emph{299}
%
%239 Não tomar pela força um penhor de um devedor \emph{299}
%
%240 Não ficar com um penhor do qual seu proprietário precise \emph{300}
%
%241 Não pegar um penhor de uma viúva \emph{300}
%
%242 Não pegar como penhor utensílios usados para a alimentação
%\emph{300}
%
%243 Não raptar um israelita \emph{301}
%
%244 Não furtar dinheiro \emph{302}
%
%245 Não cometer um roubo \emph{302}
%
%246 Não alterar os limites das terras fraudulentamente \emph{303}
%
%247 Não usurpar nossas dívidas \emph{303}
%
%248 Não negar nossas dívidas \emph{304}
%
%249 Não jurar em falso ao negar uma dívida \emph{304}
%
%250 Não enganar um ao outro em negócios \emph{304}
%
%251 Não prejudicar um ao outro com palavras \emph{305}
%
%252 Não enganar um prosélito com palavras \emph{305}
%
%253 Não enganar um prosélito nos negócios \emph{305}
%
%254 Não entregar um escravo fugitivo \emph{306}
%
%255 Não enganar um escravo fugitivo \emph{306}
%
%256 Não ser rude com crianças órfãs e com viúvas \emph{306}
%
%257 Não utilizar um servo hebreu para executar tarefas degradantes
%\emph{307}
%
%258 Não vender um servo hebreu em leilão \emph{307}
%
%259 Não utilizar um servo hebreu para fazer um trabalho desnecessário
%\emph{308}
%
%260 Não permitir que se maltrate um servo hebreu \emph{308}
%
%261 Não vender uma serva hebreia \emph{308}
%
%262 Não privar uma serva hebreia que se desposou \emph{309}
%
%263 Não vender uma prisioneira \emph{309}
%
%264 Não escravizar uma prisioneira \emph{309}
%
%265 Não planejar obter a propriedade de outrem \emph{309}
%
%266 Não cobiçar os pertences de outrem \emph{310}
%
%267 Um trabalhador contratado não pode comer das plantações em cres
%
%cimento \emph{310}
%
%268 Um trabalhador contratado não pode servir-se em demasia \emph{311}
%
%PRECEITOS NEGATIVOS 193
%
%269 Não ignorar uma propriedade perdida \emph{311}
%
%270 Não abandonar uma pessoa sobrecarregada \emph{311}
%
%271 Não trapacear nas medidas e nos pesos \emph{312}
%
%272 Não manter pesos e medidas incorretos \emph{312}
%
%273 Um juiz não pode cometer injustiças \emph{313}
%
%274 Um juiz não pode aceitar presentes de uma das partes \emph{313}
%
%275 Um juiz não pode proteger uma das partes \emph{313}
%
%276 Um juiz não pode acovardar-se com medo de pronunciar um julgamento
%justo \emph{314}
%
%277 Um juiz não pode decidir em favor de um homem pobre por piedade
%\emph{314}
%
%278 Um juiz não pode distorcer um julgamento contra uma pessoa de
%
%\begin{quote}
%má reputação \emph{314}
%\end{quote}
%
%279 Um juiz não pode ter piedade de alguém que matou um homem \emph{315}
%
%280 Um juiz não pode distorcer a justiça por prosélitos ou órfãos
%\emph{315}
%
%281 O juiz não pode ouvir uma das partes na ausência da outra \emph{315}
%
%282 Um tribunal não pode condenar por maioria de um num caso capital 316
%
%283 Um juiz não pode confiar na opinião de outro juiz \emph{316}
%
%284 Não designar um juiz inculto \emph{317}
%
%285 Não prestar um falso testemunho \emph{317}
%
%286 Um juiz não pode aceitar o testemunho de um homem mau \emph{317}
%
%287 Um juiz não pode aceitar o testemunho de um parente de uma das
%partes \emph{318}
%
%288 Não condenar baseado no depoimento de uma única testemunha
%\emph{318}
%
%289 Não matar um ser humano \emph{318}
%
%290 Não punir com a pena capital baseando-se em provas circunstanciais
%\emph{319}
%
%291 Uma testemunha não pode atuar como advogado \emph{320}
%
%292 Não matar um assassino sem julgamento \emph{320}
%
%293 Não poupar a vida de um perseguidor \emph{320}
%
%294 Não punir uma pessoa por um pecado cometido sob coação \emph{321}
%
%295 Não aceitar um resgate de alguém que tenha cometido um assassina
%
%to deliberadamente \emph{321}
%
%296 Não aceitar um resgate de alguém que tenha cometido um assassina
%
%to involuntariamente \emph{322}
%
%297 Não se descuidar de salvar um israelita em perigo de vida \emph{322}
%
%298 Não deixar obstáculos em propriedades públicas ou privadas
%\emph{323}
%
%299 Não dar um conselho enganoso \emph{323}
%
%300 Não infligir castigo corporal excessivo \emph{323}
%
%301 Não bisbilhotar \emph{324}
%
%302 Não odiar uns aos outros \emph{324}
%
%303 Não envergonhar ninguém \emph{324}
%
%304 Não se vingar um do outro \emph{325}
%
%305 Não guardar rancor \emph{325}
%
%306 Não pegar o ninho todo de um pássaro. \emph{325}
%
%307 Não raspar a tinha \emph{326}
%
%308 Não cortar ou cauterizar marcas de lepra \emph{326}
%
%309 Não lavrar um vale no qual tenha sido realizado o ritual de ``Eglá
%Arufá'' \emph{326}
%
%310 Não deixar viver um feiticeiro \emph{326}
%
%311 Não levar um recém-casado para longe de sua casa \emph{327}
%
%312 Não discordar das autoridades tradicionais \emph{327}
%
%313 Não fazer acréscimos à lei escrita ou oral \emph{327}
%
%314 Não fazer diminuições na lei escrita ou oral \emph{328}
%
%315 Não maldizer um juiz \emph{328}
%
%316 Não maldizer um chefe \emph{328}
%
%194 MAIMÔNIDES
%
%317 Não maldizer um israelita \emph{328}
%
%318 Não amaldiçoar os pais \emph{330}
%
%319 Não ferir seus pais \emph{330}
%
%320 Não trabalhar no Shabat \emph{331}
%
%321 Não viajar no Shabat \emph{331}
%
%322 Não castigar durante o Shabat \emph{331}
%
%323 Não trabalhar no primeiro dia de ``Pessah'' \emph{332}
%
%324 Não trabalhar no sétimo dia de ``Pessah'' \emph{332}
%
%325 Não trabalhar em ``Atzeret'' \emph{332}
%
%326 Não trabalhar em ``Rosh Hashaná'' \emph{332}
%
%327 Não trabalhar no primeiro dia de ``Sucot'' \emph{332}
%
%328 Não trabalhar em ``Shemini Atzeret'' \emph{332}
%
%329 Não trabalhar em ``Yom Kipur'' \emph{333}
%
%330 Não cometer incesto com sua mãe \emph{333}
%
%331 Não cometer incesto com a esposa de seu pai \emph{333}
%
%332 Não cometer incesto com sua irmã \emph{334}
%
%333 Não cometer incesto com a filha da esposa de seu pai se ela for sua
%
%irmã \emph{334}
%
%334 Não cometer incesto com a filha de seu filho \emph{334}
%
%335 Não cometer incesto com a filha de sua filha \emph{334}
%
%336 Não cometer incesto com sua filha \emph{335}
%
%337 Não se chegar a uma mulher e a sua filha 336
%
%338 Não se chegar a uma mulher e à filha do filho dela 336
%
%339 Não se chegar a uma mulher e à filha da filha dela 336
%
%340 Não cometer incesto com a irmã de seu pai \emph{336}
%
%341 Não cometer incesto com a irmã de sua mãe \emph{336}
%
%342 Não se chegar à esposa do irmão de seu pai \emph{337}
%
%343 Não se chegar à esposa de seu filho \emph{337}
%
%344 Não se chegar à esposa de seu irmão \emph{337}
%
%345 Não se chegar à irmã de sua esposa enquanto esta última for viva 337
%
%346 Não se unir a uma mulher menstruada \emph{337}
%
%347 Não se chegar à esposa de outro homem \emph{338}
%
%348 Os homens não podem deitar-se com animais 339
%
%349 As mulheres não podem deitar-se com animais \emph{339}
%
%350 Um homem não pode chegar-se a outro homem 339
%
%351 Um homem não pode chegar-se a seu pai \emph{340}
%
%352 Um homem não pode chegar-se ao irmão de seu pai \emph{340}
%
%353 Não ter intimidades com uma parenta \emph{342}
%
%354 Um ``mamzer'' não pode chegar-se a uma israelita \emph{343}
%
%355 Não se chegar a uma mulher antes do casamento \emph{343}
%
%356 Não tornar a casar-se com a esposa de quem se divorciou, depois
%
%\begin{quote}
%que ela tenha se casado novamente \emph{344}
%\end{quote}
%
%357 Não se chegar a uma mulher sujeita ao casamento levirato 344
%
%358 Não se divorciar da mulher que se violentou e com a qual se foi obri
%
%gado a casar \emph{345}
%
%359 Não se divorciar de uma mulher depois de tê-la caluniado \emph{345}
%
%360 Um homem incapaz de procriar não pode casar-se com uma israelita
%\emph{346}
%
%361 Não castrar \emph{346}
%
%362 Não nomear um rei que não seja israelita de nascimento \emph{346}
%
%363 Um rei não pode possuir muitos cavalos \emph{347}
%
%\emph{364} Um rei não pode ter muitas esposas \emph{347}
%
%365 Um rei não pode acumular grande fortuna pessoal \emph{348}

\setcounter{paragraph}{0}
\setcounter{secnumdepth}{4}

\paragraph{Não crer nem atribuir divindade a outro que não Ele}

Por esta proibição somos proibidos de crer em ou atribuir divindade a
outro que não Ele, enaltecido seja. Ela está expressa em Suas palavras
--- embora não se possa atribuir palavras a Seu Ser
transcendental\footnote{Já que Deus não é um corpo, e que portanto não se pode atribuir a Ele a ``fala'' física.} --- ``Não terás outros deuses
diante de Mim'' (Êxodo 20:3).

Foi deixado claro no final de Macot que esta proibição é um dos 613
preceitos, pois está dito ali: ``Seiscentos e treze preceitos foram
dados a Moisés no Sinai etc.''. Nós já explicamos este assunto no
primeiro preceito positivo.

\paragraph{Não fazer imagens para adorá-las}

Por esta proibição somos proibidos de fazer imagens que venham a ser
adoradas, e não há diferença entre fazê-las nós mesmos ou mandar que
outros as façam. Esta proibição está expressa em Suas palavras,
enaltecido seja Ele, ``Não farás para ti imagem de escultura etc.''
(Êxodo 20:4).

Todo aquele que transgredir este preceito negativo estará sujeito ao
açoitamento, seja por ter feito o ídolo ou por ter mandado que outra
pessoa o faça, mesmo que ele não o adore.

\paragraph{Não fazer um ídolo para que outros o adorem}

Por esta proibição somos proibidos de fazer um ídolo, mesmo que seja
para que outros o adorem, e ainda que a pedido de um idólatra. Ela está
expressa em Suas palavras, enaltecido seja Ele, ``Nem fareis vós ídolos
para vós mesmos'' (Levítico 19:4), a respeito das quais diz a Sifrá:
```Nem fareis vós': ainda que seja para os outros''.

Também está dito ali: ``Aquele que fizer um ídolo para uso próprio
transgride dois preceitos negativos''. Ou seja, ele transgride a
proibição de fazê-lo ele próprio, ainda que para uso de outros, que está
contida neste terceiro preceito, e também a proibição de adquirir um
ídolo e de guardá-lo, ainda que outra pessoa o tenha feito para ele, e
que está contida no preceito precedente. Dessa forma ele está sujeito a
ser açoitado duas vezes.

As normas deste preceito e do precedente estão explicadas no Tratado
Abodá Zará.

\paragraph{Não fazer figuras de seres humanos}

Por esta proibição somos proibidos de fazer figuras de seres humanos de
metal, pedra, madeira e similares, ainda que elas não sejam feitas para
serem adoradas. A finalidade disso é impedir-nos de fazer qualquer
imagem para que não pensemos, como fazem as massas, que elas possuem
poderes sobrenaturais. Esta proibição está expressa em Suas palavras,
enaltecido seja Ele, ``Não fareis diante de Mim, deuses de prata nem
deuses de ouro para vós'' (Exodo 20:23).

Para explicar esta proibição a Mekhiltá diz: ```Não fareis\ldots{} deuses de
prata'; a fim de que você não diga: Vou fazê-los apenas como enfeites,
como outros fazem em vários outros países, as Escrituras dizem `Não
fareis\ldots{} para vós'''.

A transgressão deste preceito negativo é punida com o açoitamento.

As normas deste preceito --- que figuras nos é permitido ou proibido
confeccionar, e de que forma, e assim por diante --- estão explicadas no
terceiro capítulo de Abodá Zará.

Está explicado em Sanhedrin que este preceito negativo --- ou seja, Suas
palavras ``Não fareis diante de Mim, deuses de prata etc.'' --- também
abrange outros aspectos que vão além do limite destes preceitos; mas o
sentido literal deste versículo é o que expusemos, como está explicado
na Makhiltá.

\paragraph{Não se curvar diante de um ídolo}

Por esta proibição somos proibidos de curvar-nos diante de um ídolo, e
está claro que o termo ``ídolo'' significa qualquer outro objeto de
adoração que não o Eterno. Esta proibição está expressa em Suas
palavras, enaltecido seja Ele, ``Não te prostarás diante deles, nem os
servirás'' (Êxodo 20:5). A intenção não é de proibir unicamente o ato de
curvar-se, excluindo os outros; foi mencionada apenas uma forma de
adoração, que é a prostração, mas estamos da mesma forma proibidos de
oferecer sacrifícios e de queimar incenso diante de um ídolo; todo
aquele que fizer uma dessas coisas proibidas, isto é, que se curvar,
oferecer sacrifícios, oferecer uma libação ou queimar incenso estará
sujeito ao apedrejamento.

A Mekhiltá diz: ```Aquele que sacrificar aos deuses, será morto' (Êxodo
22:19). Ouvimos, dessa forma, a penalidade, mas não ouvimos a
advertência. Por isso as Escrituras dizem: `Não te prostarás diante
deles, nem os servirás'.
Oferecer sacrifícios, que já está incluído\footnote{Na advertência ``nem os servirás''.}, está
destacado aqui para ensinar-nos a seguinte lição: O sacrifício, que é
algo que se realiza em sinal de adoração a Deus, é um pecado, quer seja
ele normalmente adorado dessa forma quer não; portanto também no caso de
qualquer outro ato executado a serviço de Deus, é um pecado, quer seja
ele normalmente adorado dessa maneira, quer não''. O significado dessas
palavras é que todo aquele que realizar diante de um ídolo qualquer um
desses quatro atos de adoração --- a saber, curvar-se, oferecer
sacrifícios, queimar incenso e verter uma libação --- que nós temos a
obrigação de realizar a serviço de Deus --- está sujeito ao
apedrejamento, mesmo que o ídolo não seja normalmente adorado daquela
maneira. O que se quer dizer pela expressão ``não seja normalmente
adorado'' é o seguinte: embora alguém não tenha adorado o ídolo da
maneira pela qual ele é costumeiramente adorado, adorá-lo de uma das
formas mencionadas faz com que ele fique sujeito ao apedrejamento, se
ele tiver pecado voluntariamente, e à extinção, se seu pecado não tiver
sido testemunhado ou se ele não tiver sido punido por isso. Contudo, se
o pecado foi cometido involuntariamente, o pecador deve oferecer um
Sacrifício Determinado de Pecado. Isto também se aplica a quem deificar
um objeto qualquer.

Esta proibição --- ou seja, a proibição de prestar homenagem a um ídolo
através de qualquer uma dessas quatro formas de adoração, mesmo que o
ídolo não seja normalmente adorado assim --- está repetida em Suas
palavras, enaltecido seja Ele, ``E não oferecerão mais seus sacrifícios
aos `se-irim''' (Levítico 17:7), a respeito das quais a Sifrá diz:
```Se-irim' significa demônios''.

Na Guemará de Pessahim está explicado que esta proibição se aplica em
especial ao caso de alguém que tenha abatido uma oferenda para um ídolo,
mesmo que ele não seja normalmente adorado dessa forma: ``Como sabemos
que se alguém sacrificar um animal a `Merkulis' ele está sujeito a ser
punido? Porque está escrito `Não oferecerão mais seus sacrifícios aos
demônios'. Uma vez que há uma redundância no que se refere às formas
comuns de adoração, que aparece no versículo `De que modo serviam estas
nações a seus deuses' (Deuteronômio 12:30) deve-se considerar este como
referindo-se a formas não comuns de adoração''. Dessa maneira, a
transgressão voluntária desta proibição é punida com ambas a extinção e
o apedrejamento, como explicado acima, e aquele que a transgredir
involuntariamente deve oferecer um sacrifício. As palavras das
Escrituras relativas a isto são: ``Aquele que sacrificar aos deuses será
morto''.

As normas deste preceito estão explicadas no sétimo capítulo de Sanhedrin.

\paragraph{Não adorar ídolos}

Por esta proibição somos proibidos de adorar ídolós ainda que de outras
formas além das quatro especificadas antes, desde que aquele ídolo
específico seja normalmente adorado dessa maneira, como por exemplo,
evacuar para Peor ou jogar uma pedra para Merkulis. Esta proibição está
expressa em Suas palavras, enaltecido seja Ele, ``Nem os servirás''
(Êxodo 20:5), a respeito das quais a Mekhiltá diz: ```Não te prostrarás
diante deles, nem os servirás': aqui há dois pecados separados e
independentes --- oferecer um sacrifício e prostrar-se''. Dessa forma,
aquele que atirar uma pedra para Peor ou evacuar para Merkulis não terá cometido um pecado, pois essas não são as maneiras comuns
de adorá-los, de acordo com Suas palavras, enaltecido seja Ele, ``De que
modo serviam estas nações a seus deuses, do mesmo modo também farei eu''
(Deuteronômio 12:30).

A transgressão voluntária a esta proibição é punível com o apedrejamento
e a extinção, e aquele que a violar involuntariamente deve oferecer um
sacrifício.

As normas deste preceito também estão explicadas no sétimo capítulo de
Sanhedrin, onde se lê: ``Por que a extinção está mencionada três vezes
pela idolatria? Ela está prescrita uma vez pela maneira usual, uma vez
pela maneira não usual, e uma vez por Molekh''. Ou seja, aquele que
adorar qualquer ídolo, seja de que forma for, estará sujeito à extinção,
se a maneira de adoração for a usual, tal como evacuar para Peor, jogar
pedras para Merkulis, ou afastar o cabelo diante de Quemosh. Da mesma
forma, aquele que adorar qualquer ídolo de uma das quatro maneiras
especificadas está sujeito à extinção, mesmo se a forma de adoração não
for a usual, como, por exemplo, se ele oferecer sacrifícios a Peor, ou
prostrar-se diante de Merkulis, o que seria uma forma ``não usual'' de
adoração. A terceira extinção se aplica àquele que faz com que seus
descendentes passem pelo fogo em sinal de adoração a Molekh, como
explicarei.

\paragraph{Não entregar parte de sua descendência a Molekh}

Por esta proibição somos proibidos de colocar uma parte de nossos
descendentes nas mãos do ídolo conhecido, na época da entrega da Torah,
como Molekh. Ela está expressa em Suas palavras, enaltecido seja Ele,
``E da tua semente não entregarás nenhum, para fazê-la passar pelo fogo,
a Molekh'' (Levítico 18:21).

Esta forma de idolatria, como está explicado no sétimo capítulo de
Sanhedrin, consistia em acender um fogo e abanar suas chamas, quando
então\footnote{O pai (Hilchot Rambam, Abodá Zará, 6º capítulo, Halachá 3).} entregava parte de seus descendentes ao
sacerdote a serviço daquele ídolo, e fazia com que passassem através do
fogo de um lado para o outro.

A proibição de tal conduta está repetida em Suas palavras ``Não se
achará entre ti, quem faça passar seu filho ou sua filha pelo fogo''
(Deuteronômio 18:10).

Aquele que violar voluntariamente esta proibição estará sujeito ao
apedrejamento ou à extinção, se não for apedrejado; aquele que pecar
involuntariamente deverá oferecer um Sacrifício Determinado de Pecado.

As normas deste preceito estão explicadas no sétimo capítulo de Sanhedrin.

\paragraph{Não praticar a feitiçaria do ``ob''}

Por esta proibição somos proibidos de praticar a feitiçaria de um ``ob''
o qual, depois de ter queimado um determinado incenso e realizado um
determinado ritual, imagina ouvir uma voz falando de debaixo de suas
axilas que responde a suas perguntas, sendo que essa prática é considerada como um
tipo de idolatria. Esta proibição está expressa em Suas palavras,
enaltecido seja Ele, ``Não vos voltareis para as magias'' (Levítico
19:31), a respeito das quais a Sifrá diz: ``é o Piton que fala de
debaixo de suas axilas''.

Aquele que violar voluntariamente esta proibição --- ou seja, que
praticar isto ele próprio, e que realizar o ritual --- estará sujeito ao
apedrejamento ou à extinção, se ele não for apedrejado; aquele que
cometer o pecado involuntariamente deve oferecer um Sacrifício
Determinado de Pecado.

As normas deste preceito estão explicadas no sétimo capítulo de Sanhedrin.

\paragraph{Não praticar a feitiçaria do ``yideoni''}

Por esta proibição somos proibidos de praticar a feitiçaria de um
``yideoni'', sendo que esta também é uma espécie de idolatria. O
``yideoni'' coloca na boca o osso de um pássaro chamado ``yidoa'',
queima incenso, recita determinadas palavras e cumpre um determinado
ritual até ficar como se tivesse desmaiado e cair num transe, durante o
qual ele\footnote{O osso do pássaro.} prediz o futuro. Os Sábios dizem:
```yideoni' --- aquele que coloca o osso do `yidoa' em sua boca, o qual
fala por si próprio''. A proibição desta prática está nas palavras ``Não
vos voltareis para as magias e para as feitiçarias (yideoni)''
(Levítico 19:31). Isto não deve ser considerado como um ``Lav
shebikhlalut'' porque ao falar dos castigos a serem aplicados Ele separa
os dois, dizendo ``ob'' ou ``yideoni'', e ordenando apedrejamento e
extinção no caso de violação voluntária de cada um deles. Suas palavras,
enaltecido seja Ele, são ``E homem ou mulher que fizerem magia (`ob') ou
feitiçaria (`yideoni'), serão mortos'' (Levítico 20:27). A Sifrá diz: Em
suas palavras ``E homem ou mulher que fizerem magia ou feitiçaria'' está
o castigo, mas não a advertência. Por isso as Escrituras dizem: ``Não
vos voltareis para as magias e para as feitiçarias'' (Ibid., 19:31).

Também neste caso aquele que transgredir a proibição involuntariamente
deverá levar um Sacrifício Determinado de Pecado.

As normas deste preceito estão explicadas no sétimo capítulo de Sanhedrin.

\paragraph{Não estudar as práticas da idolatria}

Por esta proibição somos proibidos de interessar-nos pela idolatria ou
de estudar suas práticas, ou seja, indagar a respeito de seus disparates
e superstições ensinados pelos seus fundadores, como, por exemplo, que
um espírito pode descer de uma determinada maneira e se comportará de
uma determinada forma; ou que se se queimar incenso para uma determinada
estrela e se seus adoradores se colocarem diante dela numa determinada
posição, ela agirá de uma determinada maneira; e assim por diante. O
simples fato de pensar a respeito desses assuntos e de se informar sobre
essas ilusões leva os tolos a aproximar-se dos ídolos e a adorá-los. O
versículo das Escrituras que contém a proibição destas práticas é:
``Não vos volteis aos ídolos'' (Levítico 19:4), a respeito do qual diz a
Sifrá: ``Se você se voltar para eles você os endeusará''. A Sifrá cita
também as palavras de Rabi Yehudá: ``Não vos volteis para vê-los'', ou seja,
nem sequer olhe para o ídolo ou estude sua forma de idolatria, para não
perder nem um momento sequer com qualquer coisa relacionada com isso.

No capítulo ``Shoel Adam'' os Sábios dizem: ``O que está escrito abaixo
de um quadro ou de uma estátua não deve ser lido no Shabat. Quanto à
estátua em si, não se deve olhá-la nem mesmo durante os dias da semana
porque foi dito `Não vos volteis aos ídolos'. Como isto deve ser
interpretado? Rabi Yohanan disse: Não vos volteis ao que vossa própria
mente concebe''.

A proibição de pensar a respeito dos ídolos está repetida em Suas
palavras ``Guardai-vos, não suceda que o vosso coração vos seduza, e vos
desvieis, e sirvais'' (Deuteronômio 11:16). Quer dizer, se sua mente
cometer o erro de pensar em ídolos, isso o levará a extraviar-se do
caminho correto e a adorá-los. Ele ainda diz, sobre o mesmo assunto, ``E
quiçá levantes os teus olhos para os céus, e vendo o sol, a lua e as
estrelas etc.'' (Ibid., 4:19). Ele não proíbe que se levante a cabeça e
se observe os corpos celestiais com os olhos; o que Ele proíbe é que se
olhe com os olhos da mente para aquilo que seus adoradores atribuem a
eles. Por isso também Suas palavras ``E não indagues acerca dos seus
deuses, dizendo: De que modo serviam estas nações a seus deuses? Do
mesmo modo também farei eu'' (Deuteronômio 12:30) são uma advertência
para que não indaguemos a respeito de suas formas de adoração, ainda que
nós próprios não os adoremos, porque tudo isso nos conduz a segui-los
por seus maus caminhos.

Você deve saber que todo aquele que transgredir esta proibição está
sujeito ao açoitamento. Isso foi deixado claro no final do primeiro
capítulo do Erubin, ao mencionar que o castigo de açoitamento está
prescrito na lei das Escrituras para aquele que violar a lei de ``erub''
dos Limites. Para confirmar isso foram citadas as seguintes palavras das
Escrituras: ``Não saia ninguém (al) de seu lugar'' (Êxodo 16:29); e
quando alguém objetou: ``É o açoitamento o castigo pela violação da
proibição expressa por `al' e não por `lo'?'' foi-lhe respondido: ``Se o
castigo pela desobediência da proibição expressa por `al' não fosse o
açoitamento, o castigo pela desobediência da proibição `Não nos (al)
volteis aos ídolos' não seria o açoitamento''. Isto mostra que a
transgressão desta proibição é punida com o açoitamento.

\paragraph{Não erguer um pilar que as pessoas se reunirão para reverenciar}

Por esta proibição somos proibidos de erguer um pilar que as pessoas se
reunirão para reverenciar, mesmo que ele seja erguido com o propósito de
adoração ao Eterno. A razão disso é que não devemos, no serviço do
Eterno, imitar os idólatras, que tinham o hábito de erguer pilares e
colocar seus ídolos sobre eles. Esta proibição está expressa em Suas
palavras, enaltecido seja Ele, ``Não levantarás para ti `matzebá',
porque o Eterno, teu Deus, odeia'' (Deuteronômio 16:22).

A transgressão desta proibição é punida com o açoitamento,

\paragraph{Não esculpir pedras para prostrar-se sobre elas}

Por esta proibição somos proibidos de esculpir pedras para prostrar-nos
sobre elas, mesmo se elas tiverem sido feitas a serviço do Eterno. Isto
também é condenado para que não imitemos os idólatras, cujo hábito era
colocar uma pedra magnificamente esculpida aos pés de um ídolo e prostrar-se
sobre ela em adoração. Suas palavras, enaltecido seja Ele, ``Assoalho
sagrado de pedras, não poreis em vossa terra para prostrar-vos sobre
ele'' (Levítico 26:1).

A punição pela transgressão desta proibição é o açoitamento.

A Sifrá diz: ``Não poreis em vossa terra'': você está proibido de
prostrar-se sobre as pedras em sua terra, mas pode prostrar-se sobre as
pedras no Santuário.

As normas deste preceito estão explicadas na Guemará de Meguilá.

\paragraph{Não plantar árvores no Santuário}

Por esta proibição somos proibidos de plantar árvores no Santuário ou
junto ao Altar com a finalidade de adorno ou embelezamento, mesmo que
seja em adoração ao Eterno, porque era costume dos idólatras plantar
árvores bonitas e simétricas em honra aos ídolos, em suas casas de
adoração. Suas palavras, enaltecido seja Ele, são: ``Não plantarás para
ti, nenhuma `ashera' nem árvore junto ao altar do Eterno, teu Deus''
(Deuteronômio 16:21).

A transgressão desta proibição é punida com o açoitamento.

As normas deste preceito estão explicadas na Guemará de Tamid, onde foi
deixado claro que é proibido todo tipo de planta no Santuário.

\paragraph{Não jurar por um ídolo}

Por esta proibição somos proibidos de jurar por um ídolo mesmo estando
com idólatras, nem devemos fazê-los jurar em nome de um ídolo, de acordo
com as palavras da Mekhiltá ``Você não deve levar um pagão a jurar por
sua divindade''. Esta proibição está expressa em Suas palavras,
enaltecido seja Ele, ``E o nome de outros deuses não mencionareis''
(Exodo 23:13): você não deve fazer com que o idólatra jure em nome de
sua divindade.

No mesmo trecho a Mekhiltá também diz: ```Não mencionareis' --- isso
significa que não se deve fazer um juramento em nome de um ídolo''.

Em Sanhedrin está dito: ```Não mencionareis' --- isto significa que não
se deve dizer a seu amigo `Espere por mim junto a tal ídolo'''.

Todo aquele que transgredir esta proibição --- ou seja, que jurar em
nome de qualquer objeto criado que pessoas mal orientadas consideram
como divindade, como se fosse um ser superior\footnote{Ver o preceito positivo 7.} ---
está sujeito ao açoitamento. A Guemará de Sanhedrin diz o seguinte, com
relação à proibição feita pelos Rabinos de beijar ou abraçar um ídolo,
de varrer o chão diante dele, ou de realizar outros atos semelhantes que
indiquem respeito e amor por ele: ``O ofensor não será açoitado por
nenhum desses atos, a não ser por fazer uma promessa ou um juramento em
nome de um ídolo''.

As normas deste preceito estão explicadas no sétimo capítulo de Sanhedrin.

\paragraph{Não convocar pessoas para a idolatria}

Por esta proibição somos proibidos de convocar pessoas para a prática
da idolatria, isto é, convocar e estimular pessoas a adorar ídolos,
ainda que o próprio incitador não os adore e não faça nada além de
convocar outros a fazê-lo. Aquele que iludir uma comunidade é chamado de
``madiá'', de acordo com Suas palavras, enaltecido seja Ele, ``Saíram
homens ímpios do meio de ti, e perverteram (va-yedihu) os moradores da
sua cidade dizendo etc.'' (Deuteronômio 13:14). Aquele que iludir um
único indivíduo é chamado de ``messit'', de acordo com Suas palavras,
enaltecido seja Ele, ``Quando te incitar (ye-sithkha) teu irmão de pai,
ou teu irmão de mãe\ldots{} em segredo, dizendo etc.'' (Ibid., 7). No
presente preceito falamos apenas do ``madiá'' cuja ação está proibida
por Suas palavras, enaltecido seja Ele, ``Não seja ouvido de tua boca''
(Êxodo 23:13).

A Guemará de Sanhedrin diz: ``Não seja ouvido de tua boca'' é a
proibição contra o `messit'. A isso foi objetado que há uma proibição
explícita contra desviar alguém do bom caminho nas palavras das
Escrituras ``E todo Israel ouvirá e temerá, e não voltará a fazer uma
coisa má como esta'' (Ibid., 12) e essa é a proibição contra o `madiá'.
A Mekhiltá de Rabi Ishmael diz: ``Não seja ouvido de tua boca é uma
advertência contra o `madiá'''.

O castigo pela transgressão desta proibição é o apedrejamento, como
lemos em Sanhedrin: ``Aqueles que levarem uma cidade apóstata pelo mau
caminho serão punidos com o apedrejamento''.

As normas deste preceito estão explicadas no décimo capítulo de Sanhedrin.

\paragraph{Não tentar persuadir um israelita a adorar ídolos}

Por esta proibição somos proibidos de desencaminhar, ou seja, tentar
persuadir um israelita a adorar ídolos. Aquele que fizer isso será
chamado de ``messit'', como foi explicado anteriormente. A proibição
está expressa em Suas palavras, enaltecido seja Ele, ``E não voltará a
fazer uma coisa má como esta, no meio de ti'' (Deuteronômio 13:12).

A punição pela transgressão desta proibição --- ou seja, por
desencaminhar um israelita --- é o apedrejamento, como está dito nas
Escrituras: ``Mas certamente o matarás'' (Ibid., 10). E o homem que o
``messit'' desejava desencaminhar é aquele que deverá matá-lo, como Ele
claramente enunciou, enaltecido seja Ele: ``A tua mão será a primeira
contra ele para o matar'' (Ibid.), a respeito de que o Sifrei diz:
``Aquele que foi desencaminhado tem por obrigação matá-lo''.

As normas deste preceito estão explicadas no sétimo capítulo de Sanhedrin.

\paragraph{Não amar a pessoa que deseja seduzi-lo para a idolatria}

Por esta proibição quem tiver sido desencaminhado\footnote{Levado a adorar ídolos.} está proibido
de amar aquele que procura enganá-lo ou de prestar atenção ao que ele
diz. Ela está expressa em Suas palavras, enaltecido seja Ele, ``Não lhe
cederás (toveh)'' (Deuteronômio 13:9).

O Sifrei diz: ``Eu poderia pensar, em virtude do princípio geral `E
amarás o teu próximo' (Levítico 19:18)\footnote{Ver o preceito positivo 206.}, que somos
ordenados a amá-lo; por isso as Escrituras dizem `Não lhe cederás
(toveh)'''.

\paragraph{Não diminuir nossa aversão pelo enganador}

Por esta proibição quem tiver sido desencaminhado está proibido de
diminuir seu ódio pelo ``messit''. É seu dever incondicional odiá-lo e
se ele não o fizer estará infringindo um preceito negativo. Esta
proibição está expressa em Suas palavras, enaltecido seja Ele, ``E não
o ouvirás'' (Deuteronômio 13:9), que estão explicadas da seguinte forma:
``Como está escrito `Auxiliá-lo-ás' (Êxodo, 23:5)\footnote{Ver o preceito positivo 202.}, eu poderia pensar que somos ordenados a ajudar o `messit'; por isso as Escrituras dizem: `E não o ouvirás'''.

\paragraph{Não salvar a vida do enganador}

Por esta proibição quem tiver sido desencaminhado está proibido de
salvar a vida do ``messit'' se ele se encontrar em perigo. Ela está
expressa em Suas palavras, enaltecido seja Ele, ``E os teus olhos não
terão piedade dele'' (Deuteronômio 13:9), que são explicadas da seguinte
forma: ``Uma vez que está escrito `Não sejas indiferente quando está em
perigo o teu próximo' (Levítico 19:16), eu poderia pensar que é
proibido ficar indiferente caso um `messit' esteja em perigo; por isso
as Escrituras dizem: `E os teus olhos não terão piedade dele'''.

\paragraph{Não defender um enganador}

Por esta proibição quem tiver sido desencaminhado está proibido de
defender o enganador, e mesmo que ele tenha conhecimento de algum
argumento em seu favor ele está proibido de sugeri-lo ao ``messit'' ou
de expô-lo ele mesmo. Esta proibição está expressa em Suas palavras,
enaltecido seja Ele, ``Não o pouparás'' (Deuteronômio 13:9), que são
explicadas como significando ``Você não deve advogar em seu favor''.

\paragraph{Não omitir uma evidência que seja desfavorável ao enganador}

Por esta proibição quem tiver sido desencaminhado está proibido de
omitir qualquer coisa de que ele tenha conhecimento que seja
desfavorável ao ``messit'' e que contribua para puni-lo. Ela está expressa em Suas
palavras, enaltecido seja Ele, ``E nem esconderás a sua culpa''
(Deuteronômio 13:9), que são explicadas como significando ``Se souberes
algo desfavorável a ele não te é permitido omiti-lo''.

\paragraph{Não tirar proveito de ornamentos que enfeitaram um ídolo}

Por esta proibição somos proibidos de tirar proveito de ornamentos com
os quais um ídolo tenha sido enfeitado. Ela está expressa em Suas
palavras, enaltecido seja Ele, ``Não cobiçarás a prata e o ouro que está
sobre eles'' (Deuteronômio 7:25). A Sifrá explica que são proibidas as
roupas que adornaram um ídolo e baseia essa proibição em Suas palavras,
enaltecido seja Ele, ``Não cobiçarás a prata e ouro que está sobre
eles''.

A punição pela desobediência a esta proibição é o açoitamento. As normas
deste preceito estão explicadas no terceiro capítulo de Abodá Zará.

\paragraph{Não reconstruir uma cidade apóstata}

Por esta proibição somos proibidos de reconstruir uma Cidade Apóstata.
Ela está expressa em Suas palavras, enaltecido seja Ele, ``E será um
montão de ruína para sempre; não será reconstruída jamais''
(Deuteronômio 13:17). O castigo pela reconstrução de qualquer parte dela
--- ou seja, por reedificá-la como ela era antes --- é o açoitamento.


As normas deste preceito estão explicadas no décimo capítulo de Sanhedrin.

\paragraph{Não tirar proveito dos pertences de uma cidade apóstata}

Por esta proibição somos proibidos de usar ou tirar proveito dos bens de
uma Cidade Apóstata. Ela está expressa em Suas palavras, enaltecido seja
Ele, ``E não haverá na tua mão nenhuma coisa do anátema'' (Deuteronômio
13:18).

O castigo por pegar alguma coisa é o açoitamento.

As normas deste preceito estão explicadas no décimo capítulo de Sanhedrin.

\paragraph{Não aumentar nossa fortuna com qualquer coisa que provenha da idolatria}

Por esta proibição somos proibidos de aumentar nossa fortuna com
qualquer coisa que provenha da idolatria; ao contrário, devemos fugir
dela, de seus templos e de seus pertences. Esta proibição está expressa
em Suas palavras, enaltecido seja Ele, ``E não trarás abominação à tua casa''
(Deuteronômio 7:26).

O castigo por beneficiar-se de algum dos pertences é o açoitamento.

Foi deixado claro no final de Macot que aquele que acender lenha de
``Ashera''\footnote{Ver o preceito negativo 13.} está sujeito a dois açoitamentos: um
por ``E não trarás abominação à tua casa'', e um por ``E não haverá''
(Deuteronômio 13:18). Isto deve ser observado.

As normas deste preceito estão explicadas no terceiro capítulo de Abodá
Zará.

\paragraph{Não fazer profecias em nome de um ídolo}

Por esta proibição um homem fica proibido de profetizar em nome de um
ídolo, ou seja, dizer que o Eterno lhe ordenou adorar um ídolo ou que o
próprio ídolo lhe ordenou adorá-lo, prometendo recompensá-lo e
ameaçando puni-lo, como pretextam os profetas de Baal e os de Ashera.

As Escrituras não contêm uma proibição precisa e específica a esse
respeito, quer dizer, uma proibição contra profetizar em nome de um
ídolo, mas prescrevem um castigo, que é a morte, por essa ofensa através
de Suas palavras, enaltecido seja Ele, ``Que falar em nome de outros
deuses, este profeta morrerá'' (Deuteronômio 18:20). Essa morte será por
estrangulamento porque temos uma regra geral de que quando as Escrituras
prescrevem a pena de morte sem maior especificação significa que deve
ser por estrangulamento.

Você já está familiarizado com a regra que expliquei nos catorze
Fundamentos expostos na Introdução a este trabalho, e que foi
estabelecida pelos Sábios pelas palavras ``As Escrituras nunca
prescrevem uma punição sem antes estabelecer uma proibição''. Aqui a
proibição se deriva das palavras ``E o nome de outros deuses não
mencionareis'' (Êxodo 23:13) pois não é impossível que um único
preceito negativo sirva como advertência para várias proibições, embora
ele não seja um ``lav shebikhlalut'', já que foi estabelecido um castigo
diferente para cada caso. Darei esclarecimentos sobre este princípio no
momento apropriado.

As normas deste preceito estão explicadas no décimo primeiro capítulo
de Sanhedrin.

\paragraph{Não fazer falsas profecias}

Por esta proibição somos proibidos de fazer profecias falsas, isto é,
de divulgar, em nome do Eterno, uma profecia que Ele, enaltecido seja,
não tenha dito ou que tenha dito a outro que não aquele que se gaba de tê-la
ouvido e que diz falsamente que o Eterno a comunicou a ele. Este
preceito negativo está expresso em Suas palavras ``Mas o profeta que propositadamente
falar alguma coisa em Meu Nome, que não lhe ordenei falar''
(Deuteronômio 18:20).
Também neste caso o castigo pela desobediência da proibição é o
estrangulamento: o Talmud inclui ``um falso profeta'' na lista dos
transgressores que devem ser estrangulados. Também está escrito no mesmo
lugar: ``Há três que estão sujeitos à morte pela mão do homem: `Que propositadamente
falar alguma coisa em Meu Nome' significa aquele que profetiza o que ele
não ouviu; `Que não lhe ordenei falar', subentendendo que isso foi
ordenado a outro, significa aquele que profetiza o que não foi dito `a
ele'; `Ou que falar em nome de outros deuses' (Ibid.) significa aquele
que profetiza em nome de um ídolo''. Com referência a todos eles está
escrito: `Este profeta morrerá', e toda vez que a pena de morte for
ordenada pelas Escrituras sem especificação, significa que deve ser por
estrangulamento.

As normas da lei relativa a um falso profeta estão explicadas no décimo
primeiro capítulo de Sanhedrin.

\paragraph{Não ouvir as profecias de quem profetiza em nome de um ídolo}

Por esta proibição somos proibidos de ouvir as profecias de alguém que
profetiza em nome de um ídolo, ou seja, não devemos entrar em debate com
ele nem fazer-lhe perguntas, dizendo-lhe ``Qual é o teu milagre e que
prova tens dele?'', como faríamos no caso de alguém que profetizasse em
nome do Eterno. Quando ouvirmos uma pessoa profetizando em nome de um
ídolo, devemos repreendê-lo, pois é nosso dever repreender todo pecador,
e caso ele insista na sua afirmação devemos aplicar-lhe o castigo que
merece, de acordo com a lei das Escrituras, sem levar em consideração
seus milagres ou suas provas.

Esta proibição está expressa em Suas palavras, enaltecido seja Ele,
``Não obedecerás às palavras daquele profeta'' (Deuteronômio 13:4).

As normas deste preceito estão explicadas no décimo primeiro capítulo
de Sanhedrin.

\paragraph{Não ter piedade de um falso profeta}

Por esta proibição somos proibidos de ter pena de um falso profeta ou de
deixar de matá-lo porque ele profetiza em nome do Eterno. Não devemos
temer estar cometendo algum pecado, uma vez que sua falsidade nos foi
provada. Esta proibição está expressa em Suas palavras, enaltecido seja
Ele, ``Não o temerás'' (Deuteronômio 18:22), a respeito das quais diz o
Sifrei: ```Não o temerás': não deixes de declará-lo culpado''.

As normas deste preceito estão explicadas na Introdução ao nosso
``Comentário sobre a Mishná''.

\paragraph{Não adotar os hábitos e costumes dos descrentes}

Por esta proibição somos proibidos de trilhar os caminhos dos descrentes
e de adotar seus costumes, inclusive quanto a suas roupas e suas
reuniões sociais. Esta proibição está expressa em Suas palavras,
enaltecido seja Ele, ``E não andareis nos costumes da nação que Eu hei
de expulsar de diante de vós'' (Levítico 20:23) e está repetida em Suas
palavras ``E não andeis segundo os seus costumes'' (Ibid., 18:3), a
respeito das quais comenta a Sifrá: ``Eu ordenei apenas aquilo que foi
estabelecido como costumes para eles e seus ancestrais''.

A Sifrá diz: ```E não andeis segundo os seus costumes': não sigam seus
costumes sociais, ou seja, as coisas que se transformaram em hábitos
para eles, tal como teatros, circos e arenas --- sendo que esses são
vários tipos de locais de reunião onde eles se juntam para a adoração de
ídolos. Rabi Meir diz: Esses se referem aos `costumes dos Amoritas'
enumerados pelos Sábios. Rabi Yehudá ben Betera diz que você não deve se
barbear ao redor da cabeça ou deixar crescer a franja de seus cabelos ou
raspar o cabelo de sua testa''.

A punição por qualquer uma dessas ofensas é o açoitamento.

Esta proibição está repetida sob outra forma, em Suas palavras
``Guarda-te não te unires a elas com receio de que sejas levado a
segui-las'' (Deuteronômio 12:30) sobre as quais diz o Sifrei: ``
`Guarda-te' é um preceito negativo; `Com receio de que' é um preceito
negativo; `Sejas levado a segui-las' --- com receio de que te compares a
elas e sigas seus costumes e elas se transformem numa armadilha para ti.
Você não deve dizer: Assim como eles se vestem em púrpura, eu me
vestirei em púrpura; assim como eles usam `telusin' --- um tipo de
adorno usado pelos soldados --- eu também usarei o `telusin'''. E você
conhece as palavras do profeta ``Todo que estiver vestido com ornamentos
estrangeiros''\footnote{Tsefânia 1:8.}. A finalidade de tudo isto é que
devemos evitar os idólatras e menosprezar todos os seus hábitos, até
mesmo o de suas roupas.

As normas deste preceito estão explicadas no sexto capítulo de Shabat, e
também no Tosseftá desse Tratado.

\paragraph{Não fazer adivinhações}

Por esta proibição somos proibidos de fazer adivinhações, isto é, fazer
uso de qualquer um dos vários meios de estimular a capacidade de
suposição, pois todos aqueles que têm o poder de predizer o futuro o
fazem porque a capacidade de suposição está altamente desenvolvida
neles, e de uma maneira geral ela opera corretamente; consequentemente,
eles têm um pressentimento do que vai acontecer, sendo que alguns deles
são superiores a outros, assim como entre todos os homens alguns superam
outros numa determinada capacidade da alma.

Contudo, os que têm esses poderes de suposição tentam estimulá-los e
ativá-los por um ou outro meio. Um deles baterá várias vezes no chão com
sua bengala, emitirá gritos estranhos, e se concentrará durante um longo
período de tempo, até que caia num tipo de transe e comece a predizer o
futuro. Uma vez eu vi isso no extremo Oeste. Outro jogará seixos num
pedaço de pele, olhará para eles por um longo tempo e então fará a
profecia --- uma prática comum em todos os lugares que visitei. Outro
ainda jogará um longo cinto de couro no chão, o observará, e fará a
profecia. O objetivo de tudo isto é estimular os poderes que ele
possui; seu ritual não produz nenhum efeito nem lhe fornece nenhuma
informação.

É nesse ponto que a maioria das pessoas se engana. Quando algumas
predições se tornam realidade, eles pensam que essas práticas realmente
revelam o futuro e persistem nesse erro ao ponto de chegar a acreditar
que algumas dessas práticas são a causa dos acontecimentos que se
seguem, tal como os astrólogos costumam acreditar. A arte da astrologia
é, na realidade, semelhante a isto, no sentido de que ambos são meios de
estimular essa capacidade. Portanto, não há dois homens iguais no que se refere à veracidade de suas
profecias, embora muitos possam ser iguais quanto a seu conhecimento da
arte.

Aquele que se envolve com uma destas práticas, ou outras práticas do
mesmo tipo, é chamado de ``kossem'', adivinho; e o Eterno, enaltecido
seja Ele, diz: ``Não se achará entre ti\ldots{} nem adivinho (kossem
kessamim)'' (Deuteronômio 18:10). A este respeito o Sifrei diz: ``O que
é um `kossem'? Aquele que tomar sua bengala em suas mãos e disser: `Devo
ou não devo ir?'''. É com relação a este tipo de adivinhação, comum
naquela época, que o profeta disse: ``Meu povo pede conselho a sua vara
e sua bengala se manifesta a eles''\footnote{Hoshea 4:12.}.

Aquele que cometer esta transgressão --- ou seja, que praticar
adivinhações, e fizer coisas que lhe permitam predizer o futuro ---
estará sujeito ao açoitamento. Isto não se aplica àquele que consulta o
adivinho, embora o ato de consultá-lo seja censurável ao extremo.

As normas deste preceito estão explicadas em vários trechos na Guemará
de Sanhedrin, na Tosseftá de Shabat, e no Sifrei.

\paragraph{Não orientar nossa conduta pelas estrelas}

Por esta proibição somos proibidos de orientar nossa conduta pelas
estrelas, ou seja, não devemos dizer ``Este dia é favorável para um
determinado empreendimento e nós vamos realizá-lo'', ou ``Este dia é
desfavorável para um determinado empreendimento e não o executaremos''.
Esta proibição está expressa em Suas palavras, enaltecido seja Ele,
``Não se achará entre ti\ldots{} nem prognosticador (meonen)'' (Deuteronômio
18:10) e está repetida em Suas palavras ``E não prognosticareis
(teonenu)'' (Levítico 19:26), a respeito das quais diz a Sifrá: ``E não
prognosticareis'': isto se refere àqueles que predizem os tempos ---
sendo que a palavra ``teonenu'' é derivada de ``oná'' (que significa
tempo, época). Isto significa que não deverá ser encontrado entre vocês
um vidente que decide que um momento é favorável e outro não.

O castigo pela transgressão desta proibição --- ou seja, por aconselhar
quanto aos momentos -- é o açoitamento. Isto não se aplica àquele que
faz a consulta, mas tal tipo de consulta também está proibido, além de
ser uma fraude. Todo aquele que deliberadamente escolher uma determinada
época para fazer algo, baseado numa previsão de boa sorte ou sucesso,
também deve ser açoitado porque ele terá realizado uma ação.

Esta proibição também se extende aos truques de ilusão de ótica. Os
Sábios dizem: ```Meonen' se refere a alguém que engana com ilusões de
ótica''; isto inclui um grande número de truques realizados por
prestidigitação, que dão aos homens a ilusão de verem coisas que não
existem. Essas pessoas têm seus truques habituais, como pegar uma corda
e diante dos olhos dos espectadores, guardá-la num canto de seu traje
para então retirar dali uma cobra; ou jogar um anel para cima e depois
retirá-lo da boca de uma das pessoas que estiver diante dele; e há
outros truques de mágica similares, muito populares. Todos os truques
desse tipo são proibidos e todo aquele que os pratica é chamado de
enganador; e como seus truques são um tipo de feitiçaria, sua punição é
o açoitamento. Ele também é um enganador de pessoas e causa grandes
danos ao fazer com que aquilo que não pode realmente existir pareça
possível aos olhos de tolos, homens, mulheres e crianças, habituando-os assim a
aceitar como possíveis coisas impossíveis. Isto deve ser observado.

\paragraph{Não praticar a vidência}

Por esta proibição somos proibidos de praticar a vidência, como fazem
as pessoas que dizem: ``Como eu interrompi minha viagem, não terei
sucesso''; ou ``A primeira coisa que vi hoje foi isto e aquilo: este
será certamente um dia proveitoso para mim''. Tais exemplos são
extremamente comuns entre as pessoas de nações atrasadas.

Todo aquele que permitir que sua conduta seja influenciada por
presságios estará sujeito ao açoitamento, de acordo com Suas palavras,
enaltecido seja Ele, ``Não se achará entre ti\ldots{} nem adivinho, nem
feiticeiro (menahesh)'' (Deuteronômio 18:10). Isso está repetido nas
palavras ``Não augurareis (tenahashu)'' (Levítico 19:26); e o Sifrei
diz: ```Menahesh': como aquele que diz `O pão caiu de minha boca', `O
bastão caiu de minha mão', `Uma cobra passou pela minha direita', ou
`Uma raposa passou pela minha esquerda' ''. E na Sifrá lê-se: ```Não
augurareis', como fazem aqueles que tiram presságios de uma doninha, ou
dos pássaros, ou das estrelas, e assim por diante''.

As normas deste preceito também estão explicadas no sétimo capítulo de
Shabat e na Tosseftá desse Tratado.

\paragraph{Não praticar feitiçaria}

Por esta proibição somos proibidos de praticar feitiçaria. Ela está
expressa em Suas palavras, enaltecido seja Ele, ``Não se achará entre
ti\ldots{} nem feiticeiro (mekhashef)'' (Deuteronômio 18:10).

Aquele que deliberadamente desobedecer a esta proibição está sujeito ao
apedrejamento. Aquele que a transgredir involuntariamente deve levar um
Sacrifício Determinado de Pecado. As Escrituras dizem: ``Feiticeira não
deixarás viver'' (Êxodo 22:17).

As normas deste preceito estão explicadas no sétimo capítulo de Sanhedrin.

\paragraph{Não praticar a arte do encantador}

Por esta proibição somos proibidos de praticar a arte do encantador
(``hober''), ou seja, dizer palavras de encantamento que se supõe que
tenham determinados efeitos positivos ou negativos. Ela está expressa em
Suas palavras, enaltecido seja Ele, ``Não se achará entre ti\ldots{} nem
encantador (hober haber)'' (Deuteronômio, 18:10-11), sobre as quais o
Sifrei diz: ```hober haber' significa um encantador de serpentes ou de
escorpiões''; quer dizer, ele recita palavras de encantamento para eles
para que --- acredita ele --- eles não o mordam, ou se ele já tiver sido
mordido, para diminuir a dor.

O castigo pela desobediência desta proibição é o açoitamento.

As normas deste preceito também estão explicadas no sétimo capítulo de
Shabat.

\paragraph{Não consultar um necromante que use o ``ob''}

Por esta proibição somos proibidos de consultar um necromante que use o
``ob'', e de tentar obter informações dele. Ela está expressa em Suas
palavras, enaltecido seja Ele, ``Não se achará entre ti\ldots{} nem
necromante (shoel ob)'' (Deuteronômio 18:10-11).

A desobediência a esta proibição --- ou seja, consultar um necromante
que se utilize de um ``ob'' --- não é punida com a morte, mas mesmo
assim a sua prática é proibida.

\paragraph{Não consultar um feiticeiro que se utilize do ``yidoa''}

Por esta proibição somos proibidos de consultar um feiticeiro que se
utilize do ``yidoa'' e de tentar obter informações dele. Ela está
expressa em Suas palavras, enaltecido seja Ele, ``Não se achará entre
ti\ldots{} ou yideoni'' (Deuteronômio 18:10-1 1). A Sifrá diz: ```Não vos
volteis para as magias (ob) e para as feitiçarias (yideoni)' (Levítico
19:31): o `ob', ou seja, o piton que fala de debaixo de suas axilas, e o
`yideoni', que fala de dentro de sua boca, são punidos com o
apedrejamento e aquele que os consulta é punido com o
açoitamento''.\footnote{Se ele se conduzir da maneira indicada pelo feiticeiro.}

\paragraph{Não tentar obter informações com os mortos}

Por esta proibição somos proibidos de tentar obter informações com os
mortos --- com aqueles que se pensa que estão mortos embora comam e
tenham sensações --- pensando que se alguém fizer determinadas coisas e
se vestir de uma determinada maneira, os mortos virão durante seu sono e
responderão suas perguntas. Esta proibição está expressa em Suas
palavras, enaltecido seja Ele, ``Não se achará em ti\ldots{} nem quem
consulte os mortos (doresh el hametim)'' (Deuteronômio 18:10-11), a
respeito das quais a Guemará de Sanhedrin diz: ```Doresh el hametim'
significa aquele que se deixa morrer de fome e passa a noite num
cemitério para que o espírito de um demônio possa descansar nele''.

O castigo pela transgressão desta proibição é o açoitamento.

\paragraph{As mulheres não devem usar roupas ou adornos masculinos}

Por esta proibição também somos proibidos de seguir os costumes dos
hereges no que se refere a mulheres usarem roupas ou adornos masculinos.
Ela está expressa em Suas palavras, enaltecido seja Ele, ``Não haverá
traje de homem na mulher'' (Deuteronômio 22:5).

Toda mulher que usar um adorno que é sabido ser usado só pelos homens
naquela região está sujeita ao açoitamento.

\paragraph{Os homens não devem usar roupas ou adornos femininos}

Por esta proibição os homens também estão proibidos de se enfeitarem
com adornos femininos. Ela está expressa em Suas palavras, enaltecido
seja Ele, ``E não usará o homem vestido de mulher'' (Deuteronômio
22:5). O homem que colocar adornos ou trajes sabidos ser de uso
exclusivamente feminino naquela região também é punido com o
açoitamento.

Você deve saber que este costume --- ou seja, as mulheres se enfeitarem
com adornos masculinos e os homens com adornos femininos --- é algumas
vezes adotado com o intuito de despertar o desejo carnal, como é comum
entre as nações, e algumas vezes com a finalidade de adoração de um
ídolo, como está explicado nos livros dedicados a esse assunto. Também é
uma prática comum estipular, com relação à confecção de determinados
talismãs, que se aquele que os fizer for um homem ele deve usar trajes
femininos e enfeitar-se com ouro, pérolas e coisas desse tipo, e se for
mulher, ela deve usar armadura e rodear-se de armas. Isto é bem sabido
daqueles que são conhecedores do assunto.

\paragraph{Não fazer marcas em nossos corpos}

Por esta proibição somos proibidos de fazer qualquer marca --- azul,
vermelha ou de qualquer outra cor --- em nossos corpos, assim como fazem
os idólatras, como é comum entre os Koptim até hoje. A proibição está
expressa em Suas palavras, enaltecido seja Ele, ``E escrita de tatuagem
não poreis em vós'' (Levítico 19:28).

O castigo pela desobediência desta proibição é o açoitamento. As normas
deste preceito estão explicadas no final do Tratado Macot.

\paragraph{Não usar roupas de lã e linho}

Por esta proibição somos proibidos de usar uma roupa tecida com lã e
linho, como os sacerdotes dos ídolos costumavam fazer naquela
época\footnote{Isto é, na época da revelação da Torá.}. Ela está expressa em Suas palavras,
enaltecido seja Ele, ``Não te vestirás com estofos misturados (shaatnez)
de lã e linho juntamente'' (Deuteronômio 22:11). Este costume é comum
entre os monges Coptas do Egito àtualmente.

O castigo por transgredir esta proibição é o açoitamento.

As normas deste preceito estão explicadas nos Tratados Quilaim e Shabat,
e no final de Macot.

\paragraph{Não raspar os cabelos das têmporas}

Por esta proibição somos proibidos de cortar o cabelo das têmporas. Ela
está expressa em Suas palavras, enaltecido seja Ele, ``Não cortareis o
cabelo de vossa cabeça em redondo'' (Levítico 19:27). O objetivo desta
proibição também é para que não imitemos os idólatras, porque eles
tinham o costume de raspar apenas o cabelo de suas têmporas. De acordo
com isso, os Sábios julgaram necessário explicar no Tratado Yebamot que
``Barbear toda a cabeça é considerado como `arredondar''', para que não
se argumente que o objetivo desta proibição é impedir-nos de raspar os
cabelos das têmporas assim como o resto do cabelo (da cabeça), como
fazem os sacerdotes idólatras, mas que raspar todo o cabelo não é
imitação deles. Por esse motivo os Sábios nos dizem que em circunstância
alguma nos é permitido raspar os cabelos das têmporas, quer se raspe
apenas as têmporas ou todo o cabelo da cabeça; e que cada têmpora é
punível com o açoitamento, de forma que aquele que barbear toda sua
cabeça deve ser açoitado duas vezes. Apesar disso, não contamos esta
proibição como dois preceitos, porque há apenas uma negação e não duas.
Se Ele tivesse dito: ``Você não deve arredondar o canto direito da
cabeça, nem o canto esquerdo'', e víssemos que os Sábios prescreveram
dois castigos, teria sido possível considerá-la como dois preceitos; mas
como é um único assunto abrangido numa única expressão, trata-se na
realidade de um único preceito. Embora esta proibição seja interpretada
como abrangendo várias partes do corpo, e estejamos sujeitos a
açoitamento por cada uma dessas partes, separadamente, isso não a
transforma necessariamente em mais de um preceito.

As normas deste preceito estão explicadas no final de Macot. Ele não é
obrigatório para as mulheres.

\paragraph{Não raspar a barba}

Por esta proibição somos proibidos de raspar a barba, que tem cinco
partes: a maxila superior direita, a maxila superior esquerda, a maxila
inferior direita, a maxila inferior esquerda e a ponta da barba. A
proibição completa está expressa nas palavras ``E não raspareis os
cantos de vossa barba'' (Levítico 19:27) porque todas as partes da
barba estão incluídas no termo ``barba''. As Escrituras não dizem ``Nem
raspareis vossa barba'', mas sim ``Nem raspareis `os cantos' de vossa
barba'', significando que não se deve raspar nem um canto da barba, a
qual, de acordo com a Tradição, compõe-se de cinco ``cantos'', como
explicado acima, e fica-se sujeito a cinco açoitamentos se se raspar
toda a barba, ainda que se raspe a barba toda de uma só vez.

A Mishná diz: ``Cinco vezes pela barba: duas vezes pelo lado direito,
duas vezes pelo esquerdo e uma vez pela parte inferior. Rabi Eliezer
diz: Se ele tirou toda a barba num só movimento, ele está sujeito a um
único castigo; disso o Talmud conclui: ``Portanto, Rabi Eliezer
considera o todo uma proibição''. Assim, temos uma prova clara que o
Primeiro Sábio\footnote{Que discorda de Rabi Eliezer.} é de opinião que há cinco
proibições, e essa é a lei.

Esse era também um costume dos sacerdotes idólatras, e é sabido que
atualmente os sacerdotes Europeus raspam suas barbas.

A razão pela qual isso não deve ser contado como cinco preceitos é que a
proibição trata de um único assunto numa única expressão, como
explicamos no preceito precedente.

As normas deste preceito estão explicadas no final de Macot. Ele não é
obrigatório para as mulheres.

\paragraph{Não fazer cortes em nossa carne}

Por esta proibição somos proibidos de fazer cortes em nossa carne, como
fazem os idólatras. Ela está expressa em Suas palavras, enaltecido seja
Ele, ``Não fareis cortes em vossa carne (lo titgodedu)'' (Deuteronômio
14:1), e também, sob outra forma, em Suas palavras ``E incisões (seret)
por um morto, não fareis em vossa carne'' (Levítico 19:28).

A Guemará de Yebamot explica que ``Lo titgodedu'' é obrigatório
pelo seu próprio contexto, já que o Todo Misericordioso disse ``Não vos
ferireis por causa dos mortos''.

A Guemará de Macot diz: ```Seritá' (cortar com a mão) e `guedidá'
(cortar com um instrumento) são a mesma coisa''. Também está explicado
lá que quem o fizer por causa dos mortos, seja com a mão ou com um
instrumento, estará sujeito ao açoitamento; na prática da idolatria,
ele estará sujeito ao castigo se o fizer com um instrumento, mas estará
isento se o fizer com a mão, pois no livro dos profetas encontramos o
seguinte: ``Eles se cortam (vayitgodedu) de acordo com seus métodos,
com espadas e lanças''\footnote{I. Reis 18:28.}.

O Talmud diz que este preceito negativo também proíbe dividir as pessoas
e criar facções e discórdia, interpretando ``lo titgodedu'' como ``Não
deveis separar-vos em facções (agudot)''. O verdadeiro significado do
versículo, contudo, é, como explicam os Sábios, ``Não vos ferireis por
causa dos mortos''; o outro é meramente um ``derash''.

Da mesma forma, o fato de eles dizerem que ``aquele que for inflexível
numa discussão viola um preceito negativo, pois está dito que ``Para que
não seja como Korah, e como sua congregação'' (Números 17:5), também é
um ``derash'', pois o verdadeiro objetivo do versículo é o de dissuadir.
Da forma como os Sábios explicam o versículo, ele contém uma declaração
negativa e não uma proibição, sendo que a interpretação deles é que o
Eterno declara que todo aquele que no futuro contestar a autoridade dos
``Cohanim'' e reivindicar o sacerdócio para si mesmo não será castigado
com a punição determinada para Korah --- ou seja, não será tragado pela
terra --- mas será punido ``conforme tinha falado o Eterno, por
intermédio de Moisés'' (Números 17:5), isto é, com a lepra, como quando
Ele disse a Moisés: ``Leva, por favor, a tua mão ao teu peito''. (Êxodo
4:6) e como está dito do rei Uziah\footnote{II. Cron. 26:16-21.}.

Para voltar ao assunto deste preceito, suas normas estão explicadas no
final de Macot, e o castigo pela desobediência desta proibição é o
açoitamento.

\paragraph{Não se fixar na terra do Egito}

Por esta proibição somos proibidos de estabelecer-nos na terra do Egito,
a fim de que não aprendamos a heresia dos Egípcios nem sigamos seus
costumes, que são repulsivos para a Torah. Esta proibição está expressa
em Suas palavras, enaltecido seja Ele, ``Nem fará voltar o povo ao
Egito'' (Deuteronômio 17:16). Ela aparece três vezes nas Escrituras e, de acordo com os
Sábios, ``Em três lugares a Torah avisa Israel para não retornar ao
Egito; mas três vezes eles retornaram e três vezes foram punidos''. Nós
já mencionamos o primeiro dos três avisos; o segundo está expresso em
Suas palavras ``Pelo caminho que te tenho dito: Não voltarás mais para
vê-lo'' (Deuteronômio 28:68); e o terceiro em Suas palavras ``Porque os
Egípcios que vedes hoje não volvereis a vê-los nunca mais'' (Êxodo
14:13). Embora de acordo com seu significado literal esta seja uma
afirmação, ela é tradicionalmente entendida como sendo uma proibição.

No final de Guemará de Sucá está explicado que Alexandria está incluída
entre os lugares onde é proibido estabelecer-se e que a totalidade da
terra do Egito, na qual não podemos viver, compreende uma área de 400
parasangas quadradas medidas a partir do mar em Alexandria. Contudo, é
permitido atravessar esta área para fins comerciais, ou passar por ela a
caminho de outro país. Está dito explicitamente no Talmud de Jerusalém:
``Você não deve retornar para estabelecer-se, mas pode retornar para
fins de comércio, negócios e conquista do país''.

\paragraph{Não aceitar opiniões contrárias às ensinadas na Torah}

Por esta proibição somos proibidos de exercer a liberdade dos
pensamentos no que se refere a aceitar opiniões contrárias às que nos
são ensinadas pela Torah; devemos, ao contrário, limitar nossos
pensamentos, e levantar uma barreira em torno deles, formada pelos
preceitos positivos e negativos da Torah. Esta proibição está expressa
em Suas palavras, enaltecido seja Ele, ``E não errareis indo atrás de
vosso coração e atrás de vossos olhos, atrás dos quais vós andais
errando'' (Números 15:39), sobre as quais diz o Sifrei: ```E não
errareis indo atrás de vosso coração' --- isto significa heresia, pois
está escrito: `E se eu encontrar mais amargo do que a
morte'\footnote{Eclesiástico, 7:26.}. `E atrás de vossos olhos' --- isto
significa prostituição, pois está escrito: `E Sansão disse a seu
pai'''\footnote{Juízes 14:3.}. ``Prostituição'' aqui significa perseguir
e pensar constantemente em prazeres físicos e indulgências.

\paragraph{Não fazer uma aliança com as Sete Nações Idólatras de Canaã}

Por esta proibição somos proibidos de fazer uma aliança com os hereges,
ou seja, com as Sete Nações\footnote{As Sete Nações de Canaã: os hiteus, os guirgasheus, os emoreus, os cananeus, os periseus, os hiveus e os jebuseus.}, e de deixá-los tranquilos em sua heresia. Ela está expressa em Suas palavras,
enaltecido seja Ele, ``Não farás aliança alguma com elas'' (Deuteronômio
7:2).

Já explicamos, ao tratar do preceito positivo 187 que a guerra contra as Sete Nações, e os outros preceitos relativos a elas devem ser
incluídos\footnote{Nos 613 preceitos.}, e não são de tempo limitado.

\paragraph{Não poupar a vida de um homem das Sete Nações Idólatras}

Por esta proibição somos proibidos de poupar a vida de qualquer homem
que pertença a uma das Sete Nações para evitar que eles corrompam as
pessoas e as levem para o caminho errôneo da idolatria. Esta proibição
está expressa em Suas palavras, enaltecido seja Ele, ``Não deixarás com
vida todo que tiver alma'' (Deuteronômio 20:16). Matá-los constitui um
preceito positivo, como explicamos ao tratar do preceito positivo 187.

Todo aquele que transgredir esta proibição, deixando de matar todo
aquele que ele poderia ter morto estará infringindo um preceito
negativo.

\paragraph{Não demonstrar compaixão para com os idólatras}

Por esta proibição somos proibidos de demonstrar compaixão para com os
idólatras ou de elogiar qualquer coisa que lhes pertença. Ela está
expressa em Suas palavras, enaltecido seja Ele, ``E não terás
misericórdia deles (lo tehonem)'' (Deuteronômio 7:2), que são
tradicionalmente interpretadas como significando: ``Não lhes
concedereis nenhuma graça (hen)''. E ainda que um idólatra tenha uma boa
aparência, somos proibidos de dizer que ``Ele tem boa aparência'' ou
``ele tem um belo rosto'', como está explicado em nossa Guemará.

A Guemará de Abodá Zará no Talmud de Jerusalém diz que há um preceito
negativo que proíbe conceder-lhes alguma graça.

\paragraph{Não permitir que idólatras residam em nossa terra}

Por esta proibição somos proibidos de permitir que idólatras residam em
nossa terra, para que não aprendamos suas heresias. Ela está expressa em
Suas palavras, enaltecido seja Ele, ``Não morarão em tua terra, quiçá te
façam pecar contra Mim'' (Êxodo 23:33). Se algum idólatra desejar
permanecer em nossa terra não devemos permitir que o faça a menos que
ele renegue a idolatria; nesse caso ele poderá se tornar um residente.
Tal pessoa é conhecida como um ``guer toshab'', o que significa que ele
é um prosélito apenas no sentido de que lhe é permitido residir em
nossa terra. Assim, os Sábios dizem: ``Quem é um `Suer toshab'? De
acordo com Rabi Yehudá, é aquele que renegou a idolatria''.

``Contudo, um adorador de ídolos não deve residir entre nós, nem podemos
vender-lhe um imóvel ou alugar a ele: isto é rigorosamente interpretado
como significando `não permitirás que se fixem (hanaya) na terra'''.

As normas deste preceito estão explicadas em Sanhedrin e Abodá Zará.

\paragraph{Não se unir pelo matrimônio a hereges}

Por esta proibição somos proibidos de unir-nos pelo matrimônio a
hereges. Ela está expressa em Suas palavras, enaltecido seja Ele, ``E
não te aparentarás com elas'' (Deuteronômio 7:3). O termo
``aparentarás'' é explicado da séguinte forma: ``Tua filha não darás a
seu filho, e sua filha não tomarás para teu filho'' (Ibid.). Está
exposto claramente no Tratado Abodá Zará que ``A Torah proíbe a união
pelo matrimônio''.

O castigo por desobedecer a esta proibição varia de acordo com as
circunstâncias. Se um homem mantiver publicamente um relacionamento com
uma mulher pagã, aquele que o matar quando ele estiver cometendo a
transgressão estará desse modo aplicando o castigo que Pinhas aplicou a
Zimri. Por isso a Mishná diz: ``Se um homem coabitar com uma mulher
pagã, ele será punido por fanáticos''. Isto, contudo, só é permitido
sob certas condições, ou seja, quando a transgressão for feita
abertamente, e enquanto o ato estiver acontecendo, como naquele
caso\footnote{Como no caso de Pinhas e Zimri (Números 25:7).}. Contudo, quando o ato não for cometido
publicamente ou não for punido pelos fanáticos naquele momento, o
transgressor está sujeito à extinção, embora isto não esteja prescrito
pela Torah. Aparece no Talmud a pergunta: ``E se os fanáticos não o
punirem?'' e a resposta é que ele está sujeito à extinção, como consta
nas palavras das Escrituras: ``Pois Yehudá tinha profanado a santidade
do Eterno que o amava, e tinha se casado com a filha de um deus
estranho. O Eterno destruirá o homem que fizer isso, o mestre e o
aprendiz''\footnote{Mal. 2:11-12.}. Está dito: ``Isto mostra que ele está
sujeito à extinção''. Portanto, quando ficar provado que um homem teve
relação com uma mulher pagã diante de testemunhas, apesar de ter sido
categoricamente advertido, ele está sujeito a açoitamento pela
autoridade da Torah. Você deve observar isso.

A lei a respeito de todos esses assuntos está explicada em Abodá Zará e Sanhedrin.

\paragraph{Não se unir pelo matrimônio a um homem Amonita ou Moabita}

Por esta proibição fica proibido o casamento com um varão amonita ou
moabita, mesmo depois que ele tenha se tornado prosélito. Ela está
expressa em Suas palavras, enaltecido seja Ele, ``Não entrará nenhum
Amonita, e nem Moabita na congregação do Eterno'' (Deuteronômio 23:4).

O castigo pela contravenção desta proibição é o açoitamento, ou seja,
se um homem amonita ou moabita prosélito se casar com uma mulher
israelita ambos estão sujeitos ao açoitamento, de acordo com a lei das
Escrituras.

As normas deste preceito estão explicadas no, oitavo capítulo de
Yebamot, e no final de Kidushin.


\paragraph{Não excluir os descendentes de Esaú}

Por esta proibição somos proibidos de excluir os descendentes de
Esaú\footnote{Irmão gêmeo de Jacob (ou Israel).} depois que eles tiverem se tornado
prosélitos, ou seja, de recusarmos a unir-nos a eles pelo matrimônio.
Esta proibição está expressa em Suas palavras, enaltecido seja Ele,
``Não abominarás o Edumeu, porque é teu irmão'' (Deuteronômio 23:8).

\paragraph{Não afastar os descendentes dos egípcios}

Por esta proibição somos proibidos de afastar os egípcios e de
recusarmos a unir-nos a eles pelo matrimônio, depois que eles tiverem se
tornado prosélitos. Esta proibição está expressa em Suas palavras,
enaltecido seja Ele, ``Nem abominarás o egípcio'' (Deuteronômio 23:8).

As normas destes dois preceitos --- relativos aos egípcios e aos edumeus
--- estão explicadas no oitavo capítulo de Yebamot, e no final de
Kidushin.

\paragraph{Não oferecer a paz a Amon nem a Moab}

Por esta proibição somos proibidos de oferecer a paz a Amon ou Moab. O
Eterno nos ordenou que quando estivermos a ponto de sitiar uma cidade
devemos pedir a seus habitantes, antes de iniciar as hostilidades, que
eles se submetam e não guerreiem conosco; e se eles nos entregarem a
cidade, ficamos proibidos de entrar em guerra com eles ou matá-los,
como explicamos ao tratar do preceito positivo 190. Mas no caso de Amon
e de Moab não devemos seguir esse procedimento; o Eterno nos proibiu de
oferecer-lhes a paz e de pedir para que se submetam. Esta proibição
está expressa em Suas palavras, enaltecido seja Ele, ``Não lhes
procurarás nem paz, nem bem''. (Deuteronômio 23:7). A este respeito o
Sifrei diz: ``Eu poderia pensar que a regra que diz: `Quando te
aproximares de uma cidade para pelejar contra ela, oferecer-lhe-ás a
paz' (Ibid., 20:10) deve ser aplicada também neste caso. Por isso as
Escrituras dizem: `Não lhes procurarás nem paz'. E como está escrito:
`No lugar que escolher' (Ibid., 23:17), eu poderia pensar que aqui
novamente a regra também se aplica. Por isso as Escrituras dizem: `Nem
bem, em todos os teus dias para sempre'''.

\paragraph{Não destruir árvores frutíferas durante um assédio}

Por esta proibição somos proibidos de destruir árvores frutíferas
durante um cerco a fim de causar escassez e sofrimento aos habitantes da
cidade sitiada. Ela está expressa em Suas palavras, enaltecido seja Ele,
``Não destruirás o seu arvoredo\ldots{} pelo que não o cortarás'' (Deuteronômio 20:19). Toda
destruição está incluída nesta proibição; por exemplo, todo aquele que
queimar uma roupa ou quebrar um recipiente desnecessariamente estará
desobedecendo à proibição ``Não destruirás'', e estará sujeito ao
açoitamento.

No final de Macot está explicado que aquele que cortar ``árvores boas''
está sujeito ao açoitamento. A esse respeito os Sábios comentam: A
advertência quanto a isso está expressa nas palavras das Escrituras
``Podeis comer de seus frutos, mas não deveis cortá-las''.

As normas deste preceito estão explicadas no segundo capítulo de Baba
Batra.

\paragraph{Não temer os hereges em tempos de guerra}

Por esta proibição somos proibidos de temer os hereges em tempos de
guerra, ou de recuar diante deles; ao contrário, é nosso dever ser
corajosos e reunir todas as nossas forças para resistir na linha de
batalha. Todo aquele que fugire recuar infringirá o preceito negativo
expresso em Suas palavras ``Não te aquebrantarás. diante deles''
(Deuteronômio 7:21) e também em Suas palavras ``Não os temais'' (Ibid.,
3:22).

O preceito contra recuar ou ceder terreno em batalha está repetido
muitas vezes nas Escrituras porque esta é uma das situações em que lhe é
possível defender a verdadeira fé.

As normas deste preceito estão explicadas no oitavo capítulo de Sotá.

\paragraph{Não esquecer o que Amalec nos fez}

Por esta proibição somos proibidos de esquecer o que Amalec nos fez e de
como ele nos atacou sem ter sido provocado\footnote{Ver Êxodo 17:8-13.}. Nós já
explicamos, ao falar do preceito positivo 189, que é um preceito
positivo lembrar o que Amalec nos fez e manter vivo nosso ódio por ele.
Da mesma forma somos proibidos, por um preceito negativo, de
negligenciar ou esquecer este assunto. Esta proibição está expressa em
Suas palavras, enaltecido seja Ele, ``Não te esquecerás'' (Deuteronômio
25:19). O Sifrei diz: ``Lembra-te (Ibid.,17) --- com palavras; `Não te
esquecerás' --- dentro de teu coração''. Quer dizer, não devemos
abrandar nosso ódio por Amalec, nem retirá-lo de nossos corações.

\paragraph{Não blasfemar o grande Nome}

Por esta proibição somos proibidos de blasfemar o grande Nome,
enaltecido seja Ele muito, muito acima de todas as palavras dos hereges.
Isto é o que é chamado de ``abençoar o Nome''. As Escrituras prescrevem
expressamente o apedrejamento como punição pela desobediência da
proibição expressa em Suas palavras, enaltecido seja Ele, ``E aquele que
blasfemar o nome real do Eterno, certamente será morto; toda a
congregação o apedrejará'' (Levítico 24:16). Mas as Escrituras não
destacam este pecado como uma proibição expressa; ele está incluído na proibição geral expressa em Suas palavras
``Aos juízes não maldigas'' (Êxodo 22:27). A Mekhiltá diz: ``Nas
palavras das Escrituras `Aquele que blasfemar o nome real do Eterno,
certamente será morto' ouvimos a penalidade por este pecado, mas não
ouvimos a sua proibição. Por isso as Escrituras dizem `Aos juízes não
maldigas'''. E, de acordo com a Sifrá, ``A penalidade pelo Nome Especial
é a morte e por um dos adjetivos é o açoitamento''. A Mekhiltá diz
ainda: ```Aos juízes não maldigas': esta é a proibição contra
blasfemar''.

As normas deste preceito estão explicadas no sétimo capítulo de Sanhedrin.

Você deve saber que este tipo de proibição, que engloba dois ou três
assuntos específicos, não está incluído na categoria de ``lav
shebikhlalut'' porque as Escrituras especificam a punição por cada
transgressão separadamente; consequentemente, sabemos que cada um deles
é proibido e é o objeto de um preceito negativo, como explicamos na
Introdução deste livro. Como é nosso princípio que não se prescreve
nenhum castigo a menos que uma proibição o tenha precedido, somos
obrigados a procurar a proibição, a qual algumas vezes descobrimos
através de uma das Leis de Interpretação, e outras vezes encontramos
numa passagem que trata de outro assunto, como explicamos na
Introdução. Um ``lav shebikhlalut'' subsiste apenas quando não
encontramos nenhuma outra base para a proibição de qualquer um dos atos
em questão a não ser naquele determinado preceito negativo, como
explicamos no Nono Fundamento. Contudo, quando nos tiver sido ensinado
que isto ou aquilo é proibido --- já que as Escrituras dizem que aquele
que fizer uma determinada coisa incorrerá num determinado castigo ---
não importa se a advertência foi mencionada explicitamente ou se foi
deduzida por raciocínio específico ou geral. Você deve compreender este
princípio pois haverá muitos preceitos mais aos quais ele se aplica.

\paragraph{Não violar um ``shebuat bitui''}

Por esta proibição somos proibidos de violar um ``shebuat bitui''. Ela
está expressa em Suas palavras, enaltecido seja Ele, ``E não jurareis
falso em Meu nome'' (Levítico 19:12).

O termo ``shebuat bitui'' significa um juramento através do qual juramos
fazer ou não fazer algo que a Lei não ordene nem proíba. Devemos
cumprir um juramento desse tipo e as palavras ``Não jurareis falso em
Meu nome'' nos proíbem de violá-lo.

A Guemará de Shebuot diz: ``O que é um `shebuat sheker'? Jurar o
contrário. Isto foi corrigido para: Jurar e inverter''; ou seja, jurar
fazer alguma coisa e fazer o contrário daquilo que se jurou.

A Guemará explica no terceiro capítulo de Shebuot e também no Tratado
Temurá que ``shebuat sheker'' (um falso juramento) é o não cumprimento
de um ``shebuat bitui''. Essa explicação está exposta da seguinte forma:
``Esse falso juramento, de que tipo é ele?'' quer dizer, de acordo com o
contexto, o que se quer dizer por um falso juramento que não acarreta
nenhuma ação? ``Devemos dizer que significa jurar não comer e depois
fazê-lo? Mas neste caso a ação foi realizada. Devemos, então, concluir
que o que se quer dizer é jurar comer e não comer. Mas existe uma
penalidade de açoitamento? Seguramente foi-nos dito\ldots{} etc''.

O castigo pela transgressão voluntária desta proibição é o açoitamento; se alguém a transgredir involuntariamente ele deve oferecer um
Sacrifício de Maior ou Menor Valor, como explicamos no preceito
positivo 72. Isto se baseia na seguinte passagem do terceiro capítulo de
Shebuot: ``Este é um `shebuat bitui', por cuja violação voluntária
fica-se sujeito ao açoitamento; se ela for violada inconscientemente
deve-se oferecer um Sacrifício de Maior ou Menor Valor''. As normas
deste preceito estão explicadas nessa passagem.

Quando afirmei que o castigo pela transgressão voluntária deste preceito
é o açoitamento você deve saber que eu não quis dizer que há um pecado
punido com o açoitamento, mesmo que ele não tenha sido cometido
deliberadamente. Toda vez que você me ouvir afirmar que uma determinada
transgressão é punível com o açoitamento --- quer seja no que precede
ou no que se segue --- você deve saber que isto se aplica unicamente a
um pecado cometido voluntariamente, na presença de testemunhas, e
desafiando uma advertência formal, como está explicado no Tratado
Sanhedrin com relação às determinações sobre testemunhas e à
advertência formal. Aquele que pecar sem querer ou sob coação, ou em
virtude de uma falsa informação em circunstância alguma estará sujeito
ao açoitamento ou à extinção, e menos ainda à execução judicial. Isto
se aplica a todos os preceitos, e deve ser registrado.

No caso de alguns preceitos, nós realmente afirmamos que a violação
voluntária é punível com o açoitamento ou a morte, porque o mesmo
pecado cometido involuntariamente acarreta oferta de um sacrifício. A
razão disso é que nem todos os pecados cometidos involuntariamente
acarretam a oferta de um sacrifício. Mas toda vez que a penalidade por
uma transgressão for o açoitamento, a extinção ou a execução judicial,
não se fica sujeito à punição a menos que o pecado tenha sido cometido
na presença de testemunhas e desafiando-se uma advertência formal. É
sabido que o único objetivo da advertência formal é para que se possa
fazer a distinção entre a violação por ignorância ou proposital.

Você deve conhecer este princípio, e não espere que eu torne a repeti-lo.

\paragraph{Não fazer um ``shebuat shav''}

Por esta proibição somos proibidos de fazer um ``shebuat shav'' (um
juramento em vão). Ela está expressa em Suas palavras, enaltecido seja
Ele, ``Não jurarás em nome do Eterno, teu Deus, em vão'' (Êxodo 20:7).
Ela nos proíbe de jurar que um objeto existente é o que de fato ele não
é, ou que algo impossível existe, ou de jurar violar qualquer um dos
preceitos da Torah. Da mesma forma, jurar um fato evidente, que nenhuma
pessoa instruída negaria ou questionaria, como por exemplo jurar-se
pelo Eterno que tudo aquilo que for degolado morrerá, isso também é um
caso de tomar o nome do Eterno em vão. Nas palavras da Mishná: ``O que é
um `shebuat shav'? Jurar o contrário dos fatos conhecidos pelo homem''.

A punição pela transgressão deliberada desta proibição é o açoitamento;
aquele que a transgredir involuntariamente está isento, assim como
todos os demais que forem culpados de transgredir um preceito negativo,
como explicamos anteriormente. Está dito em Shebuot que o castigo por um
``shebuat shav'' é o açoitamento se a ofensa for deliberada, e que não
há punição se ela for inconsciente. As normas deste preceito estão ali
explicadas.

\paragraph{Não profanar o nome de Deus}

Por esta proibição somos proibidos de profanar o Nome. Isto é o
contrário da santificação do Nome, que nos é ordenada pelo nono preceito
positivo, e que explicamos ali. Esta proibição está expressa em Suas
palavras, enaltecido seja Ele, ``E não profanareis o nome de Minha
santidade'' (Levítico 22:32).

Este pecado abrange três tipos de ações, duas das quais são possíveis
para qualquer um e a terceira apenas para certas pessoas.

O primeiro tipo de ação, que é possível para qualquer pessoa, é este.
Todo aquele que, em época de perseguição, é forçado a transgredir um dos
preceitos, se seu perseguidor tem em mente fazê-lo cometer um pecado de
maior ou menor gravidade, ou todo aquele que for forçado, mesmo sem ser
em épocas de perseguição, a cometer o pecado de idolatria, incesto ou
derramamento de sangue, deve sacrificar sua vida e submeter-se à morte
em vez de transgredir, como explicamos em relação ao nono preceito
positivo. Se ele cometer o pecado e escapar assim da morte, ele terá
profanado o Nome e infringido este preceito negativo; e se isso
acontecer em público, isto é, na presença de dez israelitas, ele estará
profanando o Nome publicamente e pecando contra Suas palavras,
enaltecido seja Ele, ``Não profanareis o nome de Minha santidade'' e seu
crime será muito grave. Mas ele não será punido com o açoitamento,
porque agiu sob coação, e um tribunal só tem o direito de impor o
açoitamento ou a morte a alguém que tenha pecado intencionalmente, de
sua livre vontade, diante de testemunhas e desafiando uma advertência
formal. Como diz o Sifrei daquele que entrega um filho a Molekh:
```Porei Eu a Minha ira contra aquele homem' (Levítico 20:5): os Sábios
dizem `aquele homem', mas não o horrlem que pecar sob coação ou sem
intenção de fazê-lo, ou devido a uma falsa informação''. Assim foi
deixado claro que aquele que adora ídolos sob pressão não está sujeito à
extinção, e menos ainda à execução judicial, mas é culpado de ter
profanado o Nome.

O segundo tipo de ação, que também é possível a qualquer um, é cometer
uma transgressão que, embora não motivada por cobiça ou desejo de lucro,
demonstre indiferença e relaxamento de comportamento. Um homem que agir
dessa forma também é culpado de profanar o Nome e está sujeito ao
açoitamento; por conseguinte as Escrituras dizem: ``E não jurareis falso
em Meu nome, profanando o nome de vosso Deus'' (Levítico 19:12), porque
ele demonstra indiferença, embora não obtenha nenhum benefício
material.

O tipo de ação possível apenas para certos indivíduos é aquele realizado
por um homem de religiosidade e virtudes conhecidas que pareça ao povo
ser uma transgressão e ser impróprio de um homem tão devoto, embora na
realidade seja algo permitido. Tal procedimento também é uma profanação
do Nome de acordo com os Sábios, que dizem: ``O que constitui uma
profanação do Nome? Se por exemplo eu levar carne do açougue e não
pagar imediatamente por ela. Rabi Yohanan diz: No meu caso, andar
quatro cúbitos sem a Torah ou sem os `tefilin'''.

Esta proibição também se encontra em outro lugar nas palavras das
Escrituras ``E não profanarás o nome de teu Deus; Eu sou o Eterno''
(Levítico 18:21).

As normas deste preceito estão explicadas em Pessahim e no final de Yoma.

\paragraph{Não testar suas promessas e advertências}

Por esta proibição somos proibidos de testar Suas promessas e ameaças,
enaltecido seja Ele, transmitidas a nós por Seus Profetas, lançando
dúvidas sobre elas, depois de saber que aquele que as enunciou é um
verdadeiro Profeta. Esta proibição está expressa em Suas palavras ``Não
experimentareis ao Eterno, vosso Deus'' (Deuteronômio 6:16)

\paragraph{Não demolir casas de adoração ao Eterno}

Por esta proibição somos proibidos de demolir casas de adoração ao
Eterno, de destruir livros de profecia, ou de apagar os Nomes Sagrados e
coisas semelhantes. Esta proibição está expressa em Suas palavras ``Não
procedereis de modo semelhante para com o Eterno, vosso Deus''
(Deuteronômio 12:4). Ao ordenar-nos destruir os ídolos, apagar seus
nomes e demolir completamente seus templos e
altares\footnote{Ver o preceito positivo 185.}, Ele acrescenta a proibição ``Não
procedereis de modo semelhante para com o Eterno, vosso Deus''.

O castigo pela transgressão de qualquer pormenor desta proibição --- tal
como demolir uma parte qualquer do Santuário, ou o Altar, ou algo
semelhante, ou apagar qualquer um dos Nomes do Eterno --- é o
açoitamento. A Guemará, depois de explicar no final de Macot que queimar
lenha pertencente ao Santuário é punível com açoitamento, acrescenta:
``Esta proibição se encontra nas palavras: `E suas árvores idolatradas,
queimareis no fogo\ldots{} Não procedereis de modo semelhante para com o
Eterno, vosso Deus''' (Ibid., 3-4). Assim também, depois de explicar
que a penalidade por apagar o Nome Divino é o açoitamento, ela continua:
``A proibição se encontra nas palavras `Fareis perecer os seus nomes
daquele lugar. Não procedereis de modo semelhante para com o Eterno'
(Ibid.)''.

As normas deste preceito estão explicadas no quarto capítulo de Shebuot.

\paragraph{Não deixar o corpo de um criminoso pendurado durante toda a
noite após sua execuçao}

Por esta proibição somos proibidos de deixar o corpo, depois da
execução, pendurado durante toda a noite numa árvore, para que sua visão
não dê origem a pensamentos sacrílegos. O enforcamento só é praticado
entre nós nos casos do blasfemador e do idólatra, a respeito de quem
também foi dito: ``Ao Eterno ele blasfema'' (Números 15:30). Esta proibição está expressa em Suas palavras, enaltecido seja Ele, ``Não pernoitará seu cadáver no
madeiro'' (Deuteronômio 21:23), a respeito das quais o Sifrei diz:
```Não pernoitará seu cadáver no madeiro' é um preceito negativo''.

As normas deste preceito estão explicadas no sexto capítulo de Sanhedrin.

\paragraph{Não interromper a vigilância do Santuário}

Por esta proibição somos proibidos de interromper a vigilância do
Santuário, que deve ser continuamente patrulhado durante toda a noite.
Esta proibição está expressa em Suas palavras ``E fareis o serviço
(ush'martem) de guarda da santidade'' (Números 18:5). Nós já explicamos,
ao tratar do preceito positivo 22, que manter vigilância e patrulhar o
Santuário é um preceito positivo e aqui mostraremos que negligenciar
isto é infringir um preceito negativo. A Mekhiltá diz: ```E farão o
serviço da guarda da tenda da assinação' (Números 18:4) é apenas um
preceito positivo; de que modo concluímos que há um preceito negativo?
Pelas palavras das Escrituras: `E fareis o serviço da guarda da
santidade'''.

As normas deste preceito estão explicadas no início de Tamid e Midot.

\paragraph{O ``Cohen Gadol'' não deve entrar no Santuário em outras ocasiões além das estabelecidas}

Por esta proibição o ``Cohen Gadol'' fica proibido de entrar no
Santuário a todo e qualquer momento, por causa do respeito devido ao
Santuário e do temor que se deve ter da Presença Divina. Esta proibição
está expressa em Suas palavras, enaltecido seja Ele, ``Que não venha a
toda a hora a santidade'' (Levítico 16:2).

Esta proibição inclui várias restrições. O ``Cohen Gadol''! está
proibido de entrar no Santíssimo até mesmo em ``Yom Kipur'' a não ser
nos momentos determinados para o serviço\footnote{Ver o preceito positivo 49.}; e todos
os ``Cohanim'' estão proibidos de entrar no Santuário a qualquer
momento durante o ano a não ser no momento do serviço. Em resumo, fica
proibido a qualquer ``Cohen'' entrar no local que lhe é permitido entrar
--- ou seja, no Santuário Interno, no caso do ``Cohen Gadol'', e no
Santuário externo, no caso de ``Cohen'' comum --- a não ser durante o
serviço.

Aquele que violar esta proibição entrando fora da hora de serviço está
sujeito à morte se ele entrar no Santíssimo, e ao açoitamento se ele
entrar no Santuário.

A Sifrá diz: ```Que ele não venha a toda a hora' se refere a `Yom
Kipur'; `A santidade' abrange também o resto do ano; `para dentro da
cortina' (Ibid.) faz com que a proibição se aplique a todo o Santuário.
Poder-se-ia pensar que entrar em qualquer lugar do Santuário é punível
com a morte; por isso as Escrituras dizem: `Diante do propiciatório que
está sobre a arca para que não morra' (Ibid.). O que significa isto?
Entrar `Diante do propiciatório' é punível com a morte, e, no resto do
Santuário, com o açoitamento''.

A Guemará de Menahot diz explicitamente: ``Fica-se sujeito ao
açoitamento por entrar no Santuário''.

\paragraph{Um ``Cohen'' com um defeito não deve entrar em nenhuma parte do
Santuário}

Por esta proibição um ``Cohen'' com um defeito está proibido de entrar
em qualquer parte do Santuário, ou seja, no Altar, ou no espaço entre o
Pórtico e o Altar, ou no próprio Pórtico, ou no Santuário propriamente
dito. Esta proibição está expressa em Suas palavras, enaltecido seja
Ele, ``Somente até o véu não virá, e o altar não se chegará'' (Levítico
21:23).

Está explicado no início de Teharot que aquele que tiver um defeito, ou
cujo cabelo estiver solto está proibido de entrar no espaço entre o
Pórtico e o Altar ou em qualquer outra parte do Santuário. A Sifrá
também explica que essas duas proibições, ``Somente até o véu não virá''
e ``Ao altar não se chegará'' não seriam suficientes uma sem a outra, e
que ambas são necessárias para complementar a lei a esse respeito, que é
a lei que define o local onde eles estão proibidos de entrar. Aquele que
voluntariamente entrar além do Altar está sujeito ao açoitamento, mesmo
se ele não entrar com a finalidade de ministrar o serviço.

\paragraph{Um ``Cohen'' com um defeito não deve ministrar no Santuário}

Por esta proibição um ``Cohen'' que tiver um defeito físico está
proibido de ministrar. Ela está expressa em Suas palavras ``(O homem\ldots{})
em que houver algum defeito, não se chegará para oferecer'' (Levítico
21:17); quer dizer, não o deixem aproximar-se para realizar o serviço.

Um ``Cohen'' defeituoso que ministrar está sujeito ao açoitamento, como
a Sifrá diz: ``Um `Cohen' defeituoso está sujeito não à morte, mas
apenas ao açoitamento''.

\paragraph{Um ``Cohen'' com um defeito temporário não deve ministrar no Santuário}

Por esta proibição um ``Cohen'' com um defeito passageiro fica proibido
de ministrar enquanto ele estiver com o defeito. Esta proibição está
expressa em Suas palavras, enaltecido seja Ele, ``Todo homem em que
houver algum defeito, não se aproximará'' (Levítico 21:18), sobre as
quais diz a Sifrá: ```Todo homem em que houver algum defeito': isto me
fala apenas de um `Cohen' permanentemente defeituoso; como saber que o mesmo se aplica àquele que tiver um defeito passageiro? Pelas palavras das Escrituras `Todo homem
em que houver algum defeito, não se aproximará'''.

Um ``Cohen'' com um defeito passageiro que ministrar também está
sujeito ao açoitamento.

As normas deste preceito, com relação a defeitos passageiros e
temporários nos homens, estão explicadas no sétimo capítulo de Bekhorot.

\paragraph{Os Levitas e os ``Cohanim'' não devem realizar as tarefas uns dos outros}

Por esta proibição os Levitas ficam proibidos de executar qualquer uma
das tarefas designadas aos ``Cohanim'', e os ``Cohanim'' de realizar
qualquer uma das tarefas designadas aos Levitas porque cada uma dessas
duas famílias, isto é, os ``Cohanim'' e os Levitas, tem sua tarefa
específica no Santuário. Portanto, há uma advertência conjunta Dele,
enaltecido seja Ele, a ambos, para que nenhum deles execute o trabalho
do outro, e para que cada um faça aquilo de que foi incumbido, como Ele
disse, ``Cada um no seu ofício e no seu cargo'' (Números 4:19).

A proibição a esse respeito está expressa em Suas palavras, dirigidas
aos Levitas, ``Salvo aos objetos da santidade e ao altar não se chegarão
para que não morram'' (Números 18:3). Depois disso, Ele diz aos
``Cohanim'' ``Para que não morram, tanto ele como vós'' (Ibid.), sendo
que as palavras ``Como vós'' significam ``Esta proibição se aplica
também a vós, os `Cohanim'. Assim como eu os proibi de realizar vosso
trabalho também vocês estão proibidos de realizar os deles''.

O Sifrei diz: ```Aos objetos da santidade e ao altar não se chegarão' é
a proibição; `Para que não morram' enuncia o castigo. O versículo me diz
apenas que os Levitas estão sujeitos à punição e estão proibidos de
executar o trabalho dos `Cohanim'. Como saber que os `Cohanim' também
estão proibidos de executar o trabalho dos Levitas? Pelas palavras das
Escrituras `Tanto eles'. Como saber também que um grupo não deve fazer o
trabalho de outro grupo? \footnote{Todos os Levitas estavam divididos em dois grupos, aqueles que cuidavam apenas dos portões no Santuário, e aqueles que cuidavam do
canto nos cultos (ver preceito positivo 23).}. Pelas palavras `Como
vós'. Aconteceu uma vez que Rabi Yehoshuá ben Hananyá quis ajudar Rabi
Yohanan ben Gudgoda; então este lhe disse: `Volte! Você já foi privado
de sua vida, pois eu sou um dos guardiães do portão e você é um dos
cantores'''.

Assim, foi deixado claro que todo Levita que executar uma tarefa que não
lhe tiver sido designada estará sujeito à morte pela mão dos Céus. Da
mesma forma, os ``Cohanim'' não devem se ocupar das tarefas dos Levitas;
mas em caso de transgressão, eles não estão sujeitos à morte, e sim
apenas ao açoitamento.

A Mekhiltá diz: ```Salvo aos objetos da santidade e ao altar não se
chegarão': poder-se-ia pensar que eles estão sujeitos à morte meramente
por tocar, por isso as Escrituras dizem `Salvo' (ach); eles só ficam
sujeitos à pena se realizarem o trabalho. Aqui novamente, o versículo me
fala apenas sobre os Levitas que realizarem o trabalho dos `Cohanim'; de
que forma fico sabendo sobre os `Cohanim' que realizam o trabalho dos Levitas? Pelas palavras `Como
vós'''. Também está dito ali: ``Levitas que realizarem o trabalho dos
`Cohanim' estão sujeitos à morte, mas `Cohanim' que realizarem o
trabalho dos Levitas são culpados apenas de transgredir um preceito
negativo''.

\paragraph{Não entrar no Santuário nem pronunciar uma
sentença sobre uma lei da Torah estando intoxicado}

Por esta proibição somos proibidos de entrar no Santuário ou de
pronunciar uma sentença a respeito de qualquer uma das Leis da Torah, se
estivermos intoxicados. Ela está expressa em Suas palavras, enaltecido
seja Ele, ``Vinho e bebida forte não bebereis\ldots{} quando entrardes à
tenda da revelação\ldots{} e para ensinar aos filhos de Israel'' (Levítico 1
0:8-1 1). O Talmud diz: ``Aquele que tiver bebido um
quarto\footnote{Um quarto de um ``log'' de vinho.} não deve pronunciar uma sentença legal''.

Há diferentes castigos por esta proibição. Todo aquele que se embebedar
com vinho está proibido de entrar entre o Pórtico e o Altar ou em
qualquer parte do próprio Santuário e, se ele o fizer, estará sujeito à
punição por açoitamento. Se ele realizar o serviço enquanto estiver
intoxicado a penalidade é a morte pela mão dos Céus; mas se a
intoxicação tiver sido ocasionada por outro intoxicante que não o vinho
o castigo será apenas o açoitamento, não a morte. Entretanto, se alguém,
``Cohen'' ou israelita, pronunciar uma sentença estando bêbado, seja
com vinho ou outro intoxicante, ele transgredirá um preceito negativo.

A Sifrá diz: `` `Vinho\ldots{} não bebereis': isto me fala apenas do vinho;
como fico sabendo que isto se aplica da mesma forma a outras bebidas
intoxicantes? Pelas palavras das escrituras `E bebida forte'. Se é
assim, por que as Escrituras mencionam especificamente o vinho? Porque
o vinho sujeita à morte, enquanto que qualquer outra bebida intoxicante
não''.

Também está escrito no mesmo lugar: ``Como saber que não se fica sujeito
ao castigo a não ser durante o serviço? Pelas palavras das Escrituras
`Tu e teus filhos contigo, quando entrardes à tenda de assinação, e não
morrereis'''. Mais adiante lemos: ``Poder-se-ia pensar que um israelita
que pronunciar uma sentença está sujeito à morte; por isso as Escrituras
dizem: `Tu e teus filhos contigo\ldots{} e não morrereis'. Tu e teus filhos
estareis sujeitos à morte, mas um israelita não está sujeito à morte
por pronunciar uma sentença''.

As normas deste preceito estão explicadas no quarto capítulo de Queretot.

\paragraph{Um ``Zar'' não deve oficiar no Santuário}

Por esta proibição um ``zar'' fica proibido de oficiar. Por ``zar'' eu
quero dizer todo aquele que não é descendente de Aarão. Esta proibição
está expressa em Suas palavras, enaltecido seja Ele, ``E o estranho
(zar) não se aproximará de vós'' (Números 18:4). As Escrituras expõem claramente que
aquele que violar esta proibição está sujeito à morte pela mão dos Céus:
``E o estranho que se aproximar será morto'' (Ibid., 7). A esse respeito
o Sifrei diz: ```E o estranho que se aproximar será morto': ouvimos a
penalidade, mas não ouvimos a proibição. Por isso as escrituras dizem:
`E o estranho não se aproximará de vós'''.

A proibição e a penalidade relacionadas com este assunto estão repetidas
em Suas palavras ``E não se aproximarão mais os filhos de Israel\ldots{} para
que não levem sobre si, pecado, e morram'' (Ibid., 22).

A Guemará de Yoma detalha os serviços cuja realização por um ``zar'' lhe
acarreta a morte: ``Há quatro serviços por cuja execução um `zar' fica
sujeito à morte: aspergir, queimar, a libação do vinho, e a libação da
água''.

As normas deste preceito estão explicadas nesse lugar e no último
capítulo do Tratado Zabahim.


\paragraph{Um ``Cohen'' impuro não deve oficiar no Santuário}

Por esta proibição um ``Cohen'' que estiver impuro fica proibido de
realizar o serviço. Ela está expressa em Suas palavras, enaltecido seja
Ele, endereçadas aos ``Cohanim'', ``Que se separem quando estão impuros
das santidades dos filhos de Israel\ldots{} e não profanem o nome de Minha
santidade'' (Levítico 22:2).

No nono capítulo de Sanhedrim lemos: ``Como sabemos que ao realizar o
serviço estando impuro ele está sujeito à morte? Porque está escrito
`Fala a Aarão e a seus filhos, para que se separem\ldots{} e não profanem'.
Está dito em outro trecho `Para não\ldots{} pois morrerão por isto quando o
profanarem' (Ibid., 8-9); assim como no caso anterior, `profanação'
envolve a penalidade de morte pela mão dos Céus, portanto aqui também as
palavras `não profanem o nome da Minha santidade' significam que se eles
profanarem o Nome, realizando o serviço enquanto estiverem em estado de
impureza, eles estão sujeitos à morte pela mão dos Céus''.

\paragraph{Um ``Cohen'' que praticou um ``Tebul yom'' não deve oficiar no
Santuário}

Por esta proibição um ``Cohen'' que tenha praticado um ``Tebul yom''
fica proibido de oficiar até o pôr do sol, ainda que ele já tenha se
purificado\footnote{Embora o ``Cohen'' impuro já tenha procedido ao banho ritual
determinado pela Torah, depois de ter-se unido a sua esposa, ele
continua impuro até o pôr do sol.}. Ela está expressa em Suas palavras,
enaltecido seja Ele, relativas aos ``Cohanim'', ``Não profanarão o nome
de seu Deus'' (Levítico 21:6).

Todo aquele que violar esta proibição --- ou seja, que oficiar após um
``Tebul yom'' --- está sujeito à morte pela mão dos Céus.

Ela não está claramente enunciada nas Escrituras, contudo ela é
tradicionalmente interpretada assim. Está escrito no nono capítulo de
Sanhedrin, com relação à interpretação de Suas palavras, enaltecido seja
Ele, ``Santos serão para Deus, e não profanarão o nome de seu Deus'': ``Uma vez que isto não
pode se referir a um `Cohen' que oficiar impuro, o que está proibido
por um outro versículo, como já foi explicado, apliquem-no a quem
oficiar depois de um `Tebul yom'. E pode-se deduzir uma analogia do uso
da palavra `profanação' neste caso e no caso da oferta de elevação''. E
se oficiar depois do ``Tebul yom'', ele estará contado ali entre os
transgressores que estão sujeitos à morte.

\paragraph{Uma pessoa impura não pode entrar em nenhuma parte do Santuário}

Por esta proibição toda pessoa impura fica proibida de entrar em
qualquer parte do Santuário --- o que equivale, em gerações
posteriores\footnote{No Templo de Jerusalém (Beit Hamikdash).}, a todo o Campo do Santuário, que
começa com o Campo dos Israelitas, a partir do Portão de Nicanor. Esta
proibição está expressa em Suas palavras ``Para que não contaminem os
seus acampamentos'' (Números 5:3), significando o Acampamento da
Presença Divina.

A Guemará de Macot diz: ``Com relação a uma pessoa impura que entrar no
Santuário as Escrituras expressam ambos a penalidade e a proibição. A
penalidade: `Será banida aquela alma\ldots{} porque o santuário do Eterno
contaminou' (Número 19:20). A proibição: `Para que não contaminem os
seus acampamentos'''. E a Mekhiltá diz: ```Ordena aos filhos de Israel
que enviem do acampamento\ldots{}' (Ibid.,5:2) é um preceito
positivo\footnote{Ver o preceito positivo 31.}. Como sabemos que há um preceito
negativo? Pelas palavras das Escrituras `Para que não contaminem'''.

Esta proibição está repetida de outra forma em Suas palavras, referentes
a uma mulher depois do parto, ``E no santuário não entrará'' (Levítico
12:4).

A Sifrá diz: ``As palavras `E separareis os filhos de Israel de suas
impurezas, e não morrerão\ldots{} (Ibid., 15:31) poderiam ser compreendidas
como aplicáveis em caso de ambos o interior e o exterior, significando
que todo aquele que tocasse o exterior do Santuário em estado que
impureza também estaria sujeito à extinção. Por isso as Escrituras dizem
'E no Santuário não entrará'''. Ali também está explicado que a lei para
uma mulher depois do parto é a mesma que para as pessoas impuras em
geral, no que se refere a esta proibição.

A Sifrá também diz, com referência a Suas palavras, enaltecido seja Ele,
``E se não lavar e não banhar sua carne, levará sobre si a sua
iniquidade'' (Levítico 17:16): ``O que significa isto? Que se não se
lavar, ele estará sujeito à extinção; que se não lavar suas roupas, ele
estará sujeito ao açoitamento. E como sabemos que o versículo se refere
apenas à contaminação do Santuário e de suas coisas sagradas? Porque
ambas a proibição e a penalidade estão enunciadas'' etc.

Está explicado em outro lugar que aquele que deliberadamente violar
esta proibição está sujeito à extinção, e aquele que a violar
involuntariamente deve levar um Sacrifício de Maior ou Menor Valor,
como explicamos no preceito positivo 72.

As normas deste preceito estão explicadas no início de Shebuot, em
Horayot, em Queretot, e em vários trechos do Tratado Zebahim.

\paragraph{Uma pessoa impura não pode entrar no acampamento dos Levitas}

Por esta proibição toda pessoa impura fica proibida de entrar no
acampamento dos Levitas --- o que equivale, em gerações
posteriores\footnote{No templo de Jerusalém (Beit Hamikdash).}, ao Monte do Templo, como explicamos
no início do Tratado Quelim, onde se fala da exclusão de pessoas
impuras do Monte do Templo. A proibição das Escrituras a este respeito
está expressa em Suas palavras, relativas àquele que ``está impuro por
causa daquilo que aconteceu com ele à noite'': ``Não entrará em nenhum
acampamento'' (Deuteronômio 23:11).

A Guemará de Pessahim também diz: ```Sairá para fora do acampamento':
isto é, para fora do Acampamento da Presença Divina'', como explicamos
ao falar do preceito positivo 31. ``Não entrará em nenhum acampamento:
isto é, dentro do Acampamento dos Levitas. A isto Rabina objetou:
Suponha que ambos se refiram ao Acampamento da Presença Divina e que
ele esteja violando, dessa forma, um preceito positivo e um negativo. Se
fosse assim, as Escrituras diriam `Sairá para fora do acampamento' e
`Não entrará nele'; quer dizer, ele teria dito simplesmente `Não entrará
nele'. Qual é a finalidade de repetir `o acampamento'? É para
determinar outro acampamento'', que é o Acampamento dos Levitas: também
nesse ele não deve entrar.

O Sifrei diz: ``Não entrará em nenhum acampamento é um preceito
negativo''.

As normas deste preceito estão explicadas em nosso ``Comentário'', no
primeiro capítulo do Tratado Quelim.

\paragraph{Não construir um altar com pedras que tenham sido tocadas por ferro}

Por esta proibição somos proibidos de construir um Altar com pedras que
tenham sido tocadas por ferro. Ela está expressa em Suas palavras,
enaltecido seja Ele, ``Não o edificarás de pedras lavradas'' (Êxodo
20:25). Um altar construído com tal tipo de pedras é inadequado e não se
devem colocar sacrifícios sobre ele.

As normas deste preceito estão explicadas no terceiro capítulo de Midot.

\paragraph{Não subir ao altar por degraus}

Por esta proibição somos proibidos de subir ao Altar por degraus, para
não darmos passos largos para chegar até ele e para que subamos com um
pé seguindo próximo ao outro. Esta proibição está expressa em Suas
palavras, enaltecido seja Ele, ``E não subas por degraus sobre Meu
altar'' (Êxodo 20:26), a respeito das quais a Mekhiltá diz o seguinte: ``O que as Escrituras querem dizer com `Para que não seja descoberta tua nudez sobre ele'
(Ibid.)? Que ao subir ao altar não se deve dar passadas largas, mas sim
andar com um pé seguindo próximo ao outro''.

O modelo do plano inclinado\footnote{O plano do altar.} e a maneira de
construí-lo estão explicados no terceiro capítulo de Midot.

A punição por dar passadas largas e expor sua nudez no altar é o
açoitamento.

\paragraph{Não apagar o fogo do altar}

Por esta proibição somos proibidos de apagar o fogo do Altar. Ela está
expressa em Suas palavras, enaltecido seja. Ele, ``Fogo contínuo estará
aceso sobre o altar; não se apagará'' (Levítico 6:6), sobre as quais a
Sifrá diz: ```Não se apagará': isto nos ensina que aquele que o apagar
estará transgredindo um preceito negativo''. E aquele que violar esta
proibição, extinguindo nem que seja uma única brasa ardente de cima do
Altar, será punido com o açoitamento.

As normas deste preceito estão explicadas no décimo capítulo de Zebahim.

\paragraph{Não oferecer nenhum tipo de sacrifício sobre o altar de ouro}

Por esta proibição somos proibidos de oferecer todo e qualquer tipo de
sacrifício sobre o Altar de Ouro no Santuário. Ela está expressa em Suas
palavras, enaltecido seja Ele, ``Não oferecereis sobre ele incenso
estranho, nem holocausto, nem oblação; e libação não derramareis sobre
ele'' (Êxodo 30:9).

Aquele que oferecer ou espalhar sobre ele sangue de qualquer outro tipo
de sacrifício que não seja aquele que lhe corresponde está sujeito ao
açoitamento.

\paragraph{Não fazer óleo igual ao óleo de unção}

Por esta proibição somos proibidos de fazer óleo igual ao Óleo de Unção.
Ela está expressa em Suas palavras, enaltecido seja Ele, ``E da mesma
composição não fareis outro como ele'' (Exodo 30:32).

A punição pela contravenção voluntária desta proibição é a extinção,
como está nas Escrituras: ``Todo homem que fizer semelhante a ele\ldots{}''
(Ibid., 33). Se o pecado for cometido involuntariamente, o transgressor
deve levar um sacrifício Determinado de Pecado.

As normas deste Preceito estão explicadas no primeiro capítulo de Queretot.

\paragraph{Não ungir ninguém a não ser os ``Cohanim Guedolim'' e os reis com o óleo de
unção preparado por Moisés}

Por esta proibição somos proibidos de ungir qualquer outra pessoa a não
ser os ``Cohanim Guedolim'' e os Reis com o Óleo de Unção que Moisés
preparou. Ela está expressa em Suas palavras, enaltecido seja Ele,
``Sobre carne de homem não será untado'' (Êxodo 30:32). Ficou claro que
todo aquele que deliberadamente se ungir com Óleo de Unção está sujeito
à extinção, como está prescrito nas palavras: ``E que usá-lo num
estranho, será exterminado de seu povo'' (Ibid., 33); e que todo aquele
que o fizer involuntariamente deve levar um Sacrifício Determinado de
Pecado.

As normas deste preceito estão explicadas no início de Queretot.

\paragraph{Não fazer incenso igual ao usado no Santuário}

Por esta proibição somos proibidos de fazer incenso igual ao usado no
Santuário; quer dizer, fazer incenso usando os mesmos ingredientes, nas
mesmas quantidades, com a intenção de queimá-lo. Esta proibição está
expressa em Suas palavras, enaltecido seja Ele, ``Como a sua composição,
não fareis para vós'' (Êxodo 30:37).

Somos claramente avisados que todo aquele que violar propositalmente
esta proibição, fazendo incenso similar com a intenção de aspirá-lo,
está sujeito à extinção, pois Ele diz: ``O homem que fizer igual a este
para o cheirar será banido do seu povo'' (Ibid., 38); e aquele que o
fizer involuntariamente deve levar um Sacrifício Determinado de Pecado.

As normas deste preceito estão explicadas no início de Queretot.

\paragraph{Não retirar as varas das argolas da Arca}

Por esta proibição somos proibidos de remover as varas das argolas da
Arca. Ela está expressa em Suas palavras, enaltecido seja Ele, ``E nas
argolas da Arca estarão as varas; não se tirarão dela'' (Êxodo 25:15). A
punição pela contravenção é o açoitamento.

No final de Macot, ao enumerar os transgressores que estão sujeitos ao
açoitamento, os Sábios perguntam: ``Por que não incluir também aquele
que remover as varas da Arca, já que essa proibição está expressa nas
palavras `Não se tirarão dela'?''. Assim, foi deixado claro que este é
um preceito negativo, e a punição por sua transgressão é o açoitamento.

\paragraph{Não desprender o peitoral do ``efod''}

Por esta proibição somos proibidos de remover o Peitoral do
``Efod''\footnote{Roupa usada pelo ``Cohen Gadol'' quando está ministrando o serviço
no Santuário.}. Ela está expressa em Suas palavras,
enaltecido seja Ele, ``E não se
desprenderá o peitoral de cima do efod'' (Êxodo 28:28), e permanecerá
amarrado a ele.

No final de Macot, ao enumerar os transgressores que estão sujeitos ao
açoitamento, os Sábios perguntam: ``Por que não incluir também aquele
que desprende o Peitoral, já que a proibição está expressa em Suas
palavras `E não se desprenderá o peitoral de cima do efod'?''. Assim,
fica claro que desprendê-lo é punido com o açoitamento.

\paragraph{Não rasgar a orla do manto do ``Cohen Gadol''}

Por esta proibição somos proibidos de rasgar a borda do manto do ``Cohen
Gadol''; a orla deve estar inteira e intacta. Esta proibição está
expressa em Suas palavras, enaltecido seja Ele, ``Como abertura de malha
será debruada, para que não se rasgue'' (Êxodo 28:32). Cortar o manto
com tesoura ou similar será punido com o açoitamento.

\paragraph{Não oferecer nenhum sacrifício fora do campo do Santuário}

Por esta proibição somos proibidos de oferecer qualquer sacrifício fora
do Campo do Santuário\footnote{Ver o preceito negativo 77.}, o que é chamado de
``sacrificar fora''. Esta proibição está expressa em Suas palavras,
enaltecido seja Ele, ``Guarda-te de ofereceres teus holocaustos em todo
o lugar que vires'' (Deuteronômio 12:13).

O Sifrei diz: ``Isto me fala apenas dos Holocaustos; como saber quanto
aos outros sacrifícios? Pelas palavras das Escrituras `E ali farás tudo
o que eu te ordeno'. Contudo, eu poderia dizer que enquanto no caso do
Holocausto há um preceito positivo\footnote{Ver o preceito positivo 84.} e um negativo,
no caso de todas as outras ofertas há apenas um preceito positivo. Por
isso as Escrituras dizem: `Ali oferecerás os teus holocaustos'. Por que
o Holocausto está destacado, se ele está incluído no enunciado geral?
Para permitir-lhe deduzir por analogia que assim como no caso do
Holocausto --- que está especificamente mencionado --- há um preceito
positivo e um negativo, assim também em todos os outros casos, onde está
estipulado apenas um preceito positivo, há um preceito negativo
envolvido também''.

Vou apresentar agora uma explicação deste texto --- embora ele seja
simples --- para que o assunto fique claramente entendido. No caso do
Holocausto, as Escrituras proíbem expressamente oferecê-lo fora, pelas
Suas palavras, enaltecido seja Ele, ``Guarda-te de ofereceres teus
holocaustos''; a seguir há uma ordem expressa para que ele seja
oferecido no Santuário, em Suas palavras, enaltecido seja Ele, ``Ali
oferecerás os teus holocaustos'', que são um preceito positivo para
oferecer o Holocausto ``No lugar que escolher o Eterno'' (Ibid.). Mas,
com relação às outras ofertas consagradas, há apenas o preceito
positivo para que sejam oferecidas no Santuário, expresso em Suas
palavras ``Ali farás tudo o que eu te ordeno'' (Ibid.,14). Entretanto as
palavras ``Ali farás etc.'' implicam que não devemos fazê-lo fora, e é
um princípio aceito entre nós que um preceito negativo derivado de um
preceito positivo tem a força de um preceito positivo. Assim sendo, as
palavras citadas acima: ``Contudo eu poderia dizer\ldots{} que no caso de
todas as outras ofertas há apenas um preceito positivo'' devem ser
compreendidas da seguinte forma: aquele que oferecer qualquer
sacrifício fora transgride apenas um preceito negativo derivado de um
preceito positivo e por isso as Escrituras dizem: ``Ali oferecerás os
teus holocaustos'', para permitir-nos argumentar, por analogia, que
assim como o oferecimento do Holocausto fora infringe um preceito
negativo, o mesmo acontece quando se leva qualquer uma das outras
ofertas.

A violação voluntária desta proibição é punida com a extinção; aquele
que a violar involuntariamente deve levar um Sacrifício Determinado de
Pecado. O castigo de extinção por ``sacrificar fora'' está estabelecido
na parte chamada ``Aharé Mot'' \footnote{Levítico 16:1 a 18:30.} das Escrituras:
``Que oferecer holocausto ou sacrifício, e à porta da tenda da assinação
trouxer, para oferecê-los ao Eterno, será banido aquele homem de seu
povo'' (Levítico 17:8-9), a respeito das quais diz a Sifrá: ``Será
banido aquele homem de seu povo: já ouvimos a penalidade; de onde vem a
proibição? Das palavras das Escrituras `Guarda-te de ofereceres teus
holocaustos'''. Ou nas palavras da Guemará de Zebahim: ``A penalidade
foi enunciada, assim como a advertência. A penalidade: `À porta da tenda
da assinação trouxer\ldots{} será banido aquele homem de seu povo'; a
advertência: `Guarda-te de ofereceres etc.'''

As normas deste preceito estão explicadas no décimo terceiro capítulo
de Zebahim.

\paragraph{Não degolar nenhum dos sacrifícios sagrados fora do campo do Santuário}

Por esta proibição somos proibidos de degolar qualquer um dos
sacrifícios Sagrados fora. Isso é chamado de ``degolar fora'' e na
enumeração de todas as transgressões que acarretam a extinção, feita no
início de Queretot, ``degolar fora'' e ``sacrificar fora'' são contadas
separadamente.

O princípio de que aquele que degola do lado de fora fica sujeito à
extinção a partir do momento do degolamento, mesmo se ele não fizer a
oferenda depois, está expressa na Torah em Suas palavras, enaltecido
seja Ele, ``Que degolar boi ou cordeiro ou cabra, no acampamento, ou
degolar fora do acampamento, e não os trouxer à porta da tenda da
assinação para o oferecer como sacrifício ao Eterno, derramador de
sangue será considerado aquele homem; sangue derramou, e será banido
aquele homem dentre seu povo'' (Levítico 17:3-4). Contudo, a proibição
de degolar os sacrifícios sagrados fora não está explicitamente
enunciada, mas se deduz do princípio de que não se prescreve uma
punição a menos que uma proibição a preceda, de acordo com os
Fundamentos que apresentamos na Introdução a estes preceitos. A Guemará
de Zebahim diz: ``Aquele que degolar e oferecer fora é culpado por
degolar e por oferecer. Isso está correto no que se refere ao
sacrifício, a respeito do qual as Escrituras prescrevem a punição e enunciam a proibição, sendo a punição: `Será banido' (Ibid., 9),
e a proibição: `Guarda-te (hishamer) de ofereceres' (Deuteronômio
12:13), a qual deve ser compreendida à luz das palavras de Rabi Abin, em
nome de Rabi Ilai, de que toda vez que as Escrituras disserem
`guarda-te' (hishamer), ou `para que não' (pen), ou `não' (al), há um
preceito negativo. Mas no caso do degolamento, embora as Escrituras
prescrevam reconhecidamente o castigo nas palavras `E não os trouxer à
porta da tenda da assinação\ldots{} será banido aquele homem', onde é que
encontramos a proibição?'' Depois de uma longa discussão a questão foi
assim resolvida: ``Ao dizer `Ali oferecerás (taaleh)' \ldots{} e `ali farás
(taaseh)' (Deuteronômio 12:14), as Escrituras nos permitem argumentar o
seguinte, por analogia de `oferecer' a `fazer': assim como no caso de
`oferecer', também no caso de `fazer' estão implicados tanto o castigo
quanto a proibição''.

``Ali oferecerás'' e ``Ali farás'' se referem às Suas palavras,
enaltecido seja Ele, ``Ali oferecerás os teus holocaustos'', que
significam sacrificar, i.e., queimar no fogo, e as palavras ``Ali farás
tudo o que eu te ordeno'', que incluem ambos queimar e degolar, uma vez
que Ele nos deu ordens quanto ao degola-mento também.

Você deve saber que aquele que degolar fora involuntariamente também
tem a obrigação de levar um Sacrifício Determinado de Pecado.

Também é importante para você saber que todo aquele que oferecer
atualmente um sacrifício sagrado fora está sujeito à extinção. Os sábios
dizem claramente: ``Se alguém sacrificar atualmente, Rabi Yohanan diz
que ele é culpado'', e essa é a lei, pois os sacrifícios ainda são
válidos, de acordo com o princípio bem estabelecido de que ``Podemos
oferecer sacrifícios embora não haja Santuário''.

As normas deste preceito também estão explicadas no décimo terceiro
capítulo de Zebahim.

\paragraph{Não destinar animais defeituosos para serem oferecidos sobre o altar}

Por esta proibição somos proibidos de dedicar animais com defeito sobre
o Altar. Ela está expressa em Suas palavras, enaltecido seja Ele, ``Todo
aquele que tiver defeito não oferecereis'' (Levítico 22:20), a respeito
das quais a Sifrá diz: ```Todo aquele que tiver defeito não
oferecereis': isto se refere à consagração''.

\paragraph{Não degolar animais defeituosos para oferecê-los como sacrifício}

Por esta proibição somos proibidos de degolar: animais com defeito como
Sacrifício. Ela está expressa em Suas palavras, enaltecido seja Ele,
relativas a animais com defeito ``Não oferecereis ao Eterno'' (Levítico
22:22), a respeito das quais diz a Sifrá: ```Não oferecereis ao Eterno':
isto se refere ao degolamento''.


\paragraph{Não aspergir o sangue de animais defeituosos sobre o altar}

Por esta proibição somos proibidos de aspergir sobre o Altar o sangue
de animais com defeito. Ela está expressa em Suas palavras, enaltecido
seja Ele, também relativas aos animais defeituosos, ``Não oferecereis ao
Eterno'' (Levítico 22:24), as quais a Tradição interpreta como uma
proibição quanto a espalhar o sangue de um animal com defeito. Essa é a
opinião do ``Taná kamá'', e essa é a lei. Contudo, Rabi Yossi ben Yehudá
diz que ela proíbe apenas o recebimento do sangue. Essa opinião aparece
na Sifrá: ```Não oferecereis ao Eterno': isto se refere a receber o
sangue''.

A Guemará de Temurá diz: ``Do ponto de vista do `Fana
kamá', por que as Escrituras dizem `Não oferecereis'? Isso é necessário
para o caso de se aspergir o sangue de um animal defeituoso. Mas é das
palavras `sobre o altar' que deduzimos essa proibição?'', referindo-se
às palavras das Escrituras ``E ofertas queimadas, não dareis destas
coisas sobre o altar, ao Eterno'' (Levítico 22:22), que nos ensinam que
tudo aquilo que for oferecido sobre o altar não deve provir de animais
defeituosos. A resposta é: ``É assim que foi escrita a
Torah''\footnote{Sendo que, de acordo com a gramática ou com o assunto, é necessário
  escrever dessa forma.}. Quer dizer, a proibição ``E ofertas
queimadas, não dareis destas coisas'' se refere apenas à queima das
partes de sacrifícios e nada pode ser deduzido do uso da expressão
``sobre o altar'' porque o versículo não poderia ter sido escrito de
outra forma, pois como poderia ele dizer simplesmente ``E ofertas
queimadas, não dareis destas coisas''? Ele ficaria incompleto!

Por tudo o que precede, aparece claramente que Suas palavras ``Não
oferecereis ao Eterno'' são a proibição de espalhar o sangue.

\paragraph{Não queimar as partes de sacrifício de um animal defeituoso sobre o altar}

Por esta proibição somos proibidos de queimar as partes de sacrifício
de um animal defeituoso. Éla está expressa em Suas palavras, enaltecido
seja Ele, ``E ofertas queimadas, não dareis destas coisas sobre o
altar'' (Levítico 22:22), sobre as quais diz a Sifrá: ```E ofertas
queimadas, não dareis destas coisas' se refere à gordura. `Não dareis
destas coisas' significa apenas a queima de todas; como fico sabendo que
se trata de uma parte qualquer delas? Pelas palavras `destas coisas',
que significam nem mesmo qualquer parte delas''.

Assim, foi deixado claro que aquele que sacrificar um animal defeituoso
transgride quatro proibições, se contarmos a queima das partes de
sacrifício como apenas um preceito; mas se o contarmos como dois, como
fez o ``Taná''\footnote{O sábio da Sifrá.}, haverá cinco preceitos
transgredidos, pois o ``Taná'' conta ``qualquer uma'' das partes de
sacrifício como um preceito, e ``todas'' elas como outro, sustentando
que ``destas coisas'' significa qualquer parte delas, embora haja
apenas uma proibição. Parece que o ``Taná'' é de opinião que a violação
de um ``lav shebikhlalut'' é punível com o açoitamento! Por isso ele diz na
Sifrá: ``Aquele que destinar para o Altar um animal defeituoso
transgride cinco proibições: a de destinar, degolar, aspergir o sangue,
queimar as partes de sacrifício, e queimar qualquer uma dessas
partes''.

A Guemará de Temurá diz: ``De acordo com Abayé, se alguém oferecer
sobre o altar os membros de animais defeituosos será punido tanto pela
violação da proibição de queimar o animal inteiro como pela violação da
proibição de queimar uma parte qualquer dele. Rabá diz que não há pena
de açoitamento pela violação de um `lav
shebikhlalut'\footnote{Ver o nono fundamento.}, mas contra ele foi citada a
afirmação `Aquele que destinar para o Altar um animal defeituoso
transgride cinco proibições, etc.', que demonstra que há uma pena de
açoitamento pela violação de um `lav shebikhlalut'. Rabá foi, por
conseguinte, contestado''.

Foi assim deixado claro que quem diz que se transgridem cinco proibições
tem essa opinião apenas porque ele sustenta que a violação de um ``lav
shebikhlalut'' é punível com o açoitamento, e consequentemente conta
como duas a proibição referente ao todo e a qualquer parte. É sabido que
Abayé adota esse ponto de vista em todos os lugares, como explicamos no
nono dos Fundamentos expostos no começo deste trabalho. Mas de acordo
com Rabá, que diz que não há pena de açoitamento pela violação de um
``lav shebikhlalut'', há apenas um açoitamento pela queima, como
dissemos. Nós deixamos claro que este é o ponto de vista correto, como
explicado na Guemará de Sanhedrin, e como enfatizado por nós no nono
Fundamento; consequentemente, há apenas quatro proibições, como foi
enunciado nas Escrituras, e aquele que dedicar e sacrificar um animal
defeituoso está sujeito a quatro açoitamentos pela violação dessas
quatro proibições, como explicamos.

Todas essas proibições se aplicam a animais permanentemente defeituosos,
tais como os mencionados nas Escrituras nas seguintes palavras: ``Que
tenham membros ou maior que o outro,\ldots{} e de testículos machucados, ou
moídos ou despendidos, ou cortados'' (Levítico 22:23-24), pois todos
esses são defeitos permanentes.

Todos os defeitos de animais, tanto permanentes como temporários, estão
detalhados no sexto capítulo de Bekhorot; e as normas das quatro
proibições que tratam especificamente da oferta de animais defeituosos
estão explicadas em diversos trechos dos Tratados Zebahim e Temurá.

\paragraph{Não sacrificar um animal com um defeito temporário}

Por esta proibição somos proibidos de sacrificar um animal com um
defeito passageiro. Ela está expressa em Suas palavras no Deuteronômio
``Não sacrificarás ao Eterno, teu Deus, boi, ou cordeiro que tenha
defeito etc.''. (Deuteronômio 17:1), que são explicadas pelo Sifrei
como se referindo a um animal com um defeito passageiro.

Também neste caso o açoitamento é o castigo por desobedecer à proibição
de sacrificar.


\paragraph{Não oferecer sacrifícios defeituosos de um gentio}

Por esta proibição somos proibidos de oferecer sacrifícios defeituosos
de um gentio. Não devemos dizer: ``Como ele é um gentio, um sacrifício
imperfeito pode ser oferecido em seu favor''. Esta proibição está
expressa em Suas palavras, enaltecido seja Ele, ``E da mão do
estrangeiro não oferecereis nenhuma dessas coisas'' (Levítico 22:25).

A punição por sacrificar transgredindo esta proibição também é o
açoitamento.

\paragraph{Não fazer com que uma oferta se torne defeituosa}

Por esta proibição somos proibidos de fazer com que uma oferta se torne
defeituosa. Isso é chamado de ``deformar ofertas consagradas'' e é
punível pelo açoitamento, desde que seja feito quando o Santuário
estiver de pé e caso a oferta seja aceitável, como explicado na Guemará
de Abodá Zará. Essa proibição está expressa em Suas palavras relativas
aos sacrifícios ``Estes deverão ser sem defeito'' (Levítico 22:21),
sobre as quais a Sifrá diz: ```Estes deverão ser sem defeito', ou seja,
não os tornem defeituosos''.

\paragraph{Não oferecer fermento ou mel sobre o altar}

Por esta proibição somos proibidos de oferecer fermento ou mel sobre o
Altar. Ela está expressa em Suas palavras, enaltecido seja Ele, ``Porque
não fareis queimar fermento algum ou mel algum como oferta queimada ao
Eterno'' (Levítico 2:11), e exposta sob outra forma em Suas palavras
``Nenhuma oblação que oferecerdes ao Eterno será preparada com
fermento'' (Ibid.).

Já explicamos no Nono Fundamento que oferecer ambos fermento e mel é
punível com um açoitamento apenas, e não com dois, porque essa
proibição é um ``lav shebikhlalut'', como explicamos ali, e ficou
claramente estabelecido que a violação de um ``lav shebikhlalut''
acarreta açoitamento apenas uma vez. Assim, por exemplo, aquele que
oferecer mel é punido uma vez com açoitamento, da mesma forma que aquele
que oferecer fermento, ou fermento e mel juntos.

\paragraph{Não oferecer um sacrifício sem sal}

Por esta proibição somos proibidos de oferecer um sacrifício sem sal.
Ela está expressa em Suas palavras, enaltecido seja Ele, ``Não deixarás
faltar o sal do pacto de teu Deus, em tua oblação'' (Levítico 2:13).
Como somos proibidos de deixar faltar o sal, segue-se que é proibido
oferecer qualquer sacrifício sem sal, e, que todo aquele que oferecer um
sacrifício ou oblação sem sal está sujeito ao açoitamento.

As normas deste preceito estão explicadas no sétimo capítulo de Zebahim.

\paragraph{Não oferecer no altar o salário de uma rameira ou o preço de um cão}

Por esta proibição somos proibidos de oferecer no Altar o salário de uma
rameira ou o preço de um cão. Ela está expressa em Suas palavras,
enaltecido seja Ele, ``Não trarás salário de rameira, nem preço de cão
à casa do Eterno, teu Deus'' (Deuteronômio 23:19).

As normas deste preceito estão explicadas no sexto capítulo do Tratado
Temurá. Todo aquele que oferecer uma destas coisas, embora seu
sacrifício seja desqualificado, está sujeito ao açoitamento, de acordo
com a lei relativa ao sacrifício de um animal com defeito.

\paragraph{Não degolar a mãe e seu filhote no mesmo dia}

Por esta proibição somos proibidos de degolar a mãe e seu filhote no
mesmo dia, seja para serem usados como sacrifício ou como simples
alimento. Ela está expressa em Suas palavras, enaltecido seja Ele, ``A
ela e a sua cria não degolareis no mesmo dia'' (Levítico 22:28).

A punição por degolar transgredindo esta proibição é o açoitamento. As
normas deste preceito estão detalhadamente explicadas no quinto
capítulo de Hulin.

\paragraph{Não colocar azeite de oliva sobre a oblação de um pecador}

Por esta proibição somos proibidos de colocar azeite de oliva sobre a
oblação de um pecador\footnote{Ver o preceito positivo 72.}. Ela está expressa em Suas
palavras, enaltecido seja Ele, ``Não porá sobre ela azeite'' (Levítico
5:11). Colocar azeite sobre ela se pune com o açoitamento.

\paragraph{Não levar incenso junto com a oblação de um pecador}

Por esta proibição somos proibidos de levar incenso junto com a oblação
de um pecador. Ela está expressa em Suas palavras, enaltecido seja Ele,
``Nem porá sobre ela incenso'' (Levítico 5:11). Colocar incenso sobre
ela se pune com o açoitamento.

A Mishná diz: ``Um homem torna-se culpado por causa do óleo por si só e
por causa do incenso por si só'', porque estes são sem dúvida alguma
dois preceitos negativos independentes.

As normas deste preceito, relativo à oblação de um pecador, estão
explicadas no quinto capítulo de Menahot.

\paragraph{Não misturar azeite de oliva com a oblação de uma mulher suspeita de
adultério}

Por esta proibição somos proibidos de misturar azeite de oliva com a
oblação de uma mulher suspeita de adultério. Ela está expressa em Suas
palavras, enaltecido seja Ele, ``Não derramará sobre ela azeite''
(Números 5:15).

A punição por oferecer a oblação com óleo é o açoitamento.

\paragraph{Não colocar incenso sobre a oblação de uma mulher suspeita de adultério}

Por esta proibição somos proibidos de colocar incenso sobre a oblação
de uma mulher suspeita de adultério. Ela está expressa em Suas palavras,
enaltecido seja Ele, ``Nem porá sobre ela incenso'' (Números 5:15), a
respeito das quais o Sifrei diz: ``Isto significa que aquele que puser
incenso em tal tipo de oferenda transgredirá um preceito negativo, pois
o que se aplica ao azeite se aplica também ao incenso''. Portanto a
transgressão da proibição se pune também pelo açoitamento.

A Mekhiltá diz: ```Não derramará sobre ela azeite, nem porá sobre ela
incenso': isto significa que há duas proibições diferentes''.

\paragraph{Não trocar um animal que tenha sido consagrado como oferenda}

Por esta proibição somos proibidos de trocar um animal que tenha sido
consagrado como oferenda. Isso é chamado de ``substituição''. A
proibição está expressa em Suas palavras, enaltecido seja Ele, ``Não o
mudará e não o trocará'' (Levítico 27:10)\footnote{Ver o preceito positivo 87.}.

Há uma proibição específica no caso do dízimo\footnote{Ver o preceito positivo 78.}. A
razão para isso está exposta na Sifrá: ``O dízimo já está incluído no
geral\footnote{I.e., na proibição geral ``Não o mudará e não o trocará''.}: então por que ele foi especificamente
destacado? Para permitir-nos argumentar por analogia que, da mesma forma
que o dízimo --- cuja substituição está proibida --- é uma das coisas
sagradas do altar, assim também todas as coisas sagradas --- cuja
substituição está proibida por Suas palavras `Não o mudará' --- são
apenas as coisas sagradas do altar'', e sua substituição se pune com o
açoitamento.

As normas deste preceito estão explicadas no Tratado Temurá.

\paragraph{Não trocar uma oferenda sagrada por outra}

Por esta proibição somos proibidos de trocar um tipo de oferenda sagrada
por outro, como por exemplo trocar um Sacrifício de Paz por um
Sacrifício de Delito, ou um Sacrifício de Delito por um Sacrifício de
Pecado. Esses e outros atos similares são proibidos por um preceito
negativo, está expresso em Suas palavras, enaltecido seja Ele,
referentes ao primogênito de um animal, ``Ninguém o consagrará''
(Levítico 27:26). De acordo com a Tradição, estas palavras se aplicam a
consagrar para o altar.

A Sifrá diz: ``Estou informado apenas com relação ao primogênito; de que
forma posso concluir que não se deve trocar nenhum sacrifício sagrado
por um sacrifício de tipo diferente? Pelas palavras das Escrituras `Do
animal\ldots{} ninguém o consagrará'''. A referência aqui é feita às
palavras ``Do animal, que já pertence ao Eterno, ninguém o consagrará'',
que é como se o Eterno tivesse dito: ``Qualquer que seja o animal que
tenha sido dedicado ao Eterno, homem algum deve dedicá-lo a outra
finalidade sagrada; ele deve continuar a ser o que era''.

As normas deste preceito estão explicadas no quinto capitulo de Temurá.

\paragraph{Não resgatar o primogênito de um animal puro}

Por esta proibição somos proibidos de resgatar o primogênito de um
animal puro. Ela está expressa em Suas palavras, enaltecido seja Ele,
``O primogênito do boi ou o primogênito do carneiro ou o primogênito do
bode, não remirás; santidades são eles'' (Números 18:17). Entretanto, o
primogênito pode ser vendido, como está explicado no Tratado Bekhorot. A
Sifrá diz: ``Com relação ao primogênito, as Escrituras dizem `Não
remirás', mas ele pode ser vendido''.

\paragraph{Não vender o dízimo do gado}

Por esta proibição somos proibidos de vender o dízimo do gado seja de
que forma for. Ela está expressa em Suas palavras, enaltecido seja Ele,
relativas ao dízimo do gado, ``Não se poderão remir'' (Levítico 27:33),
sobre as quais diz a Sifrá: ``Com relação ao dízimo do gado as
Escrituras dizem ``Não se poderão remir''; e ele não deve ser vendido,
vivo ou depois de ter sido degolado, seja ele perfeito ou defeituoso''.

As normas deste preceito e do precedente estão explicadas no Tratado
Bekhorot, e no início do Tratado Maasser Sheni.

\paragraph{Não vender uma propriedade consagrada}

Por esta proibição somos proibidos de vender uma propriedade que foi
declarada consagrada por seu proprietário,\footnote{Ver o preceito positivo 145.} nem
mesmo ao tesoureiro do Templo. Esta proibição está expressa em Suas
palavras, enaltecido seja Ele,
``Toda consagração\ldots{} não poderá ser vendido'' (Levítico 27:28), a
respeito das quais diz o Sifrei: ```Toda\ldots{} não poderá ser vendido':
(nem mesmo) ao tesoureiro''.

Aqui o termo ``consagrado'' significa consagrado sem declaração
específica de finalidade.\footnote{E nesse caso a propriedade deve ser entregue a um ``Cohen'' (preceito positivo 145).}

\paragraph{Não resgatar terra que tenha sido consagrada sem nenhuma declaração
específica de finalidade}

Por esta proibição também somos proibidos de resgatar a terra que tenha
sido declarada consagrada, sem nenhuma declaração específica de
finalidade.\footnote{Ver o preceito positivo 145.} Esta proibição está expressa em Suas
palavras, ``Toda consagração que uma pessoa fizer ao Eterno\ldots{} não
poderá ser vendido, nem remido'' (Levítico 27:28), sobre as quais a
Sifrá diz: ``Não deverá ser resgatada pelo seu proprietário. Então o que
deverá ser feito com ela? `Como campo consagrado para o ``Cohen''; a
possessão dele pertencerá aos ``Cohanim''' (Ibid.,21). Eu poderia pensar
que é assim, até mesmo no caso do proprietário tê-la expressamente
declarado consagrada ao Eterno; por isso as Escrituras dizem:
`Ele'''.\footnote{Os sábios concluíram que esta palavra vem limitar a aplicação da
lei (de que a consagração pertence ao ``Cohen'') apenas ao caso da
consagração sem declaração específica de finalidade.}

As normas deste preceito, referente às terras que foram declaradas
consagradas, estão explicadas no Tratado Arakhin, onde está explicado
que algo que tenha sido declarado consagrado sem nenhuma declaração de
finalidade deve ser dado aos ``Cohanim''. Também está dito ali: ``O que
foi consagrado para uso dos `Cohanim' não pode ser resgatado e deve ser
dado aos `Cohanim', como a Oferta de Elevação''.

\paragraph{Não cortar a cabeça do pássaro de um Sacrifício de Pecado durante a ``Meliká''}

Por esta proibição somos proibidos de cortar a cabeça do pássaro de um
Sacrifício de Pecado, durante a ``Meliká''.\footnote{Uma forma de sacrifício segundo a qual o ``Cohen'' se utiliza da
unha para cortar a nuca, e a laringe e o esôfago do pássaro.} Ela
está expressa em Suas Palavras, enaltecido seja Ele, ``E destroncará sua
cabeça pela nuca, porém não o separará'' (Levítico 5:8). Se ele a cortar
ou a dividir, ele invalidará o Sacrifício.

As normas deste preceito estão no sexto capítulo de Zebahim.

\paragraph{Não fazer qualquer trabalho com um animal consagrado}

Por esta proibição somos proibidos de fazer qualquer trabalho com um
sacrifício. Ela está expressa em Suas palavras, enaltecido seja Ele,
``Não farás nenhum serviço com o primogênito de teu boi'' (Deuteronômio
15:19) e nós deduzimos a proibição de trabalhar com qualquer outro
animal consagrado ao sacrifício a partir desta, relativa ao
primogênito.

Está explicado no final do Tratado Makoth que aquele que trabalhar com
um animal dedicado ao sacrifício é punido com o açoitamento.

\paragraph{Não tosquiar um animal consagrado}

Por esta proibição somos proibidos de tosquiar um sacrifício. Ela está
expressa em Suas palavras, enaltecido seja Ele, ``Nem tosquiarás o
primogênito do teu rebanho'' (Deuteronômio 15:19) e nós deduzimos a
proibição de tosquiar todos os outros animais consagrados ao sacrifício
a partir desta, relativa ao primogênito.

As normas destes dois preceitos, relativos a tosquiar e a trabalhar com
sacrifícios, estão explicadas no Tratado Bekhorot. Todo aquele que
tosquiar qualquer sacrifício também será punido com o açoitamento.

\paragraph{Não degolar o sacrifício de ``Pessah'' enquanto tivermos pão
levedado em nosso poder}

Por esta proibição somos proibidos de degolar o cordeiro de "Pessah"
enquanto tivermos pão levedado. Ela está expressa em Suas palavras,
enaltecido seja Ele, "Não sacrifiques, tendo pão levedado, sangue de
Meu sacrifício" (Êxodo 23:18) e aparece também em outro
trecho\footnote{Êxodo 34:25.}. Significa que a partir do momento do
degolamento da oferenda de "Pessah", que é a
tarde\footnote{No 14º dia de Nissan.}, não deverá haver pão levedado em possessão
daquele que degolar a oferenda, daquele que aspergir seu sangue, daquele
que queimar suas partes de sacrifício, nem daqueles que fazem parte da
companhia\footnote{O grupo que se reuniu para juntos comerem a carne do sacrifício de
  ``Pessah''.}; e todo aquele que estiver de posse de
pão levedado naquela ocasião será punido com o açoitamento.

A Mekhiltá diz: ```Não sacrifiques, tendo pão levedado, sangue de Meu
sacrifício': isto é, não deverás degolar o cordeiro de ``Pessah''
enquanto ainda houver pão levedado''.

As normas deste preceito estão explicadas no quinto capítulo de
Pessahim.

\paragraph{Não deixar as partes do sacrifício da oferenda de ``Pessah'' de um dia
para o outro}

Por esta proibição somos proibidos de deixar de oferecer as partes do
sacrifício da oferenda de ``Pessah'' até que elas deixem de ser
adequadas ao sacrifício e se transformem em ``notar''. Esta proibição
está expressa em Suas palavras, "E não ficará sebo do Meu sacrifício,
até a manhã" (Exodo 23:18), sobre as quais a Mekhiltá diz: "O objetivo
das Escrituras é explicar-nos que os pedaços de gordura se tornam
impróprios ao sacrifício se ficarem de um dia para o outro no chão".

Esta proibição está repetida de uma outra forma em Suas palavras: ``E
não ficará para a manhã o sacrifício do cordeiro de `Pessah''' (Ibid.
34:25).

\paragraph{Não deixar ficar nenhuma parte da carne da oferenda
de ``Pessah'' até a manhã seguinte}

Por esta proibição somos proibidos de deixar ficar alguma parte da carne
da oferenda de "Pessah" até a manhã do décimo quinto dia. Ela está
expressa em Suas palavras "E não fareis sobrar nada dele até a manhã"
(Êxodo 12:10).

Nós já explicamos\footnote{Ver preceito positivo 91.} que este preceito negativo é um
dos que estão justapostos a um preceito positivo, o qual está expresso
em Suas palavras "E a sobra dele, pela manhã a queimareis no fogo"
(Ibid.). A Mekhiltá diz: " 'E a sobra dele': as Escrituras tencionam
acrescentar um preceito positivo ao preceito negativo, para assim
indicar-nos que ele não está sujeito ao açoitamento.

\paragraph{Não deixar sobrar carne do sacrifício do Festival do décimo quarto
do ``Nissan'' até o terceiro dia}

Por esta proibição somos proibidos de deixar ficar até o terceiro dia
qualquer parte da carne do sacrifício de Festival que for levado no
décimo quarto dia (como está explicado no sexto capítulo de Pessahim).
Ela deve ser comida em dois dias. Esta proibição está expressa em Suas
palavras ``E não ficará da carne do cordeiro de `Pessah' que
sacrificares no primeiro dia à tarde, até pela manhã'' (Deuteronômio
16:4), cuja interpretação tradicional é a seguinte: ```E não ficará da
carne\ldots{} até pela manhã': as Escrituras falam aqui do sacrifício de
Festival que é levado em suplemento ao sacrifício de `Pessah', e
determinam que ele pode ser comido durante dois dias. Poder-se-ia pensar
que ele pode ser
comido apenas num só dia, mas quando as Escrituras dizem `até pela
manhã' isso significa até a manhã do segundo dia do Festival''. E a
respeito deste sacrifício de Festival que Ele diz, enaltecido seja Ele,
``E sacrificarás `Pessah' ao Eterno, teu Deus, do rebanho (tson) e do
gado (bakar)'' (Ibid., 2).

Qualquer porção de carne deste sacrifício do Festival do décimo quarto
que tiver ficado até o terceiro dia deverá ser queimada porque ela se
transforma em ``notar'', e consequentemente não há penalidade de
açoitamento por isso.

As normas deste preceito, relativo ao sacrifício do Festival do décimo
quarto dia, estão explicadas em vários trechos dos Tratados Pessahim e
Haguigá.

\paragraph{Não deixar sobrar carne do segundo sacrifício de
``Pessah'' até a manhã seguinte}

Por esta proibição somos proibidos de deixar sobrar qualquer parte da
carne do segundo sacrifício de ``Pessah''\footnote{Ver o preceito positivo 58.} até pela
manhã. Ela está expressa em Suas palavras ``Não deixarás nada dele até
pela manhã'' (Números 9:12).

Assim como a anterior, esta também é uma proibição justaposta a um
preceito positivo\footnote{Ver o preceito positivo 91.}.

\paragraph{Não deixar sobrar carne do Sacrifício de Graças até a manhã seguinte}

Por esta proibição somos proibidos de deixar sobrar qualquer parte de um
Sacrifício de Graças até pela manhã. Ela está expressa em Suas palavras,
relativas ao Sacrifício de Graças: ``Não deixareis ficar dele até pela
manhã'' (Levítico 22:30).

Por esta proibição chegamos à conclusão de que tudo o que ficar, de
qualquer outro sacrifício, além do tempo determinado para o seu consumo
se transformará em ``notar'' e deverá ser queimado, uma vez que a
proibição está justaposta a um preceito positivo. Queimá-lo é um
preceito positivo, como explicamos ao tratar do preceito positivo 91.

\paragraph{Não quebrar nenhum osso do sacrifício de ``Pessah''}

Por esta proibição somos proibidos de quebrar qualquer um dos ossos do
sacrifício de ``Pessah''. Ela está expressa em Suas palavras ``Nem o
osso quebrarão'' (Êxodo 12:46), e sua transgressão é punida com o
açoitamento, como está explicitamente exposto no Talmud: ``Aquele que
quebrar um osso de um sacrifício de `Pessah' puro será punido com
açoitamento''.


\paragraph{Não quebrar nenhum osso do segundo sacrifício de ``Pessah''}

Por esta proibição também somos proibidos de quebrar qualquer um dos
ossos do segundo sacrifício de ``Pessah''\footnote{Ver o preceito positivo 57.}. Ela
está expressa em Suas palavras, enaltecido seja Ele, ``E osso algum não
quebrará dele'' (Números 9:12), e sua contravenção também é punida com o
açoitamento.

A Guemará de Pessahim diz: ``Uma vez que está dito, em relação ao
segundo sacrifício de `Pessah', `E osso algum não quebrará dele', o que
é desnecessário, já que também está dito `Segundo todo o estatuto de
``Pessah'', o fará' (Ibid.), devo concluir que isso se refere a qualquer
tipo de osso, quer ele contenha tutano ou não.''

As normas relativas à quebra de um osso estão explicadas no sétimo
capítulo de Pessahim.

\paragraph{Não retirar o sacrifício de ``Pessah'' do lugar onde ele é comido}

Por esta proibição somos proibidos de retirar qualquer porção da carne
do sacrifício de ``Pessah'' do lugar onde a
companhia\footnote{O grupo que se reuniu para juntos comerem a carne do sacrifício de
  ``Pessah''.} se reúne para comê-lo. Esta proibição
está expressa em Suas palavras, enaltecido seja Ele, ``Não levarão para
fora da casa a carne'' (Êxodo 12:46), sobre as quais a Mekhiltá diz:
```Para fora': isto é, para fora do lugar onde ela deve ser comida''.
Qualquer porção da carne que for retirada é classificada como ``terefá''
e não pode ser comida.

A Guemará de Pessahim diz: ``Se carne do sacrifício de ``Pessah'' for
levada de uma companhia para outra, embora isso
infrinja um preceito negativo, a carne permanecerá pura, mas todo aquele
que a comer estará transgredindo um preceito negativo''. Também está
dito, no mesmo trecho, que ``Aquele que levou carne de sacrifício de
``Pessah'' de uma companhia para outra não é
culpado a menos que ele a deixe lá, pois a expressão `levarão' tem o
mesmo significado aqui que no caso do Shabat''. Mas se a deixar lá,
então ele estará sujeito ao açoitamento.

As normas deste preceito estão explicadas no sétimo capítulo de Pessahim.

\paragraph{Não cozer as sobras de uma oblação de cereal com levedo}

Por esta proibição somos proibidos de cozer as sobras de uma oblação de
cereal com levedo. Ela está expressa em Suas palavras, enaltecido seja
Ele, ``Não será cozido levedado, isto é igual à porção das minhas
ofertas queimadas, que lhe tenho dado'' (Levítico 6:10), o que equivale
a dizer que a porção deles, que é a sobra da oblação, não deve ser
cozida com levedo. Todo aquele que a cozer com levedo estará sujeito ao
açoitamento, como a Mishná enuncia claramente: ``Fica-se sujeito ao
açoitamento''.

As normas deste preceito estão explicadas no quinto capítulo de Menahot.

\paragraph{Não comer o sacrifício de ``Pessah'' cozido nem cru}

Por esta proibição somos proibidos de comer o sacrifício de ``Pessah''
cozido ou cru: ele deve ser assado. Esta proibição está expressa em Suas
palavras, enaltecido seja Ele, ``Não comais dela mal passada no fogo nem
cozida na água'' (Êxodo 12:9).

Já expliquei no Nono Fundamento deste trabalho que a contravenção a
esta proibição é punida com o açoitamento.

\paragraph{Não permitir que um ``guer toshab'' coma do sacrifício de ``Pessah''}

Por esta proibição somos proibidos de permitir que um ``guer
toshab''\footnote{Um ``prosélito residente'', ou seja, um gentio que não adora
ídolos e cumpre as sete leis de Noah, mas que não se converteu
oficialmente ao judaísmo.} coma do sacrifício de ``Pessah''. Ela
está expressa em Suas palavras, enaltecido seja Ele, ``O forasteiro
(toshab) e o mercenário estrangeiro não comerão dele'' (Êxodo 12:45).

\paragraph{Uma pessoa incircuncisa não deve comer do sacrifício de ``Pessah''}

Por esta proibição a pessoa incircuncisa fica avisada para não comer do
sacrifício de ``Pessah''. Ela está expressa em Suas palavras, ``E nenhum
incircunciso comerá dele'' (Êxodo 12:48). Toda pessoa incircuncisa que
comer dele será punida com o açoitamento.

\paragraph{Não permitir que um israelita apóstata coma do sacrifício de ``Pessah''}

Por esta proibição somos proibidos de permitir que um israelita
apóstata coma do sacrifício de ``Pessah''. Ela está expressa em Suas
palavras, enaltecido seja Ele, ``Nenhum estrangeiro comerá dele''
(Êxodo 12:43), que o Targum
(Onkelos), de acordo com a Tradição, traduz da seguinte forma: ``Nenhum
israelita apóstata''. A Mekhiltá também diz: ```Nenhum estrangeiro' se
refere a um israelita apóstata que adora ídolos''.

\paragraph{Uma pessoa impura não deve comer comida santificada}

Por esta proibição uma pessoa impura fica proibida de comer comida
santificada. Ela está expressa em Suas palavras, relativas a uma mulher
depois do parto, ``Em nenhuma santidade tocará'' (Levítico 12:4), sobre
as quais a Sifrá diz: ```Em nenhuma santidade tocará, e no santuário não
entrará': assim como entrar no Santuário em estado de impureza acarreta
o castigo de extinção, o mesmo acontece se se comer carne santificada em
estado de impureza''.

Esta aplicação da proibição ``não tocará'' (em nenhuma santidade) quanto
a ``comer'' voluntariamente está baseada no princípio estabelecido em
Macot sobre a interpretação de Suas palavras ``Em nenhuma santidade
tocará''. O que a Guemará de Macot diz é o seguinte: ``Com relação ao
impuro que comeu carne santificada, eu admito que a penalidade está
expressa em `E a alma que comer carne de sacrifício de oferta de paz\ldots{}
e tiver a sua impureza sobre si, será banida de seu povo' (Ibid., 7:20).
Mas onde encontramos a proibição necessária? Ela se encontra no texto:
`Em nenhuma santidade tocará'''.

A Guemerá diz ainda: ```Em nenhuma santidade tocará' é a proibição de
comer. Você diz que essa é uma proibição de `comer'? Será que ela não é
apenas uma proibição de `tocar'? Não. As Escrituras dizem: `Em nenhuma
santidade tocará, e no santuário não entrará', tornando assim
equivalentes a `santidade' e o `santuário'. Assim como o Santuário
acarreta a perda de uma alma (extinção), assim também todas as
santidades acarretam como penalidade a perda de uma alma. Se você disser
que é o `tocar', você conhece algum exemplo em que o `tocar' acarrete a
perda de uma alma? Portanto, o significado deve ser `comer'''.

A razão para que o Misericordioso tenha usado a palavra ``tocar'' em
relação a ``comer'' é para mostrar-nos que o tocar é igualado ao comer.

Por essas declarações fica claro que comer carne santificada é punido
com a extinção, se a ofensa for cometida deliberadamente, mas se for
involuntariamente, o transgressor deve levar um Sacrifício de Maior ou
Menor Valor, como explicamos ao tratar do preceito positivo 72.

As normas deste preceito estão explicadas no décimo terceiro capítulo
de Zebahim.

\paragraph{Não comer carne de sacrifícios consagrados que se tornaram impuros}

Por esta proibição somos proibidos de comer a carne de sacrifícios
consagrados que se tornaram impuros. Ela está expressa em Suas palavras,
enaltecido seja Ele, ``E a carne sagrada do sacrifício de paz que tocar
em tudo o que for impuro, não será comida'' (Levítico 7:19). A
penalidade por comer o que esta proibição determina é o açoitamento. Na
Tosseftá de Zebahin está explicado que uma pessoa pura que comer carne
impura recebe os quarenta açoites; e no segundo capítulo da Guemará de Pessahim está dito: ``A
impureza da `pessoa' é punida com a extinção, mas a impureza da `carne'
é um preceito negativo''.

As normas deste preceito já foram explicadas no décimo terceiro capítulo
do (Tratado) Zebahim.

\paragraph{Não comer ``notar''}

Por esta proibição somos proibidos de comer ``notar'', quer dizer, carne
de sacrifícios que foi deixada além do tempo determinado para seu
consumo.

Na Torah não consta nenhuma proibição expressa quanto a comer ``notar'',
mas ela prescreve a pena de extinção para todo aquele que o faça
através de Suas palavras, enaltecido seja Ele, na parte de Kedoshim,
relativas ao Sacrifício de Paz, ``E a sobra, até o terceiro dia, no fogo
será queimada. E se for comido no terceiro dia\ldots{} e será banida aquela
alma de seu povo'' (Levítico 19:6-8). Fica, assim, claro que se alguém o
fizer propositalmente, ele será punido com a extinção e se o fizer
involuntariamente, deverá levar um Sacrifício Determinado de Pecado. O
castigo está explicitamente enunciado nas Escrituras, mas a proibição
se deduz de Suas palavras, relativas à Consagração, ``E o estranho não
comerá delas pois elas são santidade'' (Êxodo 29:33), onde a palavra
``elas'' inclui qualquer parte de um sacrifício que se estrague e que,
como o ``notar'', não deva ser comida.

No Tratado de Meilá há o seguinte comentário sobre a afirmação na Mishná
de que ``pigul'' e ``notar'' não devem ser contados juntos, pois são
duas coisas diferentes: ``O princípio se aplica apenas quanto a
impurificar as mãos, de acordo com a lei Rabínica, mas eles devem ser
contados juntos no que se refere a comer, pois a Mishná diz, em nome de
Rabi Eliezer: `Ele não comerá delas pois elas são santidade': um
preceito negativo proíbe comer todas as santidades que se tornarem
impróprias''. Como ``pigul'' e ``notar'' são ambos santidades que se
tornaram impróprias, a proibição de comer qualquer um deles está
expressa em Suas palavras ``Ele não comerá delas pois elas são
santidade''. Já foi explicado que o castigo por comer ``notar'' é a
extinção.

\paragraph{Não comer ``pigul''}

Por esta proibição somos proibidos de comer ``pigul''. ``Pigul''
significa um sacrifício que se tornou impróprio porque, no momento em
que foi degolado ou oferecido, a pessoa que o ofertou teve intenções
inadequadas quanto à sua finalidade, tendo pensado em comê-lo ou em só
queimar as partes que devem ser queimadas depois de expirado o prazo
para fazê-lo, como explicamos claramente no segundo capítulo de
Zebahim.

A proibição de comer ``pigul'' está expressa em Suas palavras ``Ele não
comerá delas pois são santidade'' (Êxodo 29:33), como explicamos ao
tratar do preceito precedente, e deduzimos o castigo de Suas palavras
na parte de Tzav, relativas ao ``pigul'', ``E, se na hora de sacrificar,
pensou em comer da carne do sacrifício de ofertas de paz no terceiro
dia, este não será aceito, nem será levado em conta aquele que o
oferecer; impuro será (pigul), e quem comer dele, a sua iniquidade
levará'' (Levítico 7:18). A Tradição explica que este versículo se
refere a um sacrifício que se tornou inválido devido a intenções
inadequadas quanto à sua finalidade, tidas no momento da oferta, sendo
tal sacrifício conhecido como ``pigul'', e que Suas palavras ``Pensou em
comer da carne'' se referem apenas à intenção de comê-lo no terceiro
dia. Portanto a Guemará diz: ``Ouça com atenção! O versículo se refere
àquele que nesse momento tiver a intenção de comer a carne de seu
sacrifício no terceiro dia'', e nos diz que tal intenção desqualifica o
sacrifício e que todo aquele que comer dele após ter tido essa intenção
fica sujeito à extinção, pois está dito: ``E quem comer dele, a sua
iniquidade (avon) levará''; e a respeito do ``notar'' está dito: ``E
aquele que o comer, levará sobre si sua iniquidade (avon)\ldots{} e será
banida aquela alma de seu povo'' (Ibid., 19:8).

A Guemará de Queretot diz: ``Nunca negligencie um `guezerá shavá'!'' A
lei do `pigur', que é um dos preceitos essenciais da Torah, foi deduzida
apenas através de um `guezerá shavá'. Um versículo diz: `A sua
iniquidade (avon) levará' e o outro diz: `Levará sobre si sua iniquidade
(avon)'; assim como num caso há extinção, também há extinção no outro.

Aquele que comer ``pigul'' involuntariamente também fica obrigado a
levar um Sacrifício Determinado de Pecado.

As normas relativas ao ``pigur'' e ao ``notar'' estão explicadas em
vários trechos da Ordem de ``Kadashim''.

\paragraph{Um ``zar'' não deve comer ``terumá''}

Por esta proibição um ``zar''\footnote{Ver o preceito negativo 74, onde Maimônides explica que quando ele
  escreve ``zar'', está se referindo a qualquer um que não seja
  descendente de Aarão.} fica proibido de
comer qualquer ``terumá''\footnote{A Oferenda de Elevação (ver o preceito positivo 126).}. Ela está expressa em
Suas palavras ``E todo estranho não comerá
da santidade'' (Levítico 22:10), nas quais a expressão ``santidade''
significa ``terumá'', bem como primícias, pois elas também são chamadas
de ``terumá'', como explicarei; foi isso o que eu quis dizer quando
falei de ``qualquer `terumá'''.
A mesma lei se aplica a todo aquele que cometer sacrilégio
deliberadamente.
Aquele que comer ``terumá'' voluntariamente está sujeito à morte
pela mão dos Céus, mas não fica obrigado a acrescentar uma quinta parte,
como está explicado no sexto e sétimo capítulos do Tratado Terumá. No
nono capítulo de Sanhedrin, um ``zar'' que comer ``terumá'' está incluído na
lista dos
pecadores sujeitos à morte pela mão dos Céus, e isso se justifica por
Suas palavras ``E não levarão sobre si pecado, pois morrerão por isto
quando o profanarem'' (Ibid., 9), que são seguidas por ``E todo estranho
não comerá da santidade''. Da mesma forma, a Mishná diz no segundo
capítulo de Bicurim: ``Pode-se incorrer na pena de morte pelo Sacrifício de Elevação e pelas primícias; eles
estão sujeitos ao quinto adicional e são proibidos aos que não são
`Cohanim'''.
Rav discorda de todas essas leis da Mishná e diz que um ``zar'' que
come ``terumá'' é punido com o açoitamento\footnote{E não com a morte pela mão dos Céus.}, e é
sabido que, sendo um Taná, Rav pode discordar\footnote{Ou seja, apesar de ter vivido após os Tanaim (os responsáveis pela
  redação e impressão da Mishná), Rav era considerado uma autoridade tão
  alta que podia discordar da opinião unânime dos Tanaim.}. Já
explicamos em nosso ``Comentário sobre a Mishná'' que em todas as
discussões que não afetem o procedimento, e sim apenas a opinião, não darei uma decisão em favor de um ponto de vista ou de
outro. Sendo assim eu me absterei de dizer se o correto é o conceito de
Rav ou o da Mishná anônima, uma vez que todos concordam que ele está
sujeito ao açoitamento. Isto é consequência da regra explicada na
Introdução deste trabalho, segundo a qual todos aqueles que estão
sujeitos à morte pela mão dos Céus por ter violado qualquer um dos
preceitos negativos também estão sujeitos ao açoitamento. A mesma lei se
aplica àquele que deliberadamente cometer sacrilégio ao aproveitar-se
de objetos sagrados, como está demonstrado pelo que se diz sobre o caso
de um menino próximo da maioridade religiosa que faz uma consagração:
``Se ele o consagra e outros o comem, Rabi Yohanan e Resh Lakish são
ambos de opinião que eles devem ser punidos com o açoitamento''.

\paragraph{Um servo ou um criado de um ``Cohen'' não devem comer ``terumá''}

Por esta proibição até mesmo um servo ou um criado israelita de um
``Cohen'' ficam proibidos de comer ``terumá''. Ela está expressa em Suas
palavras ``Aquele que mora com o `Cohen' e o jornaleiro, não comerá da
santidade'' (Levítico 22:10). Aquele que o comer será tratado da mesma
forma que um ``zar''\footnote{Alguém que não é descendente de Aarão.}.

\paragraph{Um ``Cohen'' incircunciso não deve comer ``terumá''}

Por esta proibição um homem incircunciso fica proibido de comer
"terumá", e a mesma lei se aplica no caso de todas as outras santidades:
um homem incircunciso está proibido de comê-las. Esta proibição não está
expressamente enunciada nas Escrituras, mas provém de um ``guezerá
shavá''\footnote{``Expressão similar'', ou seja, uma analogia entre duas leis
  estabelecida com base na congruência verbal dos textos das Escrituras.}, e os guardiães da Tradição explicam ainda
que esta é uma proibição da Torah e não simplesmente uma proibição
Rabínica. A explicação se encontra em Yebamot: ``De onde se conclui que
um incircunciso não pode comer `terumá'? A Torah usa a expressão `toshab
vesakhir' (alguém que reside temporariamente e um criado) no caso do
cordeiro de Pessah e no caso do `terumá'. Portanto, assim como o
cordeiro do Pessah --- em relação ao qual foi usado `toshab vesakhir'
--- está proibido ao incircunciso, assim também o `terumá' --- em
relação ao qual também foi usado `toshab vesakhir' --- está proibido ao
incircunciso'', e a mesma lei se aplica a todos os outros sacrifícios
consagrados. Encontramos o mesmo texto na Sifrá, onde também lemos:
``Rabi Akiba diz: O uso da expressão `Todo homem' (Levítico 22:4)
significa que o homem incircunciso está incluído''.

Na Guemará de Yebamot também está explicado que a Torah permite que um
``mashukh''\footnote{Alguém que teve seu prepúcio puxado para a frente a fim de cancelar o sinal do pacto de Abraham.} coma ``terumá'', mas os Sábios proíbem
isso, porque ele parece incircunciso.

Ficou assim claro que um homem incircunciso está proibido de comer pela
Torah, e um ``mashukh'' pela lei Rabínica. Isto deve ser compreendido.

No mesmo trecho lemos que de acordo com a lei Rabínica um ``mashukh''
deve ser circuncidado novamente.

\paragraph{Um ``Cohen'' impuro não deve comer ``terumá''}

Por esta proibição um ``Cohen'' que estiver impuro fica proibido de
comer ``terumá''. Ela está expressa em Suas palavras ``Todo homem da
semente de Aarão, que for leproso, ou que tiver fluxo, das santidades
não comerá até que se purifique'' (Levítico 22:4).

Na Guemará de Macot lemos: ``De que forma deduzimos a proibição
necessária relativa à Oferenda de Elevação? Das palavras `Todo homem
(pessoa) da semente de Aarão, que for leproso\ldots{}' E que coisas são
permitidas de forma igualitária à `semente de Aarão'? Você é obrigado a
dizer: a Oferenda de Elevação'', sendo que ``de forma igualitária à
semente de Aarão'' significa que todos os que vêm de sua semente, tanto
machos como fêmeas, podem comer.

Esta proibição aparece novamente em Suas palavras, abençoado seja Ele,
``E guardarão (veshameru) este Meu mandado'' (Ibid.,9).

O castigo pela contravenção desta proibição é a morte pela mão dos Céus.
No nono capítulo de Sanhedrin um ``Cohen'' impuro que comer uma Oferenda
de Elevação pura está incluído na lista dos pecadores que estão sujeitos
à morte pela mão dos Céus, e isto se justifica pelas Suas palavras ``E
guardarão este Meu mandado e não levarão sobre si pecado''.

\paragraph{Uma ``halalá'' não deve comer alimento sagrado}

Por esta proibição uma ``halalá''\footnote{A filha de um ``Cohen'' que se casa com alguém com quem não lhe é
  permitido casar-se.} fica proibida de
comer alimento sagrado que lhe teria sido permitido comer, ou seja, o
peito e a coxa. Ela está expressa em Suas palavras ``E a filha do
`Cohen', quando se casar com um homem estranho, ela não comerá daquilo
que se separa das santidades'' (Levítico 22:12).

Na Guemará de Yebamot lemos: `` `Quando se casar com um homem estranho':
assim que ela passar a viver com um homem desqualificado, ele a
desqualificará''. As palavras ``Ela não comerá do que é separado das
santidades'' são interpretadas como significando ``Ela não comerá
daquilo que se separa das oferendas consagradas'', ou seja, o peito e a
coxa.

No mesmo trecho está dito: ``A Torah poderia ter dito `Ela não comerá
das oferendas consagradas'. Por que `das (bitrumath) santidades'? É para
ensinar-nos duas coisas'', a saber, que se a filha de um ``Cohen'' viver
com um homem desqualificado, ele a desqualifica no que se refere a comer
``terumá''\footnote{A Oferenda de Elevação.}, e que se ela esteve casada com um
``zar''\footnote{Alguém que não seja descendente de Aarão.} e ele morreu, ela recupera o direito à
Oferenda de Elevação mas não o direito ao peito e a coxa.
 
Consequentemente, esta proibição, a saber, ``Não comerá das santidades''
abrange dois assuntos: primeiro, ela proíbe uma ``halalá'' de comer
alimento sagrado; segundo, ela proíbe a filha de um ``Cohen'' que foi
casada com um ``zar'' de comer o peito e a coxa, mesmo que seu marido
tenha morrido ou se divorciado dela.

Contudo, a proibição de comer ``terumá'' enquanto ela viver com seu
marido, se ele for um ``zar'', não está baseada neste versículo, ela foi
deduzida por aqueles que interpretam a Torah de Suas palavras ``E todo
estranho (zar) não comerá da santidade'' (Ibid., 10). Enquanto ela viver
com um ``zar'', ela própria será uma ``zar'' e a ela aplicaremos a lei
do ``zar''.

Você deve compreender isto, e deve saber que ela também está sujeita ao
açoitamento se violar esta proibição.

\paragraph{Não comer a oblação de um ``Cohen''}

Por esta proibição somos proibidos de comer a Oblação de cereal de um
"Cohen". Ela está expressa em Suas palavras, enaltecido seja Ele, ``E
toda oblação de `Cohen' será queimada totalmente, não será comida''
(Levítico 6:16) e está repetida com relação aos bolos assados do ``Cohen
Gadol'', que também são uma oblação. A transgressão desta proibição
também é punida com o açoitamento.

A Sifrá diz: ```Será queimada totalmente': isso significa que somos
proibidos por um preceito negativo de comer qualquer coisa que deva ser
queimada totalmente''.

\paragraph{Não comer carne de sacrifícios de pecado cujo sangue tenha sido
levado para dentro do Santuário}

Por esta proibição os ``Cohanim'' ficam proibidos de comer a carne dos
Sacrifícios de Pecado que devem ser oferecidos no Santuário. Ela está
expressa em Suas palavras, enaltecido seja Ele, ``E todo sacrifício de
pecado cujo sangue for trazido à tenda da assinação para expiar na
santidade, não será comido; no fogo será queimado'' (Levítico 6:23).
Comê-la é punido com o açoitamento.

A Sifrá diz: ```Não será comido; no fogo será queimado': isto significa
que somos proibidos, por um preceito negativo, de comer qualquer coisa
que deva ser queimada''.

\paragraph{Não comer sacrifícios consagrados que tenham sido invalidados}

Por esta proibição somos proibidos de comer sacrifícios consagrados que
tenham sido invalidados em virtude de um defeito causado
propositalmente, (como foi explicado no Tratado Bekhorot), depois do
degolamento, por qualquer uma das formas que fazem com que um sacrifício consagrado se
torne alimento proibido. Esta proibição está expressa em Suas palavras
``Não comerás nada do que for abominável'' (Deuteronômio 14:3), sobre
as quais diz o Sifrei: ```Não comerás nada do que for abominável' se
refere a sacrifícios consagrados que foram invalidados''. Diz também:
``Rabi Eliezer ben Jacob diz: De que forma sabemos que, se alguém fizer
um corte na orelha do primogênito de um animal e comer dele, ele estará
violando um preceito negativo? Pelas palavras das Escrituras: `Não
comerás nada do que for abominável'''.

Comê-los é punido com o açoitamento.

As normas deste preceito estão explicadas no Tratado Bekhorot.

\paragraph{Não comer o segundo dízimo de cereais não remido fora de Jerusalém}

Por esta proibição somos proibidos de comer o segundo dízimo de cercais
fora de Jerusalém. Ela está expressa em Suas palavras, enaltecido seja
Ele, ``Mas não te será permitido comer em tuas cidades o dízimo de teus
cereais'' (Deuteronômio 12:17).

A punição por comer o segundo dízimo não remido é o açoitamento, de
acordo com a explicação expressa no final de Macot, ou seja, se ele for
comido fora de Jerusalém depois que ele tenha ``visto a fachada do
Templo'', ou seja, depois que ele tenha sido levado para dentro das
muralhas de Jerusalém. Isto está expresso no Talmud: ``A partir de
quando se fica sujeito à penalidade de açoitamento? A partir do momento
em que ele `vir a fachada do Templo'''.

\paragraph{Não consumir o segundo dízimo de vinho não remido fora de Jerusalém}

Por esta proibição somos proibidos de consumir o segundo dízimo de vinho
fora de Jerusalém. Ela está expressa em Suas palavras, enaltecido seja
Ele, ``Mas não te será permitido comer em tuas cidades o dízimo de teus
cereais, e de teu mosto'' (Deuteronômio 12:17).

O açoitamento é a punição por consumi-lo, desde que isso seja feito nas
mesmas condições que no caso do dízimo dos cereais.

\paragraph{Não consumir o segundo dízimo de azeite não remido fora de Jerusalém}

Por esta proibição somos proibidos de consumir o segundo dízimo de
azeite fora de Jerusalém. Ela está expressa em Suas palavras ``Mas não
te será permitido\ldots{} e de teu azeite'' (Deuteronômio 12:17). Consumi-lo
é punido com o açoitamento, desde que seja feito nas mesmas condições
que no caso do dízimo dos cereais.

Se você está surpreso por contarmos as proibições relativas aos dízimos
de cereais, da vindima e do azeite como três preceitos, você deve saber
que aquele que comer os três ao mesmo tempo está sujeito a um
açoitamento por cada um deles pois a proibição expressa neste versículo
(``Não te será permitido comer em tuas cidades o dízimo de teus
cereais, e de teu mosto, e de teu azeite'') não é um ``lav
shebikhlalut''\footnote{Uma proibição negativa geral (ver o nono fundamento).}, pelo qual não se aplica a
penalidade de açoitamento. Ao contrário, este texto indica uma divisão.
Está explicitamente dito na Guemará de Macot: ``Se alguém comer do
dízimo de cereais, de vinho e de azeite, ele estará sujeito a um castigo
por cada um deles separadamente''. Mas aplica-se o açoitamento por uma
proibição coletiva? O texto é redundante. Veja bem: na Torah já estava
dito `E comerás diante do Eterno, teu Deus\ldots{} o dízimo de teu grão, teu
mosto, e teu azeite' (Ibid., 14:23); então por que ela os expõe de novo,
detalhadamente? Deve ser para estabelecê-los separadamente.

A Guemará de Macot diz: ``Veja bem: se já estava escrito `E comerás
diante do Eterno, teu Deus\ldots{} o dízimo de teu grão, teu mosto, e teu
azeite', não poderia o Todo Misericordioso, ter simplesmente dito o
seguinte: `Não deves comê-los dentro de teus portões?' Que outro
objetivo poderia Ele ter ao enunciá-los novamente, em detalhe, a não ser
o de enfatizar separadamente a proibição relativa a cada caso?''

Assim foi deixado claro que cada um dos assuntos mencionados neste
versículo é objeto de um preceito negativo diferente. Voltarei a este
assunto e complementarei o exame das outras proibições expressas neste
versículo.

\paragraph{Não comer um primogênito sem defeito fora de Jerusalém}

Por esta proibição somos proibidos de comer um primogênito sem defeito
fora de Jerusalém. Ela está expressa em Suas palavras ``Mas não te será
permitido comer dentro de tuas cidades\ldots{} nem os primogênitos de teu
gado, e de teu rebanho'' (Deuteronômio 12:17), sobre as quais o Sifrei
diz: ```Os primogênitos' se refere à primeira cria e o objetivo do
texto é ensinar-nos que um `zar'\footnote{Alguém que não é descendente de Aarão.} que comer um primogênito, seja antes ou depois de seu sangue ter sido aspergido,
estará dessa forma transgredindo um preceito negativo''.

Assim, fica claro que esta proibição abrange dois assuntos: ela proíbe
um ``zar'' de comer um primogênito sem defeito, e um ``Cohen'' de comer
um primogênito sem defeito fora de Jerusalém. Esses dois assuntos
constituem a lei relativa ao primogênito sem defeito.

A transgressão desta proibição é punida com o açoitamento.

\paragraph{Não comer o Sacrifício de Pecado e o Sacrifício de Delito fora do campo do
Santuário}

Por esta proibição somos proibidos de comer o Sacrifício de Pecado e o
Sacrifício de Delito fora do Campo do Santuário, e esta proibição se
aplica até mesmo aos ``Cohanim''. Ela está expressa em Suas palavras, no
mesmo versículo, ``De teu gado e de teu rebanho'' (Deuteronômio 12:17).
É como se Ele
tivesse dito: ``Não deves comer dentro de teus portões o dízimo de teus
cereais, de teu gado, nem de teu rebanho''; e o Sifrei explica: ```De
teu gado e de teu rebanho': o versículo se refere apenas ao caso de uma
pessoa que viola um preceito negativo ao comer um Sacrifício de Pecado
ou um Sacrifício de Delito fora das cortinas''\footnote{Fora das cortinas do Tabernáculo.} e é
punida com o açoitamento.

Da mesma forma, aquele que comer os sacrifícios menos sagrados fora das
muralhas também será punido com o açoitamento, como está explicado na
Guemará de Macot, porque comer qualquer uma das santidades fora do local
designado para isso está incluído na proibição ``Não te será permitido''
(Ibid.).

\paragraph{Não comer carne de um Holocausto}

Por esta proibição somos proibidos de comer a carne de um Holocausto.
Ela está expressa em Suas palavras, enaltecido seja Ele, ``Mas não te
será permitido\ldots{} nem os teus votos que ofereceres'' (Deuteronômio
12:17). Isto é como se Ele tivesse dito: ``Não deves comer os votos que
ofereceres''; e o Si-frei explica: ```Nem os teus votos': o objetivo do
versículo é apenas mostrar-lhe que aquele que comer um Holocausto, seja
antes ou depois de aspergir seu sangue, seja dentro ou fora das
cortinas, estará violando um preceito negativo''.

Este preceito negativo serve como proibição contra qualquer tipo de
sacrilégio.

Todo aquele que desobedecer a este preceito voluntariamente --- ou seja,
que comer a carne de um Holocausto, ou que se tornar culpado de
sacrilégio por obter um proveito qualquer de qualquer um dos outros
sacrifícios consagrados, como explicado no Tratado Meilá --- será
punido com o açoitamento. Aquele que lhe desobedecer involuntariamente
fica obrigado a levar um Sacrifício de Delito por
sacrilégio\footnote{Ver o preceito positivo 71.} e a pagar de acordo com o valor daquilo
de que ele usufruiu, acrescido da quinta parte\footnote{Ver o preceito positivo 118.},
como explicado no Tratado Meilá.

No nono capítulo de Sanhedrin lemos: ``Se ele cometeu sacrilégio
deliberadamente, Rabi Yehudá diz que ele está sujeito à morte, mas os
Sábios dizem que ele está sujeito ao açoitamento''; e os Sábios citam em
apoio a seu ponto de vista: ```Pois morrerão por isto' (Levítico 22:9),
significando que morrerão por `isto', mas não por causa do
sacrilégio''.

\paragraph{Não comer sacrifícios menos sagrados antes de aspergir seu sangue sobre o altar}

Por esta proibição somos proibidos de comer dos sacríficios menos
sagrados antes de aspergir seu sangue. Ela está expressa em Suas
palavras ``Mas não te será permitido comer em tuas cidades\ldots{} nem tuas
ofertas voluntárias'' (Deuteronômio 12:17). É como se Ele tivesse dito:
``Não deverás comer tuas ofertas voluntárias'' e, de acordo com a
Tradição, ``o versículo se refere apenas
a alguém que viola um preceito negativo ao comer o Sacrifício de Graças
ou o de Paz antes de aspergir seu sangue''; ele também será punido com o
açoitamento.

\paragraph{Um ``Cohen'' não pode comer as primícias fora de Jerusalém}

Por esta proibição um ``Cohen'' fica proibido de comer as primícias fora
de Jerusalém\footnote{Ver os preceitos positivos 125 e 132.}. Ela está expressa em Suas palavras
``Não te será permitido comer em tuas cidades\ldots{} as oferendas (terumá)
de tua mão'' (Deuteronômio 12:17) pois, de acordo com a Tradição, ``A
expressão `As oferendas de tua mão' significa as primícias''. Como este
versículo menciona explicitamente tudo que deve ser levado, e como ele
inclui ``as oferendas de tua mão'', não pode haver dúvida de que está
dito que elas devem ser levadas\footnote{Deuteronômio 26:2-4.}. Mas nós sabemos
que a ``terumá'' não precisa ser levada; então como pode Ele ter-nos
proibido de comê-la ``em tuas cidades''?

De acordo com o Sifrei, ``Este versículo se refere apenas ao caso de uma
pessoa que, ao comer as primícias sem declamar sobre
elas\footnote{Ver o preceito positivo 132.}, viola um preceito negativo''. E está
explicado no final de Macot que alguém se torna culpado
apenas\footnote{A pessoa só se torna culpada ao comê-las antes de colocá-las no
Campo do Santuário.} antes de colocá-las no Campo do Santuário,
mas uma vez que as colocar ali ele estará inocente ainda que não tenha
declamado sobre elas.

Uma das condições impostas com relação ao segundo dízimo se aplica
também às primícias, a saber, que aquele que as comer fora de Jerusalém
não será culpado a menos que elas tenham ``visto a fachada do Templo''.
Aquele que as comer depois de elas terem visto ``a fachada do Templo'',
mas antes de que elas tenham sido colocadas no Campo do Santuário,
estará sujeito apenas ao açoitamento, se ele for um ``Cohen''; mas se
for um israelita, ele estará sujeito à morte pela mão dos Céus por ter
comido as primícias, ainda que ele tenha declamado sobre elas primeiro.

A Mishná diz explicitamente: ``A Oferenda de Elevação e das primícias
podem causar a pena de morte, sujeitam ao quinto adicional e são
proibidos para quem não for `Cohen'''. Assim, se ele as comer
intencionalmente estará sujeito à morte, e se o fizer
involuntariamente, ele deve acrescentar uma quinta parte, como no caso
da Oferenda de Elevação. Pois como as Escrituras se referem a elas como
``a oferenda (terumá) de tua mão'', segue-se que elas estão sujeitas à
mesma lei que a ``terumá''.

É importante que você compreenda bem isto para que não se engane a este
respeito. A lei é a seguinte: um ``Cohen'' que comer das primícias
depois que elas tenham ``visto a fachada do Templo'' mas antes de que
tenham sido colocadas no Campo do Santuário está sujeito ao açoitamento,
sendo que a proibição está expressa em Suas palavras ``Não te será
permitido comer em tuas cidades\ldots{} as oferendas de tua mão'', de acordo
com o que está explicado em Macot, assim como no caso de um israelita
que está sujeito ao açoitamento por comer o segundo dízimo fora de
Jerusalém, embora ele lhe pertença. Mas um israelita que comer as
primícias em qualquer local, depois que elas tenham
``visto a fachada do Templo'', está sujeito à morte pela mão dos Céus,
estando
a proibição expressa nas palavras ``E todo estranho não comerá da
santidade'' (Levítico 22:10), como explicamos ao tratar do número 133
destes preceitos.
As normas deste preceito estão explicadas na Guemará de Macot.

\paragraph{Um ``zar'' não pode comer os sacrifícios mais sagrados}

Por esta proibição um ``zar''\footnote{Alguém que não seja descendente de Aarão.} fica proibido de
comer dos sacrifícios mais sagrados. Ela está expressa em Suas palavras
``O estranho não comerá delas pois é santidade'' (Êxodo 29:33).

Não se fica sujeito ao açoitamento a menos que se coma dentro do Campo
do Santuário e depois de ter aspergido o sangue do sacrifício.

\paragraph{Não comer o segundo dízimo impuro não remido, nem mesmo em Jerusalém}

Por esta proibição somos proibidos de comer, mesmo estando em Jerusalém,
um segundo dízimo que tenha se tornado impuro, até que ele seja remido.
O princípio aceito a respeito é que um segundo dízimo que se tornou
impuro deve ser remido até mesmo em Jerusalém, como está explicado em
Macot. A proibição está expressa em Suas palavras ``Não comi dele, em
estado impuro'' (Deuteronômio 26:14) que significam, de acordo com a
Tradição: ``Nem quando eu estava impuro e ele puro, nem quando eu estava
puro e ele impuro''.

A Guemará de Macot explica ainda que é proibido comer o segundo dízimo
ou as primícias que se tornaram impuros e que uma pessoa que se tornou
impura está sujeita ao açoitamento se ela os comer, desde que ela coma o
dízimo não remido em Jerusalém em estado de impureza; somente nesse
caso ela estará sujeita ao açoitamento, como dissemos.

As normas deste preceito estão explicadas no final de Macot.

\paragraph{Não comer o segundo dízimo durante o período de luto}

Por esta proibição somos proibidos de comer o segundo dízimo durante o
período de luto\footnote{Ver o preceito positivo 37.}. Ela está expressa em Suas
palavras, enaltecido seja Ele, ``Não comi do segundo dízimo no primeiro
dia de luto'' (Deuteronômio 26:14). A Mishná diz que o dízimo e as
primícias devem ser levados a Jerusalém, devem ser
declarados\footnote{Ver o preceito positivo 131.} e são proibidos a um
``onen''\footnote{Alguém que esteja de luto.}. Da mesma forma, uma pessoa de luto fica
proibida por este versículo de comer dos sacrifícios consagrados, pois
também está escrito na Torah: ``E me aconteceram tais coisas; se eu
tivesse comido do sacrifício de pecado do dia, agradaria aos olhos do
Eterno?'' (Levítico 10:19).

As leis do luto estão explicadas no oitavo capítulo de Pessahim e no
segundo capítulo de Zebahim.

Aquele que comer dos sacrifícios consagrados ou do dízimo durante o
luto estará sujeito ao açoitamento.

\paragraph{Não gastar o dinheiro do resgate do segundo dízimo a não ser com
comida e bebida}

Por esta proibição somos proibidos de gastar o dinheiro do segundo
dízimo em outras coisas que não comida e bebida. Ela está expressa em
Suas palavras, enaltecido seja Ele, ``E não o troquei para fazer o
sepultamento de um morto'' (Deuteronômio 26:14), a respeito das quais
diz o Sifrei: ``Não usei nada dele para um caixão ou uma mortalha''.
Aquele que gastar qualquer parte do dinheiro em outras coisas deve
gastar uma quantia equivalente em comida, como está explicado no lugar
apropriado.

``O morto'' está mencionado para dar maior ênfase, como se Ele tivesse
dito: ``Embora importante, você não deve gastar o dinheiro do segundo
dízimo para esse fim''.

Parece-me que uma vez que o Enaltecido nos ordena usar o dinheiro do
segundo dízimo apenas para comida, pelas Suas palavras ``E darás este
dinheiro'' (Deuteronômio 14:26), gastá-lo em outras coisas que não
alimento equivale a dá-lo aos mortos, já que eles não podem usufruir
dele.

\paragraph{Não comer ``tebel''}

Por esta proibição somos proibidos de comer ``tebel'', isto é, um
produto do qual não se tenha separado a Oferenda de Elevação e os
dízimos. Ela está expressa em Suas palavras, enaltecido seja Ele, ``E
não profanarão as santidades dos filhos de Israel, que eles separarem
(yarimu) para o Eterno'' (Levítico 22:15).

A transgressão desta proibição --- ou seja, comer o ``tebel'' é punida
com a morte pela mão dos Céus, como se pode deduzir pelo fato de que
aqui Ele diz: ``Não profanarão'', e de que no. caso da Oferenda de
Elevação Ele diz igualmente: ``As santidades dos filhos de Israel não
profanareis'' (Números 18:32).

A referência à ``profanação'' em ambos os casos indica que aqui, assim
como no caso da Oferenda de Elevação, a penalidade é a morte, como
explicamos.
A Guemará de Sanhedrin diz: ``De que forma sabemos que aquele
que comer `tebel' está sujeito à morte? Pelo versículo `Não profanarão
as santidades dos filhos de Israel'. O versículo se refere àquilo que
ainda vai ser oferecido, e a identidade da lei se conhece pelo uso da
palavra `profanação' neste caso bem como no da Oferenda de Elevação''. A
expressão ``aquilo que ainda vai ser oferecido'' significa que é como se
Ele tivesse dito: ``Não deves profanar as santidades que as pessoas
`ainda vão separar' para o Eterno''. É por isso que Ele diz, enaltecido
seja Ele, ``et asher yarimu'' (que eles separarão), usando o verbo no
futuro. No versículo seguinte a este Ele diz: ``A fim de que não levem
sobre si delito de culpa comendo as suas santidades'' (Levítico 22:16).
A Guemará de Macot diz: ``Eu poderia pensar que se é culpado apenas por
comer o `tebel' do qual nenhuma contribuição foi separada ainda. Mas
e no caso da grande Oferenda de Elevação ter sido separada, mas a
Oferenda de Elevação do dízimo\footnote{O dízimo que o levita deve separar para dar ao ``Cohen''.} ainda não, ou quando o
primeiro dízimo tiver sido separado, mas o segundo dízimo não, ou ainda
o dízimo dos pobres, de que forma\footnote{I.e., de que forma tomamos conhecimento da proibição de comer tais
  produtos?} comer esses
produtos? Através dos seguintes textos instrutivos: `Mas não te será
permitido comer ``em tuas cidades'' o dízimo de teus cereais etc.'
(Deuteronômio 12:17); e mais adiante está dito: `A fim de que os comam e
se fartem' (Ibid., 26:12). Assim como neste último, no versículo
anterior faz-se referência ao dízimo do pobre, e o Misericordioso ordena
`Não te será permitido comer'''.

Contudo, tudo isto se refere apenas ao açoitamento. O castigo de morte
só é decorrente da grande Oferenda de Elevação e da Oferenda de
Elevação do dízimo. Pois aquele que comer do primeiro dízimo antes que
a Oferenda de Elevação do dízimo tenha sido retirada dele está sujeito
à morte, de acordo com Suas palavras, enaltecido seja Ele, aos Levitas,
ao ordenar-lhes que separassem um dízimo do dízimo, ``E as santidades
dos filhos de Israel não profanareis e não morrereis'' (Números 18:32).
Esta é a proibição de comer do primeiro dízimo que ainda é ``tebel" e
sua violação acarreta a morte, como está explicado no Tratado Demai.

O essencial de todo este debate é o seguinte: aquele que comer
``tebel'' do qual ainda não se tenha separado a grande Oferenda de
Elevação e a Oferenda de Elevação do dízimo está sujeito à morte, e a
proibição está expressa nas palavras ``Não profanarão as santidades dos
filhos de Israel\ldots{}'', como explicamos ao tratar deste preceito. Aquele
que comer ``tebel'' depois de ter separado as Oferendas de Elevação mas
antes de separar todos os dízimos está sujeito ao açoitamento, e a
proibição está expressa nas palavras ``Mas não te será permitido comer
em tuas cidades o dízimo de teus cereais\ldots{}''. Você deve se lembrar
disso e não se enganar a esse respeito.

As normas relativas ao ``tebel'' estão explicadas em vários trechos dos
Tratados Demai, Terumot e Maasserot.

\paragraph{Não alterar a ordem prescrita para separar o dízimo da colheita}

Por esta proibição somos proibidos de mudar a ordem das doações dos
produtos; devemos separá-las de acordo com a ordem determinada. A
explicação é a seguinte: quando, por exemplo, o trigo tiver sido
debulhado e arrumado numa pilha uniforme e tiver se tornado ``tebel'',
deve-se separar dele primeiro a grande Oferenda de Elevação que é uma
quinquagésima parte, e depois um décimo do que restar, que será o
primeiro dízimo, e finalmente um décimo do restante, que será o segundo
dízimo. A grande Oferenda de Elevação deve ser entregue ao ``Cohen'', o
primeiro dízimo ao Levita e o segundo dízimo deve ser comido pelo seu
proprietário, em Jerusalém. Esta é a ordem segundo a qual as doações
devem ser separadas, e a proibição de separar primeiro o que deve ser
separado por último e retardar o que deve ser separado primeiro está
expressa em Suas palavras, enaltecido seja Ele, ``Não tardarás em
oferecer da plenitude de tua colheita e do que sai de tuas prensas''
(Êxodo 22:28), que é como se Ele tivesse dito: ``Não demorarás em
trazer, da plenitude de tua colheita e do que sai de tuas prensas, aquilo que deve ser oferecido em primeiro lugar''.

Na Mishná de Terumot lemos: ``Se alguém oferecer a Oferenda de Elevação
antes das primícias, ou o primeiro dízimo antes da Oferenda de
Elevação, ou o segundo dízimo antes do primeiro, sua ação será válida,
embora ele esteja transgredindo um preceito negativo, pois está dito:
``Não tardarás em oferecer da plenitude de tua colheita e do que sai de
tuas prensas''.

A Mekhiltá diz: ```A plenitude de tua colheita': ou seja, as primícias
retiradas das colheitas plenas. `E o que sai de tuas prensas': ou seja,
a Oferenda de Elevação. `Não tardarás': não deixe que o segundo preceda
o primeiro, o primeiro preceda a Oferenda de Elevação, ou a Oferenda de
Elevação preceda as primícias''. E acrescenta: ``A partir disso deduz-se
o princípio de que se alguém oferecer a Oferenda de Elevação antes das
primícias, ou o segundo dízimo antes do primeiro, embora ele esteja
violando um preceito negativo, seu ato ainda será válido''. E está
explicado no primeiro capítulo de Temurá que aquele que proceder fora de
ordem não está sujeito ao açoitamento.

\paragraph{Não adiar o pagamento de promessas}

Por esta proibição somos proibidos de adiar promessas, sacrifícios
voluntários e outras oferendas a que nos tenhamos comprometido. Ela está
expressa em Suas palavras, enaltecido seja Ele, ``Quando fizeres algum
voto ao Eterno, teu Deus, não demorarás em pagá-lo''
(Deuteronômio23:22). E, de acordo com a Tradição, não se transgride
esta proibição até que três Festivais tenham se passado.

As normas deste preceito estão explicadas no início do Tratado Rosh Hashaná.

\paragraph{Não comparecer a um festival sem um sacrifício}

Por esta proibição somos proibidos de ir ao Santuário num Festival sem
um sacrifício para ser oferecido ali. Ela está expressa em Suas palavras
enaltecido seja Ele, ``E ninguém aparecerá diante de Mim com as mãos
vazias'' (Exodo 23:15).

As normas deste preceito estão explicadas no Tratado Haguigá. Esta
proibição não se aplica às mulheres.

\paragraph{Não deixar de cumprir uma obrigação oral, mesmo que não se tenha
feito um juramento}

Por esta proibição somos proibidos de infringir uma obrigação à qual
tenhamos nos comprometido oralmente, ainda que se tenha feito um
juramento. O que se tem em mente aqui é um voto, como quando alguém
diz: ``Se isto ou aquilo acontecer'', ou ``Se eu fizer isto ou aquilo'',
não tocarei em nada de que cresce ``no mundo'', ou ``nesta cidade'', ou
``num determinado tipo'', como por exemplo vinho, ou leite, ou peixe, ou algo assim; ou ainda se alguém
disser: ``Eu renunciarei meus direitos conjugais'', ou se fizer
qualquer outro tipo de promessa que envolva uma obrigação do tipo das
que estão citadas como exemplo no Tratado Nedarim. Se alguém fizer isso,
ele será obrigado a cumprir essa promessa\footnote{Ver o preceito positivo 94.} e estará
proibido de quebrar sua palavra, de acordo com Suas palavras, enaltecido
seja Ele, ``Não profanará (yahel) a sua palavra'' (Números 30:3), que
são interpretadas como significando: ``não fará as suas palavras
profanas'' (hulin), ou seja, ele não deixará de cumprir o que ele se
comprometeu a fazer. Nas palavras da Guemará de Shebuot: ``Os votos
estão sob a proibição `Não profanará a sua palavra'''.

No Sifrei lemos: ```Não profanará' nos diz que se transgridem duas
proibições --- `Não profanará' e `Não demorarás em pagá-lo'''
(Deuteronômio 23:22). Quer dizer, se alguém tiver prometido um
sacrifício ou algo semelhante, como por exemplo uma doação para o
tesouro do Templo, ou para a caridade, ou para uma sinagoga, ou similar,
e não tiver cumprido sua promessa até depois da passagem dos três
Festivais, ele será culpado por ambos `Não demorarás em pagá-lo' e `Não
profanará'. E alguém que cometer uma transgressão ao fazer algo que ele
se compremeteu a não fazer é punido com o açoitamento.

As normas deste preceito estão explicadas em detalhe no Tratado Nedarim.

\paragraph{Um ``Cohen'' não pode casar-se com uma ``zoná''}

Por esta proibição um ``Cohen'' fica proibido de tomar uma ``zoná'' como
esposa. Ela está expressa em Suas palavras ``Mulher prostituta (zoná) ou
profana não tomarão'' (Levítico 21:7). Se um ``Cohen'' se chegar a uma
``zoná'' ele estará sujeito ao açoitamento.

\paragraph{Um ``Cohen'' não pode casar-se com uma ``halalá''}

Por esta proibição um ``Cohen'' fica proibido de tomar uma
``halalá''\footnote{A filha de um ``Cohen'' que se casou com alguém com quem estava
  proibida de casar-se.} como esposa. Ela está expressa em Suas
palavras ``Mulher prostituta ou profana (halalá) não tomarão'' (Levítico
21:7). Se um ``Cohen'' se chegar a ela, ele estará sujeito ao
açoitamento.

\paragraph{Um ``Cohen'' não pode casar-se com uma mulher divorciada}

Por esta proibição um ``Cohen'' fica proibido de casar-se com uma mulher
divorciada. Ela está expressa em Suas palavras ``Nem mulher divorciada
de seu marido não tomarão'' (Levítico 21:7).


\paragraph{Um ``Cohen Gadol'' não pode casar-se com uma viúva}

Por esta proibição o ``Cohen Gadol'' (e apenas ele) fica proibido de
casar-se com uma viúva. Ela está expressa em Suas palavras, enaltecido
seja Ele, ``Viúva ou divorciada, ou profana ou prostituta, a estas não
tomará'' (Levítico 21:14).

Na Guemará de Kidushin se encontra a razão pela qual Ele proíbe
novamente o ``Cohen Gadol'' de casar-se com uma mulher divorciada, uma
``halalá'' ou uma ``zoná''. É que se um ``Cohen Gadol'' se chegar a uma
mulher que seja ao mesmo tempo viúva, divorciada, ``halalá'' e ``zoná'',
ele será punido por açoitamento quatro vezes, enquanto que um ``Cohen''
comum será punido apenas três. No mesmo trecho lemos: ``Se com uma
viúva, uma mulher divorciada e profana, e uma meretriz'' --- que está
explicado como significando uma única mulher que seja todas essas
coisas --- ``e se elas forem nessa ordem, ele é culpado por cada
relação''.

O significado das palavras ``Se elas forem nessa ordem'' é que as
desqualificações devem ter acontecido à mulher na ordem em que foram
mencionadas no versículo, ou seja, que primeiro ela tenha se tornado uma
viúva, depois uma divorciada, depois uma ``halalá'', e finalmente uma
``zoná''. Somos a confirmar que isso é assim, porque desejamos que ele
fique sujeito ao açoitamento quatro vezes por ter-se chegado a uma
mulher, em uma ocasião. E um princípio aceito que uma proibição não
vigora sobre algo que já está proibido por uma outra proibição, a menos
que ela seja um ``issur mossif''\footnote{Uma proibição adicional, ou seja, uma proibição que se aplica a
mais tipos de pessoas.}, um ``issur
colel''\footnote{Uma proibição inclusiva, ou seja, uma proibição que se aplica a
mais tipos de coisas.}, ou um ``issur bevad
ehad''\footnote{Uma proibição simultãnea, ou seja, duas proibições que entram em
  vigor ao mesmo tempo.}, como explicamos no lugar apropriado em
nosso ``Comentário sobre a Mishná'' de Queretot; e cada uma delas só
será um ``issur mossif'' apenas se elas ocorrerem nessa ordem, como está
ali exposto. Contudo, se várias mulheres estiverem envolvidas como por
exemplo se ele se chegar a uma mulher viúva, a uma outra que for
``halalá'', a uma outra que for ``zoná'' e a uma outra que for
divorciada --- não há dúvidas de que ele está sujeito ao açoitamento por
cada uma delas separadamente.

Você pode objetar e argumentar o seguinte: ``Uma vez que é um princípio
aceito que não se fica sujeito ao açoitamento pela violação de um ``lav
shebikhlalut''\footnote{Reprimenda negativa geral.}, por que ele ficaria sujeito ao
açoitamento por cada violação separadamente, já que elas são todas
proibidas por um único preceito negativo?'' A resposta é que o objetivo
de repetir a proibição que impede o ``Cohen Gadol'' de casar-se com uma
mulher divorciada, com uma ``zoná'' ou com uma ``halalá'' é exatamente o
de deixar claro que nesse caso ele está sujeito à mesma regra que um
``Cohen'' comum, e é punido com o açoitamento separadamente. Sabemos
que um ``Cohen'' comum está sujeito ao açoitamento por cada violação
pelo fato de que uma delas foi mencionada especificamente, a saber,
``Nem mulher divorciada de seu marido não tomarão'' (Levítico 21:7), o
que demonstra que há um castigo separado para cada transgressão; assim
como ele está sujeito ao acoitamento por causa de apenas uma mulher que
for divorciada, visto que esta proibição foi especificamente mencionada,
assim também ele está sujeito ao açoitamento por apenas uma que for
``zoná'', e por apenas uma que for ``halalá''.


Esse é o significado das palavras da Guemara de Kidushin ``Assim como
uma mulher divorciada é diferente de uma ``zoná'' e de uma ``halalá'', em
relação a um `Cohen' comum, ela também é diferente em relação ao `Cohen Gadol'''.

Também está explicado ali que caso haja várias mulheres envolvidas há
uma penalidade de açoitamento por cada violação separadamente, sem
levar em conta se foram na ordem acima mencionada ou
não.\footnote{Sem levar em consideração se as desqualificaçõeS ocorreram na ordem
mencionada.}

Ficou, assim, claro que a proibição de chegar-se a cada uma delas
constitui um preceito negativo separado, e consequentemente, sua
violação acarreta uma pena de açoitamento por cada uma delas,
separadamente.

Também está explicado ali que um ``Cohen'' comum não está sujeito ao
açoitamento por cada uma das violações a menos que ele tenha se casado
com a mulher e se tenha, por conseguinte, chegado a ela. Isto está
exposto no Talmud da seguinte forma: ``Se ele se chegar a ela, ele está
sujeito ao açoitamento; se ele não se chegar a ela, ele não o estará,
pois as Escrituras dizem: `A estas não tomará\ldots{} e não profanará\ldots{}'
(Levítico 2 1 :1 4-1 5). que significam: Por que ele não as tomará? Para
que possa não profanar''.

As normas destes quatro preceitos\footnote{I.e., preceitos negativos 158 a 161.} estão explicadas
por completo em Yebamot e Kidushin.

\paragraph{Um ``Cohen Gadol'' não pode chegar-se a uma viúva}

Por esta proibição um ``Cohen Gadol'' fica proibido de chegar-se a uma
viúva, mesmo sem casar-se com ela. Ela está expressa em Suas palavras,
enaltecido seja Ele, ``E não profanará sua semente entre seu povo''
(Levítico 21:15). Esta é a explicação para isso. Um ``Cohen'' comum é
proibido de casar-se \footnote{Com uma ``zoná'', uma "halalá" ou uma mulher divorciada.} através da proibição ``Não
tomarão'' (Ibid., 7), sendo que ``tomarão'' significa tomar por esposa,
mas ele não fica sujeito à pena de açoitamento até que se chegue a ela,
como explicamos antes. Se ele se chegar a ela fora do casamento ---
embora ele seja proibido de fazê-lo e seja avisado para não fazê-lo,
pois além de tudo ele ainda prejudica a categoria sacerdotal dela ---
ele também não fica sujeito ao açoitamento, pois isso não está
explicitamente proibido. Contudo, no caso do ``Cohen Gadol'' as
Escrituras mencionam duas proibições: a primeira é ``A estas não
tomará'' (Ibid., 14), e a segunda é ``E não profanará sua semente'', a
qual o proíbe de chegar-se a ela até mesmo fora do casamento.

A Guemará de Kidushin diz: ``Rabá admite, com relação a um `Cohen
Gadol' com uma viúva, que se ele viver com ela sem casar-se, está
sujeito ao açoitamento, pois o Todo Misericordioso diz: `Não profanará
sua semente', e ele a terá profanado''. Também está dito, no mesmo
lugar: ``Se um `Cohen Gadol'\footnote{Se ele viver com uma viúva.} com uma viúva ele
estará sujeito ao açoitamento duas vezes, uma por causa de `A estas não
tomará' e outra por causa de `Não profanará'''. O motivo pelo qual isto
está mencionado no caso de uma viúva é que ela foi especificamente
proibida apenas para um ``Cohen Gadol'', sendo permitida aos
``Cohanim'', mas se ele se chegar a ela, ele a profana e faz com que
ela não possa mais se casar com um ``Cohen'' comum. Mas com relação às
três mulheres --- a saber, a mulher divorciada, a ``zoná'', e a ``halalá'' --- a lei é a mesma que no caso do ``Cohen'' comum, ou seja, que cada uma delas está desde o
início proibida a todos os ``Cohanim'' e a única razão para a repetição
da proibição no caso do ``Cohen Gadol'' é a que foi explicada.

\paragraph{Um ``Cohen'' não pode entrar no Santuário com o cabelo solto}

Por esta proibição os ``Cohanim'' ficam proibidos de entrar no Santuário
com o cabelo solto tal como os que estão de luto, que não aparam nem
arrumam seus cabelos. Ela está expressa em Suas palavras, enaltecido
seja Ele, ``Não deixeis o cabelo de vossas cabeças solto'' (Levítico
10:6), que o Targum traduz por: ``Não deixeis crescer vosso cabelo'', e
que Hezekiel explica dizendo: ``Nem permitirão que suas madeixas
cresçam''\footnote{Ezeq. 44:20.}. A mesma expressão é usada com relação
aos leprosos: ``O cabelo de sua cabeça deixará solto'' (Ibid., 13:45), a
respeito da qual a Sifrá diz: ``Ele deverá deixar seu cabelo crescer''.
A Sifrá diz ainda: ```Não deixeis o cabelo de vossas cabeças solto': não
deixeis que ele cresça''.

Com relação ao ``Cohen Gadol'', a proibição está repetida em Suas
palavras ``Seu cabelo não deixará crescer'' (Ibid., 2 1:1 0). O objetivo
da repetição é para que você não pense que Suas palavras a Elazar e a
Itamar ``Não deixeis o cabelo de vossas cabeças solto'' foram ditas
apenas por causa do falecimento\footnote{Pela morte de seus irmãos Nadab e Abihu. Vide Levítico 10:1-7.}, e que um
``Cohen'' pode fazê-lo se não for em sinal de luto. Por isso Ele deixa
claro que no caso do ``Cohen Gadol'' ele deve usar seus cabelos curtos
em virtude de suas funções sacerdotais.

Desobedecer a esta proibição oficiando com o cabelo não cortado é punido
com a morte\footnote{A morte pela mão dos Céus.}. Deduzimos que os de cabelo não
cortado estão incluídos entre os que estão sujeitos à morte através de
Suas palavras: ``Para que não morrais'' (Ibid. 10:6). Mas aquele que
entrar no Santuário com o cabelo comprido e não oficiar está sujeito
apenas ao açoitamento, e não à morte.

\paragraph{Os ``Cohanim'' não podem usar vestes rasgadas ao entrar no Santuário}

Por esta proibição os ``Cohanim'' ficam proibidos de entrar. no
Santuário usando vestes rasgadas. Ela está expressa em Suas palavras ``E
vossos vestidos não rompais'' (Levítico 10:6), sobre as quais a Sifrá
diz: ```E vossos vestidos não rompais': não rasgem suas vestes''. Esta
proibição também está repetida com relação ao ``Cohen Gadol'' em Suas
palavras ``E suas vestes não romperá'' (Ibid., 21:10).

Você deve saber que não é permitido ao ``Cohen Gadol'' rasgar suas
vestes por um morto, mesmo que ele não esteja oficiando; e é por causa
dessa restrição adicional que a proibição é repetida. A Sifrá diz:
```Seu cabelo não deixará crescer e suas vestes não romperá' (Ibid.): nem mesmo por um
parente falecido, da maneira como as pessoas comuns rasgam e deixam seus
cabelos crescerem, quando estão de luto. De que maneira? O ``Cohen
Gadol'' rasga suas vestes de baixo\footnote{Pois o rasgão nesse local não é tão feio.}, mas as pessoas
comuns rasgam as de cima''.

Quem oficiar usando vestes rasgadas também está sujeito à morte, porque
à mesma regra se aplica ao caso do cabelo comprido e ao das vestes
rasgadas. Mas quem entrar no Santuário dessa maneira transgride uma
proibição. Apenas um ``Cohen Gadol'' está permanentemente proibido de
deixar crescer seus cabelos e de rasgar suas vestes, mesmo que ele não
entre no Santuário, e essa é a diferença em relação ao ``Cohen'' comum.

\paragraph{Os ``Cohanim'' não podem sair do Santuário enquanto estiverem oficiando}

Por esta proibição os ``Cohanim'' ficam proibidos de sair do Santuário
enquanto estiverem oficiando. Ela está expressa em Suas palavras ``E da
entrada da tenda da assinação não saireis'' (Levítico 10:7). Esta
proibição também está repetida com relação ao ``Cohen Gadol'', através
de Suas palavras ``Do Santuário não sairá'' (Ibid., 21:12).

A Sifrá diz: ```Da entrada da tenda da assinação': eu poderia pensar que
um `Cohen' comum não pode sair do Santuário, quer ele esteja oficiando
quer não; por isso as Escrituras dizem: `Do santuário não sairá e não
profanará'. Isso mostra que é enquanto ele estiver oficiando. `Porque o
azeite da unção do Eterno está sobre vós' (Ibid., 10:7): isto me diz que
apenas Aarão e seus filhos, que foram ungidos com o Azeite da Unção,
estão sujeitos à morte se eles saírem enquanto estiverem oficiando; como
fico sabendo quanto a todos os `Cohanim', por todos os tempos? Porque as
Escrituras dizem: `Porque o azeite da unção do Eterno está sobre
vós'.

Você deve saber que no caso do ``Cohen Gadol'' há uma proibição
adicional, a saber, que ele não pode acompanhar o esquife de um parente
próximo. Este é o significado literal das palavras das Escrituras ``E do
santuário não sairá'', e é dessa forma que o texto está explicado no
segundo capítulo de Sanhedrin, onde ele está citado como prova de que
se um parente próximo do ``Cohen Gadol'' morrer, ele não poderá
acompanhar seu ataúde. O mesmo texto ensina ainda que o ``Cohen Gadol''
pode oficiar no dia do falecimento de um parente. As palavras dos Sábios
em Sanhedrin são as seguintes: ```E do santuário não sairá, e não
profanará', mas qualquer outro `Cohen' que não sair do Santuário o
estará profanando'' --- referindo-se a um ``Cohen'' comum a quem não é
permitido oficiar porque nesse caso ele é um ``onen''. A advertência
contra oficiar durante o luto se deriva do texto em questão. Esta regra
de que, ao contrário de um ``Cohen Gadol'', um ``Cohen'' comum não pode
oficiar enquanto ele for um ``onen'' está explicada no final de
Horayot.

Dessa forma, foi deixado claro que Suas palavras ``E não profanará''
são uma negativa, e não uma proibição. Isto é, é simplesmente para
declarar que ele não estará profanando o Santuário se oficiar, embora sendo um
``onen''.
Consequentemente, o significado literal do versículo é que Suas palavras
``E não profanará'' são a razão para a proibição anterior, da seguinte
forma: ``E do santuário não sairá'', a fim de ``não profanar o santuário''.
Mas nas duas teorias segue-se que esta proibição não deve ser contada
como um preceito negativo separado, como ficará claro para todo aquele
que compreendeu os Fundamentos\footnote{Ver o oitavo Fundamento.} preestabelecidos
para a execução deste trabalho.

Também foi deixado claro que estas três proibições --- a saber, ``Seu
cabelo não deixará crescer e suas vestes não romperá'' (Ibid., 21:10) e
``E do santuário não sairá'' (Ibid., 1 2) --- estão repetidas com
relação ao ``Cohen Gadol'' com uma finalidade específica, assim como as
proibições que o impedem de chegar-se a uma mulher divorciada, a uma
``zoná'' ou a uma ``halalá''. Ficou também claro que o conteúdo dessas
três proibições é idêntico ao que foi proibido por Suas palavras ``Não
deixeis o cabelo de vossas cabeças solto, e vossos vestidos não
rompais\ldots{} e da entrada da tenda de assinação não saireis'' (Ibid.
10:6-7). E também que nosso Mestre Moisés, a paz esteja com ele, as deu
a conhecer a Elazar e a Itamar dizendo: ``As coisas que lhes são
proibidas não se tornam permitidas por causa de seu luto pela grande
perda; vocês ainda continuam proibidos de deixar crescer seus cabelos,
de rasgar suas vestes, e de deixar o Santuário enquanto estiverem
oficiando''. A razão pela qual a proibição está repetida com relação ao
``Cohen Gadol'' é para explicar que ela se aplica apenas enquanto durar
o ofício e que só nesse caso eles estarão sujeitos à morte, como se
pode ver pelo fato de que para explicar Suas palavras ``Da entrada da
tenda de assinação não saireis'' os Sábios citaram Suas palavras ``E do
santuário não sairá''. E embora cada proibição, repetida com relação ao
``Cohen Gadol'', amplie assim o seu alcance, como explicamos, fica claro
para todos aqueles que entenderam nossa Introdução que essas proibições
não são preceitos adicionais, uma vez que o objetivo das Escrituras é
de não permitir-lhe fazer qualquer uma dessas coisas enquanto estiver
comprometido com os ofícios. Você deve compreender isto.

\paragraph{Um ``Cohen'' comum não pode tornar-se impuro por nenhuma pessoa
morta a não ser pelas que estão determinadas na Torah}

Por esta proibição um ``Cohen'' comum fica proibido de tornar-se impuro
por qualquer pessoa morta a não ser pelos parentes especificados na
Torah. Ela está expressa em Suas palavras, enaltecido seja Ele, ``O
`Cohen', por um morto entre seu povo, não se faça impuro'' (Levítico
21:1).

Aquele que infringir esta proibição, fazendo-se impuro por qualquer
pessoa morta que não esteja entre os cinco parentes pelos quais ele deve
guardar luto\footnote{Ver o preceito positivo 37.} estará sujeito ao açoitamento. A
proibição não se aplica às mulheres. A Tradição diz: ```Filhos de
Aarão' (Ibid.), mas não as filhas de Aarão''.


\paragraph{Um ``Cohen Gadol'' não deve ficar sob o mesmo teto que um cadáver}

Por esta proibição um ``Cohen Gadol'' fica proibido de permanecer sob o
mesmo teto que um cadáver, mesmo que seja de um dos
obrigatórios\footnote{I.e., um dos parentes cujo funeral ele seria normalmente obrigado a acompanhar.}, ou seja, seus parentes. Esta
proibição está expressa em Suas palavras ``E às almas mortas não se
chegará'' (Levítico 2 1:1 1).

Aquele que se fizer impuro dessa forma, mesmo que seja pelo cadáver de
seu pai ou de sua mãe, será punido com o açoitamento.

\paragraph{Um ``Cohen Gadol'' não pode fazer-se impuro por nenhuma pessoa morta}

Por esta proibição o ``Cohen Gadol'' fica proibido de fazer-se impuro
por qualquer pessoa morta, seja de que forma for, quer seja por contato
ou por carregar seu corpo. Ela está expressa em Suas palavras ``Por seu
pai, e por sua mãe, não se fará impuro'' (Levítico 2 1:1 1).

Possivelmente você poderá pensar que esta proibição e a precedente são
uma só e que Suas palavras ``Por seu pai, e por sua mãe, não se fará
impuro'' são apenas uma explicação. Mas não é esse o caso: elas são
duas proibições separadas. Como diz a Sifrá: ``Ele será considerado
culpado por `E às almas mortas não se chegará' e também por `Por seu
pai\ldots{}'''. E um ``guezerá shavá''\footnote{Uma ``expressão similar'', ou seja, uma analogia entre duas leis,
  estabelecida com base na congruência verbal dos textos das
  Escrituras.} faz ainda com
que isto também se aplique a um ``Cohen'' comum: ``Assim como o `Cohen
Gadol' que está proibido por qualquer pessoa morta é culpado duas vezes,
assim também o `Cohen' comum, que está proibido de fazer-se impuro por
qualquer pessoa morta está sujeito ao castigo pela violação de `E às
almas mortas\ldots{}'''. Contudo, não consideramos que seja assim no caso do
``Cohen'' comum\footnote{Não consideramos que ele tenha transgredido dois preceitos
  diferentes.} pelo motivo explicado no Segundo
Fundamento mas\footnote{No caso do ``Cohen Gadol''.} contamos essas duas proibições
como preceitos diferentes porque elas estão expressas em dois
versículos diferentes e porque o significado de ``E às almas
mortas\ldots{}'' não é o mesmo que o de ``Por seu pai\ldots{}'', pois aqueles que
transcreveram a Tradição disseram: ``Ele está sujeito a castigo por `E
às almas mortas\ldots{}' e também por `Por seu pai\ldots{}'''.

\paragraph{Os Levitas não podem adquirir um pedaço da Terra de Israel}

Por esta proibição toda a tribo de Levi está proibida de tomar um pedaço
da terra de Israel. Ela está expressa em Suas palavras ``Os sacerdotes-levitas, com ou sem defeitos, de toda a tribo de Levi, não terão parte
nem herança com Israel'' (Deuteronômio 18:1).

\paragraph{Os Levitas não podem receber nenhuma parte da pilhagem da conquista
da Terra de Israel}

Por esta proibição toda a tribo de Levi fica também proibida de receber
uma parte da pilhagem obtida na conquista da terra de Israel. Ela está
expressa em Suas palavras ``Os sacerdotes-levitas, com ou sem defeitos,
de toda a tribo de Levi, não terão parte nem herança com Israel''
(Deuteronômio 18:1), a respeito das quais o Sifrei diz: ```Parte' da
pilhagem; `herança' da terra''.

Você poderia talvez objetar e perguntar-me: ``Por que você considera
essas duas proibições --- a de pegar uma parte da terra e a de pegar uma
parte da pilhagem --- como dois preceitos separados? Com certeza essa é
uma ``lav shebikhlalut''\footnote{Uma proibição negativa geral.} e como tal ela é
considerada como um único preceito, de acordo com o princípio que você
estabeleceu''.

A resposta é que esta proibição está de fato dividida em duas pelas
palavras ``Mas herança não terão'' (Ibid., 2), havendo assim duas
proibições expressas de maneira diferente: a primeira, que os proíbe de
pegar algo da pilhagem, e ``Os sacerdotes-levitas\ldots{} não terão parte'';
e a segunda, que os proíbe de ter uma parte da Terra, e ``Herança não
terão''.

Esta proibição dupla aparece novamente com relação aos ``Cohanim'' em
Suas palavras, enaltecido seja Ele, a Aarão ``Mas herança não terão em
suas terras, nem terão parte no meio de seus irmãos'' (Números 18:20)
que são interpretadas da seguinte forma: ```Herança não terão em suas
terras': isto é, quando eles dividirem as terras; `Nem terão parte no
meio de seus irmãos': isto é, parte da pilhagem''.

Caso você pense que estas duas proibições, relativas especificamente
aos ``Cohanim'', constituem dois preceitos que deveriam ser
acrescentados, você deve saber que uma vez que esta proibição está
expressa em termos genéricos, referindo-se a ``toda a tribo de Levi'',
os ``Cohanim'' já estão incluídos nela, e que ela foi repetida com
relação aos ``Cohanim'' apenas para dar maior ênfase. Nos casos como
este, em que uma lei de aplicação geral está repetida fazendo referência
especificamente a um determinado caso, o objetivo da repetição é apenas
o de enfatizar e complementar a lei que pode não ter sido formulada por
completo na primeira proibição.

Se contássemos Suas palavras a Aarão ``Mas herança não terão em suas
terras, nem terão parte no meio de seus irmãos'', em adição a Suas
palavras ``Os Sacerdotes-levitas\ldots{} não terão parte\ldots{}'',, nós
obrigatoriamente deveríamos ter contado, por analogia, as proibições
que impedem o ``Cohen Gadol'' de chegar-se a uma mulher divorciada, a
uma ``halalá'' e a uma ``zoná'', como três preceitos adicionais, além
daqueles que se aplicam da mesma forma ao ``Cohen Gadol'' --- e aos
``Cohanim'' comuns.

Entretanto, caso alguém insista em que realmente deveríamos ter feito
isso, nós lhe responderemos o seguinte: um ``Cohen
Gadol'' que se tivesse chegado a uma mulher divorciada estaria
forçosamente sujeito a dois castigos, um por ser ele um ``Cohen'', a
quem uma mulher divorciada está proibida, e outro por ser um ``Cohen
Gadol'', a quem ela está igualmente proibida sob os termos de outro
preceito. Mas está explicado no Tratado Kidushin que ele está sujeito a
castigo uma única vez; consequentemente, apenas a proibição geral deve
ser contada e o objetivo de qualquer outra proibição, menos genérica, do
mesmo tipo de conduta serve apenas para ensinar-nos alguma norma
específica ou para complementar o enunciado da lei, como explicamos ao
tratar do preceito negativo 161.

A essa categoria de leis pertence também a proibição que ordena aos
``Cohanim'': ``Não farão calva em sua cabeça, nem a sua barba rasparão e
em sua carne não farão incisões'' (Levítico 21:5). Essas três proibições
já foram impostas a todo o povo de Israel através de Suas palavras
``Não cortareis o cabelo de vossa cabeça em redondo'' (Ibid., 19:27);
``Não vos fareis raspar a cabeça'' (Deuteronômio 14:1); ``E incisões por
um morto não fareis em vossa carne'' (Levítico 19:28). Elas estão
repetidas com relação aos ``Cohanim'' apenas com a finalidade de
complementar o enunciado da lei, como foi deixado claro no final de
Macot, onde as normas desses três preceitos estão explicadas. Se elas
fossem preceitos separados a serem aplicados aos ``Cohanim'' e não
simplesmente complementos do enunciado da lei, um ``Cohen'' estaria
sujeito ao açoitamento duas vezes por cada uma das transgressões, uma
vez por ser israelita, e outra por ser ``Cohen''. Mas esse não é o caso.
Na realidade, ele está sujeito a um único açoitamento, como o resto de
Israel, como está explicado no lugar apropriado.

Este princípio deve ser compreendido na sua totalidade.

\paragraph{Não arrancar nosso cabelo pelos mortos}

Por esta proibição somos proibidos de arrancar nossos cabelos por causa
dos mortos, como fazem os loucos. Ela está expressa em Suas palavras
``Não vos fareis raspar a cabeça entre os olhos por causa de um morto''
(Deuteronômio 14:1).

Esta proibição está repetida, com relação aos ``Cohanim'', em Suas
palavras ``Não farão calva em sua cabeça'' (Levítico 21:5), a fim de
complementar o enunciado da lei. Poderíamos argumentar, em virtude das
palavras ``entre os olhos'', que a proibição se aplica apenas a raspar a
testa, por isso Ele diz: ``Não farão calva em sua cabeça'', mostrando
assim que a proibição se aplica a toda a cabeça assim como à testa. E se
Ele tivesse dito apenas ``Não farão calva em sua cabeça'' poderíamos
argumentar que a proibição se aplica tanto em relação aos mortos como
aos outros, por isso Ele explica dizendo ``por causa de um morto''.

Todo aquele que deixar em sua cabeça um buraco de calvície do tamanho de
um grão de feijão, por ter arrancado seus cabelos por causa de um morto,
seja ele ``Cohen'' ou israelita, será punido com o açoitamento, uma vez
por cada buraco de calvície.

Aqui novamente o objetivo da repetição com relação aos ``Cohanim'', nas
palavras ``Nem a sua barba rasparão e em sua carne não farão incisões''
(Levítico 21:5), é para complementar a lei do preceito, como está
explicado no final de Macot.

\paragraph{Não comer um animal impuro}

Por esta proibição somos proibidos de comer um animal impuro, doméstico
ou selvagem. Ela está expressa em Suas palavras ``Estes não comereis
dos que ruminam\ldots{}: o camelo, e a lebre, e o coelho\ldots{} e o porco''
(Deuteronômio 14:7-8).

As Escrituras não proibiram explicitamente que se comam outros animais
impuros, mas pelas Suas palavras ``E todo animal de casco fendido, e que
tem a unha separada em dois de cima até embaixo, e que rumina, entre os
animais, esses comereis'' (Ibid., 14:6)\footnote{Ver o preceito positivo 149.} sabemos
que todos os que não possuam essas duas características não são alimento
permitido. Contudo, este é o caso de um preceito negativo derivado de um
preceito positivo o qual, de acordo com o que foi estabelecido, possui a
força de um preceito positivo, e é um princípio aceito que a infração de
um preceito negativo deste tipo não é punida com o açoitamento. Mas
através de um ``kal vahomer''\footnote{``Com toda razão''.} deduzimos que
estamos proibidos de comer outros animais impuros ou selvagens e que
estamos sujeitos ao açoitamento por comê-los; portanto, se comer um
porco ou um camelo, que tem uma das duas características dos animais
puros, é punível com o açoitamento, conclui-se que comer outros animais
domésticos ou selvagens que não tenham nenhuma dessas características é
punível com o açoitamento.

Agora observem as palavras da Sifrá a este respeito: ```Esses comereis'
--- `esses' podem ser comidos, mas os animais impuros não podem ser
comidos. Isso me apresenta apenas um preceito positivo; de que forma
fico sabendo que também há um preceito negativo? Pelas palavras das
Escrituras: `Mas estes não comereis dos que ruminam' (Levítico 11:4-7).
Isso me diz apenas que `esses' são alimentos proibidos; como fico
sabendo com relação a todos os outros animais impuros? Eu o concluo por
analogia: se esses, que têm uma das características do animal puro, são
alimento proibido, não é lógico pensar que um preceito negativo nos
proíbe de comer outros animais impuros que não tenham nenhuma dessas
características? Dessa forma fica estabelecido que o camelo, a lebre, o
coelho e o porco são\footnote{I.e., são proibidos explicitamente pelas Escrituras.} pelas Escrituras, enquanto
que outros animais impuros o são por força de um `kal vahomer'. Fica
também estabelecido que o preceito positivo vem das Escrituras enquanto
que o preceito negativo relativo a eles é derivado de um `kal
vahomer'''.

Entretanto, esse ``kal vahomer'' é meramente utilizado para estabelecer
algo que as Escrituras deixaram claro, como no caso da
filha\footnote{Ver o preceito negativo 336.}, que explicaremos no local apropriado.
Portanto, aquele que comer o equivalente ao tamanho de uma oliva de
carne de um animal impuro doméstico ou selvagem, estará sujeito ao
açoitamento, de acordo com a Torah. Você deve compreender isso.


\paragraph{Não comer um peixe impuro}

Por esta proibição somos proibidos de comer um peixe impuro. Ela está
expressa em Suas palavras, relativas a esses tipos de peixe, ``E
abominação serão para vós; de sua carne não comereis'' (Levítico 11:11).

Comer o equivalente ao tamanho de uma oliva de sua carne é punido com o
açoitamento.

\paragraph{Não comer nenhuma ave impura}

Por esta proibição somos proibidos de comer uma ave impura. Ela está
expressa em Suas palavras, relativas a essas espécies, ``E isto
abominareis das aves: não serão comidas'' (Levítico 11:13).

Comer o equivalente ao tamanho de uma oliva de sua carne é punido com o
açoitamento.

As normas deste e dos dois preceitos anteriores estão explicadas no
terceiro capítulo de Hulin.

\paragraph{Não comer nenhum inseto alado}

Por esta proibição somos proibidos de comer qualquer inseto alado, tal
como moscas, abelhas, vespas e insetos similares. Ela está expressa em
Suas palavras no Deuteronômio ``E todo animal rastejante alado, impuro
será para vós: não serão comidos'' (Deuteronômio 14:19), sobre as quais
o Sifrei diz: ```Todo animal rastejante alado etc.' é um preceito
negativo''.

Comê-los será punido com o açoitamento.

\paragraph{Não comer nada que rasteje sobre a terra}

Por esta proibição somos proibidos de comer qualquer coisa que rasteje
sobre a terra, tal como vermes, escaravelhos e coleópteros, que são
chamados ``coisas que rastejam sobre a terra''. Ela está expressa em
Suas palavras, enaltecido seja Ele, ``E todo animal rastejante que se
move sobre a terra é abominação, não será comido'' (Levítico 11:41).

Comer qualquer uma dessas coisas será punido com o açoitamento.

\paragraph{Não comer nenhuma criatura rastejante que se reproduza em matéria
deteriorada}

Por esta proibição somos proibidos de comer qualquer criatura
rastejante que se reproduza em matéria deteriorada ou desfeita, mesmo
que ela não pertença a nenhuma espécie característica e não seja
resultante da união de um macho e de uma fêmea. Esta proibição está
expressa em Suas palavras
``E não façais impuras vossas almas com todo animal rastejante que se
arrasta sobre a terra'' (Levítico 11:44), sobre as quais a Sifrá diz:
```Todo animal rastejante que se arrasta sobre a terra' --- ainda que
ele não se reproduza''.

Esta é a diferença entre Suas expressões ``Todo animal rastejante que
\emph{se move} sobre a terra'' (Ibid., 41) e ``Todo animal rastejante
que \emph{se arrasta} sobre a terra''. A criatura que \emph{se move} tem
o poder de gerar seres semelhantes e se multiplica sobre a terra,
enquanto que as criaturas que \emph{se arrastam} se reproduzem em
matéria deteriorada ou desfeita e não têm o poder de gerar seres como
elas.

Também neste caso comê-las é punido com o açoitamento.

\paragraph{Não comer criaturas vivas que se reproduzam em sementes ou frutas}

Por esta proibição somos proibidos de comer criaturas vivas que se
reproduzam em sementes ou frutas uma vez que elas tenham saído da
semente ou da fruta e se movam sobre elas; mesmo se as encontrarmos
depois em nossa comida somos proibidos de comê-las e aquele que o fizer
será punido com o açoitamento. Esta proibição está expressa em Suas
palavras ``Todo animal rastejante que se move sobre a terra, não os
comereis porque abominação são eles'' (Levítico 11:42), sobre as quais a
Sifrá diz: ``Isto inclui também os que estiveram sobre a terra e que
tornaram a entrar''\footnote{Tornaram a entrar na semente ou na fruta.}.

\paragraph{Não comer nenhuma espécie de criatura rastejante}

Por esta proibição somos proibidos de comer qualquer espécie de
criatura rastejante, seja ela um ser alado, ou rasteje na água ou sobre
a terra.
Esta imbibição está expressa em Suas palavras, enaltecido seja Ele,
``Não torneis abomináveis vossas almas com nenhum animal rastejante que se move e
não vos façais impuros com eles e não sejais impuros por eles'' (Levítico
11:43). Esta é uma proibição separada, cuja transgressão é punida com o
açoitamento e ela é semelhante a um ``issur colel''. De acordo com isso,
aquele que comer qualquer tipo de animal que se arrasta sobre a terra é punido com
o açoitamento duas vezes: uma por causa da proibição ``E todo animal
rastejante que se move sobre a terra é abominação, não será comido'' (Ibid., 41) e
outra por causa da proibição ``Não torneis abomináveis vossas almas com
nenhum animal rastejante que se move''. Da mesma forma, aquele que comer
qualquer tipo de animal rastejante alado será punido com o açoitamento duas
vezes: uma por causa da proibição ``E todo animal rastejante alado, impuro será
para vós: não serão comidos'' (Deuteronômio 14:19), e outra por causa da proibição
``Não torneis abomináveis vossas almas etc.''. Além disso, aquele que comer um
inseto que tenha asas e que também se arraste pelo chão e que, dessa
forma, seja ao mesmo tempo um animal rastejante alado e um animal que se move pelo\\
chão, será punido com o açoitamento quatro vezes: Se, além disso, o
inseto também se arrastar na água, comê-lo será punido seis vezes: a quinta vez
por ser um peixe impuro, a respeito do qual está dito: ``De sua carne
não comereis'' (Levítico 11:11) e a sexta por causa de ``Não torneis
abomináveis vossas almas etc.''. Esta última abrange também tudo o que
se arrasta dentro da água, uma vez que não temos nenhum versículo
proibindo que se coma animais que rastejem na água a não ser o que diz:
``Não torneis abomináveis vossas almas com nenhum animal rastejante que
se move''. De acordo com esses princípios, a Guemará de Macot diz: ``Se
alguém comer uma enguia ele está sujeito ao açoitamento por quatro
vezes; se comer uma formiga, por cinco vezes; se comer um marimbondo,
por seis vezes''.

Entretanto, todas as vezes que ouvi alguém interpretar esse trecho, ou
seja, ``Se alguém comer uma enguia etc.'' e que li os livros que
consultei a esse respeito, a interpretação foi invariavelmente a acima
mencionada. E ela é uma interpretação incorreta, sustentável apenas se
os princípios verdadeiros claramente estabelecidos no Talmud fossem
derrubados. Porque se você examinar o que dissemos acima, você vai
chegar à conclusão de que, baseado no preceito negativo ``Não torneis
abomináveis vossas almas com nenhum animal rastejante que se move'',
eles determinam o açoitamento por três vezes; mas está demonstrado que
esse ponto de vista está errado, pois em circunstância alguma alguém
pode ser punido duas vezes por causa de uma proibição, como está
explicado na Guemará de Hulin e como nós próprios já explicamos em
diversas ocasiões e como ilustraremos com exemplos a seguir.

A verdadeira interpretação, que não admite dúvidas e que não vai
enganá-lo é a seguinte. Aquele que comer uma criatura alada que rasteja
tanto na água quanto sobre a terra estará sujeito apenas a três castigos
por açoitamento: um por ser uma criatura rastejante alada, a cujo
respeito há uma proibição específica; outro por ser uma criatura que
rasteja sobre a terra contra a qual também há uma proibição específica;
e outro por causa de ``Não torneis abomináveis vossas almas'', que
proíbe comer criaturas que rastejam na água, abrangidas pela expressão
``Nenhum animal rastejante que se move'' que está na proibição ``Não
torneis abomináveis vossas almas''. Mas se alguém comer um animal que
se arrasta apenas sobre a terra, ou uma criatura rastejante alada, ou
uma que se arrasta apenas na água, o castigo em cada caso será o
açoitamento pela desobediência referente a cada proibição, sendo que no
último caso a proibição está expressa em ``Não torneis abomináveis
vossas almas com nenhum animal rastejante que se move''. O simples fato
de que esta proibição também abrange a criatura que rasteja sobre a
terra não nos sujeita a dois açoitamentos por uma criatura desse tipo,
pois ainda que houvesse mil proibições expressas relativas às criaturas
que rastejam sobre a terra, ainda assim seríamos açoitados uma única vez
por ter desobedecido a elas, pois todas elas seriam repetições de uma
única coisa. Mesmo se Ele tivesse dito mil vezes: ``Não comereis insetos
que rastejam sobre a terra'', ``Não comerás insetos que se movem sobre a
terra'', estaríamos sujeitos a um único açoitamento. Vocês já viram
aqueles que apresentaram esta doutrina incorreta afirmar que aquele que
usar ``Shaatnez'' (i.e, vestes feitas com lã e linho) estará sujeito a
dois açoitamentos porque isso está expressamente proibido por dois
preceitos negativos? Eu nunca os vi defender tal opinião, ao contrário,
eles achariam estranho se outra pessoa o fizesse. Todavia, eles não
acham estranha sua própria opinião quando dizem que o caso de uma
criatura que rasteja sobre a terra ou de uma criatura rastejante alada é
punível com dois açoitamentos: um pela proibição da criatura comida, e
outro por causa da proibição: ``Não torneis abomináveis vossas almas''!
Isso não passaria desapercebido nem mesmo a um surdo-mudo.

Voltarei agora ao assunto que comecei a explicar.

Se acontecesse que um inseto nascesse numa determinada semente ou fruta
e saísse dela, aquele que o comesse estaria sujeito ao açoitamento uma
vez, mesmo que esse inseto nunca chegasse a tocar o chão, porque há uma
proibição específica com relação a ele, como explicamos ao tratar do
preceito precedente. Se ele caísse no chão e aí se movesse, aquele que
o comesse estaria sujeito a dois açoitamentos, um por causa de ``De todo
animal rastejante que se move sobre a terra, não os comereis porque
abominação são eles'' (Levítico 11:42), e um por causa do ``E não façais
impuras vossas almas com todo animal rastejante que se arrasta sobre a
terra'' (Ibid., 44). Se, além disso, o inseto fosse do tipo que se
feproduz, estaria sujeito a três açoitamentos: dois pelas razões acima
mencionadas e o terceiro por causa de ``E todo animal rastejante que se
move sobre a terra é abominação, não será comido'' (Ibid., 41). Se além
de tudo isso o inseto fosse alado, estaria sujeito a um quarto
açoitamento por causa de ``E todo animal rastejante alado, impuro será
para vós: não serão comidos'' (Deuteronômio 14:19). E caso o inseto,
além de ser alado, ainda se arrastasse na água, como fazem muitas
espécies, ele sujeitaria a cinco açoitamentos, sendo o quinto por causa
de uma criatura que se arrasta na água, que está abrangida na proibição
expressa em Suas palavras ``Não torneis abomináveis vossas almas com
nenhum animal rastejante que se move e não vos façais impuros com eles e
não sejais impuros por eles''. E se, finalmente, ainda por cima a
criatura gerada fosse também uma ave, sujeitaria a seis açoitamentos,
por causa de ``E isto abominareis das aves: não serão comidas, porque
elas são uma abominação'' (Levítico 11:13).

Você não deve ficar surpreso de que um pássaro nasça de uma fruta podre,
já que frequentemente vemos pássaros maiores do que uma pequena noz que
se geraram de alimentos decompostos. Não deve achar estranho que uma
criatura seja ambos uma ave impura e um animal rastejante alado, pois
não é impossível que uma criatura possua ao mesmo tempo as qualidades e
características tanto de uma ave como de um animal rastejante alado.
Dessa forma, você vai encontrar comentaristas anteriores contando entre
esses seis açoitamentos um em virtude de uma criatura que é ao mesmo
tempo um peixe impuro e um animal que se arrasta na água. Isso é
correto e eu não vou contestá-lo porque é possível que uma criatura seja
ambos um peixe e um animal que se arrasta na água, assim como é possível
que uma criatura seja ambos uma ave e um animal que se arrasta na água,
ou uma ave e um animal rastejante alado, de forma que haveria quatro
punições por comê-lo. A enguia é ambos uma ave, um animal rastejante
alado, e uma criatura que se arrasta tanto na terra como na água;
consequentemente ela sujeita a quatro açoitamentos. A formiga
mencionada é um inseto alado, gerado de frutas podres, que não se
reproduz e acarreta um\footnote{Acarreta um castigo.} por ser uma criatura
rastejante que sai do alimento, um por ser uma criatura que rasteja
sobre a terra, um por ser uma criatura que se move sobre a terra, um por
ser uma criatura rastejante alada e um por ser uma criatura que se
arrasta na água. A vespa, que também nasce de matéria em decomposição é,
além disso, uma ave e uma criatura rastejante alada.

Não é impossível que a vespa, a formiga e outros tipos de criaturas
voadoras e rastejantes nasçam de materiais ou frutos em decomposição,
mas as massas, que desconhecem as ciências naturais, não pensam assim.
Como eles veem que a maioria dos seres animados são gerados através da
união de um macho e de uma fêmea, eles imaginam que isso deve ser assim
com relação a todos os seres vivos.

Lembre e compreenda este assunto porque ele é uma palavra dita
apropriadamente''.

Eu expliquei a vocês os princípios através dos quais você pode decidir,
depois de analisá-los, que por comer um inseto a pessoa está sujeita a
um determinado número de açoitamentos, enquanto que por comer um outro
ela está sujeita a um número menor de açoitamentos.

Pelos versículos citados ficará claro para você que aquele que comer um
inseto vivo inteiro de qualquer tamanho\footnote{Estará sujeito a ser castigado.} e não
devemos perguntar se o inseto era do tamanho de uma oliva. Mesmo se
alguém comer um mosquito, ele estará sujeito a três açoitamentos: um
por ser uma criatura que rasteja sobre a terra, um por ser uma criatura
que se move sobre a terra, e um por ser uma criatura rastejante alada.

Também nos dizem: ``Aquele que prender suas fezes peca contra `Não
tornareis abomináveis Vossas almas'. Aquele que beber num copo de chifre
de cirurgião no qual ele recebe o sangue peca contra `Não tornareis
abomináveis vossas almas''' Isto se aplica também a comer imundícies ou
coisas repulsivas ou tomar líquidos pelos quais a maioria das pessoas
tenha aversão; todas essas coisas são proibidas. Entretanto, não se está
sujeito ao açoitamento por essas coisas, uma vez que o significado
literal do texto se refere apenas aos seres rastejantes. Mas elas
acarretam um ``macat mardut''.

Assim, ficou claro, através de toda esta discussão, que o versículo
``Não tornareis abomináveis vossas almas'' é a única fonte de onde
deduzimos a proibição de comer criaturas que se arrastam na água, já que
não há nenhum preceito negativo específico que se refira a isso, a não
ser este. Isto deve ser entendido.

\paragraph{Não comer ``nebelá''}

Por esta proibição somos proibidos de comer um animal que morreu por
si. Esta proibição está expressa em Suas palavras ``Não comereis nenhum
animal que morreu por si (nebelá)'' (Deuteronômio 14:21).

Comer a quantidade correspondente ao tamanho de uma oliva de ``nebelá''
é punido com o açoitamento.

\paragraph{Não comer ``terefá''}

Por esta proibição somos proibidos de comer ``terefá''. Ela está
expressa em Suas palavras, ``Carne dilacerada no campo não comereis''
(Êxodo 22:30).

O significado literal do texto é o que está exposto na Mekhiltá: ``As
Escrituras falam dos casos comuns e mencionam os lugares onde os
animais são provavelmente dilacerados''. Contudo, tradicionalmente o
versículo também é interpretado da seguinte forma: ``Qualquer carne que
estiver no campo é `terefá', portanto não deveis comê-la''. Isso
significa que qualquer carne que tenha sido levada para fora de seus
limites legais se torna ``terefá''; portanto, se a carne dos Sacrifícios
Mais Sagrados for levada para fora do Campo do Santuário, ou a carne
dos Sacrifícios Menos Sagrados para fora dos muros de Jerusalém, ou a
carne do sacrifício de ``Pessah'' para fora da companhia\footnote{O grupo que se reuniu para comer o Sacrifício de Pessah.}, ou se a cria puser sua pata dianteira para
fora\footnote{E depois a recolher novamente, durante o trabalho de parto.}, como está explicado no quarto capítulo de
Hulin, em cada um destes casos a carne é chamada ``terefá'' e aquele que
comer o equivalente ao tamanho de uma oliva dela estará sujeito ao
açoitamento, de acordo com as Escrituras. Da mesma forma, carne
arrancada de um animal vivo é chamada ``terefá'' e aquele que a comer
estará sujeito ao açoitamento. A Guemará de Hulin diz: ```Carne
dilacerada do campo não comereis' se refere à carne de um animal vivo e
também à carne de um animal que foi dilacerada por animais selvagens''.

As proibições deste preceito e do precedente estão repetidas com relação
aos ``Cohanim'' em Suas palavras ``Do animal que morre por si ou
dilacerado por outros animais não comerá para não impurificar-se por
causa deles'' (Levítico 22:8). O motivo pelo qual a proibição está
repetida no caso deles é que, uma vez que as Escrituras lhes ordenam que
comam de um pássaro de Sacrifício de Pecado que
foi\footnote{Que foi abatido.} por ``meliká''\footnote{Ver o preceito negativo 112.} --- um
método que, se fosse usado para abater carne comum certamente não seria
válido, pois ele transforma a carne em ``nebelá'' --- poderia
ocorrer-nos que eles podem comer, como alimento comum, até mesmo
``meliká'' ou um que tenha sido degolado de maneira inadequada; por
isso as Escrituras explicam que eles continuam a ser como os israelitas
no que se refere à advertência contra comer ``nebelá'' ou ``terefá''.
Esta é a explicação dada pelos Sábios, que também mencionam este
versículo com relação a outra lei, que não é relevante neste trabalho.

Mas o animal doméstico ou selvagem que comprovadamente tiver se tornado
``terefá'', de acordo com um dos métodos aceitos de interpretação, é
alimento proibido, mesmo que ele tenha sido abatido segundo os rituais;
e aquele que o abater segundo os rituais e comer sua carne será punido
com o açoitamento, de acordo com a lei Rabínica.

As coisas que transformam em ``terefá'' estão explicadas no terceiro
capítulo de Hulin. As normas deste e dos nove preceitos precedentes
estão explicadas nesse mesmo capítulo, no último capítulo de Macot, e no
primeiro capítulo de Bekhorot.

\paragraph{Não comer um membro de um animal vivo}

Por esta proibição somos proibidos de comer um membro de uma criatura
viva, ou seja, cortar um membro de um animal vivo e comer o equivalente
ao tamanho de uma oliva, na sua condição natural\footnote{I.e., junto com as veias e os tendões (embora em outras proibições as
  veias e os tendões não cheguem ao tamanho de uma oliva).}.
E ainda que não haja mais do que uma porção ínfima de carne nele, aquele
que a comer será punido com o açoitamento. A proibição está expressa em
Suas palavras ``Não comerás enquanto a alma está junto à carne''
(Deuteronômio 12:23), a respeito das quais o Sifrei diz: ```Não comerás
enquanto a alma está junto à carne' se refere a um membro de uma
criatura viva''. O versículo é interpretado da mesma maneira na Guemará
de Hulin, onde lemos também: ``Aquele que comer um membro de uma criatura viva, e também carne de uma criatura
viva, é culpado duas vezes''. A razão disso é que há duas proibições,
das quais a primeira é ``Não comerás enquanto a alma está junto à
carne'', que proíbe comer um membro, e a segunda é ``Carne dilacerada no
campo não comereis'' (Êxodo 22:30), que proíbe comer a carne de uma
criatura viva, como explicamos.

Esta proibição aparece novamente, sob outra forma, em Suas palavras a
Noé, proibindo-o de comer um membro de uma criatura viva: ``Porém, a
carne com sua alma e seu sangue, não comereis'' (Gênesis 9:4).

\paragraph{Não comer ``guid hanashé''}

Por esta proibição somos proibidos de comer os tendões encolhidos. Ela
está expressa em Suas palavras ``Por isso não comem os filhos de Israel
o tendão encolhido'' (Gênesis 32:33). Todo aquele que comer o tendão
todo, ainda que ele seja pequeno, ou o equivalente ao tamanho de uma
oliva, será punido com o açoitamento.

As normas deste preceito estão explicadas no sétimo capítulo de Hulin.

\paragraph{Não comer sangue}

Por esta proibição somos proibidos de comer sangue. Ela está expressa
em Suas palavras ``E sangue não comereis'' (Levítico 7:26). Ela aparece
várias vezes nas Escrituras\footnote{Levítico 3:17 e 17:14.} e está declarado
expressamente que o castigo por sua contravenção voluntária é a
extinção: ``Todo aquele que comer dele será banido'' (Ibid., 17:14).
Aquele que a infringir involuntariamente deverá oferecer um Sacrifício
Determinado de Pecado.

As normas deste preceito estão explicadas no quinto capítulo de Queretot.

\paragraph{Não comer gordura de um animal puro}

Por esta proibição somos proibidos de comer a gordura de um animal
puro. Ela está expressa em Suas palavras ``Todo sebo de boi, e de
carneiro, e de cabra, não comereis'' (Levítico 7:23). Ela aparece mais
uma vez nas Escrituras\footnote{Levítico 3:17.} e a pena de extinção está
explicitamente prescrita em caso de transgressão voluntária. Se alguém a
infringir involuntariamente ele deve oferecer um Sacrifício Determinado
de Pecado.

As normas deste preceito estão explicadas no sétimo capítulo de Hulin.

\paragraph{Não cozinhar carne no leite}

Por esta proibição somos proibidos de cozinhar carne no leite. Ela está
expressa em Suas palavras ``Não cozinharás cabrito com o leite de sua
mãe'' (Êxodo 23:19). Aquele que cozinhar carne no leite será punido com o
açoitamento, mesmo que ele não a coma, como está explicado em vários
trechos do Talmud.

\paragraph{Não comer carne cozida em leite}

Por esta proibição somos proibidos de comer carne em leite. Ela está
expressa na repetição de Suas palavras ``Não cozinharás cabrito com o
leite de sua mãe'' (Êxodo 24:26), cujo objetivo é proibir-nos de comer.
A Guemará de Hulin diz: ``Aquele que cozinhar carne no leite estará
sujeito ao açoitamento, e aquele que comer dela estará sujeito ao
açoitamento.''. E a Guemará de Macot diz: ``Aquele que, num dia de festa
cozer o tendão da coxa no leite e o comer estará sujeito a cinco
açoitamentos: um por comer o tendão, um por
cozinhar\footnote{Por cozinhar em dia de festa, sem necessidade.}, um por cozer carne no leite, um por
comer carne no leite, e um por acender o fogo''. Diz ainda: ``Troque
`acender o fogo' por `lenha que pertence ao Santuário'; quanto à
proibição necessária, ela está expressa no texto `Suas árvores
idolatradas queimareis no fogo\ldots{} Não procedereis de modo semelhante
para com o Eterno, vosso Deus''' (Deuteronômio 12:3-4).

A Guemará de Hulin diz: ``O Misericordioso expressa a proibição de comer
pelo termo `cozinhar' porque assim como se fica sujeito ao açoitamento
por cozinhar, também se fica sujeito ao açoitamento por comer''. E no
segundo capítulo de Pessahim lemos o seguinte, com relação à carne no
leite: ``Comer não está mencionado especificamente para mostrar-nos que
podemos ficar sujeitos ao açoitamento por causa de comida até mesmo se
não a usarmos da maneira convencional''. Isto deve ser lembrado.

Neste ponto é conveniente que eu chame atenção para um princípio
importante, que não mencionei até agora.

Suas palavras ``Não cozinharás cabrito com leite de sua mãe'' aparecem
três vezes na Torah\footnote{Êxodo 23:19; Ibid., 34:19; Deuteronômio 14:21.}, e de acordo com aqueles que
transcrevem a Tradição, cada uma dessas proibições tem um objetivo
diferente. ``Uma'', dizem eles, ``proíbe comê-la, outra proíbe tirar
algum proveito disso, e a outra proíbe de cozinhá-la''.

Alguém poderia argumentar o seguinte: ``Por que você conta a proibição
de comer e a proibição de cozinhar como dois preceitos, e não conta a
proibição de tirar proveito disso como um terceiro preceito?'' Essa
pessoa deve ser informada que a proibição de tirar proveito disso não
pode ser contada propriamente como um preceito separado porque essa e a
proibição de comer são da mesma natureza, já que comer é uma forma de
tirar proveito. Toda vez que ele disser, com relação a uma determinada
coisa, ``Isso não deve ser comido'', comer serve apenas como exemplo de
tirar proveito, e a intenção é de proibir-nos de tirar algum proveito da
coisa, seja comendo ou de qualquer outra forma. Isto está expresso pelos
Sábios da seguinte forma: ``Toda vez que as Escrituras disserem `Isto
não deve ser comido, Não deveis comer, Não deves comer', compreende-se a
proibição de comer e de tirar proveito, a menos que as Escrituras
declarem expressamente que não é assim, como no caso do `nebelá', cujo
uso Ele permite explicitamente ao dizer: ``Ao peregrino incircunciso que está em tuas cidades o darás, e o comerá'' (Deuteronômio 14:21).

De acordo com este princípio, não é correto contar-se a proibição de
comer e de tirar proveito dela como dois preceitos; e se contássemos
dois preceitos no caso da carne cozida com leite, deveríamos ter feito a
mesma coisa nos casos ``hametz'' \footnote{Alimento que contém fermento.}, de
``orlá''\footnote{Nome que se dá ao fruto de uma árvore antes da mesma completar três
  anos.}, e de ``quil-ei ha
querem''\footnote{Um vinhedo onde se colocou, juntamente com as sementes de uvas,
  outras espécies de sementes, como a de trigo ou as de vegetais.}, contando a proibição de tirar proveito
como um preceito independente em si, em cada um desses quatro casos.
Entretanto, como neles contamos apenas a proibição de comer, porque ela
inclui a proibição de tirar proveito, como explicamos, fazemos o mesmo
no caso da carne no leite.

Resta apenas uma pergunta. Pode ser perguntado por que as Escrituras
tiveram que mencionar a terceira proibição no caso de carne no leite, a
fim de proibir-nos de tirar proveito disso, como explicamos, se a
proibição de tirar proveito se deduz, como foi explicado, da proibição
de comer. A resposta é que, na realidade, as Escrituras não dizem, com
referência à carne no leite, ``Não deveis \emph{comê-la}'', o que
proibiria ambos de comer e tirar proveito dela; consequentemente, era
necessário que houvesse um outro preceito para proibir que se tirasse
proveito.

Nós já mencionamos a razão pela qual o Misericordioso não escreveu
``comer'' no caso da carne no leite, que é que toda vez que ``comer''
for mencionado não se é culpado a menos que se sinta prazer com isso.
Se, contudo, alguém devesse abrir a boca e engolir comida proibida, ou
comê-la enquanto estivesse quente demais e como resultado queimasse a
garganta com ela, causando-lhe dor ao engoli-la, ou em qualquer caso
semelhante, ele estaria isento. As únicas exçessões são o caso da carne
no leite e o de ``quil-ei ha querem'', como explicaremos posteriormente;
nesses casos ele é culpado por comer, mesmo que não sinta prazer com
isso. Você deve compreender e lembrar-se de todos esses princípios.

As normas deste preceito estão explicadas no oitavo capítulo de Hulin.

\paragraph{Não comer a carne de um boi apedrejado}

Por esta proibição somos proibidos de comer a carne de um boi que tenha
sido apedrejado, mesmo que ele tenha sido abatido antes de ser
apedrejado, porque uma vez que a sentença foi pronunciada ele passou a
ser alimento proibido, mesmo se ele tiver sido abatido de acordo com os
requisitos rituais. Esta proibição está expressa em Suas palavras,
enaltecido seja Ele, ``Não será comida a sua carne'' (Exodo 21:28); e a
Mekhiltá diz: ``Se os proprietários de um boi que está sendo conduzido
ao apedrejamento o abaterem de acordo com os ritos, antes de seu
apedrejamento, sua carne será mesmo assim alimento proibido. Esse é o
significado de ``Não será comida da sua carne''.

Aquele que comer o equivalente ao tamanho de uma oliva de sua carne será
punido com o açoitamento.


\paragraph{Não comer pão feito com grãos da nova ceifa}

Por esta proibição somos proibidos de comer pão feito com grãos da nova
ceifa, antes do final do décimo sexto dia de Nissan. Ela está expressa
em Suas palavras, enaltecido seja Ele, ``E pão, e farinha feita de grãos
das espigas verdes, torrada no forno, e grãos verdes de cereais não
comereis'' (Levítico 23:14).

Aquele que comer o equivalente ao tamanho de uma oliva disso será punido
com o açoitamento.

\paragraph{Não comer grãos da nova ceifa torrados}

Por esta proibição somos proibidos de comer grãos da nova ceifa
torrados, antes do final do décimo sexto dia de Nissan. Ela está
expressa em Suas palavras, enaltecido seja Ele, ``E pão, e farinha feita
de \emph{grãos das espigas verdes, torrada} no forno, e grãos verdes de
cereais não comereis'' (Levítico 23:14).

Aquele que comer o equivalente ao tamanho de uma oliva disso será punido
com o açoitamento.

\paragraph{Não comer grãos verdes de cereais}

Por esta proibição somos proibidos de comer grãos verdes de cereais
antes do final do décimo sexto dia de Nissan. Ela está expressa em Suas
palavras, enaltecido seja Ele, ``E pão, e farinha feita de grãos das
espigas verdes, torrada no forno, e \emph{grãos verdes de cereais} não
comereis'' (Levítico 23:14).

Já nos referimos às palavras do Talmud: ``Aquele que comer do pão e da
farinha feita de grãos de espigas verdes, torrada no forno, e dos grãos
verdes de cereais será culpado por cada um deles separadamente'' e
também já explicamos isto em detalhes no nono dos Fundamentos que
prefaciam este trabalho, e aos quais você deve se reportar.

As normas da lei sobre a nova ceifa estão explicadas no sétimo capítulo
de Menahot, e em diversos trechos de Shebiit, Maasserot e Halá.

\paragraph{Não comer ``orlá''}

Por esta proibição somos proibidos de comer ``orlá''. Ela está expressa
em Suas palavras ``Por três anos vos será proibido; não se comerá''
(Levítico 19:23).

Aquele que comer o equivalente ao tamanho de uma oliva disso será punido
com o açoitamento.

As normas deste preceito estão explicadas no Tratado Orlá.

A proibição contra comer ``orlá'' fora da Terra de Israel está expressa
numa lei dada a Moisés no Sinai. O texto da Torah proíbe comê-los apenas
na Terra de Israel.

\paragraph{Não comer ``quil-ei ha querem''}

Por esta proibição somos proibidos de comer ``quil-ei ha querem''. Ela
está expressa em Suas palavras, enaltecido seja Ele, relativas a eles,
``Para que não se profane (pen tikdash) o produto com o que haja a mais
na semente que semeares'' (Deuteronômio 22:9), sobre as quais a Tradição
diz: ```Pen tidkash --- pen tudak esh' (para que não seja consumido
pelo fogo)", ou seja, é proibido tirar-se algum proveito disso. Já nos
referimos ao princípio de que ``Toda vez que as Escrituras dizem
`guarda-te' (hishamer), ou `para que não' (pen), ou `não' (al), há um
preceito negativo''.

O segundo capítulo de Pessahim, depois de estabelecer que ``Os artigos
proibidos pela Torah não acarretam o açoitamento a não ser na sua
maneira habitual de consumo'', ou seja, que só se fica sujeito ao
castigo por se ter comido um alimento proibido se se sentiu prazer em
comê-lo. Diz ainda: ``Abayé disse: Todos concordam que se deve impor
açoitamento no caso de um `quil-ei ha querem', mesmo que não de acordo
com seu uso habitual. Qual é a razão? Porque não se menciona `comer'
neste caso'', uma vez que as Escrituras dizem apenas ``pen tidkash''
--- (i.e.) pen tudak esh (para que não seja consumido pelo fogo).

As normas deste preceito estão explicadas no Tratado Quil-Aim. De acordo
com as Escrituras este preceito também só se aplica na Terra de Israel.

\paragraph{Não beber ``yain nessech''}

Por esta proibição somos proibidos de beber ``yain nessech'' (i.e.,
vinho de libação que foi usado para adoração de ídolos). Esta proibição
não está enunciada explicitamente nas Escrituras; mas elas dizem, a
respeito da idolatria: ``De cujos sacrifícios comiam a gordura e de
cujas libações bebiam o vinho'' (Deuteronômio 32:38), e isso demonstra
que a proibição que se aplica a sacrifícios oferecidos a um ídolo se
aplica da mesma forma ao vinho de libação.

Você já está familiarizado com o princípio, frequentemente mencionado
no Talmud, de que é proibido tirar proveito, sob pena do açoitamento.

Encontramos a prova de que o ``yain nessech'' é uma das proibições da
Torah, e que essa proibição deve ser contada entre os preceitos
negativos, na Guemará de Abodá Zará: ``Rabi Yohanan e Rabi Shimeon ben
Lakish declararam: Com todas as coisas proibidas pela Torah, quer
consista a mistura da mesma variedade ou de variedades
diferentes\footnote{A mistura será desqualificada para o consumo quando o elemento
  proibido transmitir um sabor.}, quando transmite um sabor, com exceção
de ``tebel'' e do ``yain nessech''. Nesses casos, com a mesma
variedade\footnote{A mistura está desqualificada.}, pela menor
quantidade\footnote{Do elemento proibido.}, mas com variedades diferentes,
quando\footnote{Esse elemento proibido.} revelar um sabor''. Isso é uma prova clara
de que a proibição do ``yain nessech'' está nas Escrituras.

Da mesma forma no Sifrei, numa descrição do erro de Israel em Shitim ao
praticar a prostituição com as filhas de Moab\footnote{Ver Números 25:1.},
lemos: ``Entraram; uma garrafa de vinho amonita estava junto dela, e naquela época o vinho pagão ainda não havia sido proibido a Israel. Ela disse a ele: `Você
gostaria de beber?' etc.''. As palavras ``naquela época o vinho pagão
ainda não havia sido proibido a Israel'' provam acima de qualquer dúvida
que depois ele foi proibido.

As afirmações no Talmud de que a proibição de vinho pagão está entre as
Dezoito Regras prescritas e de que ``O caso do `yain nessech' é
diferente, visto que \emph{os Sábios} impuseram maior restrição a esse
respeito'', se referem na realidade apenas ao vinho pagão, não ao
``yain nessech'', que está proibido pelas escrituras.
E você conhece a máxima dos Rabinos: ``Há três tipos de vinho'' etc.

As normas deste preceito estão explicadas nos últimos capítulos de Abodá Zará.

\paragraph{Não comer nem beber em excesso}

Por esta proibição somos proibidos de dedicar-nos excessivamente à
comida e à bebida enquanto formos jovens, de acordo com as condições
descritas no caso de um filho inflexível e rebelde. Esta proibição está
expressa em Suas palavras, enaltecido seja Ele, ``Não comereis sobre o
sangue'' (Levítico 19:26).

A explicação disso é a seguinte. O filho inflexível e rebelde é um dos
que estão sujeitos à morte por condenação judicial, e a Torah determina
expressamente que no seu caso a morte seja por
apedrejamento\footnote{Ver Deuteronômio 21:21.} Nós já explicamos na Introdução a
este trabalho que cada vez que as Escrituras estabelecem a pena de
extinção ou morte por condenação judicial há um preceito negativo,
exceto nos casos dos sacrifícios de ``Pessah'' e da circuncisão, como
explicamos\footnote{Ver o preceito positivo 55.}. Portanto, como a lei determina que o
glutão e beberrão está sujeito ao apedrejamento, nas condições
mencionadas, sabemos que isso é algo que está totalmente proibido; e,
uma vez estabelecida a punição, resta-nos encontrar a proibição, já que
é um princípio aceito que a Torah nunca prescreve um castigo sem ter
primeiro enunciado uma proibição\footnote{Ver o preceito positivo 4.}. Na Guemará de
Sanhedrin lemos: ``De que forma chegamos à proibição dirigida contra um
filho teimoso e rebelde? Pelo versículo `Não comereis sobre o sangue'''.
Quer dizer, não comereis de forma tal que isso ocasione derramamento de
sangue, e é assim que come o glutão e o beberrão, cujo castigo é a
morte. Se alguém comer dessa forma ruim e repreensível ele transgredirá
um preceito negativo e o fato desta proibição ser um ``lav
shebikhlalut'' não tem consequências, como explicamos no Nono
Fundamento, pois como o castigo está formulado explicitamente nas
Escrituras, não importa se a proibição está expressa numa lei ou num
``lav shebikhlalut''. Nós já explicamos isso várias vezes e já
apresentamos exemplos disso antes.

As normas deste preceito estão explicadas no oitavo capítulo de Sanhedrin.

\paragraph{Não comer durante um ``Yom Kipur''}

Por esta proibição somos proibidos de comer durante ``Yom Kipur''. A
Torah não contém nenhuma proibição expressa a esse respeito, mas ela
menciona o castigo --- a saber, que aquele que comer nesse dia estará
sujeito à extinção --- em Suas palavras ``Porque toda alma que não se
afligir neste mesmo dia, será banida de seu povo'' (Levítico 23:29).
Consequentemente, sabemos que é proibido comer durente ``Yom Kipur''.

No início de Queretot aquele que comer em ``Yom Kipur'' está incluído
entre aqueles que estão sujeitos à extinção, e está explicado que há
um preceito negativo toda vez que se fica sujeito à extinção, exceto nos
casos do sacrifício de ``Pessah'' e da circuncisão. Assim sendo, fica
claro que comer em ``Yom Kipur'' é um preceito negativo; consequentemente aquele que o
transgredir voluntariamente será punido com a extinção e aquele que o violar
involuntariamente, com um Sacrifício Determinado de Pecado, de acordo
com o que está explicado no início de Queretot e no Tratado Horayot: que
esta obrigação se aplica apenas aos preceitos \emph{negativos,} uma vez
que com relação a quem é obrigado a levar um Sacrifício Determinado de
Pecado, Ele diz, enaltecido seja Ele, ``Por fazer um dos preceitos do
Eterno, daqueles que são de \emph{não} fazer'' (Levítico 4:27).

A Sifrá diz: ```Porque toda a alma que não se afligir neste mesmo dia,
será banida do seu povo'. Isto determina o castigo por deixar de afligir
a alma, mas não apresenta nenhuma proibição explícita com relação a afligir-se
nesse dia. Não havia necessidade de que as Escrituras enunciassem o castigo
acarretado pelo fato de trabalhar\footnote{Levítico 23:30.}, porque isso se
deduz através de um `cal vahomer'\footnote{Isto é,``com toda razão''.}. Se deixar de
afligir a alma em `Yom Kipur' é algo que deva ser punido, conclui-se que
certamente a execução de um trabalho, que \emph{está} determinada com
relação aos Festivais e Shabatot, deve ser punida. Então por que está\\
expresso o castigo decorrente do fato de trabalhar? Para que se derive
dele a \emph{proibição} relativa à aflição da alma: assim como o castigo por
trabalhar está precedido por uma proibição, o castigo por afligir a alma
também está precedido por uma proibição''. Dessa forma explicamos o que havíamos prometido explicar.

As normas deste preceito estão explicadas no final do Tratado Yoma.

\paragraph{Não comer ``hametz'' durante ``Pessah''}

Por esta proibição somos proibidos de comer ``hametz'' durante
``Pessah''. Ela está expressa em Suas palavras, enaltecido seja Ele,
``Não comereis coisa levedada'' (Êxodo 13:3).

A transgressão voluntária desta proibição será punida pela extinção, de
acordo com o que está claramente expresso em Suas palavras ``Pois todo
aquele que comer coisa levedada, será banida aquela alma de Israel''
(Ibid., 12:15). Aquele que a transgredir involuntariamente é obrigado a
levar um Sacrifício de Pecado.

As normas deste preceito estão explicadas no Tratado Pessahim.

\paragraph{Não comer nada que contenha ``hametz'' durante ``Pessah''}

Por esta proibição somos proibidos de comer qualquer coisa que contenha
uma mistura de ``hametz'', mesmo que não seja pão, como por exemplo
``cutá''\footnote{Uma conserva composta de leite azedo, crostas de pão e sal.}, todo tipo de condimento, e similares.
Esta proibição está expressa em Suas palavras ``Nenhuma coisa levedada
comereis'' (Êxodo 12:20), sobre as quais diz a Mekhiltá: ``Incluindo
`cutá' da Babilônia, cerveja Central, e vinagre Idumeano. Eu poderia
pensar que o castigo é causado por eles; por isso as Escrituras dizem:
`Todo aquele que comer levedura, será banida aquela alma' (Ibid., 19),
ou seja, por comida levedada, mas não por comida com uma mistura de
levedo. Por que então essas coisas foram mencionadas? Porque são uma
desobediência a um preceito negativo''.

Está explicado no Tratado Pessahim que embora tais coisas sejam
proibidas e comê-las seja proibido, aquele que as comer não estará
sujeito ao açoitamento a menos que ele coma o equivalente ao tamanho de
uma oliva de ``hametz'' dentro do tempo em que se comeria meio
pão\footnote{A duração do consumo de três ovos (Mishné Torah, 1º cap., Hametz e
  Matzah, 6ª Halachá).}; de outra forma ele não será culpado.

\paragraph{Não comer ``hametz'' depois da metade do decimo quarto de ``Nissan''}

Por esta proibição somos proibidos de comer ``hametz'' depois da metade
do décimo quarto dia de Nissan. Ela está expressa em Suas palavras ``Não
comerás nela levedo'' (Deuteronômio 16:3), nas quais a expressão
``nela'' se refere ao cordeiro de ``Pessah'', que devemos abater no
décimo quarto dia ``ao anoitecer''. De acordo com isso Suas palavras
São: ``Não comereis pão levedado a partir da hora do abate''.

Na Guemará de. Pessahim lemos: ``Como ficamos sabendo que aquele que
come `hametz' a partir das seis horas transgride um preceito negativo?
Pelo versículo `Não comereis nela levedo'''. No mesmo Tratado também
lemos: ``Todos concordam que o `hametz' é proibido pelas Escrituras a
partir das seis horas''. Esse é o comentário que encontramos em todos os
textos corretos que foram lidos antes dos mais sábios
Talmudistas\footnote{Os ``Gaonim".}. Esse Tratado também dá a seguinte
razão para a proibição de comer ``hametz'' na sexta hora: ``Os Sábios
estabeleceram uma proteção para uma proibição das Escrituras''.

Aquele que transgredir ao comer ``hametz'' depois da metade do dia será
punido com o açoitamento.

As normas deste preceito estão explicadas no início de Pessahim.

\paragraph{Não pode ser visto ``hametz'' em nossas moradias durante ``Pessah''}

Por esta proibição está proibido que se veja ``hametz'' em qualquer uma
das nossas moradias durante os sete dias\footnote{Os sete dias de Pessah.}. Esta
proibição está expressa em Suas palavras ``Não será vista por ti coisa
levedada, e não será visto contigo fermento, em todo o teu território''
(Êxodo 13:7).

Essas não são duas proibições, relativas a dois assuntos diferentes, mas
sim referem-se a um único tópico. O Talmud explica: ``As Escrituras
iniciam com levedo e terminam com pão levedado para ensinar-nos que
levedo e pão levedado são a mesma coisa''. Quer dizer, não há diferença
entre o levedo e aquilo que é levedado.

Aquele que transgredir, guardando ``hametz'' em seu poder, não estará
sujeito ao açoitamento a menos que ele tenha comprado ``hametz''
durante ``Pessah'' e tenha adquirido direito a ele, realizando dessa
forma um ato específico relacionado com ele. Pelas palavras da
Tosseftá: ``Aquele que guardar `hametz' durante `Pessah' e aquele que
permitir que diversas sementes cresçam\footnote{Em seu vinhedo.}, não estará
sujeito ao açoitamento''.

\paragraph{Não possuir ``hametz'' durante ``Pessah''}

Por esta proibição somos proibidos de ter ``hametz'' em nossa posse,
ainda que ele não esteja visível ou mesmo que ele tenha sido deixado com
alguém. Esta proibição está expressa em Suas palavras ``Sete dias
levedura não será encontrada em vossas casas'' (Êxodo 12:19). Também por
esta proibição o castigo será o açoitamento, desde que haja um ato
envolvido, como mencionamos, de acordo com os princípios estabelecidos
no Tratado Shebuot. Os sábios dizem explicitamente, em vários lugares:
``Transgride-se `não será visto' e `não será encontrado'''.

No início do Tratado Pessahim estão explicadas as normas desses dois
preceitos, bem como as coisas proibidas por Suas proibições ``Não será
visto por ti fermento, em todo o teu território'' (Ibid., 13:7) e
``Levedura não será encontrada em vossas casas''. Está explicado ali que
cada um desses preceitos negativos obtém alguma coisa a mais do outro, e
que aquele que permitir que ``hametz'' fique em seu poder durante
``Pessah'' transgride dois preceitos: ``não será visto'' e ``não será
encontrado''.

\paragraph{Um Nazir não pode beber vinho}

Por esta proibição um Nazir fica proibido de beber vinho ou qualquer
outro tipo de bebida forte da qual o suco da uva seja um componente
importante. Ela está expressa em Suas palavras, enaltecido seja Ele, ``E
todo remolho de beberagem de uvas não beberá'' (Números 6:3).

Ele levou essa proibição tão longe a ponto de transcrevê-la e está
declarado que mesmo se o vinho ou a bebida feita com ele azedar, ele não
pode bebê-la. Mas esta proibição, expressa em Suas palavras ``Vinagre
de vinho novo e vinagre de vinho velho não beberá'' (Ibid.), não é um
preceito independente. Se Ele tivesse dito: ``Ele não beberá vinho nem
vinagre de vinho'' haveria dois preceitos diferentes; mas Suas
palavras, ``Vinagre de vinho\ldots{} não beberá'', complementam a proibição
do vinho.

A Guemará de Nazir explica que Suas palavras ``mishrat'anab im''
(``licor de uvas'') ``significam que o sabor é equivalente à substância
em si''.

Uma prova de que elas são um único preceito é que se ele beber vinho e
vinagre de vinho ele não será punido por açoitamento duas vezes; é esse
o castigo aplicado se ele beber um ``rebiit''\footnote{Um quarto de um ``log''.} de
vinho ou de vinagre.

\paragraph{um Nazir não pode comer uvas frescas}

Por esta proibição um Nazir fica proibido de comer uvas frescas. Ela
está expressa em Suas palavras ``E uvas frescas\ldots{} não comerá''
(Números 6:3). Se ele comer o equivalente ao tamanho de uma oliva delas
ele será punido com o açoitamento.

\paragraph{um Nazir não pode comer uvas secas}

Por esta proibição um Nazir fica proibido de comer uvas secas. Ela está
expressa em Suas palavras ``E uvas\ldots{} secas não comerá'' (Números 6:3).
Se ele comer o equivalente ao tamanho de uma oliva delas ele será punido
com o açoitamento.

\paragraph{um Nazir não pode comer caroços de uvas}

Por esta proibição um Nazir fica proibido de comer caroços de uvas. Ela
está expressa em Suas palavras ``Desde \emph{as grainhas} até a casca
das uvas não comerá'' (Números 6:4). Se ele comer o equivalente ao
tamanho de uma oliva delas ele será punido com o açoitamento.

\paragraph{um Nazir não pode comer bagaços de uvas}

Por esta proibição um Nazir fica proibido de comer bagaços de uvas. Ela
está expressa em Suas palavras ``Desde as grainhas \emph{até a casca das
uvas} não comerá'' (Números 6:4). Se ele comer o equivalente ao tamanho
de uma oliva delas ele será punido com o açoitamento.

A prova de que essas cinco proibições --- a do vinho, a de uvas
frescas, a de uvas secas, a de caroços e a de bagaços de uvas --- são
cada uma um preceito individual é o fato de que o castigo por qualquer uma delas é a
punição por açoitamento. A Mishná diz: ``Há um castigo separado por
causa do vinho, das uvas, dos caroços das uvas e dos bagaços das uvas''
e na Guemará de Nazir está explicitamente dito: ``Se ele comesse uvas
frescas, uvas secas, os caroços de uvas, e os bagaços de uvas, e se
espremesse um cacho de uvas e bebesse\footnote{O suco resultante.}, ele estaria
sujeito ao açoitamento por cinco vezes''. Na tentativa de provar que o
``Taná'' não mencionou todas as vezes, e que na realidade um Nazir
poderia estar sujeito a mais do que cinco açoitamentos, a Guemará diz
que ele omitiu o preceito ``Não profanará a Sua palavra'' (Números
30:3); mas ela não diz que ele omitiu o vinagre de vinho porque não se é
culpado uma vez por beber vinho e outra por beber vinagre de vinho. O
vinagre só é proibido porque ele é essencialmente vinho, como
explicamos, e é como se Ele tivesse dito que ao se tornar vinagre, o
vinho não perde a característica principal que faz com que ele seja
proibido.

É importante que você saiba que todas as coisas proibidas a um Nazir
podem ser postas juntas para perfazer um volume total equivalente ao
tamanho de uma oliva, que acarreta a pena de açoitamento para aquele
que o comer.

\paragraph{um Nazir não pode fazer-se impuro pelos mortos}

Por esta proibição um Nazir fica proibido de fazer-se impuro pelos
mortos. Ela está expressa em Suas palavras ``Por seu pai, por sua mãe,
por seu irmão, e por sua irmã, não se impurificará'' (Números 6:7).
Aquele que se fizer impuro por qualquer pessoa morta, seja a impureza
uma daquelas pelas quais ele deve cortar seu cabelo ou não, ele será
punido com o açoitamento.

\paragraph{um Nazir não pode fazer-se impuro entrando numa casa onde haja um morto}

Por esta proibição um Nazir fica proibido de fazer-se impuro numa casa
com um morto. Ela está expressa em Suas palavras ``Não se aproximará de
um morto'' (Números 6:6), e a Guemará diz explicitamente que as
Escrituras fazem uma afirmação clara: ```Não se impurificará' (Ibid.,7);
ao acrescentar `Não \emph{se aproximará}', proíbe que se
contamine e que entre''. Está explicado ali que se ele entrar numa casa
com um morto depois de ter-se tornado impuro, ele será punido com o
açoitamento apenas uma vez; mas se ele se tornar impuro e entrar na casa
ao mesmo tempo --- como, por exemplo, se ele entrar numa casa na qual
houver um moribundo e ficar ali até que o homem morra, de maneira que
os fatos de ter-se tornado impuro e o de ter entrado na casa onde há um
morto ocorram ao mesmo tempo --- ele estará sujeito ao açoitamento por
duas vezes. Contudo, se ele entrar numa casa onde haja um morto, ele se
torna impuro antes de sua entrada, como está explicado ali, de acordo
com os princípios estipulados em Oholoth.

\paragraph{um Nazir não pode raspar a cabeça}

Por esta proibição um Nazir fica proibido de raspar a cabeça. Ela está
expressa em Suas palavras, enaltecido seja Ele, ``Lâmina não passará por
sua cabeça'' (Números 6:5).

Aquele que raspar a cabeça de um Nazir também será punido com o
açoitamento, porque quem barbeou e quem se deixou barbear são
igualmente culpados e a pena de açoitamento incide a partir do momento
em que ele tenha raspado um único fio de cabelo.

Todas as normas da lei dos Nazirim estão explicadas no Tratado dedicado
especialmente a este assunto.

\paragraph{Não ceifar toda a colheita}

Por esta proibição somos proibidos de ceifar a totalidade da colheita;
deve-se deixar uma parte num canto do campo para os pobres. Esta
proibição está expressa em Suas palavras, enaltecido seja Ele, ``Não
acabarás de segar o canto de teu campo'' (Levítico 19:9).

Este preceito negativo está justaposto a um preceito positivo porque se
alguém o transgredir e segar toda a colheita ele deve dar a medida de
``peá'' da colheita ceifada aos pobres. Isto aparece em Suas palavras,
enaltecido seja Ele, ``Para o pobre e o imigrante os deixarás'' (Ibid.,
10), como explicamos ao tratar dos preceitos
positivos\footnote{Ver o preceito positivo 120.}. A ``peá'' é obrigatória tanto no caso
das árvores como no caso dos campos.

De acordo com as Escrituras, este preceito só é obrigatório na Terra de
Israel. Suas normas estão explicadas no Tratado especialmente dedicado a
este assunto.

\paragraph{Não recolher as espigas de cereais que caíram durante a colheita}

Por esta proibição somos proibidos de recolher as espigas de cereais que
caírem durante a colheita; elas devem ser deixadas para os pobres. Esta
proibição está expressa em Suas palavras, enaltecido seja Ele, ``E as
espigas caídas no recolhimento de tua ceifa, não recolherás'' (Levítico
19:9).

Este também está justaposto a um preceito
positivo\footnote{Ver o preceito positivo 121.}, como explicamos no caso da ``peá''. Suas normas estão explicadas no Tratado Peá.

\paragraph{Não recolher todo o produto do vinhedo na época da vindima}

Por esta proibição somos proibidos de recolher todo o produto do vinhedo
na época da vindima. Esta proibição está expressa em Suas palavras,
enaltecido seja Ele, ``E tua vinha não rebuscarás'' (Levítico 19:10); os
cachos de uvas que não estiverem totalmente desenvolvidos devem ser
deixados para os pobres.

Isto não se aplica a outras árvores, ainda que elas sejam similares às
videiras, porque a proibição expressa em Suas palavras ``Quando bateres
a tua oliveira, não tornarás a colher o que resta nos ramos''
(Deuteronômio 24:20) significa que não devemos recolher uma oliva
\emph{esquecida,} e essa lei sobre a oliva esquecida se aplica também
às outras árvores\footnote{Mas não há menção quanto à lei das frutas não desenvolvidas em outras
  Arvores.}.

Este também está justaposto a um preceito
positivo\footnote{Ver o preceito positivo 123.}, e suas normas estão explicadas no
Tratado Peá.

\paragraph{Não recolher os bagos das uvas que caírem durante a colheita}

Por esta proibição somos proibidos de recolher os bagos soltos que
caírem durante a vindima; eles devem ser deixados para os pobres. Esta
proibição está expressa em Suas palavras ``E bago de tua vinha não
recolherás'' (Levítico 19:10).

Este também está justaposto a um preceito positivo\footnote{Ver o preceito positivo 124.}
e suas normas estão explicadas no Tratado Peá.

\paragraph{Não voltar para buscar uma gavela esquecida}

Por esta proibição somos proibidos de ir buscar um feixe de espigas.
esquecido. Ela está expressa em Suas palavras ``E esqueceres uma gavela,
no campo, não voltarás a tomá-la'' (Deuteronômio 24:19). A obrigação do
que é esquecido se aplica tanto ao solo quanto à
árvore\footnote{A proibição de recolher o que foi esquecido se aplica a todos os
  produtos do solo e das árvores.}. Este preceito também está justaposto ao
preceito positivo expresso em Suas palavras, enaltecido seja Ele, ``Para
o imigrante, o órfão, e a viúva será'' (Ibid.)\footnote{Ver o preceito positivo 122.} e
se alguém pecar e o recolher, deverá devolvê-lo ao pobre. As normas
deste preceito estão explicadas no Tratado Peá.

Você deve saber que é um princípio aceito entre nós, no caso de um
preceito negativo que está acompanhado de uma ordem a uma ação positiva,
que se o transgressor executar a ordem positiva ele não será punido com
o açoitamento, mas se não a executar, ele o será. No caso da ``peá'',
por exemplo, se ele ceifar a colheita inteira não ficará sujeito ao
açoitamento na época da colheita, e poderá dar aos pobres espigas de
cereais. Da mesma forma, se ele debulhar o grão e o transformar em
farinha, e fizer massa com essa farinha, ele pode dar a medida de
``peá'' da massa. Mas se acontecer de o grão se perder ou se queimar
completamente --- e sobretudo se por sua livre iniciativa ele causar o
seu desaparecimento como por exemplo, comendo-o
todo --- ele estará sujeito ao açoitamento, uma vez que terá deixado de
cumprir um preceito positivo.

Vocês não devem interpretar mal as palavras da Guemará de Macot: ``Nós
só temos este caso e um outro'', onde ``um outro'' significa a ``peá'',
deduzindo, como vocês poderiam fazer, que essa regra se aplica
\emph{apenas à} ``peá''. Não é assim. Na realidade, ``um outro''
significa o caso da ``peá'' e todos os casos semelhantes, já que os
bagos caídos, as espigas, o que foi esquecido e os cachos de uvas não
totalmente desenvolvidos são cada um preceito negativo acompanhado de
uma ordem a uma ação positiva; e em cada um desses casos, como no caso
da ``peá'', há possibilidades alternativas de ``kiyemu ve lo
kiyemu''\footnote{Se ainda houver uma possibilidade de cumprir o preceito positivo, ele
  não estará sujeito à punição, mas se não houver mais nenhuma
  possibilidade de cumpri-lo, ele estará sujeito à punição.} ou de ``bitlo ve lo
bitlo''\footnote{Se ele próprio anulou a possibilidade de cumprir o preceito
  positiivo, ele estará sujeito à punição, mas se não foi ele mesmo
  quem anulou essa possibilidade, então ele não ficará sujeito à
  punição.}. Pois o texto através do qual ficamos
sabendo que o caso da ``peá'' está associado a uma ordem de uma ação
positiva é o de Suas palavras, enaltecido seja Ele, ``Para o pobre e o
imigrante os deixarás'' (Levítico 19:10), e essas palavras se aplicam à
``peá'', espigas, bagos soltos de uvas, e cachos de uvas não totalmente
desenvolvidos. Suas palavras são: ``Não acabarás de segar o canto de teu
campo, e as espigas caídas no recolhimento de tua ceifa, não recolherás.
E tua vinha não rebuscarás, e o bago de tua vinha não recolherás; para
o pobre e o imigrante os deixarás'' (Ibid., 9-10). E Ele diz mais
adiante, sobre a gavela esquecida: ``Não voltarás a tomá-la; para o
imigrante, o órfão, e a viúva será''. E como encontramos na Guemará que
a ``peá'' é um preceito negativo justaposto a um preceito positivo,
encontrado em Suas palavras ``Para o pobre e o imigrante os deixarás'',
segue-se que todas essas cinco proibições\footnote{As proibições estabelecidas nos preceitos negativos 210 a 214.} são
preceitos negativos justapostos a preceitos positivos;
consequentemente, se alguém cumprir o preceito positivo não será punido
com o açoitamento, como mencionamos, e se não lhe for possível
cumpri-1o, ele será punido com o açoitamento. Mas enquanto houver a
possibilidade de cumpri-lo, ele ficará isento do castigo mesmo que ele
ainda não o tenha cumprido, e nós simplesmente lhe ordenaremos que o
cumpra. Ele só passa a ficar sujeito ao açoitamento quando soubermos
que ele infringiu a proibição e que não resta nenhuma possibilidade de
cumprir o preceito positivo.

Você precisa saber e compreender isto.

\paragraph{Não semear ``quil-aim''}

Por esta proibição somos proibidos de semear
``quil-aim''\footnote{Ou seja, semear diversos tipos de sementes num só campo, tais como
  trigo com aveia. Também conhecido como ``quil-ei
  zeraim''.}. Ela está expressa em Suas palavras,
enaltecido seja Ele, ``Teu campo não semearás com diversas sementes''
(Levítico 19:19).

Semear ``quil-ei zeraim'' só é proibido na Terra de Israel e aquele que
o fizer estará sujeito, de acordo com as Escrituras, ao açoitamento. E
permitido fazê-lo fora da Terra de Israel.

As normas deste preceito estão explicadas no Tratado Quil-Aim.

\paragraph{Não semear grãos nem vegetais num vinhedo}

Por esta proibição estamos proibidos de semear grãos ou vegetais num
vinhedo. Esta forma de ``quil-aim'' é chamada ``quil-ei ha querem'', e a
proibição está expressa em Suas palavras, enaltecido seja Ele, ``Não
semearás a tua vinha com diferentes espécies de sementes'' (Deuteronômio
22:9). Sobre isso o Sifrei diz: ```Não semearás a tua vinha': Por que
isto é necessário? Já não nos foi dito `Teu campo não semearás com
diversas sementes' (Levítico 19:19), o que sem dúvida alguma, inclui
ambos o vinhedo e o campo?'' A resposta é que isto é para ensinar-nos
que aquele que permite que grãos ou vegetais cresçam em seu vinhedo
infringe dois preceitos negativos.

Vocês precisam saber que, de acordo com as Escrituras, ``quil-ei ha
querem'' só é proibido na Terra de Israel; aquele que semear na Terra de
Israel estará sujeito ao açoitamento, de acordo com as Escrituras, se
ele semear, de uma só vez, trigo e cevada misturados com caroços de
uvas. A lei dos Rabinos proíbe semear fora da Terra de Israel também, e
aquele que semear trigo e cevada junto com caroços de uvas, com um
único movimento, fora da Terra de Israel, estará sujeito ao açoitamento,
de acordo com a lei Rabínica. Mas o enxerto de árvores, cuja proibição
está incluída em Suas palavras ``Teu campo não semearás com diversas
sementes'' é punido com açoitamento em qualquer lugar.

As normas deste preceito estão explicadas no Tratado Quil-Aim.

\paragraph{Não cruzar animais de espécies diferentes}

Por esta proibição somos proibidos de cruzar animais de espécies
diferentes. Ela está expressa em Suas palavras ``O teu animal não farás
juntar com outra espécie'' (Levítico 19:19). A penalidade por isso é o
açoitamento, desde que se auxilie o cruzamento da forma como se coloca
um pincel num canudo. O Talmud diz explicitamente: ``Em caso de
adultério eles\footnote{As testemunhas.} devem tê-los visto em atitude de
adultério; mas com relação a espécies diversas, eles devem tê-lo visto
auxiliando tal como alguém que estivesse colocando um pincel num
canudo''. Só assim se está sujeito ao açoitamento.

As normas deste preceito estão explicadas no oitavo capítulo do Quil-Aim.

\paragraph{Não trabalhar com duas espécies diferentes de animais juntos}

Por esta proibição somos proibidos de trabalhar com animais de duas
espécies diferentes juntos. Ela está expressa em Suas palavras ``Não
lavrarás com boi e jumento juntamente'' (Deuteronômio 22:10). Aquele que trabalhar
--- por exemplo, lavrar, debulhar, ou puxar --- com eles será punido com
açoitamento. A palavra ``juntos'' significa que não devemos juntá-los
para fazer qualquer espécie de trabalho.

De acordo com as Escrituras só se incorre na penalidade de açoitamento
se os dois animais de espécies diferentes forem um animal puro e um
impuro, como, por exemplo, um boi e um jumento; se alguém arar ou puxar
com dois animais assim juntos, ou se os conduzir juntos, ele será pu
nido com o açoitamento. De acordo com a lei Rabínica, contudo,
incorre-se nessa penalidade se se trabalhar com animais de duas espécies
diferentes quaisquer.

As normas deste preceito estão explicadas no oitavo capítulo de Quil-Aim.

\paragraph{Não impedir um animal de comer do produto no meio do qual ele esteja
trabalhando}

Por esta proibição somos proibidos de impedir um animal de comer do
produto no meio do qual ele esteja trabalhando. Assim, se um animal
estiver esmagando cereais, ou carregando palha no seu dorso de um lugar
para outro, não devemos impedi-lo de comer do grão ou da palha. Esta
proibição está expressa em. Suas palavras, enaltecido seja Ele, ``Não
amarrarás a boca do boi quando estiver debulhando'' (Deuteronômio 25:4),
e está explicado que a proibição de amarrar a boca se aplica da mesma
forma ao boi e aos outros animais, embora as Escrituras mencionem apenas
o de uso comum. Quer esteja o animal amassando cereais ou fazendo
qualquer outro trabalho, ele não deve ser impedido de comer do produto
no meio do qual ele esteja trabalhando; toda vez que alguém o impedir de
fazê-lo, nem que seja apenas dizendo ao animal para que não coma, ele
será punido com o açoitamento.

As normas deste preceito estão explicadas no sétimo capítulo de Baba Metzia.

\paragraph{Não cultivar o solo no sétimo ano}

Por esta proibição somos proibidos de cultivar\footnote{Ver o preceito positivo 135.} o
solo no sétimo ano. Ela está expressa em Suas palavras, enaltecido seja
Ele, ``E no sétimo ano, sábado de descanso será para a terra\ldots{} teu
campo não semearás'' (Levítico 25:4).

A contravenção a esta proibição será punida com o açoitamento. Suas
normas estão explicadas no Tratado Shebiit.


\paragraph{Não podar árvores no sétimo ano}

Por esta proibição somos proibidos de cultivar árvores no sétimo ano.
Ela está expressa em Suas palavras, enaltecido seja Ele, ``E tua vinha
não podarás'' (Levítico 25:4). A penalidade pela contravenção desta
proibição também é o açoitamento.

A Sifrá diz: ``Semear e podar já estão incluídos no preceito
geral''\footnote{Relativo ao sétimo ano.}; então por que eles foram mencionados
especificamente? Para permitir que se faça uma analogia: como semear e
podar têm a característica específica de serem trabalhos comuns ao
campo e ao pomar, eu deduzo\footnote{Eu deduzo que a proibição se aplica \emph{apenas} aos tipos de trabalho etc.} os tipos de trabalho
que sejam comuns ao campo e ao pomar.

As normas deste preceito também estão explicadas no Tratado Shebiit.

\paragraph{Não ceifar uma planta que nasceu por si só no sétimo ano da maneira
como se faz num ano comum}

Por esta proibição somos proibidos de ceifar qualquer coisa que cresça
por si só no sétimo ano da maneira como o faríamos num outro ano
qualquer. O significado disto é que somos proibidos de cultivar o solo
ou árvores no ano de Shabat, como já mencionamos antes, mas podemos
comer, durante o sétimo ano, aquilo que crescer das sementes, que
tiverem caído no solo durante o sexto ano, conhecido como ``produção
tardia'' --- com a diferença de que ela deve ser colhida de maneira
diferente. Esta proibição está expressa em Suas palavras, enaltecido
seja Ele, ``O que nascer por si mesmo, depois da ceifa, não segarás''
(Levítico 25:5). Isto não significa que a produção tardia não deva ser
colhida, pois Ele diz: ``Serão os produtos do descanso da terra, livres
para comer, para vós e para todos, igualmente'' (Levítico 25:6). O
significado é que a maneira de ceifar deve ser diferente da dos outros
anos: deve-se ceifar a produção tardia como se ceifaria um produto que
não pertença a ninguém, sem precauções e sem preparativos, como
explicaremos.

\paragraph{Não colher uma fruta que tenha crescido por si so no sétimo ano, da
mesma maneira como se faz num ano comum}

Por esta proibição somos proibidos de colher uma fruta que tenha
crescido no sétimo ano da maneira como a colheríamos num ano comum: devemos agir de modo diferente para indicar que ela não tem dono. Esta
proibição está expressa em Suas palavras, enaltecido seja Ele, ``As
uvas separadas para ti, da tua vinha, não colherás'' (Levítico 25:5),
que são interpretadas da seguinte forma: ``Não deves colhê-las da
maneira como aqueles que as colhem''. Daí as palavras dos Sábios: ``Os
figos do sétimo ano não devem ser cortados com uma faca para figos, mas
podem ser cortados com uma faca comum; as uvas não devem ser esmagadas
com uma prensa de vinho, mas podem ser esmagadas num tonel; e as olivas
não podem ser preparadas numa prensa de olivas nem com um triturador de
olivas, mas podem ser trituradas e colocadas em prensas pequenas''.

As normas deste e do preceito precedente estão explicadas no Tratado Shebiit.

\paragraph{Não cultivar o solo no Ano do Jubileu}

Por esta proibição somos proibidos de cultivar o solo no Ano do Jubileu.
Ela está expressa em Suas palavras, relativas a esse ano, ``Não
semeareis'' (Levítico 25:11), que correspondem as Suas palavras,
relativas ao Ano de Shabat, ``Teu campo não semearás'' (Ibid., 4).

O cultivo tanto do solo como das árvores está proibido no Ano do Jubileu
assim como no Ano de Shabat e por essa razão Ele diz ``Não semeareis'',
uma expressão generalizada, que abrange ambos o solo e as árvores.

A contravenção a esta proibição também será punida com o açoitamento.

\paragraph{Não ceifar a produção tardia do Ano do Jubileu da maneira como se
faz num ano comum}

Por esta proibição somos proibidos de ceifar a produção tardia do Ano do
Jubileu da maneira como a ceifamos num ano comum, como ex plicamos em
relação ao sétimo ano. Esta proibição está expressa em Suas pa lavras,
enaltecido seja Ele, ``E não segareis o que nascer por si mesmo''
(Levítico 25:11)

\paragraph{Não colher frutas no Ano do Jubileu da maneira como se faz num ano comum}

Por esta proibição somos proibidos de colher frutas no Ano do Jubileu da
maneira como as colhemos nos outros anos. Esta proibição está expressa
em Suas palavras ``E não colhereis as uvas da vinha, separadas para
vós'' (Levítico 25:11), como explicamos no caso do sétimo ano. A Sifrá
diz: ```E não segareis\ldots{} e não colhereis as uvas da vinha': assim como
isto está determinado com relação ao sétimo ano, também está determinado
com relação ao Jubileu''; quer dizer, ambos estão na mesma situação com
relação a todas essas proibições.

As normas do Ano de Shabat e do Ano do Jubileu não são obrigatórias a
não ser na Terra de Israel.

\paragraph{Não vender definitivamente nossas terras em Israel}

Por esta proibição somos proibidos de vender nossas terras na Terra de
Israel em caráter definitivo. Ela está expressa em Suas palavras ``A
terra não será vendida em perpetuidade'' (Levítico 25:23).

As normas deste preceito estão explicadas no final de Arakhin.

\paragraph{Não vender as terras dos arredores dos Levitas}

Por esta proibição somos proibidos de vender as terras dos arredores
que pertençam aos Levitas. Ela está expressa em Suas palavras ``E o
campo do arrabalde de suas cidades não será vendido'' (Levítico 25:34).

A Torah diz\footnote{Em Números}, como vocês sabem, que se devem dar
cidades aos Levitas, com arredores e campos, isto é, mil cúbitos de
arredores e dois mil cúbitos depois deles para campos e vinhedos, como
está explicado no Tratado Sotá. A proibição é dirigida aos Levitas, que
ficam proibidos de alterar essa demarcação, transformar terreno urbano
em arredores e arredores em terreno urbano, ou campos em arredores, ou
arredores em campos. Esta proibição está contida em Suas palavras ``Não
será vendido'', que a Tradição interpreta como significando que isso não
deve ser alterado.

As normas deste preceito estão explicadas no final de Arakhin.

\paragraph{Não abandonar os Levitas}

Por esta proibição somos proibidos de abandonar os Levitas, de deixar
de dar-lhes suas porções completas ou de alegrar seus corações nos
Festivais. Ela está expressa em Suas palavras ``Guarda-te de abandonar
o levita'' (Deuteronômio 12:19), sobre as quais diz o Sifrei:
```Hishamer' (guarda-te) é um preceito negativo; `pen ta azov' (de
abandonar) é um preceito negativo''.

\paragraph{Não cobrar as dívidas depois do Ano de Shabat}

Por esta proibição somos proibidos de cobrar as dívidas no Ano de
Shabat\footnote{O sétimo ano}; elas devem ser perdoadas por
completo\footnote{Ver o preceito positivo 141.}. Esta proibição está expressa em Suas
palavras, enaltecido seja Ele, ``Todo credor, que emprestou a seu
companheiro, o deixará; não reclamará a seu companheiro nem a seu
irmão'' (Deuteronômio 15:2).

De acordo com as Escrituras, este preceito é obrigatório apenas na Terra
de Israel nas ocasiões em que a exoneração da terra estiver em vigor
ali, ou seja, no Jubileu. Pela lei Rabínica, contudo, ele é obrigatório
em todos os lugares e para sempre, e não se permite que se peça o
pagamento de uma dívida depois do Ano de Shabat: a dívida deve ser
cancelada.

As normas deste preceito estão explicadas no final do Tratado Shebiit.

\paragraph{Não recusar um empréstimo que deva ser cancelado no Ano de Shabat}

Por esta proibição somos proibidos de recusar um empréstimo porque ele
será cancelado pelo Ano de Shabat\footnote{O sétimo ano.}. As Escrituras
proíbem tal tipo de relutância pelas palavras ``Guarda-te que não haja
uma coisa perversa no teu coração, nem digas\ldots{}'' (Deuteronômio 15:9). A
esse respeito diz o Sifrei: ```Hishamer' (guarda-te) é um preceito
negativo: `pen yi-yeh' (que não haja) também é um preceito negativo''.
Quer dizer, a finalidade desses dois preceitos, colocados um após o
outro, é dar maior ênfase.

\paragraph{Não deixar de fazer caridade a nossos irmãos necessitados}

Por esta proibição somos proibidos de deixar de fazer caridade e de dar
assistência a nossos irmãos necessitados ao ficarmos cientes de sua
situação infeliz, sabendo que está em nosso poder
ajudá-los\footnote{Ver o preceito positivo 195.}. Esta proibição está expressa em Suas
palavras, enaltecido seja Ele, ``Não endurecerás teu coração, e não
fecharás tua mão a teu irmão o mendigo'' (Deuteronômio 15:7). Isto nos
proíbe de agirmos de maneira avarenta e mesquinha a ponto de deixar de
dar aos necessitados.

\paragraph{Não mandar embora de mãos vazias um servo hebreu}

Por esta proibição somos proibidos de mandar embora de mãos vazias um
servo hebreu que nos serviu quando ele for liberto, ao final de seis
anos. Ao contrário, nós devemos oferecer-lhe presentes de nossa
propriedade\footnote{Ver o preceito positivo 196.}. Esta proibição está expressa em Suas
palavras, enaltecido seja Ele. ``Quando o deixares ir livre de ti, não
o deixarás ir vazio'' (Deuteronômio 1 5:1 3).

As normas deste preceito, relativo aos presentes, estão explicadas no
primeiro capítulo do Tratado Kidushin.

\paragraph{Não cobrar uma dívida de alguém que se sabe que não pode pagar}

Por esta proibição ficamos proibidos de cobrar uma dívida de alguém que
sabemos que não pode pagá-la. Ela está expressa em Suas palavras,
enaltecido seja Ele, ``Não serás para ele como credor'' (Êxodo 22:24).

Na Guemará de Baba Metzia lemos: ``De que forma sabemos que se alguém
emprestou um ``maneh''\footnote{Cem ``shekalim''.} a seu vizinho e souber que
ele não tem nada, ele nem deve passar diante dele? Pelas palavras das
Escrituras `Não serás para ele como credor'''. E a Mekhiltá diz: ```Não
serás para ele como credor': você não deve estar sempre diante dele''.

Vocês devem saber que esta proibição se aplica também quanto a pedir o
pagamento dos juros de uma dívida e por essa razão a Mishná diz que
aquele que emprestar seu dinheiro a juros também estará infringindo Suas
palavras ``Não serás para ele como credor'', como explicarei adiante.

\paragraph{Não emprestar a juros}

Por esta proibição somos proibidos de emprestar a juros. Ela está
expressa em Suas palavras ``Teu dinheiro não lhe darás com lucro
(neshekh), e com usura (marbit) não lhe darás tua comida'' (Levítico
25:37).

Esta repetição da proibição de uma única ofensa lhe dá maior força,
fazendo com que aquele que empresta a juros seja culpado de infringir
duas proibições. Elas não são duas ofensas, pois ``neshekh'' e ``ribit''
são a mesma coisa. A Guemará de Baba Metzia diz: ``Você não vai
encontrar `neshekh' sem `tarbit', nem `tarbit' sem `neshekh', e o único
objetivo das Escrituras ao mencionar cada um deles separadamente
é\footnote{Para ensinar-nos que se transgride etc.} que se transgridem duas proibições''. Também diz
ali: ``Nas Escrituras `neshekh' e `tarbit' são sinônimos''; e ainda:
``Como está escrito `Teu dinheiro não lhe darás com ``neshekh'' e com
``marbit'' não lhe darás tua comida', leia-se isso da seguinte forma:
`Teu dinheiro não lhe darás com ``neshekh'' e com ``marbit'', e com
``neshekh'' e com ``marbit'' não lhe darás tua comida'''

Dessa forma, aquele que emprestar dinheiro ou provisões com juros
transgredirá dois preceitos, além das outras proibições que também são
feitas a quem empresta, para dar maior ênfase, pois esta proibição está
repetida sob outra forma em Suas palavras ``Não tomarás dele lucro nem
usura'' (Ibid., 36). Como está explicado na Guemará de Baba Metzia, esta
também é uma proibição imposta aos que emprestam, mas como explicamos
no Nono Fundamento, todas essas proibições são redundantes, pois elas
são repetições do preceito que proíbe emprestar com juros.

As normas deste preceito estão explicadas no quinto capítulo de Baba Metzia.

\paragraph{Não tomar emprestado com juros}

Por esta proibição quem tomar emprestado também fica proibido de fazê-lo
com juros, porque se não houvesse uma proibição imposta também àquele
que pede emprestado, proibindo-o de fazê-lo com juros, poderíamos
imaginar que só peca aquele que empresta, porque ele prejudica alguém, e
que aquele que pede emprestado, submetendo-se a ser prejudicado, não
estaria cometendo nenhum pecado. Este caso seria semelhante ao de um
prejuízo, onde aquele que prejudica peca, mas o prejudicado não. Por
essa razão se impõe uma proibição também sobre quem pede emprestado, o
qual está proibido de fazê-lo com juros, de acordo com Suas palavras,
enaltecido seja Ele, ``Lo tashikh le'ahikha'' (Não pagarás a teu irmão
juros) (Deuteronômio 23:20), que a Tradição explica como significando:
Não deixes que nenhum juro te seja cobrado! E a Guemará de Baba Metzia
diz explicitamente: ``Quem pede emprestado transgride `lo tashikh' e
`Diante do cego não porás tropeço' (Levítico 19:14)'', como explicaremos
ao falar deste último preceito.

\paragraph{Não participar de um empréstimo a juros}

Por esta proibição somos proibidos de participar de uma transação entre
quem pede emprestado e quem empresta que envolva juros, seja como
fiador, como testemunha ou para registrar o contrato entre eles para o
pagamento de juros com os quais eles tenham concordado. Ela está
expressa em Suas palavras, enaltecido seja Ele, ``Não porás juros sobre
ele'' (Êxodo 22:24)

A Guemará de Baba Metzia diz: ``O fiador e a testemunha transgridem
apenas `Não porás juros sobre ele''' e está explicado ali que quem fizer
o registro estará na mesma situação que a testemunha e o fiador. Também
está explicado ali que embora o preceito ``Não porás juros sobre ele''
se aplique apenas a terceiros numa transação. Ele inclui também quem
empresta, e que consequentemente aquele que empresta a juros transgride
seis preceitos negativos: um, ``Não serás para ele como credor'' (Êxodo
22:24): dois, ``Teu dinheiro não lhe darás com lucro'' (Levítico
25:37); três, ``Com usura não lhe darás tua comida'' (Ibid.); quatro,
``Não tomarás dele lucro, nem usura'' (Ibid., 36); cinco, ``Não porás
juros sobre ele''; e seis, ``Diante do cego não porás tropeço'' (Ibid.,
19:14).

Também lemos ali: ``Os que transgridem preceitos negativos são os
seguintes: quem empresta, quem pede emprestado, o fiador e a testemunha.
Os Sábios acrescentam: quem registrar a dívida também. Eles transgridem
`Teu dinheiro não lhe darás', `Não tomarás dele', `Não serás para ele
como credor', `Não porás juros sobre ele', e `Diante do cego não porás
tropeço'''. E na Guemará lemos: ``Abayé disse: aquele que empresta
infringe todos; aquele que toma emprestado, `Não pagarás a teu irmão
juros' e `Diante do cego não porás tropeço'; o fiador é a testemunha,
apenas `Não porás juros sobre ele'''.

No caso de violação deste preceito, se o juro era ``ribit ketsutsa''
(juro fixo), ele deve ser tomado de quem emprestou e devolvido à pessoa
de quem ele o cobrou.

\paragraph{Não oprimir um empregado atrasando o pagamento de seus salários}

Por esta proibição somos proibidos de prejudicar um trabalhador
atrasando o pagamento de seus salários. Ela está expressa em Suas
palavras ``Não ficará a paga de um jornaleiro contigo até pela manhã''
(Levítico 19:13). Isto se refere, como mostram as palavras ``até pela
manhã'', a um trabalhador contratado para o dia, que pode pedir seus
soldos a qualquer momento da noite; mas um trabalhador contratado para a
noite, que pode pedir pagamento de seu soldo durante toda a noite e todo
o dia, deve ser pago antes do anoitecer, de acordo com Suas palavras
``No seu dia, lhe pagarás a sua diária, e isto o farás antes do pôr do
sol'' (Deuteronômio 24:15). Como diz a Mishná: ``Um trabalhador
contratado para o dia pode receber a qualquer hora durante a noite; um
contratado para a noite pode receber a qualquer hora durante o dia''.

Esses dois versículos não contêm dois preceitos, e sim apenas um e o
objetivo das duas proibições é de complementar o enunciado da lei. A
partir das duas juntas ficamos sabendo qual é a hora de efetuar os
pagamentos.

As normas deste preceito estão explicadas no nono capítulo de Baba
Metzia, onde fica claro que é apenas no caso de um trabalhador israelita
contratado que aquele que atrasar o pagamento do soldo viola um
preceito negativo; no caso de um trabalhador não israelita, ele
transgride um preceito positivo, contido em Suas palavras ``No seu dia,
lhe pagarás a sua diária'' (Deuteronômio 24: 15).\footnote{Ver o preceito positivo 200.}

\paragraph{Não tomar pela força um penhor de um devedor}

Por esta proibição somos proibidos de tomar pela força um penhor de um
devedor, a não ser por ordem de um juiz e através de seu emissário. Nós
próprios não devemos entrar na casa do devedor contra sua vontade e
pegar um penhor dele. Esta proibição está expressa em Suas palavras,
enaltecido seja Ele, ``Não entrarás em sua casa para lhe tomar o seu
penhor'' (Deuteronômio 24:10). Como a Mishná diz: ``Se um homem
empresta a seu companheiro, ele pode pegar um penhor dele apenas através
do Tribunal e não deve entrar em sua casa para ir buscá-lo pois está
escrito `Do lado de fora ficarás etc.''' (Ibid., 11).

Este preceito negativo está justaposto a um preceito positivo, que está
expresso em Suas palavras, enaltecido seja Ele, ``Restituir-lhe-ás o
penhor'' (Ibid., 13)\footnote{Ver o preceito positivo 199.} e é assim que está explicado
na Guemará de Macot.

Mas vocês devem saber que se ele não o devolver, deixando de cumprir o
preceito positivo relativo a isso, ele estará sujeito ao açoitamento e
deverá pagar pelo penhor, como está explicado no final de Macot.

As normas deste preceito estão explicadas no nono capítulo de Baba Metzia.

\paragraph{Não ficar com um penhor do qual seu proprietário precise}

Por esta proibição somos proibidos de ficar com um penhor quando seu
proprietário precisar dele; devemos devolver de dia o artigo usado
durante o dia, e a noite o artigo usado durante a noite, como diz a
Mishná: ``Ele deve devolver um travesseiro à noite e um arado de dia''.
A proibição a esse respeito está expressa em Suas palavras, enaltecido
seja Ele, ``Não passarás a noite com o seu penhor'' (Deuteronômio
24:12), sobre as quais o Sifrei diz: ```Não passarás a noite' enquanto
seu penhor estiver contigo'' e deverás devolver-lhe aquilo sem o qual
ele não pode passar, por causa de sua pobreza, como Ele explicou nas
palavras ``Pois só esta é a sua coberta, a túnica para a sua pele!''
(Êxodo 22:26).

As normas deste preceito estão explicadas no nono capítulo de Baba Metzia.

\paragraph{Não pegar um penhor de uma viúva}

Por esta proibição somos proibidos de pegar um penhor de uma viúva,
quer seja ela pobre ou rica. Ela está expressa em Suas palavras ``Não
tomarás em penhor a roupa da viúva'' (Deuteronômio 24:17). Como diz a
Mishná: ``Não se deve pegar uma garantia de uma viúva, seja ela pobre ou
rica, pois está escrito `Não tomarás em penhor a roupa da viúva'''.

As normas deste preceito estão explicadas no nono capítulo de Baba Metzia.

\paragraph{Não pegar como penhor utensílios usados para a alimentação}

Por esta proibição somos proibidos de pegar como penhor utensílios que
são usados na preparação dos alimentos, tais como vasilhas para moer,
amassar ou cozinhar, apetrechos para o abate de gado, e todos os outros
objetos que estão na categoria de ``utensílios necessários para a
preparação de alimentos''. Esta proibição está expressa em Suas
palavras, enaltecido seja Ele, ``Não lhe tomará em penhor nem a mó
abaixo, nem a mó de em cima, porque são coisas com as quais se elabora o
alimento do homem'' (Deuteronômio 24:6), sobre as quais a Mishná diz:
``Isto significa não apenas a mó de baixo e a mó de cima, mas qualquer
coisa utilizada na preparação de comida para o consumo do homem, pois
está escrito: `Porque são coisas com as quais se elabora o alimento do
homem'''.

Resta-nos explicar-lhes a afirmação ``É culpado por causa de dois
utensílios, pois está dito que `Não lhe tomará em penhor nem a mó
abaixo, nem a mó de em cima'''. Vocês poderiam deduzir que aqui há dois
preceitos separados, e tal dedução seria ainda mais prontamente
confirmada pela afirmação deles de que ``Ele é culpado por causa da mó
de baixo e por causa da mó de cima, separadamente''. Mas o significado
dessas afirmações é o seguinte.

Aquele que pegar como penhor um utensílio com o qual se prepara o
alimento necessário infringe um preceito negativo, como foi explicado.
Aquele que pegar como penhor vários utensílios, todos eles usados na
preparação do alimento necessário --- como, por exemplo, para moer ou
para assar, ou para amassar --- é culpado por cada um deles
separadamente. Este ponto não necessita de explicação, pois esse caso é
semelhante ao de pegar como penhor as vestes da viúva de Reuben, as da
viúva de Shimon e as da viúva de Levi, quando se cometeria um pecado
separado em relação a cada uma das vestes.

Mas uma dúvida pode surgir quando se pegar como penhor dois utensílios
que realizam juntos uma única operação no preparo do alimento
necessário, sendo um insuficiente sem o outro. Devemos dizer que, uma
vez que a comida só pode ser completamente processada com o uso dos
dois utensílios juntos, eles constituem um único instrumento e o
transgressor é culpado por causa de um utensílio apenas? Ou devemos
dizer que, como há dois utensílios, ele é culpado por causa de cada um
deles separadamente? Foi-nos explicado que ele é culpado por causa de
dois utensílios, embora o trabalho seja executado pelos dois juntos,
como no caso da mó de baixo e da mó de cima, sendo que a moagem seria
impossível se faltasse qualquer uma das duas pedras, e que pegar as
duas mós como penhor seria como pegar a tina de amassar e a faca de
degolar, sendo que cada uma delas tem uma finalidade específica. Este é
o significado da afirmação de que ele é culpado por causa de dois
utensílios; ela não significa que haja dois preceitos separados.

Eis o que o Sifrei nos diz a respeito do assunto que acabo de explicar
a vocês: ``Assim como a mó de baixo e a mó de cima são diferenciadas
como sendo dois utensílios que executam juntos uma única operação, e se
incorre num castigo por cada um deles em separado, também no caso de
todas as coisas compostas de dois utensílios usados juntos numa
operação incorre-se num castigo por cada um deles separadamente''. O
significado e o propósito desta afirmação são que, embora eles sejam
usados na realização de uma única operação, fica-se sujeito a um castigo
por cada um deles separadamente.

Aquele que transgredir e tomar como penhor um utensílio desses deve
colocá-lo à disposição e devolvê-lo a quem o usa. E se ele tiver sido
perdido ou queimado antes de sua restituição, ele será punido com o
açoitamento. O mesmo se aplica àquele que pegar as roupas de uma viúva
como penhor.

As normas deste preceito estão explicadas no nono capítulo de Baba Metzia.

\paragraph{Não raptar um israelita}

Por esta proibição somos proibidos de raptar um israelita. Ela está
expressa em Suas palavras, nos Dez Mandamentos, ``Não furtarás'' (Êxodo
20:15), sobre as quais diz a Mekhiltá: ```Não furtarás': esta é a
proibição de raptar''. A Guemará de Sanhedrin diz: ``De que forma
deduzimos a proibição de raptar? Rabi Yoshiá disse: De `Não furtarás';
Rabi Yohanan disse: De `Não se venderão como são vendidos os servos'
(Levítico 25:42). Mas não há disputa entre eles; um Sábio determina a
proibição de roubar, e o outro Sábio a proibição de vender'', já que
não se impõe o castigo a menos que o transgressor rapte e venda
e a penalidade por infringir essas duas proibições é o estrangulamento.
Suas palavras, enaltecido seja Ele, são ``Aquele que roube um homem e
o vender, e for encontrado em sua mão, será certamente morto'' (Êxodo 21:16).

As normas deste preceito estão explicadas no décimo primeiro capítulo de Sanhedrin.
  

\paragraph{Não furtar dinheiro}

Por esta proibição somos proibidos de furtar dinheiro. Ela está
expressa em Suas palavras ``Não furtareis'' (Levítico 19:11), sobre as
quais a Mekhiltá diz: ```Não furtareis': esta é a proibição, de furtar
dinheiro''.

Aquele que infringir esta proibição deverá, de acordo com o que está
prescrito nas Escrituras, fazer uma restituição dobrada, ou
quadruplicada ou quintuplicada\footnote{Êxodo 21:37.}, ou simplesmente
devolver o que furtou.

A Sifrá diz: ``Pelas palavras, relativas ao furto, `Pagará o dobro'
(Êxodo 22:3), nós sabemos qual é a penalidade; mas como ficamos sabendo
da proibição? Pelas palavras das Escrituras `Não furtareis', mesmo que
o objetivo seja aborrecer''', ou seja, mesmo que o objetivo seja irritar
o proprietário e causar-lhe problemas e devolvê-lo a ele depois. ```Não
furtareis' nem com o objetivo de pagar uma restituição quadruplicada ou
quintuplicada''.

As normas deste preceito estão explicadas no sétimo capítulo de Baba Kamma.

\paragraph{Não cometer um roubo}

Por esta proibição somos proibidos de cometer um roubo, isto é, de tomar
abertamente pela força e violência qualquer coisa a que não tenhamos
direito. Ela está expressa em Suas palavras, enaltecido seja Ele, ``Não
extorquirás'' (Levítico 19:13). A Tradição também explica: ```Não
extorquirás (tigzol)': é como na expressão `Ele ``arrancou'' (vayigzol)
a lança das mãos dos Egípcios'''\footnote{II Samuel 23:21.}.

Este preceito negativo está justaposto ao preceito positivo, expresso
em Suas palavras ``Devolverá o que roubou'' (Levítico
5:23)\footnote{Ver o preceito positivo 194.}; mas mesmo que ele elimine o preceito
positivo, ele não será punido com o açoitamento, uma vez que um homem
não pode ser castigado com ambos o açoitamento e a restituição, já que este é um preceito negativo sujeito a restituição.

Portanto, se o ladrão queimar o artigo roubado ou o jogar no mar, ele
deverá pagar o seu valor; e.se ele negar a culpa sob juramento ele deve
acrescentar um quinto e deve levar um Sacrifício de Delito, como está
explicado no lugar apropriado\footnote{Ver o preceito positivo 71.}. Esta é a explicação
dada no final de Macot.

As normas deste preceito estão explicadas no nono e no décimo capítulos de Baba Kamma.

\paragraph{Não alterar os limites das terras fraudulentamente}

Por esta proibição somos proibidos de alterar os limites das terras
fraudulentamente, isto é, deslocar as marcas entre nossas terras e as do
vizinho, de maneira a reivindicar como nossas as terras de outro. Esta
proibição está expressa em Suas palavras, enaltecido seja Ele, ``Não
removerás o limite da herança, de teu companheiro'' (Deuteronômio
19:14); a esse respeito o Sifrei diz: ```Não removerás o limite da
herança de teu companheiro'; mas já não foi dito `Não extorquirás teu
próximo'? (Levítico 19:13). Então por que acrescentar `Não removerás'?
Para ensinar-nos que aquele que usurpar as terras de seu vizinho
transgride dois preceitos. Eu poderia pensar que isto se aplica também
fora da Terra de Israel, por isso as Escrituras dizem: `Na herança que
herdares'. Na Terra de Israel desobedece-se a dois preceitos negativos,
mas fora dela a apenas um", que é o ``Não extorquirás teu próximo''.
Fica assim demonstrado que este preceito negativo se aplica apenas à
Terra de Israel.

\paragraph{Não usurpar nossas dívidas}

Por esta proibição somos proibidos de usurpar as dívidas que fizermos,
isto é, recusar-nos a pagá-las e ficar com seu valor. Ela está expressa
em Suas palavras, enaltecido seja Ele, ``Não sonegarás teu próximo e não
extorquirás'' (Levítico 19:13).

``Furtar'' é o ato de tirar os pertences de alguém por meios
traiçoeiros e secretos, e está proibido por Suas palavras, ``Não
furtareis'' (Ibid., 19:11), como explicamos. A ``extorsão'' é o ato de
tirar os pertences de alguém por meio de força e violência, como fazem
os ladrões nas estradas, e está proibido por Suas palavras ``Não
extorquirás'' (Ibid.,13). A ``sonegação'' ocorre quando você deve uma
determinada importância a alguém, quer dizer, você tem em seu poder e é
responsável por uma quantia de dinheiro que pertence a outra pessoa, e
não a devolve a ela, utilizando-se de força ou apenas de subterfúgios e
trapaças. Tal conduta também está proibida pelas Suas palavras,
enaltecido seja Ele, ``Não sonegarás teu próximo'' que são explicadas da
seguinte forma na Sifrá: ```Não sonegarás teu próximo' significa
enganá-lo com relação a dinheiro, como por exemplo, reter o salário de
um empregado'' e todos os atos similares. O salário de um empregado é
dado como exemplo apenas porque ele representa uma dívida claramente
sua, embora ele não lhe tenha dado nenhum dinheiro dele e você não
tenha recebido dinheiro algum dele; apesar disso, como você tem uma
dívida certa para com ele, você está proibido de recusar-lhe o dinheiro.

A proibição relativa a este assunto está repetida, e este caso
específico foi dado como exemplo em Suas palavras ``Não defraudarás o
jornaleiro pobre e necessitado'' (Deuteronômio 24:14), que significam:
``Não oprimirás um empregado porque ele é pobre e necessitado'', como
Ele diz mais adiante: ``E isto o farás antes do pôr do sol, porque é
pobre'' (Ibid.,15).

O Sifrei diz: ```Não defraudarás o jornaleiro pobre e necessitado'; já
não foi dito `Não extorquirás'? É para ensinar-nos que aquele que
retiver o salário de um empregado transgredirá `Não defraudarás', `Não
extorquirás' e `Não ficará a paga de um jornaleiro contigo' (Ibid.) e
`No seu dia, lhe pagarás a sua diária''' (Ibid.)\footnote{Ver o preceito positivo 200.}. E acrescenta, interpretando Suas
palavras ``pobre e necessitado'': ``Eu puno mais rapidamente quando se
trata dos pobres e dos necessitados''.

A lei é a mesma para aquele que engana como para aquele que extorque;
Ele diz, enaltecido seja Ele, ``E negar ao seu companheiro a coisa que
lhe foi entregue sob custódia, ou um empréstimo em dinheiro, ou roubou,
ou extorquiu o seu companheiro (etc.)'' (Levítico 5:21).

\paragraph{Não negar nossas dívidas}

Por esta proibição somos proibidos de negar nossas dívidas ou aquilo
que tenha sido confiado a nós. Ela está expressa em Suas palavras ``Não
enganareis'' (Levítico 19:11) que se referem, como está explicado, a
transações de dinheiro.

A Sifrá diz: ``Pelas palavras `E negou, e jurou em falso (etc.)'''
(Ibid., 5:22), ficamos sabendo do castigo. De que forma ficamos sabendo
da proibição? Pelas palavras das Escrituras ``Não enganareis''.

Vocês já sabem que aquele que negar algo que ele tiver em depósito está
desqualificado como testemunha, mesmo que ele não o tenha feito sob
juramento, porque ele terá transgredido Suas palavras, enaltecido seja
Ele, `Não enganareis'.

As normas deste preceito estão explicadas em vários trechos do Tratado
Shebuot.

\paragraph{Não jurar em falso ao negar uma dívida}

Por esta proibição somos proibidos de jurar em falso ao negar uma dívida
nossa. Ela está expressa em Suas palavras ``E não mentireis cada um ao
seu companheiro'' (Levítico 19:11). Por exemplo: aquele que negar
falsamente ter recebido algo em custódia transgride Seu preceito,
abençoado seja Ele, ``Não enganareis'' (Ibid.); se a isso ele
acrescentar um falso juramento, ele transgride o preceito ``Não
mentireis''.

A Sifrá diz: ```Não mentireis cada um ao seu companheiro': qual é a
finalidade disto? Pelas palavras ``E jurou em falso, (etc.)'' (Ibid.,
5:22) conhecemos o castigo; mas como ficamos sabendo da proibição? Pelas
palavras das Escrituras: ``Não mentireis''.

As normas deste preceito estão explicadas no quinto capítulo de
Shebuot, onde fica claro que aquele que jurar um falso ao negar uma
dívida transgride dois preceitos negativos: ``Não jurareis falso em Meu
nome'' (Ibid., 19:12) e ``Não mentireis cada um ao seu companheiro''.

\paragraph{Não enganar um ao outro em negócios}

Por esta proibição somos proibidos de enganar um ao outro em negócios
de compra e venda. Ela está expressa em Suas palavras, enaltecido seja
Ele, ``Quando fizerdes uma venda a vosso companheiro, ou comprardes da
mão de vosso companheiro, não enganareis cada qual ao seu irmão'' (Levítico
25:14), sobre as quais diz a Sifrá: ```Não enganareis cada qual ao seu
irmão': enganar em termos de dinheiro''.

As normas deste preceito estão explicadas no quarto capítulo de Baba Metzia.

\paragraph{Não prejudicar um ao outro com palavras}

Por esta proibição somos proibidos de prejudicar um ao outro com
palavras, isto é, dizendo ao outro coisas que vão feri-lo e humilhá-lo e
causar-lhe dor insuportável, tal como fazer uma pessoa lembrar dos erros
que cometeu na juventude, pelos quais ela já tenha se arrependido, e
dizer-lhe: ``Agradeça aos Céus por tê-lo desviado de tal e tal
procedimento para conduzi-lo a seu atual estilo virtuoso de vida''; ou
como fazer referências cruéis a respeito de seus defeitos físicos
graves. Esta proibição está expressa em Suas palavras, enaltecido seja
Ele, "Não enganareis cada um ao seu companheiro; e temerás a teu Deus"
(Levítico 25:17), a respeito das quais diz o Talmud: ``Isto se refere a
enganar com palavras'', e a Sifrá: ``As palavras das Escrituras `Não
enganareis cada um ao seu companheiro' se referem a enganar com
palavras. O que isto significa? Se um homem se corrigiu, não se lhe
deve dizer `Lembre-se de seu antigo comportamento'. Se um homem estiver
sofrendo, se ele estiver doente ou tiver enterrado seu filho, não se
lhe deve dizer o que o companheiro de Yob disse: `Não é teu temor a Deus
a tua confiança, e tua esperança a totalidade de teus caminhos? Recorda,
eu te suplico, quem pereceu, sendo inocente'. Se virmos arrieiros em
busca de grãos não devemos dizer-lhes `Tal pessoa vende grãos' se
soubermos que ela nunca os vendeu. Também não se deve dizer ao
proprietário, quando não se tiver dinheiro, `Quanto custa este
artigo?'''. O Talmud diz ainda: ``O prejuízo verbal é mais odioso do que
o prejuízo monetário porque está escrito, a respeito do prejuízo
verbal: `Temerás a teu Deus'''.

As normas deste preceito estão explicadas no quarto capítulo de Baba Metzia.

\paragraph{Não enganar um prosélito com palavras}

Por esta proibição somos proibidos de enganar um prosélito com palavras.
Ela está expressa em Suas palavras, enaltecido seja Ele. ``Ao imigrante
não o fraudareis'' (Êxodo 22:20), sobre as quais diz a Mekhiltá: ```Ao
imigrante não o fraudareis': com palavras''.

A proibição relativa a este assunto está repetida em Suas palavras ``Não
o enganareis'' (Levítico 19:33), que a Sifrá explica assim: ``Não lhe
diga: `Ontem você adorava ídolos e agora você está sob a proteção da
``Shekhiná'' (Presença Divina)'''.

\paragraph{Não enganar um prosélito nos negócios}

Por esta proibição somos proibidos de enganar um prosélito e
prejudicá-lo numa compra e venda. Ela está expressa em Suas palavras
``Não o oprimireis'' (Êxodo 22:20), sobre as quais diz a Mekhiltá: ```Não o
oprimireis' --- no que se refere a dinheiro''.

Está explicado na Guemará de Baba Metzia que aquele que engana um
prosélito transgride ``Não enganareis cada um ao seu companheiro''
(Levítico 25:17) e ``Não o fraudareis'' (Êxodo 22:20). E aquele que o
engana em assuntos de dinheiro transgride ``Não o oprimireis'', além da
proibição na qual ele está incluído juntamente com todos os israelitas,
ou seja, a de enganar em assuntos de dinheiro.

\paragraph{Não entregar um escravo fugitivo}

Por esta proibição somos proibidos de entregar a seu senhor um escravo
que foi buscar refúgio na Terra de Israel. Mesmo que seu senhor seja um
israelita, como ele fugiu do exterior para a Terra de Israel ele não
deve ser entregue, e o senhor deve libertá-lo, recebendo dele uma nota
de reconhecimento de dívida pelo seu valor. Esta proibição está
expressa em Suas palavras, enaltecido seja Ele, ``Não entregarás ao seu
senhor o escravo'' (Deuteronômio 23:16). E está explicado no quarto
capítulo de Guitin que as Escrituras se referem aqui a um escravo que
tenha fugido do exterior em busca de refúgio na Terra de Israel e que a
lei diz que ele deve escrever ao seu senhor uma nota de reconhecimento
de dívida por seu valor, e que o senhor deve redigir-lhe um Estatuto de
Liberdade. Em circunstância alguma ele retornará a ser seu escravo pois
ele foi residir na terra pura, que foi escolhida para as pessoas
enaltecidas.

As normas deste preceito estão explicadas ali.

\paragraph{Não enganar um escravo fugitivo}

Por esta proibição somos proibidos de enganar o escravo que fugiu e veio
a nós. Ela está expressa em Suas palavras, enaltecido seja Ele,
``Contigo ficará no meio de ti\ldots{} não o enganarás'' (Deuteronômio
23:17), a respeito das quais diz novamente a Sifrá: ```Não o enganarás'
se refere a enganá-lo com palavras''. Pois assim como o Enaltecido
acrescentou uma proibição contra enganar um prosélito por causa de sua
condição de pessoa infeliz e sem amigos, Ele acrescentou também uma
terceira proibição contra enganar um escravo, que é ainda mais infeliz e
humilde que o prosélito, para que vocês não digam: Este escravo não se
incomodará se eu o enganar com palavras.

Está claro que o escravo referido aqui pelas Escrituras e o prosélito a
quem somos proibidos de enganar são pessoas que aceitaram a Torah como
sua lei, quer dizer, são Prosélitos Virtuosos.

\paragraph{Não ser rude com crianças órfãs e com viúvas}

Por esta proibição somos proibidos de tratar de maneira rude as
crianças órfãs e as viúvas. Ela está expressa em Suas palavras,
enaltecido seja Ele, ``A nenhuma viúva ou órfão afligireis'' (Êxodo
22:21).


Esta proibição inclui o tratamento áspero por palavras ou atos. Devemos
falar com eles de maneira muito gentil e amável, tratá-los tão bem
quanto nos for possível, mostrar-lhes nossa boa vontade em relação a
eles e estabelecer para nós mesmos um padrão superior para lidar com
todos estes assuntos. Todo aquele que violar qualquer uma dessas coisas
estará violando este preceito negativo, e o Enaltecido decretou
claramente seu castigo em Suas palavras: ``Acender-se-á a Minha ira, e
matar-vos-ei com a espada'' (Ibid., 23).

\paragraph{Não utilizar um servo hebreu para executar tarefas degradantes}

Por esta proibição somos proibidos de utilizar um escravo hebreu para
executar tarefas domésticas degradantes, como as que são executadas
pelos escravos cananeus. Ela está expressa em Suas palavras, enaltecido
seja Ele, ``Não o farás servir com serviço de escravo'' (Levítico
25:39). A Sifrá diz: ``Ele não deve levar um `belinta' atrás de você,
nem levar suas coisas antes de você ao banho''. Um ``belinta'' é uma
pequena esteira onde se senta ou se descansa quando se está cansado
depois de ter feito exercício físico e que o escravo leva para seu
senhor. É proibido impor qualquer tarefa de tal natureza a um escravo
hebreu, que só deve ser empregado em trabalhos como os que são
executados por um trabalhador contratado ou por um artesão, com o acordo
de seu empregador. Isto está estabelecido em Suas palavras ``Como
jornaleiro, como imigrante, estará contigo'' (Ibid., 40).

\paragraph{Não vender um servo hebreu em leilão}

Por esta proibição somos proibidos de vender um servo hebreu de maneira
como são vendidos os escravos, ou seja, pô-lo à venda em leilão no
mercado de escravos. Isso não deve ser feito em hipótese alguma; deve
ser feito em local fechado e em condições adequadas. A proibição
relativa a este assunto está expressa em Suas palavras, enaltecido seja
Ele, ``Não se venderão como são vendidos os servos'' (Levítico 25:42),
sobre as quais diz a Sifrá: ```Não se venderão como são vendidos os
servos': não se deve colocar uma plataforma e pô-lo na pedra de
leilão''.

Este preceito negativo inclui sem dúvida alguma a proibição de raptar
um israelita porque se alguém o vender, o fará como faria comum escravo
cananeu, transgredindo assim Suas palavras ``Não se venderão como são
vendidos os servos''. Nós já nos referimos a esse assunto
anteriormente. E a Torah deixa claro que quem o fizer será
morto\footnote{Êxodo 21:16.}.

As normas deste preceito e as dos precedentes estão explicadas no
primeiro capítulo da Guemará de Kidushin.


\paragraph{Não utilizar um servo hebreu para fazer um trabalho desnecessário}

Por esta proibição somos proibidos de utilizar um servo hebreu num
trabalho desnecessário, que é chamado de ``trabalho inclemente''. Esta
proibição está expressa em Suas palavras, enaltecido seja Ele, ``Não
dominarás sobre ele com rigor'' (Levítico 25:43). Não devemos fazê-lo
trabalhar a não ser quando somos forçados a isso pela necessidade de
que um trabalho específico seja feito. A Sifrá comenta: ```Não dominarás
sobre ele com rigor' significa que não lhe dirás `Aqueça-me esta bebida'
se isso não for necessário'', e coisas similares. O exemplo dado é a
menor e a mais simples tarefa, contudo nem mesmo isso é permitido a não
ser que seja necessário.

\paragraph{Não permitir que se maltrate um servo hebreu}

Por esta proibição somos proibidos de permitir que um pagão que more em
nossa terra seja severo com um servo hebreu que se vendeu a ele. Esta
proibição está expressa em Suas palavras, enaltecido seja Ele, ``Não
dominará sobre ele, com rigor, à tua vista'' (Levítico 25:53). Não
devemos dizer que, uma vez que esse hebreu pecou contra si mesmo e se
vendeu a um pagão, nós o deixaremos sofrer as consequências de seu ato.
Devemos controlar o pagão e evitar que ele seja severo. A Sifrá diz:
``Não dominará sobre ele com rigor, à tua vista'': o preceito se aplica
apenas quando ele estiver à tua vista. Quer dizer, não somos
obrigados a supervisioná-lo quando ele estiver em sua própria casa para
verificar se ele está sendo severo ou não; mas toda vez que o virmos
fazendo isso, devemos proibi-lo de agir assim.

\paragraph{Não vender uma serva hebreia}

Por esta proibição o proprietário de uma serva hebreia fica proibido de
vendê-la. Ela está expressa em Suas palavras, enaltecido seja Ele, ``Não
a poderá vender após ter-se servido dela'' (Êxodo 21:8).

As normas deste preceito estão explicadas em sua totalidade no início
do Tratado Kidushin.

\paragraph{Não privar uma serva hebreia que se desposou}

Por esta proibição o proprietário de uma serva hebreia que a tenha
desposado fica proibido de privá-la --- e por isso quero dizer diminuir
sua comida, roupas ou direitos conjugais --- de maneira tal a
causar-lhe dor e sofrimento. Esta proibição está expressa em Suas
palavras, enaltecido seja Ele, ``Sua manutenção, seu vestuário, e o seu
direito conjugal não lhe diminuirá'' (Êxodo 21:10).

Este preceito negativo também se aplica a todo aquele que se casar com
uma israelita para que não a prive de nenhuma dessas três coisas a fim
de causar-lhe dor e infelicidade. Por Suas palavras, enaltecido seja Ele,
relativas a uma serva hebreia, proibindo-nos de diminuir sua comida,
suas roupas ou seus direitos conjugais, e pelo fato de acrescentar
``Trata-la-á como se tratam as filhas'' (Ibid., 9) ficamos sabendo que
``Como se tratam as filhas'' significa que não devemos diminuir sua
comida, roupas ou direitos conjugais. Isso está estipulado na Mekhiltá:
``O que este texto nos ensina com relação a `como se tratam as filhas'?
Ele serve para esclarecer outro texto, mas na realidade ele é
autoexplicativo''. Também está dito ali: ```Sheerá.
significa seu alimento'; `quessutá', sua roupa, no sentido literal;
`onatá', seus direitos conjugais''.

\paragraph{Não vender uma prisioneira}

Por esta proibição somos proibidos de vender uma mulher formosa depois
de tê-la tomado, por ocasião da captura de uma cidade, como está
explicado no lugar apropriado. Esta proibição está expressa em Suas
palavras, enaltecido seja Ele, ``E se não a quiseres, a deixarás ir em
liberdade; e não a venderás por dinheiro'' (Deuteronômio 21:14).

\paragraph{Não escravizar uma prisioneira}

Por esta proibição somos proibidos de escravizar uma mulher formosa
depois de tê-la tomado; quer dizer, não se deve fazer dela sua escrava e
tratá-la como outras servas que fazem trabalhos vis. Esta proibição está
expressa em Suas palavras, enaltecido seja Ele, ``Não te servirás dela,
porque a afligiste'' (Deuteronômio 21:14), sobre as quais diz o Sifrei:
```Não te servirás dela' significa que não a `utilizarás'''.

Ficou assim claro que estes dois preceitos negativos proíbem duas coisas
diferentes: vendê-la a outra pessoa e guardá-la, tratando-a como
escrava. Deve-se observar Seu preceito, enaltecido seja Ele, ``A
deixarás ir em liberdade'' (Ibid.). Assim também está explicado o texto
relativo ao raptor: ``E se servir dele, e depois o vender'' (Ibid.,
24:7): ``Ele não incorre em nenhuma culpa a menos que o tenha sob seu
próprio controle e o ponha a seu serviço''.

As normas relativas à mulher formosa estão explicadas no início de Kidushin.

\paragraph{Não planejar obter a propriedade de outrem}

Por esta proibição somos proibidos de ocupar nossas mentes com planos
para obter o que pertence a um de nossos irmãos. Ela está expressa em
Suas palavras, enaltecido seja Ele, ``Não cobiçarás a casa de teu
próximo'' (Êxodo 20:17), sobre as quais diz a Mekhiltá: ```Não
cobiçarás': eu poderia pensar que se refere à simples expressão de um
desejo, por isso as Escrituras dizem `Não cobiçarás a prata e o ouro que
está sobre eles, nem os \emph{tomarás} para ti' (Deuteronômio 7:25).
Assim como naquele caso, neste também é apenas quando se coloca o
desejo em prática''.

Ficou dessa forma claro que este preceito negativo nos proíbe de
planejar obtermos algo que cobicemos e que pertença aos nossos irmãos,
mesmo que o compremos e que paguemos por ele seu preço real. Qualquer
ato desse tipo é uma desobediência a ``Não cobiçarás''.

\paragraph{Não cobiçar os pertences de outrem}

Por esta proibição somos proibidos de concentrar nossos pensamentos na
cobiça e no desejo de coisas que pertençam a outra pessoa, porque isso
levará a planejar obtê-las. As palavras a esse respeito são: ``E não
desejarás a casa do teu próximo'' (Deuteronômio 5:18).

Esses dois preceitos negativos não se referem ao mesmo assunto. O
primeiro, ``Não cobiçarás'' (Êxodo 20:17), proíbe obter efetivamente o
que pertence a outra pessoa; o segundo nos proíbe até mesmo desejá-lo e
cobiçá-lo. A Mekhiltá diz: ``Aqui está dito `Não \emph{cobiçarás} a casa
do teu próximo' e, mais adiante, `Não \emph{desejarás} a casa do teu
próximo'. Portanto, incorre-se na culpa por desejar apenas, bem como por
apenas cobiçar''. Também diz: ``Como sabemos que se alguém começar por
desejar, ele acabará por cobiçar? Porque as Escrituras dizem: `Não
cobiçarás.e não desejarás' (Deuteronômio 5:18). Como sabemos que se
alguém começar por cobiçar, ele acabará por roubar usando de violência?
Porque as Escrituras dizem: `Eles cobiçam campos, e se apoderam
deles'''\footnote{Micah 2:2.}.

A explicação disto é que se alguém vir um belo objeto que pertença a seu
irmão e se interessar por ele e passar a desejá-lo, ele estará
infringindo a proibição expressa em Suas palavras, enaltecido seja Ele,
``Não desejarás''. Então seu amor pelo objeto ficará cada vez mais forte
até que ele comece a arquitetar um plano para obtê-lo, e não cesse de
pedir e pressionar o proprietário para que o venda a ele ou o dê em
troca de algo melhor e mais valioso; se ele o conseguir, estará dessa
forma infringindo outro preceito, que é o ``Não cobiçarás'', pois
devido à sua persistência e ardis ele obteve algo que o proprietário não
desejava vender. Dessa forma ele infringiu dois preceitos, como
explicamos. Contudo, se o proprietário se recusar a vender ou trocar o
objeto, por sua grande estima por ele, e se aquele que o cobiçar, devido
a seu grande desejo de tê-lo, o tomar pela força e coação, ele também
transgredirá o preceito negativo ``Não extorquirás'' (Levítico 19:13).
Para compreender isto vocês deveriam ler a história do rei Ah-Ab e
Nabot\footnote{Reis 1, cap. 21.}.

Agora deve estar clara para vocês a diferença entre ``Não desejarás'' e
``Não cobiçarás''.

\paragraph{Um trabalhador contratado não pode comer das plantações em crescimento}

Por esta proibição um trabalhador contratado fica proibido de comer das
plantações em crescimento entre as quais ele esteja trabalhando. Ela
está expressa em Suas palavras, enaltecido seja Ele, ``Foice não porás
na seara de teu companheiro'' (Deuteronômio 23:26), sobre as quais diz o
Talmud: ```Foice': isto estende a lei a tudo o que necessita de foice
em época de cortar com a foice'', ou seja, na época da colheita não
deves colher para ti mesmo.

É sabido que este versículo se refere apenas a um trabalhador contratado e que Suas palavras ``Quando entrares\ldots{}'' (Ibid.) significam ``quando
um trabalhador entrar'', como o Targum o traduz: ``Quando fores
contratado''.

No sétimo capítulo de Baba Metzia lemos: ``Eles podem comer de acordo
com a lei das Escrituras: daquilo que está no solo, e para o qual ele
foi contratado, depois que o trabalho estiver terminado''.

As normas deste preceito estão explicadas nesse capítulo.

\paragraph{Um trabalhador contratado não pode servir-se em demasia}

Por esta proibição um trabalhador contratado fica proibido de pegar
mais das plantações nas quais ele estiver trabalhando do que aquilo que
ele necessitar para sua refeição. Ela está expressa em Suas palavras,
enaltecido seja Ele, ``Poderás comer uvas conforme teu desejo, até te
fartares, porém na tua bolsa não porás'' (Deuteronômio 23:25).

As normas deste preceito estão explicadas no sétimo capítulo de Baba
Metzia, onde também está explicado o que lhe é e o que não lhe é
permitido comer, e que ele não pode comer sem violar a proibição ``Na
tua bolsa não porás''.

\paragraph{Não ignorar uma propriedade perdida}

Por esta proibição somos proibidos de fechar nossos olhos a uma
propriedade perdida; devemos recolhê-la e devolvê-la a seu proprietário.
Esta proibição está expressa em Suas palavras, enaltecido seja Ele, "Não
farás como se não os visses" (Deuteronômio 22:3). Nós já citamos o que
a Mekhiltá diz com relação à propriedade perdida: ``Aprendemos assim que
se viola um preceito positivo e um preceito negativo''. E a Guemará diz:
``Devolver uma propriedade perdida apoia-se sobre um preceito positivo e
sobre um preceito negativo''.

Este assunto aparece novamente no Deuteronômio, onde há um preceito
negativo separado em Suas palavras ``Vendo o boi de teu irmão, ou o seu
cordeiro, extraviados, não farás como se não os visses'' (Ibid., 1),
sobre as quais diz o Sifrei: ```Vendo\ldots{}' é um preceito negativo''; e
diz ainda, mais adiante: ``Quando encontrares (Êxodo
23:4)\footnote{Ver o preceito positivo 204.} é um preceito positivo''.

As normas deste preceito estão explicadas no segundo capítulo de Baba Metzia.

\paragraph{Não abandonar uma pessoa sobrecarregada}

Por esta proibição somos proibidos de abandonar alguém que esteja
sobrecarregado e se atrase na estrada. Devemos ajudá-lo retirando dele
sua carga até que ele possa arrumá-la e devemos ajudá-lo a colocá-la nas
costas ou sobre seu animal, como está explicado nas normas deste
preceito. A proibição está expressa em Suas palavras ``Não te recusarás
a ajudá-lo'' (Êxodo 23:5), sobre as quais diz a Mekhiltá: ```Não te recusarás a ajudá-lo;
auxilia-lo-ás' nos ensina que se viola ambos um preceito
positivo\footnote{Ver o preceito positivo 202.} e um preceito negativo''.

Além disso há um preceito separado a esse respeito em Suas palavras no
Deuteronômio ``Vendo o jumento de teu irmão'' (Deuteronômio 22:4), sobre
as quais diz o Sifrei: ```Vendo o jumento de teu irmão' é um preceito
negativo''; e diz mais adiante: ```Quando vires o asno daquele que te
aborrece' é um preceito positivo''.

As normas deste preceito também estão explicadas no segundo capítulo de
Baba Metzia.

\paragraph{Não trapacear na medida e nos pesos}

Por esta proibição somos proibidos de trapacear ao medir a terra ou de
usar medidas e pesos incorretos. Ela está expressa em Suas palavras
``Não fareis iniquidade no juízo, nem na medida de comprimento, nem no
peso, e nem na medida de capacidades'' (Levítico 19:35), que a tradição
explica como significando ``Não fareis iniquidade ao medir''. E a Sifrá
diz, ao explicar este preceito negativo: ```Não fareis iniquidade no
juízo': se isto se refere ao proferir um julgamento, isso já foi dito.
Então porque diz aqui `no juízo'? Para ensinar-nos que aquele que mede é
chamado de juiz'''.

Também está dito ali: ```Na medida de comprimento' se refere à medição
de terras'', ou seja, a medição e o cálculo devem ser feitos de acordo
com as rigorosas leis da matemática, com precisão e com conhecimento dos
métodos corretos; não devemos utilizar suposições sem base, como faz a
maioria dos funcionários.

``Peso'' inclui ambos, pesos e balanças.

\paragraph{Não manter pesos e medidas incorretos}

Por esta proibição somos proibidos de manter pesos e medidas incorretos
em nossas casas, ainda que não os utilizemos para fins comerciais. Ela
está expressa em Suas palavras, enaltecido seja Ele, ``Não terás no teu
bolso pesos diversos um grande e um pequeno'' (Deuteronômio 25:13) e
isso se aplica também a diversas medidas. Como diz a Guemará de Baba
Batra: ``Uma pessoa está proibida de manter em sua casa uma medida muito
pequena ou muito grande, ainda que seja para a coleta de urina''.

Vocês não devem concluir, por Suas palavras ``Não terás diversas
medidas'' (Ibid. 25:14) e ``Não terás diversos pesos'', que estes são
dois preceitos separados. O objetivo das duas proibições é completar as
normas do preceito para que elas cubram os dois tipos de medidas, a de
peso e a de tamanho. Portanto, é como se Ele tivesse dito: ``Não terás
dois padrões de peso nem de medida'', como explicamos com relação ao
preceito positivo\footnote{Ver o preceito positivo 208.} Seu preceito ``Não terás pesos
diversos\ldots{} Não terás diversas medidas'' é uma simples proibição que
inclui vários casos, todos regidos pela mesma lei, tal como acontece no
preceito ``Não pagarás a teu irmão juro de dinheiro, nem juro de comida,
nem juro de coisa alguma que se dá como juro'' (Deuteronômio 23:20).
Proibições repetidas da mesma coisa não devem ser contadas como
preceitos separados, como explicamos na Introdução, no Nono Fundamento.
Nós já demos um exemplo disto no preceito negativo 200, que está
expresso em Suas palavras ``Pães ázimos serão comidos sete dias e não
será vista por ti coisa levedada'' (Êxodo 13:7).

\paragraph{Um juiz não pode cometer injustiças}

Por esta proibição um juiz fica proibido de cometer injustiças num
julgamento. Ela está expressa em Suas palavras ``Não fareis injustiça no
juízo'' (Levítico 19:15). O significado deste preceito é que não se deve
afastar-se dos princípios que a Torah estabeleceu com uma condenação ou
uma absolvição.

\paragraph{Um juiz não pode aceitar presentes de uma das partes}

Por esta proibição um juiz fica proibido de aceitar um presente das
partes, mesmo que seja para que ele proceda a um julgamento justo. Ela
está expressa em Suas palavras ``E suborno não tomes'' (Êxodo 23:8). A
proibição relativa a este assunto está repetida em outro lugar. O Sifrei
diz: `` `Não tomarás suborno' (Deuteronômio 16:19) --- nem mesmo para
absolver o inocente e condenar o culpado''.

As normas deste preceito estão explicadas em vários trechos de Sanhedrin.

\paragraph{Um juiz não pode proteger uma das partes}

Por esta proibição um juiz fica proibido de proteger um dos litigantes
num julgamento. Mesmo que ele seja um homem de alta posição e de grande
distinção, o juiz não deve reverenciá-lo se ele aparecer diante dele
juntamente com a outra parte, nem tratá-lo com deferência e respeito.
Esta proibição está expressa em Suas palavras, enaltecido seja Ele,
``Nem honrarás as faces do poderoso'' (Levítico 19:15), a respeito das
quais diz a Sifrá: ``Não dirás `Este é um homem rico, de uma família
ilustre; como posso envergonhá-lo e testemunhar seu embaraço?'
Certamente não o envergonhará\footnote{A pessoa que pensa de tal forma certamente não o envergonhará.}; e é por essa razão
que as Escrituras dizem: `Nem honrarás as faces do poderoso'''.

As normas deste preceito estão explicadas em vários trechos de Sanhedrin e de Shebuot.

\paragraph{Um juiz não pode acovardar-se com medo de pronunciar um julgamento justo}

Por esta proibição um juiz fica proibido de acovardar-se com medo de
pronunciar um julgamento justo contra um malfeitor inclemente e
perverso. E seu dever pronunciar o julgamento sem pensar nos danos que
o malfeitor pode causar-lhe. Suas palavras, enaltecido seja Ele, são:
``Não temereis a homem algum'' (Deuteronômio 1:17), sobre as quais o
Sifrei diz: `` `Não temereis a homem algum'; para que você não diga
`Tenho medo deste homem porque ele pode matar meu filho, ou queimar meu
trigo, ou destruir minha plantação', as Escrituras dizem `Não temereis a
homem algum'''.

\paragraph{Um juiz não pode decidir em favor de um homem pobre por piedade}

Por esta proibição um juiz fica proibido de ter piedade de um homem
pobre e distorcer um julgamento em seu favor por piedade. Ele deve
tratar os ricos e os pobres da mesma forma, e fazer com que se cumpra a
pena imposta. Esta proibição está expressa em Suas palavras ``Ao pobre
não favorecerás em sua briga'' (Êxodo 23:3).

O preceito negativo relativo a este assunto se encontra novamente em
Suas palavras ``Não favorecerás as faces do mendigo'' (Levítico 19:15),
a respeito das quais diz a Sifrá: ``Para que você não diga: `Este é um
homem pobre é como eu e este homem rico somos obrigados a sustentá-lo,
vou sentenciar em seu favor e assim permitir-lhe viver sem perder o
respeito por si mesmo', as Escrituras dizem: `Ao pobre não favorecerás
em sua briga'''.

\paragraph{Um juiz não pode distorcer um julgamento contra uma pessoa de má reputação}

Por esta proibição um juiz fica proibido de distorcer um julgamento em
detrimento de uma das partes que ele saiba ser um pecador perverso. O
Enaltecido nos proíbe de punir tal homem distorcendo seu julgamento
através de Suas palavras, enaltecido seja Ele, ``Não perverterás o
julgamento de teu indigente em sua causa'' (Êxodo 23:6). A esse
respeito a Mekhiltá diz: ``Para que você não diga, num caso entre um
homem mau e um homem honesto, `Como este homem é mau vou distorcer o
julgamento contra ele' as Escrituras dizem: `Não perverterás o
julgamento de teu \emph{indigente} em sua causa' --- significando
`indigente' no que se refere a boas ações'', isto é, mesmo que ele seja
pobre de boas ações você não deve distorcer o julgamento contra ele.

\paragraph{Um juiz não pode ter piedade de alguém que matou um homem}

Por esta proibição um juiz fica proibido de apiedar-se de alguém que
matou um homem ou lhe causou a perda de um membro, ao estabelecer a
pena. Ele não pode dizer: ``Este é um pobre homem que cortou a mão do
outro ou o cegou de um olho sem querer'' e assim ter compaixão dele e
ser indulgente ao avaliar a importância dos prejuízos. Esta proibição
está expressa em Suas palavras ``Alma por alma, olho por olho''
(Deuteronômio 19:21). Este preceito negativo aparece novamente em Suas
palavras ``Não o olharás com piedade e vingarás o sangue inocente de
Israel'' (Ibid.,13).

\paragraph{Um juiz não pode distorcer a justiça por prosélitos ou órfãos}

Por esta proibição um juiz fica proibido de distorcer a justiça devido
a prosélitos ou órfãos. Ela está expressa em Suas palavras ``Não
perverterás o juízo do imigrante e do órfão'' (Deuteronômio 24:17).

Já lhes foi explicado que aquele que distorcer a justiça por causa de um
israelita transgride o preceito negativo expresso em Suas palavras ``Não
fareis injustiça no juízo'' (Levítico 19:15); mas aquele que distorcer
a justiça por causa de um prosélito transgride dois preceitos negativos,
como diz o Sifrei: ```Não perverterás o juízo do imigrante' nos ensina
que aquele que distorcer a justiça devido a um prosélito violará dois
preceitos negativos''. E se o prosélito for órfão, ele transgredirá
três.

\paragraph{Um juiz não pode ouvir uma das partes na ausência da outra}

Por esta proibição um juiz fica proibido de ouvir os argumentos de uma
das partes quando a outra não estiver presente. Ela está expressa em
Suas palavras ``Não dês ouvido à maledicência'' (Êxodo 23:1). Já que na
maioria dos casos o que um dos litigantes alega enquanto o outro não
está presente é incorreto, o juiz fica proibido de ouvi-lo para que ele
não tenha uma visão inexata e falsa do caso. A Mekhiltá diz: `` `Não dês
ouvido à maledicência' proíbe um juiz de ouvir uma das partes até que a
outra também esteja presente e proíbe uma parte de apresentar seu caso
ao juiz até que a outra também esteja presente''. E para proibir tal
conduta que Ele diz: ``Da palavra falsa afasta-te'' (Ibid.,7), como está
explicado no quarto capítulo de Shebuot.

De acordo com o Talmud, este preceito negativo também proíbe caluniar,
ou ouvir uma calúnia, ou prestar um falso testemunho, como está
explicado em Macot.

\paragraph{Um tribunal não pode condenar por maioria de um num caso capital}

Por esta proibição um tribunal fica proibido de condenar por maioria de
um. O significado disto é que se houver uma divisão na opinião dos
juízes, ficando alguns a favor da pena de morte e outros não, e se
houver maioria de apenas um em favor da condenação, não se permite que
se mate o pecador, pois o Eterno proibiu o Tribunal de fazê-lo a menos
que a maioria favorável à condenação seja de dois. Esta proibição está
expressa em Suas palavras ``Não concorras com a maioria para que alguém
seja condenado'' (Êxodo 23:2); quer dizer, ao sentenciar a pena de morte
você não deve fazê-lo por causa de uma maioria casual. Este é o
significado da expressão restritiva ``para que alguém seja condenado''.
Como a Mekhiltá diz, ``Se onze forem a favor da absolvição e doze a
favor da condenação, eu poderia pensar que o veredito é de culpa, por
isso as Escrituras dizem: `Não concorras com a maioria para que alguém
seja condenado'''. Também está dito ali: ``Um veredito de absolvição
precisa da maioria de um, mas um veredito de condenação precisa de uma
maioria de dois''.

As normas deste preceito estão explicadas no quarto capítulo de Sanhedrin.

\paragraph{Um juiz não pode confiar na opinião de outro juiz}

Por esta proibição um juiz fica proibido de confiar na opinião de outro
juiz ao condenar o culpado ou absolver o inocente sem que ele próprio
tenha examinado o assunto baseado em sua própria investigação e deduções
dos princípios da lei. Esta proibição está expressa em Suas palavras
``Não declares numa causa de forma a desviar-te (de acordo com a
opinião da maioria)'' (Êxodo 23:2), cujo significado é: no caso de uma
polêmica, seu objetivo não deve ser apenas adotar uma opinião, seguir a
opinião da maioria ou dos juízes superiores e abolir seu próprio ponto
de vista sobre o assunto. A Mekhiltá diz: `` `Não declares\ldots{}'. Não
digas, quando forem contados\footnote{Quando os votos forem contados.}, `Basta que eu siga tal pessoa'; dê a sua própria opinião. Poder-se-ia pensar que a mesma
lei se aplica a casos não econômicos, por isso as Escrituras dizem: `de
maneira a desviar-te, de acordo com a opinião da maioria'''.

Deste preceito também se deduz a proibição que impede um juiz de
argumentar em favor de uma condenação quando ele próprio já tiver se
declarado em favor da absolvição, proibição essa expressa em Suas
palavras, enaltecido seja Ele, ``Não declares numa causa de forma a
desviar-te'', isto é, não te desvies para mudar o veredito para a
condenação.

Da mesma forma, um caso capital não pode ser aberto pela condenação
porque as Escrituras dizem: ``Não declares numa causa de forma a
desviar-te''. E esse mesmo texto também é a base das regras de que um
veredito de condenação pode ser revertido, mas não um veredito de
absolvição, e que não devem começar pelo mais
velho\footnote{A leitura do veredito não deve começar pelo juiz mais velho.}, como foi deixado claro no quarto capítulo
de Sanhedrin, onde estão explicadas as normas deste preceito.

\paragraph{Não designar um juiz inculto}

Por esta proibição fica proibido ao Grande Tribunal e ao Exilarca
designar, por causa de suas outras qualidades, um juiz que não seja
versado na sabedoria da Torah. Eles ficam proibidos de fazer isso e ao
fazer nomeações para a Torah\footnote{Ao nomear as pessoas que devem apresentar decisões em questões da Torah.}, eles devem basear-se
apenas na sabedoria que o homem tem da Torah, em seus conhecimentos
sobre seus preceitos e proibições e em sua conduta absolutamente
irrepreensível. O preceito que proíbe que uma pessoa seja designada em
virtude de outras qualidades está expressa em Suas palavras, enaltecido
seja Ele, ``Não conheçais faces no juízo'' (Deuteronômio 1:17), sobre
as quais o Sifrei diz: ``Não conheçais faces no juízo se refere a alguém
cuja função seja designar juízes''. Quer dizer, a proibição é dirigida
apenas àquele que tem o direito de nomear juízes para os israelitas e o
proíbe de fazê-lo baseado em qualquer uma das razões que mencionamos. O
Sifrei acrescenta: ``Você não deve dizer: `Vou nomear tal pessoa porque
ele é simpático, ou porque é rico, ou porque é meu parente, ou porque
ele me emprestou dinheiro, ou porque ele conhece muitos idiomas'. O
resultado será que ele absolverá o culpado e condenará o inocente não
porque ele seja mau, mas porque lhe falta a sabedoria. É por essa razão
que as Escrituras dizem: `Não conheçais faces no juízo'''.

\paragraph{Não prestar um falso testemunho}

Por esta proibição somos proibidos de prestar falsos testemunhos. Ela
está expressa em Suas palavras ``Não darás falso testemunho (ed shaker)
contra teu próximo'' (Êxodo 20:16), e aparece novamente sob outra
forma, nas palavras ``Não darás falso testemunho contra o teu próximo''
(Deuteronômio 5:17). Contra aquele que transgredir este preceito
negativo as Escrituras decretam que ``Fareis a eles como pensavam fazer
a seu irmão'' (Ibid., 19:19). E a Mekhiltá diz: ```Não darás falso
testemunho' é uma advertência contra testemunhas que tencionem causar
danos''.

A desobediência a este preceito também acarreta o açoitamento, como foi
deixado claro no início do Tratado Macot, onde estão explicadas as
normas deste preceito.

\paragraph{Um juiz não pode aceitar o testemunho de um homem mau}

Por esta proibição um juiz fica proibido de aceitar o testemunho de um
homem mau e agir levando seu testemunho em consideração. Ela está
expressa em Suas palavras, enaltecido seja Ele, ``Não acompanhes o mau
para servir de falso testemunho'' (Êxodo 23:1), que a Tradição explica
assim: ``Não deixes que o mau testemunhe e não deixes que o injusto
testemunhe; dessa forma os injustos e os ladrões ficam excluídos de ser
testemunhas'', de acordo com o versículo ``Não se levantarão testemunhos
injustos contra alguém (etc.)'' (Deuteronômio 19:16).

As normas deste preceito estão explicadas no terceiro capítulo de Sanhedrin.

\paragraph{Um juiz não pode aceitar o testemunho de um parente de uma das partes}

Por esta proibição um juiz fica proibido de admitir o testemunho de
parentes, seja a favor ou contra ele. Ela está expressa em Suas
palavras, enaltecido seja Ele, ``Não se fará morrer os pais pelos
filhos, nem os filhos pelos pais'' (Deuteronômio 24:16), cuja explicação
Tradicional se encontra no Sifrei: ``Pais não deverão morrer pelo
testemunho de seus filhos nem os filhos pelo testemunho de seus pais''.

A mesma lei se aplica a casos envolvendo reivindicações de dinheiro,
mas as Escrituras a estabeleceram, de uma forma hiperbólica,
referindo-se a casos capitais, para ensinar-nos que não devemos
reciocinar da seguinte forma: ``Como esta acusação envolve a pena de
morte, não devemos duvidar da veracidade do testemunho de um parente, e
sim devemos agir de acordo com ele, já que o caso envolve a morte de seu
parente, não deixando lugar para suspeita''. E por essa razão que as
Escrituras destacam como exemplo o laço mais forte e mais profundo de
afeição que é o amor de um pai por seu filho e de um filho por seu pai;
os Sábios dizem que estamos proibidos de aceitar até mesmo o testemunho
de um pai contra seu filho, mesmo que isso o condene à morte, e que isso
é um decreto das Escrituras para o qual não há razão, seja ela qual for.
Vocês devem compreender isso.

As normas deste preceito estão explicadas no terceiro capítulo de Sanhedrin.

\paragraph{Não condenar baseado no depoimento de uma única testemunha}

Por esta proibição somos proibidos de infligir uma punição ou uma multa
baseados no depoimento de uma única testemunha, mesmo que esta seja
digna de toda a confiança. Ela está expressa em Suas palavras,
enaltecido seja Ele, ``Não valerá uma testemunha contra um homem por
qualquer delito ou por qualquer pecado'' (Deuteronômio 19:15), a
respeito das quais o Sifrei diz: ``Não valerá por qualquer delito, mas
valerá por um juramento''.

As normas deste preceito estão explicadas em diversos trechos de
Yebamot, Quetubot, Sota, Guitin e Kidushin, e em vários trechos da Ordem
Nezikin.

\paragraph{Não matar um ser humano}

Por esta proibição somos proibidos de matar-nos uns aos
outros. Ela está expressa em Suas palavras ``Não matarás'' (Êxodo
20:13), e todo aquele que violar este preceito negativo será decapitado.
O Enaltecido diz: ``Do meu altar o tirarás, para que morra'' (Ibid., 21:14).

As normas deste preceito estão explicadas no nono capítulo de Macot
Sanhedrin, e no segundo capítulo de Macot.


\paragraph{Não punir com a pena capital baseando-se em provas circunstanciais}

Por esta proibição somos proibidos de executar uma sentença baseando-se
numa forte suspeita, mesmo que ela seja quase conclusiva. Assim, se um
homem perseguir seu inimigo com a intenção de matá-lo e o perseguido se
refugiar numa casa, seguido pelo perseguidor, e se ao entrarmos depois
deles encontrarmos o homem perseguido dando seu último suspiro e seu
inimigo, o perseguidor, junto a ele com uma faca na mão, estando ambos
ensanguentados, o perseguidor não deve ser condenado à morte pelo
Tribunal, no desempenho da justiça já que não há testemunhas para depor
que eles viram o assassinato ser cometido. A Verdadeira Torah proíbe que
se condene um homem à morte através de Suas palavras, enaltecido seja
Ele, ``O inocente e o justo não mates, pois não justificarei ao mau''
(Êxodo 23:7).

A Mekhiltá diz: ``Suponha que eles vejam alguém perseguindo seu
companheiro com a intenção de matá-lo e que eles lhe façam uma
advertência legal dizendo: O homem é um israelita, um filho da Aliança;
se você o matar você será morto; e que depois disso eles o percam de
vista e mais tarde encontrem o outro dando seu último suspiro, e sangue
pingando da espada na mão do perseguidor, eu poderia pensar que ele
deveria ser declarado culpado. Por isso as Escrituras dizem: `O inocente
e o justo não mates'''.

Não deixem que isto os confunda nem pensem que a lei é injusta. Entre os
acontecimentos que estão entre os limites da possibilidade, alguns são
muito prováveis e outros altamente improváveis, e outros ainda estão
entre esses dois. Os limites da possibilidade são muito amplos. Se a
Torah nos tivesse permitido decidir casos capitais com base numa forte
probabilidade, Sue pode parecer absolutamente convincente, como no caso
do exemplo dado, no caso seguinte estaríamos decidindo baseados numa
probabilidade ligeiramente menor, e assim por diante gradualmente, até
que estivéssemos julgando casos capitais e condenando pessoas à morte
baseados em suposições injustificáveis, de acordo com os caprichos do
juiz. Por isso o Enaltecido fechou essa porta, por assim dizer,
ordenando que nenhum castigo seja aplicado a menos que haja testemunhas
que aleguem saber de fato o que aconteceu, sem uma dúvida qualquer, e
que não haja outra explicação possível. Se não sentenciarmos, ainda que
com base numa suspeita muito forte, o pior que pode acontecer é que o
pecador será absolvido; mas se punirmos baseados na força de suspeitas e
suposições pode ser que um dia enviemos à morte uma pessoa inocente. E é
melhor e mais satisfatório absolver mil pessoas culpadas do que enviar à
morte uma única pessoa inocente, uma única vez.

Da mesma forma, se duas testemunhas declararem que um homem cometeu duas
transgressões, cujo castigo por cada uma delas é a morte, e se cada uma
das duas testemunhas o tiver visto cometendo apenas uma das
transgressões --- por exemplo, se uma testemunha declarar que o acusado
trabalhou no Shabat, e que ele o avisou para que não o fizesse, e a
outra declarar que o acusado adorou ídolos, e que ele o avisou para que
não o fizesse --- o acusado não deverá ser apedrejado. ``Suponha'', diz
a Mekhiltá, ``que uma testemunha declare que uma determinada pessoa
adora o sol, e outra que ela adora a lua;
eu poderia pensar que eles devam ser reunidos\footnote{Que esses dois testemunhos devam ser reunidos e que o acusado deva ser declarado culpado.}; por
isso as Escrituras dizem: `O inocente e o justo não mates'''.

\paragraph{Uma testemunha não pode atuar como advogado}

Por esta proibição uma testemunha fica proibida de atuar como advogado
num caso no qual ela preste depoimento. Mesmo que ele seja culto e bem
informado, ele não deve atuar como testemunha, juiz e advogado, e sim
depor com relação ao que ele viu, ficando em silêncio enquanto os juízes
fizerem uso de seu depoimento como eles julgarem apropriado. A
testemunha está proibida de dizer o que quer que seja em acréscimo ao
seu depoimento. Esta proibição, que se aplica apenas a casos capitais,
está expressa em Suas palavras, enaltecido seja Ele, ``Uma testemunha
não deporá contra alguém para que morra'' (Números 35:30), e novamente
em Suas palavras ``Não será morto por depoimento de uma testemunha''
(Deuteronômio 17:6). Quer dizer, ele não será morto pela argumentação
das testemunhas.

Na Guemará de Sanhedrin lemos: `` `Uma testemunha não deporá contra
alguém' seja por sua absolvição ou pela sua condenação'', e o seguinte
motivo é dado: ``Ele fica parecendo uma testemunha
interessada''\footnote{Como se ele tivesse interesse próprio em sua testemunha.}
É apenas em casos capitais que está proibido advogar em favor da
absolvição ou da condenação.

\paragraph{Não matar um assassino sem julgamento}

Por esta proibição somos proibidos de matar quem cometeu um crime, a
quem vimos fazer algo que se pune com a morte, antes que ele seja levado
a julgamento. Ele deve ser levado a julgamento e devem ser apresentadas
provas contra ele ao Tribunal; nós só podemos depor contra ele, e o
Tribunal o sentenciará por qualquer ofensa que ele tenha cometido. A
proibição está expressa em Suas palavras, enaltecido seja Ele, ``Não
morrerá o homicida antes de ser apresentado diante da congregação para
o julgamento'' (Números 35:12). Sobre isso a Mekhiltá diz: ``Eu poderia
pensar que ele pode matá-lo, se ele tiver cometido assassinato ou
fornicação, por isso as Escrituras dizem: `Antes de ser apresentado
diante da congregação'''. Ainda que aqueles que o viram cometer o
assassinato sejam membros do Grande Tribunal, eles se tornam todos
testemunhas que prestarão depoimento diante de outro Tribunal, e esse
outro Tribunal é quem o condenará à morte. E a Mekhiltá diz: ``Suponha
que `uma congregação' veja um homem cometer um assassinato; eu poderia
pensar que eles podem matá-lo antes que ele seja condenado corretamente
por um Tribunal. Por isso as Escrituras dizem: `Não morrerá o homicida
antes de ser apresentado diante da congregação'''.

\paragraph{Não poupar a vida de um perseguidor}

Por esta proibição somos proibidos de poupar a vida de um perseguidor. A
explicação disto é a seguinte. O preceito precedente, que proíbe a
testemunha de matar o criminoso antes que ele seja condenado pelo Tribunal,
se aplica apenas ao caso de alguém que já tenha executado o ato pelo
qual ficou sujeito à morte; mas enquanto ele ainda estiver tentando
executá-lo, é chamado de ``perseguidor'', e temos a obrigação de detê-lo
e impedi-lo de levar a cabo suas más intenções. Se ele se recusar
obstinadamente, devemos lutar com ele; e se for possível detê-lo de
fazer o que ele tenciona privando-o de um de seus membros ---como por
exemplo, cortando sua mão ou seu pé, ou cegando-o --- muito bem; mas se
for impossível detê-lo a não ser que se lhe tire a vida, ele deve ser
morto antes que possa executar a ação. A proibição que nos proíbe de
poupar um perseguidor e de recusar-nos a matá-lo está expressa em Suas
palavras ``Cortar-lhe-ás a mão, o teu olho não terá piedade dela''
(Deuteronômio 25:12)\footnote{Ver o preceito positivo 247.}, sobre as quais o Sifrei diz:
```Cortar-lhe-ás a mão' nos ensina que você deve salvá-lo com a mão do
atacante. Como sabemos que, se a mão não salvar a vítima, devemos
salvá-la tirando a vida do atacante? Porque as Escrituras dizem: `Teu
olho não terá piedade'''. Também está dito ali: ```E lhe pegar pelas
\emph{suas vergonhas}' (Ibid.): assim como as vergonhas estão
especificadas aqui porque elas envolvem risco de vida, é assim acarretam
`Cortar-lhe-ás a mão', da mesma forma todas as vezes que a vida for
posta em perigo este mesmo princípio deve ser aplicado''.

O princípio que nós estabelecemos, de que um perseguidor deve ser morto,
não se aplica a todos os que viriam a ser malfeitores e sim apenas
àquele que persegue outra pessoa com a intenção de matá-la, mesmo que
ele seja um menor, ou com a intenção de tomá-la de uma das maneiras
proibidas, o que obviamente inclui uma agressão por parte de um varão. O
Enaltecido diz ``A moça desposada gritou e não houve quem a salvasse''
(Deuteronômio 22:27), de onde se conclui que se houver um salvador, ele
deve salvá-la de qualquer forma possível; e Ele compara o caso de alguém
que persegue uma donzela, ao de quem persegue seu companheiro com
intenção de matá-lo, através de Suas palavras ``Porque como no caso do
homem que se levanta contra o seu companheiro, e o mata, assim também é
este caso'' (Ibid., 26).

As normas deste preceito estão explicadas no oitavo capítulo de Sanhedrin.

\paragraph{Não punir uma pessoa por um pecado cometido sob coação}

Por esta proibição somos proibidos de punir uma pessoa por um pecado
cometido sob coação, pois ela terá agido sob pressão. Esta proibição
está expressa em Suas palavras, enaltecido seja Ele, ``Mas à moça não
farás nada'' (Deuteronômio 22:26). Em Sanhedrin lemos: ``O
Misericordioso isenta de punição aquele que peca sob coação, pois está
dito: `Mas à moça não farás nada'''..

\paragraph{Não aceitar um resgate de alguém que tenha cometido um assassinato deliberadamente}

Por esta proibição somos proibidos de aceitar resgate de alguém que
tenha cometido um assassinato deliberadamente. Tal pessoa deve ser morta
em todos os casos. A proibição está expressa em Suas palavras, enaltecido
seja Ele, ``Não aceitarás resgate pela vida do homicida, que é condenado
a morrer'' (Números 35:31).

As normas deste preceito estão explicadas em Macot.

\paragraph{Não aceitar um resgate de alguém que tenha cometido um assassinato involuntariamente}

Por esta proibição somos proibidos de aceitar resgate por alguém que
tenha cometido um assassinato involuntariamente de forma a livrá-lo do
exílio\footnote{Ver o preceito positivo 182.}. Ele deve ser banido, em todos os casos. A
proibição está expressa em Suas palavras, enaltecido seja Ele, ``E não
aceitarás resgate por aquele que fugiu para a cidade de refúgio''
(Números 35:32).

As normas deste preceito estão explicadas na Guemará de Macot.

\paragraph{Não se descuidar de salvar um israelita em perigo de vida}

Por esta proibição somos proibidos de descuidar-nos de salvar a vida de
um israelita que virmos correndo risco de vida ou de destruição e a quem
estiver em nosso poder salvar, como por exemplo se uma pessoa estiver se
afogando e se formos bons nadadores e pudermos salvá-la, ou se um pagão
estiver tentando matar alguém, e estivermos em condições de impedir sua
tentativa ou de salvar a pessoa ameaçada. Num caso assim somos proibidos
de manter-nos à parte e recusar-nos de ir em seu socorro por Suas
palavras ``Não sejas indiferente quando está em perigo o teu próximo''
(Levítico 19:16).

Os Sábios dizem que esta proibição abrange também o caso de alguém que
negar provas, pois ele verá o dinheiro de seu amigo perder-se, estando
em posição de restituí-lo se contar a verdade. As Escrituras se referem
novamente a este assunto: ``Se não o denunciar, levará seu pecado''
(Ibid., 5:1)\footnote{Ver o preceito positivo 178.}.

A Sifrá diz: ``De que forma sabemos que se você tem conhecimento de
alguma prova favorável a ele você não deve omiti-la? Pelo texto: `Não
sejas indiferente quando está em perigo o teu próximo'. E de que forma
sabemos que se você vir alguém afogando-se ou sendo atacado por ladrões
ou por um animal selvagem, você tem a obrigação de salvá-lo? Pelo texto:
`Não sejas indiferente quando está em perigo o teu próximo'. E de que
forma sabemos que se alguém persegue seu vizinho com a intenção de
matá-lo você tem a obrigação de salvá-lo mesmo que isso custe uma vida?
Pelo texto: `Não sejas indiferente quando está em perigo o teu
próximo'''.

As normas deste preceito estão explicadas no Tratado Sanhedrin.

\paragraph{Não deixar obstáculos em propriedades públicas ou privadas}

Por esta proibição somos proibidos de deixar obstáculos ou empecilhos
em propriedades públicas ou privadas para que não causem acidentes
fatais. Ela está expressa em Suas palavras, enaltecido seja Ele, ``Para
que não ponhas culpa de sangue em tua casa'' (Deuteronômio 22:8). Sobre
isso o Sifrei diz: `` `Farás um parapeito' (Ibid.) é um preceito
positivo\footnote{Ver o preceito positivo 184.}; `Para que não ponhas culpa de sangue em
tua casa' é um preceito negativo''.

As normas deste preceito estão explicadas no primeiro capítulo de
Shekalim no Talmud de Jerusalém, e em vários trechos em Nezikin.

\paragraph{Não dar um conselho enganoso}

Por esta proibição somos proibidos de dar um conselho enganador.
Portanto, se alguém pedir seu conselho sobre um assunto que ele não
compreenda muito bem você está proibido de enganá-lo ou desencaminhá-lo;
você deve dar o que você considera ser a orientação correta. A proibição
está expressa em Suas palavras, enaltecido seja Ele, ``Diante do cego
não porás tropeço'' (Levítico 19:14), sobre as quais a Sifrá diz: ``Se
alguém é `cego' de alguma maneira, e lhe pedir um conselho, não lhe dê
um conselho que não lhe seja apropriado''.

De acordo com os Sábios, este preceito negativo também se aplica a
ajudar ou levar alguém a cometer uma transgressão, porque fazer isso é
ajudar e incitar a cometer um delito um homem cuja paixão o tenha
privado de sua capacidade de raciocinar e o tenha cegado, ou
apresentar-lhe oportunidades para o pecado. É nesse sentido que os
Sábios dizem, com relação a uma transação envolvendo um empréstimo com
juros, que tanto quem empresta como quem pede emprestado transgridem
``Diante do cego não porás tropeço'', uma vez que um ajuda o outro a
completar a transgressão. Há muitos casos desse tipo, nos quais os
Sábios dizem que se transgride o ``Diante do cego não porás tropeço''.
Contudo, o significado literal do versículo é o que explicamos acima.

\paragraph{Não infligir castigo corporal excessivo}

Por esta proibição um juiz fica proibido de infligir a um malfeitor um
castigo corporal tão severo que lhe cause dano permanente. Isto deve ser
explicado da seguinte forma: a quantidade máxima de açoites que pode ser
imposta a um homem sujeito ao açoitamento foi fixada pela Tradição em
trinta e nove, mas nenhum homem pode ser submetido a um castigo corporal
até que seja feita uma estimativa do número de açoites que ele pode
suportar, levando-se em consideração sua idade, temperamento e físico.
Se ele puder suportar o castigo pleno, ele será aplicado; se não, ele
deverá receber tantos açoites quantos for capaz de suportar, com um mínimo de
ires. Isto se baseia nas palavras do Enaltecido ``Com o número de açoites segundo a
sua culpa'' (Deuteronômio 25:2). O castigo total é quarenta açoites
menos um, e o preceito proíbe que se exceda nem que seja por um a
quantidade que o ofensor pode suportar, de acordo com a estimativa do
juiz. A proibição está expressa em Suas palavras ``Com o número de
açoites segundo a sua culpa. Quarenta açoites lhe fará dar, não irá
além'' (Ibid., 2-3).

O Sifrei diz: ``Se exceder o limite, ele violará um preceito negativo.
Isso me foi dito apenas com relação aos quarenta. De que forma fico
sabendo que isso se aplica a qualquer quantidade que ele possa suportar,
de acordo com a estimativa do Tribunal? Pelas palavras das Escrituras
`Não irá além, com receio que (pen) suceda que indo além\ldots{}'''
(Ibid.,3).

Este preceito negativo também proíbe bater num israelita, seja ele quem
for. Se já somos proibidos de bater num pecador, quanto mais numa outra
pessoa! Os Sábios, a paz esteja com eles, também nos proíbem de ameaçar
de bater numa pessoa, mesmo que não o façamos realmente: ``Aquele que
simplesmente levantar a mão contra seu vizinho com a intenção de bater
nele é chamado de `malfeitor' (rasha), como está dito: `E diz ao mau
(la-rasha): Por que feres a teu próximo?''' (Exodo 2:13).

\paragraph{Não bisbilhotar}

Por esta proibição somos proibidos de bisbilhotar. Ela está expressa em
Suas palavras, enaltecido seja Ele. ``Não andarás com mexericos (rachil)
entre o teu povo'' (Levítico 19:16), sobre as quais o Sifrei diz: ``Não
se deve ser indulgente com um e severo com outro''. Outra interpretação
é: ``Não deves ser como um vendedor ambulante, que carrega sua
mercadoria de um lugar para outro''.

Este preceito negativo também proíbe a difamação\footnote{Ver o preceito positivo 219.}.

\paragraph{Não odiar uns aos outros}

Por esta proibição somos proibidos de odiar uns aos outros. Ela está
expressa em Suas palavras ``Não odiarás a teu irmão em teu coração''
(Levítico 19:17), sobre as quais a Sifrá diz: "Eu falo apenas de rancor
no coração. Contudo, se alguém revelar seu ódio e
disser à pessoa que odeia que ele é seu inimigo, ele não violará este
preceito negativo, mas transgredirá ``Não te vingarás e nem guardarás
ódio'' (Ibid., 18), e também violará o preceito positivo expresso em
Suas palavras ``Amarás o teu próximo como a ti mesmo''
(Ibid.)\footnote{Ver o preceito positivo 206.}. Mas o ódio no coração é o pecado mais
grave de todos.

\paragraph{Não envergonhar ninguém}

Por esta proibição somos proibidos de envergonhar alguém, o que é
chamado de ``embranquecer a face de seu companheiro'', ou seja,
envergonhá-lo em público. A proibição relativa a este assunto está expressa em
Suas palavras ``Repreenderás a teu companheiro e não levarás sobre ti
pecado'' (Levítico 19:17)\footnote{Ver o preceito positivo 205.}

A Sifrá diz: ``De que forma ficamos sabendo que mesmo que alguém tenha
repreendido um homem quatro ou cinco vezes ele deve continuar a fazê-lo?
Pelas palavras das Escrituras `Repreenderás'. Eu poderia pensar que é
assim mesmo quando sua repreensão transforma seu semblante, por isso as
Escrituras dizem: `E não levarás sobre ti pecado'''. O sentido literal
do versículo, contudo, é que somos proibidos de guardar qualquer
pensamento sobre seu peca, do ou de recordá-lo.

\paragraph{Não se vingar um do outro}

Por esta proibição somos proibidos de vingar-nos um do outro, ou seja,
se alguém nos tiver feito algum mal, não devemos insistir em persegui-lo
até que tenhamos retribuído sua maldade ou que o tenhamos ferido como
ele nos feriu. O Eterno proíbe essas coisas com as palavras ``Não te
vingarás'' (Levítico 19:18).

A Sifrá diz: ``Até onde vai o poder de vingança? Se `A' disser a `B':
`Empreste-me sua foice' e ele recusar, e se no dia seguinte `B' disser a
`A': `Empreste-me sua machadinha', e ele responder: `Não a emprestarei a
você, da mesma forma que você se recusou a emprestar-me sua foice',
contra esse tipo de conduta está dito: `Não te vingarás'''. Podemos
fazer deduções por analogia, a partir deste exemplo, para todos os
outros casos.

\paragraph{Não guardar rancor}

Por esta proibição somos proibidos de guardar rancor --- isto é,
guardar na lembrança um mal que alguém nos tenha feito e lembrar disso
contra ele --- mesmo que não nos vinguemos. Esta proibição está expressa
em Suas palavras, enaltecido seja Ele, ``Não te vingarás e nem
guardarás ódio'' (Levítico 19:18).

A Sifrá diz: ``Até onde vai o poder do rancor? Se `A' disser a `B':
`Empreste-me sua foice', e ele recusar e se no dia seguinte `B' disser a
`A': `Empreste-me sua machadinha', e ele responder: `Aqui está ela; eu
não sou como você, que não quis emprestar-me sua foice', contra esse
tipo de conduta está dito: `Nem guardarás ódio'''.

\paragraph{Não pegar o ninho todo de um pássaro}

Por esta proibição somos proibidos, quando estivermos caçando, de pegar
o ninho todo de um pássaro, com a mãe e Os filhotes. Ela está expressa
em Suas palavras, enaltecido seja Ele, ``Não tomarás a mãe estando com
os filhos'' (Deuteronômio 22:6).

Este é um preceito negativo justaposto a um preceito positivo, a saber,
``Deixarás ir livremente a mãe'' (Ibid.,7)\footnote{Ver o preceito positivo 148.} Se não
se obedecer ao preceito positivo relacionado, deixando a mãe sair, e caso a mãe morra antes de ser liberta, ele será punido com o açoitamento.

As normas deste preceito estão explicadas no final de Hulin.

\paragraph{Não raspar a tinha}

Por esta proibição somos proibidos de raspar o cabelo em volta da tinha.
Ela está expressa em Suas palavras ``O lugar da tinha não se raspará''
(Levítico 13:33).

Nas palavras da Sifrá: ``De que forma ficamos sabendo que aquele que
tirar sinais de impureza de sua tinha viola um preceito negativo? Pelas
palavras das Escrituras `O lugar da tinha não se raspará'''.

\paragraph{Não cortar ou cauterizar marcas de lepra}

Por esta proibição somos proibidos de cortar ou de cauterizar marcas de lepra de maneira a modificar sua aparência\footnote{Ver os preceitos positivos 101 a 103.}.
Esta proibição está expressa em Suas palavras ``Guarda-te da chaga da
lepra'' (Deuteronômio 24:8), sobre as quais diz o Sifrei: `` `Guarda-te
da chaga da lepra' é um preceito negativo'', e a Mishná diz: ``O homem
que tira as marcas da impureza ou cauteriza a lepra transgride um
preceito positivo'' e está sujeito ao açoitamento, como explicamos no
lugar apropriado.

\paragraph{Não lavrar um vale no qual tenha sido realizado o ritual de ``Eglá Arufá''}

Por esta proibição somos proibidos de lavrar ou cultivar um vale virgem
no qual se tenha destroncado o pescoço de uma
vaca\footnote{Ver o preceito positivo 181.}. Ela está expressa em Suas palavras ``Que
não se lavra nem se semeia'' (Deuteronômio 21:4). A contravenção a esta
proibição é punida com o açoitamento.

A Guemará de Macot, ao enumerar as transgressões puníveis com o
açoitamento, diz: ``Por que não incluir também aquele que semeia num
`vale virgem', já que a proibição necessária está expressa nas palavras
`Que não se lavra nem se semeia'?'' Dessa forma, fica claro que este é
apenas um preceito negativo, e que sua desobediência é punida com o
açoitamento.

As normas deste preceito estão explicadas no final de Sotá.

\paragraph{Não deixar viver um feiticeiro}

Por esta proibição somos proibidos de permitir que um feiticeiro viva.
Ela está expressa em Suas palavras ``Feiticeira não deixarás viver''
(Êxodo 22:17). Permiti-lo é quebrar um preceito negativo, e não somente um
preceito positivo\footnote{Ver o preceito positivo 229.}, como no caso de perdoar um
malfeitor que esteja sujeito à morte por sentença
judicial\footnote{Nesse caso transgride se também um preceito positivo.}.

\paragraph{Não levar um recém-casado para longe de sua casa}

Por esta proibição somos proibidos de levar um recém-casado para longe
de sua casa durante um ano para fazer qualquer tipo de serviço, seja
militar ou civil. Ao contrário, devemos liberá-lo, durante um ano
inteiro, de todos os deveres que possam afastá-lo de
casa\footnote{Ver o preceito positivo 214.}. A proibição está expressa em Suas palavras,
enaltecido seja Ele, ``Nem lhe será imposto carga alguma; livre estará
para cuidar de sua casa'' (Deuteronômio 24:5).

Na Guemará de Sotá lemos: `` `Não servirá o exército' (Ibid.): eu
poderia pensar que ele não sairá com o exército, mas que preparará armas
e fornecerá água e comida. Por isso as Escrituras dizem: `Nem lhe será
imposto carga alguma'. A \emph{ele} não será imposta carga alguma, mas
você pode impô-la a outros. Contudo, se está escrito `Nem lhe será
imposto carga alguma', qual é a finalidade de `Não servirá o exército'?
Para que a transgressão da lei envolva duas proibições''.

Nós já explicamos no Nono Fundamento que nem toda transgressão que nos
torna culpados por duas proibições envolve dois preceitos.

Vocês devem saber que o próprio esposo está proibido de deixar sua casa,
ou seja, de sair numa viagem, durante um ano inteiro.

As normas deste preceito estão explicadas no oitavo capítulo de Sotá.

\paragraph{Não discordar das autoridades tradicionais}

Por esta proibição somos proibidos de discordar dos guardiães
autorizados da Tradição, a paz esteja com eles, ou de afastar-nos do que
quer que eles ordenem em assuntos da Torah. Ela está expressa em Suas
palavras ``Não te desviarás da sentença que te anunciarem''
(Deuteronômio 17:11), sobre as quais o Sifrei diz: `` `Não te
desviarás\ldots{}' é um preceito negativo''.

Aquele que infringir este preceito negativo é chamado de ``o velho
rebelde'' e está sujeito à morte por estrangulamento, pelas condições
estabelecidas pela Tradição, que estão expostas no final de Sanhedrin,
onde as normas deste preceito estão explicadas.

\paragraph{Não fazer acréscimos à lei escrita ou oral}

Por esta proibição somos proibidos de fazer acréscimos à lei escrita ou
oral. Ela está expressa em Suas palavras, enaltecido seja Ele, ``Não
acrescentareis\ldots{} a isso nada'' (Deuteronômio 13:1). Os Sábios dizem
frequentemente: ``Ele transgride a lei `Não acrescentareis a isso
nada''', ou ``Você transgrediu a lei `Não acrescentareis a isso nada'''.

\paragraph{Não fazer diminuições na lei escrita ou oral}

Por esta proibição somos proibidos de fazer diminuições na lei escrita
ou oral. Ela está expressa em Suas palavras ``Nem diminuireis a isso
nada'' (Deuteronômio 13:1). Os Sábios dizem frequentemente: ``Ele
transgride a lei.`Nem diminuireis a isso nada' '', ou ``Você transgrediu
a lei `Não diminuireis a isso nada'''.

\paragraph{Não maldizer um juiz}

Por esta proibição somos proibidos de maldizer um juiz. Ela está
expressa em Suas palavras ``Aos juízes (Elohim) não maldigas'' (Exodo
22:27) A contravenção a esta proibição será punida com o açoitamento.

\paragraph{Não maldizer um chefe}

Por esta proibição somos proibidos de maldizer um chefe. Ela está
expressa em Suas palavras ``Ao chefe (Nassi) de teu povo não maldigas''
(Èxodo 22:27).

Pela palavra ``Nassi'' as Escrituras querem dizer o
rei que governa, como em Suas palavras, enaltecido seja Ele, ``Quando o
\emph{príncipe da nação} pecar'' (Levítico 4:22). Contudo, os Sábios
usam as palavras apenas com relação ao Chefe da Academia dos Setenta
Anciões (isto é, do Grande Sanhedrin). Assim, em todo o Talmud e na
Mishná eles falam de: ``Os Nessiim \emph{e} os Juízes Principais'', ``O
Nassi e o Juiz Principal''. Eles também dizem: ``Se um Nassi perdoar
sua honra, seu perdão será aceito, mas se um rei perdoar sua honra, ele
não será aceito''.

Vocês devem saber que esse preceito negativo se refere ao Nassi assim
como ao rei, pois seu objetivo é advertir-nos contra maldizer qualquer
um que tenha uma posição de autoridade suprema, seja na esfera da
autoridade governamental ou na da Torah, como Cabeça da Academia. É
isso o que se depreende das normas deste preceito.

A contravenção a esta proibição é punida com o açoitamento.

\paragraph{Não maldizer um israelita}

Por esta proibição somos proibidos de maldizer qualquer israelita. Ela
está expressa em Suas palavras ``Não amaldiçoarás ao surdo'' (Levítico
19:14).

Explicarei agora o significado do termo ``heresh'' (surdo).

Quando uma pessoa é movida pelo desejo de vingar-se de alguém que o
enganou, causando-lhe um dano do tipo que ele acredite ter sofrido, ele
não ficará satisfeito enquanto não tiver devolvido o mal dessa forma, e
seus sentimentos só serão mitigados e sua mente só abandonará essa ideia
quando ele tiver se vingado. Algumas vezes o desejo de vingança de um
homem se satisfaz meramente maldizendo e insultando porque ele sabe
quanto dano e vergonha isso causará ao seu inimigo Mas algumas vezes o assunto será mais
sério e ele não se contentará enquanto não tiver arruinado completamente
o outro, e assim se satisfará ao pensar na dor que a perda de seus bens
causou a seu inimigo. Em outros casos o assunto será ainda mais sério, e
ele não se satisfará até que tenha surrado seu inimigo ou o tenha
ferido. Ou pode ser, mais sério ainda, e seu desejo de
vingança não ficará satisfeito enquanto ele não chegar ao extremo de
tirar a vida de seu inimigo e destruir toda sua existência. Por outro
lado, algumas vezes, devido à leveza da ofensa, o desejo de vingança não
será grande, de forma que ele se sentirá aliviado praguejando e
maldizendo enraivecido, mesmo se o outro não o ouvisse, se estivesse
presente. É sabido que pessoas geniosas e coléricas encontram alívio
dessa forma com relação a ofensas triviais, ainda que o ofensor
desconheça sua raiva e não ouça suas explosões.

No entanto poderíamos supor que a Torah nos proíbe de maldizer um
israelita pela vergonha e a dor que a imprecação lhe causaria ao
ouvi-la, mas que não há pecado em maldizer um surdo, pois como ele não
pode ouvir, não há de se sentir ofendido. Por essa razão Ele nos diz que
não se deve maldizer, proibindo de fazê-lo no caso do surdo, já que a
Torah se refere não apenas ao ofendido, mas também ao ofensor, a quem se
diz que não deve ser vingativo nem genioso. É dessa maneira que ficamos
sabendo que os guardiães da Tradição deduzem a proibição de maldizer um
israelita das palavras das Escrituras ``Não amaldiçoarás ao surdo''.

A Sifrá diz: ``Está dito que não devo amaldiçoar o surdo; de que forma
fico sabendo que não se deve amaldiçoar ninguém? Pelas palavras das
Escrituras `De teu povo não maldigas' (Êxodo 22:27). Isto é para
excluir os mortos, os quais, embora sendo como os
surdos\footnote{No sentido de que podem ouvir e se sentir ofendidos.}, se diferenciam deles por não estarem mais
vivos''.

A Mekhiltá diz: ```Não amaldiçoarás ao surdo': as Escrituras mencionam
o mais infeliz dos seres humanos''.

Ao dizer que se pune com o açoitamento, queremos dizer apenas se\footnote{Só se viola o preceito se a maldição vier acompanhada do
pronunciamento do Nome Divino.} com o Nome Divino. Aquele que amaldiçoar a si
mesmo também será punido com o açoitamento.

Ficou, dessa forma, claro para vocês que aquele que amaldiçoar seu
companheiro com o Nome Divino viola o preceito ``Não amaldiçoarás ao
surdo''; aquele que amaldiçoar um juiz é culpado duas vezes; e aquele
que amaldiçoar um chefe é culpado três vezes. A Mekhiltá diz: ``Quando
as Escrituras dizem: `Ao chefe de teu povo' (Ibid.), eu interpreto isso
como significando ambos um chefe e um juiz. Então por que elas dizem
`Aos juízes não maldigas'? Para deixar claro que se é culpado ao
maldizer qualquer um dos dois''. Dessa forma, a Mekhiltá diz: ``É
possível, através de um simples pronunciamento, tornar-se culpado
quatro vezes. O filho de um chefe que amaldiçoa seu pai é culpado quatro
vezes: por amaldiçoar seu pai, por amaldiçoar um juiz, por amaldiçoar um
chefe, e por amaldiçoar um israelita (incluído em `teu povo')''.

E assim demos a explicação prometida acima.

As normas deste preceito estão explicadas no quarto capítulo de Shebuoth.

\paragraph{Não amaldiçoar os pais}

Por esta proibição cada um de nós está proibido de amaldiçoar seus pais.

A Torah enuncia claramente a punição nas palavras ``Aquele que maldisser
a seu pai ou a sua mãe, será certamente morto'' (Êxodo 21:17), e esta é
uma das ofensas que se pune com o apedrejamento. Mesmo aquele que
deliberadamente amaldiçoar com o Nome Divino o pai ou a mãe que não
estiverem mais vivos será apedrejado. Esta proibição, contudo, não está
claramente expressa nas Escrituras, pois Ele não diz: ``Não amaldiçoarás
teu pai''. Mas, como foi explicado antes, há uma proibição contra
amaldiçoar um israelita, e ela inclui nosso pai entre os outros.

A Mekhiltá diz: ```Aquele que maldisser a seu pai ou a sua mãe, será
certamente morto': ouvimos o castigo por fazê-lo, mas onde encontramos
essa proibição? Nas palavras das Escrituras `Aos juízes (Elohim) não
maldigas'. (Êxodo 22:27). Se seu pai for um juiz, ele está incluído em
`Elohim'; se ele for um chefe, ele está incluído em
`Nassi' (Ibid.); e se ele for uma pessoa comum, ele está incluído em
``Não amaldiçoarás ao surdo'' (Levítico 19:14). E você pode estabelecer
uma regra geral com base no que há de comum aos três, a saber, que eles
são `de teu povo' (Ibid.) e portanto você está proibido de
amaldiçoá-los. Assim também com relação a seu pai. Ele pertence a `teu
povo' e portanto você está proibido de amaldiçoá-lo''.

A Sifrá diz: ```O homem que amaldiçoar a seu pai, e a sua mãe' (Levítico
20:9): ouvimos o castigo por isso, mas não ouvimos a proibição, por isso
as Escrituras dizem: `Aos juízes não maldigas''' sendo este texto igual
ao da Mekhiltá, citado acima.

As normas deste preceito estão explicadas no sétimo capítulo de Sanhedrin.

\paragraph{Não ferir seus pais}

Por esta proibição somos proibidos de ferir nossos pais.

Também neste caso não há uma proibição expressa nas Escrituras, mas o
castigo está mencionado em Suas palavras ``Aquele que ferir a seu pai ou
a sua mãe, será certamente morto'' (Êxodo 21:15), e deduzimos a
proibição contra ferir nosso pai através do método usado com relação a
amaldiçoar nosso pai, o qual explicamos no preceito negativo 300, ou
seja, que nosso pai está incluído no preceito e nos proíbe de ferir
qualquer israelita.

A Mekhiltá diz: `` `Aquele que ferir a seu pai ou a sua mãe': ouvimos o
castigo, mas não ouvimos a proibição, por isso as Escrituras dizem:
`Quarenta açoites lhe fará dar, não irá além' (Deuteronômio 25:3), e
pelo método de `kal vahomer'\footnote{Literalmente, ``com mais razão'': argumento do menor ao maior.} raciocinamos da
seguinte forma: se no caso de alguém em quem temos a obrigação de bater
somos proibidos de fazê-lo\footnote{Somos proibidos de dar mais do que um determinado número de açoites.}, conclui-se que no caso
dos pais, em quem temos o dever de não bater, somos totalmente
proibidos de fazê-lo''.

Aquele que transgredir este preceito negativo --- ou seja, que deliberadamente ferir seu pai ou sua mãe, fazendo com que eles sangrem --- está sujeito à morte por estrangulamento.

As normas deste preceito estão explicadas no final de Sanhedrin.

\paragraph{Não trabalhar no Shabat}

Por esta proibição somos proibidos de fazer qualquer
trabalho\footnote{Inclui 39 tipos de trabalhos proibidos no Shabat. Ver Mishné Torá
  Milchot Shabat, 7º Capítulo, 1ª Lei.} no Shabat. Ela está expressa em Suas
palavras ``Não farás nenhuma obra'' (Êxodo 20:10). As Escrituras
prescrevem expressamente a pena de extinção pela desobediência deste
preceito negativo, se ela não chegar ao conhecimento do Tribunal; mas se
houver o depoimento de testemunhas, o castigo será a morte por
apedrejamento. Isto se aplica à transgressão voluntária; aquele que
pecar involuntariamente deve oferecer um Sacrifício Determinado de
Pecado\footnote{Ver o preceito positivo 69.}.

As normas deste preceito estão explicadas no Tratado Shabat.

\paragraph{Não viajar no Shabat}

Por esta proibição somos proibidos de viajar no Shabat. Ela está
expressa em Suas palavras ``Não saia ninguém de seu lugar no sétimo
dia'' (Êxodo 16:29). A Tradição fixa o limite além do qual é proibido ir
em dois mil cúbitos além dos limites da cidade; nem um cúbito a mais. É
permitido ir até dois mil cúbitos em qualquer direção. A Mekhiltá diz:
`` `Não saia ninguém de seu lugar': isto é, além de dois mil cúbitos''.

A Guemará de Erubin diz: ``A pena de açoitamento por transgressão da lei
de `erub' de limites está prescrita por lei das Escrituras''. As normas
deste preceito estão explicadas nesse Tratado.

\paragraph{Não castigar durante o Shabat}

Por esta proibição somos proibidos de aplicar o castigo de um malfeitor
no Shabat. Ela está expressa em Suas palavras, enaltecido seja Ele,
``Não acendereis fogo em todas as vossas habitações, no dia de sábado''
(Exodo 35:3), cujo significado é: ``Não queime o réu que deve ser
queimado''; e a mesma lei se aplica a todas as outras formas de
execução. A Mekhiltá diz: ```Não acendereis fogo'. Acender fogo, que
está incluído entre os tipos de trabalho proibidos no Shabat, foi
especialmente destacado para ensinar-nos que assim como as leis do
Shabat não podem ser transgredidas nem mesmo no caso especificamente
mencionado de execução pelo fogo, elas também não podem ser
transgredidas no caso de qualquer uma das formas de execução judicial''.

No Talmud lemos que acender o fogo está destacado porque é um preceito
negativo\footnote{E a punição por sua transgressão é apenas o açoitamento e não a
  extinção e a pena capital, como no caso dos outros trabalhos.}; mas o parecer aceito é que isso está
destacado porque a execução de cada tipo diferente de trabalho acarreta
uma penalidade separada, como está explicado ali.

Na Guemará de Jerusalém lemos: `` `Em todas as vossas habitações': Rabi
Ilai, em nome de Rabi Lanai, comenta: `Em todas as vossas habitações':
desta forma ficamos sabendo que os Tribunais não devem julgar no
Shabat''.

\paragraph{Não trabalhar no primeiro dia de ``Pessah''}

Por esta proibição somos proibidos de trabalhar no primeiro dia de
``Pessah''. Ela está expressa em Suas palavras ``Nenhum trabalho será
feito neles'' (Êxodo 12:16)\footnote{Ver também Levítico 23:7.}.

\paragraph{Não trabalhar no sétimo dia de ``Pessah''}

Por esta proibição somos proibidos de trabalhar no sétimo dia de
``Pessah''. Ela está expressa em Suas palavras ``Nenhum trabalho será
feito neles'' (Êxodo 12:16)\footnote{Ver também Levítico 23:8.}, ou seja, no primeiro
e no sétimo dia.

\paragraph{Não trabalhar em ``Atzeret''}

Por esta proibição somos proibidos de trabalhar em ``Atzeret'', isto é,
``Shabuot''. Ela está expressa em Suas palavras ``Nenhum trabalho servil
fareis'' (Levítico 23:21).

\paragraph{Não trabalhar em ``Rosh Hashaná''}

Por esta proibição somos proibidos de trabalhar no dia de ``Rosh
Hashaná''. Ela está expressa em Suas palavras, relativas a esse dia,
``Nenhum trabalho servil fareis'' (Levítico 23:25).

\paragraph{Não trabalhar no primeiro dia de ``Sucot''}

Por esta proibição somos proibidos de trabalhar no primeiro dia de
``Sucot''. Ela está expressa em Suas palavras, relativas a esse dia,
``Nenhum trabalho servil fareis'' (Levítico 23:35).

\paragraph{Não trabalhar em ``Shemini Atzeret''}

Por esta proibição somos proibidos de trabalhar em ``Shemini Atzeret''.
Ela está expressa em Suas palavras, relativas a esse dia, ``Nenhum
trabalho servil fareis'' (Levítico 23:36).

Vocês devem saber que todo aquele que fizer qualquer tipo de trabalho em
qualquer desses dias estará sujeito ao açoitamento, a menos que o
trabalho seja relativo ao preparo do alimento necessário, como dizem as
Escrituras, referindo-se a um deles, ``Salvo o que é
para comer para toda alma, isto só será feito para vós'' (Êxodo 12:16).
O mesmo se aplica em relação ao resto dos festivais.

As normas deste preceito estão explicadas no Tratado Betzá.

\paragraph{Não trabalhar em ``Yom Kipur''}

Por esta proibição somos proibidos de trabalhar em ``Yom Kipur''. Ela
está expressa em Suas palavras, relativas a esse dia, ``Nenhum trabalho
fareis'' (Levítico 23:31).

Aquele que transgredir voluntariamente este preceito negativo estará
sujeito à extinção, como determinam as Escrituras; aquele que pecar
involuntariamente deverá oferecer um Sacrifício Determinado de Pecado.

As normas deste preceito estão explicadas nos Tratados Betzá, Meguilá, e
outros.

\paragraph{Não cometer incesto com sua mãe}

Por esta proibição um homem fica proibido de cometer incesto com sua
mãe. Ela está expressa em Suas palavras ``Tua mãe é ela, não descobrirás
sua nudez'' (Levítico 18:7).

A contravenção a esta proibição será punida com a extinção. Se
testemunhas depuserem contra o transgressor ele será apedrejado, caso
tenha pecado deliberadamente; se o tiver feito involuntariamente, ele
deverá oferecer um Sacrifício Determinado de Pecado.

\paragraph{Não cometer incesto com a esposa de seu pai}

Por esta proibição um homem fica proibido de cometer incesto com a
esposa de seu pai. Ela está expressa em Suas palavras ``Nudez da mulher
de teu pai não descobrirás'' (Levítico 18:8).

A contravenção a esta proibição será punida com a extinção. Se
testemunhas depuserem contra o transgressor, ele será morto por
apedrejamento se tiver pecado deliberadamente, mas se o tiver feito
involuntariamente, ele deverá oferecer um Sacrifício Determinado de
Pecado.

Ficou dessa forma claro para vocês que o homem que cometer incesto com
sua mãe será culpado uma vez por ser ela sua ``mãe'' e outra vez por ser
ela a ``esposa de seu pai'', quer seja durante a vida de seu pai ou
depois de sua morte, como está explicado em Sanhedrin.

\paragraph{Não cometer incesto com sua irmã}

Por esta proibição um homem fica proibido de cometer incesto com sua
irmã. Ela está expressa em Suas palavras ``Nudez de tua irmã, filha de
teu pai ou filha de tua mãe\ldots{} não descobrirás sua nudez'' (Levítico
18:9).

Aquele que deliberadamente transgredir este preceito está sujeito à
extinção; se o violar involuntariamente, ele deverá oferecer um
Sacrifício Determinado de Pecado.

\paragraph{Não cometer incesto com a filha da esposa de seu pai se ela for sua irmã}

Por esta proibição um homem fica proibido de cometer incesto com a filha
da esposa de seu pai, se ela for sua irmã. Ela está expressa em Suas
palavras, enaltecido seja Ele, ``Nudez da filha da mulher de teu pai,
gerada de teu pai, tua irmã ela é, não descobrirás sua nudez'' (Levítico
18:11).

O objetivo desta proibição é fazer da filha da esposa do pai uma relação
proibida separada, para que aquele que cometer incesto com sua irmã por
parte de pai, se a mãe dela for casada com o pai dele, seja culpado duas
vezes: porque ela é sua irmã e porque ela é filha da esposa de seu pai,
assim como um homem que comete incesto com sua mãe é culpado duas vezes:
porque ela é sua mãe e porque ela é a esposa de seu pai, como
explicamos. Isso aparece no seguinte trecho do segundo capítulo de
Yebamot: ``Nossos Sábios nos ensinaram: Um homem que comete incesto com
sua irmã, que também é a filha da esposa de seu pai, é culpado ambos
porque ela é sua irmã e é a filha da esposa de seu pai. Rabi Yossi ben
Yehudá disse que ele é culpado apenas por ser ela sua irmã mas não por
ser a filha da mulher de seu pai. Por que motivo ele disse isso?
Observem, diriam eles, que está escrito `Nudez de tua irmã, filha de teu
pai etc.' (Ibid., 9); qual é a finalidade das palavras `Nudez da filha
da mulher de teu pai, gerada de teu pai etc.'? É para declarar que ele
é culpado por ser ela ambas sua irmã e filha da mulher de seu pai''.

Aquele que violar esta proibição cometendo incesto com uma irmã cuja mãe
esteja casada com seu pai está sujeito à extinção, mas apenas se seu
pecado tiver sido deliberado. Se ele tiver pecado involuntariamente, ele
deverá oferecer um Sacrifício Determinado de Pecado.

\paragraph{Não cometer incesto com a filha de seu filho}

Por esta proibição um homem fica proibido de cometer incesto com a filha
de seu filho. Ela está expressa em Suas palavras ``Nudez da filha de teu
filho\ldots{} não descobrirás'' (Levítico 18:10).

\paragraph{Não cometer incesto com a filha de sua filha}

Por esta proibição um homem fica proibido de cometer incesto com a filha
de sua filha. Ela está expressa em Suas palavras ``Ou filha de tua
filha, não descobrirás sua nudez, porque tua nudez são `elas'''
(Levítico 18:10).

\paragraph{Não cometer incesto com sua filha}

Por esta proibição um homem fica proibido de cometer incesto com sua
própria filha.

Esta proibição não está explicitamente enunciada na Torah; as
Escrituras não dizem ``Não descobrirás a nudez de tua filha''. O motivo
dessa omissão é que a proibição é evidente, pois uma vez que o incesto
com a filha de um filho ou com a filha de uma filha é proibido, é óbvio
que o incesto com uma filha é proibido.

A Guemará de Yebamot diz: ``Chegou-se ao princípio que fundamenta a
proibição com a filha por interpretação, pois Rabá disse: `Rabi Isaac
ben Abdmei me disse que ficamos sabendo da existência da lei pelo fato
de que ``hená'' (elas) e ``zimá'' (maldade) aparecem ambos em dois
trechos relacionados'''. Quer dizer, ao proibir o incesto com a filha de
um filho e com a filha de uma filha, as Escrituras dizem: ``Pois tua
nudez são elas (hená)'' (Levítico 18:10); e ao proibir de chegar-se a
uma mulher e a sua filha, ou a uma mulher e à filha de seu filho, ou a
uma mulher e à filha de sua filha, está dito: ``Elas (hená) são parentes
próximas; mau pensamento (zimá) é'' (Ibid., 17). Assim como na proibição
de chegar-se a uma mulher e à filha do filho dela ou à filha da filha
dela, também está proibido chegar-se à filha dela, de forma que a
proibição de chegar-se à filha de um filho ou à filha de uma filha
inclui a proibição de chegar-se a sua própria filha. Além disso, as
Escrituras dizem, com relação ao castigo: ``E um homem que tomar uma
mulher e a sua mãe, obra de pensamento mau (zimá) é; no fogo queimarão a
ele e a elas'' (Ibid., 20:14); e chegar-se a uma mulher e à filha do
filho dela ou à filha da filha dela também é punido pelo fogo, porque a
palavra ``hená'' foi usada nos dois casos.

Na Guemará de Queretot lemos: ``Nunca trate um `guezerá shavá'
levianamente pois, observe, o preceito da filha é um dos mais
importantes da Torah, e ele nos foi ensinado pelas Escrituras apenas
através de um `guezerá shavá', isto é, através do aparecimento da
palavra `hená' em dois trechos relacionados, e da palavra `zimá' em dois
outros''. Observem que a expressão é ``ele nos foi ensinado'' e não
``nós o aprendemos'' porque todas essas coisas nos foram transmitidas
pela Tradição através do Emissário\footnote{Do emissário do Eterno ou seja, Moisés.} e elas constituem a explicação tradicional, como explicamos na Introdução à
nossa obra, o ``Comentário sobre a Mishná''. A razão pela qual as
Escrituras não o mencionam explicitamente é apenas porque ele pode ser
deduzido através de um ``guezerá shavá''. É isso o que o Talmud quer
dizer com as palavras: ``Ele nos foi ensinado pelas Escrituras apenas
através de um `guezerá shavá'''; e a referência à filha como ``um dos
mais importantes preceitos da Torah'' é o bastante.

Por todo o exposto ficou claro que o incesto com uma filha, com a filha de uma filha, ou com a filha de um filho e punida com o fogo. Se
ninguém tiver conhecimento da transgressão ou se não houver evidências
concretas contra o transgressor, ele estará sujeito à extinção, caso o
tenha feito voluntariamente. Aquele que cometer alguma dessas
transgressões involuntariamente deverá oferecer um Sacrifício
Determinado de Pecado.

\paragraph{Não se chegar a uma mulher e a sua filha}

Por esta proibição fica proibido chegar-se a uma mulher e a sua filha.
Ela está expressa em Suas palavras ``Nudez de uma mulher e de sua filha
não descobrirás'' (Levítico 18:17).

Aquele que violar esta proibição --- sendo uma das mulheres sua esposa
--- está sujeito a morrer no fogo, se a prova contra ele for
evidenciada. Ele está sujeito à extinção se a transgressão for
voluntária, mas se pecou involuntariamente ele deverá oferecer um
Sacrifício Determinado de Pecado.

\paragraph{Não se chegar a uma mulher e à filha do filho dela}

Por esta proibição fica proibido de chegar-se a uma mulher e à filha do
filho dela. Ela está expressa em Suas palavras, abençoado seja Ele, "É à
filha de seu filho" (Levítico 18:17). Também neste caso o transgressor
será queimado e punido com a extinção, se tiver pecado deliberadamente,
e oferecerá um Sacrifício Deterrriinado de Pecado se o tiver feito
involuntariamente.

\paragraph{Não se chegar a uma mulher e à filha da filha dela}

Por esta proibição fica proibido de chegar-se a uma mulher e à filha da
filha dela. Ela está expressa em Suas palavras, enaltecido seja Ele, ``E
à filha de sua filha'' (Levítico 18:17). O transgressor será punido com
a extinção e morrerá queimado se tiver pecado deliberadamente; sé tiver
pecado involuntariamente ele deverá oferecer um Sacrifício Determinado
de Pecado.

\paragraph{Não cometer incesto com a irmã de seu pai}

Por esta proibição um homem fica proibido de cometer incesto com a irmã
de seu pai. Ela está expressa em Suas palavras ``Nudez da irmã de teu
pai não descobrirás'' (Levítico 18:12). O transgressor está sujeito à
extinção, se tiver pecado deliberadamente, e se o tiver feito
involuntariamente, ele deverá oferecer um Sacrifício Determinado de
Pecado.

\paragraph{Não cometer incesto com a irmã de sua mãe}

Por esta proibição um homem fica proibido de cometer incesto com a irmã
de sua mãe. Ela está expressa em Suas palavras, enaltecido seja Ele,
``Nudez da irmã de tua mãe não descobrirás'' (Levítico 18:13). Quem
transgredir este preceito deliberadamente está sujeito à extinção, e
quem pecar involuntariamente deverá oferecer um Sacrifício Determinado
de Pecado.

\paragraph{Não se chegar à esposa do irmão de seu pai}

Por esta proibição um homem fica proibido de chegar-se à esposa do irmão
de seu pai. Ela está expressa em Suas palavras ``Não te chegarás a sua
mulher; ela é tua tia'' (Levítico 18:14). O transgressor deste preceito
está sujeito à extinção se tiver pecado voluntariamente; se o tiver
feito involuntariamente, ele deverá oferecer um Sacrifício Determinado
de Pecado.

\paragraph{Não se chegar à esposa de seu filho}

Por esta proibição um homem fica proibido de chegar-se a esposa de seu
filho. Ela está expressa em Suas palavras ``Nudez de tua nora não
descobrirás'' (Levítico 18:15). O transgressor será punido com o
apedrejamento; mas se a prova contra ele não tiver sido evidenciada, ou
se ninguém souber da transgressão, a pena é a extinção se ele tiver
pecado voluntariamente; se ele o tiver feito involuntariamente, ele
deverá oferecer um Sacrifício Determinado de Pecado.

\paragraph{Não se chegar à esposa de seu irmão}

Por esta proibição um homem fica proibido de chegar-se á esposa de seu
irmão. Ela está expressa em Suas palavras ``Nudez da mulher de teu
irmão não descobrirás'' (Levítico 18:16). O transgressor deste preceito
está sujeito à extinção se tiver pecado voluntariamente; se o tiver
feito involuntariamente, ele deverá oferecer um Sacrifício Determinado
de Pecado.

\paragraph{Não se chegar à irmã de sua esposa enquanto esta última for viva}

Por esta proibição um homem fica proibido de chegar-se à irmã de sua
esposa enquanto esta viver. Ela está expressa em Suas palavras,
enaltecido seja Ele, ``E a mulher com sua irmã não tomarás'' (Levítico
18:18). Aquele que transgredir este preceito voluntariamente está
sujeito à extinção; aquele que pecar involuntariamente deverá oferecer
um Sacrifício Determinado de Pecado.

\paragraph{Não se unir a uma mulher menstruada}

Por esta proibição um homem fica proibido de unir-se a uma mulher
menstruada durante o período de sua impureza, ou seja, durante os sete
dias completos. Ela está expressa em Suas palavras ``E a mulher na impureza
de sua menstruação não te chegarás'' (Levítico 18:19); e enquanto ela
não tiver feito uma imersão\footnote{Num banho ritual.} depois de completados
os sete dias, ela será considerada menstruada.

O transgressor voluntário deste preceito está sujeito à extinção; aquele
que pecar involuntariamente deverá oferecer um Sacrifício Determinado de
Pecado.

\paragraph{Não se chegar à esposa de outro homem}

Por esta proibição um homem fica proibido de chegar-se à esposa de um
outro homem. Ela está expressa em Suas palavras ``E com a mulher de teu
companheiro não te deitarás para dar sêmen'' (Levítico 18:20).

A punição pela violação deste preceito varia de acordo com as
circunstâncias. Se a mulher for noiva\footnote{A mulher noiva, ou prometida legalmente, está no estágio preliminar ao casamento, que acarreta todas as consequências legais deste.} ambos ficam
sujeitos ao apedrejamento, como determinam as
Escrituras\footnote{Vide Deuteronômio 22:23-24.}. Se ela for a filha de um ``Cohen'',
ela deverá morrer queimada e o homem estrangulado. Se ela for a filha de
um israelita, ambos estão sujeitos à morte por estrangulamento. Tudo
isso se aplica se a prova for evidenciada, caso contrário o homem fica
sujeito à extinção. Também neste caso tudo isso se aplica se o pecado
tiver sido cometido voluntariamente pelo homem, mas se ele o cometeu
involuntariamente, ele deverá oferecer um Sacrifício Determinado de
Pecado.

A proibição desta transgressão aparece em outro lugar, nos Dez
Mandamentos, em Suas palavras ``Não adulterarás'' (Êxodo 20:14), que
significam que não se deve chegar-se à esposa de outro homem. Nas
palavras da Mekhiltá: ``Por que foi dito `Não adulterarás'? Porque nas
palavras `Certamente serão mortos, o adúltero e a adúltera' (Levítico
20:10) ouvimos a penalidade, mas não ouvimos a proibição. Por essa razão
as Escrituras dizem: `Não adulterarás'''. Da mesma forma, a Sifrá diz:
```O homem que cometer adultério com a mulher de outro homem, que
adulterar com a mulher de seu próximo' (Ibid.): ouvimos aqui a
penalidade, mas não ouvimos a proibição. Por essa razão as Escrituras
dizem: `Não adulterarás' a ambos o homem e a mulher''. Eles não
encontraram a proibição nas palavras ``E com a mulher de teu
companheiro não te deitarás para dar sêmen'' porque essa proibição não
inclui ambos o adúltero e a adúltera, mas é dirigida apenas ao adúltero.
Da mesma forma, no que se refere às relações proibidas em geral, eles
tiveram que aplicar a proibição também à mulher, e por isso lemos na
Sifrá: ```Nenhum \emph{de vós} se chegará\ldots{} para descobrir a sua
nudez' (Ibid., 18:6) proíbe ambos o homem através da mulher e a mulher
através do homem''.

A Guemará de Sanhedrin diz: ``Todos estão incluídos nos termos
`adúltero' e `adúltera' mas as Escrituras excluem a filha de um `Cohen',
ensinando que ela deve ser queimada, e a moça noiva, ensinando que ela
deve ser apedrejada''.

Nós explicamos este assunto na Introdução ao presente trabalho.


\paragraph{Os homens não podem deitar-se com animais}

Por esta proibição um homem fica proibido de se deitar com um animal,
macho ou fêmea. Ela está expressa em Suas palavras ``E com qualquer
animal não te deitarás'' (Levítico 18:23). O transgressor voluntário
está sujeito à morte por apedrejamento, e se não for apedrejado, à
extinção Se ele pecou involuntariamente, deverá oferecer um Sacrifício
Determinado de Pecado.

\paragraph{As mulheres não podem deitar-se com animais}

Por esta proibição as mulheres ficam proibidas de se deitarem com
animais. Ela está expressa em Suas palavras, enaltecido seja Ele, ``Nem
a mulher se porá diante de um animal para se juntar com ele'' (Levítico
18:23). Este é um preceito independente, não incluído no precedente, uma
vez que as Escrituras, ao proibir os homens de se deitar com animais,
não impõem a mesma proibição às mulheres, na ausência de um preceito
negativo específico a elas. Assim, no princípio de Queretot lemos: ``Há
trinta e seis ofensas pelas quais a Torah prescreve a extinção'', e a
enumeração delas que se segue inclui as cometidas por um homem que se
deita com um animal e por uma mulher que se deita com um animal, embora
sejam enumeradas apenas categorias gerais, como explicamos em nosso
``Comentário''. Dessa forma fica claro que esta proibição é um preceito
independente, e deve ser incluída na lista dos preceitos negativos.

Aquela que violar este preceito voluntariamente está sujeita ao
apedrejamento; se a prova não for evidenciada, ela está sujeita à
extinção se tiver pecado voluntariamente. Se tiver pecado
involuntariamente, deverá oferecer um Sacrifício Determinado de Pecado.

\paragraph{Um homem não pode chegar-se a outro homem}

Por esta proibição um homem fica proibido de chegar-se a um varão. Ela
está expressa em Suas palavras ``E com um homem não te deitarás como se
fosse uma mulher'' (Levítico 18:22), e aparece também em outro lugar, em
Suas palavras ``Nem haverá destinado à pederastia dentre os filhos de
Israel'' (Deuteronômio 23:18). Este preceito negativo está repetido
para dar maior força e não para dirigir a proibição à vítima. As
palavras das Escrituras ``E com um homem não te deitarás'' estipulam a
advertência às duas partes.

A Guemará de Sanhedrin diz que é Rabi Ishmael que considera ``Nem haverá
destinado à pederastia dentre os filhos de Israel'' como a proibição
dirigida à vítima. Consequentemente, ``Aquele que cometer pederastia
ativamente e que também permitir que o ofendam dessa forma, de maneira
negligente, está sujeito, de acordo com o ponto de vista de Rabi
Ishmael, a duas penalidades; mas Rabi Akiba diz que isso é
desnecessário porque `E com um homem não te deitarás (lo tishcab) como
se fosse mulher' pode ser lido como `Não serás deitado (lo
tishacheb)'''. Portanto, aquele que cometer pederastia e também permitir que o ofendam dessa forma, de maneira negligente, estará
sujeito a uma penalidade apenas, pois ``lo tishcab'' (Não te deitarás) e
``lo tishacheb'' (Não serás deitado) são um único preceito, e na opinião
de R. Akiba o objetivo de ``Nem haverá destinado à pederastia'' é apenas
para reforçar ó preceito, da mesma forma que a fim de reforçar ``Não
adulterarás'' (Êxodo 20:14) que é, como já explicamos, a proibição da
mulher de outro homem, temos ``E com a mulher de teu companheiro não te
deitarás para dar sêmen'' (Levítico 18:20).

Há muitos casos deste tipo, como explicamos no Nono Fundamento.

O transgressor deste preceito será punido com o apedrejamento. Se ele
não for apedrejado, estará sujeito à extinção se o pecado tiver sido
voluntário; se ele o tiver cometido involuntariamente, ele deverá
oferecer um Sacrifício Determinado de Pecado.

\paragraph{Um homem não pode chegar-se a seu pai}

Por esta proibição um homem fica proibido de chegar-se a seu pai. Ela
está expressa em Suas palavras ``A nudez de teu pai\ldots{} não descobrirás''
(Levítico 18:7). Também neste caso o transgressor está sujeito ao
apedrejamento. Assim, um homem que se chegar a seu pai será culpado duas
vezes: por ser um varão e por ser seu pai.

A Guemará de Sanhedrin explica que Suas palavras ``A nudez de teu pai\ldots{}
não descobrirás'' se referem na realidade ao pai. A isso se objetou o
seguinte: ``Mas não sabemos disso através do versículo `E com homem não
te deitarás como se fosse mulher'?'' (Ibid., 22), ao que se respondeu:
``Isso nos ensina que se incorre numa penalidade dupla. Rabi Yehudá
disse que um pagão que cometer pederastia com seu pai incorre numa
penalidade dupla''. Para explicar isso eles dizem ali: ``Essas palavras
de Rabi Yehudá se referem supostamente a um judeu\footnote{Que cometer a ofensa.},
inconscientemente, e\footnote{Que deverá oferecer.} um sacrifício; e o termo
`pagão' é um eufemismo''. Quer dizer, aquele que inconscientemente se
chegar a seu pai deverá oferecer dois Sacrifícios de Pecado, da mesma
forma que aquele que inconscientemente se chegar a duas mulheres
proibidas; mas se ele não for seu pai, ele deverá oferecer apenas um
Sacrifício de Pecado.

\paragraph{Um homem não pode chegar-se ao irmão de seu pai}

Por esta proibição um homem fica proibido de chegar-se ao irmão de seu
pai. Ela está expressa em Suas palavras ``Nudez do irmão de teu pai não
descobrirás'' (Levítico 18:14). Portanto, aquele que involuntariamente
se chegar ao irmão de seu pai deverá oferecer dois Sacrifícios de
Pecado, como explicamos no caso de seu pai. Na Guemará de Sanhedrin
lemos: ``Todos concordam que aquele que cometer pederastia com seu tio
paterno incorre numa penalidade dupla, pois as Escrituras dizem: `Nudez
do irmão de teu pai não descobrirás'''.

Vocês devem saber que toda vez que eu usar a expressão ``a prova
for evidenciada'' eu quero dizer que deve haver duas ou mais testemunhas
qualificadas que tenham feito uma advertência, e que tenham apresentado
a prova diante de um Tribunal qualificado de vinte e três juízes, e que
isto se aplica apenas enquanto estiverem em vigor as leis da pena
capital.

Está claro que, no caso de todos os delitos carnais mencionados, as
Escrituras estabelecem explicitamente a penalidade de extinção porque
após enumerá-los elas dizem: ``Aquele que fizer alguma destas
abominações, serão banidas as almas que o fizerem'' (Levítico 18:29). Da
mesma forma, toda vez que afirmamos que uma ofensa desse tipo acarreta a
morte por sentença judicial, essa penalidade também está prescrita nas
Escrituras. Mas quanto ao modo de execução, que em alguns casos está
determinada que seja pelo apedrejamento, em outros pelo estrangulamento,
e em outros pelo fogo, a autoridade é, em alguns casos a Tradição, e em
outros as Escrituras.

As normas das leis relativas a todos esses delitos carnais estão
explicadas nos Tratados Sanhedrin e Queretot, e em vários trechos de
Yebamot, Quetubot e Kidushin.

Está explicado no início de Queretot que no caso de qualquer pecado
cuja penalidade é a morte se cometido voluntariamente e a oferta de
Sacrifício Determinado de Pecado se cometido involuntariamente, a
pessoa de cuja culpa se tem dúvidas deverá oferecer um Sacrifício
Suspensivo de Delito\footnote{Ver o preceito positivo 70.}. O termo ``Sacrifício
Determinado de Pecado'' significa que o sacrifício deve ser sempre de
rebanho, isto é, uma ovelha ou uma cabra.

Se vocês observarem todos os preceitos negativos e examinarem os
castigos relativos a cada um deles vocês perceberão que no caso de cada
pecado pelo qual a pena é a extinção, se cometido voluntariamente, e a
oferta de um Sacrifício de Pecado, se cometido involuntariamente, o
sacrifício referido é um Sacrifício Determinado de Pecado, com a exceção
de dois pecados, pelos quais a pena é a extinçào, se cometidos
voluntariamente, e um Sacrifício de Maior ou Menor Valor, se cometidos
involuntariamente, ao invés de um Sacrifício Determinado de
Pecado\footnote{Ver o preceito positivo 72.}. Esses dois pecados são a profanação do
Santuário e a profanação de suas Santidades. Por ``profanação do
Santuário'' eu quero dizer a entrada de uma pessoa impura no Campo do
Santuário (``az-hara'') e por ``profanação de suas Santidades'' eu quero
dizer que uma pessoa que tenha se tornado impura coma a carne das
ofertas consagradas.

Da mesma forma, ficará claro para vocês que pela violação de qualquer
preceito negativo, cuja penalidade é a extinção se cometida
voluntariamente, fica-se obrigado a oferecer um Sacrifício de Pecado se
ela tiver sido cometida involuntariamente, exceto no caso da blasfêmia,
a qual acarreta a extinção se cometida voluntariamente mas não obriga a
um Sacrifício de Pecado se cometida involuntariamente.

Ficará também claro para vocês que todos aqueles que estiverem sujeitos
à morte por uma das quatro formas de execução judicial estarão sujeitos
à extinção se o Tribunal não os executar ou não souber de suas
transgressões, exceto em dez casos, nos quais as pessoas ficam sujeitas
à morte por sentença judicial, mas não à extinção, a saber: quem desviar
outrem\footnote{Ver o preceito negativo 15.} do caminho certo, quem desencaminhar outras
pessoas do caminho certo\footnote{Ver o preceito negativo 16.}, num falso
profeta\footnote{Ver o preceito negativo 27.}, quem profetizar em nome de um ídolo \footnote{Ver o preceito negativo 26.}, um ancião
que menosprezar a decisão do Supremo Tribunal\footnote{Ver o preceito negativo 312.}, um
filho teimoso e rebelde\footnote{Ver o preceito negativo 195.}, quem raptar um
israelita\footnote{Ver o preceito negativo 243.}, um assassino\footnote{Ver o preceito negativo 289.},
quem bater em seu pai ou em sua mãe\footnote{Ver o preceito negativo 319.} e quem
amaldiçoar seu pai, ou sua mãe\footnote{Ver o preceito negativo 318.}. Em cada um desses
casos, se a prova for evidenciada, ele será morto; mas se o Tribunal não
souber de sua transgressão ou não tiver podido condená-lo à morte, ele
se expôs à morte mas não corre o perigo de extinção.

Vocês devem conhecer e lembrar-se desses princípios.

\paragraph{Não ter intimidades com uma parenta}

Por esta proibição somos proibidos de procurar prazer no contato com
qualquer parenta que esteja na categoria das mulheres proibidas, mesmo
que seja apenas através de abraços, beijos e coisas assim. A proibição
de tal conduta está expressa em Suas palavras, enaltecido seja Ele,
``Nenhum de vós se chegará àquele que lhe é próximo por carne, para
descobrir a sua nudez'' (Levítico 18:6) que tem o mesmo significado que
se Ele tivesse dito: ``Vocês não devem se aproximar deles de forma tal
que possa conduzir a uma relação proibida''. Sobre isso a Sifrá diz:
```Se chegará\ldots{} para descobrir a sua nudez' proíbe apenas o fato de
chegar-se a ela; de que forma fico sabendo que é proibido
`aproximar-se'? Pelas palavras das Escrituras: `E à mulher na impureza
de sua menstruação \emph{não te chegarás}' (Ibid., 19). Contudo, isto
proíbe apenas aproximar-se de uma mulher menstruada e chegar-se a ela;
de que forma fico sabendo que as duas proibições se aplicam a todas as
mulheres das categorias proibidas? Pelas palavras `\emph{Se chegará}
àquele que lhe é próximo por carne'''.

A Sifrá diz também: ```Serão banidas as almas que o fizerem' (Ibid.,
29). Por que foi dito isso? Porque pelas palavras `Se aproximar' eu
poderia pensar que se fica sujeito à extinção pela simples
`aproximação'. Por isso as Escrituras dizem: `Que o \emph{fizerem}' mas
não as almas que simplesmente se aproximarem delas''.

A proibição de tal conduta indecente aparece novamente nas palavras
``Para não fazer nenhum dos costumes abomináveis'' (Ibid., 30). Mas o
versículo contendo as duas proibições, a saber, ``Segundo as obras da
terra do Egito, na qual estivestes, não fareis, e segundo as obras da
terra de Canaã, à qual Eu vos levo, não fareis'' (Ibid., 3) nos proíbe
não apenas a prática dos ``costumes abomináveis'', mas proíbe também os
atos abomináveis específicos que Ele expõe nos versículos seguintes.
Portanto estas duas proibições são de extensão global e cobrem todas as
categorias proibidas, bem como nos proíbem de fazer ``Segundo as obras
da terra do Egito\ldots{} e segundo as obras da terra de Canaã'', que
englobam todos os seus costumes, tanto os de libertinagem, como os de
agricultura, os de criação de gado e os de vida social. Depois Ele
passa a explicar que essas ``obras'' que Ele proíbe de praticar são
relações ilícitas com este e aquele, como fica claro pelas palavras que
concluem o relato: ``Porque todas estas abominações fizeram os homens da
terra que estavam antes de vós'' (Ibid., 27).

A Sifrá diz: ``Eu poderia pensar que não devemos construir nossas casas
nem plantar vinhedos da maneira como o fazem as outras nações; por isso
as Escrituras dizem: `Não andeis segundo os seus costumes' (Ibid. 3),
que significam que o decreto se aplica apenas aos costumes que eles e
seus ancestrais prescreveram por lei''. Diz também: ``O que costumavam
eles fazer? Um homem se casava com um homem, uma mulher se casava com
uma mulher, e uma mulher se casava com dois homens''. Portanto, fica
claro que esses dois preceitos negativos --- a saber, ``Segundo as obras
da terra do Egito\ldots{} não fareis e segundo as obras da terra de Canaã\ldots{} não fareis'' --- proíbem em termos gerais todas as relações ilícitas
e são seguidos por proibições específicas relativas a cada uma das
relações proibidas individualmente\footnote{Ver o preceito positivo 38.}.

Nós mesmos já explicamos todas as normas deste preceito em nosso
``Comentário sobre a Mishná'', no sétimo capítulo de Sanhedrin, onde
explicamos que são punidas pelo açoitamento.

Também é importante para vocês saber que o fruto de uma relação pela
qual se está sujeito à extinção é chamado ``um bastardo''. O Eterno
chamou tal fruto de ``bastardo'', e quer o pecado tenha sido cometido
voluntária ou involuntariamente, seu fruto será ``um bastardo''. A única
exceção é o fruto da relação com uma mulher menstruada; nesse caso esse
fruto não será um bastardo, mas será chamado de ``filho de uma mulher
menstruada''. Isto está explicado no quarto capítulo de Yebamot.

\paragraph{Um ``mamzer'' não pode chegar-se a uma israelita}

Por esta proibição um bastardo fica proibido de chegar-se a uma
israelita. Ela está expressa em Suas palavras, enaltecido seja Ele,
``Não entrará bastardo na congregação do Eterno'' (Deuteronômio 23:3).

A contravenção a esta proibição será punida com o açoitamento. As normas
deste preceito estão explicadas no oitavo capítulo de Yebamot e no
final de Kidushin.

\paragraph{Não se chegar a uma mulher antes do casamento}

Por esta proibição um homem fica proibido de chegar-se a uma mulher sem
estar devidamente casado com ela. Ela está expressa em Suas palavras,
enaltecido seja Ele, ``Não haverá mulher destinada à prostituição dentre
as filhas de Israel'' (Deuteronômio 23:18), e aparece sob outra forma
em Suas palàvras, enaltecido seja Ele, ``Não profanarás a tua filha
para fazê-la prostituta'' (Levítico 19:29), sobre as quais a Sifrá diz:
```Não profanarás a tua filha para fazê-la prostituta' se refere a
alguém que entregue sua filha solteira à luxúria ou a uma mulher que se
entregue à luxúria''.

Deixem que eu explique porque Ele repete este preceito dessa forma e o
que a repetição acrescenta. Nós já havíamos recebido Sua lei, enaltecido
seja Ele, segundo a qual um homem que seduzir ou forçar uma moça não
está sujeito a nenhum castigo a não ser a pagar uma multa em dinheiro e
a casar-se com a moça, como determinam as
Escrituras\footnote{Vide Deuteronômio 22:28-29 e Êxodo 22:15-16.}. De acordo com isso poderíamos pensar
que, uma vez que envolve apenas uma penalidade em dinheiro, este caso é
como qualquer outro que envolve dinheiro e que assim como uma pessoa é
livre de dar seu dinheiro a quem ele quiser ou de liberar uma outra
pessoa de uma importância devida a ele, assim também seria permitido que
alguém deixasse que um homem se chegasse a sua filha solteira e
renunciasse ao pagamento em dinheiro, uma vez que isso --- ou seja, os
cinquenta siclos de prata --- pertence ao pai da moça por direito; ou
que alguém desse sua filha a um homem em troca de uma soma em dinheiro.
Qualquer ideia desse tipo está impedida pela proibição ``Não profanarás
a tua filha para fazê-la prostituta'' porque o dinheiro está prescrito
apenas nos casos de sedução ou força, e fazer tal coisa de mútuo acordo
é totalmente ilegal. A razão para isso está ém Suas palavras ``Para que
a terra não seja entregue à prostituição e que não se encha a terra de
pensamentos maus'' (Levítico 19:29). Sedução e força são raros mas se
tal conduta fosse permitida e legalizada, a impudicícia se tornaria
comum. Esta é uma bela e adequada explicação do versículo em questão, e
está em harmonia com os ensinamentos de nossos Sábios e com as
prescrições da Torah.

A contravenção a esta proibição relativa a moças solteiras será punida
com o açoitamento.

As normas deste preceito estão explicadas em Quetubot e Kidushin.

\paragraph{Não tornar a casar-se com a esposa de quem se divorciou, depois que
ela tenha se casado novamente}

Por esta proibição um homem fica proibido de tornar a casar-se com a
esposa de quem ele se divorciou se ela tiver estado casada com outro.
Ela está expressa em Suas palavras, enaltecido seja Ele, ``Não poderá
seu primeiro marido, que a despediu, tornar a tomá-la, para que seja sua
mulher, depois que foi contaminada'' (Deuteronômio 24:4).

A contravenção a esta proibição será punida com o açoitamento. As normas
deste preceito estão explicadas em várias passagens em Yebamot.

\paragraph{Não se chegar a uma mulher sujeita ao casamento levirato}

Por esta proibição outros homens ficam proibidos de chegar-se a uma
viúva sujeita ao casamento levirato\footnote{Ver o preceito positivo 216.}. Ela está
expressa em Suas palavras ``A mulher do defunto não se casará com homem
estranho de fora'' (Deuteronômio 25:5).

A contravenção a esta proibição será punida com o açoitamento do homem e
da mulher.

As normas deste preceito estão explicadas em Yebamot.

\paragraph{Não se divorciar da mulher que se violentou e com a qual se foi obrigado a casar}

Por esta proibição um homem fica proibido de divorciar-se da mulher que
ele violentou. Ela está expressa em Suas palavras ``Ela lhe será por
mulher\ldots{} e não a poderá despedir por todos os seus dias''
(Deuteronômio 22:29).

Este preceito negativo está precedido pelo preceito positivo ``Ela lhe
será por mulher''\footnote{Ver o preceito positivo 218.}, e isso está exposto na Guemará
de Macot, que prossegue assim: ``Um violentador israelita que tiver se
divorciado de sua mulher, poderá casar-se novamente com ela sem ficar
sujeito ao açoitamento; mas se ele for um `Cohen', ele será açoitado e
não poderá casar-se novamente com ela''.

Vocês devem saber que se um israelita se divorciar de uma mulher com
quem foi obrigado a casar-se, e se ela morrer antes que ele torne a se
casar com ela, ou se ela se casar com outro, ele será punido com o
açoitamento pois não terá cumprido o preceito positivo em questão. Isto
está de acordo com o principio aceito de que ``Se ele tiver
cumprido\footnote{O preceito positivo, ele estará isento.}, mas se ele não o tiver
cumprido''\footnote{Embora ele não o tenha invalidado por decisão própria, ele é
  considerado culpado.}.

As normas deste preceito estão explicadas no terceiro e quarto capítulos de Quetubot.

\paragraph{Não se divorciar de uma mulher depois de tê-la caluniado}

Por esta proibição um homem fica proibido de divorciar-se de sua mulher
depois de tê-la caluniado. Ela está expressa em Suas palavras (que são
usadas neste caso também) ``Não a poderá despedir, por todos os seus
dias'' (Deuteronômio 22:19).

Este preceito negativo também está precedido pelo positivo que está em
Suas palavras ``E lhe será por mulher'' (Ibid.)\footnote{Ver o preceito positivo 219.}.
Portanto, se ele se divorciar dela estará sujeito ao açoitamento, assim
como quem for culpado de forçar uma mulher, de acordo com o que está
explicado no final do Tratado Macot.

Ali, e no terceiro e no quarto capítulos de Quetubot, as normas deste
preceito estão explicadas.

\paragraph{Um homem incapaz de procriar não pode casar-se com uma israelita}

Por esta proibição um homem que tiver sofrido um acidente que o tenha
privado do poder da procriação está proibido de casar-se com uma mulher
israelita. Esta proibição está expressa em Suas palavras, enaltecido
seja Ele, 'Não entrará aquele que tem os testículos trilhados e aquele
cujo derrame de sêmen é deficiente, na congregação do Eterno '
(Deuteronômio 22:2), Se um homem assim chegar-se a uma mulher israelita
depois do noivado\footnote{O noivado, como preliminar ao casamento, contém todas as
consequências legais deste.}, ele será punido com o
açoitamento.

As normas deste preceito estão explicadas no nono capítulo de Yebamot.

\paragraph{Não castrar}

Por esta proibição somos proibidos de castrar um macho de qualquer que
seja a espécie, homem ou animal. Ela esta expressa nas últimas palavras
do versículo ``De testículos machucados, ou moídos, ou desprendidos, ou
cortados, não oferecereis ao Eterno, nem fareis estas coisas na terra''
(Levítico 22:24) que a Tradição explica como significando: ``Nem fareis
isso a vós mesmos''.

A contravenção a esta proibição, ou seja, a castração de qualquer
espécie,é punida com o açoitamento.

No capítulo ``Shemona Sheratsim'' lemos: ``Como sabemos que a castração
de um homem é proibida? Através do versículo `Nem fareis estas coisas na
terra', portanto não fareis isso a vós mesmos. Mesmo se alguém castrar
depois que um outro já o tenha feito, ele será culpado. Rabi Hiya ben
Abun diz, em nome do Rabi Yohanan: Todos concordam que se alguém
preparar levedado depois que alguém já o tenha preparado levedado ele é
culpado, porque está dito: `Não será cozido levedado' (Levítico, 2:10) e
`Será preparada com fermento' (Ibid., 2:11). Se alguém castrar depois
que outro já tenha castrado ele será culpado porque está dito: `De
testículos machucados, ou moídos, ou desprendidos, ou cortados\ldots{}'. Se
alguém é culpado por cortá-los, imaginem o quanto não o será por
despedaçá-los! Isso é para ensinar-nos que se alguém cortá-los depois de
outro despedaçá-los, ele será culpado''.

As normas deste preceito estão explicadas em várias passagens em Shabat
e Yebamot.

\paragraph{Não nomear um rei que não seja israelita de nascimento}

Por esta proibição somos proibidos de nomear um rei sobre nós que não
seja israelita de nascimento, mesmo que ele seja um Prosélito Justo. Ela
está expressa em Suas palavras, enaltecido seja Ele, ``Não poderás pôr
sobre ti um homem estranho, que não seja teu irmão'' (Deuteronômio
17:15), sobre as quais o Sifrei diz: ```Não poderás pôr sobre ti um homem estranho' é
um preceito negativo''.

Da mesma forma, com referência a todas as outras nomeações, sejam elas
religiosas ou governamentais, não podemos nomear sobre nós mesmos um
homem prosélito, a menos que sua mãe seja uma israelita; isto é
decorrente de Suas palavras, enaltecido seja Ele, ``Poderás,
certamente, pôr sobre ti o rei\ldots{} dentre teus irmãos, porás rei sobre
ti'' (Ibid.), que o Talmud interpreta como significando: ``Todas as
nomeações que fizerdes deverão ser apenas `dentre teus
irmãos'''\footnote{Ver o preceito positivo 173.}

No que se refere à realeza, vocês já sabem, pelas sagradas escrituras
dos profetas, que Davi foi considerado digno do título pois o Talmud diz
claramente: ``A coroa da realeza, Davi mereceu e recebeu'', assim como
todos os de sua linhagem até o fim de todas as gerações. Para aquele que
crê na Torah de Moisés, nosso mestre, não pode haver rei que não seja
descendente de Davi através de Salomão apenas; e alguém que não seja
dessa linhagem nobre é considerado um ``estrangeiro'', no que se refere
à realeza, assim como todo aquele que não for descendente de Aarão é
considerado um ``estranho'', no que se refere a oficiar no Santuário.
Isto está claro, e não há dúvida alguma a esse respeito.

As normas deste preceito estão explicadas em várias passagens de
Yebamot, Sanhedrin, Sotá e Nidá.

\paragraph{Um rei não pode possuir muitos cavalos}

Por esta proibição o rei fica proibido de possuir muitos cavalos. Ela
está expressa em Suas palavras, enaltecido seja Ele, ``Não multiplicará
para si cavalos'' (Deuteronômio 1 7: 16).

O limite permitido é que ele não deve ter cavalos correndo diante dele,
nem possuir um único cavalo a não ser aquele que ele monta. Ele pode
manter cavalos em seus estábulos para que seu exército monte em tempos
de guerra, mas só lhe é permitido ter um animal para seu uso privado.

As normas deste preceito estão explicadas no segundo capítulo de Sanhedrin.

\paragraph{Um rei não pode ter muitas esposas}

Por esta proibição um rei fica proibido de ter muitas esposas. Ela está
expressa em Suas palavras, enaltecido seja Ele, ``Não multiplicará para
si mulheres'' (Deuteronômio 17:17). O limite permitido é que ele não
pode ter mais do que dezoito esposas, por matrimônio devidamente
contraído.

Os estatutos deste preceito já foram explicados no segundo capítulo de
Sanhedrin.

\paragraph{Um rei não pode acumular grande fortuna pessoal}

Por esta proibição o rei fica proibido de acumular uma grande fortuna
para si próprio. Ela está expressa em Suas palavras, enaltecido seja
Ele, ``Prata e ouro não multiplicará muito para si'' (Deuteronômio
17:17). O limite permitido é que ele não deve ir além do que é
estritamente necessário para a manutenção de seu exército e de seus
criados pessoais. Contudo, é-lhe permitido acumular riquezas para as
necessidades de todo o povo de Israel.

O Enaltecido explica nas Escrituras a razão desses três preceitos, a
saber, ``Não multiplicará para si cavalos'', ``Não multiplicará para si
mulheres'' ``Prata e ouro não multiplicará muito para
si''\footnote{Sam. 12:25.}; e o conhecimento dessas razões levou a sua
desobediência, como no caso notório de Salomão, a paz esteja com ele,
apesar da superioridade de seu conhecimento e sabedoria, e de ser ele
``o amado do Eterno''\footnote{469}.

Nossos Sábios aprenderam com isso que, se os homens soubessem as razões
para todos os preceitos, eles encontrariam meios de lhes desobedecer.
Pois se um homem tão perfeito supôs erroneamente que sua atitude não o
levaria de modo algum a uma transgressão, as massas, providas de mentes
simples, seriam mais facilmente levadas a desobedecer a eles,
argumentando o seguinte: Ele proibiu isto e ordenou aquilo apenas por
tal e tal razão, por isso vamos evitar o pecado cuidadosamente para
impedir aquilo que este preceito estabeleceu, mas não seremos
minunciosos com relação ao preceito em si; e isto destruiria a própria
base da Religião. Por essa razão o enaltecido não expôs as razões, mas
não há um único preceito que não tenha uma razão e uma causa, remotas ou
imediatas. A maioria dessas causas e razões, contudo, está acima da
inteligência e da compreensão das massas; contudo, o Profeta testemunha com relação a
todos\footnote{A todos os preceitos.}: ``Os preceitos do Eterno são justos e
alegram o coração; o mandamento do Eterno é puro e ilumina os
olhos''\footnote{Salmos 19:9.}.

Eu imploro a ajuda do Eterno para cumprir tudo o que ele ordenou
e para abster-me de tudo o que Ele proibiu.

Aqui termina o que tencionamos incluir neste trabalho.


