%MAIMÔNIDES
%
%Os 613 Mandamentos
%
%OS 613 MANDAMENTOS
%
%\textbf{(TARIAG HA-MITZVOTH)}
%
%MOSHÉ BEN MAIMON
%
%\textbf{HA RAMBAM}
%
%\begin{quote}
%2 2 EDIÇÃO
%
%TRADUÇÃO • BIOGRAFIA
%\end{quote}
%
%\textbf{GIUSEPPE NAHAISSI}
%
%\begin{quote}
%\includegraphics[width=0.41667in,height=0.41667in]{media/image1.png}
%\end{quote}
%
%NOVA STELLA
%
%\textbf{Dados de Catalogação, na Publicação (CIP) Internacional\\
%(Câmara Brasileira do Livro, SP, Brasil)}
%
%\begin{quote}
%Maimônides, 1135-1204.
%
%Os 613 Mandamentos : Tariag Ha-Mitzvoth / Moshé ben Maimon ; tradução,
%biografia Giuseppe Nahaïssi. --- São Paulo : Nova Stella, 1990.
%
%1. Judaísmo - Doutrinas 2. Maimônides, 1135-1204 3. Seiscentos e treze
%Mandamentos I. Título. II. Título: Tariag Ha-Mitzvoth.
%\end{quote}
%
%CDD-296.092
%
%-296
%
%90-0763 -296.172
%
%\begin{quote}
%\textbf{Índices para catálogo} sistemático:
%\end{quote}
%
%\begin{enumerate}
%\def\labelenumi{\arabic{enumi}.}
%\item
  %\begin{quote}
  %Judaísmo 296
  %\end{quote}
%\item
  %\begin{quote}
  %Mandamentos : 613 : Teologia social judaica 296.172
  %\end{quote}
%\item
  %\begin{quote}
  %Teólogos judeus : Biografia e obra 296.092
  %\end{quote}
%\end{enumerate}
%
%MAIMÔNIDES\\
%OS 613 MANDAMENTOS
%
%Título original\\
%TARIAG HA-MITZVOTH
%
%Tradução\\
%GIUSEPPE NAHAÏSSI
%
%Produção e Capa\\
%LUCIANO GUIMARÃES
%
%Revisão\\
%CRISTIANE REGINA BARBIERI\\
%CRISTINE BAENA FONTELLES\\
%NICOLE WEXLER
%
%Composição, Paginação e Filmes\\
%HELVÉTICA EDITORIAL LTDA.
%
%1 edição: 1990\\
%3000 exemplares
%
%2 edição: 1990\\
%2000 exemplares
%
%Copyright de tradução\\
%GIUSEPPE NAHAÏSSI
%
%NOVA STELLA EDITORIAL\\
%R. Antonio de Souza Noschese, 289\\
%05324 --- São Paulo --- SP\\
%Tel.: 268.4214 --- FAX: 268.0987

\chapter*{}
\thispagestyle{empty}
\begin{flushright}
\begin{vplace}[30]
\textsc{do rei david}

\emph{Ajuda-me a trilhar os caminhos de Teus Mandamentos,\\ pois é neles
que eu encontro deleite}
\end{vplace}
\end{flushright}

\chapter*{}
\thispagestyle{empty}
\begin{verse}
\versal{DA TORÁ}\\[10pt]

Neste dia o Eterno, teu Deus, te ordena\\
Que cumpras estes estatutos e leis.\\
Deverás cumpri-los diligentemente\\
Com todo o teu coração e com toda a tua alma.\\[10pt]

Com relação ao Eterno, confessaste, hoje, que Ele é teu \qb{}Deus\\
que andarás em seus caminhos\\
E observarás seus estatutos, preceitos e leis,\\
E que atenderás a suas determinações.\\[10pt]
 
E o Eterno confessou hoje, a teu respeito, que fazes parte \qb{}de seu povo,\\
Como Ele te havia prometido,\\
Portanto, deves observar todos os seus preceitos,\\
Para que possas ser um povo consagrado ao Eterno, teu \qb{}Deus\ldots{}\\[10pt]

E agora, Israel, o que o Eterno, teu Deus, pede de ti\\
A não ser que temas o Eterno, teu Deus,\\[10pt]

Que andes em todos os caminhos e que O ames\\
E que sirvas o Eterno, teu Deus, com todo o teu coração e \qb{}com toda a tua alma;\\
Que observes, para o teu bem, os mandamentos do Eterno\\\bigskip
\hfill{}(Deuteronômio 26:16-19, 10:12-13) 
%\enlargethispage{\textheight}
\end{verse}

%Agradeço
%
%Ao Rabino SHMUEL ZAIONTZ, ROSH YECHIVA, TOMCHEI TMIMIM LUBAVITCH de Nova
%York pela revisão e aconselhamento na preparação desta tradução.
%
%Ao Professor ELIE SOUCAR pela grande ajuda na preparação do Glossário e
%seus comentários.
%
%A Sra MARY VANSTREELS pela ajuda e colaboração em ordenar esta obra,
%sem a qual a mesma seria absolutamente impossível.
%
%Dedico estra tradução da obra do Rambam, Rabi Moshé ben Maimon, o Grande
%Maimônides, à minha mulher Sarah e aos meus filhos Moshé, Nathan e
%Carmelah na esperança de que continuem a trilhar o caminho da verdade.

\chapter{Introdução}

\begin{flushright}
\textsc{giuseppe nahaïssi}
\end{flushright}

\noindent{}``De Moisés a Moisés não houve outro igual a Moisés''. Esta frase
singular está gravada na pedra tumular de Maimônides, ou Moisés filho
de Maimon, na cidade de Tiberíades --- não longe das margens do mar da
Galileia, onde o grande mestre está sepultado. Ela foi escrita por
seus discípulos querendo dizer que, desde Moisés filho de Amram --- o
maior legislador Hebreu, autor por inspiração divina dos dez mandamentos
e da lei, a Torá --- até Moisés filho de Maimon --- o seu maior
intérprete e autor da Mishná Torá ---, não houve outro que
pudesse ser comparado em grandeza e sabedoria ao primeiro Moisés.

Quando Moisés foi incumbido por Deus de empreender a gigantesca tarefa de resgatar os filhos de Israel da escravidão do Egito e levá-los à terra prometida, em Canaã, era necessário muito mais que libertar escravos para torná-los aptos a guiar seus próprios destinos como indivíduos e como nação para os séculos que viriam. Precisava uni-los sob um estatuto que iria reger seus modos de vida, criar costumes e tradições nacionais, dar a eles um conjunto de leis que deveria servir tanto na guerra como na paz e conter instruções para a conquista, para o tratamento a ser dispensado aos inimigos, aos cativos de combates, aos estrangeiros e peregrinos e aos aliados, conter o estatuto e divisão da terra, o direito civil de paz, a obrigatoriedade do ensino, as leis do casamento, as uniões lícitas e ilícitas, as leis do divórcio e da herança, as instruções de higiene e de saúde e o regime alimentar. Era também necessário afirmar este povo num código moral e ético apoiado num único Deus universal, intangível, justo e onipotente; um código sobre o bem e o mal, a sua relatividade e impactos sobre a vida humana, sobre a virtude e a promiscuidade, a verdade e a mentira, a pureza e as impurezas. Exige também a luta constante contra os inimigos da moral estabelecida, personificados pelos filhos de Amalec\footnote{Ver \emph{Amalec} no glossário.} que cultuam o caos e, por outro lado, insiste que os homens se aproveitem corretamente das dádivas divinas, de serem férteis, de crescerem e multiplicarem-se. Neste código são determinados os deveres e os direitos dos cidadãos, dos sacerdotes, dos reis e dos juízes. O amor ao próximo só é superado pelo amor a Deus, provedor de todas as coisas.

Esta obra monumental intitulada Torá ou \emph{a Lei} é a
primeira constituição escrita e distribuída a um povo para lhe servir
de estatuto e guia. Escrita sobre pergaminho e dividida em cinco capítulos
ou livros (Gênesis, Êxodo, Levíticos, Deuteronômio, Números), contém 613
artigos de lei mais conhecidos como os 613 preceitos ou 613 mandamentos.
Estes preceitos são divididos em duas grandes seções: os preceitos
positivos ou \emph{Farás} e os negativos ou \emph{Não farás}. São 248 os
preceitos positivos e 365 os negativos, pois usará as 248 partes que
compõem o seu corpo para fazer os seus deveres para com Deus e seu
próximo e se recusará a fazer o mal os 365 dias do ano. Os dez
mandamentos resumem os 613 preceitos.

Esta obra é de tal profundidade que não somente guardou o povo de Israel
por 3500 anos como a mais velha nação do mundo, mas influenciou a
humanidade com sua essência. Os ``dez mandamentos'' são aceitos e
respeitados em todos os tribunais do planeta como a lei magna. Duas das
maiores religiões se inspiráram na moral deste código, o Cristianismo e
o Islamismo, que juntas contam hoje com mais de 2 bilhões de fiéis. Os
cinco livros de Moisés são os cinco primeiros livros da Bíblia,
considerada sagrada e divina tanto por cristãos, muçulmanos como judeus,
e base de sua ética e de suas orações.

Esta lei também chamada de \emph{Lei de Moisés} ou \emph{Lei Mosaica}, aplicada na disciplina do povo hebreu durante os quarenta anos no deserto de Sinai,
foi a lei dos juízes de Israel, dos profetas; teve sua corte instalada
durante o período dos reis,
e sua autoridade excedia a do próprio monarca. Com a destruição do
primeiro Templo e de Jerusalém, houve uma enorme necessidade de se
preservarem a lei e seus valores, e começou a era dos estatutos
acadêmicos e a redação do resto dos livros que compõem a Bíblia hoje.
Com a volta a Tzión, setenta anos mais tarde, reestabeleceu-se a Suprema
Corte ou Sanhedrim,\footnote{Ver \emph{Sanhedrim} no glossário.} mais conhecida como \emph{Sinédrio}, onde os juízes
decidiam todas as questões. Com a invasão helenística, judeus e gregos
se influenciaram mutuamente; a dialética da filosofia grega tornou-se
presente nas academias israelitas e a moral judaica invadiu os gregos.
Com a dispersão criada pela invasão romana, era imperativo salvar a lei.
Rabbi Iohanan, filho de Zaccai, vendo desmoronarem o Templo e a Suprema
Corte, conseguiu escapar do desastre e montou uma academia na cidade de Iávne,
Israel, formando setenta e dois mestres ou rabis.

O Talmud ou os estatutos da jurisprudência da Lei Mosaica foi a coluna
mestre da sustentação da lei no exílio. As cortes rabínicas espalhadas
pela comunidade na diáspora resolviam as questões entre os judeus e,
desta forma, os rabinos mantiveram a comunidade, o ensino, a moralidade
e a fé.

Em mais de 2500 anos de jurisprudência em todos os estilos, os mestres
da Lei Mosaica, os rabis, compuseram uma obra monumental que os judeus
convencionaram chamar de lei oral, pois foi proferida verbalmente
pelos juízes e acadêmicos e, posteriormente, redigida pelos escribas. A
lei oral explica a lei e contém os relatos e as opiniões de centenas de
mestres. Maimônides decidiu colocar em um só trabalho o sumo da lei oral e
escreveu a Mishiná Torá pela segunda vez.
Para resumir decidiu escrever este trabalho em dois volumes com curtas
explicações sobre cada preceito e referências de onde encontrar mais
sobre o assunto na literatura talmúdica.

Esta é a tradução do livro dos preceitos de Maimônides, intitulado em
hebraico \emph{Sefer Hamitzvot}. Recomendamos a sua leitura com muita
atenção e meditação e, apesar de muitos dos preceitos não caberem mais
nos dias de hoje, pelo menos de maneira literal --- como os serviços dos
sacerdotes no templo, ou dos castigos a serem aplicados ---, no bojo de
cada um deles há uma lição de vida, por exemplo no preceito de como retirar
cinzas do santuário em que está embutido o respeito ao passado: já que a
juventude foi o fogo de ontem e a velhice tem que ser respeitada,
cuidada e levada em segurança. Quanto aos castigos por açoitamento, o
exílio ou até a pena de morte, hoje nos servem para dimensionar a
gravidade do crime cometido.

A maneira mais prática de estudar a Lei Mosaica é começando pelo livro
dos preceitos de Maimônides, e isso foi exatamente o que eu fiz quando,
no ano de 1985, festejavam-se os 850 anos do seu nascimento e os rabinos
do mundo inteiro --- e especialmente o Rabi Menahem Meyer Schneersohn,
rabino chefe do movimento Chabad ---, recomendaram o estudo da obra do
grande mestre.

No estudo dos textos em hebraico, tendo ao lado a tradução em inglês, eu
tomava notas em português para minha memória. As notas se acumularam e
alguns amigos, sabendo do meu trabalho, me recomendaram sua publicação.
Refinei o texto e parti para a tradução literal. A fidelidade ao
espírito do texto foi fundamental e, muitas vezes, achei conveniente
sacrificar o vernáculo a favor do assunto.

\section{Maimônides: vida e obra}

No ano de 1166, aos 31 anos, desembarca em Alexandria, no Egito, Moisés,
filho de Rabi Maimon, o homem que viria a ser conhecido como o Moisés
do Egito e respeitado pelo mundo todo como uma das mais relevantes
figuras do pensamento judaico.

Os estudiosos costumam se referir a esse homem como o grande sábio cujo legado foi decisivo 
para a manutenção da fé e da união do povo judeu no século XII.
Sua glória se estendeu também aos círculos não judaicos, e nos meios cultos de
Bagdad ele passou a ser considerado como um dos mais eminentes homens da
época. Maimônides foi o responsável, entre outros feitos, pela
subordinação do valor moral ao valor teórico, e pela análise
contemplativa abstrata como objetivo final, ao invés do julgamento
concreto dos atos --- por mais que a introdução da inteligência no espírito
religioso já houvesse sido feita na época tanaica, e que o valor
religioso da compreensão talmúdica já fosse conhecido pelo povo havia
muito tempo. A superioridade da contemplação sobre o rito e a moral
constitui o pilar central de seu pensamento e, embora o Talmud
ensine que não são as pesquisas e sim os fatos que importam, ele insiste
nas pesquisas porque tem a profunda convicção de que o amor de Deus é
tanto maior quanto mais desenvolvida e aperfeiçoada for nossa
inteligência.

Talmudista, codificador da Torá, filósofo, místico, matemático,
médico e dono de um talento literário ímpar, ele iria transformar a
comunidade judaica do Egito, trazer uma nova ordem para os judeus do
mundo e viria a ser o único pensador da Idade Média cujas teorias
exerceram influência significativa ao mesmo tempo sobre pensadores cristãos, muçulmanos 
e judeus de sua época. Sua obra foi, aliás,
frequentemente citada por filósofos como Tomás de Aquino, Alberto, o
Grande, Roger Bacon, Inácio de Loyola, Alexandre de Halle, Nícolas de
Coves, Leibniz, Baruch de Espinoza e muitos outros.

Homem de personalidade densa e complexa, Maimônides estabeleceu para si
mesmo uma conduta estrita e complicada, mas soube simplificar o que
desejava transmitir de forma tal que seus leitores pudessem
compreendê-lo facilmente. Fanático pela brevidade, Maimônides se
preocupa sempre em construir parágrafos claros, sem nenhum interesse em engrandecer 
seus pensamentos nem glorificá-los com uma retórica exagerada. 
São suas estas palavras: ``Se me fosse possível resumir o Talmud inteiro em uma
frase, não quereria fazê-lo em duas''. Enquanto algumas de suas obras
são muito eruditas, outras são escritas de maneira muito fácil e de
compreensão extremamente simples. Quando interpelado sobre o porquê
dessa diferença entre uma obra e outra, ele respondeu: ``O pão e o
leite são para as crianças, e a carne e o vinho são para os adultos''.
Fiel a essa filosofia, Maimônides conduz seu aluno, fazendo-o crescer em
suas mãos, e levando-o a passar por vários estágios de ``pão e leite''
primeiro, para que ele possa chegar a compreender e a apreciar ``a carne
e o vinho'' da metafísica, a ciência superior que lhe abriu os caminhos
na sua busca da verdade.

Ele acredita que todos os homens devotos, sem exceção, que vivam de
acordo com a virtude e que sigam os mandamentos bíblicos e mantenham
sempre boa conduta, serão recompensados com o mundo futuro,
independentemente de seu credo ou religião. Respeita e tem íntimos
amigos no mundo islâmico, e costuma afirmar que a doutrina cristã não
tem nenhuma contradição com o judaísmo, pois ela também reconhece a
força e a necessidade dos mandamentos e da moral bíblica, e que seus
adeptos, se quiserem aprofundar-se no estudo contemplativo dos textos,
descobrirão a verdade.

Além de ter sido considerado o maior talmudista do século, esse homem,
que se torna o médico da corte do Egito servindo ao grão-vizir de
Saladin, Al Fadil, e depois ao sultão Al Afdal, gozava da reputação de
ser o melhor médico de seu tempo. Os relatos sobre seu saber e sua
competência se estendem de tal forma que chegam ao conhecimento do rei
Ricardo Coração de Leão, da Inglaterra, que o convida para ser seu
médico particular. Maimônides, no entanto, prefere permanecer no Egito,
pois lá ele acumula também o cargo de \emph{naguid}, e pode utilizar-se de sua
digna posição para proteger a comunidade judaica através do mundo
islâmico. O \emph{naguid} era o líder e o porta-voz dos judeus egípcios,
nomeado pelo sultão, que representava a autoridade moral e política
de todas as comunidades judaicas no país dos fatímidas. Ele era escolhido
dentre a comunidade rabínica, mas tinha também direito de justiça sobre
os caraítas e os samaritanos.

Mas seu brilhantismo e seu sucesso não são fortuitos. Na sua juventude,
Maimônides aprende astronomia com o filho do célebre astrônomo Ibn
Afla, de Sevilha, estuda o Almagesto, o tratado astronômico de
Ptolomeu, as proposições de Álgebra, o tratado das seções cônicas, a
geometria, a mecânica, a medicina, tratados astrológicos, bem como
livros teológicos de outras religiões, para adquirir um conhecimento
geral das religiões de seu tempo. Aprofunda-se ainda nas doutrinas
filosóficas de Aristóteles, Filo, Afrodísias, Themistius, Alfarabi, Gazali, 
Gaon Saadia, Bachija, Rabi Iehuda Halevi, Rabi Abraham bar Chiha e de Rabi Abraham 
Ibn Ezra, mas baseia suas explicações metafísicas mais profundas no pensamento aristotélico.

Aos 16 anos Maimônides escreve uma introdução à lógica, e aos
23 uma dissertação matemática e astronômica, tratando das questões
principais da determinação do calendário judaico. Pelos lugares por onde
passa durante o êxodo em que vive durante vinte anos, estuda
atentamente a flora dos países e se interessa por suas arquiteturas. No
Egito ele estuda os usos e as particularidades da língua, os hábitos e
a moral dos judeus egípcios, e chega a redigir um comentário sobre essas
observações.

O RAMBAM, sigla de Rabi Moisés ben Maimon, ou simplesmente Maimônides,
do grego ``filho de Maimon'', nasceu em Córdoba, na Espanha, em 1135,
filho do \emph{daián}\footnote{\emph{Juiz rabínico}, do hebraico.} Rabi Maimon, 
descendente de uma longa linhagem de \emph{daianim}, remontando a Rabi Iehuda 
Hanassi, o autor da Mishná, sábio que havia atingido a perfeição moral e
intelectual, e que, por sua vez, era descendente direto da casa real de
Davi.

Tendo ficado órfão de mãe ao nascer, Maimônides se revela uma criança que
desde cedo se habitua a entregar-se a meditações profundas sobre a vida
e a morte e a confiar-se sozinho a Deus. Para isso, ele se refugia na
sinagoga durante a semana, na parte reservada às mulheres, para meditar,
onde tem certeza de que ninguém virá interrompê-lo.

\section{As disputas religiosas da Espanha medieval}

No ano de 1148, quando o jovem Moisés completa 13 anos, os Almóadas,
liderados por Abde Almumine, invadem a cidade de Córdoba. Esses
Almóadas, ou \emph{confessores da unidade}, eram uma tribo berbere que
conquistara o poder na Espanha e no Marrocos, após vinte anos de lutas
sangrentas. Abde Almumine era o sucessor de Ibn Tumarte, jovem e ardoroso
muçulmano que vivia no sudoeste do que atualmente é o Marrocos e que,
insatisfeito com os ensinamentos teológicos básicos que ali lhe haviam
sido ministrados, decidiu aprofundar-se no assunto, indo para isso às
faculdades de Córdoba, de Meca e de Bagdad, onde entrou em contato com
os ensinamentos de Gazali. Depois de ter aprendido a ciência teológica
oriental, Ibn Tumarte voltou para sua região natal e, declarando-se
descendente de Maomé, liderou uma guerra santa contra as altas esferas
do governo as quais, segundo ele, eram as responsáveis --- entre outras
tantas coisas inadmissíveis --- pelo relaxamento religioso, luxo e
pela decadência moral da corte e da alta sociedade, representação
material de Deus --- o que era uma blasfêmia ---, e pelo \emph{politeísmo},
que ele atribuía aos antigos fiéis da África do Norte, os quais
afirmavam, tal como os cristãos, a pluralidade do ser divino. A
revolução teológica, aliada ao desejo de conquista, levou a um sucesso
sem precedentes, e o reino dos Almóadas se estendeu da Síria ao oceano
Atlântico. Eles destruíam as igrejas e as sinagogas, e aos povos que
não aceitavam converter-se à \emph{verdadeira religião islâmica},
propagada por eles, restava a opção entre a imigração ou a morte.

O jugo dos Almóadas se fazia sentir na mesma época em que as Cruzadas
partiam da França e da Alemanha, a fim de conquistar a Terra Santa e
apossar-se do túmulo de Cristo. Intolerantes, os Cruzados arrasavam, na
sua marcha para Jerusalém, tudo o que encontravam de não cristão,
massacrando em seu caminho as populações judias indefesas. Mais uma vez
a história se repetia e os judeus se defrontavam com uma nova e grave
crise de identidade: dobrar-se aos conquistadores islâmicos ou à
barbárie dos cruzados em marcha. O fanatismo religioso imperava tanto
no levante quanto no ocidente, e continuar professando o credo judaico
representava um risco de vida. Talvez esse tenha sido um dos momentos
mais difíceis e trágicos da história da sobrevivência do judaísmo; eram
necessárias grandes forças para sustentar a fé. E um dos personagens
mais importantes dessa época foi Maimônides, pois a clareza de suas
ideias e de seus escritos mantiveram acesas no povo judeu as chamas da
crença e da liberdade da ciência e do conhecimento para enfrentar esse
período singular e sinistro.

Preferindo o êxodo ao massacre ou à renúncia de sua fé, a família de
Rabi Maimon sai de Córdoba quando os Almóadas chegam. Depois de dez
anos de vida errante, nos quais passam por diversas cidades do sul da
Espanha, chegam a Fez, capital do Marrocos --- a África do Norte
sempre fora o local de asilo para os judeus que fugiam das perseguições
religiosas na Espanha. Munido de coragem ímpar, Rabi Maimon opta por
Fez, onde os \emph{confessores da unidade} haviam instalado sua corte,
porque tem esperança de ser introduzido ao líder deles, o califa
Abde Almumine, homem que tinha reputação de interessar-se pelas coisas 
do espírito e que procurava cercar-se de sábios, a fim de expor-lhe o 
pensamento judeu relativo a Deus e tentar assim obter uma mudança na 
política do governo em relação aos judeus.

Muitos judeus, no entanto, para escapar à morte ou ao abandono do lar,
optavam pela conversão aparente à doutrina dos \emph{confessores da
unidade}. Essa conversão, que os obrigava a uma vida dupla vergonhosa
e sem dignidade, era algo suportável apenas enquanto eles contassem com
a providência divina e enquanto essa situação de sofrimentos tivesse
algum sentido compreensível. Contudo, esse conflito se tornava um
sofrimento intolerável quando o sustentáculo moral da fé começava a
desmoronar, abalando sua confiança em Deus e em si próprios. A crença na
unidade absoluta de Deus, apregoada pelos Almóadas, parecia, ao povo
mais simples, idêntica à doutrina judaica, e eles começavam a acreditar
que a missão do povo eleito tinha chegado a seu fim e a se perguntar se
o profeta Maomé não era realmente superior a Moisés.

Extremamente preocupado com essa situação, Rabi Maimon decide escrever,
em 1159, uma carta em árabe que ele envia às comunidades judias da
África do Norte, recordando-lhes a infalibilidade divina, a existência
de uma aliança permanente entre Deus e Israel, a superioridade de Moisés
e o profundo significado da prece. Nessa carta ele diz: ``Um rei, ao
demitir um de seus funcionários, tem o hábito de nomear imediatamente
outro, a fim de transmitir-lhe o cargo e as funções do primeiro. Um
marido que repudia sua mulher, geralmente coloca outra em seu lugar, e
lhe dá os adornos e a cama da primeira. O sinal da mudança consiste em
dar ao sucessor os direitos e as honras do predecessor. Onde está,
afinal, o povo a quem o Eterno apareceu, ao qual Ele deu uma
Torá e sobre o qual Ele espalhou sinais de sua benevolência,
semelhantes àqueles com os quais Ele favoreceu os judeus?''. Rabi
Maimon refere-se aqui às maravilhas do \emph{Êxodo}, quando Deus libertou o
povo de Israel da opressão faraônica, às dez pragas impostas ao inimigo,
à abertura do mar para a passagem do povo eleito, ao milagre do maná
para sua alimentação diária durante quarenta anos, à presença de suas
colunas de fogo para guiá-los à terra prometida e à entrega de Seus
mandamentos a viva voz, desde o cume do Sinai, não através de um
emissário, nem de um intermediário, nem sequer de um anjo, mas através
d'Ele próprio, em sua glória. 

A carta segue assim: ``Enquanto nenhum
outro povo puder mostrar sinais de clemência e de benevolência
similares, só se pode considerar como falatório o abandono de Israel em
favor de outro povo. Ainda que vivamos incessantemente na angústia,
ainda que pela manhã desejemos a chegada da noite, e à noite a chegada
da manhã, ainda assim devemos pensar na seguinte profecia: `Deus não
esquecerá a aliança que ele fez com teus pais'\,''. E ``\ldots{} Deus não 
quer destruir, mas purificar Israel. Devemos considerar nossa aflição atual 
como um ensinamento, como uma prova. Como acreditar na ira do Eterno, no 
repúdio de Israel? A missão de Moisés, nosso incomparável mestre, prova a 
eleição de Israel\ldots{} O sucesso material não prova o valor de uma nação. 
A preferência de Deus por Moisés e por Israel, preferência confirmada em 
várias ocasiões pela benevolência divina, garante a efetivação das promessas 
do Senhor, mas não se pode saber quando ocorrerá sua realização, trazida 
pelo arrependimento e pela oração\ldots{}''

Seguindo o exemplo de seu pai, aos 24 anos, Maimônides decide sair da
vida de estudo e trabalho solitários que levava até então para
redigir um tratado sobre um julgamento feito por um rabino que condenara
como traidores do judaísmo os judeus que se convertiam em aparência à
doutrina dos Almóadas. Maimônides considera que professar a fé
islâmica para continuar vivo não é apostasia, baseado no fato de que 
outros judeus haviam tido
atitudes similares anteriormente, sem que por isso tenham provocado a
ira do Senhor, e que o mais importante é a sobrevivência do povo de
Israel. Nessa sua primeira obra, publicada em Fez, ele diz: ``Se as
colunas do mundo, Moisés, Elias, Isaías, e até mesmo um anjo, foram
punidos porque ousaram elevar a voz contra Israel, então quanto não
deve ser censurado um homem suficientemente audacioso para dizer que
nas comunidades judaicas há malfeitores, pagãos, homens indignos de
prestar testemunho a Deus, ateus! \ldots{} Então esse rabino estrangeiro e de
pouca reflexão não sabia que os que se convertem pela força não pecam
por negligência? \ldots{} O Senhor não os abandonará; Ele não os rejeitará:
Ele nunca menosprezou a miséria dos infelizes''.

Orientando-se pelo provérbio \emph{Não se consegue nada sem penar}, o
objetivo a que Maimônides se fixa é o de \emph{compreender} Deus, até onde
isso for possível ao homem. Para tanto ele julga que deve iniciar pela
lógica, seguida das ciências matemáticas, das naturais, e por fim da
metafísica, numa progressão do concreto para o abstrato. Assim, ele se
dedica ao estudo de várias ciências para exercitar seu espírito e suas
capacidades intelectuais, a fim de discernir a lógica demonstrativa dos
outros métodos de raciocínio. Ele se consagra com zelo ao estudo das
ciências gerais, mas apenas como elementos necessários à aquisição de
uma cultura global, e não por uma necessidade interna, pois esta ele
satisfaz através do estudo da Torá.

\section{Os mistérios da tradição}

Na sua adolescência, Maimônides procura compreender e aprofundar-se
nos mistérios proféticos e suas reflexões a esse respeito formam o ponto
culminante de toda sua vida intelectual. Ele tem a convicção de que a
sabedoria, a integridade e a modéstia são os atributos que preparam o
espírito do homem para o advento da profecia. Acredita também que, por
mais profundo que possa parecer o saber acumulado por um homem, ele deve
colocar tudo nas mãos do Todo Poderoso, pois o conhecimento é um dom de
Deus. No entanto, sua tendência especulativa o leva a buscar
incessantemente o sentido da existência individual, já que a crença na
necessidade do pensamento é a ideia condutora de sua vida. Para ele o
pensamento é sagrado, e ele só consegue aceitar a crença através da
inteligência e do entendimento, afirmando que a inteligência filosófica
é uma condição \emph{sine qua non} para a imortalidade da alma e para a
participação no reino eterno.

A solução, a resposta, não é o essencial para Maimônides. A disciplina e 
a dedicação são as qualidades fundamentais de sua inteligência, daí sua 
reafirmação sempre de não desejar construir um sistema filosófico, e sim 
apenas facilitar o caminho para alcançar o conhecimento de Deus. Ele considera 
como fraqueza de espírito acomodar-se na crença tradicional toda vez que a lógica se inclina diante da religião, como, por exemplo, no caso das posições dogmáticas, e diz, a esse respeito, o seguinte: ``\ldots{} se alguma coisa não tem um motivo compreensível e se ela não traz nenhum benefício nem evita nenhum mal, por que diríamos daquele de quem ela é o objeto de crença ou a regra de conduta, que ele é sábio e inteligente, e que ele ocupa uma posição elevada? Que haveria de surpreendente para os povos nisso? \ldots{} Diríamos que\ldots{} o homem é mais perfeito que seu criador, pois o homem falaria e agiria visando a um determinado objetivo, enquanto que Deus, ao invés de agir dessa forma, nos ordenaria\ldots{} a fazer o que não tem nenhuma utilidade para nós e nos proibiria ações que não nos trazem nenhum prejuízo''. 

Ele conhece os limites da razão, mas considera como imperativo viver sob o domínio dela, pois para ele a inteligência não é um local para descarregar suas dúvidas, pois já faz parte do reino de Deus. Em 1158, durante sua fuga através da Espanha, ele inicia a redação de seu \emph{Comentário sobre a Mishná}, obra que leva sete anos para ser concluída e na qual, à guisa de prefácio ao décimo capítulo do tratado \emph{Sanhedrin}, ele faz uma exposição da tradição e da doutrina do judaísmo. Redigida para proporcionar uma resposta às dificuldades e às necessidades do povo judeu na época, e com o intuito de preservar a unidade de seu povo, que ameaçava desmoronar diante de tantas provações e de tantos conflitos, essa introdução levou Maimônides a sacrificar seus princípios liberais e a propor um quadro quase dogmático que representa, sob seu próprio ponto de vista, o verdadeiro credo do judaísmo, e que pode ser resumido da seguinte forma:

\begin{enumerate}
\def\labelenumi{\arabic{enumi}.}
\item
  Eu acredito plenamente que o criador, que o seu nome seja bendito, é o
  criador e guia de todos os seres, que Ele, e apenas Ele, criou, cria e
  criará todas as coisas.

\item
  Eu acredito plenamente que o criador, que o seu nome seja bendito, é
  um e único e que não existe nada mais único do que Ele; que apenas Ele
  é nosso Deus, era, é e será.

\item
  Eu acredito plenamente que o criador, que o seu nome seja bendito, é
  etéreo; que Ele não tem nenhuma propriedade antropomórfica; que nada
  é parecido com Ele.

\item
  Eu acredito plenamente que o criador, que o Seu nome seja bendito, é
  primeiro e último.

\item
  Eu acredito plenamente que o criador, que o Seu nome seja bendito, é o
  único a quem é apropriado rezarmos e que não é apropriado rezar a mais
  ninguém.

\item
  Eu acredito plenamente que todas as palavras dos profetas são
  verdadeiras.

\item
  Eu acredito plenamente que a profecia de Moisés, nosso mestre, que
  esteja em paz, foi verdadeira, que foi ele o pai de todos os profetas,
  daqueles que o precederam como daqueles que o seguiram.

\item
  Eu acredito plenamente que a totalidade da Torá que está em
  nossas mãos foi dada a Moisés, nosso mestre, que descanse em paz.

\item
  Eu acredito plenamente que esta Torá não será modificada e que
  não haverá outra Torá dada pelo criador, bendito seja seu
  nome.

\item
  Eu acredito plenamente que o criador, bendito seja seu nome, conhece
  todas as ações e todos os pensamentos de todos os seres humanos, como
  está escrito: ``É Ele que amolda o coração de todos, Ele que capta
  todas as suas ações''.\footnote{Salmos 33:15.}

\item
  Eu acredito plenamente que o criador, bendito seja o seu nome,
  recompensa aqueles que observam seus mandamentos, e pune aqueles que
  os transgridem.

\item
  Eu acredito plenamente na vinda do Messias, ainda que possa tardar, no
  entanto espero a cada dia pela sua vinda.

\item
  Eu acredito plenamente que haverá ressurreição dos mortos no momento
  em que assim o desejar nosso criador, bendito seja seu nome, exaltada
  seja a Sua recordação para todo o sempre.
\end{enumerate}

Incorporados depois à liturgia de várias populações judias, esses
princípios foram recebidos com grande alegria pelas comunidades
carentes, transformando-se em hinos de glorificação a Deus. O mais
famoso deles é o \emph{Igdál}, de autor desconhecido, com força poética sem
igual e totalmente baseado nas palavras do grande mestre. Este hino é
cantado até os dias de hoje em todas as sinagogas.

Para elaborar seu \emph{Comentário sobre a Mishná}, Maimônides se
inspira, tanto no pensamento como na forma, na própria Mishná,
redigida por seu antepassado, o rabino Iehuda Hanassi. A Mishná, escrita no
início do século III, é a obra que condensa as explicações e os
resultados dos estudos e das pesquisas intelectuais que haviam sido
feitos até então em torno da Sagrada Escritura. Esse trabalho separa o
conteúdo da doutrina daquilo que está diretamente ligado ao texto da
Bíblia, tal como havia sido transmitido pela tradição, e o reduz a
regras e a decisões.

De acordo com seu autor, o \emph{Comentário} deveria trazer novidades e
melhorias aos estudos. Com o passar do tempo a Guemará, que é um
comentário elaborado sobre a Mishná, havia suplantado o estudo
desta; grandes sábios e profundos conhecedores da \emph{Guemará}
ignoravam a Mishné, e Maimônides os recriminava por isso. Os
objetivos de Maimônides ao redigir o seu \emph{Comentário} eram,
portanto o de restabelecer a Mishná na sua posição preponderante
e o de fazer um resumo dos debates que estão nessa obra, de modo a
ter-se uma referência rápida e fácil sobre todas as questões da Lei e de
modo a servir aos debutantes como uma preparação para a dialética
superior.

Publicado em 1168, o \emph{Comentário} foi concebido da seguinte forma:
as introduções sistemáticas, escritas livremente, se distinguem
totalmente das explicações curtas dos textos da Mishné. A
amplidão e a profundidade da sabedoria do autor aparecem aí de maneira
mais forte e mais clara do que nas partes explicativas, necessariamente
limitadas pelo próprio texto da Mishné. Maimônides coloca, nesse
seu trabalho, a seguinte advertência: ``Leia várias vezes meu livro e
reflita atentamente. Se sua imaginação lhe fizer crer após a primeira
leitura ou mesmo após a décima que você o compreendeu, então ela o
enganou. Pois você não deve fazer a leitura deste livro de maneira
rápida: eu não o escrevi simplesmente, como é o caso às vezes; ele é o
fruto de muitas pesquisas e reflexões''.

A publicação dessa obra, no entanto, não provoca aparentemente nenhuma
polêmica; Maimônides não possuía as condições habitualmente necessárias
para ser reconhecido como uma autoridade: passar pela escola, tornar-se
professor ou \emph{gaon}, ou fazer parte dos trabalhos efetuados numa academia
representativa. Mas ele não queria dever nada a uma posição nem a uma
dignidade.

Com a morte de Abde Almumine em 1163, o qual havia sido de certa maneira
tolerante com a comunidade judia de Fez, seus sucessores retomam as
perseguições aos judeus, e a família de Maimônides emigra então
para Ceuta, cidade situada à beira-mar, na extremidade norte de
Marrocos, e que naquela época ocupava um lugar preponderante no mundo
das artes e das ciências. Mas os Almóadas disputavam acirradamente o
governo de Ceuta e em meio a tumultuados golpes e contragolpes
políticos, em 1165, Rabi Maimon decide partir novamente, desta vez com
destino à Terra Santa.

\section{As viagens de Maimônides}

A Terra Santa era na época o ponto mais cobiçado do ocidente nas
disputas religiosas. As Cruzadas haviam conquistado o país e quando a
família de Maimônides chega lá o governador é o franco Amaury, homem
ambicioso e ávido de poder e de riquezas. Contudo Nur ad-Din, governador
muçulmano da região do Tigre, decide partir para a Guerra Santa contra
os cristãos das cruzadas e faz uma aliança com o Egito e a Síria para
cercar a Terra Santa. É nesse momento que Maimônides se dirige para lá.

Maimônides desembarca na florescente cidade de São João de Acre em 16 de
maio de 1165, em cujas ruas se ouviam todos os idiomas do oriente e do
ocidente, e é asilado pelo Rabi Jafet, que presidia a vida das duzentas
famílias judias da cidade.

Na Terra Santa as antigas tradições judaicas se haviam perpetuado, pois
tinham sido transmitidas de maneira ininterrupta dentro do país. Maimônides 
descobre ali que a ordem das quatro partes do Pentateuco nos
\emph{teffilin}, tal como ele a havia aprendido em Córdoba, difere da ordem
estabelecida de acordo com a opinião de conhecidos \emph{gaonin}, que eram os
chefes das grandes academias da Babilônia, e de acordo com antigos
textos do Talmud, e decide então corrigir a ordem de seus \emph{teffilin}. Esse
fato é um acontecimento importante, sobretudo por tratar-se ele de um
homem que dedica sua existência à pesquisa, à explicação e à
representação da lei judaica, pois é uma demonstração de grande
humildade e de aceitação do conhecimento daqueles sábios no que se
refere à tradição.

Em outubro de 1165 Maimônides se dirige à Jerusalém para rezar diante
do Muro das Lamentações, e de lá ele vai para Hebron, a fim de rezar
sobre o túmulo dos patriarcas.

Mas não tardaria muito para que a depravação que ele observava nos
imigrantes estrangeiros que chegavam à Terra Santa o convencesse a
partir de lá, já que ele acreditava que ``É inato no homem curvar-se,
com relação aos seus hábitos e atos, aos costumes dos países e dos
amigos ou companheiros que ele encontra ali''. Todo homem, diz ele, deve
procurar assimilar os hábitos e a conduta dos sábios. Para isso, é
preciso que ele faça tudo que estiver ao seu alcance para viver junto
aos justos e afastar-se dos maus, a fim de não correr o risco de se
integrar a eles e de se sentir inclinado a agir como eles: ``Se acontece
de vivermos num lugar onde os habitantes não seguem o caminho correto, é
preciso imigrar para um lugar onde os habitantes sejam devotos e tenham
bons costumes''.

Diante disso, ele parte com direção ao Egito onde, ao contrário do que
ocorria nos outros países muçulmanos, os judeus podiam contar com a
tolerância religiosa dos califas fatímidas. Dentre os 50 mil habitantes
do país naquela época havia 3 mil famílias judias que viviam em paz e
gozavam de completa liberdade civil e religiosa. Chegando a Alexandria
em 1166, a família de Maimônides se depara com uma cidade internacional,
que embora não mais fosse a capital do Egito nem a segunda cidade do
mundo, continuava sendo grande e bela, ``cidade do comércio de todos os
povos'', como descrevia o comerciante Benjamim de Tudela, para onde se
dirigiam comerciantes tanto da Europa cristã, como do sul da Arábia, da
África do Norte e das Índias, e onde cada nação possuía seu próprio
armazém.

Mas Rabi Maimon pouco desfruta dessa merecida paz, depois de tantos
anos de peregrinação, pois vem a falecer poucos meses depois de terem
chegado ao Egito. Davi, seu filho mais moço, assume então a manutenção
material da família, dedicando-se ao comércio de pedras preciosas, e
liberando assim Maimônides dessa preocupação para que ele pudesse
continuar seus estudos.

Se as comunidades judias da Espanha e do Marrocos estavam ameaçadas de
extinção pela fé implacável dos Almóada, e as da Terra Santa pelas
cruzadas e pelos maus costumes, o conforto material e a comodidade em
que viviam os judeus do Egito também representavam uma ameaça, só que
neste caso para a vida intelectual, como acontece toda vez que a
opulência se instala na existência do homem. Eles ali negligenciavam a
observação das leis religiosas, ignoravam os sábios e a falta de
conhecimentos se generalizava, provocando a decadência religiosa, como
se podia observar pelo desenvolvimento e prosperidade da doutrina
caraíta no país.

Os caraítas constituíam uma seita judia separatista, que desprezava a
tradição oral legada pelas instituições rabínicas, e que se guiava ao pé
da letra pela Torá. Ao contrário do que ocorria nos outros
países, onde essa seita estava em vias de extinção, seu distanciamento
da maioria da comunidade judia fazia com que ela prosperasse no Egito, 
pois lá encontrava um ambiente
favorável, junto aos maometanos, que acreditavam estarem os caraítas
mais próximos do Islã do que os judeus seguidores das leis talmúdicas.
Vários judeus se juntaram aos caraítas e foram recompensados com favores
políticos, já que essa seita gozava da confiança dos fatímidas xiitas. A
influência deles se fazia sentir, e aos poucos desapareciam na
comunidade os ritos tradicionais, fazendo com que os próprios rabinos
se sentissem impotentes com relação ao progresso dessa assimilação.

Como estimasse, por tudo isso, que os caraítas eram inadequados para
executar os deveres religiosos dos judeus tradicionais, Maimônides
começou a propor uma cisão de cultos a fim de eliminá-los
definitivamente da vida religiosa judia. Isso provocou a cólera dos
caraítas a tal ponto que ele se viu forçado a partir de Alexandria.

Portanto, por volta de 1168, Maimônides parte para Fostat, onde se erige
atualmente a antiga cidade do Cairo. Dois anos depois de sua chegada,
ele já ocupa ali um rabinato e trabalha intensamente na ajuda aos
necessitados. Sabe-se, por exemplo, que em 1169 ele envia diversas
cartas circulares às comunidades egípcias a fim de conseguir o dinheiro
para o resgate dos judeus que haviam sido aprisionados por Amaury,
governador franco de Jerusalém, para evitar que eles fossem vendidos
como escravos.

Depois de ter tentado afastar o perigo que os caraítas representavam
para a vida do judaísmo autêntico, Maimônides se lança na reforma dos
costumes no que se refere às preces feitas na sinagoga. Ele percebera
que, enquanto o ``hazan'' fazia em voz alta, na sinagoga, a oração
silenciosa da comunidade, as pessoas conversavam ao invés de escutar com
recolhimento a oração. Considerando isso um desrespeito a Deus, ele
ordenou que se desse início a essa oração primeiramente em voz alta, e
que ela fosse seguida depois por todos em silêncio e com recolhimento,
em vez do que se fazia até então. Essa melhoria, que ele ousa impor a
despeito da ordem da oração talmúdica, encontra o apoio e o
reconhecimento dos sábios contemporâneos e é aceita no Egito.

Incansável na sua busca da perfeição, ele deseja também unificar os
ritos dos dois grupos em que estavam divididos os judeus do Egito, ou
seja, os babilonianos e os jerusalemitas. Os babilonianos dividiam a
Torá de maneira que ela pudesse ser lida completamente num ano,
enquanto que os Jerusalemitas utilizavam um ciclo de três anos. Cada um
desses grupos tinha sua sinagoga e não tinha outros ritos em comum a
não ser o Simchat Torá e Shavuot. Essa oposição de usos dentro da
comunidade judia chocava Maimônides que, como judeu espanhol, estava
habituado a uma liturgia uniforme, fundada na ordem das orações de
\emph{amram}. A diversificação dos ritos pareceu-lhe imprópria, e embora ela se
apoiasse na tradição local, Maimônides sentiu que a lógica da lei e do
pensamento deveria substituir a comodidade dos hábitos. Contudo, se
defrontou aqui com a oposição de Zuta, o \emph{naguid} da época, homem
ambicioso que aproveitou a ocasião para afastar aquele estrangeiro
atrevido e audacioso, dizendo que a reforma que Maimônides queria
instaurar no Egito devia ser considerada como um ato de hostilidade
contra o governo. Não resta então a Maimônides alternativa a não ser
abandonar temporariamente essa luta e afastar-se de Fostat por algum
tempo --- que ele aproveita para iniciar sua obra magna, o
\emph{Mishná Torá}.

A morte de seu irmão Davi no naufrágio de um navio, por volta de 1171,
representa um grande golpe para Maimônides. Além de minar
irremediavelmente sua saúde, essa grande dor desencadeia uma crise
decisiva em sua alma e acarreta uma mudança profunda em seu pensamento
e na sua visão do mundo. Junto com o irmão desaparece também
toda a fortuna da família, e
como estimasse que nenhum sábio devesse viver às custas da comunidade
para poder prosseguir seus estudos, ``pois nem na Torá, nem nos
livros posteriores dos sábios não encontra alguma coisa que apoie essa
tese'' --- como ele próprio diz ---, Maimônides decide então tornar-se
médico e ganhar dessa forma seu sustento. Ele dava, dessa forma, o
exemplo e recomendava também a todos os estudiosos sábios que ganhassem
seu pão graças a seu trabalho e não às custas da religião. Médico
devoto, escreve um juramento para todos os médicos, judeus ou não, no
qual reafirma o dever que eles têm de curar, e faz uma oração para que
Deus lhes preste assistência e intervenha por eles. Essa oração diz o
seguinte:

``Oh Deus! O Senhor formou o corpo do homem com uma bondade infinita!
--- O Senhor uniu nele inúmeras forças que trabalham incessantemente
como tantos instrumentos a fim de preservar em seu todo esta casa
maravilhosa, contendo uma alma imortal, e essas forças atuam com toda a
ordem, concordância, e harmonia imaginável. Mas se fraqueza ou paixão
violenta perturbam esta harmonia, estas forças agem uma contra a outra
e o corpo volta ao pó de onde ele veio. O Senhor então envia ao homem
seus mensageiros, as doenças, que anunciam a aproximação do perigo e
pede que se prepare para vencê-las. A eterna providência me apontou para
cuidar da vida e da saúde de suas criaturas. Que o amor a minha arte me
deixe agir em todos os momentos, que a avareza e a mesquinhez, tanto
como a sede da glória ou da reputação, não tomem conta dos meus
pensamentos, porque sendo inimigos da verdade e da filantropia poderiam
me decepcionar e me fazer esquecer da minha meta de fazer o bem aos Seus
filhos. Me enriqueça com força de alma e mente, para que ambas estejam
prontas a servir ao rico e ao pobre, ao bom e ao mau, amigo e inimigo e
que jamais enxergue o paciente senão como um ser igual, adoecido.

Se médicos mais cultos que eu desejam me aconselhar, me inspirar
confiança e obediência, aceito de bom grado, pois o estudo da ciência é
maravilhoso. Um só ser não pode enxergar tudo. Que eu seja moderado em
tudo, exceto na sabedoria da ciência; que eu seja insaciável até o ponto
certo; permita-me ter sempre a força e a oportunidade de corrigir
minhas aquisições, sempre estendendo meu domínio, porque a sabedoria
não tem limite e o espírito do homem também se estende infinitamente,
para que diariamente se enriqueça com novos conhecimentos. Hoje ele
pode descobrir os erros de ontem, e amanhã ele pode obter nova
iluminação sobre o que ele deu certeza hoje.

Deus, o Senhor me apontou para cuidar da vida e da morte de suas
criaturas; eis-me pronto para minha vocação.''

Por volta de 1172, comovido e preocupado com a situação dos judeus que
viviam no Iêmen, acossados que estavam pelos \emph{confessores da unidade}
no oeste, pelos xiitas no leste, e desnorteados depois do anúncio que
havia sido feito por um pobre e ingênuo lunático da breve chegada do Messias, 
Maimônides lhes envia três cartas alentadoras, que ele
redige em árabe para que pudessem ser compreendidas por todos. Numa
delas ele diz o seguinte: ``Devemos ficar satisfeitos por sofrer todos
esses infortúnios, essas perseguições, esse exílio, a perda de nossos
bens e as injúrias de todos, pois todas essas misérias são uma honra que
Deus nos concede''. Ele diz ainda que todo o mal que se sofria era um
sacrifício que se levava ao altar e lhes recorda que Deus havia
prometido que nenhuma opressão duraria muito tempo e que seu povo nunca
seria destruído. Afirma também que o que eles estavam passando naquele
momento não era um sofrimento, e sim um mal preliminar que anunciava o
reino do verdadeiro Messias. Maimônides acredita que a inveja é a
verdadeira motivação que leva os outros povos a perseguir e a oprimir
constantemente os judeus: não podendo atacar-se ao Todo Poderoso por
ter Ele escolhido o povo de Israel como herdeiro e guardião de Seu
estatuto e de sua doutrina, eles se lançam contra o povo em si, numa
Guerra Santa que dura desde os tempos de Amalec.

Uma vez concluído seu \emph{Comentário sobre a Mishná}, Maimônides
concebe a Mishná Torá, uma obra que o ocuparia de 1170 a 1180, e
que deveria guiar os leitores através das obscuridades e das
imprecisões do Talmud, no qual está contida a vida interior e exterior
do judaísmo.

\section{O livro dos mandamentos}

O Talmud (tanto o da Babilônia quanto o de Jerusalém) se
desenvolveu durante séculos como o comentário da Mishná. Uma
enorme quantidade de opiniões e de novos conhecimentos foram expressos
ali e fixados como continuação do texto da Mishná. Esse trabalho
só terminou no século V, mas ficou rapidamente demonstrado que o povo
era incapaz de compreender esse alto ensinamento e é por isso que apenas
um pequeno número de pessoas se consagrava ao estudo do Talmud.
As perseguições e os sofrimentos que vinham castigando os judeus anos a
fio os obrigavam a deixar os estudos religiosos cada vez mais em segundo
plano e a dar prioridade à preservação das próprias vidas; a sapiência
dos sábios e o raciocínio dos filósofos se perdiam, as explicações sobre
o Talmud que os \emph{gaonim} davam, e que eles julgavam estar ao
alcance de todos, começavam a não mais ser compreendidas, assim como os
próprios textos do Talmud, da \emph{Sifrá}, dos \emph{Sifris} e
da \emph{Toseftá}, pois a compreensão dessas obras exige uma grande
inteligência, uma alma preparada e extensos e aprofundados estudos.

Maimônides constata que o povo em si não tinha a sua disposição um
código onde pudesse encontrar regras seguras, sem mistura de
controvérsias e de opiniões. Assim sendo, ele deseja expor em sua obra,
numa linguagem clara e breve, o que é proibido e o que é permitido, o
que é puro e o que é impuro, bem como tudo o que se refere às questões
da Torá, tudo isso para que a lei possa ser conhecida por todos,
sem deixar dúvidas. Ele quer que a Lei esteja, em palavras claras, na
boca de todos os homens e faz sua exposição de maneira direta e
didática, buscando dar um esclarecimento simples e satisfatório a
questões que, de outra forma, poderiam ser interpretadas erroneamente
pelo povo. É assim, por exemplo, que ele explica que o preceito que nos
ordena temer a Deus implica, na realidade, não o medo do Eterno, mas
sim o respeito ao nome de Deus, a reserva em utilizá-lo, o cuidado para
não cometer uma blasfêmia, glorificando-o e louvando-o cada vez que ele
for pronunciado. Portanto, o temor ao Eterno significa a obrigação de
santificar Seu nome e de estar sempre alerta para não profaná-lo, o que
envolve, entre outras coisas, o dever de se deixar matar antes de
renegar o Senhor em benefício de um deus pagão ou antes de entregar aos
gentios um de seus irmãos israelitas para que ele seja morto ou
desonrado. Santifica também o nome do Senhor aquele que ``se afasta de
toda transgressão ou observa os preceitos sem ser levado a fazer isso
por alguma consideração de ordem profana, terror, temor, ou busca de
reconhecimento'', ao passo que profana o santo nome ``todo aquele que
transgride espontaneamente e na ausência de qualquer tipo de pressão,
por desdém e com o intuito de escandalizar, nem que seja apenas um dos
preceitos anunciados pela Lei''. Assim, com este que pode ser considerado
um código metódico de referência para o dia a dia, Maimônides tem a
convicção de poder levar seu leitor a descobrir, passo a passo, qual é a
atitude correta e qual é o caminho a ser seguido para alcançar a
perfeição do corpo e da alma que o Senhor espera de nós.

A grande dificuldade para a execução de sua obra estava principalmente
no fato de que não havia nenhuma preparação anterior que pudesse
ajudá-lo em seu trabalho, pois, de acordo com suas palavras, ``A
sapiência dos sábios de nossa época consiste em julgar a verdade de uma
sentença não de acordo com seu conteúdo, e sim de acordo com sua
conformidade com a sentença de um predecessor, sem examiná-la''.
Maimônides decide então executar sua obra como uma codificação, como um
resumo sistemático, e não como um comentário, que era o que o
Talmud fazia com relação à Mishná. Decide também não citar
as opiniões discutidas e refutadas, mas sim fornecer apenas as decisões
que têm força de lei. Ele deseja expor todas as doutrinas da
Mishná e do Talmud, sem dar o nome do autor de cada uma
delas após cada citação, mas dizendo apenas que todas as frases da
Torá que constituem a lei oral haviam sido transmitidas por este
e aquele, desde Ezra e nosso mestre Moisés.

Seguindo fielmente a tradição judaica, Maimônides se limita estrita e
logicamente ao Talmud, tomando por lei o que ele encontra
explicitamente decidido. Como na maioria dos casos as questões estão
dissertadas, sem apresentar uma recomendação firme e final, ele próprio
as resolve, e é exatamente isso o que há de mais importante em seu
trabalho, não a compilação em si. Os princípios e os métodos que ele
aplica, com uma lógica minuciosa, em suas decisões pessoais, são de uma
elevação de pensamento que prepara um terreno novo para todos os séculos
seguintes. Ele distingue no Talmud os elementos \emph{haláchicos}
obrigatórios e os elementos \emph{hagádicos}, que não o são. Isso lhe permite
ter uma independência de opinião completa em relação às decisões dos
sábios talmúdicos que não eram de origem religiosa, sempre que ele não
pôde prová-las cientificamente. O equilíbrio resultante da independência
e da fidelidade de seu espírito original está repleto da mais legítima
e autêntica autoridade e é a obra-prima de sua atitude intelectual.

Sua objetividade científica faz com que ele atribua a mesma importância
a todas as matérias da \emph{Halachá}, sem levar em conta sua relação com a
atualidade. Ele fala sobre todos os preceitos, inclusive aqueles que não
estão mais em aplicação desde a destruição do Templo, e lastima que
ninguém mais se interesse em pesquisar ou conhecer essas leis, pois
assim elas terminam por ser esquecidas.

Maimônides tem consciência de escrever um livro definitivo: ``Ninguém
terá necessidade de ajuda para conhecer a lei judaica, se ele tiver
minha obra que forma uma coletânea completa de todas as instituições,
usos e decretos, desde Moisés até o fim da redação do Talmud,
incluindo as explicações posteriores dos \emph{gaonim}'', e é por isso que ele
intitula seu código de Mishná Torá, que significa ``repetição da
lei''. Isso representa uma revolução e uma reforma no ensino da
religião: quem estudasse primeiro as Escrituras e depois o código de
Maimônides conheceria toda a doutrina da tradição oral e poderia, em
tese, se abster de pesquisar em outras obras. Através da utilização de
seu código ganha-se o tempo que outrora se dedicaria ao estudo do
Talmud, tempo esse que podia ser consagrado aos estudos
filosóficos. Maimônides deseja despertar o interesse dos estudantes pela
posição filosófica do problema para depois dirigir seus pensamentos para
a metafísica, pois da mesma forma que ele também prefere a análise
contemplativa ao julgamento das ações, ele prefere o estudo da
metafísica, \emph{raízes da doutrina}, ao estudo da dialética do
Talmud, \emph{ramificações da doutrina}, embora ele considere
indispensáveis as partes dialéticas do Talmud e considere que as
partes hagádicas são a fonte da ciência filosófica.

Ele abre mão da discussão, que é o que acontece no Talmud, em
favor da decisão. Na ciência talmúdica as pesquisas analíticas não são
levadas em consideração; sua pedagogia inculca \emph{a doutrina pela
doutrina} e seu estudo conduz a uma teoria, e não a uma decisão
essencial para a prática. É esse aspecto do ensino e dos métodos de
pensamento que Maimônides pretende reformular.

Aquela era a primeira vez na história que um homem ousava recolher numa
só obra a totalidade que constitui a ciência hebraica. As tentativas
feitas nesse sentido até então haviam fracassado, devido à imensidade de
material disperso existente. O caráter de sua codificação consiste em
citar a ideia em vez do acontecimento, a lei em vez do caso, a coisa em
vez das pessoas, trocando a história pela especulação, a situação
concreta pela abstração.

Maimônides utiliza um método novo para a forma do livro. Ele devia
repartir um grande número de prescrições, de leis e de decisões
especiais nos 613 \emph{compartimentos} de seu livro, correspondentes cada
um deles a um dos preceitos derivados da Torá. Resolve, então,
redigir 613 parágrafos, ordenando-os em 83 seções e distribuindo-os em
14 livros. Para que sua grande obra pudesse ser lida com maior clareza e
para preservar a unidade do conjunto, Maimônides decide escrever um
prefácio onde apresentaria um resumo dos \emph{Tariág mitzvot}, ou dos
613 preceitos divinos mas, ao estudar as enumerações já existentes
desses preceitos, chega à conclusão de que nunca se haviam estabelecido
normas claras para se chegar à classificação do que era e do que não era
um preceito, e de que muitas vezes o conteúdo havia cedido lugar à
forma. Sendo os \emph{613 preceitos} o fio condutor da vida e da crença do
povo judeu, já que eles representam os desejos do Eterno expressos na
Torá, era de suprema importância que o povo soubesse com
clareza e com precisão quais eram esses desejos para que pudesse
cumpri-los.

Assim sendo, Maimônides inicia a redação de seu \emph{O livro dos
preceitos divinos}, o \emph{Sêfer hamitzvot}, com o intuito de esclarecer
e ordenar os 613 preceitos, e de dirimir as dúvidas dos leigos. Esse
livro, escrito em linguagem precisa e acessível, proporcionava também
aos jovens da época um caminho mais fácil para adquirir algum
conhecimento a respeito dos preceitos não mais em aplicação e das coisas
do Templo e do Santuário, pois Maimônides estava preocupado com o fato
de que não só eles ignoravam esses aspectos da tradição, como também
não demonstravam nem ao menos alguma curiosidade ou interesse pelas
coisas do passado.

Maimônides divide o livro em duas partes. Na primeira estabelece os
14 Fundamentos ou os princípios lógicos que utiliza para determinar o
que é e o que não é preceito. Na segunda apresenta de maneira
detalhada os 613 preceitos, divididos em 248 positivos e 365 negativos.
Ao longo de todo o livro, Maimônides se empenha em esclarecer seu
raciocínio, citar fontes, argumentos, literatura rabínica --- sem no
entanto deixar de apresentar sempre uma conclusão clara e decisiva,
para que não restem dúvidas ao leitor quanto ao que deve e ao que não
deve ser feito.

Os princípios são apresentados agrupados de acordo com os assuntos neles tratados. 
Assim, vemos agrupadas as obrigações ---
bem como as proibições --- referentes ao homem para com Deus, para com
seus semelhantes, para com sua família, relativas ao Templo, 
impurezas, festas religiosas, cultivo da terra, justiça, Estado, etc.

\emph{O livro dos preceitos divinos} serviu de base para muitas outras
obras, inclusive para o \emph{Livro do ensino}, o \emph{Sêfer Hachinuch}, a
grande obra do talmudista espanhol do século XIII Aaron Halevi, que
trata especificamente desses preceitos.

\section{Judaísmo como fé e filosofia}

Uma vez concluído esse livro, em 1170, Maimônides pôde então voltar a
dedicar-se de corpo e alma ao Mishná Torá, obra que foi
recopiada por escribas profissionais e que se espalhou pelo mundo
inteiro, conquistando sábios, estudiosos, rabinos e juízes. Várias
comunidades o adotaram como código. O Talmud era uma obra vasta
e complexa; o Mishná Torá foi escrito de maneira ordenada e
clara, com uma linguagem fácil, onde os leitores encontravam a verdade
e aprendiam o que a doutrina da moral encerrava de mais profundo.
Maimônides deixou o Mishná Torá como seu grande legado
organizador que iria exercer uma influência definitiva na vida de seu
povo. Havia muito tempo a voz de um homem não tinha tal influência sobre
os judeus.

Contudo, em breve começam a aparecer os opositores de Maimônides, que
colocam em questão sua autoridade como legislador. Eles censuram a obra
de Maimônides porque ele expõe as regras legais sem citar a origem, sem
dar o nome do autor, sem provas nem peças de apoio. Até então só se
reconheciam como obrigatórias as decisões dos \emph{gaonim}, pois elas se
apoiavam não apenas nas suas qualidades pessoais, mas também na
autoridade das academias, e os judeus reconheciam a instituição e não a
pessoa. Pesa ainda contra ele o fato de ter escrito, na introdução de
sua obra, que depois de ter estudado seu Código se poderia renunciar ao
estudo da literatura pós-bíblica. Isso foi interpretado como uma
tendência a afastar o Talmud das escolas, o que representava uma
profanação e um perigo. Maimônides deixa passar a ocasião de contestar a
tempo essa falsa interpretação de sua doutrina e acaba por começar a
ser considerado por alguns como herege!

Depois de ter mantido uma correspondência preliminar com o grande
mestre, na qual solicitava ser aceito como discípulo e poder estudar com
ele, em 1185 chega a Alexandria o jovem Iossef Ibn Iehuda. Maimônides,
que antes nunca se havia interessado pelo ensino público e imediato, e
cuja necessidade de ensinar estava voltada mais para a redação do que
para a exposição oral, decide no entanto transmitir seus ensinamentos a
esse rapaz que viera de Ceuta, e que acaba por suscitar a simpatia do
mestre. A decisão de concordar em transmitir seus ensinamentos a um
aluno se prendeu principalmente ao fato de que o número dos que se
interessavam pela ciência e pela filosofia no Egito era muito reduzido.
Há muito ele esperava por uma pessoa a quem pudesse transmitir seus
ensinamentos e descobertas, e a afinidade e a afeição que ele acaba
desenvolvendo por Ibn Iehuda é tão grande que ele passa a chamá-lo de
\emph{meu filho}.

Esse jovem, para quem Maimônides era o representante de uma ciência mais
elevada, se interessava sobretudo pela questão da interpretação das
Escrituras e desejava iniciar-se nos mistérios da ciência superior.
Embora tendo durado pouco menos de dois anos, o convívio com Ibn Yehuda
foi muito estimulante para Maimônides, que procura fazer com que esse
jovem entusiasta e impaciente por atingir a \emph{ciência interior}
seguisse o caminho do estudo metódico e lento. Ele lhe diz, numa carta:
``E quando você fez comigo seus estudos de lógica, eu depositava minha
esperança em você, e o julgava digno de revelar-lhe os mistérios dos
livros proféticos, para que você compreendesse aquilo que os homens
perfeitos devem compreender.''

Contudo, antes que Maimônides chegasse a introduzir seu discípulo nos
assuntos metafísicos, este se muda para Alep, por motivos ignorados. Mas
não parte sem antes obter de seu mestre a promessa de que ele redigiria
um tratado no qual responderia as questões que o preocupavam e cuja
resposta tinha ido buscar junto a ele.

E é assim que, para cumprir a promessa feita ao seu único discípulo,
Maimônides redige, de 1187 a 1190, e não sem grande hesitação, a sua
última obra, o \emph{Guia dos Perplexos}, um trabalho que deveria
``explicar os pontos obscuros da lei e manifestar o verdadeiro sentido
das alegorias, que estão acima da inteligência comum''. Esse é um
trabalho difícil para ele, pois trata-se de desvendar os mistérios que
lhe parecem invioláveis, e se algumas considerações incidentais em suas
obras anteriores já haviam provocado grandes oposições, o que não
poderia resultar da apresentação completa de suas concepções
filosóficas? Ele chega, no entanto, à conclusão de que, apesar de tudo,
é importante que o livro seja escrito, e explica nele porque tomou essa
decisão: ``\ldots{} eu sou o homem que, vendo-se apertado numa arena estreita
e não encontrando a forma de ensinar uma verdade bem demonstrada, a não
ser de uma maneira que convenha a um só homem notável e que desagrade a
dez mil ignorantes, prefere falar para essa única pessoa, sem prestar
atenção à reprovação da grande multidão, e espera tirar esse único
homem notável da confusão em que ele caiu e mostrar-lhe o caminho para
sair de sua perplexidade, a fim de que ele se torne perfeito e de que
obtenha o repouso''. Em outro trecho dessa mesma obra ele diz também
que ``A verdade não se torna mais verdadeira pelo fato de todo mundo
acreditar nela, tampouco pelo fato de todo mundo discordar dela''.

Ele estava preocupado com o desinteresse dos judeus pela filosofia. A
maioria deles sentia uma separação entre a fé e o saber, entre o
conteúdo da revelação e as doutrinas filosóficas. Para Maimônides, no
entanto, a Hagadá é também uma das fontes da ciência filosófica: tudo o
que ali se encontra em forma de parábolas coincide com os ensinamentos
da filosofia abstrata. Com o \emph{Guia dos Perplexos} Maimônides torna
possível o acesso da razão àqueles aspectos da Torá que não
estão ao alcance da capacidade humana. Ele deseja guiar ``o homem
religioso, no qual a verdade de nossa lei está estabelecida na alma e se
tornou um objetivo de crença, que é perfeito na sua religião e nos seus
costumes, que estudou as ciências dos filósofos e conhece os diversos
assuntos delas, e que foi atraído e guiado pela razão humana, para
fazê-lo entrar em seu domínio''. Sua obra-prima filosófica se destina,
portanto, apenas aos sábios e aos estudiosos, pois os leigos e os
ignorantes jamais poderiam compreender as revelações nela contidas. Na
introdução desse livro ele diz: ``Meu pensamento vai guiá-los no caminho
do verdadeiro e vai torná-lo mais fácil. Venham, caminhem pela sua
senda, vocês que vagam no campo da religião! O impuro e o ignorante não
passarão por ele; ele será chamado de caminho secreto.''

Essa obra visava, em primeiro lugar, proteger os estudiosos das
comunidades judias da sedução que as filosofias árabe e grega exerciam
no século XII, e a grande originalidade de Maimônides nesse trabalho foi
a de estabelecer um diálogo entre o mosaísmo e a filosofia, ao invés de
se limitar a utilizar-se de seus conhecimentos filosóficos para fazer a
apologia do judaísmo. Ele não renuncia a nenhuma das tradições do
pensamento judeu, nem tampouco alimenta a ilusão de poder \emph{conciliar}
a verdade bíblica e a verdade filosófica. Ao invés disso, ele confronta
as duas tradições, de maneira a sobrepô-las. Assim, ele se outorga a
missão de guiar os estudiosos para o conhecimento metafísico o qual,
segundo ele, é uma possessão original do judaísmo que havia sido perdida
durante o exílio, e é essa perda que torna o exílio tão trágico. Ele tem
a convicção de que o renascimento da compreensão mais elevada, obtida
graças à introdução da filosofia nos estudos religiosos, é o fato
libertador que conduzirá ao acontecimento messiânico, teoria essa que,
aliás, acredita-se ser ele o primeiro a introduzir naqueles tempos de
exílio.

Seu \emph{Guia dos Perplexos} se espalha extraordinariamente na
literatura mundial e seu sistema é de grande importância para Nicolau de
Coues, Leibniz e Spinoza. A negação da tese da eternidade do mundo em
favor de sua perpetuidade, a modificação da doutrina de Aristóteles
sobre Deus, sua profetologia independente, sua explicação do significado
dos preceitos divinos e seu método de explicação da Bíblia entram no
sistema de pensamento dos escolásticos cristãos como Tomás de Aquino.
Alexandre de Halle, Alberto, o Grande e Inácio de Loyola também
aceitaram suas teorias como elementos de seus sistemas.

Embora ciente de que essa obra não esclareceria todas as dúvidas,
Maimônides espera que afaste as principais. A aspiração a conhecer
Deus, necessidade de seu pensamento presente desde sua juventude, faz
com que ele se lance à procura de um sistema metafísico. Ele adota a
razão como método de meditação, esperando chegar ao conhecimento de Deus
pela eliminação de negativas e das imperfeições. A própria existência do
indivíduo se torna para Maimônides o ponto de partida para o
conhecimento e para a compreensão de Deus. Ele conclui que ``Não há na
realidade no ser nada além de Deus e todas as suas obras; estas são
tudo o que o ser encerra fora d'Ele. Não há nenhuma outra forma de
conceber Deus a não ser por Suas obras; são elas que indicam sua
existência e o que se deve crer a seu respeito''.

Uma questão que preocupa muito Maimônides é como ascender à profecia, o
mais alto grau de perfeição possível a um homem. Ele faz no seu
\emph{Guia} a exposição de seu ponto de vista a esse respeito, e diz que
ela é um dom que Deus concede, quando Ele quer, a alguns poucos homens
que Ele escolhe dentre aqueles que possuem os três requisitos básicos: a
perfeição do pensamento, atingida através da especulação e da
concentração da inteligência no estudo e na busca do conhecimento de
Deus; a perfeição da imaginação, que foge ao controle do homem e que
depende de um cérebro perfeitamente formado; e, por último, a perfeição
do caráter e da moral, que se alcança libertando-se o espírito do desejo
de se obter satisfações terrenas e erradicando a ambição pelo domínio e
pelo poder. Essas condições são indispensáveis porque a revelação
profética é uma emanação de Deus que se manifesta, a um homem justo,
primeiramente por intermédio do \emph{intelecto ativo}, em relação à
faculdade de pensar, e depois em relação à imaginação.

Se essa emanação divina atingir unicamente a faculdade racional do
homem, seja porque sua faculdade imaginativa não foi perfeitamente
formada, seja porque a emanação foi insuficiente para alcançar a
faculdade imaginativa desse homem, ele pertencerá à categoria dos sábios
que dedicam suas vidas ao estudo e às pesquisas. Se ela agir apenas
sobre a faculdade imaginativa, seja porque a faculdade racional é
primitiva ou porque ela foi pouco exercitada, o que teremos é um homem
que pertence ao grupo onde se classificam os legisladores, os
adivinhos, os ocultistas, os videntes e os feiticeiros. Essas pessoas
têm sonhos e visões semelhantes aos dos profetas, e isso faz com que
eles acreditem ser um deles. 

De acordo com as palavras de Maimônides,
elas pensam ``que adquiriram ciências sem ter estudado, e provocam uma
grande confusão nas coisas importantes e especulativas, misturando, de
uma maneira surpreendente, as coisas verdadeiras e as quimeras. Tudo
isso ocorre porque sua faculdade imaginativa é forte, enquanto a
faculdade racional é fraca e não obteve absolutamente nada''. Mas se
essa emanação divina se espalhar pela faculdade racional e daí passar à
faculdade imaginativa, tendo ambas atingido a perfeição que explicamos
anteriormente, esse homem estará na categoria dos verdadeiros
profetas. E como recebe o homem uma profecia? À exceção de Moisés, a
quem Deus falou diretamente, os profetas recebem essas revelações através de
um mensageiro de Deus, um anjo, que lhes fala num sonho ou numa visão,
situação esta em que seus sentidos ficam paralisados e a emanação
divina se espalha por sua faculdade racional, passando daí para a
faculdade imaginativa, fazendo com que ela se aperfeiçoe e entre em
funcionamento. Maimônides escreve que ``Às vezes a revelação começa por
uma visão profética; depois essa agitação e essa forte emoção
decorrentes da ação perfeita da imaginação, vão aumentando e então
ocorre a revelação''.

Convencido, portanto, de que ``o vigor humano do espírito não é
suficiente para se alcançar o conhecimento de todas as coisas'',
Maimônides crê que a profecia é a única capaz de decifrar todas as
verdades para a limitada inteligência humana, e de conduzir à
compreensão da criação do mundo e da origem das espécies. Para ele o
sábio está no cume da hierarquia humana, uma vez que este se prepara
para receber as revelações de Deus: ``pois apenas aquele que chegou à
perfeição especulativa pode obter a seguir outros conhecimentos, quando
o intelecto divino se extravasa sobre ele''. O verdadeiro profeta é
aquele que prevê os acontecimentos, com absoluta verdade, a médio e a
longo prazo, e cujas previsões de coisas boas sempre se realizam.
Contudo, caso o profeta tenha previsto um mal que não venha a ocorrer,
isso não significa que ele não é mais um profeta, mas sim que Deus, na
sua infinita bondade, teve misericórdia de seu povo e mudou o mal em
bem.

Para ilustrar seu pensamento a esse respeito ele conta no seu
\emph{Guia} uma parábola através da qual afirma que o estudo da
literatura talmúdica, fundada na tradição e sagrada para todos os
judeus, não é a chave que abre a porta do \emph{palácio} -- onde vive o
\emph{soberano}, ou seja, Deus ---, mas que filosofar a esse respeito,
aprofundando-se cada vez mais nesse estudo, é o caminho que abre essa
porta e que conduz a Deus. Sem dúvida, essa nova ordem das potências
espirituais é extremamente audaciosa, sobretudo vinda de um homem que
consagra o melhor de sua existência ao estudo das tradições talmúdicas,
e só pode ser legítima se corresponder a um autêntico desejo dele de
estabelecer uma nova ordem das coisas. Essa nova ordem é o coroamento
da obra de Maimônides.

Quando escreveu seu \emph{Comentário sobre a Mishná}, ele pensava ainda
que tudo o que existe no universo só existe em função dos homens.
Contudo, essa concepção antropocentrista vai se dissipando ao longo dos
anos para dar lugar à crença de que o homem em si não pode ser imaginado
como o último objetivo do desenvolvimento universal, de que a espécie
humana é bem pouco com relação ao mundo superior das esferas e dos
astros, e de que não há nada que indique que o mundo só existe por causa
do homem: ``\ldots{} se a Terra inteira não passa de um ponto imperceptível
com relação à esfera das estrelas, qual será a relação da espécie
humana com o conjunto das coisas criadas? E como então alguém dentre nós
poderia imaginar que elas existem em seu favor e por sua causa, e que
elas devem servir-lhe como instrumentos?''.

Essa nova concepção do homem como mero componente de um cosmo ordenado e
perfeito abre um caminho que o leva a encontrar a explicação filosófica
do mal, explicação essa que ele renunciara a descobrir aos 25 anos,
dando-se como justificativa o fato de que os sábios que o haviam
precedido também não o haviam conseguido. Agora, vários anos após a
morte de seu irmão Davi, a impetuosidade de sua juventude desaparece,
deixando lugar a uma calma e ponderação maiores, e essa maturidade de
espírito lhe traz consolo e paz de espírito. Maimônides destrói o mal
filosoficamente e conclui que sua ocorrência não passa de exceções
esporádicas, quando se leva em conta a harmonia de toda a criação
divina: ``Todo ignorante imagina que o universo inteiro só existe em favor 
de sua pessoa, como se não houvesse nele nenhum
outro ser a não ser ele próprio. Portanto, se o que lhe acontece é
contrário aos seus desejos, ele julga que todo o ser é o mal; mas se o
homem considerasse e concebesse o universo, e se soubesse quão pequeno é
o lugar que ele aí ocupa, a verdade lhe apareceria claramente''.

Maimônides passa a definir o mal como uma simples privação, como algo
que aparece em decorrência da ausência de uma ação. Ora, como uma ação
só pode resultar em algo que venha a existir e não numa privação, o mal
não pode ser resultante de uma ação direta se essa ação não existe, e
portanto esse mal também não existe. Caso assim não fosse, como diz
Maimônides, ``se alguém produz uma matéria incapaz de receber
determinada capacidade, poder-se-ia dizer que ele fez tal privação; da
mesma forma, se alguém tivesse sido capaz de salvar uma pessoa da
morte, mas tivesse se abstido de fazê-lo e não a tivesse salvo,
poder-se-ia dizer dele que ele a matou''. Ele conclui que o bom é o ser,
aquilo que existe, e o ruim é o não ser: ``No homem, por exemplo, a
morte é um mal e é sua não existência; da mesma forma, sua doença, sua
pobreza, sua ignorância são males em relação a ele e são privações de
capacidade\ldots{} A destruição nada mais é do que a privação da forma''.

Ele classifica os males em três tipos: o primeiro é o que recai sobre
determinados indivíduos, devido a nossa própria natureza humana, e que é
o responsável pelas doenças e deformações congênitas ou resultante de
alterações ocorridas na natureza, tal como os terremotos, conforme ele
próprio exemplifica. Esses males são resultantes naturais da própria
imperfeição da matéria da qual o homem é feito, pois ``a coisa mais
eminentemente perfeita que possa formar-se a partir do sangue e do
esperma é a espécie humana, com sua bem conhecida natureza de ser vivo,
racional e mortal''. Mas levando-se em conta o total da humanidade em
todos os tempos, tais tipos de males não passam de meras exceções. A
segunda categoria abrange os males que os homens se infligem uns aos
outros, tal como a tirania e a barbárie provocadas, entre outras causas,
pelas paixões e pelas divergências de opiniões e de crenças; segundo
Maimônides: ``Esses grandes males\ldots{} são todos decorrentes de uma
privação, pois todos eles resultam da ignorância, ou seja, da privação
da ciência\ldots{} porque o conhecimento da verdade faz cessar a inimizade
e o ódio e impede que os homens se façam mal uns aos outros\ldots{}''. E há,
finalmente, os classificados no terceiro tipo, que são os mais
frequentes e que constituem os males que acontecem aos seres humanos
como decorrência de seus abusos dos prazeres mundanos, tal como a
bebida, a comida e a luxúria. Ele escreve que ``a maioria dos males que
atingem os indivíduos provêm deles próprios, quero dizer, dos indivíduos
humanos, que são imperfeitos. Se sofremos é por causa dos males que nós
mesmos nos infligimos espontaneamente, mas que atribuímos a Deus\ldots{}''
Este último tipo de mal é o mais nocivo, já que além de atingir o corpo
ele prejudica também a alma, seja porque, sendo uma força corporal, ela
é influenciada diretamente pelas alterações ocorridas no corpo, ou
porque a alma acaba por se familiarizar com o supérfluo e a se habituar
a ele, levando o homem a desenvolver uma ambição sem termo que o faz
buscar riquezas e grandezas não essenciais e totalmente desnecessárias.

Maimônides explica também no \emph{Guia dos Perplexos} que toda vez que
Deus deseja manifestar sua vontade, seja para desencadear algum
acontecimento, seja para intervir no desenvolvimento dos fatos, Ele o
faz por meio dos anjos, as \emph{inteligências separadas} que Ele cria e
das quais se serve para reger o universo. Assim sendo, os anjos são
todas as forças propulsoras e todas as faculdades, tal como a força
formadora, que existe no esperma e que dá formas ao feto, e como a faculdade 
que faz com que um animal aja de uma
determinada maneira, num dado momento, de acordo com os desígnios de
Deus, assim como se vê na citação que Maimônides faz das Escrituras:
``Meu Deus enviou seu anjo e fechou a goela dos leões, que não me
fizeram nenhum mal''.\footnote{Daniel 6:22.}

As Escrituras dizem também que os anjos formam um terço do universo.
Isso significa que eles são uma das três coisas criadas por Deus e
existentes fora d'Ele, a saber: as \emph{inteligências separadas} (ou anjos),
os corpos das esferas celestes, e a matéria que se encontra abaixo das
esferas celestes, e da qual é feito tudo o que existe no nosso mundo. As
Inteligências separadas, elementos incorpóreos criados por Deus são os
intermediários entre Deus e todos os corpos celestes e fonte de
extravasamento de benefícios e de luz para os corpos das esferas
celestes. Quanto a estas esferas, Maimônides diz que elas são entidades
que possuem uma alma, no sentido de que elas foram postas em movimento
pelas Inteligências criadas por Deus e que elas desenvolvem o desejo de
mover-se eternamente em círculos, o ponto mais alto da perfeição que um
corpo possa alcançar, já que esse é o movimento perpétuo.

De acordo com Maimônides, as esferas celestes são nove: uma
que engloba tudo, uma onde estão as estrelas fixas, e uma esfera para
cada um dos sete planetas existentes (de acordo com os conhecimentos da
época). Cada uma dessas esferas também emana benefícios e forças que
regem a matéria de nosso mundo, \emph{o mundo do nascimento e da
corrupção}, como diz Maimônides. Segundo os filósofos, temos a
evidência da influência que a lua exerce sobre as águas e da que os
raios do sol exercem sobre o elemento do fogo e do calor. Baseado
nisso, ele chega à conclusão de que cada esfera ``pode possuir um dos
quatro elementos de tal forma que tal esfera seja o princípio de força
de tal elemento em particular, ao qual, graças a seu próprio movimento,
ela dê o movimento do nascimento. Assim, portanto, a esfera da lua seria
o que move a água; a esfera do sol o que move o fogo; a esfera dos
outros planetas, o que move o ar (e seus movimentos múltiplos, sua
desigualdade, seu recuo, sua retidão e sua estação produzem as diversas
configurações do ar, sua variação e sua rápida contração e dilatação);
e, finalmente, a esfera das estrelas fixas seria o que move a terra, e é
talvez por isso que esta última se move com dificuldade, por receber a
impressão e a mistura, porque as estrelas fixas têm o movimento
lento''.

Ele conclui também que é como se todas essas forças constituíssem a
força de um só corpo, já que o universo todo é um único indivíduo. Tudo
o que nele existe é elaborado não a partir de um ato concreto e
particular, mas sim a partir do que Maimônides chama de ``extravasamento
divino'', a fonte inesgotável de bondade, de criação e de continuidade
do universo, que se espalha como numa cascata, derramando-se primeiro
sobre os anjos, que extravasam seus benefícios sobre as esferas
celestes as quais, por sua vez, os extravasam sobre os corpos
perecíveis. De acordo com ele, ``\ldots{} tal como foi demonstrada a
incorporeidade do Criador, e tal como foi estabelecido que o universo é
obra sua e que Ele é sua causa eficiente, \ldots{} foi dito que o mundo vem
do extravasamento de Deus e que Deus extravasou sobre ele tudo o que
nele ocorre. Da mesma forma, foi dito que Deus extravasou sua ciência
sobre os profetas. Tudo isso significa que essas ações são a obra de um
ser incorpóreo, e a ação de um ser assim é chamada de extravasamento''.

Com a mesma clareza que Maimônides utiliza para explicar que o
\emph{extravasamento} divino é o ponto de origem e de renovação de todas
as coisas, corpóreas ou incorpóreas, ele escreve também um capítulo no
qual faz a declaração, espantosa a princípio, de que Deus não tem nenhum poder sobre
o impossível. ``O impossível'', explica ele, ``tem uma natureza estável
e constante que não é obra de um agente e que não varia sob condição
alguma; é por isso que não se pode atribuir a Deus nenhum poder a esse
respeito''. Ele esclarece, a seguir, que já que as coisas impossíveis
não são obra de um agente e que já que a existência delas é
inadmissível, o fato de que Deus não tenha poder sobre elas não pode
significar, consequentemente, nenhum tipo de fraqueza por parte d'Ele.
As divergências que possam surgir entre os pensadores a esse respeito
limitam-se à classificação do que eles consideram como possível e como
impossível. Assim, todos concordam em identificar como impossível a
união de contrários no mesmo instante e sobre o mesmo assunto, a
transformação da substância em acidente e do acidente em substância, a
existência de uma substância corporal sem acidente, e o fato de que Deus
crie um ser semelhante a si próprio ou que Ele se corporifique, ou
ainda que se transforme. As divergências aparecem quando se trata de
saber, por exemplo, se é possível produzir algo que possua um corpo,
sem para isso servir-se de uma matéria preexistente. ``No entanto'',
diz ele, ``está claro que, segundo todas as opiniões e todos os
sistemas, há coisas impossíveis cuja existência é inadmissível e a
respeito das quais não se pode atribuir poder a Deus''.

Ainda no seu \emph{Guia dos Perplexos} Maimônides faz uma revelação
surpreendente com relação às bases da idolatria. De acordo com ele, a
idolatria em todos os tempos teve sua origem na personificação dos
astros, feita pelos sabianos, que acreditavam que os astros eram a
divindade e que o sol era o deus supremo. O próprio Abraão foi educado
dentro dessa religião, e sua oposição a ela lhe valeu a prisão, o
confisco de seus bens, e o exílio da Síria. Os sabianos adoravam os sete
planetas e os doze signos do zodíaco, e diziam ainda que Adão era o
apóstolo da Lua, e que Noé foi encarcerado porque não aprovava o culto
dos ídolos e porque se dedicava ao culto de Deus. Assim, eles
``ergueram estátuas aos planetas, estátuas de ouro ao sol e estátuas de
prata à lua, e distribuíram os metais e os climas pelos planetas,
dizendo que tal planeta era o deus de tal clima. Eles construíram
templos nos quais colocaram estátuas e afirmaram que as forças dos
planetas se derramavam sobre essas estátuas, de tal forma que elas
falavam, compreendiam, pensavam, inspiravam os homens e lhes davam a
conhecer o que lhes era útil''. Eles acreditavam ainda que se uma árvore
fosse plantada em nome de um planeta e consagrada a ele, de acordo com
determinados ritos e cuidados, a força espiritual desse planeta passava
para essa árvore --- como, por exemplo, no caso da Asherá e do Baal ---,
inspirava os homens e lhes falava durante seu sono, dando assim origem
aos augúrios, à feitiçaria, às previsões, à mágica, etc.

Por não ser uma ciência completa, a idolatria leva à dúvida e à
superstição, transformando aqueles que nela acreditam em vítimas da
ignorância, e sujeitando-os a situações de miséria e de destruição,
tornando-se necessário que o ser humano saiba diferenciar entre os
verdadeiros profetas de Deus e os outros. Foi, portanto, para afastar
esses cultos dos hábitos dos homens que Deus se preocupou em estabelecer
os preceitos relativos à interdição da idolatria, pois ``para
aproximar-se do verdadeiro Deus e para se obter a sua benevolência não
se precisa de todas essas práticas penosas, mas\ldots{} basta amá-Lo e
temê-Lo, duas coisas que são o verdadeiro objetivo do culto divino''.
Assim sendo, não faz nenhum sentido que haja pessoas que imaginem que
seus destinos possam ser regidos por esses astros, a quem se dedicavam
templos e oferendas. Além do mais, se os astros se encarregassem de
dirigir a vida das pessoas, estabelecendo-lhes um destino que variaria
de acordo com a posição deles no momento do
nascimento de cada uma delas, e se elas nada pudessem fazer para
intervir ou mudar isso, a Lei que Deus nos deu não teria nenhuma razão
de ser: tudo já estaria traçado e determinado, de forma definitiva e
inexorável, independentemente da atitude e do comportamento de cada um,
quer vivesse ele no caminho dos justos e dos bons ou na delinquência e
na depravação. ``Nesse caso'', escreve ele, ``toda recompensa e todo
castigo seriam injustiças manifestas que não poderiam ser permitidas nem
entre nós, nem por parte de Deus com relação a nós''.

E no entanto, fica evidente que a lei divina tem um objetivo bem claro:
``Cada qual dos 613 preceitos serve para inculcar as atitudes corretas
ou para eliminar algumas concepções errôneas, para estabelecer uma
legislação justa ou para eliminar injustiças, para nos imbuir de
virtudes exemplares ou para nos dissuadir de inclinações nocivas''. O
conjunto dos preceitos está, portanto, ligado a três coisas: às
opiniões, à moral e à prática dos deveres sociais, e visa fazer com que
o homem possa alcançar a perfeição do corpo e a da alma. A perfeição da
alma, objetivo máximo e superior da existência, através da qual o homem
atinge a \emph{permanência perpétua}, a comunhão com Deus, só pode ser, no
entanto, alcançada numa segunda etapa, depois que o bem-estar do corpo,
embora segundo em importância, tenha sido atingido, já que ``é
impossível que o homem, atormentado por uma dor, pela fome, a sede, o
calor ou o frio, compreenda as ideias que se deseja fazê-lo
compreender''.

A concepção da virtude maimonidiana, situada no termo médio,
equidistante do excesso e da escassez, e que ele expunha no tratado de
ética que escrevera na juventude, vinha de Aristóteles. Mas ele agora
faz uma exceção a essa média ideal com relação à humildade, sobre a qual
ele diz: ``É preciso atingir seu ponto culminante e exercê-la no seu
mais alto grau. Pois na Escritura, toda vez que se fala da grandeza de
Deus, fala-se também de sua humildade. E Deus louvou a humildade de
Moisés, que possuía todas as qualidades morais e intelectuais e que era
o mestre da doutrina, da ciência e da profecia''.

A filologia é uma preocupação constante para Maimônides, em toda a sua
obra. Ele estuda os textos a fundo e compara vários manuscritos do
Talmud antes de tomar uma decisão, pois muitas vezes eles diferem
em pontos essenciais. Ele chega até mesmo a conseguir uma cópia do
Talmud do século VII, escrita em pergaminho, e parece que sua
leitura é de extrema importância para a interpretação dessa obra.

Ele percebe que várias questões religiosas só podiam receber uma solução
com a ajuda da ciência geral, e é o caráter religioso de sua obra que
lhe dá unidade. O equilíbrio de sua alma transparece em seu estilo e ao
longo de suas milhares de frases, cada uma das partes de seus livros se
integra no todo, cada linha guarda sua medida, cada palavra está em
harmonia com o resto do livro, tanto no valor como na forma.

Ao mesmo tempo em que escreve seu \emph{Guia} ele trabalha num tratado
de medicina que viria a ser muito difundido, e no qual ele expõe as
teorias fisiológicas, anatômicas, terapêuticas e higiênicas da medicina.
Em 1186, a pedido do sultão Al-Malik Al-Muzaffar, Maimônides redige o
\emph{Livro dos segredos}, onde enumera vários remédios conhecidos
graças aos mais profundos segredos da medicina e que fazem parte da
literatura médica e do Talmud, além de algumas fórmulas
resultantes de suas próprias pesquisas. Escreve ainda, a pedido de Al
Fadil, grão-vizir de Saladin, um tratado conciso de primeiros socorros
para os casos de envenenamento, que adquire uma grande autoridade nos
meios competentes e que é frequentemente citado pelos médicos durante
toda a Idade Média.

\section{O médico Maimônides}

No ano de 1187 Al Fadil nomeia Maimônides médico da corte. Como
consequência desse emprego e da consideração que o cerca ele recebe
pouco depois o título de \emph{naguid}. A alta posição política que ele adquire
na qualidade de chefe das comunidades judaicas, e o respeito que ele havia
adquirido graças a sua personalidade, permitem-lhe defender os judeus
nas diversas regiões do reino. Um de seus primeiros decretos, ao assumir
esse posto, é o de proclamar que nas cidades egípcias apenas os juízes
expressamente qualificados podem promulgar casamentos e divórcios. Esse
decreto visa a uma supervisão geral dos registros do estado civil e a
uma proteção à mulher, para evitar que os emigrantes, já casados em
seus países de origem, tornem a casar-se no Egito. Seu novo cargo vai
ajudá-lo a combater a influência dos caraítas de forma decisiva e a
obter sucesso na reforma dos ritos, que ele havia iniciado anos antes.
Ele considera, no entanto, que o dever de guiar os judeus é muito mais
uma carga extenuante do que um benefício, pois ele tem consciência de
quanto é fácil ser caluniado e mal interpretado.

Sua situação como médico da corte o obriga a passar o dia inteiro no
Cairo, mas essa grande fama não lhe traz nenhum prazer: ele se sente
deprimido, e o cansaço físico provocado por seu trabalho na corte
durante o dia e como médico e como \emph{naguid} para os que o procuram em sua
casa à noite e nos fins de semana, o impedem de encontrar tempo para
continuar dedicando-se aos estudos. Esse cansaço acaba por provocar uma
doença que o deixa de cama durante todo um ano e da qual ele nunca se
recupera totalmente, pois seu organismo fica debilitado e minado.

Após o \emph{Guia dos Perplexos} ele não escreve mais nenhuma obra de
caráter religioso, a não ser alguns conselhos e cartas e uma dissertação
para explicar os três princípios fundamentais da metafísica da época: a
existência de Deus, as relações da origem do mundo com relação a Deus, e
a eternidade ou a criação do mundo. Para curar os doentes ele renuncia
ao seu projeto de escrever um livro sobre a Hagadá, que deveria colocar
em evidência a filosofia do judaísmo, e assim legitimar suas próprias
teorias. Renunciando até mesmo a traduzir para o hebraico os livros que
havia escrito em árabe, a acabar seus comentários sobre o Talmud,
iniciados em sua juventude, e a redigir o livro de referências de que
dependia o futuro de seu Mishná Torá, obra que tantas
controvérsias havia provocado após sua publicação, ele consagra seus
últimos anos de vida inteiramente à medicina, e é nomeado, em 1198,
médico particular do sultão Al Afdal, sucessor de Al Aziz no trono de
Damasco e da Síria, que reinou depois da morte de Saladin, em 1195.
Portanto, no ponto culminante de sua vida, Maimônides se dirige ao povo
e essa é sua última transformação: ele passa da contemplação à prática,
da metafísica à medicina. Renunciando ao isolamento, Maimônides
consegue agora falar com as outras pessoas sem por isso deixar de pensar
em Deus e de ter seu coração aberto a Ele. Ele não mais sente a
necessidade de pensar nas coisas sagradas para se sentir próximo de
Deus. A superioridade da profecia sobre a filosofia torna-se-lhe clara:
``a profecia não é demonstrada por nenhuma prova, toda tentativa de
exame científico deve ser posta de lado. Seria como querer colocar
todas as coisas do mundo num pequeno vaso'', escreve ele em seus últimos
anos de vida.

Mas o \emph{corpus} de sua doutrina logo começa a ser sacrificado por
aqueles que, não se contentando em ler seus livros, decidem criticá-los,
sem no entanto ter compreendido corretamente seu significado. A questão
relativa à redenção dos corpos, por exemplo, foi mal compreendida e os
iemenitas enviam ao \emph{gaon} de Bagdad uma carta onde se queixam de que a
ressurreição dos mortos, tal como era concebida pelo povo, havia sido
negada na obra de Maimônides. Seus adversários conseguem o apoio do poderoso 
Gaon Samuel ben Ali. A notícia de que o \emph{gaon}, representante legítimo da tradição
judaica, havia tomado partido contra Maimônides e de que o havia
condenado e provado claramente que ele negava a redenção, espalha-se
pelo mundo judeu. Para desfazer esse mal-entendido Maimônides se vê
obrigado a redigir um \emph{Tratado sobre a ressurreição}, onde ele diz:
``\ldots{} Eu julguei incorreto estudar apenas os galhos da doutrina e
negligenciar as raízes. E por isso que discuti os princípios
fundamentais da fé\ldots{} Mas como a compreensão das provas dessas
doutrinas fundamentais pressupõe o conhecimento de muitas ciências, eu
apenas apresentei os meios de prová-las, sem citá-los\ldots{} Eu expressei a
opinião que devemos imaginar o mundo futuro sem relação alguma com a
redenção, mas ressaltei expressamente que a reencarnação dos mortos é
um pilar fundamental da doutrina religiosa. \ldots{} Mas dizer que afirmei
que a alma não retorna jamais a seu corpo é uma difamação, porque essa
negação constituiria a negação dos próprios milagres e equivaleria ao
repúdio da religião''. Uma vez mais, Maimônides se vê forçado a fazer
aqui uma conceção dogmática, mas o faz provavelmente apoiado em estudos
metafísicos que lhe permitem conciliar seus conhecimentos místicos com
o que foi estabelecido pela tradição.

Contudo, se Maimônides encontra oposição e controvérsia por parte de
alguns, é admirado e respeitado por muitos outros, como pelos
estudiosos que vivem nos países cristãos, onde não há nem \emph{gaon} nem
exilarca --- descendente do rei Davi e representante da comunidade no exílio ---, 
onde a autoridade das
escolas é fraca em relação à da academia de Bagdad, e onde se fazem
verdadeiras pesquisas. Esse é o caso da região de Provença, na França,
onde florescem a ciência talmúdica e a especulação filosófica, e onde as
obras de Maimônides produzem uma sensação sem precedentes na história
dos judeus junto a um grupo de sábios respeitados, chamados de \emph{Sábios
de Lunel}. Em 1195 chega a Fostat um tratado redigido em hebraico por
esses sábios, cujo porta-voz se chama Jonathan Cohen, no qual se
cantavam as glórias de Maimônides e se fazia uma declaração de admiração
e fidelidade àquele que eles consideram como o maior mestre que havia
aparecido desde a conclusão do Talmud.

Por volta de 1201 chegam da região do Midi da França cartas com várias
assinaturas, pedindo a Maimônides para traduzir ele próprio para o
hebraico o seu \emph{Guia dos Perplexos}, mas ele se vê forçado a
recusar, pois não tem mais tempo nem sequer para concluir os livros que
havia iniciado. A tradução é então entregue a Samuel Ibn Tibbon, que
trabalha nela com afinco, pois deseja submetê-la a Maimônides para
verificação e revisão, porém infelizmente ele não consegue terminá-la em
tempo.

Maimônides tem a firme convicção de que a imortalidade é a vida eterna
do espírito que tem conhecimentos, mas ``as almas que sobrevivem após a
morte não são a mesma coisa que a alma que nasce com o homem no momento
de seu nascimento; pois aquela que nasce ao mesmo tempo que ele é apenas
algo em potencial e uma disposição, enquanto que a coisa que fica em
separado depois da morte é aquilo em que ela se transformou''. Portanto,
a imortalidade depende da quantidade de conhecimentos adquiridos. A
morte é acontecimento mais importante para o sábio que adquiriu o
conhecimento de Deus, pois a compreensão da inteligência se fortifica no
momento em que ela se separa do corpo, já que ``Há um limite para a
compreensão humana: enquanto a alma estiver no corpo ela não poderá
compreender o sobrenatural\ldots{} A matéria é um grande véu''. Uma vez
atingida a separação, essa inteligência fica para sempre nesse estado
de plenitude, gozando continuamente dessa que é a verdadeira e a grande
felicidade.

Essa importante separação chega para Maimônides na noite
de 13 de dezembro de 1204. Ele é levado, de acordo
com seu desejo, para Tiberíades na Terra Santa, e ali é enterrado num
bosque próximo ao local onde descansam os grandes talmudistas dos
séculos II, III e IV, e onde seu antepassado, o rabino Iehuda Hanassi,
havia estado por diversas vezes.

Nas paredes de mármore que circundam sua tumba, vários seguidores,
estudiosos e amigos deixaram gravados seus testemunhos de admiração e de
respeito pelo grande sábio, um dos quais diz o seguinte:

``Aqui jaz um homem --- e, no entanto, ele não foi um homem; Se fostes
um homem, então foram os seres celestes que te criaram''.


\chapter{Os 14 fundamentos} 

Começarei agora a mencionar os Fundamentos --- em número de 14 ---
que nos guiarão na enumeração dos preceitos. Começarei dizendo que a
soma total dos preceitos que nos foram ordenados por Deus, conforme
consta no Rolo da Torá, é de 613. Deles,
248 correspondem ao número de membros do corpo
humano e são preceitos positivos, e 365
correspondem aos dias de um ano solar e são preceitos negativos. Este
número está mencionado no texto do Talmud, no final do Tratado Macot,
onde está dito: ``613 preceitos foram ditos a Moisés no
Sinai: 365 correspondendo aos dias de um ano
solar e 248 correspondendo aos membros do corpo
humano''. A título de derash\starr{} eles disseram
também, com relação ao fato de que os preceitos positivos correspondem
ao número de membros, que é como se cada membro dissesse à pessoa:
``Cumpra um preceito comigo''; e com relação ao fato de que os
preceitos negativos correspondem ao número de dias no ano solar, eles
disseram que é como se cada dia dissesse à pessoa: ``Não cometa uma
transgressão hoje''. O fato de que eles constituem o número dos
preceitos não passou despercebido a nenhum dos que se empenharam na
enumeração dos preceitos; mas no processo da enumeração em si eles
contaram assuntos que são produtos de imaginação sem base, como será
explicado neste trabalho. Isto se deveu ao fato de que eles
desconheciam estes 14 Fundamentos, os quais passarei a explicar.

O \textbf{Primeiro Fundamento:} Não devemos incluir nesta enumeração os
preceitos adicionais de autoria rabínica.

\textbf{O Segundo Fundamento:} Não devemos incluir nesta enumeração o
que se pode deduzir das Escrituras por meio de qualquer um dos treze
princípios exegéticos, pelos quais se interpreta a Torá, ou por meio
de inclusão.

\textbf{O Terceiro Fundamento:} Não devemos incluir nesta enumeração
preceitos que não sejam obrigatórios por todos os tempos.

\textbf{O Quarto Fundamento:} Não se devem incluir nesta enumeração
ordens que abranjam todos os preceitos da Torá.

\textbf{Quinto Fundamento:} A razão dada para um preceito não deve ser
contada como um preceito separado.

\textbf{O Sexto Fundamento:} Quando um preceito contém tanto uma ordem
positiva quanto uma negativa, cada uma das partes deve ser contada
separadamente, uma entre os preceitos positivos e a outra entre os
negativos.

\textbf{Sétimo Fundamento:} Não devemos contar as leis detalhadas de
um preceito.

\textbf{Oitavo Fundamento:} Não se deve incluir entre os preceitos
negativos uma declaração negativa que exclua um caso particular de um
determinado assunto.

\textbf{Nono Fundamento:} Esta enumeração não deve ser baseada no
número de vezes que um determinado preceito negativo ou positivo está
repetido nas Escrituras, mas sim na Natureza da ação proibida ou
ordenada.

\textbf{O Décimo Fundamento:} Não se devem contar os atos estipulados
como preliminares ao cumprimento do preceito.

\textbf{O Décimo Primeiro Fundamento:} Não se devem contar separadamente
os diversos elementos que compõem um só preceito.

\textbf{O Décimo Segundo Fundamento:} Não se devem contar separadamente
as etapas sucessivas na execução de um preceito.

\textbf{O Décimo Terceiro Fundamento:} Quando um determinado preceito
tiver que ser cumprido por vários dias não se deve contar um preceito
por cada dia.

\textbf{Décimo Quarto Fundamento:} De que forma os tipos de castigo
devem ser contados como preceitos positivos.

E agora voltarei a explicar cada um dos fundamentos e trazer provas
sobre eles, se Deus quiser.

\chapter*{O primeiro fundamento\subtitulo{Não se devem incluir nesta enumeração os\\ preceitos adicionais de autoria rabínica}}

Vocês devem saber que não seria necessário comentar este
assunto, pois está perfeitamente claro. Se o texto do Talmud diz que
``613 preceitos foram ditos a Moisés no Sinai'', como
seria possível dizer que algo que vem dos rabinos se inclui nesta
enumeração? Mas fomos forçados a comentá-lo, pois outros já se enganaram
a este respeito e contaram --- como preceito --- a luz de Chanucá e
a leitura do Rolo (de Esther) entre os preceitos positivos. Também o
recitar de cem bençãos diariamente, o consolo aos que estão de luto, a
visita aos doentes, o enterro dos mortos, o vestir os que estão
despidos, o cálculo das estações, e os dezoito dias nos quais
completamos o ``Halel''.

De fato, deve-se olhar com espanto para quem ouve as palavras ``Foram
ditos a Moisés no Sinai'' e ainda assim conta a leitura do ``Halel'',
com a qual Davi exaltou ao Eterno, enaltecido seja Ele, como se ela
tivesse sido ordenada a Moisés e ainda conta a luz de ``Hanucá'', que
os Sábios estabeleceram na época do Segundo Templo, e a leitura do Rolo
(de Esther)! Contudo, não creio que haja alguém que pudesse imaginar ou
ousasse supor que foi dito a Moisés no Sinai que ele deveria nos ordenar
acender a luz de ``Hanucá'', se no final de nossa soberania um
determinado acontecimento relacionado com os gregos ocorresse entre nós.

Parece-me, no entanto, que o que os levou a errar e a enganar-se a esse
respeito é que, ao recitar a bênção, dizemos: ``que nos santificastes
através de Vossos preceitos e nos ordenastes com relação à leitura do
Rolo'', ou ``a acender a luz `Hanucá'\,'', ou ``a completar o `Halel'\,''.
Além disso, o Talmud pergunta: ``Onde nos foi ordenado''? E eles
responderam: ``Em Suas palavras `Não te desviarás'\,''.\footnote{Deuteronômio
17:11.}

Mas se foi essa a razão pela qual contaram dessa maneira, eles também
deveriam ter contado tudo o que foi imposto pelos Mestres, pois já havia
sido ordenado a nosso mestre Moisés, no Sinai, que nos obrigasse a
executar tudo o que os Sábios nos ordenaram ou proibiram, por Suas
palavras: ``Conforme o mandamento da lei que te ensinarem, e conforme o
juízo que te disserem, farás'';\footnote{Deuteronômio 17:11.} também mais
adiante Ele nos proibiu de desobedecer-lhes em tudo o que decretassem e
estabelecessem, ao dizer: ``Não te desviarás da sentença que te
anunciarem, nem para a direita nem para a esquerda''.\footnote{Ibid.} Mas se
fosse correto contar entre os 613 preceitos tudo o que está revestido de
autoridade rabínica --- levando-se em consideração que se inclui em
palavras, enaltecido seja Ele, ``Não te desviarás da sentença'' e
``Conforme o mandamento da lei que te ensinarem, farás'' --- por que
foram estes enfatizados? Assim como contaram a luz de ``Hanucá'' e a
leitura do Rolo (de Esther) ---, eles deveriam ter contado a lavagem das
mãos e o preceito de eruv,\starr{} pois em relação a
isso recitamos as bênçãos ``que nos santificastes
através de Vossos preceitos e nos ordenastes com relação à lavagem das
mãos'' ou ``com relação ao preceito de `erub', da mesma forma que
recitamos a bênção ``com relação à leitura do Rolo'' ou ``com relação a
acender a luz de `Hanucá'\,''. E tudo isso vem de lei rabínica! Eles
dizem explicitamente: ``As primeiras águas são
preceitos.\footnote{A lavagem das mãos antes da refeição (por oposição às ``águas
  finais'', que são a lavagem das mãos após a refeição).} Como? Abayé disse: É preceito obedecer às
palavras dos Sábios''. Isto é semelhante ao que dizem com relação à
leitura do Rolo (de Esther) e à luz de Hanucá: ``Onde nos foi ordenado
isso? Nas palavras `Não te desviarás'\,''. Também foi deixado claro que
tudo aquilo que ordenaram os profetas --- a paz esteja com eles --- que
vieram depois de Moisés, nosso mestre, também é de autoridade rabínica.
Assim, eles dizem expressamente: ``Quando Salomão estabeleceu o `erubin'
e as mãos, uma voz divina apareceu e disse: Meu
filho, seja sábio e alegre Meu coração'\,''. Em outros trechos eles
explicaram que o `erubin' é lei rabínica e que as
mãos é decreto dos Escribas. Assim, foi explicado
que tudo o que foi ordenado depois de Moisés, nosso mestre, é chamado de
``lei rabínica''.

Eu lhes expliquei isso a fim de que não pensem que pelo fato de ter sido
ordenada pelos profetas, a leitura do Rolo (de Esther) deve ser
considerada ``lei das Escrituras'', pois o ``erubin'' é chamado de
``lei rabínica'', embora tenha sido ordenado por Salomão, o filho de
Davi, e seu Tribunal.

Foi isso o que confundiu alguém, que por esse motivo contou vestir os
despidos, pois ele encontrou em Isaías o seguinte: ``Quando vires um
despido, o cobrirás''. Mas ele não sabia que isto está incluído em Suas
palavras ``E lhe emprestarás o suficiente para o que lhe faltar''.\footnote{Deuteronômio 15:8.} Está sem dúvida alguma claro que o significado
deste preceito é que devemos alimentar os famintos, vestir os despidos,
dar um colchão e um cobertor a quem não os tiver, ajudar a casar-se
aquele que não tiver meios para fazê-lo e providenciar uma montaria
para quem não a tiver, pois sabe-se, pelo texto do Talmud, que tudo
isso está incluído em Suas palavras globais ``O suficiente para o que
lhe faltar''.

Parece, contudo, que na opinião dessas pessoas a linguagem do Talmud
foi composta ``com lábios vacilantes e com língua estranha'', caso
contrário eles não teriam contado a leitura do Rolo (de Esther) e
similares como preceitos ditos a Moisés no Sinai!

A Guemará de Shebuot diz: ``Eu só tenho conhecimento dos preceitos que foram ordenados no Sinai;\footnote{Dos que o povo de ``Israel'' recebeu no monte Sinai.} de que forma fico sabendo dos que foram
destinados a ser estabelecidos como novos artigos, tais como a leitura
do Rolo
(de Esther)? Pelas palavras das Escrituras: `Os judeus cumpriram e
assumiram'\footnote{Esther 9:27.} --- eles cumpriram aquilo que já haviam
assumido''. Isso significa que eles aceitaram crer em todos os preceitos
que os profetas e os Sábios ordenassem depois.

Eu estou surpreso. Por que eles contaram preceitos positivos de
autoridade rabínica, como os que mencionamos, mas não contaram também
preceitos negativos de autoridade rabínica? Assim como eles contaram
entre os preceitos positivos a lei de ``Hanucá'', a leitura do Rolo (de
Esther), as cem bênçãos e o ``Halel'', eles deveriam igualmente ter
contado entre os preceitos negativos os vinte casos de
Sheniyot\starr{} como vinte preceitos negativos. Pois
assim como toda e qualquer relação proibida pela Torá constitui um
preceito negativo de lei das Escrituras, assim também cada relação
proibida pelos Mestres constitui um preceito negativo de autoridade
rabínica. Eis exatamente o que os Sábios dizem: ``As
Sheniyoe foram estabelecidas pelos Escribas''. E
está também explicado no Talmud que as palavras da Mishné,
``proibida por um preceito'' se referem às ``Sheniyot''. A esse respeito
eles comentaram: ``Que proibição por um preceito? O preceito que ordena
obedecer às palavras dos Sábios''. Da mesma forma, eles deveriam ter
contado nesta enumeração ``a irmã de uma mulher (na qual foi feita a)
Halitzá, que é proibido por decreto dos Escribas!

Resumindo, se devêssemos contar todos os preceitos positivos e
negativos impostos pelos Rabinos, o número chegaria a muitos milhares.

Isto está, sem dúvida alguma, claro. Tudo o que for de autoridade
rabínica não deve ser contado na soma total dos 613 preceitos, uma vez
que esses estão todos baseados em versículos da Torá, não havendo entre
eles nada que seja de autoridade rabínica, como explicamos. Mas o fato
de contar alguns preceitos de autoridade rabínica e deixar
arbitrariamente outros de fora não pode ser aceito em hipótese alguma,
seja quem for seu autor.

Explicamos, assim, o teor deste fundamento e sua prova, para que ninguém
venha a ter nem sombra de dúvida a este respeito.

\chapter*{O segundo fundamento\subtitulo{Não devemos incluir nesta enumeração o que se\\ pode deduzir das escrituras por meio de qualquer\\ um dos treze princípios exegéticos,
pelos quais\\ se interpreta a Torá, ou por meio de inclusão}}

Nós já explicamos no início de nosso \emph{Comentário sobre a Mishné}
que a maioria das leis da Torá são deduzidas por meio dos treze
princípios exegéticos pelos quais se interpreta a Torá, e que uma lei
deduzida dessa forma está algumas vezes sujeita a uma diferença de
opiniões. Também explicamos ali que há leis que são interpretações
tradicionais recebidas de nosso mestre Moisés, e portanto não sujeitas
a diferenças de opinião, mas que foram provadas por meio de algum
desses treze princípios. Na realidade, a sabedoria das Escrituras é tal
que é possível encontrar nelas um indício ou uma semelhança que nos
conduza à interpretação recebida.

Assim sendo, conclui-se que nem toda lei deduzida pelos Sábios por meio
de um dos treze princípios pode ser declarada como tendo sido dita a
Moisés no Sinai; da mesma forma, não devemos concluir que pelo fato de
os Sábios do Talmud encontrarem respaldo para uma determinada lei num
dos treze princípios, ela será de autoridade rabínica, visto que é
possível que uma
determinada lei seja uma interpretação recebida.

Resumindo: toda lei que não estiver explicitamente enunciada na Torá,
mas que tenha sido deduzida do Talmud por meio de um dos treze
princípios, deve ser contada se aqueles que receberam a Tradição
declararem explicitamente que ``ela pertence ao corpo da Torá'' ou que
``ela é lei da Torá''. Mas se eles não disserem ou explicarem
claramente que é assim, então ela é uma lei de autoridade rabínica, uma
vez que não há nenhum versículo que a indique diretamente.

Também com relação a este fundamento um outro se enganou e contou o
temor aos Sábios entre os preceitos positivos.

Quer-me parecer que ele o fez por causa das palavras de Rabi Akiba: `Ao
Eterno, teu Deus, temerás'\footnote{Deuteronômio 6:13.} inclui os sábios''. E
eles pensaram que tudo o que se deduz pela Inclusão é semelhante àquilo
que lhe deu origem.

Mas se fosse correto o que eles pensaram, então por que eles não
contaram também o dever de honrar o padrasto e a madrasta e o irmão mais
velho, além dos pais --- a quem temos o dever de honrar conforme o que
foi determinado através do princípio de Inclusão? (Como os sábios
disseram, ```A teu pai'\footnote{Êxodo 20:21.} inclui teu irmão mais velho assim
como teu padrasto. `E a tua mãe' inclui tua madrasta''). Isso é análogo
ao que os Sábios disseram: ```Ao eterno, teu Deus, temerás' inclui os
sábios''. Então por que eles contaram este último e não o anterior?

O fato de não terem o conhecimento necessário já os levou a cometer um
erro maior do que esse: ao encontrar uma interpretação de algum
versículo que exigisse a execução ou a proibição de um determinado ato
--- obrigações que são sem dúvida impostas por lei rabínica --- eles as
contaram entre os preceitos, ainda que o significado literal do
versículo não indicasse de maneira alguma aquelas obrigações. Isso
contraria o princípio que eles, abençoada seja sua memória, nos
ensinaram: ``Um versículo da Torá não perde nunca seu sentido
literal''. Também contraria o processo de raciocínio encontrado em todo
o Talmud, e que está demonstrado pelo fato de que quando os Sábios falam
de um versículo, que dá origem a tópicos derivados por meio de
interpretação e de várias provas, eles perguntam: ``Mas de que trata o
versículo em si?''

Mas eles baseados em comparações sem fundamento, contam entre os
preceitos positivos a visita aos doentes, o consolo aos que estão de
luto, e o enterro dos mortos, tudo isso por causa da seguinte
interpretação, mencionada com relação às Suas palavras, enaltecido seja
Ele, ``E fa-lo-ás saber o caminho por onde andarão, e a obra que
farão'':\footnote{Êxodo 18:20.} ```O caminho' se refere aos atos de bondade;
`andarão' se refere a visitar os doentes; `por onde' se refere a
enterrar os mortos; `e a obra' se refere às leis; `que farão' se refere
ao que ultrapassa o estritamente requerido pela lei''. Eles pensaram que
cada uma das obrigações mencionadas constitui um preceito em si, mas
eles não sabiam que todas essas obrigações, bem como outras
semelhantes, estão incluídas nos termos de um dos preceitos
explicitamente enunciado na Torá, que é o que está expresso em Suas
palavras, enaltecido seja Ele, ``Amarás o teu próximo como a ti mesmo''.\footnote{Levítico 19:18.} De maneira semelhante eles contaram o cálculo das
estações do ano como um preceito, baseados na seguinte interpretação
dada pelos Sábios ao versículo ``Porque isto é a vossa sabedoria e o
vosso entendimento à vista dos povos'':\footnote{Deuteronômio 4:6.} ``Qual é a
sabedoria e qual o entendimento que está à vista dos povos? Devo dizer
que é o cálculo das estações e das constelações''.

Se eles tivessem contado questões ainda mais claras do que esta, e
portanto mais fáceis de supor que é correto enumerá-las --- ou seja, as
leis deduzidas através de um dos treze princípios de interpretação da
Torá --- o número de preceitos atingiria vários milhares! Talvez vocês
possam pensar que nós desistimos de contá-las porque elas não são
suficientemente claras, ou porque haja dúvidas quanto ao fato de
determinada lei deduzida através daquele princípio estar correta ou não,
mas a razão não é essa. O motivo pelo qual não as contamos é que tudo o
que se deduzir dessa forma são ramificações das raízes que foram
explicitamente declaradas a Moisés no Sinai e que constituem os 613
preceitos.

Mesmo que tenha sido o próprio Moisés quem as deduziu, elas não devem
ser contadas. A prova de tudo isto é que eles disseram na Guemará de
Temurá: ``Mil e setecentas deduções de menor a maior., analogias de
frases e aspectos especiais nos decretos dos Escribas foram esquecidos
durante os dias de luto por Moisés. Contudo Ataniel, filho de Kenaz, os
reconstituiu com seu raciocínio, como foi dito: `E Caleb disse: Aquele
que ferir a ``Kiriat Sefer'' e a tomar\ldots{} E Ataniel, o filho de Kenaz a
tomou'\,''. E esse número já era tão grande, quantas então não seriam as
leis originais aprendidas através de Moisés! Pois é inconcebível que se
tenha esquecido tudo o que foi aprendido. Sendo assim não há dúvida de
que as leis aprendidas por dedução de menor a maior, ou por algum dos
outros princípios, chegavam a vários milhares e que todas elas já eram
conhecidas nos dias de Moisés, nosso mestre, visto que foram esquecidas
nos dias de luto.

Assim, foi-lhes explicado que mesmo no tempo de Moisés já se falava de
``aspectos especiais nos decretos dos Escribas'', pois tudo o que eles
não ouviram explicitamente no Sinai é considerado como ``decreto dos
Escribas''. Da mesma forma, foi explicado que mesmo na época de Moisés,
a paz esteja com ele, não se contava entre os 613 preceitos ditos a ele
no Sinai nenhuma lei deduzida através dos treze princípios, e que nós
certamente não devemos contar o que tenha sido deduzido num período
posterior. Em vez disso, devemos contar o que constitui uma
interpretação formulada em seu nome, desde que os guardiães da Tradição
nos digam claramente que um ato específico nos foi proibido e que essa
proibição é lei da Torá, ou que eles nos digam que ``ela é parte da
própria Torá''. Nesse caso a contaremos, pois a aprendemos pela
Tradição e não por um dos treze princípios. Em tais casos a referência
feita a um dos treze princípios ou a apresentação de provas por meio
deles serve apenas para demonstrar a sabedoria contida na Torá, como
explicamos no \emph{Comentário sobre a Mishné}.

\chapter*{O terceiro fundamento\subtitulo{Não devemos incluir nesta enumeração preceitos\\ que não sejam obrigatórios por todos os tempos}}

Você deve saber que as palavras ``Os 613 preceitos foram declarados a
Moisés no Sinai'' ensinam que este número constitui a quantidade de
preceitos obrigatórios por todos os tempos, isso porque os que assim não
forem não têm relação específica com o Sinai, quer tenham eles sido ali
proclamados ou não. A expressão ``no Sinai'' significa apenas a
Revelação essencial da Torá, que ocorreu no Sinai. Isto está expresso
em suas palavras, enaltecido seja Ele, ``Sobe a Mim, ao monte, e fica
ali; e dar-te-ei\ldots{}''.\footnote{Êxodo 24:12.} E eles disseram expressamente:
``Onde nas Escrituras está dito que os 613 preceitos foram declarados a
Moisés no Sinai? No versículo `E a Lei que nos ordenou Moisés, herança
e\ldots{}'.\footnote{Deuteronômio 33:4.} Quer dizer, ele nos ordenou a soma das
letras-números TORAH, que totaliza 611. Eles ouviram o `Eu sou o
Eterno, teu Deus'\footnote{Êxodo 20:2.} e o `Não terás outros deuses diante de
Mim'\footnote{Ibid., 3.} do próprio Todo Poderoso''. Com mais esses dois
preceitos se completa o número 613.

O propósito desta rubrica é demonstrar que a Palavra que nos foi
ordenada por Moisés e que nós ouvimos apenas dele é a soma das
letras-números da palavra TORAH, e que é isso o que Ele declarou ser a
``Herança para a congregação de Jacob''.\footnote{Deuteronômio 33:4.} Um preceito
que não seja obrigatório por todos os tempos não é ``uma herança'' para
nós, pois ``uma herança'' é apenas aquilo que perdura para sempre,
assim como foi dito: ``Por todos os dias que os céus estiverem sobre a
terra''.\footnote{Ibid., 11:21.} Assim também a afirmação deles, de que é como
se cada um dos membros de uma pessoa lhe ordenasse cumprir um preceito
e cada dia do ano o aconselhasse a não cometer uma transgressão, é uma
prova de que este número nunca vai diminuir. Mas se os preceitos que não
são obrigatórios para sempre devessem ser incluídos nesta enumeração,
esse número diminuiria toda vez que um determinado preceito, ao alcançar
seu objetivo, tivesse sido completamente cumprido, e assim aquela
afirmação teria sido correta apenas durante um determinado momento.

Uma vez mais errou o outro com relação a este fundamento, e contou ---
ao se ver pressionado --- ``E não entrarão para ver, quando cobrirem os
objetos da santidade''\footnote{Números 4:20.} e ``Não farão mais o trabalho de
carregamento'',\footnote{Ibid., 8:25.} relativo aos levitas. Mas esses preceitos
eram obrigatórios apenas no deserto, e não para sempre. Embora digam:
``Há uma insinuação contra roubar um vaso sagrado em `E não entrarás
para ver'\,'', o termo ``insinuação'' já é prova suficiente para indicar
que este não é o significado literal do versículo; tampouco se inclui
esta transgressão entre as passíveis de morte pelas mãos dos Céus, como
foi explicado na Tosseftá e em Sanhedrin.

De fato, surpreendo-me com quem contou essas proibições. Por que eles
não contaram o versículo relacionado com o maná, ``Ninguém deixe sobrar
dele até a manhã''\footnote{Êxodo 16:19.} assim como o versículo ``Não molestes a
Moab, e não faças a ele guerra'',\footnote{Deuteronômio 2:9.} e o versículo
relativo a Amon, ``Não os molestes, e não combatas com eles''?\footnote{Ibid.,
19.} Eles também deveriam ter contado entre os preceitos positivos os
versículos ``Faze para ti uma serpente abrasadora e põe-na sobre uma
haste''\footnote{Números 21:8.} e ``Toma um vaso, põe nele a quantia de um `omer'
de maná''\footnote{Êxodo 16:33.} da mesma forma como contaram
``a oferenda de elevação do tributo'' e a dedicação do altar. E também
deveriam ter contado ``Estejam prontos para o terceiro dia'',\footnote{Êxodo
19:15.} bem como ``Tampouco o rebanho, o gado, aparecerão'',\footnote{Ibid.,
34:3.} ``Que não transpassem o termo para subir ao Eterno'',\footnote{Ibid.,
19:24.} e muitos outros versículos semelhantes.

Nenhum ser racional duvidará que todos esses preceitos --- positivos e
negativos --- foram de fato ditos a Moisés no Sinai, só que eles foram
aplicáveis durante um determinado período de tempo e não são
obrigatórios para sempre, e por isso não devem ser incluídos.

De acordo com este fundamento não devemos contar nem as Bênçãos e as Maldições que lhes foram ordenadas no Gerizim e no Ebal, nem a
edificação do altar que nos foi ordenado construir quando entrássemos
na terra de Canaã, porque esses eram todos preceitos aplicáveis a um
determinado período de tempo.

Tampouco devemos contar o preceito positivo de que se desejássemos
comer carne de algum animal só poderíamos fazê-lo depois de levá-la
como oferenda de pazes, porque isso foi um decreto especificamente
aplicável no deserto, como aparece em Suas palavras ``Os trarão ao
Eterno'',\footnote{Levítico 17:5.} sobre as quais a Sifrá comenta: ```E os
trarão' constitui um preceito positivo'' --- mas obrigatório apenas no
deserto, pois Ele explicou no Deuteronômio a permissão perene de comer
uma refeição de carne ao dizer: ``Com todo o desejo de tua alma poderás
comer carne''.\footnote{Deuteronômio 12:20.}

Se fosse necessário contar tudo o que está nesta categoria, todos os
preceitos ordenados a Moisés desde o dia em que ele se tornou profeta
até o dia de sua morte incluindo tudo o que lhe foi ordenado no Egito,
durante a Consagração,\footnote{Do Tabernáculo.} e outros preceitos além
desses, todos contidos na Torá, alguns positivos e outros negativos
--- teríamos mais de trezentos preceitos, além dos que são vigentes por
todos os tempos. Mas como é impossível contá-los todos, fatalmente não
se deve contar nenhum, e não fazer como fizeram outros, usando apenas
alguns deles para assim completar o número que não conseguiram atingir.

Isto é o que desejávamos alcançar com este Fundamento.

\chapter*{O quarto fundamento\subtitulo{Não se devem incluir nesta enumeração ordens\\ que abranjam todos os preceitos da Torá}}

Há preceitos positivos e negativos na Torá que não se referem a uma
obrigação concreta, mas incluem todos os preceitos, como se o Eterno,
enaltecido seja Ele, estivesse dizendo: ``Faça tudo o que eu lhe ordenei
e cuide-se para não fazer todas as coisas que eu lhe proibi'', ou ``Não
se rebele contra algo que eu lhe ordenei''. Tal ordem não deve ser
contada como um preceito separado já que ela não se refere a uma
obrigação específica, o que faria dela um preceito positivo, nem adverte
contra um ato determinado, o que a transformaria num preceito negativo.

Assim por exemplo é o que Ele disse: ``E de tudo o que vos disse,
guardá-lo-eis'';\footnote{Êxodo 23:13.} ``Meus estatutos guardareis'';\footnote{Levítico
19:19.} ``Os Meus juízos cumprireis'';\footnote{Ibid., 18:4.} ``E guardareis
Minha aliança'';\footnote{Êxodo 19:5.} ``E guardareis o Meu mandado'',\footnote{Levítico
18:30.} e muitas outras afirmações análogas.

Já se enganaram com relação a este Fundamento, contando ``Santos
sereis''\footnote{Ibid., 19:2.} como um preceito positivo, sem saber que os
versículos
``Santos sereis'' e ``Santificar-vos-ei e sereis santos''\footnote{Ibid., 11:44}
são ordens para que cumpramos a totalidade da Torá, como se Ele
dissesse: ``Seja santo fazendo tudo o que Eu lhe ordeno e afaste-se de
tudo o que lhe proibi de fazer''.

A Sifrá diz: ```Santos sereis' --- fique distante''. Ou seja, fiquem
longe das abominações contra as quais Eu os adverti.

Na Mekhiltá disseram: ```Homens de santidade sereis para Mim'.\footnote{Êxodo
22:30.} Issi, o filho de Yehudá, diz: a cada novo preceito que o Santo,
enaltecido seja Ele, impõe a Israel, Ele lhes acrescenta mais
santidade''. Quer dizer, esta não é uma obrigação independente, mas está
relacionada aos preceitos que lhes foram ordenados ali, pois todo aquele
que cumprir aquela obrigação será chamado de ``santo''.

Dessa forma não há diferença entre Suas palavras ``Santos sereis'' e
``Cumpre Meus preceitos''. Assim como não diríamos que esta intimação
global constitui um preceito positivo a ser acrescentado a todos os
outros preceitos, assim também não podemos dizer que Suas palavras
``Santos sereis'' e outras expressões semelhantes constituem preceitos
separados, uma vez que não há nelas nada de específico além do que já
sabemos.

O Sifrei diz: ```E sejais santos'\footnote{Números 15:40.} --- isto se refere à
santidade dos preceitos''.

Assim, ficou claro aquilo a que nos propusemos.

A partir deste Fundamento segue-se também que Suas palavras ``E tirareis
o entupimento de vosso coração''\footnote{Deuteronômio 10:16.} significam: Sede
humilde e ouvi todos os preceitos que Ele mencionou anteriormente. Da
mesma forma, o versículo ``E vossa cerviz não endurecereis mais''.\footnote{Ibid.} significa: Não vos rebeleis, aceitando tudo o que vos ordenei e
não lhe desobedeçais.

\chapter*{O quinto fundamento\subtitulo{A razão dada para um preceito não deve\\ ser contada como um preceito separado}}

Ocasionalmente encontramos razões para os preceitos em forma de
preceitos negativos que poderíamos pensar ser apropriados contar como
preceitos independentes. Assim, por exemplo, são Suas palavras ``Não
poderá seu primeiro marido, que a despediu, tornar a tomá-la, para que
seja sua mulher\ldots{} e não farás condenar a terra'',\footnote{Deuteronômio 24:4.}
onde as palavras ``E não farás condenar a terra'' são as razões da
proibição que as antecede, como se Ele tivesse dito: ``Se fizeres assim,
aumentará a corrupção na terra''.

Outro exemplo são Suas palavras ``Não profanarás a tua filha para
fazê-la prostituta, para que a terra não seja entregue à prostituição'',\footnote{Levítico 19:29.} onde as palavras ``Para que a terra não seja entregue
à prostituição'' constituem o motivo, como se Ele tivesse dito: ``A
razão desta proibição é para que a terra não seja entregue à
prostituição''.

O mesmo ocorre no versículo ``E não vos façais impuros com eles e não
sejais impuros por eles''.\footnote{Ibid., 11:43.} Depois de ter mencionado a
proibição contra comer determinadas coisas, Ele deu a razão para isso
dizendo: ``Não se tornem impuros comendo-os'', dando a entender que a
transgressão desta proibição causa a impurificação da alma.

O Sifrei diz expressamente, com relação a Suas palavras, enaltecido seja
Ele, sobre a proibição de pedir um resgate pela vida de um assassino:
``O versículo `E não contaminarás a terra'\footnote{Números 35:34.} nos ensina
que o derramamento de sangue impurifica a terra''. Portanto, foi
explicado que esta ordem negativa constitui uma razão para a proibição
anterior e não uma declaração independente.


Da mesma forma dizem, com relação ao seguinte versículo ``E do
santuário não sairá e não profanará'':\footnote{Levítico 21:12.} ``Mas se sair,
ele profanará''.
Também com relação a este Fundamento errou um outro, incluindo essas
ordens, sem compreendê-las. Todavia, se se perguntasse a alguém que as
incluiu qual é a obrigação específica que essa ordem estabeleceu ele
ficaria confuso e não teria resposta. E isso anula a alegação de que
elas possam ser contadas.

\chapter*{O sexto fundamento\subtitulo{Quando um preceito contém tanto uma ordem positiva quanto uma negativa, cada uma das partes deve ser contada separadamente, uma entre os preceitos positivos e a outra entre os negativos}}

Você deve saber que um assunto pode ser regulamentado por meio de ambos
um preceito positivo e um negativo, de uma destas três maneiras:

\begin{enumerate}
\def\labelenumi{\arabic{enumi}.}
\item
  Se o cumprimento de uma determinada obrigação acarretar um preceito
  positivo e a sua transgressão um preceito negativo, como, por
  exemplo, no Shabat, nos festivais e no Ano Sabático, quando se alguém
  fizer certos trabalhos estará violando um preceito negativo e se
  descansar estará cumprindo um preceito positivo, como será explicado
  posteriormente. Da mesma forma, o jejum em ``Yom Kipur'' constitui um
  preceito positivo e comer nesse dia é um preceito negativo.

\item
  Se houver um preceito negativo precedido por um preceito positivo,
  tal como se vê em Suas palavras com relação a quem seduziu ou
  maldisse: ``E lhe será por mulher'',\footnote{Deuteronômio 22:19.} que
  constituem um preceito positivo, e ``Não a poderá despedir por todos
  os seus dias'',\footnote{Ibid., 29.} que constituem um preceito negativo.

\item
  Se houver um preceito negativo enunciado primeiro, e depois
  justaposto a um preceito positivo, como aparece, por exemplo, em Suas
  palavras ``Não tomarás a mãe estando com os filhos'',\footnote{Ibid., 6.} que
  são seguidas de ``Deixarás ir livremente a mãe''.\footnote{Ibid., 7.}
\end{enumerate}

Em cada um destes casos devemos contar a ordem positiva entre os
preceitos positivos e a negativa entre os preceitos negativos, pois os
Sábios falam explicitamente em cada caso do preceito positivo e do preceito
negativo. Assim, eles dizem muitas vezes: ``O positivo ou o negativo,
relativos a isto''. Isto é perfeitamente compreensível, visto que o
significado do positivo é diferente do negativo e portanto eles são
duas obrigações diferentes: num Ele nos dá uma ordem e no outro Ele nos
faz uma advertência.

Eu não me recordo no momento de alguém que tenha se enganado com relação
a este Fundamento.

\chapter*{O sétimo fundamento\subtitulo{Não se devem contar as leis detalhadas de um preceito}}

Saiba que cada preceito está expresso nas Escrituras e a partir dessa
base seguem-se muitas obrigações e advertências em relação às leis que
regem esse preceito. Este é um exemplo disso: a
Halitzá\footnote{Ver o preceito positivo 217.} e o casamento
levirato\footnote{Ver o preceito positivo 216.} são dois preceitos positivos. Não há
controvérsia quanto a isso. Mas quando estudamos as leis desses dois
preceitos positivos e o que deve ser cumprido de acordo com os
postulados da lei, verifica-se que algumas mulheres realizarão a
Halitzá e não o casamento levirato e que outras realizarão o
casamento levirato e não a Halitzá, enquanto outras ainda farão ou
um ou outro, e outras não farão nem um nem outro. O mesmo se aplica aos
homens, ou seja, aos cunhados. Alguns se submetem à Halitzá mas não
realizam o casamento levirato, outros se casam, mas não se submetem à
Halitzá; outros ainda não fazem nenhum dos dois, e outros podem
fazer ou um ou outro: contrair o casamento levirato ou submeter-se à
Halitzá. Da mesma forma, verifica-se que algumas das cunhadas devem
fazer a Halitzá e outras devem casar-se com um dos cunhados; algumas
farão a Halitzá a todos; a algumas mulheres era permitido que ela se
casasse com o homem que se tornou seu marido, mas não com os irmãos
dele, enquanto que no caso de outra mulher ela poderia ter-se casado com
um dos irmãos de seu marido, mas estava proibida de casar-se com o homem
que de fato se tornou seu marido; em outros casos, tanto o marido como
os irmãos dele eram homens proibidos para ela e, em outro caso ainda,
era-lhe possível casar-se tanto com o homem que se tornou seu marido
como com os irmãos dele.

Se fôssemos considerar cada lei dessas como um preceito independente,
só as leis do Tratado Yebamot somariam mais de duzentos preceitos!
Contudo, nenhuma delas é por si só um preceito positivo ou um negativo;
ao contrário, devemos dizer que em determinadas circunstâncias uma
cunhada deve fazer a Halitzá ou o casamento levirato, e que em
outras ela está proibida de casar-se com um determinado cunhado, ou
então que tanto a Halitzá quanto o casamento levirato são
impossíveis no seu caso. E assim deve ser necessariamente com relação a
cada um dos preceitos.

Sendo assim --- e isto é um assunto acima de qualquer discussão
---conclui-se que ainda que as leis de um preceito estejam
explicitamente enunciadas na Torá elas não devem ser contadas. O mero
fato de as Escrituras terem explicado as leis ou as condições de um
determinado preceito não significa que devamos contar cada condição e
cada detalhe da lei como um preceito individual.

Mas muitos já se enganaram a este respeito, contando tudo o que
encontraram nas Escrituras, sem refletir sobre a substância do preceito,
suas leis e condições. A título de exemplo podemos citar que as
Escrituras, no livro de ``Vayikrá'' (Levítico), obrigam uma pessoa que
tenha tornado impuros o Santuário, suas ofertas consagradas e outras
coisas ali mencionadas, a levar um Sacrifício de Pecado. Isto
constitui, sem dúvida, um preceito positivo. Logo depois as Escrituras
explicam as leis relativas a essa oferenda, dizendo que ela deve
constar de uma ovelha ou uma cabra, e que se ele não tiver posses
suficientes para isso deverá levar duas rolas ou dois pombinhos, e se
ele também não puder se permitir isso, deverá levar a décima parte de
uma ``efá'' de farinha. Isto constitui um Sacrifício de Maior ou Menor
Valor. É óbvio que todas estas leis são apenas uma explicação sobre qual
é o sacrifício imposto, e que de forma alguma elas devem ser contadas
como três preceitos --- o de oferecer o animal, o de oferecer as aves,
e o de oferecer a décima parte de uma ``efá'' pois elas não são três
ordens e sim apenas um único preceito, a saber, que um transgressor deve
oferecer um sacrifício por pecado e que esse sacrifício deve ser isto ou
aquilo, dependendo de seus recursos.

O mesmo princípio se aplica no caso de um sacrifício por um erro
cometido com relação aos preceitos. Assim, as Escrituras explicaram no
livro ``Vayikrá'' que aquele que transgride involuntariamente um dos
preceitos do Eterno deve levar um Sacrifício de Pecado, e que isto
constitui um preceito positivo se o erro for um dos que acarretam a
extinção se cometido voluntariamente, se houver algum ato relacionado
com ele e se ele acarretar a transgressão de um preceito negativo, como
explicamos no comentário a Horayot e Queretot. Em seguida, a Escritura
detalhou as leis relativas a esse sacrifício, dedicando a isso vários
versículos e dizendo que se a pessoa que cometeu o pecado for alguém do
povo, ela deve levar uma ovelha ou uma cabra; se for o
líder,\footnote{De todo o povo judeu.} ele deve levar um bode; e se for o Cohen
Gadol, ele deve levar um boi. E se o erro cometido for especificamente
com relação à idolatria, o transgressor --- seja ele o líder, alguém do
povo ou o Cohen Gadol --- deve oferecer uma cabra. Mas o fato de
oferecer diferentes tipos de animais não altera a natureza do sacrifício
em si, que é a oferta a ser levada por um pecado não intencional, e não
o transforma em vários sacrifícios, de modo a dar origem a vários
preceitos. Se assim fosse, deveríamos da mesma forma contar Suas
palavras ``uma ovelha'' ou ``uma cabra'' como dois preceitos separados,
e ``duas rolas'' ou ``duas pombinhas'' como outros dois preceitos.
Obviamente isto não estaria correto, pois o que constitui o preceito
positivo é a obrigação de levar um sacrifício; e o fato de que uma
pessoa ofereça uma cabra como sacrifício, e a outra um bode, é apenas
uma condição dessa oferenda, mas nem toda condição de um preceito deve
ser considerada como um preceito independente.

Compreenda bem este ponto, pois um erro acerca disto pode ser disfarçado
e só será percebido por alguém dotado de compreensão.

Também entra nesta categoria o que Ele disse, enaltecido seja Ele,
em complemento à lei do castigo de uma moça
comprometida\footnote{Através do noivado legal, que acarreta as responsabilidades legais
do casamento.} que cometer adultério: que se uma moça
comprometida cometer adultério, seu castigo será
o apedrejamento, e que se ela for a filha de um Cohen, seu castigo
será ser
queimada. Com relação a este assunto, todos os que consultei se
enganaram pois eles contaram um preceito separado para a mulher casada,
um para a moça comprometida, e um para a filha de um Cohen. Mas isso
não está correto, como explicarei a seguir.

Suas palavras, enaltecido seja Ele, ``Não adulterarás'',\footnote{Êxodo 20:14.}
que constituem um preceito negativo, são explicadas pela Tradição como
sendo uma advertência às mulheres casadas. Essa advertência é seguida
por um versículo que declara que quem violar esta proibição está sujeito
à pena de morte. Isso está expresso em Suas palavras ``Certamente serão
mortos, o adúltero e a adúltera''.\footnote{Levítico 20:10.} A seguir, as
Escrituras completam a lei desta punição estabelecendo condições,
dizendo que o versículo ``Certamente serão mortos, o adúltero e a
adúltera'' está sujeito às seguintes condições: se a mulher casada for
filha de um Cohen, seu castigo será ser queimada; se ela for uma
moça virgem comprometida, seu castigo será ser apedrejada; e se ela não
for mais virgem e não for a filha de um Cohen, seu castigo será o
estrangulamento. As leis detalhadas relativas ao tipo de morte a ser
aplicado não transformam esse único preceito em vários, pois, apesar de
todos esses detalhes, não nos afastamos da proibição básica imposta à
mulher casada.

Os Sábios dizem explicitamente em Sanhedrin: ``Estavam todos incluídos
nos termos `adúltero e adúltera';\footnote{Ibid.} as escrituras apenas
especificaram que a filha de um israelita está sujeita a ser apedrejada
e que a filha de um Cohen está sujeita a ser queimada''. Com essa
afirmação eles pretenderam dizer o seguinte: a proibição imposta à
mulher casada através de Suas palavras ``Certamente serão mortos, o
adúltero e a adúltera'' inclui a todos; as Escrituras apenas
estabeleceram uma diferença quanto ao tipo de morte, impondo à queima a
algumas e o apedrejamento a outras.

Se devêssemos contar as leis detalhadas de um preceito por estarem elas
mencionadas nas Escrituras, então deveríamos contar muitos preceitos ao
invés de um só na lei que determina que um homicida não intencional deve
exilar-se numa cidade de refúgio, visto que as Escrituras mencionaram
especificamente os detalhes das leis relativas a esse preceito.
Deveríamos, então, contar da seguinte forma: ``E se com instrumento de
ferro ferir''\footnote{Números 35:16.} --- um preceito; ``E se com uma pedra que
cabe na mão''\footnote{Ibid., 17.} --- o segundo preceito; ``Ou se com
instrumento de madeira, com o qual se pode matar, ferir alguém''\footnote{Ibid.,
18.} --- o terceiro preceito; ``O vingador do sangue, matará o homicida''\footnote{Ibid., 19.} --- o quarto preceito; ``E se com ódio empurrar alguém''\footnote{Ibid., 20.} --- o quinto preceito; ``Ou jogar alguma coisa sobre ele, de
emboscada''\footnote{Ibid.} --- o sexto preceito; ``Ou por inimizade o ferir
com a mão''\footnote{Ibid., 21.} --- o sétimo preceito; ``E se por acaso, sem inimizade o empurrou''\footnote{Ibid., 22.} --- o oitavo preceito; ``Ou jogou sobre ele algum instrumento sem ser  de emboscada''\footnote{Ibid.} --- o nono preceito; ``Ou não o vendo\ldots{} alguma
  pedra que possa causar-lhe a morte''\footnote{Ibid., 23.} --- o décimo
  preceito; ``Jogou sobre ele\ldots{} e este morrer, não sendo ele seu
  inimigo''\footnote{Ibid.} --- o décimo primeiro preceito; ``E salvará a
  congregação ao homicida''\footnote{Ibid., 25.} --- o décimo segundo preceito;
  ``E a congregação o fará voltar à sua cidade de refúgio''\footnote{Ibid.} ---
  o décimo terceiro preceito; ``E ficará nela até morrer o `Cohén
  Gadol'\,''\footnote{Ibid.} --- o décimo quarto preceito; ``E se sair o matador fora''\footnote{Ibid., 26.} --- o décimo quinto preceito; ``E depois da morte do Cohen Gadol voltará o homicida''\footnote{Ibid., 28.} --- o décimo sexto preceito.

Se devêssemos fazer isso em todo e cada preceito, o número de preceitos
ultrapassaria os dois mil! É óbvio, contudo, que isso não seria
racional, já que todas essas leis são apenas detalhes de um preceito, e
que o preceito a ser contado é a lei do homicídio, ou seja, que devemos
lidar com o assunto em questão de acordo com a lei estabelecida nesses
versículos. Na realidade, o Eterno se referiu a eles como ``mishpatim''
(leis) e não como ``mitzvot'' (preceitos). Ele disse: ``Então julgará a
congregação entre quem feriu e o vingador do sangue, segundo estas
leis''.\footnote{Ibid., 24.}

O autor do ``Halachot Guedolot'' já se preocupou com relação a este
assunto e prestou atenção a ele, mas quando se deparou com complicações
ele começou a contar seções, enumerando ``a seção de herança'', ``a
seção de promessas e juramentos'', ``a seção do difamador'' e muitas
outras assim. Entretanto, este conceito não lhe ficou totalmente claro
e nem foi completamente compreendido por ele, e assim ele colocou
nessas seções, sem notar, assuntos que já havia enumerado
anteriormente. Assim, devido ao fato de desconhecer este Fundamento,
ele contou onze preceitos com relação à lepra, sem perceber que nesse
caso há apenas um preceito e que tudo o que está mencionado nas
Escrituras não é mais do que a enumeração detalhada de suas leis e
condições.

O significado disto é o seguinte. Foi-nos ordenado que uma pessoa, ao
tornar-se impura por causa da lepra, deve cumprir todas as obrigações
impostas aos impuros, ou seja, afastar-se do Santuário e de suas
ofertas consagradas, e sair do campo da Presença Divina. Contudo, como
ainda não sabemos qual tipo de lepra impurifica uma pessoa e qual não,
as Escrituras começaram, consequentemente, a explicar e a detalhar a
lei: se for assim, ele estará puro; se for de outra forma, ele estará
impuro; e se ela for de uma determinada maneira, ele deverá afastar-se
por um certo período de tempo.

Os Sábios dizem explicitamente: ``Para declará-las puras ou impuras''\footnote{Levítico 13:59.} --- assim como é obrigatório declará-la pura, também é
obrigatório declará-la impura. Assim, é óbvio que o preceito consiste
apenas em declará-lo ``impuro'' ou ``puro'', mas os detalhes que tornam
a pessoa impura ou pura não devem ser contados, uma vez que eles nada
mais são do que as condições e os detalhes da lei. Isso é o mesmo que
dizer que não se deve oferecer um animal defeituoso como sacrifício, o
que seguramente constitui um preceito negativo; resta-nos saber apenas
o que é considerado como defeito. Mas devemos contar todo defeito como
um preceito independente? Se assim fosse, o número chegaria a cerca de
70! Portanto, assim como não contamos os defeitos --- qual deles é
considerado como defeito e qual não --- mas sim apenas a advertência que
nos previne para não oferecer um animal defeituoso, assim também não
devemos contar as manifestações da lepra --- qual é impura e qual é pura
--- mas sim contar apenas que a lepra é impura, sendo todo o resto uma
explicação sobre de que consiste a lepra.

É segundo esse método que devemos contar como um único preceito cada um
dos (treze) tipos de impureza, e não contar os detalhes das leis e
condições de um tipo específico de impureza, como será esclarecido em
nossa enumeração.

Compreenda este Fundamento, pois ele é uma ``coluna central'' de apoio
neste assunto.


\chapter*{O oitavo fundamento\subtitulo{Não se deve incluir entre os
preceitos negativos\\ uma declaração negativa que exclua
um caso\\ particular de um determinado assunto}}

Você deve saber que uma proibição é uma das duas partes de uma ordem.
Quer dizer, pode-se ordenar a uma pessoa que faça algo ou que não o
faça, como, por exemplo, quando você mandar alguém comer alguma coisa e
lhe disser: ``Coma'', ou quando mandar que ele se abstenha de comer e
lhe disser: ``Não coma''. No idioma árabe, porém, não há uma palavra que
abranja esses dois significados. Os mestres do Kalam dizem, na teoria da
lógica, algo assim: ``Em árabe o preceito e a advertência não têm uma
palavra em comum para expressá-los, e por isso fomos obrigados a
referir-nos a ambos com o mesmo termo, que é o preceito''. Assim fica
explicado que uma advertência e um preceito são a mesma coisa. A palavra
usada em árabe no sentido de advertência é ``LA'' (não imperativo).
Este aspecto da comunicação --- a saber, ordenar que uma pessoa faça ou
deixe de fazer alguma coisa --- é encontrado sem dúvida alguma em todos
os idiomas. Está, portanto, claro que tanto o preceito positivo quanto
o negativo englobam ordens absolutas: a de fazer determinadas coisas e
a de prevenir-nos para não fazer outras. O termo que define as coisas
que somos ordenados a fazer é ``preceito positivo'' e o que define as
coisas que somos avisados para não fazer é ``preceito negativo'', e o
nome que abrange os dois juntos no idioma hebreu é ``Guezerá''
(decreto). Dessa forma os Sábios se referem a todos os preceitos,
positivos e negativos, como sendo ``os decretos do Rei''.

Contudo, uma simples declaração negativa que exclua alguma coisa de um
determinado assunto é algo diferente, pois não há uma ordem com relação
a ela. Esse é o caso, por exemplo, quando se diz: ``Aquela pessoa não
comeu ontem, e aquela outra não bebeu o vinho, e Zaid não é o pai de
Ornar'', e outras declarações semelhantes. Todas elas são meras negações
e não têm a menor semelhança com uma ordem.

A palavra usada em árabe para negar uma determinada coisa é quase sempre
``MA'', mas algumas vezes eles usam as palavras ``LA'' ou ``LAISA''. Por
outro lado, os hebreus negam a maioria das vezes com a mesma palavra
``LO'', também usada para fazer uma advertência. Eles negam também com a
palavra ``AYIN'', bem como com as formas que essa palavra adquire quando
se lhe acrescentam pronomes como sufixos, tais como ``EINO'' (ele não
é), ``EINAM'' (eles não são), ``EINCHEM'' (você não é), e outras.

Encontram-se negações em hebreu com a palavra ``LO'' em versículos tais
como: ``E não (VE'LO) se levantou mais, em Israel, profeta algum como
Moisés'';\footnote{Deuteronômio 34:10.} ``Deus não (LO) é homem, para que minta'';\footnote{Números 23:19.} ``Não (LO) haverá desgraça duas vezes'';\footnote{Nahum 1:9.}
``E não (VE'LO) ficou homem com ele'';\footnote{Gênesis 45:1.} ``E ele não
(VE'LO) se levantou nem foi em sua direção'';\footnote{Esther 5:9.} e em muitos
outros casos como esses. Negações com a palavra ``EIN'' aparecem, por
exemplo, nos versículos ``E homem não (AYIN) existia'';\footnote{Gênesis 2:5.} ``Mas os mortos não (EINAM)
sabem nada'',\footnote{Ecc. 9:5.} e muitos outros versículos além destes.

Ficou, assim, clara a diferença entre uma negação e uma advertência. A
advertência tem um caráter de obrigatoriedade e é, na realidade, o verbo
em sua forma de ordem. Quer dizer, da mesma forma que uma ordem, uma
advertência aparece sempre no futuro; assim como é inconcebível num
idioma que uma ordem seja dada no tempo passado, o mesmo ocorre com uma
advertência; assim como é impossível introduzir uma ordem numa sentença
que tenha um relato ou uma narração --- pois uma sentença assim deve
ter um sujeito e um objeto, enquanto que uma ordem constitui por si só
uma expressão completa, como foi explicado nos livros que tratam desse
assunto --- também não se pode introduzir uma advertência numa narração.
Mas nada disto se aplica a uma negação. Uma declaração negativa pode
entrar numa narração e pode referir-se ao passado, futuro ou presente.
Tudo isso fica evidente quando se reflete a respeito.

Sendo assim, não devemos em hipótese alguma contar entre os preceitos
negativos as declarações que forem apenas negações. Naturalmente, isto é
evidente por si só, sem necessidade de provas, a não ser o que foi
mencionado com relação ao esclarecimento do conteúdo de certas
expressões a fim de permitir a distinção entre a advertência e a
negação.

Um outro, contudo, não estava a par disto e consequentemente contou
``Não sairá como saem os escravos'',\footnote{Êxodo 21:7.} sem perceber que esta
era uma declaração negativa, e não uma proibição.

Vou explicar este assunto. O Eterno decretou que se um senhor ferir seu
servo ou serva cananeus, causando-lhes a perda de um de seus órgãos
externos, eles deverão ser libertados. A partir disso poderíamos pensar
que essa lei se aplica com toda certeza à serva hebreia, e que caso o
seu senhor lhe cause a perda de um de seus principais órgãos externos,
ela deverá ser libertada; é por isso que Ele a excluiu dessa lei,
dizendo: ``Não sairá como saem os escravos'', que é como se Ele
estivesse dizendo que não há obrigação de libertá-la caso ele lhe tenha
causado a perda de um de seus órgãos. Dessa forma, o versículo nega a
aplicação de uma determinada lei com relação a ela, mas não é uma
proibição. Os transmissores da Tradição explicaram isto, dizendo o
seguinte na Mekhiltá: `Não sairá como saem os escravos' significa que
ela não será libertada por causa da perda de um dos órgãos principais,
como acontece com os escravos cananeus''. Foi, assim, explicado que esta
é apenas a negação de aplicar a ela uma determinada lei, mas que não
constitui uma advertência.

Basicamente também não há diferença entre Suas palavras ``Não sairá como saem os escravos'' e ``Não procurará o Cohen pelo louro;
impuro
é ele''.\footnote{Levítico 13:36.} Este versículo constitui igualmente uma
negação absoluta, e não uma advertência. Quer dizer, Ele nos ensina que ao
apresentar um determinado sinal\footnote{De lepra.} o isolamento se
torna desnecessário e o Cohen não deve hesitar em declará-lo impuro.

Da mesma forma Suas palavras ``Ele não morrerá, pois ela não era
libertada''\footnote{Levítico 19:20.} não constituem uma advertência, mas sim uma
negação que significa, na realidade, que ele não está sujeito à pena de
morte porque sua liberdade não estava completa. Não se deve traduzir
esse versículo como ``Eles não deverão ser mortos'' pois assim ele se
transformaria numa advertência. Suas palavras ``Ele não morrerá, pois
ela não era libertada'' são semelhantes ao que Ele disse: ``A moça não tem pecado de morte''.\footnote{Deuteronômio 22:26.} Assim como ele negou a ela a pena de morte porque
ela estava sob coação, Ele também negou a ele a pena de morte por causa
da escravidão da moça, como se tivesse dito que ele não está sujeito à
pena de morte porque ela não era livre.

Um outro caso de negação é o que está em Suas palavras ``Para que não
seja como Korah e como sua congregação''.\footnote{Números 17:5.} Os Sábios
explicaram que este versículo é uma negação e interpretam seu
significado dizendo o seguinte: o Eterno declara que o castigo imposto
a todo aquele que vier e disputar o sacerdócio, reclamando-o para si
próprio não será o mesmo que se abateu sobre Korah e sua congregação ---
a saber, ser tragado pela terra e devorado pelo fogo --- e sim será
``Conforme tinha falado o Eterno, por intermédio de Moisés'', isto é, a
lepra. Isto está expresso em Suas palavras ``Leva, por favor, a tua mão
ao teu peito\ldots{} E a tirou e eis que sua mão estava leprosa como a neve''.\footnote{Êxodo 4:6.} Eles oferecem como prova disto o que aconteceu com Uziah,
rei de Yehudá. Embora encontremos uma opinião diferente dos Sábios na
Guemará de Sanhedrin, afirmando que ``Todo aquele que estimula a
discórdia viola um preceito negativo, pois está dito: `Para que não seja
como Korah e como sua congregação'\,'', isso é apenas o significado moral,
mas não é o significado literal do versículo. Na realidade, a
advertência quanto a esse conceito está em outro preceito, como
explicarei no lugar apropriado.

Não há nenhuma regra precisa para se fazer a distinção entre uma
declaração negativa e uma advertência, a não ser o sentido da
declaração. Não há uma palavra específica que diferencie a negação da
advertência, pois ambas são expressas em hebreu pela mesma palavra:
``LO''. Assim, é importante que aquele que analisa o assunto
profundamente pese com cuidado em sua mente o significado das palavras e
então ele perceberá facilmente qual expressão negativa constitui uma
negação e qual uma advertência, como já explicamos anteriormente.

Os Sábios, a paz esteja com eles, comentaram este assunto com
referência a uma controvérsia que surgiu entre eles devido a uma
determinada expressão negativa, para saber se ela era uma simples
negação ou uma advertência. Isso ocorreu por causa do que está expresso
em Suas palavras, enaltecido seja Ele, acerca do pássaro de Sacrifício
de Pecado: ``E destroncará sua cabeça pela nuca, porém não o separará''.\footnote{Levítico 5:8.} ``Nosso Taná'',\footnote{O ``Taná'' é o rabi mencionado na Mishné. O ``Nosso Taná'' é a
parte anônima da Mishné que geralmente expressa a opinião unânime dos
Sábios.} que fala na Mishné,
é de opinião que é uma advertência, e portanto diz: ``Se ele o separar
completamente ele o invalidará''. Consequentemente, esta declaração
negativa constitui um preceito, pois se ele o separar, ele o invalida,
assim como quem oferece levedura ou mel. Por outro lado, Rabi Elazar, o
filho do Rabi Shimon, é de opinião que este versículo não é uma
advertência, e sim uma negativa, e que as palavras ``Não o separará''
significam que não é necessário separar a cabeça e que será suficiente
se ele cortar apenas uma parte dela; dessa maneira, na sua opinião, o
sacrifício também será válido se ele a separar completamente. Os Sábios
dizem o seguinte, na Guemará de Zebahim: ``Rabi Elazar, o filho de Rabi
Shimon, costumava dizer: Ouvi dizer que eles separam completamente o
pássaro de oferta de pecado, mas Suas palavras `não o separará'
significam que ele não precisa separá-lo''. A respeito dessas palavras
os Sábios perguntaram o seguinte: ``Então você também diria que no caso
de um poço, sobre o qual está
dito: `E não o cobrir',\footnote{Êxodo 21:33.} isso significa que ele não precisa
cobri-lo?''. A resposta a isso foi: ``O versículo diz que `O dono do
poço pagará'.\footnote{Ibid. 34.} Isso deixa claro que ele deve cobri-lo''.

Foi, assim, explicado que os Sábios apresentam provas sobre se se trata
de uma negação ou de uma advertência a partir do sentido da própria
declaração.

Também foi explicado que Suas palavras ``Não o separará'' constituem um
preceito negativo, como está evidenciado na Mishné. E, finalmente, isso
também deixará claro que Suas palavras com relação ao pássaro de
holocausto, ``E o rasgará pelas asas, mas não o dividirá'',\footnote{Levítico
1:17.} não devem ser contadas, pois todos os Sábios concordam que mesmo
que ele o divida completamente, o sacrifício ainda será válido. Isso
porque, como no caso do animal que se oferece como holocausto Ele diz:
``E o cortará em seus pedaços'',\footnote{Ibid., 12.} poder-se-ia pensar que a
mesma regra se aplica ao pássaro de holocausto, por isso Ele diz que
não é necessário dividi-lo, e sim apenas rasgá-lo, e caso ele o divida
completamente, o sacrifício ainda será válido, como será explicado no
devido lugar.

Um outro caso de um versículo que pertence à categoria de declarações
negativas é o versículo ``Toda a consagração para pagar o resgate da
avaliação da pessoa condenada à morte, não poderá ser feita''.\footnote{Ibid.,
27:29.} Uma vez que você saiba qual é o significado dessa afirmação,
ficará claro para você que se trata de uma negação, e não de uma
advertência. E o seguinte. As Escrituras estipularam que um determinado
pagamento seja feito por ``avaliações'', de acordo com a idade da pessoa
avaliada e dependendo se ela for homem ou mulher. Com relação a este
conceito não faz diferença se alguém disser: ``Minha avaliação cabe a
mim'' ou ``A avaliação de tal pessoa cabe a mim'', pois nesses casos
vemos o que ela é e qual a sua idade, e ela pagará de acordo com isso.
Mas se a pessoa avaliada for uma que ficou sujeita à pena de morte pelo
Tribunal e foi julgada culpada, e se alguém então disser: ``A avaliação
dessa pessoa cabe a mim'', ela não precisa pagar nada, pois a partir do
final do julgamento ela é considerada como morta, e não se avalia os
mortos. Portanto, este é o sentido de Suas palavras ao dizer: ``Não
poderá ser feito'':\footnote{Ibid.} não é preciso pagar por essa pessoa o
resgate que normalmente aquele que fez o voto de avaliação deveria
pagar. Esta é uma das leis e estatutos sobre as avaliações que foi
mencionada nas Escrituras, mas não é uma advertência.

A Mishné diz: ``Não se fazem votos nem avaliações com relação a um
moribundo ou a alguém que foi condenado à morte''. O Talmud explica que
isto se aplica a alguém condenado à morte por um tribunal israelita. A
Mekhiltá também diz: ``As pessoas sujeitas à morte pelo tribunal não têm
resgate, pois as Escrituras dizem: `Toda a consagração para pagar o
resgate da avaliação da pessoa condenada à morte, não poderá ser
feita'\,''. Avalie a exatidão e a profundidade das palavras dos Sábios,
pois quando dizem ``não tem resgate'' --- e não ``não deverão ser
resgatadas'' --- eles deixam claro que esta afirmação é uma declaração
negativa e não uma advertência.

Esse mesmo assunto está explicado pelos Sábios na Sifrá, na seção de
Avaliações, onde dizem: ``De que forma sabemos, se uma pessoa disser com
relação a um condenado a morte pelo Tribunal: `Sua avaliação cabe a
mim', que suas palavras não têm efeito?'' --- significando que ele não é
obrigado a pagar nada? ``Pelas palavras das Escrituras: `Ele não poderá
ser resgatado'\,''.

Este assunto foi tão perfeitamente explicado que na minha opinião até
mesmo uma pessoa obtusa não terá mais dúvidas a este respeito.

E já que estamos falando deste assunto, você deve saber que há quatro
palavras na Torá --- a saber, ``Hishamer'' (guarda-te de), ``pen''
(para que não), ``aI'' (não faça), e ``lo'' (que não haja) --- que são
usadas para estabelecer uma advertência, e tudo o que tiver sido
advertido através de uma dessas palavras se chama preceito negativo. Os
Sábios dizem claramente: ``Toda vez que aparece `guarda-te de', `para
que não', `não faça' e `que não haja' há um preceito negativo''.

Resta-nos explicar o seguinte ponto para que possamos completar o
propósito desta seção. Toda vez que aparece na Torá que somos obrigados
a proclamar que não fizemos um determinado ato, para assim isentarmo-nos
de toda a responsabilidade quanto a ele, esse ato específico deve ser
contado entre os preceitos negativos, ainda que a proibição que aparece
nele seja apenas uma negação e não uma advertência. Pois se Ele nos
obriga a isentarmo-nos dizendo: ``Eu não fiz isto ou aquilo'',
fatalmente concluímos que estamos sendo advertidos para não fazer essas
coisas. Tal é o caso, por exemplo, em que a Torá nos obriga a dizer:
``Não comi do segundo dízimo no primeiro dia de luto, e não comi dele em
estado de impureza, e não o troquei para fazer o sepultamento de um
morto''.\footnote{Deuteronômio 26:14.} Através dessas palavras fica óbvio que
fomos advertidos para não realizar nenhum desses atos, como será
explicado no devido lugar, quando falarmos desses preceitos.

\chapter*{O nono fundamento\subtitulo{Esta enumeração não deve ser baseada no número\\ de vezes que um determinado preceito negativo ou\\ positivo está repetido nas escrituras, mas sim\\ na natureza da ação proibida ou ordenada}}

Você deve saber que todas as obrigações e advertências da Torá se
referem a quatro coisas: as opiniões, os atos, os traços de caráter e o
que se diz. Assim, a Torá nos ordena a acatar certas opiniões, tais
como acreditar na unidade, amar a Deus e temê-lo, enaltecido seja Ele,
ou nos adverte para que não acreditemos em certas opiniões, tais como
acreditar em e atribuir divindade a outro que não Ele. Da mesma forma
ela nos manda realizar certos atos, tais como oferecer os sacrifícios e
construir o Santuário ou nos previne quanto a certas ações, tais como a
advertência para não oferecer sacrifícios a outros que não Ele,
enaltecido seja Ele, ou curvar-se diante de outros que não Ele. Assim
também ela ordenou que nos conduzamos de acordo com determinados traços
de caráter, tais como a benevolência, a misericórdia, a piedade e o
amor, conforme está no versículo ``E amarás o teu próximo como a ti
mesmo'',\footnote{Levítico 19:18.} ou nos adverte com relação a determinados
outros traços de caráter, tais como guardar rancor, recompensar o mal
ou vingar-se, e outros mais, como explicarei. A Torá nos ordena recitar
certas palavras, tais como expressar a nossa gratidão a Ele, orar a
Ele, confessar os pecados e outros assuntos,
similares, como explicarei, ou nos previne quanto a dizer certas coisas,
tais como a advertência contra pronunciar um falso juramento, fazer
intrigas, falar mal dos outros, amaldiçoar, e outros além destes.

Quando estes aspectos forem compreendidos ficará claro que é a natureza
do assunto ordenado ou proibido --- quer se refira ele a um ato, a algo
que se disse, a uma opinião ou a um traço de caráter --- que deve ser
contado, e que não devemos levar em consideração o número de vezes que
determinada ordem ou advertência está repetida --- dependendo se se
trata de um preceito positivo ou de um negativo ---, pois a finalidade
de todas as repetições é dar maior ênfase, realçando, com a repetição da
lei, o assunto proibido ou ordenado. Apenas quando você encontrar uma
declaração dos Sábios relativa à divisão dos assuntos, e quando tiver
sido explicado pelos Intérpretes que cada um desses preceitos negativos
ou positivos contém um assunto específico, não abrangido pelo outro, só
assim deve-se contá-los, mesmo que à primeira vista possa parecer que
eles tratam de um mesmo assunto, pois o objetivo das repetições não
será enfatizar e sim dar instruções a respeito de aspectos adicionais.
Somente quando não tivermos escolha, e não encontrarmos apoio nas
palavras dos Intérpretes, os guardiães da Tradição, dizendo que o
versículo foi repetido para acrescentar alguma instrução, diremos
necessariamente que ele foi repetido para dar maior realce. Mas se
encontrarmos uma Tradição de que uma certa ordem ou proibição se refere
a um determinado assunto, e que a repetição dessa declaração acrescenta
alguma coisa, será certamente verdadeiro e correto afirmar que o
versículo foi repetido para ensinar-nos algum princípio novo, e nesse
caso cada versículo deverá ser contado separadamente. Mas onde nada de
novo tiver sido acrescentado, o objetivo da repetição será dar maior
ênfase, informar-nos que a transgressão daquela lei é muito grave --- já
que as Escrituras nos advertem várias vezes a esse respeito ---,
completar a lei de um determinado preceito, ou deduzir a partir dela
certa lei para outro preceito, como explica o Talmud ao dizer: ``Ela
está repetida a fim de constituir a base para uma analogia ou para
ensinar-nos através da dedução de frases semelhantes''.

Verificamos que os Sábios, a paz esteja com eles, comentam este ponto
no segundo capítulo da Guemará Pessachim ao discutir uma determinada
proibição que parece estar repetida porque já havia sido deduzida de
outro preceito, e que por isso querem aplicá-la como uma instrução
adicional. Eles dizem, como discussão e consideração: ``Rabina disse a
Rav Ashei: Talvez seja porque a pessoa transgrediu dois preceitos
negativos''. Em outras palavras, por que procurar a solução tentando
aplicar esta proibição a algo diferente daquilo que concluímos com a
proibição original? Talvez ela tenha sido repetida com referência ao
mesmo assunto, de modo que aquele que desobedecer a ela será culpado por
transgredir duas proibições. A resposta a isso foi: ``Disse-lhe: Sempre
que há uma possibilidade de interpretar o versículo nós o fazemos e não
fazemos dele uma proibição adicional''. Portanto, foi deixado claro que
uma proibição que não estabelece alguma instrução nova é chamada de
``adicional'', ou seja, é repetitiva. Assim, embora os Sábios falem de
``culpado por haver transgredido dois preceitos negativos'', fica
claro, por toda esta explicação, que se trata apenas de uma proibição
adicional e que por essa razão ela não deve ser contada. Dessa forma
foi deixado claro que a enumeração dos preceitos não deve ser baseada no
número de vezes que um determinado preceito negativo ou positivo foi
repetido.

Sabe-se que o preceito que nos ordena descansar no Shabat está
mencionado doze vezes na Torá. Acaso alguém que enumera os preceitos
diria ``Entre os preceitos positivos está descansar no Shabat, que
consiste de 12 preceitos''? Uma pessoa sensata diria que a proibição de comer sangue
consiste de sete preceitos? Ninguém vai se enganar com relação ao fato
de que descansar no Shabat é apenas um dos preceitos positivos e de que
a advertência quanto a comer sangue é apenas um dos preceitos
negativos.

Você deve saber que mesmo que se encontre uma frase dos Sábios dizendo
que aquele que comete uma certa transgressão viola dessa forma um certo
número de proibições, ou que aquele que deixa de fazer um certo ato
viola por causa disso um certo número de obrigações, não se deve
concluir que devemos contar separadamente cada uma dessas proibições ou
obrigações, pois a natureza da ação é uma só, e não várias. O fato de
dizerem que a pessoa transgride tantos preceitos positivos ou negativos
só se deve à repetição da ordem ou da advertência feita com relação
àquele preceito em particular, pois ele violou aquele número de
advertências ou ordens. Apenas quando os Sábios disserem que ``ele deve
ser açoitado duas vezes'' ou que ``ele deve ser açoitado três vezes'',
aí então cada advertência deve ser contada separadamente, já que ninguém
é açoitado duas vezes por violar um único preceito, como explicarei de
acordo com o que se sabe pelos textos do Talmud --- em Macot, Hulin, e
outros trechos. Contudo, administram-se dois açoitamentos por dois
preceitos, isto é, por dois assuntos independentes para os quais haja
advertências separadas.

Portanto essa é a diferença entre eles dizerem ``ele violou tantos e
tantos'' e ``ele deve ser açoitado duas ou três vezes''.

Encontramos a prova de tudo o que dissemos nas palavras dos Sábios
``Aquele que não tiver tsitsit em suas vestes violará cinco preceitos
positivos'', porque a ordem relativa a eles aparece cinco vezes: ``Que
façam para eles tsitsit\ldots{} e porão sobre os tsitsit\ldots{} E será para
vós pôr tsitsit''.\footnote{Números 15:38-39.} ``Tsitsit farás para ti e os
porás nos quatro cantos de tua vestimenta''.\footnote{Deuteronômio 22:1.} Contudo,
encontramos uma declaração explícita dos Sábios com relação ao preceito
dos tsitsit dizendo que se trata de um único preceito, como
explicarei\footnote{Ver o preceito positivo 14.} quando tratar dele.


Similar a isso é o que eles disseram: ``Quem não coloca tefilin viola oito preceitos positivos'', porque a ordem relativa a eles --- isto
é, o tefilin da cabeça e o do braço --- aparece oito vezes. Também disseram:
``O Cohen que não subir a plataforma\footnote{A plataforma onde os Cohanim fazem a bênção do povo.} viola três
preceitos positivos'' porque a ordem relativa a isto está repetida três vezes. Mas ninguém que enumerasse os
preceitos pensaria que a Bênção do Cohen constitui três preceitos e
que o tefilin constitui oito.

Sendo assim, conclui-se que não devemos contar ``enganar um prosélito''
como três preceitos devido à reiteração da proibição e às palavras dos
Sábios na Guemará de Metzia: ``Aquele que enganar um prosélito violará
três proibições e aquele que o oprimir violará três proibições''. Ao
contrário, devemos contar apenas os dois preceitos seguintes: ``E ao
peregrino não o fraudareis e não o oprimireis'',\footnote{Êxodo 22:20.} sendo os
outros repetições destas proibições. Não há dúvidas quanto a isto.

Os Sábios dizem explicitamente na Guemará de Metzia: ``Por que a
Torá adverte em trinta e seis lugares para que não se trate mal o
prosélito?
Porque seu temperamento\footnote{Significando seu instinto do mal.} é mau''. Alguém pensaria em incluir esses trinta e seis preceitos nos seiscentos e treze preceitos? É totalmente inconcebível.



Dessa forma, foi explicado e esclarecido que nem todo preceito negativo
ou positivo encontrado na Torá deve ser contado, pois pode ser que ele
seja apenas uma repetição; somente deve ser contado o conceito da ação
ordenada ou proibida. Apenas um professor --- um dos transmissores do
Comentário, a paz esteja com eles --- pode instruir-nos quanto a se um
determinado preceito positivo ou negativo reaparece a fim de
estabelecer alguma instrução adicional ou não.

Também não se deixe confundir por uma proibição que aparece sob
diferentes formas, tal como: ``E tua vinha não rebuscarás'',\footnote{Levítico
19:10.} ``E esqueceres uma gavela no campo, não voltarás a tomá-la'',\footnote{Deuteronômio 24:19.} e ``Quando bateres a tua oliveira (lo tefo'er),
não tornarás a colher o que resta nos ramos''.\footnote{Ibid., 20.} Na realidade,
elas não são duas proibições, e sim uma única advertência relativa a um
único assunto, a saber, que não se deve voltar para buscar o cereal ou
as frutas esquecidas durante a colheita e por isso Ele mencionou dois
exemplos: as uvas e as olivas. O significado da palavra ``Lo tefo'er''
é: não corte o que você esqueceu no fim dos galhos, isto é, os ramos.

Explicarei agora o que deve ser acrescentado a este fundamento. É o
seguinte. O que dissemos, a saber, que devemos contar os conceitos que
nos foram ordenados ou proibidos, está condicionado ao fato de que para
cada um desses conceitos haja um preceito negativo específico ou uma
prova de que os mestres da Tradição separam um conceito do outro,
resultando cada um numa advertência. Mas se uma proibição inclui muitos
assuntos, então contamos apenas essa proibição e não cada um dos vários
conceitos incluídos nela. Essa é a proibição global (Lav shebikhlalut),
cuja violação não acarreta açoitamento, como vamos explicar a seguir.

Ao comentar Suas palavras, enaltecido seja Ele, ``Não comereis sobre o
sangue'',\footnote{Levítico 19:26.} os Sábios dizem: ``De que forma sabemos que é
proibido comer da carne de um animal antes que a vida o tenha abandonado
por completo? Pelo versículo: `Não comereis sobre o sangue'. Outra
interpretação: `Não comereis sobre o sangue' --- não coma a carne
enquanto o sangue ainda estiver na tigela.\footnote{Ou seja, antes do sangue ser aspergido sobre o altar.} Rabi
Dossá diz: `De que forma sabemos que não devemos dar alimentos para a
Refeição de Conforto\footnote{Dada ao enlutado após o funeral.} por alguém que foi executado
judicialmente?' Pelo versículo `Não comereis sobre o sangue'. Rabi Akiba
diz: `De que forma sabemos que um Sanhedrin que realizou uma execução
não deve comer nesse dia?'. Pelo versículo `Não comereis sobre o
sangue'. Rabi Yossi, o filho de Rabi Haniná, disse: `De que forma
deduzimos a advertência com relação ao filho impertinente e rebelde?'.
Pelo versículo `Não comereis sobre o sangue'\,''.\footnote{Que quer dizer ``Não comereis a comida que acarreta a pena de
  morte''.}
Assim, pois, temos cinco assuntos sujeitos todos eles a uma advertência
e incluídos todos nessa proibição. Com relação a eles os Sábios dizem
explicitamente, na Guemará de Sanhedrin: ``Nenhum deles acarreta o
açoitamento pois trata-se de uma proibição global e não se aplica o
açoitamento por uma proibição global''. Mais adiante eles explicam que
uma proibição global é aquela que dá origem a duas ou três proibições.
Portanto fica claro que não devemos contar cada proibição incluída
nesse preceito negativo como sendo um preceito separado, mas sim como um
único preceito negativo que abrange todas elas.


Semelhante à proibição ``Não comereis sobre o sangue'' são Suas
palavras ``E diante do cego não porás tropeço''\footnote{Levítico 19:14.} porque
elas também incluem muitas proibições, como vamos explicar. Da mesma
forma Suas palavras ``Não dês ouvidos à maledicência''\footnote{Êxodo 23:1.}
incluem muitos conceitos, como explicaremos. Este é o primeiro dos dois
tipos de proibições globais.

O segundo tipo consiste de um preceito negativo que proíbe várias coisas
juntas e acrescidas umas as outras, tal como quando Ele diz: ``Não faça
isto e aquilo''. Por sua vez, este tipo se divide em duas partes, e de
acordo com a explicação do Talmud uma delas acarreta o açoitamento por
cada um dos conceitos, e a outra causa um único açoitamento, por ser
uma proibição global. Segundo os Sábios, devemos contar como preceitos
separados cada uma das proibições que ocasionam um açoitamento por cada
conceito, e devemos contar como um único preceito as proibições que
ocasionam o açoitamento uma única vez, de acordo com o que estabelecemos
neste Fundamento --- que em circunstância alguma se é açoitado duas
vezes por um único preceito. Por outro lado, onde eles estabeleceram
claramente que se está sujeito ao açoitamento por cada um dos assuntos
ligados e relacionados entre si, de maneira que aquele que fizer todos
ao mesmo tempo estará sujeito a vários açoitamentos, concluímos com
certeza que eles constituem vários preceitos, a serem contados
separadamente.

Mencionarei agora vários exemplos das duas partes desse segundo tipo. É
possível até que mencione todos os preceitos negativos desta categoria
para que o assunto fique completamente elucidado.

Contamos a proibição expressa em Suas palavras, enaltecido seja Ele,
relativa ao cordeiro Pascal ``Não comais dela mal passada no fogo nem
cozida na água''\footnote{Êxodo 12:19.} como um preceito, e não ``Não comais dela
mal passada no fogo'' como um e ``Não comais dela cozida na água'' como
outro preceito, pois Ele não expressou especificamente uma declaração
proibitiva em cada assunto dizendo ``Não comais dela mal passada nem
tampouco a comais cozida na água''. Em vez disso, Ele expressou uma
proibição que inclui dois assuntos ligados e relacionados entre si.

No capítulo de Pessachim os Sábios dizem: ``Abayé disse que se ele a
comeu mal passada, ele deve ser açoitado duas vezes; se a comeu cozida
na água, duas vezes; e se a comeu mal passada e cozida na água, três
vezes''. Isso se deve ao fato de que ele acredita que se deve ser
açoitado por desobedecer a uma proibição global; portanto, quando alguém
a comer mal passada, estará transgredindo duas proibições: a que diz
``Não comais dela mal passada no fogo'' e a que se conclui por dedução,
que é como se Ele tivesse dito: ``Coma-a apenas grelhada'', ao passo que
ele a comeu de outra forma. E se ele a comeu malpassada e cozida na
água, ele será açoitado três vezes, de acordo com Abayé: uma por
comê-la mal passada, outra por comê-la cozida na água, e uma terceira
por tê-la comido sem ser grelhada.

Continuando este debate, disseram ali: ``Mas Rabá diz que não se fica
sujeito ao açoitamento por uma proibição global. Há quem diga que se
fica sujeito a pelo menos um açoitamento'', isto é, se a comer mal
passada e cozida na água, será açoitado uma vez. ``Outros dizem que não
se fica sujeito a açoitamento algum, pois esta não é tão específica
quanto a proibição de colocar uma focinheira''. Este é o preceito ``Não
amarrarás a boca ao boi quando estiver debulhando'',\footnote{Deuteronômio
25:4.} que consiste de uma proibição advertindo contra fazer uma única
coisa, enquanto que a outra proibição adverte com relação a duas coisas
--- mal passada e cozida em água --- e consequentemente não sujeita o
transgressor a açoitamento algum.


Você já está familiarizado com o que está explicado na Guemará de
Sanhedrin: ``Não se fica sujeito ao açoitamento por uma proibição
global''. Consequentemente, as palavras de Abayé são rejeitadas, sendo
a opinião correta a que diz que se fica sujeito a apenas um açoitamento,
quer se tenha comido malpassada ou cozida na água, ou malpassada e
cozida na água. Por essa razão devemos contar Suas palavras, enaltecido
seja Ele, ``Não comais dela malpassada no fogo nem cozida na água''
como um preceito apenas.

Ali disseram também os Sábios: ``Abayé disse que se\footnote{Um Nazir.\starr}
comesse a casca da uva, ele seria açoitado duas vezes; se comesse
caroços de uva, duas vezes; se comesse cascas e caroços de uva, três
vezes. Mas Rabá diz que não se fica sujeito ao açoitamento por causa de
uma proibição global'', fazendo alusão a Suas palavras ``De tudo o que
sai da videira''\footnote{Números 6:4.} pelo que, na opinião de Abayé, fica-se
sujeito ao açoitamento.

Da mesma forma eles dizem, no quarto capítulo de Menahot: ``Abayé disse
que aquele que oferecer lêvedo e mel sobre o altar deverá ser açoitado
uma vez pelo lêvedo, uma vez pelo mel, uma vez por ter misturado o
lêvedo e uma vez por ter misturado o mel''.\footnote{Ibid., 12.} Ou seja, a palavra
``col'' inclui duas coisas: que essas coisas não devem ser oferecidas
separadas nem misturadas, seja em que quantidade for. Como você já sabe,
tudo isso está de acordo com a teoria essencial de Abayé de que se está
sujeito ao açoitamento por uma proibição global. E os Sábios
prosseguem, dizendo: ``Mas Rabá diz que não se fica sujeito a castigo
por causa de uma proibição global. Alguns dizem que se está sujeito a
pelo menos um açoitamento, e outros dizem que não se está sujeito a
nenhum, uma vez que ela não é tão específica como a proibição de colocar
uma focinheira''.

Então, como foi explicado que os versículos ``Não comais dela mal
passada no fogo nem cozida na água'' e ``Não fareis queimar fermento
algum ou mel''\footnote{Levítico 2:11.} constituem um preceito cada um, assim
também contaremos cada um dos seguintes versículos como um só preceito:
``Não entrará nenhum amonita e nem moabita'';\footnote{Deuteronômio 23:4.} ``A
nenhuma viúva ou órfão afligireis'';\footnote{Êxodo 22:21.} ``Não perverterás o
juízo do peregrino e do órfão'';\footnote{Deuteronômio 24:17.} ``Sua
manutenção, seu vestuário, e seu direito conjugal não lhe diminuirá'';\footnote{Exodo 21:10.} cada uma dessas proibições é idêntica às proibições
mencionadas: ``Não comais dela mal passada no fogo nem cozida na água''
e ``Não fareis queimar fermento algum ou mel''. Não há diferença entre
elas.

Do mesmo modo, Suas palavras ``Não trarás salário de rameira nem preço
de um cão''\footnote{Deuteronômio 23:19.} constituem um preceito negativo. E esse
também é o caso de Suas palavras ``Vinho e bebida forte não bebereis\ldots{}
quando entrardes à tenda da revelação\ldots{} e para ensinar''.\footnote{Levítico
10:9-11.} Quer dizer, num único preceito Ele advertiu para que não se
entre no Santuário nem se dê instruções sobre a Torá em estado de
embriaguez. Esta é uma das duas partes do segundo tipo de proibição
global.

A segunda parte trata precisamente do mesmo tipo de expressão que a
primeira, exceto que aqui a instrução da Tradição é de que cada assunto
ligado e acrescido acarreta um açoitamento em separado e que caso eles
sejam todos transgredidos, mesmo que sejam todos de uma só vez, fica-se
sujeito ao açoitamento por cada um deles. É em casos como estes que
devemos contar cada conceito como uma proibição separada.



Esse é o caso de Suas palavras ``Não te será permitido comer em tuas
cidades o dízimo de teus cereais, e de teu mosto, e de teu azeite'',\footnote{Deuteronômio 12:17.} com relação às quais os Sábios dizem na Guemará
Queretot: ``Se alguém comer do dízimo dos cereais, do mosto e do azeite
ele será culpado por cada um deles separadamente''. A esse respeito eles
perguntaram: ``Mas é-se açoitado por uma proibição global?'' E a
resposta foi: ``O texto é repetitivo. Veja: a Torá já disse: `Comerás
diante do Eterno, teu Deus,\ldots{} o dízimo de teu grão, teu mosto e teu
azeite'.\footnote{Ibid., 14:23.} Por que ela determina `Não te será permitido
comer em tuas cidades'? E se você disser que é para estabelecer uma
proibição, então que a Torá diga: `Não te será permitido comê-los'.
Por que ela enuncia outra vez todos detalhadamente? Só pode ser para
estabelecê-los em separado.

Está explicado ali que se é culpado por cada uma separadamente também
no caso de Suas palavras, enaltecido seja Ele, ``E pão, e farinha feita
de grãos de espigas verdes, torrada no forno, e grãos verdes de
cereais\footnote{Da nova colheita, até que se traga a oferenda do ``omer''.} não comereis''.\footnote{Levítico 23:14.} Os Sábios
dizem: ``Aquele que come pão, farinha de grãos de espiga verdes e grãos
verdes de cereais é culpado por cada um deles separadamente. Mas
fica-se sujeito ao açoitamento por uma proibição global? O texto é
repetitivo. Que o Misericordioso escreva uma e as outras serão deduzidas
dela''. Depois de uma discussão a respeito foi explicado que não havia
necessidade de que Ele mencionasse a ``farinha de grãos de espigas
verdes'', e que isso foi mencionado para estabelecer uma separação: para
sujeitar-nos ao açoitamento no caso dessa farinha separadamente. O
Talmud continua o debate perguntando: Talvez se fique sujeito ao
açoitamento por causa da farinha de grãos, uma vez que ela foi
mencionada com esse objetivo, mas será que se fica sujeito a um único
açoitamento por comer pão e farinha de grãos verdes? A resposta foi:
``Por que motivo o Misericordioso escreveu `farinha de grãos verdes' no
meio? Para ensinar-nos que o pão é como a farinha de grãos, e que esta é
como os grãos verdes'', de maneira que se é culpado por cada um
individualmente.

O mesmo direi com relação a Sua declaração, enaltecido seja Ele, ``Não
se achará entre ti quem faça passar seu filho ou sua filha pelo fogo,
nem agoureiro, nem prognosticador, nem adivinho, nem feiticeiro, nem
encantador, nem necromante ou Yideonita, nem quem consulte os mortos'';\footnote{Deuteronômio 18:10-11.} cada uma das nove coisas enumeradas são
contadas como um preceito individual e nenhuma delas pertence à primeira
parte do segundo tipo. Prova disso é que Suas palavras, enaltecido seja
Ele, ``Nem prognosticador, nem adivinho'' estão no meio da frase, pois
já foi explicado que em Suas palavras ``Não augurareis e não
prognosticareis''\footnote{Levítico 19:26.} cada uma dessas proibições constitui
um preceito em si. Assim como o prognosticador e o adivinho ---
mencionados no meio --- são casos separados, também todos os outros
casos mencionados antes e depois são semelhantes a eles, tal como
explicam os Sábios no caso do ``pão, da farinha de grãos e dos grãos
verdes''.

Outros se enganaram a respeito desta questão, seja porque suas mentes
não compreenderam em absoluto estes assuntos, ou então porque eles se
esqueceram e se desviaram do rumo correto. Assim, eles contaram Suas
palavras, enaltecido seja Ele, com respeito aos Cohanim, ``Mulher
prostituta ou profana não tomarão, nem mulher divorciada de seu marido
não tomarão''\footnote{Ibid., 21:7.} como um único preceito, embora já estivesse
explicado na Guemará de Kidushin que se fica sujeito a castigo por cada
uma dessas desqualificações, mesmo que todas elas digam respeito a uma
única mulher, como explicaremos no local apropriado. De fato, poderíamos
encontrar uma desculpa por contar uma
prostituta e uma mulher profana como um preceito, porque tendo
compreendido alguns detalhes da proibição global ele considerou Suas
palavras, enaltecido seja Ele, ``Mulher prostituta ou profana não
tomarão'' semelhante a ``Não comais dela mal passada no fogo nem cozida
na água'' e não percebeu que a primeira proibição estabelece uma
separação e a segunda não. Também não diferenciou o versículo ``E pão,
e farinha feita de grãos de espigas verdes, torradas no forno, e grãos
verdes de cereais não comereis'' de ``Sua manutenção, seu vestuário e
seu direito conjugal não lhe diminuirá''. Contudo, não vou criticá-lo
em casos como estes. Mas não há desculpa por ter contado uma mulher
divorciada junto com uma prostituta e uma profana, incluindo-as todas
num só preceito, pois a mulher divorciada constitui claramente uma
proibição separada, como Ele disse, enaltecido seja Ele: ``Nem mulher
divorciada de seu marido não tomarão''.

Portanto, deixamos claro este grande Fundamento --- isto é, a proibição
global --- e explicamos as dúvidas relativas a ele. Também esclarecemos
em que casos ele estabelece uma separação e em que casos há apenas uma
proibição global, sujeitando-nos ao castigo apenas uma vez.
Esclarecemos ainda que quando há uma separação devem-se contar todos
como preceitos separados e que quando não há separação deve-se contar um
só preceito. Tenha sempre todo este Fundamento diante de si, pois ele é
um guia da maior importância para a enumeração correta dos princípios.

\chapter*{O décimo fundamento\subtitulo{Não se devem contar os atos estipulados como preliminares ao cumprimento do preceito}}

Ocasionalmente a Torá menciona ordens que não constituem preceitos em
si, mas apenas preliminares para o cumprimento de um preceito, como se
Ele estivesse descrevendo a maneira como o preceito deve ser executado.
Um exemplo disso é o versículo ``E tomarás a flor da farinha de trigo''.\footnote{Levítico 24:5.} Não seria correto contar o fato de tomar a farinha
como um preceito, e o cozimento dela como outro; o que deve ser contado
é apenas o que Ele diz: ``E porás sobre a mesa o pão da proposição
diante de Mim, continuamente'',\footnote{Êxodo 25:30.} pois o preceito consiste
apenas em que haja sempre pão diante do Eterno. Depois Ele descreve como
deve ser esse pão e a partir de que ele deve ser feito, dizendo que ele
deve ser feito com flor de farinha e que consiste de doze pães.

Do mesmo modo não devemos contar Suas palavras ``Que te tragam azeite de
oliveira puro'',\footnote{Ibid., 27:20.} e sim apenas ``para acender a lamparina
contínua'',\footnote{Ibid.} que é o preceito de manter as lamparinas acesas,
como foi explicado no Tamid.

Assim também não devemos contar Suas palavras ``Toma para ti
especiarias'',\footnote{Êxodo 30:34.} mas sim a queima diária de incenso, como
dizem as Escrituras: ``Pela manhã, quando limpar as lamparinas, o
queimará. E ao acender Aarão os fogos\ldots{}''.\footnote{Ibid.,78.} É este versículo que constitui o
preceito a ser contado, ao passo que Suas palavras ``Toma para ti
especiarias'' são apenas uma preparação, explicando como o preceito deve
ser realizado e de que especiarias deve ser feito esse incenso.

Da mesma forma não devemos contar Suas palavras ``Toma para ti
especiarias principais''.\footnote{Ibid., 23.} O que deve ser contado é a ordem
que nos obriga a ungir os Cohanim Gadol, os reis e os utensílios
sagrados com o Óleo de Unção descrito.

Você deve julgar todos os casos semelhantes a esses baseando-se neste
critério para que você não aumente a enumeração com tópicos que não
fazem parte dela. Este é nosso objetivo com este Fundamento, e isso está
perfeitamente claro. Porém nós o mencionamos e o comentamos porque
também com relação a este assunto muitos se enganaram, contando um
preceito e seus atos preliminares como dois preceitos, como ficará
claro para quem observar a enumeração das seções feitas por Shimon
Kayará, bendita seja sua memória, e por seus seguidores.

\chapter*{O décimo primeiro fundamento\subtitulo{Não se devem contar separadamente os diversos elementos que compõem um só preceito}}

Ocasionalmente um preceito pode consistir de muitas partes, como é o
caso do preceito do ramo de palma (``lulav'' e ``etrog''), que
compreende quatro tipos. Nesse caso não devemos dizer que ``o fruto das
árvores formosas'' e ``os galhos das árvores frondosas'', e ``o
salgueiro do regato'' e ``os ramos das palmeiras'' constituem cada um
um preceito separado; ao contrário, eles são todos partes de um
preceito, pois Ele ordenou juntá-los e o preceito consiste em segurá-los
na mão, todos juntos, num dia determinado.

Um caso exatamente igual: não devemos contar Suas palavras que dizem que
o leproso deve purificar-se com dois pássaros, um pau de cedro,
carmezin, hissopo, água corrente e uma vasilha de barro como sendo seis
preceitos; o que constitui o preceito é a purificação do leproso
através de todos os elementos estabelecidos --- os já citados e mais a
raspagem --- sendo que os diferentes elementos nos são impostos para
fazer essa purificação, que é feita de tal e tal maneira.

A mesma lei se aplica com relação aos sinais de reconhecimento que nos
foi ordenado fazer no leproso quando ele estiver em estado de impureza,
a fim de que nos mantenhamos afastados dele, como está dito: ``Suas
vestes serão rasgadas e seu cabelo não será cortado, e com seu bigode se
cobrirá; e impuro! impuro! clamará''.\footnote{Levítico 13:43.} Cada um desses
atos não constitui um preceito em si; ao invés disso, é o preceito que
consiste no conjunto deles,
isto é, que devemos fazer com que o leproso possa ser identificado para
que possamos nos manter afastados dele e que essa identificação se
compõe disto e daquilo, tal como nos foi ordenado alegrar-nos diante do
Eterno no primeiro dia dos Tabernáculos, sendo que Ele explicou que essa
alegria consiste em levar determinados objetos.

Há um aspecto difícil de ser compreendido neste Fundamento e a razão
disso é o que explicarei a seguir: Toda vez que os Sábios disserem, com
relação a um determinado preceito, que um de seus elementos prejudica a
validade de outro, é óbvio que ele constitui um preceito, como, por
exemplo, as quatro variedades usadas no caso do ramo de palma e o
incenso levado junto com o pão da proposição sobre o qual os Sábios
disseram: ``As fileiras e os pratos comprometem um a validade do
outro''; nesses casos fica claro que eles constituem um único preceito.
Da mesma forma, toda vez que ficar claro que o objetivo desejado não
será obtido através de um dos elementos, também é óbvio que é o conjunto
deles que deve ser contado. Tal é, por exemplo, o caso explicado sobre o
reconhecimento do leproso, pois se apenas suas roupas forem rasgadas
mas se ele não tiver deixado crescer seu cabelo, não tiver coberto o
lábio superior e não gritar ``impuro, impuro'', ele não terá efetivado
nada; ele só será identificável quando fizer tudo. Assim também sua
purificação não será alcançada até que ele se utilize de todas as coisas
mencionadas: os pássaros, o pau de cedro, o hissopo, o carmezin, e a
raspagem. Só assim sua purificação será alcançada.

Contudo, o aspecto difícil surge quando os Sábios dizem, com relação
aos elementos, que ``eles não comprometem a validade uns dos outros''.
Num primeiro raciocínio você concluirá que como cada um desses vários
elementos não precisa do outro, cada um deles deve ser contado como um
preceito independente. Esse é o caso, por exemplo, de sua afirmação:
``O azul\footnote{O cordão azul do tsitsit.} não prejudica a validade do
branco,\footnote{O cordão branco do tsitsit.} e o branco não compromete o azul''. Isto
pode levá-lo a concluir que o branco e azul devem ser contados como dois
preceitos, se não fosse pela afirmação explícita que encontramos na
Mekhiltá de Rabi Ishmael de que ``Poderíamos pensar que estes são dois
preceitos --- o do azul e o do branco ---; por isso as Escrituras
afirmam `E será para vós pôr tsitsit'',\footnote{Números 15:39} mostrando
assim que se trata de um preceito e não de dois''.

Assim, foi explicado que mesmo que os elementos não prejudiquem a
validade uns dos outros, eles às vezes são um único preceito, desde que
o significado deles seja um só. Tal é o caso dos tsitsit, onde o
objetivo é ``Para que vos lembreis''\footnote{Números 15:40.} e portanto o que
deve ser contado é o conjunto de coisas que vai fazer com que nos
lembremos.

Assim sendo, quando formos enumerar os preceitos, resta-nos não prestar
atenção quanto a se cada elemento compromete a validade do outro ou não,
e sim fixarmo-nos em seu conceito para saber se se trata de um ou de
vários, tal como explicamos no nono dos Fundamentos que estamos tentando
elucidar.



\chapter*{O décimo segundo fundamento\subtitulo{Não se devem contar separadamente as etapas sucessivas na execução de um processo}}

Às vezes somos ordenados a executar uma determinada ação, e logo em
seguida a Torá começa a explicar como essa ação deve ser executada,
elucidando a expressão que usou e definindo o que está incluído nela. Em
casos assim não devemos contar cada ordem contida na explicação como um
preceito individual. Por exemplo: Suas palavras ``E me farão um
santuário''\footnote{Êxodo 25:8.} constituem um dos preceitos positivos, que é o
de que devemos construir uma casa para a qual devemos nos dirigir, para
onde devemos ir a fim de oferecer os sacrifícios e onde as assembleias
terão lugar, durante os festivais. Depois disso Ele começa a descrever
seus detalhes e como eles devem ser executados; esses atos específicos
--- cada um deles precedido pela expressão ``E me farão'' --- não devem
ser contados como preceitos separados.

O mesmo acontece com relação aos sacrifícios mencionados no
``Vayikrá'', onde um preceito consiste de todo o ritual descrito em
cada um dos vários tipos de sacrifícios. Um exemplo disso é o
holocausto, cujo ritual que nos foi ordenado é o seguinte: que ele seja
degolado, que sua pele seja retirada, que seja cortado em pedaços, que
seu sangue seja derramado de tal e tal maneira, é que sua gordura seja
queimada, seguida da queima de toda sua carne, juntamente com uma
determinada medida de flor de farinha misturada com óleo e com certa
quantia de vinho --- que são as libações --- e que o couro seja dado ao
Cohen que estiver oficiando. A totalidade deste ritual constitui um
preceito positivo, que é a lei do holocausto, sendo que a Torá nos
obriga a executar cada holocausto dessa maneira.

Um caso semelhante é o do ritual completo do Sacrifício de Pecado:
o degolamento, a retirada do couro, a queima do que deve ser queimado, a
lavagem das vasilhas nas quais deve ser derramado parte do sangue, e a
lavagem ou a quebra das vasilhas nas quais foi cozida a carne. Tudo
isso é a lei do Sacrifício de Pecado e constitui um único preceito.

Da mesma forma, a lei do Sacrifício de Delito constitui um só preceito,
assim como a ``lei do Sacrifício de Oferta de Paz'', oferecido como
sacrifício de graças, com ou sem pão, quando o Cohen pega o peito e
a coxa e os levanta, sendo que tudo isso é o ritual do ``Sacrifício de
Oferta de Pazes'' e
constitui um só preceito.

Esses compõem a totalidade dos sacrifícios cuja obrigação cabe ao
indivíduo e à congregação, com a exceção do Sacrifício de Delito que é
sempre uma oferta individual, como explicamos na introdução à Ordem de
Kadashim.
O que constitui o preceito positivo é o procedimento nos vários rituais
e não se deve contar cada detalhe desses rituais como um preceito
separado, a não
ser que eles sejam ordens que abranjam todos os diversos tipos de
sacrifícios e que não sejam específicos para apenas um deles, excluindo
os outros; tais ordens devem ser contadas como preceitos em separado,
já que elas não são meros detalhes do ritual de algum sacrifício
específico. Esse é o caso, por exemplo, de Sua advertência para não
trazer um sacrifício defeituoso, ou de sua ordem para que seja
perfeito, ou de que não ofereçamos um animal que não tenha atingido a
idade de ser aceito, conforme consta em Suas palavras ``Do oitavo dia
em diante\ldots{}'';\footnote{Levítico 22:27.} ou de que ofereçamos sal com todos os
sacrifícios, como Ele disse: ``Toda tua oferta de oblação temperarás com
sal'';\footnote{Ibid., 2:13.} ou de que não deixemos faltar o sal num sacrifício,
como Ele disse: ``Não deixarás faltar o sal'';\footnote{Ibid.} ou de que sejam
comidas as partes que devem ser comidas. Cada um desses casos constitui
um preceito independente, pois eles não são meros detalhes no ritual de
um sacrifício específico; ao contrário, eles são ordens que abrangem
todos os sacrifícios, como explicaremos em nossa enumeração.

Está claro que o que o Cohen toma como sua parte do sacrifício é
apenas um dos detalhes do preceito, tal como mencionamos com relação ao
couro da oferta de Holocausto. Esse também é o caso da primeira tosquia,
onde a essência do preceito consiste em separar a primeira lã do
carneiro e dá-la ao Cohen, assim como no caso do primeiro dízimo,
que devemos separar e dar ao Levita.

Outros se enganaram a este respeito, e contaram os vinte e quatro tipos
de presentes ao Cohen como vinte e quatro preceitos, depois de ter
contado alguns preceitos nos quais alguns desses presentes eram meros
detalhes, tal como explicamos com relação ao couro da oferenda de
Holocausto e ao peito e coxa do Sacrifício de Paz.

Além disso, devido ao fato de que eles não conheciam este Fundamento,
não se aperceberam dele, nem lhe prestaram atenção, resultando que eles
contaram, como preceitos separados, verter (óleo nos utensílios),
embeber (com óleo), cortar em pedaços, salgar, levar junto ao altar,
levantar, retirar um punhado e queimar, sem saber que todos esses atos
são detalhes do ritual da oblação. Ou seja, depois de ter-nos ordenado
oferecer a oblação de trigo, Ele começou a explicar a que se refere esse
nome --- o da ``lei da oblação de trigo'' --- e disse que se tratava de
flor de farinha, ou pão assado de uma certa forma --- sobre uma chapa,
numa panela a vapor ou no forno ---; depois deve-se embebê-la numa
determinada quantia de óleo, separá-la em pedaços e por nela sal e
incenso e deve-se levá-la junto ao altar e erguê-la, tomar um punhado
dela e queimá-la, de acordo com o procedimento que explicamos e
elucidamos no lugar apropriado: o Tratado Menahot. Todos esses são
detalhes do ritual que, quando executado de acordo com todo este
procedimento, é chamado de oblação. Assim sendo, o preceito é o
seguinte: somos obrigados a que o ritual do sacrifício do pão ou da flor
de farinha esteja de acordo com o procedimento assim definido. No caso
do preceito da oblação a oferenda consiste do seguinte: verter,
embeber, separar em pedaços, salgar, levantar, levar junto ao altar,
pegar um punhado e queimá-la, tal como Ele disse no caso do preceito
único da Halitzá: ``E lhe descalçará o sapato do pé, e cuspirá no
chão, diante dele, e responderá dizendo\ldots{}'',\footnote{Deuteronômio 25:9.} onde
não contamos o fato de tirar o sapato, o de cuspir e o de proferir as
palavras como preceitos independentes, uma vez que eles estão incluídos
no ritual de Halitzá, que constitui um preceito. Da mesma forma que
nesse caso, também não devemos contar separadamente as seguintes ordens:
``E deitarás sobre ela azeite'',\footnote{Levítico 2:6.} ``E porás sobre ela
incenso'',\footnote{Ibid., 15.} ``Temperarás com sal'',\footnote{Ibid., 13.} ``E
movimentará o sacerdote'',\footnote{Ibid., 23:20.} e a aproximará, ``E tirará''
dali um punho cheio\ldots{} e o fará queimar''.\footnote{Ibid., 2:2.}

Isto só passará despercebido a quem compreender os assuntos
superficialmente, sem examiná-los e avaliá-los em sua mente, tal como
dizem os Sábios de abençoada memória: ``Ele disse isso por ser
precipitado''. Quer dizer, ele disse isso sem refletir a respeito,
baseado apenas no primeiro pensamento que lhe veio à mente.

Assim, este Fundamento nos explicou as leis de todos os sacrifícios e a
maneira como elas devem ser contadas para que não haja nenhum engano nem
confusão, como as explicaremos em nossa enumeração, com a ajuda do Todo
Poderoso.

\chapter*{O décimo terceiro fundamento\subtitulo{Quando um determinado preceito tiver que\\ ser cumprido por vários dias não se deve\\ contar um preceito por cada dia}}

Está claro que há preceitos que são obrigatórios durante o transcorrer
de um determinado período de tempo. Às vezes esse período é contínuo,
quer dizer, a obrigação de cumprir um determinado preceito se sucede dia
após dia, tal como o Tabernáculo\footnote{Sentar na sucá\starr{} durante sete dias.} e os ramos de
palma.\footnote{O preceito do ``lulav''.} Outras vezes ele corresponde a certos dias,
como no caso dos sacrifícios. A título de exemplo, se devêssemos dizer
que o sacrifício adicional da lua nova constitui um preceito, isto
significaria que fomos ordenados a levar um sacrifício adicional toda
vez que houver lua nova.

Se alguém perguntasse por que não contamos o sacrifício adicional de
cada lua nova como um preceito em si, diríamos que, se assim fosse, você
também deveria contar o Holocausto de cada dia como um preceito em si,
assim como contar queimar o incenso e manter as lamparinas acesas, os
quais são obrigatórios todos os dias do ano, também como preceitos
individuais. Mas como contamos apenas o conceito do que nos foi
ordenado, sem levar em consideração o fator tempo com relação a sua
execução, nós contamos o sacrifício adicional da lua nova como um único
preceito, assim como o sacrifício adicional do Shabat e de cada um dos
cinco festivais, mesmo que eles sejam obrigatórios por vários dias
seguidos. Pois assim como Ele diz: ``E vos alegreis diante do Eterno,
vosso Deus, por sete dias'',\footnote{Levítico 23:40.} Ele diz também ``Sete dias
oferecereis ofertas queimadas ao Eterno'';\footnote{Ibid., 36.} assim como o
preceito do ramo de palma é um só, também é um só o preceito do
sacrifício adicional de Páscoa. A mesma regra se aplica aos sacrifícios
adicionais de cada uma das estações.

Com base neste Fundamento também ficará claro que o sacrifício de festa
é um único preceito, embora ele seja obrigatório nas três estações,
como também o é o de comparecer e o de alegrar-se. Ninguém deve
enganar-se nem pensar de maneira diferente.

Contudo, alguns cometeram um erro extremamente sério e estranho com
relação a este Fundamento: eles contaram todos os sacrifícios adicionais
--- o do Shabat, o das luas novas e o das festas --- como um único
preceito! Se assim fosse, eles deveriam ter contado o descanso em todos
os festivais como um preceito, mas não o fizeram. Mas o Eterno sabe e é
testemunha de que eles não devem ser criticados por isso, pois, de uma
maneira geral, eles não seguiram uma teoria ao fazer suas enumerações;
ao contrário, ``Eles subiram até o céu, eles desceram às profundezas''.\footnote{Ps. 107:26.} A verdade é o que eu lhes mencionei: que cada sacrifício
adicional constitui um preceito independente, assim como o descanso em
cada um dos festivais constitui um preceito diferente. Esta é a teoria
correta.

\chapter*{O décimo quarto fundamento\subtitulo{De que forma os tipos de castigo devem\\ ser contados como preceitos positivos}}

Você deve saber que todos os preceitos, positivos e negativos, estão
primeiramente divididos em duas partes, de acordo com o propósito deste
Fundamento. A primeira parte é aquela em que a Escritura não
especificou castigo algum, mas apenas estabeleceu uma ordem ou uma
proibição, e não sujeitou o transgressor a qualquer castigo nem lhe
designou um castigo determinado por transgredir aquela ordem ou
proibição específica. A segunda parte é a que estabelece a recompensa e
o castigo.

Entre os preceitos nos quais Ele explica o castigo estão os preceitos
que nos ordenam a apedrejar os transgressores de determinados preceitos,
a queimá-los, a executá-los com a espada como foi indicado na explicação
da Tradição, a estrangulá-los, e a açoitá-los com uma correia. A
determinados transgressores Ele impôs a extinção, isto é, o
transgressor que morrer em estado de pecado não terá um lugar no Mundo
que Há de Vir, conforme explicamos no capítulo ``Helek''; a outros Ele
impôs apenas a morte, isto é, Ele fará com que morram por seu pecado e
sua morte lhes trará a absolvição.

Os Sábios explicam no início de Macot que no caso de uma proibição cujo
castigo é a extinção ou apenas a morte pela Mão dos Céus --- se se
concluir que o transgressor pecou premeditadamente diante de testemunhas
e apesar das advertências --- o transgressor está sujeito ao
açoitamento, mesmo que seu castigo principal consista em que seu
julgamento caberá ao Céu. Há também preceitos nos quais Ele nos ordenou
castigar os transgressores de certos preceitos apenas com seu dinheiro,
não com seu corpo, tal como determinou a um assaltante que ele acrescentasse um quinto adicional e a um
ladrão que pagasse o dobro do que roubou. E também há preceitos em que
Ele nos ordenou que os violadores levem um sacrifício por seu pecado
para serem assim perdoados.

As aplicações de todas essas formas de castigo constituem preceitos
positivos, pois fomos ordenados a matar um, a açoitar outro, a apedrejar
aquele outro, ou a levar um sacrifício pelo que fizemos. Quanto a
inclui-los na enumeração, devemos contar as quatro penas de morte
impostas pelo Tribunal como quatro preceitos positivos. Tal é, na
realidade, a expressão da Mishné: ``Este é o preceito dos que devem ser
apedrejados''. Eles dizem também: ``De que maneira deve ser o preceito
de queimar?'', ``De que maneira deve ser o preceito de estrangular?'',
``De que maneira deve ser o preceito de decapitar?''.

Os Sábios também dizem que Suas palavras, enaltecido seja Ele, ``Não
acendereis fogo''\footnote{Êxodo 35:3.} são uma advertência para que não se
aplique castigos no Shabat. Ou seja, este versículo nos proíbe de
executar um preceito que nos ordene queimar alguém, pois a expressão
``Em todas as vossas habitações''\footnote{Ibid.} significa que não se deve
acender fogo no Tribunal, mesmo que isso seja um preceito positivo.
Portanto os Sábios dizem: ``Acender um fogo, que está incluído nas
categorias de tarefas proibidas no Shabat, está destacado para
ensinar-nos que assim como as leis do Shabat não podem ser
desconsideradas no que se refere ao tipo de execução especificamente
mencionado, elas também não podem ser desconsideradas no caso de outros
tipos de execução judicial''. Isto está claro e ninguém terá dúvidas a
respeito. Da mesma forma, devemos contar o açoitamento com uma correia
como um preceito individual.

Entretanto, não se deve fazer o que fizeram outros sem meditar, ou seja,
contar cada castigo em particular como um preceito separado dizendo, a
título de exemplo, que a ordem de apedrejar aquele que profanar o Shabat
é um preceito, que o apedrejamento daquele que pratica o ``Ob'' é um
segundo preceito, e que o apedrejamento daquele que adora ídolos é um
terceiro preceito, resultando que o número de preceitos corresponderá ao
número de pessoas sujeitas às quatro penas de morte impostas pelo
Tribunal. Se assim fosse, nós fatalmente teríamos que contar cada
açoitamento em separado, fazendo com que o açoitamento de quem come
``nebelá'' seja um preceito individual, o de quem come carne de porco um
segundo preceito, o de quem come carne cozida no leite um terceiro
preceito, o de quem usa ``shaatnez'' um quarto preceito, resultando
assim que teríamos tantos preceitos positivos quanto o número de
preceitos negativos que acarretam o açoitamento. Dessa maneira o número
de preceitos positivos aumentaria e chegaria com certeza a mais de
quatrocentos! Ao invés disso, assim como não contamos separadamente
todos os que estão sujeitos ao açoitamento, e sim apenas o tipo de
castigo --- a saber, o açoitamento com uma correia ---, devemos contar
também nas penas de morte apenas as formas de execução, que são pelo
fogo, por apedrejamento, por estrangulamento e por decapitação. Da
mesma forma, não devemos contar separadamente todos os que estão
sujeitos a oferecer um sacrifício, dizendo que o sacrifício de pecado
daquele que viola o Shabat sem querer é preceito, e que o Sacrifício de
Pecado de quem adora ídolos sem querer é preceito; em vez disso devemos
contar apenas o tipo de sacrifício, tal como fizemos nos tipos de pena
de morte.

Você já sabe que o tipo de sacrifício que se é obrigado a oferecer varia
de acordo com o tipo de pecado que se cometeu. Há pecados pelos quais se
oferece o Sacrifício de Pecado, ou o Sacrifício Suspensivo de Delito, ou
o Sacrifício Incondicional de Delito, ou o Sacrifício de Maior ou Menor
Valor.
É por essa razão que não devemos contar o Sacrifício de Pecado
juntamente com o Sacrifício de Delito; ao invés disso, contaremos a
obrigação do Sacrifício de Pecado, a do Sacrifício Suspensivo de
Delito, a do Sacrifício Incondicional de Delito e a do Sacrifício de
Maior ou Menor Valor como preceitos separados, sendo que as obrigações
são de responsabilidade da pessoa que deve oferecer aquele tipo
específico de sacrifício. Não voltaremos nossa atenção para os vários
pecados pelos quais se é obrigado a oferecer um sacrifício específico,
da mesma forma que contamos o açoitamento como um só preceito e
desconsideramos os vários pecados que acarretam esse castigo. Da mesma
forma, a Escritura dedicou uma seção especial a cada tipo.

Outros já fizeram tanta confusão com relação a este Fundamento que se
torna desnecessário refutá-los, nem seria fácil fazê-lo, tamanha é a
desordem que eles implantaram a este respeito.

De fato, devemos espantar-nos e surpreender-nos com uma pessoa que conta
um a um, entre os preceitos negativos, todos aqueles que estão sujeitos
a alguma das penas de morte aplicadas pelo Tribunal, bem como os que
estão sujeitos à extinção e à morte, além de contar também os atos
proibidos cuja violação implica numa daquelas formas de morte! Foi isso
o que fez o autor do Halachot Guedolot. Ele contou ``quem profanar o
Shabat'' entre aqueles que estão sujeitos à morte por apedrejamento, e
depois contou ``Não farás nenhuma obra''.\footnote{Êxodo 20:10.} Devemos de fato
concluir que eles pensaram inicialmente que a execução judicial
constitui um preceito negativo em si. Mas se assim fosse, como poderiam
eles conter o castigo e a proibição específica pela qual se aplica o
castigo?

Ainda mais surpreendente é o fato de que eles contaram entre os
preceitos negativos os que estão sujeitos à extinção, bem como os que
estão sujeitos à morte pela Mão dos Céus, os quais não envolvem
execução! Deve ser porque eles imaginam que ficar sujeito à extinção e
que a aplicação do castigo constituem a natureza desse preceito
específico. De fato, foi assim que o autor do ``Sefer Hamitzvot''
explicou isso ao resumir o conteúdo do primeiro capítulo nas seguintes
palavras: ``Entre estes há trinta e dois conceitos sobre os quais Ele
nos informa que Ele --- abençoado e enaltecido seja ---, e não nós, o
fará seguramente cumprir.'' ``Entre estes'' significa entre os conceitos
mencionados naquele capítulo. Os ``trinta e dois conceitos''
compreendem, de acordo com sua enumeração, vinte e três casos sujeitos
apenas à extinção e nove sujeitos à morte pela Mão dos Céus, e procede a
sua enumeração. Com a palavra ``seguramente'' ele quer dizer que o
Eterno, enaltecido seja Ele, assegurou que Ele aplicará a extinção a
este e a morte ao outro. Não há dúvidas de que esse homem se separou
completamente da ideia de que \emph{todos} os seiscentos e treze
preceitos são obrigação \emph{nossa}; em vez disso, alguns seriam nossa
obrigação e outros obrigação d'Ele, enaltecido seja Ele, tal como ele
afirmou claramente: ``Ele os fará cumprir, e não nós''. Deus sabe e é
testemunha de que na minha opinião tudo isto é uma confusão absoluta e
não há necessidade alguma de falar a respeito pois a invalidez de suas
palavras é óbvia. O motivo deste erro é que ao contar os castigos como
preceitos eles se confundiram, e algumas vezes contaram os castigos e
também as ações que os acarretam, estabelecendo tudo como preceito
negativo, sem refletir a respeito.

Contudo, a forma correta de enumeração é como eu mencionei: cada
\emph{tipo} de castigo constitui um preceito positivo. De acordo com
isso, a lei de restituição referente a um ladrão é um preceito positivo,
a saber, que somos ordenados a impor-lhe uma determinada quantia. Assim
também são as seguintes leis: o quinto adicional, a obrigação do
Sacrifício de Pecado, o Sacrifício
Incondicional de Delito, o Sacrifício Suspensivo de Delito e o
Sacrifício de Maior ou Menor Valor. Da mesma forma, cada um dos
seguintes castigos: apedrejamento, queima, decapitação, estrangulamento
e enforcamento constitui um preceito individual que se aplica a todo
aquele que ficar sujeito a eles, tal como o açoitamento com a correia
constitui um só preceito que se aplica a todo aquele que ficar sujeito a
esse castigo. Isso é o que desejávamos expor neste Fundamento, e com
ele completamos os Fundamentos, cuja introdução ajudará naquilo em que
nos empenhamos.

É importante acrescentar a seguinte introdução. Todo pecado, pelo qual a
penalidade é a execução judicial ou a extinção, é necessariamente um
preceito negativo, a não ser o sacrifício de Páscoa e a circuncisão, que
acarretam a extinção mesmo sendo preceitos positivos, como explicamos
no início do Tratado Queretot. Não temos nenhum outro preceito positivo
a não ser esses por cuja transgressão se fique sujeito à extinção --- e
mais ainda, à execução judicial. Sendo assim, toda vez que a Torá
disser que aquele que cometer um determinado ato deverá ser morto ou
estará sujeito à extinção saberemos perfeitamente que esse ato
específico está proibido e constitui um preceito negativo.

Às vezes as Escrituras apresentam uma proibição sem expor o castigo,
embora tanto o castigo como a advertência estejam claros. Esse, por
exemplo, é o caso da profanação do Shabat e da adoração de ídolos, com
relação aos quais Ele disse: ``Não farás nenhuma obra''\footnote{Êxodo 20:10.} e
``Nem os servirás''\footnote{Ibid., 5.} e depois disso Ele declara que aquele
que trabalha e aquele que serve estão sujeitos ao apedrejamento.

Algumas vezes as Escrituras não mencionam claramente a proibição de um
determinado ato mas declaram apenas a punição, omitindo a advertência.
Contudo, como é regra para nós que ``não há nenhum castigo estipulado na
Torá sem que uma advertência o tenha precedido'' deve haver de alguma
forma uma advertência com relação ao ato que nos sujeita ao castigo.
Isto é o que os Sábios sempre dizem: ``Ouvimos o castigo, mas não
ouvimos a proibição. Por isso a Torá diz isto e aquilo''. E se a
advertência não estiver explicitamente enunciada nas Escrituras, eles a
deduzem através de um dos Princípios. Esse é, por éxemplo, o caso do que
os Sábios dizem com relação às proibições de amaldiçoar ou de bater no
pai, as quais não estão de modo algum explicitamente mencionadas nas
Escrituras, pois em lugar algum está dito ``Não amaldiçoe seu pai'' nem
``Não bata em seu pai''; em vez disso Ele declarou que aquele que bater
ou amaldiçoar está sujeito à morte. Á partir dessas palavras deduzimos
que estes atos são proibidos e que os Sábios extraíram delas, através de
um dos princípios exegéticos, as advertências referentes a esses atos,
assim como fizeram em casos semelhantes, em outros lugares.

Este método de dedução de uma advertência não contradiz de forma alguma
o que os Sábios frequentemente dizem: ``Uma lei derivada por analogia
não é considerada uma lei específica pela qual se possa ser castigado'',
nem o princípio de que ``Acaso se pode castigar pela violação de uma lei
cuja advertência se deduziu por analogia?'' O objetivo da declaração
``Uma lei derivada por analogia não é considerada uma lei específicg
pela qual se possa ser castigado'' é apenas de proibir-nos de derivar
por analogia um assunto com relação ao qual não haja proibição
\emph{alguma} mencionada; mas se encontrarmos a punição contra fazer um
determinado ato claramentç exposta na Torá, saberemos com certeza que
esse ato está proibido e que fomos advertidos para não fazê-lo. E apenas
para estar de conformidade com a regra de que ``Nenhum castigo está
estabelecido na Torá a menos que uma advertência proibitiva o tenha
precedido'' que fazemos deduções a partir de um dos princípios, quando
Ele tiver se referido a essa advertência. E uma vez encontrada a
advertência contra esse ato, o violador que o fizer fica sujeito à
extinção ou à execução judicial.

Assim, fique ciente desta introdução e recorde-se dela e de todos os
Fundamentos precedentes juntamente com tudo o que tencionamos mencionar
a seguir.

Agora começarei a mencionar todos os preceitos, um a um, explicando-os
apenas a fim de elucidar o título do preceito, como prometemos no início
de nosso estudo, pois esse é o objetivo deste trabalho. Entretanto,
creio que é aconselhável acrescentar o seguinte ao nosso objetivo.
Quando me referir a um preceito, positivo ou negativo, que acarreta
algum castigo, mencionarei o castigo dizendo ``Aquele que o violar está
sujeito à morte, ou à extinção, ou a oferecer determinado sacrifício, ou
ao açoitamento, ou a uma das penas de morte impostas pelo Tribunal, ou a
pagamento''. E todas as vezes que nenhum castigo for mencionado você
deverá saber que se for com relação a um dos preceitos negativos a regra
a ser aplicada é, como dizem os Sábios: ``Como um homem que viola o
preceito do Rei'' e não cabe a nós puni-lo. Mas com relação a todos os
preceitos positivos, quando sua execução ainda for aplicável, devemos
açoitar com uma correia aquele que se recusar a fazê-lo até que
ele morra ou cumpra, ou até que passe o momento da obrigação pois aquele
que violar o preceito positivo de viver num
Tabernáculo\footnote{Sentar na sucá\starr{} durante sete dias.} não deve ser açoitado por seu pecado depois dos Tabernáculos. Saiba disso.

Além disso, quando eu mencionar os preceitos, positivos ou negativos,
que não são obrigatórios para as mulheres, direi: ``E este não é
obrigatório para as mulheres''. É sabido que as mulheres não estão
qualificadas para julgar nem testemunhar, nem oferecer elas próprias os
sacrifícios, nem tomarem parte numa guerra opcional. Consequentemente,
não será necessário que eu diga: ``E este não é obrigatório para as
mulheres'' com respeito a todos os preceitos relativos ao Tribunal, a
testemunhos ou ao Serviço, pois isto seria apenas redundante e
desnecessário.

Também quando eu mencinar os preceitos, positivos ou negativos, que são
obrigatórios apenas na terra de Israel ou enquanto existir o Templo,
direi: ``E este é obrigatório apenas na terra de Israel, ou enquanto o
Templo existir''.

É sabido que todos os sacrifícios eram levados apenas ao Templo, e que
tal ritual está proibido fora do Campo. Da mesma forma, as leis de
punição capital são impostas apenas durante a existência do Templo. A
Mekhiltá diz: ``De que forma sabemos que a execução judicial só pode ter
lugar enquanto o Templo existir? Pelo que dizem as Escrituras: ``Do meu
altar o tirarás, para que morra''.\footnote{Êxodo 21:14.} Também é sabido que a
profecia e o Reino ficarão desaparecidos de nosso meio até o momento em
que desistamos dos pecados a que nos habittiamos, quando então Deus nos
perdoará e terá misericórdia de nós, de acordo com o que Ele nos
prometeu, e nos restituirá, como disse com relação ao retorno da
profecia: ``E virá depois, verterei Meu espírito sobre toda carne, e
vossos filhos e filhas profetizarão'';\footnote{Joel 3:1.} e com relação ao
retorno do reino e do poder Ele disse: ``Nesse dia levantarei o
Tabernáculo de Davi que caiu, fecharei suas brechas, levantarei suas
ruínas e o edificarei como nos velhos dias''.\footnote{Amos 9:11.} E também é
sabido que a guerra e a conquista da terra só será feita com um rei, sob
o comando do Grande Sanhedrin e do Cohen Gadol, tal como Ele disse:
``E diante de Elazar, o Cohen''.\footnote{Números 27:21.}


Como todos estes assuntos são do conhecimento da maioria das pessoas,
toda vez que houver um preceito positivo ou negativo relativo a
sacrifícios, rituais, a penas de morte impostas pelo Tribunal, pelo
Sanhedrin, ou pelo profeta e rei, ou à guerra obrigatória ou opcional,
não será necessário que eu diga que ``Este preceito se aplica apenas
durante a existência do Templo'', pois isto ficou claro, de acordo com o
que mencionamos. Contudo, nos casos em que possa surgir alguma dúvida ou
engano eu comentarei a respeito, se Deus quiser.

E agora começarei a mencionar todos os preceitos, com a ajuda do Todo
Poderoso.

\begingroup\makeatletter\@openrightfalse
\part{\textsc{os 248 preceitos positivos}}

\vspace*{\fill}
\thispagestyle{empty}
\begin{tabular}{lr}
\textbf{\textsc{tema}} & \textbf{\textsc{preceitos}} \\
\textsc{a crença em deus} & 1 a 19 \\
\textsc{o santuário} & 21 a 95 \\
\textsc{a purificação} & 96 a 113 \\
\textsc{os dízimos e as doações} & 114 a 152 \\
\textsc{os festivais} & 153 a 171 \\
\textsc{a ética do estado} & 171 a 193 \\
\textsc{os deveres para com os semelhantes} & 194 a 209 \\
\textsc{os deveres para com a família} & 210 a 223 \\
\textsc{a aplicação das leis criminais} & 224 a 231 \\
\textsc{o direito da propriedade} & 232 a 248
\end{tabular}

\@openrighttrue\makeatother \endgroup

\pagebreak
\movetooddpage


\setcounter{paragraph}{0}
\setcounter{secnumdepth}{4}

\paragraph{Crer em Deus}

Por este preceito somos ordenados a crer em Deus, ou seja, a acreditar
que há um agente supremo que é o criador de tudo o que existe. Ele está
expresso em suas palavras, enaltecido seja Ele, ``Eu sou o Eterno, teu
Deus, que te tirei da terra do Egito''.\footnote{Êxodo 20:2.}

No final do Tratado Macot está dito: ``Seiscentos e treze preceitos
foram comunicados a Moisés no Sinai, como diz o verso: `A Lei que nos
ordenou Moisés'\,'';\footnote{Deuteronômio 33:4.} ou seja, ele nos ordenou obedecer a tantos preceitos quantos há na soma das \emph{letras-números} Torá. 

A isso objetou-se que as \emph{letras-números} da palavra Torá somam apenas seiscentos e onze,\footnote{A palavra \emph{Torá}, em valores numéricos, significa 611, ou seja, Tav (400); Vav (6); Resh (200); He (5).} e a resposta foi: ``Os dois preceitos `Eu sou o Eterno, teu Deus' e `Não terás outros
deuses diante de Mim'\footnote{Êxodo 20:3.} foram ouvidos do próprio Todo
Poderoso''.

Portanto, foi deixado claro que o versículo ``Eu sou o Eterno, teu
Deus'' é um dos 613 preceitos, e é o que nos ordena a crer em Deus, como
explicamos.

\paragraph{A unidade de Deus}

Por este preceito somos ordenados a crer na unidade de Deus, ou seja, a
acreditar que o criador de todas as coisas existentes e primeiro agente
delas é uno. Este preceito está expresso em suas palavras, enaltecido
seja Ele, ``Escuta, Israel! O Eterno é nosso Deus, o Eterno é Uno!''\footnote{Deuteronômio 6:4.}

Em quase todo midrashot você vai encontrar palavras que
expõe como devemos declarar a unidade do nome de Deus, ou a unidade
de Deus, ou algo nesse sentido. A intenção dos sábios era ensinar que
Deus nos tirou do Egito e nos cumulou de bondade apenas com a condição
de que acreditemos em sua unidade, e esse é nosso dever.

O preceito de crer na unidade de Deus está mencionado em muitos lugares,
e os sábios também o chamam de preceito de crer no Reino dos Céus,
pois eles falam da obrigação ``de tomar a nosso cargo a união com o
Reino dos Céus'', ou seja, de declarar a unidade de Deus e de crer
n'Ele.

\paragraph{Amar a Deus}

Por este preceito somos ordenados a amar ao Eterno, enaltecido seja
Ele, ou seja, a deter-nos e a meditar sobre seus preceitos, suas ordens,
e seus trabalhos, de maneira a obter uma concepção d'Ele, e ao
concebê-lo, alcançar o júbilo absoluto, e isto é o amor que nos foi
ordenado. Como diz o Sifrei: ``Uma vez que dizemos `E amarás ao Eterno,
teu Deus',\footnote{Deuteronômio 6:5.} o estudioso perguntará: `Como se deve
manifestar seu amor pelo Eterno?'. As Escrituras dizem: `E estarão
estas palavras que Eu te ordeno hoje, no teu coração',\footnote{Deuteronômio
6:6.} porque é através disto que você aprenderá a conhecer aquele cuja
palavra ordenou ao universo sua existência''.

Desta forma ficou claro que através deste ato de meditação você vai
alcançar a concepção de Deus e alcançar o estado de júbilo no qual o
amor a Deus será uma consequência necessária.

Os sábios dizem que este preceito também inclui a obrigação de convocar todos os descendentes de Adam\footnote{Os filhos de Adam no contexto da humanidade em sua totalidade.} para servi-Lo, louvado seja Ele, e ter fé n'Ele. 

Porque da mesma forma que você exalta e glorifica a
quem ama, e convoca os outros homens para amá-lo, se você ama o Eterno
até a concepção de sua verdadeira natureza, que já alcançou através do
conhecimento, sem dúvida convocará os tolos e os ignorantes a procurar
o conhecimento da verdade que você já encontrou.

Como diz o Sifrei: ```E amarás ao Eterno, teu Deus': isso significa que
você deverá fazer com que Ele seja amado pelos homens, como o fez o seu
pai Abraham, como foi dito: `E as almas que haviam adquirido em Haran'\,''.\footnote{Gênesis 12:5.} Ou seja, da mesma forma que Abraham, sendo um amante do Eterno --- como a Torá testemunha, quando designado pelo Eterno como
sendo: ``Meu amado Abraham''\footnote{Isa 41:8.} --- pela força de sua
concepção de Deus e pelo seu grande amor por Ele, convocou a humanidade a crer, assim você deve amá-lo de forma tal a atrair a humanidade para Ele.

\paragraph{Temer a Deus}

Por este preceito somos ordenados a crer no temor a Deus, enaltecido
seja Ele, de maneira a não ficar acomodados e autoconfiantes, e sim a
esperar sempre seu castigo. Este preceito está expresso em suas
palavras ``Ao Eterno, teu Deus, temerás''.\footnote{Deuteronômio 6:13; 10:20.}

A Guemará do Tratado Sanhedrin\starr{} comenta da seguinte forma o versículo
``Aquele que insultar (\emph{nokev}) o nome real do Eterno certamente
será morto''.\footnote{Levítico 24:16.} ``Talvez a palavra \emph{nokev} devesse
significar `declarar', já que encontramos em outro lugar `Estes homens
que foram declarados (\emph{nikevu}) por nomes',\footnote{Números 1:17.} provindo a advertência do versículo `Ao Eterno, teu Deus, temerás'\,''. 

Ou seja, o versículo ``Aquele que insultar o nome real do Eterno'' deve ser
entendido como significando aquele que simplesmente mencionar o nome do
Eterno sem louvá-lo; e se você perguntar: ``Que pecado há nisso?'' responderemos que quem o fizer estará abandonando o temor
ao Eterno, porque faz parte do temor ao Eterno não pronunciar seu nome
em vão.

Os sábios respondem a esta pergunta, e contestam a perspectiva envolvida
nela, como segue: ``Primeiro, para que constitua um insulto, o nome deve
ser utilizado, e neste caso essa condição está ausente''; ou seja, ele
deve ser culpado de insultar o nome em nome d'Ele, tal como eles dizem:
`Deixe Iossi castigar Iossi'\,''.

``Além do mais, a advertência que você cita está na forma de um preceito
positivo, e é um princípio aceito que tal tipo de advertência não é
válida''. Quer dizer, sua teoria de que a proibição do mero
pronunciamento do Nome de Deus pode provir do versículo ``Ao Eterno, teu
Deus, temerás'' é inadmissível porque esse versículo é um preceito
positivo e uma proibição não pode ser baseada num preceito positivo.

Assim, foi-lhe deixado claro que as palavras ``Ao Eterno, teu Deus,
temerás'' estipulam um preceito positivo.

\paragraph{Servir a Deus}

Por este preceito somos ordenados a servir a Deus, enaltecido seja Ele.
Este preceito está repetido várias vezes nas Escrituras, como foi dito
em: ``E servireis ao Eterno, vosso Deus'';\footnote{Êxodo 23:25.} ``e a Ele
servireis'';\footnote{Deuteronômio 13:5.} ``e a Ele servirás'';\footnote{Ibid 6:13.} ``e servi-Lo''.\footnote{Ibid 11:13.}

Embora este seja da categoria dos preceitos gerais que estão excluídos dos 613 preceitos pelo Quarto Fundamento, ainda assim ele impõe uma obrigação
específica, que é a oração. O Sifrei diz: ``Servi-Lo\footnote{Ibid 11:13.}
significa oração''. Os sábios também dizem: ```Servi-lo' significa
estudar a lei''.

Na Mishná de Rabi Eliezer, filho de Rabi Iossi Hagalili, está dito:
``De que maneira ficamos sabendo que a oração é obrigatória? Através do
versículo `Ao Eterno, teu Deus, temerás, e a Ele servirás'\,''.\footnote{Ibid
6:13.}

Os sábios também dizem: ```Servi-lo' através da sua Torá e
`servi-lo' em seu santuário'', o que significa que devemos aspirar a
rezar no Templo ou voltados em sua direção, como disse claramente
Salomão.\footnote{Reis 8:30.}

\paragraph{A aproximação de Deus}

Por este preceito somos ordenados a juntarmo-nos e a associarmo-nos com
os homens sábios, a estar sempre em sua companhia, a unirmo-nos a eles e
a seguir seus caminhos através de toda forma possível de
companheirismo: comendo, bebendo e a negócios, com a finalidade de
conseguir ser como eles, quanto as suas ações, e de acreditar nos
conceitos verdadeiros através de suas palavras. Este preceito está
expresso em Suas palavras, enaltecido seja Ele, ``E d'Ele te
aproximarás''\footnote{Deuteronômio 10:20.} que estão 
repetidas no versículo ``E aproximando-vos d'Ele''.\footnote{Ibid 11:22.}
O Sifrei diz: ```E aproximando-vos d'Ele' significa que devemos aproximar-nos dos homens sábios e de seus discípulos''.

Os sábios também usam as palavras ``E d'Ele te aproximarás'' como prova
de que é nosso dever casarmo-nos com a filha de um homem sábio, dar
nossa própria filha em casamento a um discípulo de um homem sábio,
beneficiar os homens sábios e manter negócios com eles. Como está dito:
``Existe a possibilidade para um ser humano de aproximar-se da presença
divina, visto que está escrito `Porque o Eterno, teu Deus, é um fogo
consumidor'?\footnote{Ibid 4:24.} Dessa forma devemos concluir, de acordo com
este versículo, que o casamento com a filha de um sábio deve ser
considerado como um meio de aproximar-se do Eterno''.

\paragraph{Jurar em nome de Deus}

Por este preceito somos ordenados a jurar exclusivamente em seu nome,
enaltecido seja Ele, toda vez que nos for solicitado confirmar ou negar
alguma coisa sob juramento, porque fazendo isso nós o estaremos
exaltando, honrando e magnificando. Este preceito está expresso em suas
palavras, enaltecido seja Ele, ``E pelo seu nome jurarás'',\footnote{Deuteronômio 6:13; 10:20.} que os Sábios explicam assim: ``a Torá diz
`Em seu nome jurarás' e novamente `Não jurarás em nome do Eterno, teu
Deus, em vão'\,''.\footnote{Êxodo 20:7.} Da mesma forma que estamos proibidos de
fazer um juramento sem necessidade, e isto é um preceito
negativo,\footnote{Ver preceito negativo 62.} assim também somos ordenados a fazer um juramento quando necessário, e é um preceito positivo.

É por essa razão que não é permitido jurar por nenhuma entidade criada,
tal como os anjos ou as estrelas, exceto quando o juramento é elíptico
como, por exemplo, quando se jura pela realidade do sol, significando
``pela realidade do Deus do sol''. E assim que nossa nação jura pelo
nome de Moisés nosso Mestre --- honrado seja seu nome --- como se aquele
que jura estivesse dizendo ``pelo Deus de Moisés'' ou ``por Aquele que
mandou Moisés''. Mas quando aquele que jurar não tiver a intenção de
dizer isso dessa forma e jurar por um ente criado, acreditando que essa
entidade contém em si própria uma verdade tal que se possa jurar por
ela, ele estará cometendo um pecado ao colocar alguma outra entidade em
pé de igualdade com o Nome dos Céus; a Tradição\footnote{Sucá 45:b.}
diz, a esse respeito: ``Aquele que colocar o Nome dos Céus em pé de
igualdade com algum outro ente deverá ser erradicado da face da terra''.

Este era o significado pretendido no versículo ``Pelo Seu Nome jurarás
apenas'': em nome d'Ele você deve atribuir fé a uma verdade que seja
digna de um juramento.

Está dito no início do Tratado Temurá:\footnote{Temurá 3:B.} ``De que
modo ficamos sabendo que podemos nos comprometer por um juramento a cumprir os preceitos? Pelo versículo `E pelo Seu Nome jurarás'\,''.

\paragraph{Trilhar os caminhos de Deus}

Por este preceito somos ordenados a assemelhar-nos a Deus, enaltecido
seja Ele, o tanto quanto nos for possível. Este preceito está expresso
em Suas palavras ``E andares por Seus caminhos''.\footnote{Deuteronômio 28:9.}
Este preceito foi repetido várias vezes, e disse: ``que andes em todos os Seus
caminhos''.\footnote{Ibid. 11:22.}

Com relação a este último versículo os Sábios comentam o seguinte:
``Assim como o Sagrado, enaltecido seja Ele, é chamado Misericordioso,
você também deve ser misericordioso; assim como Ele é chamado
Benevolente, você também deve ser benévolo; assim como Ele é chamado
Justo, você também deve ser justo; assim como Ele é chamado Hassid\starr,
você também deve ser hassid\starr''.

Este preceito já apareceu sob outra forma em Suas palavras ``Após o
Eterno, vosso Deus, andareis''\footnote{Deuteronômio 13:5.} que os Sábios
explicam como significando que devemos imitar as boas ações e os
elevados atributos pelos quais o Eterno, enaltecido seja Ele, é
figurativamente descrito, uma vez que Ele é de fato sublime de forma
imensuravelmente superior a toda essa descrição.



\paragraph{Santificar o nome de Deus}

Por este preceito somos ordenados a santificar o Nome de Deus. Este
preceito está expresso em Suas palavras, ``E serei santificado entre os
filhos de Israel''.\footnote{Levítico 22:32.} O conteúdo deste preceito é que
temos o dever de pregar esta verdadeira religião pelo mundo, sem temer
danos de qualquer espécie. Mesmo se um tirano tentar forçar-nos a
negá-Lo, não devemos obedecer, e ao invés disso devemos preferir a
morte; e não devemos sequer enganar o tirano fazendo-o crer que O
negamos, embora em nossos corações continuemos a crer n'Ele, enaltecido
seja Ele.

Este é o preceito relativo à Santificação do Nome que foi imposto a cada
um dos filhos de Israel: que devemos estar prontos a morrer nas mãos de
um tirano por nosso amor a Ele, enaltecido seja Ele, e por nossa fé em
Sua Unidade, assim como fizeram Hananiah\starr, Mishael\starr{} e Azariah\starr{} no tempo do
perverso

Nabucodonosor, quando ele forçou o povo a prostrar-se diante do
ídolo\footnote{Dan. 3:1.} e todos assim o fizeram, inclusive os
israelitas, e não havia mais ninguém lá para santificar o Nome dos
Céus, pois estavam todos aterrorizados. Esta foi uma grande desgraça
para Israel, pois este preceito foi descumprido por todos eles.

Este preceito só se aplica em ocasiões como aquela, quando todo
o mundo estava aterrorizado e era um dever declarar publicamente,
naquela ocasião, a Sua Unidade. O Eterno já havia prometido, através de
Isaías, que Israel não seria desgraçada por completo naquele momento
difícil, pois entre eles surgiriam jovens sem medo da morte que
derramariam seu sangue e proclamariam a Fé, santificando o Nome
publicamente, como ele nos ordenou através
de Moisés, nosso Mestre. Essa promessa está nas palavras ``Agora Jacob\starr{}
não ficará envergonhado, nem ficará seu rosto pálido; quando ele vir que seus
filhos, o
trabalho de Minhas mãos, no meio dele, santificam o Meu
nome''.\footnote{Isa. 29:22-23.}

A Sifrá\starr{} diz: ``Eu vos tirei da terra do Egito com a condição de que
vocês santifiquem Meu nome publicamente''.

Na Guemará do Tratado Sanhedrin\starr{} está dito: ``Um Noachid\starr{} é obrigado a
santificar Seu Nome, ou não? Ouçam isto: `Aos Noachidos\starr{} foi ordenado que obedecessem a sete preceitos; mas se a eles foi ordenado que
santificassem Seu Nome, então seriam oito''.\footnote{O termo ``Noachidos'' (descendentes de Noach) significa os não
israelitas ou pagãos de todos os tempos que, de acordo com a lei
judaica, são obrigados a obedecer aos sete preceitos seguintes:
(1) estabelecer tribunais de justiça; (2) não praticar idolatria;
(3) não blasfemar; (4) não cometer incesto; (5) não matar;
(6) não roubar; (7) e não comer carne retirada de animais enquanto vivos. Os Noachidos que observam estes sete preceitos herdarão uma parte do Mundo Vindouro.}

Assim, ficou claro que este é um dos preceitos obrigatórios para
Israel, tendo os Sábios deduzido este preceito das palavras ``Eu serei
santificado entre os filhos de Israel''. As leis detalhadas sobre este
preceito estão expostas no sétimo capítulo de Sanhedrin\starr.

\paragraph{Ler o Shemá}

Por este preceito somos ordenados a ler o Shemá\starr{} diariamente, à noite e
pela manhã. Este preceito está expresso em Suas palavras, enaltecido
seja Ele, ``E delas falarás sentado em tua casa, andando pelo caminho, e
ao deitar-te e ao levantar-te''.\footnote{Deuteronômio 6:7.}

As leis referentes a este preceito estão explicadas em Berakhot\starr, onde
está demonstrado que a leitura do Shemá\starr{} foi ordenada pela Torá.

A Tosseftá\starr{} diz: ``Da mesma maneira que a Torá ordenou um horário
determinado para a leitura do Shemá\starr{}, os Sábios determinaram um horário
para a Oração''; ou seja, os horários da Oração não são determinados
pela Torá, mas o dever da oração em si é imposto pela Torá, como já
explicamos, e os Sábios determinaram os horários da Oração.

Os Sábios fixaram os horários da Oração de maneira a corresponder aos
horários em que os sacrifícios eram trazidos, nos tempos dos grandes
templos de Jerusalém.

Este preceito não é obrigatório para as mulheres.

\paragraph{O estudo da Torá}

Por este preceito somos ordenados a ensinar e a estudar a sabedoria da
Torá, que é chamada de Talmud Torá. Este preceito está expresso em
Suas palavras ``E as inculcarás a teus filhos''.\footnote{Deuteron 6:7.}

O Sifrei diz: ```A teus filhos' significa a teus estudantes: verificamos
que em todo lugar os discípulos de um homem são chamados de seus filhos,
como está dito em `E os filhos dos profetas saíram'\,''.\footnote{Reis II, 2:3.} O Sifrei também diz,
no mesmo trecho: ```E as inculcarás a teus filhos', o que significa que
elas devem fluir facilmente de sua boca, para que quando uma pessoa faça uma
pergunta sobre elas, você não vacile em sua resposta, e responda com
presteza''.

Este preceito está repetido muitas vezes: ``E ensiná-las-eis'';\footnote{Deuteronômio 11:19.} ``E para que aprendam''.\footnote{Deuteronômio 31:12.} A
importância deste preceito e a obrigação de cumpri-lo estão enfatizadas
em várias passagens do Talmud.

As mulheres não são obrigadas a obedecer a este preceito, de acordo com
Suas palavras ``E ensiná-las-eis a vossos filhos''\footnote{Ibid., 11:19.} sobre
as quais os Sábios comentam: ```Filhos', mas não filhas'', como foi
explicado na Guemará de Kidushin.

\paragraph{O tefilin da cabeça}

Por este preceito nos é ordenado o uso do tefilin da Cabeça. Este
preceito está expresso em Suas palavras ``E serão por marca entre os
teus olhos''.\footnote{Deuteronômio 6:8.}

Este preceito está repetido quatro vezes.\footnote{Êxodo 13:9; ibid., 16;
Deuteronômio 6:8; ibid., 11:18.}

\paragraph{O tefilin do braço}

Por este preceito nos é ordenado o uso do tefilin do Braço. Este
preceito está expresso em Suas palavras, enaltecido seja Ele, ``E os
atarás, como sinal na tua mão''.\footnote{Deuteronômio 6:8.} Também este preceito
está repetido quatro vezes.\footnote{Êxodo 13:1; ibid., 16; Deuteronômio 6:8;
ibid., 11:18.}

Na Guemará Menahot encontramos a prova de que o uso do tefilin da
Cabeça e do tefilin do Braço são dois preceitos, quando os Sábios se
mostram surpresos com a ideia de que o tefilin da Cabeça e o do
Braço não possam ser usados um sem o outro, e sim apenas os dois juntos.
``Se'', dizem eles, ``alguém não puder cumprir dois preceitos, ele não
deve cumprir um?''. Quer dizer, se alguém não puder cumprir os dois
preceitos ele não deve cumprir nenhum deles? Não, ele deve cumprir
aquele que ele puder. Dessa maneira ele deve usar o tefilin que
possuir.

Assim ficou claro que os Sábios consideram o tefilin do Braço e o
tefilin da Cabeça como sendo dois preceitos.

Estes dois preceitos não são obrigatórios para as mulheres, pois quando
explicou sobre sua obrigatoriedade, Ele disse, enaltecido seja Ele,
``Para que esteja a Torá do Eterno em tua boca''\footnote{Êxodo 13:9.} e as
mulheres não têm a obrigação de estudar a Torá. Esta é uma explicação
dada na Mekhiltá.

Todas as leis sobre estes dois preceitos estão explicadas no quarto
capítulo de Menahot.

\paragraph{Os tsitsit}

Por este preceito nos é ordenado o feitio dos tsitsit. Este
preceito está expresso em Suas palavras, enaltecido seja Ele, ``Façam
para eles tsitsit sobre as bordas de suas vestes, pelas suas gerações;
e porão sob tsitsit da borda, um cordão azul celeste''.\footnote{Números
15:38.}

Este não é contado como dois preceitos, embora seja regra entre nós que
o azul\footnote{O cordão.} não prejudica a validade do
branco, nem o branco invalida
o azul. A razão disso aparece no Sifrei:
``Poder-se-ia pensar que são dois preceitos --- o preceito do azul e o
preceito do branco; por esse motivo, a Torá estabelece: `E será para
vós pôr tsitsit',\footnote{Ibid., 39.} mostrando assim que se trata de um
preceito e não de dois''.

Este preceito não é obrigatório para as mulheres, como está explicado no
início de Kidushin. Todas as leis deste preceito estão explicadas em
Menahot.

\paragraph{A mezuzá}

Por este preceito nos é ordenado a confecção\footnote{Ou seja, foi-nos ordenado que fixemos a mezuzá. Vide Halachot Tefilin Cap. 5, lei 7.} da
mezuzá. Este
preceito está expresso em Suas palavras, enaltecido seja Ele, ``E as
escreverás
nos umbrais de tua casa e em teus portões''.\footnote{Deuteronômio 6:9.} Este
preceito está repetido novamente na Torá.\footnote{Deuteronômio 11:20.}

Todas as suas leis estão explicadas no terceiro capítulo de Menahot.

\paragraph{A reunião do povo no Santuário durante a Festa dos
Tabernáculos\protect\footnote{festa de sucot.}}

Por este preceito somos ordenados a que todas as pessoas se reúnam no
segundo dia dos Tabernáculos, depois do final de cada sétimo ano, e a
que vários versículos de Deuteronômio sejam lidos para elas. Este
preceito está expresso em Suas palavras, enaltecido seja Ele, ``Congrega
o povo, os homens e as mulheres, e as crianças etc.'',\footnote{Deuteronômio
31:12.} as quais são o preceito da Assembleia.

Em Kidushin está dito: ``O cumprimento de todos os preceitos positivos
que estão ligados a uma ocasião não é obrigatório para as mulheres''. O
Talmud levanta a questão: ``Não é a Assembleia um preceito positivo
ligado a uma ocasião e mesmo assim obrigatório para as mulheres?''. A
conclusão a que se chegou ao final da discussão foi que ``Não se deve
argumentar a partir de uma regra geral''.

As normas deste preceito, a saber, como deve ser feita a leitura, quem
deve ler, e que trechos devem ser lidos, estão explicadas no sétimo
capítulo do Tratado Sotá.

\paragraph{Um rei deve transcrever o Rolo da Torá}

Por este preceito somos ordenados a que todo rei de nossa nação que
ocupar o trono real transcreva um Rolo da Torá\footnote{O Sefer Torá (Pentateuco) que lemos na sinagoga, aos sábados.}
para si mesmo, do qual ele não deve se separar. Este preceito está
expresso em Suas palavras, enaltecido seja Ele, ``E quando se sentar
sobre o trono de seu reino, escreverá para si o traslado desta Lei''.\footnote{Deuteronômio 17:18.}

Todas as normas deste preceito estão explicadas no segundo capítulo de
Sanhedrin.

\paragraph{Obter um Rolo da Torá}

Por este preceito somos ordenados a que todo varão entre nós tenha um Rolo da Torá\footnote{Sefer Torá} para si próprio. Se ele o transcrever de próprio punho, isto será muito apreciado e, de preferência, é assim que deve ser feito
pois os Sábios dizem:\footnote{Menahot 30:A.} ``Se ele o transcrever de
próprio punho, será considerado pelas Escrituras como se ele o tivesse recebido do Monte Sinai''. Se ele
próprio não puder fazê-lo, ele é obrigado a comprar um, ou a contratar
alguém que o transcreva por ele. Este preceito está expresso em Suas palavras,
enaltecido seja Ele, ``E agora escrevei para vós este cântico''.\footnote{Deuteronômio 31:19.} Contudo, como não é permitido transcrever alguns
trechos dela, as palavras ``este cântico'' devem necessariamente
significar a Torá completa, que inclui ``este cântico''.

A Guemará de Sanhedrin diz:\footnote{Sanhedrin 21:B.} ``Rabá diz: Ainda que
seus pais lhe tenham deixado um Rolo da Lei, ele deverá transcrever o seu próprio,
como está dito, `E agora, escrevei para vós este cântico'. Abaye
objetou: Será que transcrever um Rolo da Torá em seu próprio nome,
porque ele não deve desejar se apoiar nos seus pais, é apenas uma
obrigação do Rei e não do plebeu? A resposta a isso foi: A regra é
necessária apenas para obrigar o Rei a transcrever dois Rolos, pois nos
foi ensinado que `Ele escreverá para si o traslado desta Lei'\footnote{Deuteronômio 17:18.} significa que ele deve transcrever para si mesmo
duas cópias''. Ou seja, a diferença entre o Rei e um plebeu é que cada
homem deve transcrever um Rolo da Lei, mas o Rei deve transcrever dois,
como está explicado no segundo capítulo de Sanhedrin.

As regras para a transcrição de um Rolo da Torá e as condições
relativas a isto estão explicadas no terceiro capítulo de Menahot, no
início de Baba Batra e em Shabat.

\paragraph{Dar graças após as refeições}

Por este preceito somos ordenados a dar graças ao Eterno, enaltecido
seja Ele, após cada refeição. Ele está expresso em Suas palavras,
enaltecido seja Ele, ``E comerás e te fartarás, e louvarás ao Eterno,
teu Deus''.\footnote{Deuteronômio 8:10.}

A Tosseftá diz: ``Ficamos sabendo que dar graças após uma refeição é um
preceito imposto pela Torá através do versículo `E comerás e te
fartarás, e louvarás ao Eterno, teu Deus'\,''.

As normas deste preceito estão explicadas em vários trechos do Tratado
Berakhot.

\paragraph{A construção do Santuário}

Por este preceito somos ordenados a construir uma Casa para Seu serviço.
Lá deverão ser oferecidos sacrifícios e deverá arder o fogo perpétuo,
para lá serão feitas as peregrinações e lá terão lugar todos os anos as
festas e assembleias. Este preceito está expresso em Suas palavras,
enaltecido seja Ele, ``E Me farão um santuário''.\footnote{Êxodo 25:8.}

O Sifrei diz: ``Os israelitas foram ordenados a cumprir três preceitos
ao entrar na Terra: designar um rei para si próprios, construir o
Santuário, e destruir os descendentes de Amalec''. Fica, dessa forma,
claro que a construção do Santuário constitui um preceito em si.

Já explicamos\footnote{Décimo Segundo Fundamento.} que este preceito global inclui
preceitos individuais e que o Castiçal, a Mesa, o Altar, e as outras coisas são todos partes do Santuário e que tudo isso junto é chamado ``o Santuário'', embora
haja um preceito específico para cada uma das partes.

É verdade que Ele disse, com relação ao Altar: ``um altar de terra você
deverá fazer para Mim'',\footnote{Êxodo 20:21.} de maneira que se poderia pensar
que este é um preceito independente, separado do de construir o
Santuário, mas o significado verdadeiro, neste caso, é o que vou
explicar a vocês. O sentido literal do versículo se refere aos tempos
em que os Altos Lugares nos eram permitidos e nós tínhamos autorização
para fazer um Altar de terra em qualquer lugar, e oferecer sacrifícios
nele; e os Sábios já declararam que o objetivo do versículo era
ordenar-nos a construção de um Altar ligado à terra, que não fosse
móvel, como era no deserto.\footnote{Números 4:14; Êxodo 27:7.} Isto foi dito por eles
na Mekhiltá de Rabi Ishmael, onde o versículo é interpretado assim:
``Quando você entrar na Terra de Israel, você deverá erguer um altar
para Mim ligado à terra''. Sendo assim, o preceito é um dos que são
obrigatórios por todas as gerações, e conclui-se que ele faz parte dos
deveres do Templo, e quer dizer que o altar a ser construído deve ser
de pedra. Ao explicar as palavras ``E se você Me fizer um altar de
pedra'' a Mekhiltá diz: ``Rabi Ishmael diz: `Toda palavra \emph{se} na
Torá implica permissão, exceto em três ocasiões, uma das quais é `E se
você Me fizer um altar de pedra'\,''. Então os Sábios disseram: ``O
versículo `E se você Me fizer um altar de pedra' estabelece uma
obrigação. Você afirma que isto é uma obrigação; e se fosse apenas uma
permissão? A Torá nos diz `Com pedras inteiras edificarás o altar do
Eterno, teu Deus'\,''.\footnote{Deuteronômio 27:6.}

As normas relacionadas à construção do Santuário, seu modelo e suas
divisões, a construção do Altar, e as leis relativas, estão explicadas
num Tratado consagrado especialmente ao assunto, que é o Tratado Midot.
Da mesma forma, o modelo do Castiçal, da Mesa e do Altar de ouro, e suas
posições no Santuário estão explicados na Guemará Menahot e Yoma.

\paragraph{Respeitar o Santuário}

Por este preceito somos ordenados a ter uma atitude de grande e profunda
admiração e de temor para com o Santuário, e a venerá-lo em nossos
corações com receio e temor, pois é esse o respeito ao Santuário que foi
imposto por Suas palavras, enaltecido seja Ele, ``E Meu santuário
temereis''.\footnote{Levítico 19:30.}

A definição desse respeito está na Sifrá: ``O que significa respeito?
Que não se deve entrar no Monte do Templo com seu cajado, ou suas
sandálias, ou sua mochila, ou com poeira em seus pés, ou usar o templo
como passagem; e de forma alguma se deve cuspir lá dentro''. Está
explicado em várias passagens do Talmud que não é permitido a ninguém
ficar sentado no Tribunal, com exceção dos Reis da Dinastia de David.
Tudo isso é decorrente de Suas Palavras, enaltecido seja Ele, ``E Meu
Santuárió temereis'', às quais temos a obrigação de obedecer por todos
os tempos, até mesmo em, nossos dias, quando, em virtude de nossos
muitos pecados, ele foi destruído.

A Sifrá diz: ``De que modo deduzimos que não somente enquanto o
Santuário existia, mas também, depois que ele deixou de existir, o
respeito deve continuar? A Torá diz: `Meus \emph{Shabatot} guardareis,
e Meu Santuário temereis',\footnote{Levítico 19:30 e 26:2.} ou seja, assim como a observação do
Shabat é para sempre, da mesma forma o respeito pelo Santuário é para
sempre''.

No mesmo trecho lemos: ``Seu respeito não deve ser para com o Santuário,
mas sim para com Ele que nos deu as ordens referentes ao Santuário''.

\paragraph{A guarda do Santuário}

Por este preceito somos obrigados a manter a guarda ao Santuário e a
vigiá-lo todas as noites e durante toda a noite, e dessa forma honrá-lo,
exaltá-lo e glorificá-lo. Este preceito está expresso em Suas Palavras,
enaltecido seja Ele, a Aarão ``E tu e teus filhos contigo, estareis
diante da tenda da assinação''\footnote{Números 18:2.} ou seja, você deverá
manter a guarda no Santuário para sempre. Este preceito também é
encontrado sob outra forma: ``E manterão o serviço da guarda da tenda da
assinação''.\footnote{Ibid., 4.}


E está escrito no Sifrei: ```Tu e teus filhos contigo, estareis diante
da tenda do testemunho': os Cohanim dentro, e os Levitas fora''. Ou
seja, eles devem guardar e vigiar o Santuário e ao redor dele,
revezando-se.\footnote{O revezamento deve ser feito entre os membros de cada grupo apenas, e
  não entre os grupos, ou seja, a guarda do interior do Santuário está
  restrita aos Cohanim e a do exterior aos Levitas.}

A Mekhiltá diz: ```E manterão o serviço da guarda da tenda da assinação'
nos dá apenas um preceito positivo. De que maneira sabem que há um
preceito negativo envolvido também? Na Torá está escrito: `E mantereis
o serviço da guarda da santidade'\,''.\footnote{Ibid. 5. Ver o preceito negativo 67.} Fica, assim, claro que a guarda do Santuário constitui um preceito positivo.

No mesmo lugar lemos: ``Engrandece o Santuário que haja guardas nele,
porque um palácio que tem guardas é diferente de um palácio que não os
tem'', e é sabido que palácio (Palturim) é um nome para o Santuário. O
significado disto é que se exalta e se glorifica o Santuário ao se
designarem guardas para vigiá-lo.

Todas as normas deste preceito estão explicadas nos Tratados Tamid
(primeiro capítulo) e Midot.

\paragraph{Os serviços dos Levitas no Santuário}

Por este preceito ordena-se que apenas os Levitas realizem determinados
serviços específicos no Santuário, tal como fechar os portões e cantar
durante a oferta de sacrifícios. Este preceito está expresso em Suas
palavras ``E servirão os Levitas no serviço da tenda de assinação''.\footnote{Números 18:23.}

O Sifrei diz: ``Eu poderia supor que ele pode querer escolher entre
executar o serviço ou não e por isso a Torá diz: `E servirão os Levitas
no serviço', ou seja, eles devem fazê-lo''. É, portanto, um dever que
tem que ser executado querendo ou não.

A natureza do serviço dos Levitas está explicada em diversos trechos em
Tamid e em Midot, e no segundo capítulo de Arakhin está explicado que
apenas os Levitas devem cantar.

Este preceito aparece novamente sob outra forma: ``Servir em nome do
Eterno, seu Deus, como todos os seus irmãos levitas'',\footnote{Deuteronômio
18:7.} a respeito de que o segundo capítulo de Arakhin
diz:\footnote{Arakhin 11:b.} ``O que significa servir em nome do Eterno?
Significa cantar''.

\paragraph{As abluções dos Cohanim}

Por este preceito os Cohanim são ordenados a lavar suas mãos e pés
quando tiverem que entrar na Arca ou quando forem celebrar o ofício.
Este é o preceito da Santificação das Mãos e dos Pés, e está expresso em
Suas palavras, enaltecido seja Ele, ``E lavarão Aarão e seus filhos,
nele, suas mãos e seus pés ao entrarem na tenda da assinação etc.''.\footnote{Êxodo 30:19-20.}

Aquele que violar este preceito positivo fica sujeito à pena de morte
pela mäo dos Céus, ou seja, um Cohen que serve no Santuário sem
lavar suas mãos e seus pés fica sujeito a tal penalidade. Isto baseia-se
em Suas palavras, enaltecido seja Ele, ``Lavar-se-ão com água, e não
morrerão''.\footnote{Ibid., 20.}

As regras detalhadas deste preceito estão expostas na íntegra no segundo
capítulo de Zebahim.

\paragraph{A obrigação dos Cohanim de manter as lamparinas acesas}

Por este preceito os Cohanim são ordenados a manter as lamparinas
continuamente acesas perante o Eterno. Ele está expresso em Suas
Palavras, enaltecido seja Ele, ``A conservará em ordem Aarão e seus
filhos, desde a tarde até a manhã, diante do Eterno'',\footnote{Êxodo 27:21.} e
este é o preceito de manter as lamparinas acesas.

As normas detalhadas deste preceito estão explicadas no oitavo capítulo
de Menahot, no primeiro capítulo de Yoma e em diversos lugares do
Tratado Tamid.

\paragraph{A obrigação dos Cohanim de abençoar os israelitas}

Por este preceito os Cohanim são ordenados a abençoar os israelitas
todos os dias. Ele está expresso em Suas palavras, enaltecido seja Ele,
``Assim abençoareis aos filhos de Israel; dir-lhes-eis: o Eterno te
abençoe e te guarde. Faça resplandecer o Eterno o Seu rosto sobre ti, e
te agracie. Tenha o Eterno misericórdia de ti, e ponha em ti a paz''.\footnote{Números 6:23-26.}

As normas deste preceito estão explicadas no último capítulo de Meguilá
e de Taanit, e no sétimo capítulo do Tratado Sotá.

\paragraph{O pão da proposição}

Por este preceito somos ordenados a colocar o Pão da Proposição diante
d'Ele continuamente. Ele está expresso em Suas palavras, enaltecido
seeja Ele, ``E porás sobre a mesa o pão da proposição diante de mim,
continuamente''.\footnote{Êxodo 25:30.}

Você já sabe que a Torá prescreve que deve ser colocado pão novo a cada Shabat, juntamente com incenso, e que os Cohanim devem
comer do pão que havia sido colocado no Shabat anterior.

As normas deste preceito estão explicadas no décimo primeiro capítulo
de Menahot.

\paragraph{A queima do incenso}

Por este preceito os Cohanim são ordenados a colocar incenso
diariamente, duas vezes por dia, no Altar de Ouro. Ele está expresso em
Suas palavras, enaltecido seja Ele, ``E Aarão fará queimar sobre ele,
incenso de especiarias; pela manhã, quando limpar as lamparinas, o
queimará''.\footnote{Êxodo 30:7.}

As normas deste preceito e o procedimento a ser seguido na queima diária
do incenso estão explicadas no início de Queretot e em várias passagens
de Tamid.

\paragraph{O fogo perpétuo do altar}

Por este preceito somos ordenados a manter o fogo aceso no Altar, todos
os dias, constantemente. Ele está expresso em Suas palavras ``Fogo
contínuo estará aceso sobre o altar'',\footnote{Levítico 6:6.} o que só pode
significar que eles são obrigados a colocar lenha no fogo todos os dias,
sem falta, pela manhã e ao anoitecer, como está explicado no segundo
capítulo de Yoma e no Tratado Tamid.

O Talmud diz claramente: ``Embora o fogo venha dos Céus, é um dever
mantê-lo queimando por meios comuns''.

As normas deste preceito, que é o preceito relativo à preparação diária
do Fogo sobre o Altar estão expostas no quarto capítulo de Yoma, e no
segundo capítulo de Tamid.

\paragraph{Remover as cinzas do altar}

Por este preceito os Cohanim são ordenados a remover as cinzas do
Altar diariamente. Isto é chamado a Retirada das Cinzas, e o preceito
está expresso em Suas palavras, enaltecido seja Ele, ``E vestirá o
Cohen a sua túnica de linho\ldots{} e separará a cinza''.\footnote{Levítico 6:3.}

As normas deste preceito estão explicadas em várias passagens dos
Tratados Tamid e Kipurim.

\paragraph{Retirar os impuros}

Por este preceito somos ordenados a expulsar do Templo as pessoas
impuras. Ele está expresso em Suas palavras, enaltecido seja Ele, ``Que
enviem do acampamento todo o leproso, e todo aquele que padece de fluxo
e todo o impuro por ter tocado os ossos de cadáver de pessoa''.\footnote{Números
5:2.}

A palavra ``acampamento'' aqui significa o Acampamento da Presença
Divina, que, em gerações posteriores, foi o equivalente à Corte do
Santuário, como explicamos no início da Ordem Teharot, em nosso
\emph{Comentário sobre a Mishné}. O Sifrei diz: ```Que enviem do acampamento'
é uma advertência para que pessoas impuras não entrem no Santuário em
estado de impureza''.

Este preceito aparece também sob outra forma: ``Se houver entre vós
um homem que não estiver puro por causa de derramamento de sêmem, de
noite, sairá para fora do acampamento''.\footnote{Deuteronômio 23:11.} O
``acampamento'' aqui deve ser entendido como o Acampamento da Presença
Divina, uma vez que o próprio preceito diz: ``Fora do acampamento os
enviareis''\footnote{Números 5:3.} e que na Guemará de Pessachim se lê: ```Sairá
para fora do acampamento' significa o Acampamento da Presença Divina''.

A Mekhiltá diz: ```Ordena aos filhos de Israel que enviem do
acampamento': este é um preceito positivo. De que maneira concluímos que
também há um preceito negativo envolvido? A Torá diz: `Para que não
contaminem os seus acampamentos'\,''.

O Sifrei diz: ```Sairá para fora do acampamento' é um preceito positivo''.

\paragraph{Honrar os Cohanim}

Por este preceito somos ordenados a exaltar os descendentes de Aarão,
para demonstrar-lhes honra e respeito, e a conferir-lhes alto grau de
santidade e dignidade, mesmo contrariando suas próprias objeções. Tudo
isso é para a glória do Eterno, enaltecido seja Ele, já que Ele os
escolheu para Seu serviço e para as ofertas de Seus sacrifícios. Este
preceito está expresso em Suas palavras ``E santifica-lo-ás, porque o
sacrifício de teu Deus ele oferece, santo será para ti''\footnote{Levítico 21:8.}
que os Sábios interpretam assim: ```E santifica-lo-ás', quer dizer, ele
será o primeiro em todos os assuntos sagrados como, por exemplo, na
leitura da Torá; ele deverá ter a prioridade para ler as Bençãos nas
refeições; e ele deverá ser o primeiro a receber uma porção justa''.

A Sifrá também diz: ```E santifica-lo-ás' --- mesmo contra sua
vontade''; ou seja, este é um preceito estabelecido para nós, e não
depende da vontade do Cohen.

Da mesma forma ele diz: ```Santos serão para seu Deus'\footnote{Ibid., 6.} ---
mesmo contra sua vontade. `E serão santidade'\footnote{Ibid.} --- inclusive os
que tiverem um defeito.\footnote{Ver os preceitos negativos 70 e 71.} Portanto não devemos argumentar:
``Já que este Cohen não está capacitado a oferecer o sacrifício de seu Deus, por que deveríamos dar-lhe prioridade e demonstrar-lhe honra e respeito?''. Porque Ele
disse: ``E serão santidade'', significando toda essa honrada família, incluindo tanto os
perfeitos como os defeituosos.

As cláusulas apropriadas nas quais está estabelecido que devemos
tratá-los desta forma estão explicadas em diversos trechos da Guemará de
Macot, Hulin, Bekhorot, Shabat, e em outros trechos.

\paragraph{As vestes dos Cohanim}


Por este preceito os Cohanim são ordenados a ornar-se com vestes de
especial esplendor e beleza antes de servir no Santuário. Ele está
expresso em Suas palavras, enaltecido seja Ele, ``E farão vestidos de
santidade para Aarão, teu irmão, para o esplendor e para a beleza'';\footnote{Êxodo 28:4.} ``E a
seus filhos farás chegar, e os farás vestir as túnicas''.\footnote{Ibid., 29:8.} Estas
são as Vestes dos Cohanim: oito vestimentas para o Cohen
Gadol e quatro para o
Cohen comum. Caso o Cohen celebre o ofício com menos ou com mais
vestimentas do que o número designado para aquele ofício específico, o
ofício fica invalidado, e ele fica sujeito à morte pela mão dos Céus ---
quero dizer, aquele que celebrar o ofício com menos vestes do que o
número designado. Na Guemará de Sanhedrin ele também está entre os que
podem ser mortos pela mão dos Céus. Isto não está explicitamente dito
nas Escrituras, mas é derivado do versículo ``E lhes cingirás cintos\ldots{}
e será para eles o sacerdócio'',\footnote{Êxodo 29:9.} cuja interpretação é:
``Enquanto usam as vestimentas, eles estão revestidos de seu sacerdócio; quando não usam suas vestimentas, não estão
revestidos de seu sacerdócio'', e se transformam em leigos. Será explicado a
seguir que um leigo que oficia está sujeito à pena de
morte.\footnote{Ver o preceito negativo 74.}

A Sifrá diz: ```E pôs sobre ele o peitoral':\footnote{Levítico 8:8.} esta passagem nos ensina regras que se aplicam a essa ocasião específica e também regras
que se aplicam permanentemente; regras para os ofícios diários, e também
as regras para o culto do Dia do Perdão. Todos os dias\footnote{O Cohen Gadol.}
devia oficiar com trajes dourados, mas no Dia do Perdão com vestes de linho branco''.

Pela seguinte passagem da Sifrá fica claro que a colocação dessas vestes
é um preceito positivo: ``Como concluímos que Aarão não colocou as
vestes do Cohen apenas para seu próprio enaltecimento, e sim como
aquele que obedece à ordem de seu Rei? Pelas palavras da Torá `E fez
como ordenou o Eterno a Moisés'\,''; ou seja, embora essas vestes, com seu
ouro, ônix, jásper e outras pedras preciosas, fossem de beleza
inigualável, não seria o gozo desta beleza que o Cohen deveria tomar
em consideração, mas apenas o cumprimento do preceito que Deus impôs a
Moisés, a saber, que ele deveria sempre usar essas vestes no Santuário.

Todas as normas deste preceito estão explicadas no segundo capítulo do
Zebahim e em vários trechos de Kipurim e de Sucá.

\paragraph{Os Cohanim devem carregar a Arca Sagrada}


Por este preceito somos ordenados a que os Cohanim carreguem
a Arca\footnote{Que continha as duas tábuas com os 10 Mandamentos.} sobre os ombros, quando desejarmos
transportá-la de um lugar para
outro. Esse preceito está expresso em Suas palavras, enaltecido seja
Ele, ``Por que o serviço da santidade estava sobre eles; eles o levavam aos
ombros''.\footnote{Números 7:9} Embora este preceito tenha sido imposto naquela
época aos Levitas, isso foi apenas por causa do número limitado de
Cohanim, já que Aarão
foi o primeiro. Na realidade, o cumprimento deste preceito compete aos
Cohanim, e são eles que devem carregar, como fica claro no livro de
Joshuá e no livro de Samuel.\footnote{Josh. 3:14; II Sam. 15:25.} quando Davi ordenou trazer a
Arca pela segunda vez, o livro das Crônicas registra: ``Assim os Cohanim e os Levitas se santificaram
para levantar a Arca do Eterno, Deus de Israel. E os filhos dos Levitas
carregaram a Arca de Deus com as barras sobre seus ombros, como Moisés
ordenara, de acordo com a palavra do Eterno''.\footnote{I Cron. 15:14-15 e 5:16.}


Da mesma forma, ao se referir à divisão dos Cohanim em 24 Grupos, o
livro de Crônicas diz: ``Essas eram suas posições em seus serviços para
entrar na casa do Eterno, de acordo com as leis dadas a eles pela mão de
Aarão, seu pai, como o Eterno, Deus de Israel, lhe havia ordenado''. Os
Sábios explicam este versículo como significando que é dever dos
Cohanim carregar a Arca nos ombros e que é isso o que o Eterno, Deus
de Israel, ordenou. O Sifrei diz: ```De acordo com as leis dadas a eles\ldots{}
como o Eterno, Deus de Israel, lhe havia ordenado': onde foi que
Ele lhe deu essa ordem? Em `Porém aos filhos de Kehat não deu; porque o
serviço da santidade estava sobre eles; eles o levavam aos ombros'\,''.

Fica assim claro que este é um dos preceitos.

\paragraph{O óleo da unção}

Por este preceito somos ordenados a ter óleo feito para nós de acordo
com uma composição específica, pronto para a Unção de todo Cohen Gadol que venha designado, como Ele diz: ``E o Cohen Gadol, entre seus
irmãos, sobre cuja cabeça for derramado o óleo da unção''.\footnote{Levítico
21:10.} Com esse óleo também se deveriam ungir alguns dos reis, como
está explicado nas normas deste preceito.\footnote{Ver também o preceito negativo 84.}

O Tabernáculo e todos os seus vasos foram ungidos com este óleo,
mas os vasos não serão ungidos com ele no futuro, pois o Sifrei diz
explicitamente: ``Com a unção destes'', --- ou seja, dos vasos do
Tabernáculo --- ``todos os vasos foram santificados para sempre'', como
Ele disse, enaltecido seja Ele: ``Óleo de unção de santidade será este
para Mim por vossas gerações''.\footnote{Exodo 30:31.}

As cláusulas deste preceito estão explicadas no início de Queretot.

\paragraph{Os Cohanim devem oficiar em grupos, revezando-se no serviço}

Por este preceito os Cohanim são ordenados a oficiar em grupos,
sendo que cada grupo deve oficiar durante uma semana, e a não oficiar
todos ao mesmo tempo, exceto durante os Festivais, quando todos os grupos
devem participar igualitariamente, e quando qualquer
um\footnote{Qualquer Cohen.} presente pode oferecer sacrifícios. Aparece nas Crônicas que Davi e Samuel os dividiram em 24 grupos,\footnote{1 Cron. 24:4-18.}
e na Guemará Sucá está explicado que durante os Festivais todos
participavam de forma igual.

O trecho das Escrituras no qual está expresso este preceito é o
seguinte: ``E quando vier o Cohen, o qual descende da tribo de Levi,
de alguma das tuas cidades, de todo o Israel, onde ele habita e vier com
todo o desejo de sua alma ao lugar que escolheu o Eterno; e servir em
nome do Eterno, seu Deus, como todos os seus irmãos Levitas, que servem
ali diante do Eterno, igual porção receberão todos''.\footnote{Deuteronômio
18:6-8.} O Sifrei diz: ```E vier com todo o desejo de sua alma' poderia ser a qualquer momento, por isso nas
Escrituras está dito: `De alguma de tuas cidades', ou seja, quando todo
o povo de Israel estiver reunido em uma cidade durante os Festivais.
Poder-se-ia pensar que todos os grupos participavam igualitariamente das
oferendas nos Festivais, mesmo daquelas que não decorriam
especificamente dos Festivais, por isso a Torá esclarece: `Exceto a
parte dos patrimônios paternos'.\footnote{Deuteronômio 18:8.} O que significa o
patrimônio paterno? `Celebre você durante sua semana, que eu celebrarei
durante minha semana'\,''; ou seja, eles concordaram com o revezamento
dos grupos, e com todo o arranjo do ofício em grupos, com um novo grupo
celebrando a cada semana. O Targum explica o versículo da seguinte
maneira: ``Exceto o grupo daquela semana, pois assim o decretaram os
pais''.

As normas deste preceito estão explicadas no final da Guemará de Sucá.

\paragraph{Os Cohanim devem fazer-se impuros pelos parentes mortos}

Por este preceito os Cohanim são ordenados a fazerem-se impuros por
seus parentes lembrados na Torá.\footnote{Lembrados ou enumerados na Torá. Levítico 21:2-3.} Uma vez que as
Escrituras os proíbem, por respeito, de fazerem-se impuros pelos
mortos, mas lhes permitem fazê-lo por parentes, poder-se-ia pensar que o
Cohen pode escolher entre fazer-se impuro ou não. Ele lhes impôs uma
obrigação positiva ao dizer: ``Por ela se fará impuro''.\footnote{Levítico 21:3.}

A Sifrá diz: ```Por ela se fará impuro' é um preceito positivo. Se ele
não deseja se fazer impuro, deverá fazê-lo contra sua vontade. Isso
aconteceu com o Cohen Yossi, cuja esposa morreu na véspera de
Pessach e ele se recusou a se fazer impuro por ela; os Sábios então
se utilizaram de força e o obrigaram a fazer-se impuro, contra sua
própria vontade''.

Neste preceito está baseado o dever do luto, ou seja, a obrigação de
todo o povo de Israel de ficar de luto pelos parentes, que são em número
de seis.\footnote{Mãe, pai, filho, filha, irmão, irmã; o marido e a mulher só têm essa
  obrigação por ordem posterior contida no Talmud.} É para confirmar essa obrigação que Ele
declarou expressamente que no caso do Cohen, o qual está normalmente proibido de fazer-se impuro, ele deverá fazê-lo a qualquer custo, como todos os outros israelitas, de forma que a lei de luto não seja julgada com leviandade.

Tem sido demonstrado que o luto do primeiro dia está prescrito pela lei
das Escrituras. Na interpretação eles disseram explicitamente na Guemará
de Moed Catan que o luto não deve ser guardado durante um Festival: ``Se
o luto começou antes do Festival, o preceito positivo abrangendo todo o
povo de Israel\footnote{De que os Festivais devem ser cheios de alegria.} se sobrepõe ao preceito que lhe foi
imposto individualmente''. Portanto, está claro que as Escrituras obrigam a guardar o luto, mas apenas no primeiro dia, pois os outros seis dias foram impostos pelos Rabinos; e que
até mesmo um Cohen é obrigado a observar o luto no primeiro dia, e
fazer-se impuro por seus parentes. Compreenda isso.

As regras detalhadas deste preceito estão expostas no Tratado Mashkin, em vários trechos de Berakhot, Quetubot, Yebamot e Abodá Zará, e
na Sifrá, na passagem que começa com ``Fala aos Cohanim''.\footnote{Levítico
21:1.}

A obrigação do Cohen de fazer-se impuro por seus parentes não se
estende às mulheres porque o Cohen, que está proibido de fazer-se
impuro por outros que não sejam seus parentes, tem a obrigação de
fazê-lo por parentes, mas a mulher da família do Cohen não está proibida de
fazer-se impura por qualquer pessoa morta, como vou explicar
oportunamente\footnote{No preceito negativo 166.} e consequentemente não tem o dever nem a obrigação de fazê-lo. Ela deve guardar
o luto, mas quanto a fazer-se impura ou não, depende de sua vontade.
Compreenda isso.

\paragraph{A obrigação do Cohen Gadol de casar-se apenas com uma virgem}

Por este preceito o Cohen Gadol é ordenado a casar-se com uma
virgem. Está expresso em Suas palavras, enaltecido seja Ele, ``E ele,
mulher em sua virgindade, tomará''.\footnote{Levítico 21:13.}

Nas Escrituras\footnote{Quetubot 30:A.} está explicitamente dito: ``Rabi
Akiba afirmava que até mesmo o nascimento de um filho contrário a este
preceito positivo seria considerado um bastardo'' e como exemplo de uma
união meramente contrária a um preceito positivo eles citam o caso de um
Cohen Gadol que tenha um relacionamento com uma mulher que não seja
virgem; porque é um princípio aceito que um preceito negativo derivado
de um preceito positivo tem a força de um preceito positivo. Assim, fica
claro que este é um preceito positivo. E o Talmud diz ainda, mais
adiante:\footnote{Horayot 11:B.} O Cohen Gadol é obrigado a casar-se com
uma virgem''.

As normas deste preceito estão explicadas no sexto capítulo de Yebamot,
em vários trechos de Quetubot e de Kidushin.

\paragraph{O holocausto diário}

Por este preceito somos ordenados a oferecer no Santuário todos os dias
dois cordeiros que são chamados de Oferendas Contínuas. Ele está
expresso em Suas palavras, enaltecido seja Ele, ``Dois para cada dia,
em holocausto contínuo''.\footnote{Números 28:3.}

As normas que regem este preceito, a ordem dos sacrifícios e os métodos
a serem seguidos estão explicados no segundo capítulo de Yoma e no
Tratado Tamid.

\paragraph{A oferta diária do alimento pelo Cohen Gadol}

Por este preceito somos ordenados a que o Cohen Gadol ofereça todos
os dias uma oblação pela manhã e uma ao anoitecer, chamada de Bolo
do Cohen Gadol, conhecida também como oblação do Cohen ungido.
Este preceito está expresso em Suas palavras, enaltecido seja Ele,
``Esta é a oferta de Aarão e seus filhos''.\footnote{Levítico 6:13.}

As normas deste preceito, bem como a hora e a maneira de fazer a
oferenda, estão expostas no sexto e nono capítulos de Menahot e em
vários trechos em Yoma e Tamid.

\paragraph{A oferta adicional do Shabat}

Por este preceito somos ordenados a oferecer um sacrifício todos os
Shabatot, além do holocausto diário. Ele está expresso em Suas palavras,
enaltecido seja Ele, ``E no dia de Shabat, dois cordeiros de um ano de
idade etc.''.\footnote{Números 28:9.}

A ordem dos sacrifícios está explicada no segundo capítulo de Yoma e em
Tamid.

\paragraph{A oferta adicional da lua nova}

Por este preceito somos ordenados a oferecer um sacrifício a cada lua
nova, além do holocausto diário, sendo esta a Oferta Adicional da Lua
Nova. Este preceito está expresso em Suas palavras, enaltecido seja
Ele, ``E nos princípios de vossos meses oferecereis em holocausto ao
Eterno''.\footnote{Números 28:11.}

\paragraph{A oferta adicional de Pessach}

Por este preceito somos ordenados a oferecer um sacrifício em cada um
dos sete dias de Pessach, além do holocausto diário, sendo esta uma
Oferta Adicional do Festival do Azimo. Este preceito está expresso em
Suas palavras, enaltecido seja Ele, ``E oferecereis por sete dias
oferta queimada ao Eterno''.\footnote{Levítico 23:8.}

\paragraph{A oblação da nova cevada}

Por este preceito somos ordenados a fazer a oblação da cevada no
sexagésimo dia de Nissan, juntamente com o holocausto de um carneiro de
um ano, sem defeito. Este preceito está expresso em Suas palavras,
enaltecido seja Ele, ``Trareis ao Cohen um `omer' das primícias de
vossa ceifa''.\footnote{Levítico 23:10.}

Esta oblação é chamada ``A Oferta das Primícias'' e ela está mencionada
em Suas palavras, enaltecido seja Ele, ``E se ofereceres oblação de
primícias''.\footnote{Levítico 2:14.} A Mekhiltá diz: ``Todo `se' na Torá
implica uma opção, exceto em três casos, em que é usado com relação a
uma obrigação; um deles é este: `E se ofereceres oblação de primícias'.
Você está certo de que isto é uma obrigação? Talvez seja apenas uma
permissão. E dizem: `Assim oferecerás a oblação de tuas primícias etc.',\footnote{Ibid.} mostrando que isso é uma obrigação e não uma permissão''.

Todas as normas deste preceito estão explicadas na íntegra no décimo
capítulo de Menahot.


\paragraph{A oferta adicional de \textit{Shabuot}}

Por este preceito somos ordenados a oferecer uma Oferta Adicional também
no quinquagésimo dia após a oferta do ``Omer'', que é no sexagésimo dia
de Nissan. Esta é a Oferta Adicional da Festa das Semanas mencionada no
livro de Números. Este preceito está expresso em Suas palavras ``E no
dia das primícias, quando oferecerdes oblação nova ao Eterno\ldots{} E
oferecereis holocausto, para ser aceito com agrado pelo Eterno''.\footnote{Números 28:26-27.}

\paragraph{Levar dois pães em ``Shabuot''}

Por este preceito somos ordenados, como está prescrito, a levar ao
Santuário dois pães na Festa das Semanas, juntamente com os sacrifícios
obrigatórios da Oferenda do Pão, e a oferecer sacrifícios, como
prescrito no livro de Levítico; e depois que esses pães forem movidos,
os Cohanim devem comê-los juntamente com os cordeiros das Ofertas de
Paz. Este preceito está expresso em Suas palavras, enaltecido seja Ele,
``De vossas habitações trareis dois pães para serem movidos, de duas
décimas partes de uma `efa'\,''.\footnote{Levítico 23:17.}

Está explicado no quarto capítulo de Menahot que este sacrifício, que
era um complemento da Oferta do Pão, é uma oferta a parte e diferente da
Oferta Adicional do dia. Já fornecemos explicação suficiente a este
respeito em nosso comentário no Tratado Menahot.

Todas as normas deste preceito estão explicadas em Menahot nos capítulos
quatro, cinco, oito e onze.

\paragraph{A oferta adicional do Ano Novo}

Por este preceito somos ordenados a oferecer uma Oferenda Adicional no
primeiro dia de Tishrei, que é a Oferta Adicional do Ano Novo. Este
preceito está expresso em Suas palavras, enaltecido seja Ele, ``E no
sétimo mês, no primeiro dia do mês\ldots{} e oferecereis como holocausto,
para ser aceito com agrado pelo Eterno''.\footnote{Números 29:1-2.}

\paragraph{A oferta adicional do décimo dia de Tishrei}

Por este preceito somos ordenados a oferecer uma Oferenda Adicional no
décimo dia de Tishrei. Ele está expresso em Suas palavras,
enaltecido seja Ele, ``E no décimo dia, deste sétimo mês\ldots{} E
oferecereis holocausto ao Eterno, para ser aceito com agrado''.\footnote{Números
29:7-8.}

\paragraph{O ofício de ``Yom Kipur''}

Por este preceito somos ordenados a celebrar o Ofício do Dia, ou seja,
todos os sacrifícios e profissões de fé ordenados pelas Escrituras para
o Dia do Perdão, para expiar todos os nossos pecados. Esta é a instrução
que está expressa na porção ``Aharé Mot''.\footnote{Levítico 16:1-34.}

A prova de que ela constitui, em sua totalidade, apenas um preceito se
encontra no final do quinto capítulo de Kipurim: ``Com relação a cada
celebração de `Yom Kipur', mencionada na ordem prescrita, se algum
ofício for celebrado fora da ordem estabelecida é como se nenhum deles
tivesse sido celebrado''.

Todas as normas deste preceito estão explicadas no Tratado dedicado
exclusivamente a este assunto, que é o Tratado de Yoma.

\paragraph{A oferta adicional da Festa dos Tabernáculos}

Por este preceito somos ordenados a oferecer uma Oferenda Adicional no
Festival dos Tabernáculos. Ele está expresso em Suas palavras,
enaltecido seja Ele, ``E oferecereis por holocausto, oferta queimada
para ser aceita com agrado pelo Eterno''.\footnote{Números 29:13.} Esta é a
Oferta Adicional dos Tabernáculos.

\paragraph{A oferta adicional de ``Shemini Atzeret''}

Por este preceito somos ordenados a oferecer uma Oferenda Adicional no
oitavo dia da Festa dos Tabernáculos, que é a Oferta Adicional do oitavo
Dia da Assembleia solene.\footnote{Números 29:35-38.}

O que nos faz considerar esta Oferta Adicional como sendo uma oferta em
separado, diferente das oferecidas diariamente durante o Festival dos
Tabernáculos, é o princípio aceito que o oitavo Dia da Assembleia Solene
é, por si só, um outro Festival. Os Sábios dizem claramente: ``Ele é um
Festival separado, com ofertas separadas''. Isto prova que sua oferenda
é específica, deixando assim o assunto perfeitamente claro.

\paragraph{As três peregrinações anuais}

Por este preceito somos ordenados subir ao Santuário três vezes por ano.
Ele está expresso em Suas palavras, enaltecido seja Ele, ``Três vezes
celebrarás, para mim, festas no ano''.\footnote{Êxodo 23:14.} As Escrituras deixam
claro que ``subir'' significa ir até lá com uma oferenda.\footnote{Deuteronômio 16:16.}
Este preceito está repetido várias vezes,\footnote{Ibid., 15; Êxodo 34:23.} e o
Sifrei diz: ``Três preceitos devem ser seguidos num Festival, a saber: festejar, comparecer diante do Eterno, e alegrar-se''. A Guemará
de Haguigá também diz:\footnote{Haguigá 6:B.} ``Três preceitos são
impostos a Israel num Festival: festejar, comparecer diante do Eterno e alegrar-se''. ``Festejar'' significa levar uma Oferta de Paz, o que não é obrigatório para as
mulheres.

As normas deste preceito estão explicadas no Tratado Haguigá.


\paragraph{Comparecer diante do Eterno durante os Festivais}

Por este preceito somos ordenados a comparecer\footnote{Ao Santuário Sagrado.}
durante os Festivais. Está expresso em Suas palavras, enaltecido seja
Ele, ``Três vezes ao ano, aparecerão todos os teus homens diante do
Eterno, teu Deus''.\footnote{Deuteronômio 16:16.} O significado deste preceito é
que todo homem deve subir ao Santuário com todos os seus filhos homens
que possam andar sozinhos e oferecer um holocausto quando subir. Este é
chamado o Holocausto do Comparecimento. Nós já nos referimos às palavras
dos Sábios: ``Três preceitos são obrigatórios durante um festival:
festejar, comparecer e alegrar-se''.

As normas deste preceito, ou seja, o Preceito do Comparecimento, também
estão expostas no Tratado Haguigá. Este preceito também não é
obrigatório para as mulheres.

\paragraph{Alegrar-se nos Festivais}

Por este preceito somos ordenados a alegrar-nos nos Festivais. Está
expresso em Suas palavras, enaltecido seja Ele, ``E alegrar-te-ás na tua
festa''.\footnote{Deuteronômio 16:14.} Este é o terceiro dos três preceitos
observados num festival.

A obrigação mais importante imposta por este preceito é a das Ofertas
de Paz obrigatórias. Essas são outras Ofertas de Paz do Festival
adicionais às ofertas de Haguigá e são chamadas no Talmud de ``Ofertas
de Paz da Alegria''.

Com relação a essas Ofertas de Paz nos é dito o seguinte: ``As mulheres
têm a obrigação de tomar parte na alegria''. Como está nas Escrituras:
``E sacrificarás ofertas de pazes e comerás ali; e te alegrarás diante
do Eterno, teu Deus''.\footnote{Deuteronômio 27:7.} As normas deste preceito
também estão expostas no Tratado Haguigá.

As palavras ``E alegrar-te-ás na tua festa'' incluem o preceito dos
Sábios de que devemos alegrar-nos de todas as maneiras possíveis, como
comendo carne nos festivais, bebendo vinho, vestindo roupas novas,
distribuindo frutas e doces às crianças e mulheres, e alegrando-nos com
instrumentos musicais e dançando no Santuário especificamente, sendo que
essa será a Alegria de ``Beit Hashoeba''. Todos esses tipos de regozijo
estão compreendidos em Suas palavras ``E alegrar-te-ás na tua festa''.

Dentre as maneiras de alegrar-se, a mais obrigatória é a de beber vinho, porque ela está especialmente ligada à alegria.

Como diz a Guemará:\footnote{Pessachim 109:A.} ``Um homem tem o dever de fazer
com que seus filhos e sua família se alegrem durante um Festival\ldots{} De que
maneira? Com vinho''.

Diz ainda, mais adiante: ``Foi-nos ensinado: Rabi Yehudá ben Betera diz
que quando o Santuário existia não podia haver outro tipo de regozijo a
não ser o de comer carne, como está escrito: `E sacrificarás ofertas de
pazes\ldots{}'. Mas agora, que o Santuário não mais existe, só há regozijo com
vinho, como está dito: `O vinho faz alegre o coração do homem'\,''.\footnote{Salmos 104:15.} Diz também:
``Os homens\footnote{Os homens devem alegrar-se de maneira apropriada a eles.} de maneira apropriada a eles, e as
mulheres de maneira apropriada a elas''.

A Torá nos obriga a incluir nesse regozijo os pobres, os necessitados
e os estranhos. Suas palavras, enaltecido seja Ele, são: ``E
alegrar-te-ás diante do Eterno, teu Deus, tu\ldots{} e o peregrino, e o
órfão, e a viúva''.\footnote{Deuteronômio 16:11.}

\paragraph{Abater a oferta de Pessach}

Por este preceito somos ordenados a sacrificar o cordeiro de Pessach
no décimo quarto dia de Nissan. Está expresso em Suas palavras,
enaltecido seja Ele, ``E o degolará toda a assembleia da congregação de
Israel, pela tarde''.\footnote{Êxodo 12:6.} Aquele que não cumprir este preceito
e deliberadamente negligenciar a oferenda deste sacrifício no momento
determinado está sujeito à extinção, seja homem ou mulher, uma vez que
está claramente expresso na Guemará de Pessachim que a primeira Oferta
de Pessach é obrigatória para as mulheres da mesma forma que para
todo homem do povo de Israel, e que essa oferta tem prioridade sobre o
Shabat, ou seja, que ela deve ser oferecida no décimo quarto dia de
Nissan, mesmo que esse dia seja Shabat.

A pena de extinção está prescrita em Suas palavras ``E o homem que está
puro, e não estiver em viagem, e deixar de fazer o Pessach, essa alma
será banida de seu povo''.\footnote{Números 9:13.}

Na enumeração dos preceitos --- negativos --- cuja transgressão incorre
na penalidade de extinção, no início do Tratado Queretot, estão
incluídos os preceitos positivos de Pessach e da circuncisão, como
foi mencionado na introdução.

As regras detalhadas deste preceito estão expostas no Tratado Pessachim.

\paragraph{Comer a oferta de Pessach}

Por este preceito somos ordenados a comer a Oferta de Pessach na
décima quinta noite de Nissan, de acordo com as condições especificadas,
ou seja, ela deve ser grelhada, deve ser comida numa casa, e deve ser
comida com pão ázimo e ervas amargas. Ele está expresso em Suas
palavras, enaltecido seja Ele, ``E comerão a carne nesta noite, grelhada
no fogo, e pães ázimos, com ervas amargas comerão''.\footnote{Êxodo 12:8.}

Se alguém perguntar: ``Por que você conta comer a Oferta de Pessach, o
pão ázimo e as ervas amargas como um único preceito, e não como três,
uma vez que comer pão ázimo é um preceito, comer ervas amargas é outro
e comer a carne da Oferta de Pessach é outro?'', eu direi que é verdade
que comer pão ázimo é um preceito por si só, como explicarei
posteriormente;\footnote{Ver o preceito positivo 158.} da mesma forma, comer a carne da
Oferta de Pessach também é por si só um preceito, como já foi
mencionado. Mas comer ervas amargas está colocado como dependente de
comer a Oferta de Pessach, e não deve ser contado como um preceito
separado. Isto está provado pelo fato de que se deve comer a carne da
Oferta de Pessach em cumprimento ao preceito, quer haja ervas amargas
disponíveis ou não, mas não se comem ervas amargas
a não ser com a carne da Oferta de Pessach, porque Ele diz,
enaltecido seja Ele, ``Com pães ázimos e ervas amargas comerá o
sacrifício''.\footnote{Números 9:11.} Ao comer ervas amargas sem carne não se
cumpre nenhuma obrigação, e não se pode dizer que se cumpre o preceito
de comer ervas amargas. Nas palavras de Mekhiltá: ```Grelhada no fogo, e
pães ázimos, com ervas amargas comerão'; isso nos ensina que o preceito
referente à Oferta de Pessach estipula que ela deve ser comida
grelhada, com pão ázimo e ervas amargas'', ou seja, o preceito consiste
na totalidade destes três.

Também está escrito: ``De que modo você conclui que na ausência do pão
ázimo e das ervas amargas a obrigação pode ser cumprida comendo-se
apenas a Oferta de Pessach? Pelas palavras: `A comerão'\,'', isto é, a
carne apenas. Também se poderia pensar que, da mesma forma que na
ausência de pão ázimo e de ervas amargas, a obrigação pode ser cumprida
comendo-se a Oferta de Pessach, assim também na falta da Oferta de
Pessach a obrigação pode ser cumprida comendo-se pão ázimo e ervas
amargas, argumentando-se que uma vez que comer a Oferta de Pessach é
um preceito positivo, e comer pão ázimo e ervas amargas também é um
preceito positivo, se na ausência de pão ázimo e ervas amargas a
obrigação pode ser cumprida comendo-se apenas a Oferta de Pessach,
então deveria concluir-se que, na falta da Oferta de Pessach, a
obrigação pode ser cumprida comendo-se pão ázimo e ervas amargas. As
Escrituras dizem, a esse respeito: ``A comerão''.\footnote{Ou seja, ``comerão'' a Oferta de Pessach.}

Está também escrito: ```A comerão': destas palavras deve-se concluir que
a Oferta de Pessach deve ser comida até a saciedade, enquanto que o pão
ázimo e as ervas amargas devem ser comidos antes que o estado de
saciedade seja atingido'', uma vez que a essência do preceito consiste
em comer a carne, como Ele disse: ``Comerão a carne nesta noite'',
enquanto que comer as ervas amargas é uma obrigação complementar a comer
a carne, como estas citações deixam claro para os que as compreendem.

Uma prova evidente da exatidão de nosso parecer está numa afirmação do
Talmud: ``Comer ervas amargas hoje em dia é apenas uma regra
estabelecida pelos Rabinos'' porque no que diz respeito à Torá, não há
obrigatoriedade de comê-las sozinhas, mas elas devem ser comidas com a
carne da Oferta de Pessach. Isto constitui prova clara e evidente de
que elas são acessórias ao preceito e não constituem um preceito
individual.

As normas deste preceito também estão expostas no Tratado Pessachim.

\paragraph{Abater a segunda oferta de Pessach}

Por este preceito aquele que não tenha podido oferecer a primeira Oferta
de Pessach é ordenado a abater a Segunda Oferta de Pessach. Este
preceito está expresso em Suas palavras, enaltecido seja Ele, ``No
segundo mês, aos 14 dias do mês, pela tarde, a celebrará''.\footnote{Números
9:11.}

Aqui novamente se poderia objetar e perguntar: ``Por que você conta a segunda Oferta de Pessach,\footnote{Como um preceito diferente.} transgredindo dessa forma a regra que você estabeleceu no Sétimo Fundamento de que uma lei pertencente a um
preceito não é por si só considerada como um preceito separado? A pessoa que
assim argumentar deve saber que os Sábios sustentaram diferentes
opiniões sobre a questão da Segunda Oferta de Pessach ser considerada
como uma continuação da primeira ou como um preceito independente e a
conclusão a que se chegou foi que essa obrigação é um preceito
diferente e que, consequentemente, deve ser enumerado em separado.

A Guemará de Pessachim diz:\footnote{Pessachim 93:A.} ``Na opinião de Rabi,
fica-se sujeito à extinção\footnote{Punição à pessoa que não abater a oferta de Pessach.} por causa da primeira e fica-se sujeito
à extinção por causa da segunda. Mas Rabi Nathan diz: fica-se sujeito à
extinção por causa da primeira,
mas não pela segunda; e Rabi Hananyá ben Akabya diz: Não se fica sujeito
à extinção nem por causa da primeira, a menos que se deixe de levar a
segunda''.

A Guemará passa a perguntar: ``Em que eles diferem?''. E responde:
``Rabi sustenta que o segundo é um Festival em separado, e Rabi Nathan
sustenta que ele é apenas complementar ao primeiro''. Isso explica nossa
afirmação.\footnote{De que os Sábios diferem quanto à classificação da Segunda Oferta de
  Pessach.}

A Guemará diz ainda, no mesmo trecho: ``De acordo com isso, se alguém
deliberadamente negligenciar ambas'', ou seja, se alguém deliberadamente
deixar de levar tanto a Primeira Oferta de Pessach quanto a Segunda,
``todos concordam que ele é culpado. Se ele negligenciar as duas
involuntariamente, todos concordam que ele não é culpado. Se ele
negligenciar a primeira intencionalmente e a segunda involuntariamente,
ele é culpado e está sujeito à extinção, de acordo com o Rabi e com Rabi
Nathan, é inocente e não está sujeito à punição, de acordo com Rabi
Hananyá ben Akabya. Da mesma forma, se seu erro for intencional no caso
da primeira, e ele trouxer a oferenda na segunda, ele é culpado, de
acordo com o Rabi'', porque na sua opinião o segundo Pessach não é
meramente complementar ao primeiro, e a lei segue, em todos esses casos,
o parecer do Rabi.

Este preceito não é obrigatório para as mulheres porque está explicado\footnote{No Tratado Pessachim.} que a Segunda\footnote{A Segunda Oferta de Pessach.} é opcional para as mulheres.

As normas deste preceito estão explicadas na Guemará de Pessachim.

\paragraph{Comer a segunda oferta de Pessach}

Por este preceito somos ordenados a comer a carne da segunda Oferta de
Pessach na noite do décimo quinto dia de Iyar juntamente com pão
ázimo e ervas amargas. Ele está expresso em Suas palavras, enaltecido
seja Ele, referentes a esta também: ``A comerão com pão ázimo e ervas
amargas''.\footnote{Números 9:11.}

As normas detalhadas deste preceito também estão expostas em Pessachim.

É óbvio que este preceito não é obrigatório para as mulheres, já que
elas não são obrigadas a fazer o abatimento, como explicamos, portanto
não resta dúvida de que elas não são obrigadas a comer esta oferenda.

\paragraph{Tocar as cornetas no Santuário}

Por este preceito somos ordenados a fazer soar as cornetas no Santuário
ao oferecer qualquer um dos sacrifícios sazonais, e ele está expresso em
Suas palavras, enaltecido seja Ele, ``E também no dia de vossa alegria,
nas vossas solenidades fixas, e nos princípios de vossos meses,
tocareis as cornetas sobre vossos holocaustos etc.''.\footnote{Números 10:10.}
Os Sábios dizem explicitamente que este é o preceito das cornetas.

As normas deste preceito estão explicadas no Sifrei, em Rosh Hashaná e
Taaniot, uma vez que somos ordenados a tocar as cornetas em épocas de
dificuldades e infortúnios, quando clamamos pelo Eterno, enaltecido
seja Ele, de acordo com Suas palavras ``E quando estiverdes em guerra
em vossa terra, contra o adversário, que vos oprime, tocareis retinindo
o `Shofar'\,''.\footnote{Números 10:9.}

\paragraph{Oferecer gado com idade mínima determinada}

Por este preceito somos ordenados a que todo gado que trouxermos como
oferenda tenha oito dias ou mais de idade, e não menos. Este é o
preceito da oferenda cujo momento de ser aceita ainda não chegou por
motivos físicos e ele está expresso em Suas palavras, enaltecido seja
Ele, ``Ficarão por sete dias atrás de sua mãe''.\footnote{Levítico 22:27.}

Este preceito também nos é dado de uma outra forma: ``Sete dias estará
com sua mãe''.\footnote{Êxodo 22:29.} Isto se aplica a todas as oferendas de
todos os tipos, privadas e públicas.

Das palavras ``E do oitavo dia em diante serão aceitos por sacrifício,
como oferta queimada ao Eterno''\footnote{Levítico 22:27.} concluímos que antes
disso eles não seriam aceitáveis. Assim, está claramente proibido
oferecer um animal que não tenha atingido idade para ser aceito; mas,
como este é um preceito negativo derivado de um preceito positivo, sua
transgressão não acarreta a pena de flagelo, e aquele que trouxer um que
não tenha alcançado a idade certa não será açoitado, como está explicado
no capítulo ``Ele e seus filhotes'', onde também se lê: ``Desconsidere
a oferenda cuja época ainda não tenha chegado, pois a Escritura a
justapôs a um preceito positivo''.

As normas deste preceito estão explicadas na Sifrá, no final do Tratado
Zebahim.

\paragraph{Oferecer apenas sacrifícios perfeitos}

Por este preceito somos ordenados a oferecer ao Eterno apenas espécimes
perfeitos, sem os defeitos mencionados nas Escrituras, e livres de todas
as imperfeições consideradas como defeitos pela Tradição. Este preceito
está expresso em Suas palavras, enaltecido seja Ele, ``Estes deverão ser
sem defeito, para que sejam aceitos'',\footnote{Levítico 22:21.} sobre as quais a
Sifrá diz: ```Estes deverão ser sem defeitos para que sejam aceitos':
este é um preceito positivo''.\footnote{Ver também preceitos negativos 91 a 96.}

As palavras ``Ser-vos-ão eles sem defeito, igualmente as suas libações''\footnote{Números 28:31.} foram citadas como sendo a prova de que os vinhos das
libações e seus óleos e farinha fina devem ser absolutamente perfeitos
e sem qualquer tipo de defeito.

As normas deste preceito estão explicadas no oitavo capítulo de Menahot.

\paragraph{Levar sal com cada sacrifício}

Por este preceito somos ordenados a oferecer sal com cada sacrifício.
Ele está expresso em Suas palavras, enaltecido seja Ele, ``E toda tua
oferta de oblação temperarás com sal''.\footnote{Levítico
2:13. Ver também o preceito negativo 99.}

As normas deste preceito estão explicadas na Sifrá e em Menahot.

\paragraph{O Holocausto}

Por este preceito somos ordenados quanto ao procedimento a seguir ao
oferecermos o Holocausto. Ou seja, todo Holocausto, seja ele uma oferta
privada ou pública, deve ser oferecido de uma maneira preestabelecida.
Este preceito está expresso em Suas palavras no Levítico, ``Quando algum
de vós oferecer sacrifício ao Eterno\ldots{} Se seu sacrifício for holocausto
de gado'' etc.\footnote{Levítico 1:2-3.}

\paragraph{O Sacrifício de Pecado}

Por este preceito somos ordenados a oferecer o Sacrifício de Pecado,
seja ele de que tipo for, da maneira especificada. Este preceito está
expresso em Suas palavras, enaltecido seja Ele, ``Esta é a lei do
sacrifício de pecado'' etc.\footnote{Levítico 6:18.}

No Levítico está explicado também de que forma o sacrifício deve ser
oferecido, que parte dele deve ser queimada e que parte deve ser comida.

\paragraph{O Sacrifício de Delito}

Por este preceito somos ordenados a oferecer o Sacrifício de Delito de
uma determinada maneira. Este preceito está expresso em Suas palavras,
enaltecido seja Ele, no Levítico: ``E esta é a lei do sacrifício de
delito'' etc.\footnote{Levítico 7:1.}

As Escrituras explicam como este sacrifício deve ser oferecido, que
parte dele deve ser queimada e que parte deve ser comida.

\paragraph{O Sacrifício de Paz}

Por este preceito somos ordenados a oferecer o Sacrifício de Paz da
forma especificada. Este preceito está expresso em Suas palavras ``E se
sacrifício de pazes é sua oferta'' etc.\footnote{Levítico 3:1.} e ``É esta a lei
do sacrifício de pazes\ldots{} Se por ação de graças a oferecer'' etc.\footnote{Levítico 7:11-12.}

Esses quatro rituais --- o Holocausto, o Sacrifício de Pecado, o
Sacrifício de Delito e o Sacrifício de Paz --- compõem todo o ritual
dos sacrifícios, uma vez que todas as ofertas de animais, sejam elas
trazidas por um indivíduo ou pela congregação, pertencem a uma dessas
quatro categorias, embora o sacrifício de Delito seja sempre uma oferta
individual, como explicamos em diversas ocasiões.

O Tratado Zebahim contém as normas destes quatro preceitos e assuntos
relacionados a eles, e explica quais são as cerimônias obrigatórias, o
que não pode ser feito sem infringir a lei,\footnote{E se a lei for infringida, isso implica punição.} o que
invalida um sacrifício e qual é o procedimento correto.

\paragraph{A Oblação}

Por este preceito somos ordenados a oferecer a Oblação de acordo com as
regras estipuladas para cada um de seus vários tipos. Este preceito está
expresso em Suas palavras ``E quando uma alma oferecer uma oblação ao
Eterno'' etc.;\footnote{Levítico 2:1.} ``E se tua oferta for oblação feita na
assadeira'' etc.;\footnote{Ibid., 5.} ``E se oblação de panela é tua oferta,''
etc.,\footnote{Ibid., 7.} que são complementadas pelas seguintes palavras,
encontradas mais adiante: ``E esta é a lei da oblação''.\footnote{Ibid., 6:7.}

As normas deste preceito, com seus vários aspectos, estão explicadas no
Tratado Menahot, que se dedica especificamente a este assunto.

\paragraph{O sacrifício de um tribunal que cometeu um erro}

Por este preceito o Tribunal é ordenado a oferecer um sacrifício, caso
ele tenha tomado uma decisão errada.\footnote{Que tenha levado a congregação a cometer um pecado.}

Este preceito está expresso em Suas palavras, enaltecido seja Ele, ``E
se a congregação de Israel pecar por ignorância, e for oculto o caso aos
olhos da assembleia'' etc.\footnote{Levítico 4:13.}

As normas e condições deste preceito estão todas explicadas no Tratado
Horayot e em vários trechos do Tratado Zebahim.

\paragraph{O Sacrifício Estabelecido de Pecado}

Por este preceito o indivíduo que
tenha cometido involuntariamente um dos pecados graves
conhecidos\footnote{Ver Mishné Torá Hilchot Shegagot, 1º capítulo, 4ª Halachá.} é ordenado a oferecer um Sacrifício de
Pecado. Ele está expresso em Suas palavras, enaltecido seja Ele, ``E se
uma alma do povo da terra pecar por erro'' etc.\footnote{Levítico 4:27.}

Esta oferta é um Sacrifício Estabelecido de Pecado, ou seja, um Sacrifício de Pecado que deve ser constituído de um animal.

Já explicamos que os pecados passíveis da penalidade de Sacrifício de
Pecado são os mesmos que acarretam a penalidade de extinção, caso sejam
cometidos deliberadamente, desde que o pecado implique a violação de um
preceito negativo e que haja algum ato relacionado a ele, como explicado
no início do Tratado Queretot.

As normas deste preceito estão explicadas nos Tratados Horayot e
Queretot, e em vários trechos de Shabat, Shabuot e Zebahim.

\paragraph{O Sacrifício Suspensivo de Delito}

Por este preceito somos ordenados a oferecer um sacrifício determinado
em caso de dúvida quanto a um dos pecados capitais que implicam a
penalidade de extinção, se cometidos voluntariamente, e em Sacrifício
estabelecido de Pecado, se cometidos involuntariamente. Este sacrifício
é chamado
de Sacrifício Suspensivo de Delito. Um exemplo de um caso de dúvida que
implicaria um Sacrifício Suspensivo de Delito é o seguinte: suponha que
uma
pessoa tenha diante de si dois pedaços de gordura, uma de rins\footnote{A gordura dos rins não é ``Casher''.}
e a outra de coração.\footnote{A gordura do coração é ``Casher''.} Ele come um dos dois pedaços
e o outro é comido por outra pessoa, ou perdido. Uma dúvida surge então
em sua mente quanto a se o pedaço de gordura que ele comeu foi o
permitido ou o proibido. Nesse caso, ele deve oferecer um sacrifício
expiatório, pela dúvida que surgiu, chamado Sacrifício Suspensivo de
Delito. Se posteriormente ele se certificar de que o pedaço de gordura
que comeu era o de rins, fica confirmado que pecou involuntariamente, e
ele deverá então oferecer um Sacrifício Estabelecido de Pecado.

O versículo referente a esta oferenda está no livro de Levítico: ``E se
alguma alma pecar e fizer um dos preceitos do Eterno, daqueles que não
se devem fazer, e não souber, e for culpado, levará a sua iniquidade. E
trará do rebanho, um carneiro sem defeito, no valor de dois siclos, por
sacrifício de delito, ao Cohen; e expiará por ele, o Cohen, pelo
erro que cometeu sem saber'',\footnote{Levítico 5:17-18.} ou seja, porque ele
não sabia se tinha pecado ou não. Isto é o que os Sábios chamam de
Pecado Cometido Involuntariamente.

As normas deste preceito estão explicadas no Tratado Queretot.

\paragraph{O Sacrifício Incondicional de Delito}

Por este preceito aquele que cometer certas transgressões é ordenado a
oferecer um Sacrifício de Delito para obter o perdão. Essa oferenda é
chamada de Sacrifício Incondicional de Delito.

As transgressões que requerem este sacrifício são: sacrilégio, roubo,
manter uma ligação com uma escrava prometida em casamento e jurar em
falso no caso de algo entregue sob custódia. Aquele que cometer um
sacrilégio sem intenção, ou seja, que usufruir de uma ``perutá'' com
algo que pertença ao Santuário, seja dos Objetos Consagrados do tesouro
do Santuário ou dos do Altar; aquele que roubar uma ``perutá'' ou mais
de um amigo e fizer juramento falso a respeito; e aquele que mantiver
uma ligação com uma escrava comprometida, seja sem intenção ou
voluntariamente, todos eles terão a obrigação de trazer uma oferenda por seus pecados --- não um Sacrifício de Pecado, mas
sim um Sacrifício de Delito chamado Sacrifício Incondicional de Delito.

Com relação ao sacrilégio, Suas palavras, enaltecido seja Ele, são: ``E
pecar por erro nas santidades do Eterno, trará por seu delito'' etc.\footnote{Levítico 5:15.}

E disse:\footnote{Com relação ao roubo e falso testemunho em caso de algo entregue sob
  custódia.} ``E negar ao seu companheiro a coisa que
lhe foi entregue sob custódia\ldots{} e jurou em falso\ldots{} e como
oferta de delito trará ao Eterno, do rebanho, um carneiro sem defeito''
etc.\footnote{Ibid., 21-22, 25.} E disse:\footnote{Com relação a manter uma ligação com uma escrava noiva.} ``E o homem que
se deitar com uma mulher\ldots{} e ela for escrava desposada com um
homem\ldots{} e ele trará sua oferta de delito ao Eterno'' etc.\footnote{Ibid.,
19:20-21.}

As normas detalhadas deste preceito estão explicadas no Tratado
Queretot.

\paragraph{O Sacrifício de Maior Valor ou de Menor Valor}

Por este preceito somos ordenados a oferecer um Sacrifício de Maior
Valor ou de Menor Valor por determinadas transgressões.

As transgressões que requerem este sacrifício são: impurificar o
Santuário ou seus Objetos Santificados, perjurar,\footnote{Ou seja, alguém perjura que não pode testemunhar quando na realidade
  pode.} e
jurar em falso em relação a um testemunho. Se alguém, que tiver se
tornado impuro por alguma das fontes primárias de impureza, como explicamos na
introdução à Ordem de Teharot, entrar sem querer no Santuário, o que
implicaria impurificação do Santuário; se alguém sem querer comer carne
sagrada, o que implicaria impurificação dos Objetos Santificados do
Santuário; se alguém pronunciar um juramento e involuntariamente deixar
de cumpri-lo; se alguém, jurar em falso em relação a um testemunho, seja
sem querer ou intencionalmente, em todos esses casos ele terá que
oferecer o que chamamos de um Sacrifício de Maior ou Menor Valor.

Este preceito está expresso em Suas palavras, enaltecido seja Ele, ``E
quando alguma alma pecar, sob juramento\ldots{}; se alguma alma tocar em
alguma coisa impura\ldots{} e lhe for oculto que estava impuro e o souber
depois, e se tornar culpado; ou quando alguma alma jurar, pronunciando
com os lábios\ldots{} e lhe for oculto e o souber depois que foi culpado de
uma dessas coisas \ldots{} trará como sacrifício\ldots{} E se as suas posses não
lhe permitirem trazer\ldots{}'' etc.\footnote{Levítico 5:1-11.}

Este sacrifício é chamado um Sacrifício de Maior ou Menor Valor porque
ele não está especificado, variando de acordo com as posses do
transgressor que tem que oferecê-lo.

As normas deste preceito também estão explicadas nos Tratados Queretot
e Shabuot.

\paragraph{Confessar}

Por este preceito somos ordenados a fazer a confissão oral dos pecados
que tivermos cometido contra o Eterno, enaltecido seja Ele, depois de
nos termos arrependido deles. Esta é a maneira de fazer a confissão:
``Oh, Deus, eu pequei, eu cometi injustiça, eu infringi\ldots{}''. Deve-se elaborar a
confissão e pedir perdão com toda a eloquência de que se for capaz.

Deve-se notar que mesmo no caso de pecados pelos quais se deve oferecer
um dos sacrifícios anteriormente especificados, a fim de obter-se o
perdão prometido pelo Eterno, é preciso confessar-se no momento da
oferta. Isso se depreende de Suas palavras, enaltecido seja Ele, ``Fala
aos filhos de Israel: Quando homem ou mulher fizer algum dos pecados do
homem\ldots{} E confessará os seus pecados que cometerá''.\footnote{Números 5:6-7.}
Comentando esse versículo, a Mekhiltá diz: ``Uma vez que está escrito:
`Confessará aquilo em que pecou',\footnote{Levítico 5:5.} deduzimos que ele deve
confessar `junto com o pecado' que cometeu, ou seja, junto ao sacrifício
de pecado enquanto o mesmo ainda estiver vivo, e não depois de abatido.
O versículo não nos diz que o indivíduo deve confessar-se por qualquer
pecado a não ser pelo de entrar no Santuário em estado de impureza''.
Isso é assim porque o versículo ``Confessará aquilo em que pecou''
aparece no trecho ``Vayikra'' das Escrituras que se refere à
impurificação do Santuário e dos Objetos Santificados, bem como às
transgressões mencionadas nele, como explicamos. É por isso que a
Mekhiltá diz que desse versículo se pode deduzir a obrigação de
confessar-se apenas no caso de se ter impurificado o Santuário. ``De que
modo você conclui sua aplicação em caso de violação de qualquer um dos
outros preceitos? Pelas palavras das Escrituras: `Fala aos filhos de
Israel\ldots{} E confessará'. De que modo você conclui que a obrigação se
aplica aos casos de pena de extinção ou morte? Pelas palavras `Seus
pecados', ou seja, todos os seus pecados, estendendo a aplicação aos
preceitos negativos. `Que cometerá' abrange as transgressões dos
preceitos positivos.

A Mekhiltá diz, no mesmo trecho: ```Algum dos pecados do homem': isto
significa transgressão dos preceitos relacionados com outros homens,
como roubo, assalto e maledicência. `Por falsear em nome do Eterno':
isso inclui juramento em falso e blasfêmia. `Será culpada aquela alma': isso estende a obrigação de confessar-se a todo aquele que estiver
sujeito à pena de morte.\footnote{Por sentença judicial.} Poder-se-ia pensar que
isto se aplica mesmo àqueles que estão para ser condenados à morte em
virtude de um falso testemunho;\footnote{Ver preceito positivo 180.} as Escrituras
dizem: `Será culpada aquela alma', ou seja, a confissão não é obrigatória para
aquele que sabe que não cometeu nenhum delito e contra o qual foi prestado falso
testemunho''.

Assim, fica claro que todos os tipos de pecado, sejam eles grandes ou
pequenos, inclusive os casos de transgressões dos preceitos positivos,
acarretam a obrigação de confessar-se.

Como o preceito ``Confessará'' só está mencionado com relação à
obrigação de oferecer um sacrifício, poderia ocorrer-nos a ideia de que
a confissão não é uma obrigação em si, mas apenas algo acessório ao
sacrifício. Por isso\footnote{Os Sábios.} foram obrigados a explicar
isso na Mekhiltá da seguinte forma: ``Poder-se-ia pensar que a
confissão é necessária quando se oferece um sacrifício; de
que maneira sabemos que ela é necessária mesmo quando não há uma
oferenda? Pelas palavras das Escrituras: `Fala aos filhos de Israel: \ldots{} e confessará'. Poder-se-ia ainda pensar que a confissão só é
obrigatória na Terra de Israel; de que modo se conclui que ela também é
obrigatória na Diáspora? Pelo versículo `E confessarão a sua iniquidade, e a iniquidade de seus pais'.\footnote{Levítico 26:40. Esse versículo se refere à época em que o povo está disperso.} Daniel disse: `A ti, ó Eterno, pertence a justiça'\,''.\footnote{Daniel, 9:7. Daniel viveu fora de Israel.}

Fica assim claro, por tudo o que dissemos, que a confissão é por si
só uma obrigação independente e é obrigatória ao transgressor por cada
pecado que ele cometer, seja na Terra de Israel ou fora dela, quer ele
tenha oferecido um sacrifício ou não. Em todos os casos ele fica
obrigado a confessar, de acordo com Suas palavras, enaltecido seja Ele,
``E confessará os seus pecados''.

A Sifrá diz também: ```E manifestará':\footnote{Levítico 16:21.} isso significa confissão oral''.

As normas deste preceito estão explicadas no último capítulo de Kipurim.

\paragraph{A oferenda levada por um zav\starr}

Por este preceito um zav\starr{} é ordenado a levar, ao
se curar, um sacrifício de ``duas rolas ou dois pombinhos, um por holocausto e o
outro por sacrifício de pecado''. Este é o sacrifício do ``zab'', cujo perdão não
é alcançado até que ele o leve. Este preceito está expresso em Suas
palavras, enaltecido seja Ele, ``E quando estiver limpo de seu fluxo
quem o tiver\ldots{} E no oitavo dia tomará para si duas rolas'' etc.\footnote{Levítico 15:13-14.}

\paragraph{A oferenda dada por uma zava\starr}

Por este preceito uma zava\starr{} é ordenada a levar,
ao se curar, uma oferenda de ``duas rolas ou dois pombinhos''. Este é o
sacrifício da ``zaba'', cujo perdão não é alcançado até que ela o leve.

Talvez se pudesse objetar o seguinte: ``Uma vez que o sacrifício de um
`zab' é o mesmo que o de uma `zaba' e que você leva em consideração
apenas o tipo de oferenda envolvido e não se preocupa com o tipo de
transgressor --- como acontece no caso do Sacrifício de Pecado, do
Sacrifício Incondicional de Delito, do Sacrifício Suspensivo de Delito
e do Sacrifício de Maior ou Menor Valor, onde você enumera cada um deles
como um preceito separado, sem se preocupar sobre os tipos de
transgressões pelas quais o sacrifício em questão pode ser requerido ---
por que você não adota o mesmo método neste caso e não deixa de lado o
fato de que são indivíduos diferentes, uma vez que deles se exige o
mesmo tipo de sacrifício?''.

Quem usar essa argumentação deve saber que a oferenda do homem ou da
mulher que padecem de fluxo não é levada por causa do pecado, mas é
obrigatória em certas circunstâncias. Se a natureza do fluxo fosse
idêntica no homem e na mulher, como idêntico é o nome, um sendo chamado
de ``zab'' e a outra de ``zaba'', então seria apropriado contar-se os
dois preceitos como um só. Mas este não é realmente o caso, porque o que
faz com que o homem deva levar uma oferenda é a saída do sêmem, enquanto
que se a mulher tivesse algum tipo de emissão de sêmem, ela não seria
uma ``zaba''; o que obriga a mulher a levar uma oferenda é um fluxo de sangue, mas um fluxo de sangue
no homem não o obriga ao sacrifício. A palavra ``fluxo'' significa
apenas ``fluir'' mas o que flui não é necessariamente a mesma coisa. Os
Sábios dizem explicitamente: ``Um homem transmite a impureza através do
sêmem, e a mulher através do sangue''.

A lei do ``zab'' e da ``zaba'' não é igual à lei do leproso e da
leprosa. Isso está claramente provado pelo que está dito em Queretot:
``Há quatro pessoas cujo perdão fica pendente: o homem e a mulher com
fluxo, a mulher depois do parto, e o leproso''. Você verifica que o
``zab'' e a ``zaba'' são contados como dois, porque o fluxo do homem é
diferente do fluxo da mulher, enquanto que o leproso e a leprosa não
são contados separadamente.

O versículo referente à oferenda da ``zaba'' é: ``E se limpar-se de seu
fluxo\ldots{} no oitavo dia, tomará para si duas rolas, ou dois pombinhos''
etc.\footnote{Levítico 15:28-29.}

\paragraph{O sacrifício depois do parto}

Por este preceito a mulher que tiver dado à luz a uma criança é
ordenada a levar uma oferenda, a saber, um cordeiro de menos de um ano
de idade como Holocausto e uma pombinha ou uma rola como Sacrifício de
Pecado. Se ela for pobre, ela pode levar duas rolas ou dois pombinhos:
um como Holocausto e o outro como Sacrifício de Pecado. Ela faz parte
daqueles cujo perdão só é alcançado depois de ter levado o sacrifício,
de acordo com o que Ele disse, enaltecido seja Ele: ``E ao cumprirem-se
os dias de sua purificação, pelo filho ou pela filha, trará um cordeiro
de idade de um ano, por holocausto, e um pombinho e uma rola por
sacrifício de pecado\ldots{}''.\footnote{Levítico 12:6.}

\paragraph{O sacrifício levado por um leproso}

Por este preceito um leproso, ao se curar de uma lepra, é ordenado a
levar uma oferenda de três animais --- um Holocausto, um Sacrifício de
Pecado e um Sacrifício de Delito --- e um log\starr{}
de óleo. Se ele for pobre, ele pode levar um cordeiro como Sacrifício de
Delito e duas rolas ou dois pombinhos, um como Holocausto e o outro como
Sacrifício de Pecado. Ele é a quarta pessoa cujo perdão fica pendente
até que tenha levado seus sacrifícios. Este preceito está expresso em
Suas palavras, enaltecido seja Ele, ``E no oitavo dia tomará dois
cordeiros sem defeito e uma ovelha da idade de um ano, sem defeito''.\footnote{Levítico 14:10.}

Pode-se perguntar: ``Por que você não conta como um único preceito a
obrigação imposta a todos aqueles cujo perdão fica pendente, uma vez que
a pendência do perdão é comum a todos eles? Se você fizesse isso, esse
seria um dos meios de purificar-se e você poderia dizer: `Preceito tal é
o que obriga certas pessoas impuras, a saber, um \emph{zab}, uma
\emph{zaba}, uma mulher depois do parto e um leproso a levar um
sacrifício para que sua purificação possa ser considerada completa'.
Assim como você conta a obrigação de ser purificado por um `mikvá' como
um único preceito, sem se preocupar com quem é o impuro ou de que tipo é
a sua impureza, você poderia, da mesma forma, ter contado o sacrifício daqueles cujo perdão fica pendente como um único
preceito, sem preocupar-se com o tipo de impureza''.

O Eterno sabe, e é minha testemunha, que isso seria perfeitamente válido
se o sacrifício obrigatório àqueles cujo perdão fica pendente fosse o
mesmo em todos os casos, e nunca fosse alterado, como no caso de ser
purificado pela água, que é um tipo de purificação obrigatória igual
para todas as pessoas impuras. Mas devido à diversidade de seus
sacrifícios somos forçados, como vocês veem, a contar cada sacrifício
separadamente, porque o que completa a purificação num caso não é o
mesmo que em outro. Este caso\footnote{O caso das quatro pessoas cujo perdão fica pendente.} é igual ao da água
de aspersão, ao da água do ``mikvá'' e ao das quatro coisas pelas quais
o leproso é purificado. Esses são três preceitos separados, embora eles
todos se refiram à purificação de pessoas impuras, como explicarei
depois.

As normas detalhadas referentes às quatro pessoas cujo perdão fica
pendente e às suas oferendas estão explicadas global e detalhadamente no
primeiro e no segundo capítulos de Queretot, nos segundos capítulos de
Arakhin e Nezikin, no oitavo capítulo de Nazir, no final de Negaim, no
Tratado Kinim e em diversos trechos do Talmud, mas quase todas elas
estão nos trechos referidos.

\paragraph{O Dízimo do Gado}

Por este preceito somos ordenados a separar o dízimo dos nossos animais
puros nascidos todo ano, oferecer sua banha e seu sangue e comer o resto
em Jerusalém. Este preceito está expresso em Suas palavras, enaltecido
seja Ele, ``E todo o dízimo do gado e do rebanho, todo o que passar
debaixo da vara marcadora, o décimo será santidade ao Eterno''.\footnote{Levítico 27:32.} Esse é o Dízimo do Gado.

As normas deste preceito estão explicadas no último capítulo de Bekhorot. Lá também está explicado que este preceito é obrigatório também
fora da Terra de Israel e depois da destruição do Templo. Essa é a lei
estabelecida na Torá. Mas para que ele não fosse comido,\footnote{Fora de Jerusalém.} mesmo
que sem defeito ---
uma vez que não temos Santuário ---, os Sábios ordenaram que este
preceito fosse obrigatório apenas quando o Templo estivesse de pé, e que quando o
Templo fosse reconstruído ele seria obrigatório tanto na Terra quanto
fora dela.

\paragraph{Santificar o primogênito}

Por este preceito somos ordenados a santificar os primogênitos, ou seja,
a separá-los e deixá-los de lado para aquilo que deverá ser feito com
eles no momento exato. Ele está expresso em Suas palavras, enaltecido
seja Ele, ``Consagra para Mim todo primogênito\ldots{} no homem e no
animal''.\footnote{Êxodo 13:2.} A Torá explica que ``animal'' aqui significa
apenas o gado, os carneiros e os jumentos. Este preceito está repetido
com relação aos primogênitos dos animais puros em Suas palavras,
enaltecido seja Ele, ``Todo primogênito que nascer do teu gado'' etc.\footnote{Deuteronômio 15:19.}

Esta lei do primogênito de animais puros estipula que ele deve ser dado
aos Cohanim, que deverão oferecer sua banha e seu sangue e comer
o que restar de sua carne. As regras detalhadas deste preceito estão
largamente explicadas no Tratado Bekhorot.

No final do Tratado Halá está explicado que este preceito é obrigatório
apenas na Terra de Israel. O Sifrei diz: ``Poder-se-ia pensar que é
obrigatório levar seus primogênitos nascidos fora de Israel até a Terra
de Israel, por isso as Escrituras dizem: `E o comerás diante do Eterno,
teu Deus,\ldots{} o dízimo de teu grão\ldots{} e os primogênitos do teu gado e do 
teu rebanho' ou seja, os primogênitos devem ser levados do mesmo lugar
de onde vem o dízimo do grão, e não do exterior, pois o grão não é
levado do exterior''.

Fica, portanto, claro que este preceito só é obrigatório na Terra de
Israel. Contudo, um primogênito nascido no exterior, embora não precise
ser levado como sacrifício, deve ser comido somente se for defeituoso,
quer esteja o Templo erguido ou como está agora. Este preceito não é
obrigatório para os Levitas.

\paragraph{Resgatar o primogênito}

Por este preceito somos ordenados a resgatar nossos filhos primogênitos
e dar o dinheiro do resgate ao Cohen. Ele está expresso em Suas
palavras, enaltecido seja Ele, ``O primogênito de teus filhos dar-me-ás''.\footnote{Êxodo 22:28.}

A maneira de ``dar'' o primogênito é explicada da seguinte forma: o
primogênito é resgatado do Cohen, que o tem por direito e é tirado
dele por cinco selaim.\starr{} O preceito está expresso em Suas palavras, enaltecido seja Ele, ``Porém
resgatarás os primogênitos do homem'',\footnote{Números 18:15.} sendo este o
preceito do Resgate do filho primogênito.

Esta obrigação não se aplica às mulheres; é um dever do pai em relação
ao seu filho, como está explicado em Kidushin. Todas as normas deste
preceito estão explicadas em Bekhorot.

\paragraph{Resgatar o primogênito de um jumento}

Por este preceito somos ordenados a resgatar o primogênito de um jumento
apenas com um cordeiro --- se não o resgatarmos pelo seu valor
real\footnote{A pessoa pode resgatar o jumento com dinheiro e dá-lo ao Cohen.} e a dar o cordeiro ao Cohen. Este
preceito está expresso em Suas palavras,
enaltecido seja Ele, ``E todo que abrir a matriz da jumenta, remi-lo-ás
por carneiro''.\footnote{Êxodo 34:20 e 13:13.}

As normas deste preceito estão explicadas no Tratado Bekhorot. Este
preceito também não se aplica aos Levitas.

\paragraph{Quebrar a cerviz do primogênito de um jumento}

Por este preceito somos ordenados a quebrar a cerviz do primogênito de
uma jumenta, caso não se deseje resgatá-lo. Este preceito está expresso
em Suas palavras, enaltecido seja Ele, ``E se não o remires,
quebrar-lhe-ás a cerviz''.\footnote{Êxodo 13:13 e 34:20.} As regras detalhadas
deste preceito também estão explicadas no Tratado Bekhorot.

Poderia ser-me feita a seguinte pergunta: ``Por que você conta resgatar
e quebrar sua cerviz como dois preceitos, e não como um, considerando a
quebra da cerviz como uma das regras detalhadas do preceito, de acordo
com o Sétimo Fundamento?''.

O Eterno sabe, e é minha testemunha de que quem formulasse essa pergunta
teria razão se não fosse pela prova de que eles são dois preceitos
separados, encontrada nas seguintes palavras:\footnote{Bekhorot 13:A.} ``O
dever de redimir vem antes do dever de quebrar a cerviz, e o dever de
casamento levirato\footnote{Ver Preceitos Positivos 216 e 217.} vem antes do dever do
Halitzá''. Ou seja, da mesma forma que a viúva
do irmão sem filhos tem direito ao casamento levirato ou ao Halitzá,
sendo que o casamento levirato é um preceito, como mencionado, e o
Halitzá um outro, assim também o primogênito de uma jumenta pode ser
resgatado ou ter sua cerviz quebrada, sendo cada um deles um preceito
diferente, como foi exposto.

\paragraph{Levar os sacrifícios devidos durante o primeiro Festival}

Por este preceito somos ordenados a executar todos os deveres a nós
impostos na chegada do primeiro dos três Festivais de forma que com a
passagem de qualquer um dos três Festivais cada um de nós tenha levado
todos os
sacrifícios devidos. Este preceito está expresso em Suas palavras,
enaltecido seja
Ele, ``E lá ireis. E levareis ali vossos holocaustos''.\footnote{Deuteronômio
12:5-6.} O significado deste preceito é que quando fordes lá durante
cada um dos três Festivais, tereis a obrigação de levar todos os
sacrifícios que vos tenham sido impostos.

A esse respeito diz o Sifrei: ```E lá ireis. E levareis ali'; por que
foi dito isso? Para estabelecer a obrigação de levar os sacrifícios no
início do primeiro Festival''. Também diz, no mesmo trecho: ``Não se
transgride, `não demorarás em pagá-lo'\footnote{Deuteronômio 23:22.} até que não
tenham terminado os três Festivais --- os Festivais do ano todo''. Ou seja, se os três
Festivais tiverem passado e não se tiver levado seu sacrifício, ter-se-á violado um
preceito negativo,\footnote{Ver o preceito negativo 155.} mas se apenas um Festival tiver passado,
ter-se-á violado apenas um preceito positivo.

Na Guemará de Rosh Hashaná lemos: ``Rabi Meir diz: Assim que um festival
terminar, ele terá transgredido o preceito `Não demorarás em pagá-lo'\,''.
O Talmud pergunta: ``Por que motivo Rabi Meir diz isso?''. E a resposta
é: ``Porque está escrito `E lá ireis. E levareis ali', o que significa
que você deve levá-los quando for lá''. Mas os Sábios dizem que esse
versículo constitui apenas um preceito positivo.

Assim, ficou claro que as palavras ``E levareis ali'' são um preceito
positivo, significando que se devem cumprir todas as suas obrigações
para com o Eterno e que isso deve ser feito em cada Festival, quer sejam elas
quaisquer tipos de sacrifícios, ou que sejam donativos,\footnote{Ver o preceito positivo 114.} valores, coisas consagradas\footnote{Ver o preceito positivo 145.} coisas dedicadas ao
Santuário,\footnote{Ver o preceito positivo 121.} respigaduras,\footnote{Ver o preceito positivo 121.}
feixes esquecidos\footnote{Ver o preceito positivo 122.} e ``peá''.\footnote{Ver o preceito positivo 120.}
O cumprimento de todos esses tipos de obrigações no primeiro Festival
que houver é um preceito positivo, como está explicado na Guemará de
Rosh Hashaná.

\paragraph{Levar todas as ofertas apenas ao Santuário}

Por este preceito somos ordenados a oferecer todos os sacrifícios apenas
no Santuário. Ele está expresso em Suas palavras, enaltecido seja Ele,
``Ali oferecerás os teus holocaustos, e ali farás tudo que Eu te
ordeno''.\footnote{Deuteronômio 12:14.} Procurando alguma citação que deixasse
clara a proibição de levar qualquer tipo de sacrifício a outro
lugar,\footnote{Os Sábios.} encontraram Suas palavras, enaltecido seja
Ele, ``Guarda-te de ofereceres teus holocaustos em todo o lugar que
vires''.\footnote{Deuteronômio 12:13.}

O Sifrei diz: ``Eu sei disso apenas com relação ao Holocausto. De que
forma fico sabendo que é assim para com todos os outros sacrifícios?
Pelas palavras das Escrituras `E ali farás tudo o que eu te ordeno'.
Aqui novamente eu poderia dizer que apenas no caso do Holocausto há
ambos, um preceito positivo e um preceito negativo; como sei que o mesmo se aplica a todos os outros
sacrifícios? Pelas palavras das Escrituras `Ali farás tudo'\,''. Eu
voltarei a este assunto mais tarde, nas proibições.\footnote{Ver os preceitos negativos 89 e 90.}

O fato de dizer que no de um Holocausto estão envolvidos um preceito
positivo e um negativo significa que aquele que o levar a outro lugar
estará violando ambos: um preceito positivo e um preceito negativo,
sendo o preceito negativo ``Guarda-te de ofereceres teus holocaustos'' e
o positivo ``Ali oferecerás'', porque ele não os terá levado ``ali''. Os outros
sacrifícios poderiam abranger apenas o preceito positivo ``E ali farás tudo o que eu te
ordeno'', mas está explicado ali\footnote{No Sifrei.} que também no caso deles se
viola um preceito negativo, além do positivo. E está explicado no final do Tratado Zebahim que todos
os sacrifícios levados a outro lugar envolvem um preceito positivo e um
negativo, e a penalidade de extinção. Assim, por
tudo o que expliquei, fica claro que as palavras ``Ali farás tudo o que
Eu te ordeno'' são sem dúvida alguma um preceito positivo.

\paragraph{Levar ao Santuário, desde fora da Terra de Israel, todos os sacrifícios devidos}

Por este preceito somos ordenados a levar ao Santuário tudo o que
tenhamos a obrigação de oferecer --- quer seja um Sacrifício de Pecado,
um Holocausto, um Sacrifício de Delito ou um Sacrifício de Paz --- ainda
que o motivo responsável pelo sacrifício esteja fora da Terra de
Israel, ou seja, embora a obrigação tenha sido ocasionada fora da Terra
de Israel, somos ordenados a levar os sacrifícios ao Santuário, e temos
o dever de fazê-lo, não importa a que distância. Este preceito está
expresso em Suas palavras, enaltecido seja Ele, ``De certo, tuas coisas
sagradas, e tuas ofertá de votos, tomarás e levarás ao lugar que o
Eterno escolher'',\footnote{Deuteronômio 12:26.} sobre as quais o Sifrei diz:
```Tuas coisas sagradas' se refere apenas aos sacrifícios fora da Terra
de Israel. `Tomarás e levarás' nos ensina que devemos nos preocupar com
o transporte dos sacrifícios até o Santuário''. Está explicado ali que
isso se aplica apenas nos casos de Sacrifício de Pecado, de Sacrifício
de Delito, de Holocausto e de Sacrifício de Paz que a pessoa tem a
obrigação de oferecer.

\paragraph{Redimir Oferendas Defeituosas}

Por este preceito somos ordenados a redimir qualquer oferenda que tiver
se tornado defeituosa, liberando-a assim para uso normal e
permitindo-nos abatê-la e comê-la. Este preceito está expresso em Suas
palavras, enaltecido seja Ele, ``Todavia, com todo o desejo de tua alma,
poderás degolar e comer carne, em todas as tuas cidades'',\footnote{Deuteronômio
12:15.} a respeito das quais o Sifrei diz: ```Todavia, com todo o desejo
de tua alma, poderás degolar e comer carne, em todas as tuas cidades':
isto se refere apenas a oferendas defeituosas que tiverem sido
redimidas''.

As regras detalhadas do preceito de redenção de oferendas estão
explicadas no Tratado Bekhorot e Temurá, e em vários trechos de Hulin,
Arakhin e Meilá.

\paragraph{A santidade de uma oferenda substituída}

Por este preceito somos ordenados a considerar como sagrado um animal
que for substituído por outro. Ele está expresso em Suas palavras,
enaltecido seja Ele, ``Tanto o que for trocado como aquele pelo qual
trocou, serão santidade''.\footnote{Levítico 27:10.} Está expressamente
declarado no início do Tratado Temurá que Suas palavras, enaltecido
seja Ele, ``E não trocará''\footnote{Ibid., 33.} constituem um preceito negativo
justaposto a um preceito positivo. Os Sábios perguntam: ``Não é o caso
da substituição um exemplo de um preceito negativo justaposto a um
preceito positivo?``. No mesmo texto aparece um argumento suplementar
que sujeita aquele que fizer a substituição a ser açoitado, embora esse
seja um preceito negativo justaposto a um preceito positivo: ``Um
preceito positivo não pode se sobrepor a dois preceitos negativos''. Ou
seja, a proibição de fazer a substituição está expressa duas vezes, uma
em ``Não o mudará''\footnote{Ibid., 17:10. Ver preceito negativo 106.} outra em ``E
não o trocará'',\footnote{Ibid.} enquanto há apenas um preceito positivo que é
``Tanto o que for trocado como aquele pelo qual trocou, serão
santidade''.\footnote{Ibid., 27:10.} Dessa forma, fica ratificado o que
desejávamos provar.

As normas deste preceito, ou seja, o que valida ou invalida a
substituição, quais são suas regras, e como ela deve ser oferecida,
estão explicadas no Tratado Temurá.

\paragraph{Os Cohanim devem comer os resíduos das oblações}

Por este preceito os Cohanim são ordenados a comer os resíduos das
Oblações. Ele está expresso em Suas palavras, enaltecido seja Ele, ``E o
que ficar dela, comerão Aarão e seus filhos; comer-se-á sem fermento''.\footnote{Levítico 6:9.} A Sifrá diz, a esse respeito: ```Comer-se-á' é um
preceito positivo, da mesma forma que `O irmão de seu marido estará com
ela e a tomará por mulher'\footnote{Deuteronômio 25:5.} também é um preceito
positivo. Ou seja, comer os resíduos das Oblações e casar-se com a viúva
de um irmão que morreu sem deixar filhos são dois preceitos positivos e
não meras opções.

As normas destes preceitos estão explicadas no Tratado Menahot. A Torá
determina que este preceito se aplique apenas aos homens, de acordo com
Suas palavras, enaltecido seja Ele, ``Todo varão dos filhos de Aarão o
comerá''.\footnote{Levítico 6:11.}

\paragraph{Os Cohanim devem comer a carne dos Sacrifícios Consagrados}

Por este preceito os Cohanim são ordenados a comer a carne dos
Sacrifícios Consagrados, a saber, o Sacrifício de Pecado e o Sacrifício
de Delito, que estão entre os Sacrifícios Mais Sagrados. Este preceito
está expresso em Suas palavras, enaltecido seja Ele, ``E comerão das
coisas com que for feita a expiação''.\footnote{Êxodo 29:33.}

A Sifrá diz: ``De que maneira saber que o fato de que comam os
Sacrifícios Consagrados concede o perdão para toda Israel? Pelas
palavras da Torá `E o Eterno vo-lo deu para levardes a iniquidade da
congregação, a fim de perdoar por eles, diante do Eterno!'.\footnote{Levítico
10:17.} De que forma? O Cohen o come, e Israel recebe o perdão.''

Uma das condições deste preceito é que só se deve comê-los durante um
dia e uma noite até a meia-noite. Depois disso, fica proibido comer
deles; ele é uma obrigação apenas durante o espaço de tempo
estabelecido.

Fica claro que também este preceito se aplica apenas aos elementos
masculinos das famílias dos Cohanim, e não às mulheres, pois as
mulheres não podem comer dos Sacrifícios Mais Sagrados a que se refere
este preceito. Os outros sacrifícios --- a saber, os Sacrifícios Menos
Sagrados --- podem ser comidos no espaço de dois dias e uma noite,
exceto a Ação de Graças e o carneiro dos Nazirim os quais, embora sejam
Sacrifícios Menos Sagrados, devem ser comidos em um dia e uma noite até
a meia-noite. Outrossim, as mulheres podem comer desses Sacrifícios
Menos Sagrados.

O fato de comê-los\footnote{O fato dos Cohanim comerem os Sacrifícios Menos Sagrados.} também é parte deste preceito,
bem como o de comer a Oferta de Elevação. Contudo, comer os Sacrifícios
Menos Sagrados e os Sacrifícios de Elevação não é como comer a carne dos
Sacrifícios de Pecado e de Delito, pois o ato de comer a carne dos
Sacrifícios de Pecado e de Delito completa o perdão de quem ofereceu os
sacrifícios, como explicamos, e o ato de comer é ordenado explicitamente
no caso deles mas não no caso dos Sacrifícios Menos Sagrados e dos Sacrifícios de Elevação.
Consequentemente, isso constitui apenas uma parte do preceito que se
aplica aos outros sacrifícios, e ao comê-los ele executa um preceito. O
Sifrei diz: ```O serviço de vosso sacerdócio dei-o como dádiva a vós'\footnote{Números 18:7.} faz com que o ato de comer as Coisas Consagradas dentro
da Terra de Israel seja como o serviço no Santuário: assim como ele
tinha que lavar suas mãos antes de iniciar o serviço no Santuário, ele
também tinha que lavar as mãos antes de comer as Coisas Sagradas fora de
Jerusalém''.

As normas deste preceito estão explicadas em diversos trechos de Zebahim.

\paragraph{Queimar Sacrifícios Consagrados que se tornaram impuros}

Por este preceito somos ordenados a queimar os Sacrifícios Consagrados
que se tenham tornado impuros. Ele está expresso em Suas palavras,
enaltecido seja Ele, ``E a carne Sagrada do sacrifício de pazes que
tocar em tudo o que for impuro, não será comida, no fogo será queimada''.\footnote{Levítico 7:19.}

A Guemará de Shabat discute a questão de por que é proibido usar óleo de
Oferta de Elevação que tenha se tornado impuro para iluminação, em dia
de Festival. Lê-se ali: ``Descansar é um preceito, de forma que
descansar num dia de Festival baseia-se num preceito positivo e num negativo, e um
preceito positivo\footnote{I.e., o preceito de queimar todos os sacrifícios que tenham se tornado impuros.} não pode se sobrepor a ambos, um
preceito negativo e um positivo.\footnote{Ver os preceitos negativos 323 a 329.}

O significado deste trecho é que é proibido trabalhar num dia de
Festival e aquele que o fizer estará infringindo um preceito positivo,
tendo anulado o preceito de que o dia de Festival ``será para vós
descanso solene''.\footnote{Levítico 23:24.} Ele também estará infringindo um
preceito negativo, porque estará fazendo algo que lhe foi proibido, de
acordo com as palavras ``Nenhuma obra será feita neles'',\footnote{Êxodo 12:16.}
ou seja, nos dias de Festival. Como a queima de Sacrifícios Consagrados
que se tornaram impuros é apenas um preceito positivo, não é permitido
queimá-los num dia de Festival, de acordo com o princípio já mencionado
de que um preceito positivo não pode se sobrepor a ambos um preceito
negativo e um positivo. Também está dito: ``Da mesma forma que é
obrigatório queimar Sacrifícios Consagrados que se tornaram impuros,
assim também é obrigatório queimar o óleo da Oferta de Elevação que
tiver se tornado impuro''.

As normas deste preceito estão explicadas em Pessachim e no final de Temurá.

\paragraph{Queimar as sobras dos Sacrifícios Consagrados}

Por este preceito somos ordenados a queimar as
sobras.\footnote{Dos Sacrifícios Consagrados.} Ele está expresso em Suas palavras,
enaltecido seja Ele, ``E o que ficar da carne do sacrifício, no terceiro dia, no fogo será queimado''.\footnote{Levítico 7:17.} Com
relação a Suas palavras, enaltecido seja Ele, relativas ao cordeiro de
Pessach, ``E não fareis sobrar nada dele até a manhã; e a sobra
dele, pela manhã a queimareis no fogo'',\footnote{Êxodo 12:10.} a Mekhiltá diz:
``O objetivo das Escrituras é estipular um preceito positivo para o
preceito negativo''. Em vários trechos de Pessachim e Macot, além de
outros, está explicitamente declarado que o preceito negativo relativo
às sobras está justaposto a um preceito positivo, e dessa forma não há
punição por açoitamento. O preceito positivo a que nos referimos está
expresso nas palavras já mencionadas, ou seja, ``E o que ficar da
carne\ldots{} no fogo queimado''.

A lei de Recusa e a das Sobras são semelhantes, como explicarei
nos preceitos negativos,\footnote{Ver os preceitos negativos 131 e 132.} uma vez que a palavra notar\starr{} é usada como recusa.

As normas deste preceito estão explicadas em Pessachim e no final de Temurá.

\paragraph{O Nazir deve deixar crescer seus cabelos}

Por este preceito o Nazir é ordenado a deixar crescer seus cabelos. Ele
está expresso em Suas palavras, enaltecido seja Ele, ``Sagrado será ele;
deixará crescer o cabelo de sua cabeça''.\footnote{Números 6:5.} A esse respeito
diz a Mekhiltá: ```Sagrado será o seu cabelo': ele deixará crescer seu
cabelo em sinal de santidade. `Deixará crescer\ldots{} cabeça' é um preceito
positivo. De que forma ficamos sabendo que também há um preceito
negativo? Pelas palavras das Escrituras: `Lâmina não passará pela sua
cabeça'\,''.\footnote{Ver o preceito negativo 209.}

Também está dito na Mekhiltá: ``O preceito positivo se aplica
ao\footnote{Aplica-se ao Nazir.} que esfregar terra ou aplicar produtos
químicos''; ou seja, se ele aplicar produtos químicos em sua cabeça que provoquem a queda dos cabelos ele não
estará infringindo o preceito negativo, porque ele não terá colocado em
sua cabeça nada semelhante a uma lâmina, mas estará violando o preceito
positivo ``Deixará crescer o cabelo de sua cabeça'' ao não permitir que
o cabelo cresça. E, de acordo com nossos Fundamentos, um preceito
negativo derivado de um preceito positivo é um preceito positivo.

As normas deste preceito estão explicadas no lugar apropriado, no Tratado Nazir.

\paragraph{A obrigação do Nazir de consumar seu voto}

Por este preceito o Nazir é ordenado a raspar sua cabeça e a levar seus
sacrifícios quando os dias de sua consagração estiverem terminados. A
Sifrá diz: ``Há três casos em que a raspagem da cabeça constitui um
preceito positivo: para o Nazir, o leproso e para os Levitas''.
Contudo, a raspagem da cabeça dos Levitas\footnote{Números 8:7.} só era
obrigatória enquanto eles estavam no deserto e deixou de sê-lo depois disso, enquanto que a raspagem da cabeça do leproso e do Nazir será sempre obrigatória.

Está claro que o Nazir tem duas ocasiões para raspar sua cabeça: ele
deve fazê-lo se tiver se tornado impuro, como prescrito nas palavras ``E
quando alguém morrer subitamente, junto a ele'' etc.,\footnote{Números 6:9.} e
em estado de pureza, como prescrito nas palavras ``No dia em que se
completarem os dias de seu Nazirado''.\footnote{Ibid. 6:13.}

Contudo, essas duas obrigações de raspar a cabeça não podem ser contadas
como dois preceitos diferentes, uma vez que a raspagem a ser feita por
causa de uma impureza é um dos detalhes do regulamento relativo ao
preceito do voto do Nazir --- o preceito positivo de deixar crescer seu
cabelo em santidade, como já explicamos. Depois disto as Escrituras
especificam que se o Nazir se tornar impuro ele deve raspar sua cabeça e
trazer um sacrifício e depois deixar seu cabelo crescer novamente em
santidade durante o período de Nazirado a que ele se propôs; assim como
no caso do leproso, as duas obrigações de raspar a cabeça constituem um
único preceito, como explicarei no lugar apropriado.

Também explicarei mais tarde por que, no caso do Nazir, nós contamos
raspar a cabeça e trazer seus sacrifícios como um preceito, enquanto que
no caso do leproso nós os contamos como dois.

As normas deste preceito, ou seja, a raspagem da cabeça do Nazir, estão
explicadas no texto a esse respeito do Tratado Nazir.

\paragraph{Cumprir todos os compromissos orais}

Por este preceito somos ordenados a cumprir toda obrigação que
assumirmos através de palavras --- todo juramento, promessa, oferenda ou
equivalente. Ele está expresso em Suas palavras, enaltecido seja Ele, ``O
que sair de teus lábios guardarás''.\footnote{Deuteronômio 23:24.}
Embora\footnote{Os Sábios.} tenham analisado minuciosamente este versículo e tenham explicado cada palavra separadamente,
o sentido global de tudo o que eles dizem é que este é um preceito
positivo que obriga o homem a cumprir todo compromisso que ele tenha
assumido, e que deixar de fazê-lo é transgredir um preceito negativo.
Isto será explicado quando eu tratar dos preceitos
negativos.\footnote{Ver o preceito negativo 157.}

O Sifrei diz: ```O que sair dos teus lábios' é um preceito positivo''.

Você sabe, porém, que não se pode derivar nenhum preceito a partir das
simples palavras ``O que sair dos teus lábios'', portanto o significado
delas deve ser o que mencionei como significado literal das Escrituras,
ou seja, que um homem tem a obrigação de cumprir o que seus lábios
tenham proferido. Este preceito também se encontra em outro trecho em
Suas palavras, enaltecido seja Ele, ``Como tudo que saiu de sua boca,
assim fará''.\footnote{Números 30:3.}

As normas deste preceito estão explicadas em vários trechos de Shabuot
e de Nedarim, no final de Menahot, e também no Tratado Kinim; ou seja,
foi deixado claro que devemos ser corretos no cumprimento de qualquer
obrigação a que tenhamos nos comprometido, e de que forma podemos ser
absolvidos se tivermos dúvidas quanto ao que nos tenhamos imposto.

\paragraph{A revogação de promessas}

Por este preceito somos ordenados a aplicar as leis relativas à
revogação de promessas. Este preceito, contudo, não significa que
sejamos obrigados a revogar as promessas em todos os casos. Você deve
entender que a mesma coisa se aplica a todas as leis que eu enumerar:
um preceito não nos obriga necessariamente a fazer uma determinada
coisa, mas estipula como devemos tratar do assunto em questão, de acordo
com as leis.

Evidentemente, está explícito nas Escrituras que um marido e um pai
podem fazer a revogação\footnote{Podem revogar as promessas da esposa e da filha, respectivamente.} e estão indicados os
procedimentos para isso. A Tradição também autoriza um sábio a liberar alguém de uma promessa ou de um
juramento, baseada em Suas palavras ``Não profanará a sua palavra'',\footnote{Números 30:2.} ou seja, ``Ele não pode quebrar sua promessa, mas os outros podem
fazê-lo por ele''. De uma maneira geral, não há base
escrita\footnote{Na Torá escrita.} para isso, e eles\footnote{Os Sábios.} dizem: ``As regras sobre liberar alguém de uma promessa pairam no ar e não têm
nada de concreto para apoiá-las'', a não ser a verdadeira
Tradição.\footnote{A Tradição oral.}

As normas deste preceito estão explicadas no Tratado que trata
especificamente deste assunto, que é o Tratado Nedarim.

\paragraph{Tornar-se impuro com carcaças de animais}

Por este preceito somos ordenados quanto à impureza da carcaça de um
animal, e ele inclui a impureza de uma carcaça e todas as normas a esse
respeito.

À guisa de prefácio mencionarei algo que você deve ter em mente com
relação a tudo que será dito a seguir, quanto aos vários tipos de
impureza. O fato de que contemos cada um dos vários tipos de impureza
como um preceito positivo não significa que seja uma obrigação ou que
seja proibido tornar-se impuro de uma ou de outra dessas maneiras, como
se isso fosse um preceito negativo. O que queremos dizer é que quando a
Torá diz que quem tocar este ou aquele tipo\footnote{Tipo de impureza.}
torna-se impuro, ou que este ou aquele objeto torna impuro de uma
determinada forma aquele que o tocar, isto constitui um preceito
positivo; ou seja, esta lei que somos obrigados a seguir é um preceito
que estabelece que quem tocar determinadas coisas sob determinadas
condições se tornará impuro mas, caso seja sob condições diferentes, ele
não se tornará impuro. Na realidade, o fato de se tornar impuro é
opcional: se um homem quiser, ele se tornará impuro, e se não o quiser,
não o fará.

A Sifrá diz: ```No seu cadáver não tocareis':\footnote{Levítico 11:8.}
poderíamos pensar que se uma pessoa tocar uma carcaça ele estará
sujeito ao açoitamento; por isso as Escrituras dizem: `E por estes vos
tornareis impuros'.\footnote{Ibid., 24.} Poderíamos pensar que se uma pessoa vê
uma carcaça ele deve tocá-la e assim tornar-se impuro. Por isso as
Escrituras dizem: `No seu cadáver não tocareis'. Como fazer para
conciliar esses dois versículos? Devemos concluir que tocar uma carcaça
é opcional''.


Portanto o conteúdo deste preceito é que todo aquele que
tocar\footnote{Que tocar qualquer tipo de impureza.} se torna impuro, e ao se tornar impuro ele
está sujeito a todas as obrigações impostas às pessoas impuras: ele deve
sair do Acampamento da Presença Divina; não deve comer nem tocar Coisas
Sagradas, e assim por diante. A essência deste preceito é que uma pessoa
se torna impura ao tocar um determinado objeto ou, em determinadas
circunstâncias, ao estar próximo dele.

Tenha isso em mente com relação aos vários tipos de impureza.

\paragraph{Tornar-se impuro através das carcaças de determinados animais rastejantes}

Por este preceito somos ordenados quanto à impureza de oito tipos
de criaturas rastejantes.\footnote{Esse preceito está expresso em Levítico 11:29-30.} Este preceito abrange a
lei da impureza de animais rastejantes e as regras detalhadas referentes
a ela.

\paragraph{Tornar-se impuro através de comida e bebida}

Por este preceito somos ordenados a lidar com a impureza da comida e da
bebida de acordo com as leis prescritas. Este preceito inclui todas as
leis relativas à impureza de comida e bebida de todos os
tipos.\footnote{Essas leis estão contidas em Levítico 11:34.}

\paragraph{A mulher menstruada}

Por este preceito somos ordenados quanto à impureza da mulher
menstruada. Este preceito inclui todas as regras a esse
respeito.\footnote{Esse preceito se encontra em Levítico 15:19-24.}

\paragraph{Depois do nascimento de uma criança}

Por este preceito somos ordenados quanto à impureza de uma mulher
depois de um parto. Este preceito inclui todas as regras a esse
respeito.\footnote{Ele está expresso em Levítico 12:2-5.}

\paragraph{O leproso}

Por este preceito somos ordenados quanto à impureza de um leproso. Este
preceito inclui toda a regulamentação referente à lepra: quais são casos
impuros e quais os puros, quais necessitam segregação e quais não, quais
requerem, além da segregação, que a cabeça seja raspada, e quais não,
bem como outros detalhes relativos às suas regras e à natureza de sua
impureza.\footnote{Todas as leis referentes a esse preceito estão contidas em Levítico
  13:1-59.}



\paragraph{As roupas contaminadas pela lepra}

Por este preceito somos ordenados quanto à impureza de uma roupa
contaminada pela lepra. Ele inclui toda a regulamentação a esse
respeito: como as roupas se tornam impuras e como elas causam impureza,
quais devem ser separadas, rasgadas, queimadas, lavadas ou purificadas, bem
como tudo o mais que está prescrito nas Escrituras e o que a Tradição
diz a esse respeito.\footnote{As regras que regem este preceito estão em Levítico 13:47-59.}

\paragraph{A casa de um leproso}

Por este preceito somos ordenados quanto à impureza da casa de um
leproso. Este preceito inclui todas as regras a esse respeito: que elas
devem ser isoladas, quais devem ter suas paredes parcialmente demolidas,
ou ser completamente demolidas, como elas se tornam impuras e como elas
causam impurezas.\footnote{Os detalhes deste preceito se encontram em Levítico
14:36-48.}

\paragraph{O ``Zab''}

Por este preceito somos ordenados quanto à impureza de um ``zab''.
Este preceito inclui toda a regulamentação relativa aos sintomas de um
``zab'' e à maneira como ele torna outras pessoas
impuras.\footnote{As regras que regem este preceito estão em Levítico 14:35-54 e
  15:1-12.}

\paragraph{O sêmen}

Por este preceito somos ordenados quanto à impureza do sêmen. Este
preceito inclui toda a regulamentação a esse respeito.\footnote{As regras que regem este preceito estão em Levítico 15:16-18.}

\paragraph{A ``Zaba''}

Por este preceito somos ordenados quanto à impureza de uma ``zaba''.
Ele inclui as regras referentes aos sintomas que tornam uma mulher uma
``zaba'' e à maneira pela qual ela torna outras pessoas
impuras\footnote{As regras que regem este preceito estão em Levítico 15:25-30.} após ter-se tornado uma ``zaba''.

\paragraph{A impureza de um cadáver}

Por este preceito somos ordenados quanto à impureza de um cadáver. Este
preceito inclui todas as regras a esse respeito.\footnote{As regras que regem este preceito estão em Números 19:14-16.}

\paragraph{A lei da água de aspersão}

Por este preceito somos ordenados quanto às regras referentes à água
de aspersão\footnote{Números XIX, 9-21.} a qual, sob certas circunstâncias,
purifica e, sob outras, impurifica, como será explicado no estudo
detalhado deste preceito.

Você deve saber que os treze tipos de impurezas que foram enumerados
--- a saber, a impureza de uma carcaça, de animais rastejantes, de
comidas, de uma mulher menstruada, de uma mulher depois do parto, de um
leproso, de roupas contaminadas pela lepra, de uma casa contaminada
pela lepra, de um ``zab'', de uma ``zaba'', do sêmen, de um cadáver, e
da água de aspersão --- e a purificação para cada um deles que estão
todos explicados na Torá; deve saber também que há vários textos, regulamentações e condições relativos a cada um desses preceitos expostos nos trechos Vayehi Bayom
Hashemini,\footnote{Levítico IX, 1-11; 47.} Tazria,\footnote{Levítico XII, 1-13; 59.} Zot
Tih'ye\footnote{Levítico XIV, 1-15; 33.} e no trecho Veyikemu Eilecha para
Adoma.\footnote{Números XIX, 1-22.}
Nesses quatro textos está tudo o que se refere aos vários tipos de
impureza. Mas todas as regras e regulamentos relativos a esses tipos de
impurezas estão contidos na Ordem de Teharot (Pureza). Três dos
Tratados dessa Ordem, a saber, Teharot, Makhshirin e Okatzin, contêm as
impurezas referentes à comida, e tratam exclusivamente desse assunto, e
qualquer menção sobre outras impurezas nesses Tratados é meramente
acidental. Da mesma forma, o Nidá contém os regulamentos referentes à
mulher mestruada, à ``zaba'' e à mulher depois do parto, sendo que
também se encontram algumas das regras referentes a esta última no
Tratado Queretot. O Tratado Negaim contém todos os regulamentos sobre a
lepra nos homens, roupas e casas; o Tratado Zabim contém os
regulamentos referentes ao ``zab'', à ``zaba'' e ao sêmen; o Tratado
Ohalot trata dos cadáveres e o Tratado Pará traz as regras sobre a água
de aspersão como um agente de purificação ou de impurificação. Por outro
lado, não há Tratados específicos sobre a impureza de carcaças e de
animais rastejantes; os regulamentos sobre esses assuntos estão
espalhados em vários trechos da Ordem, sobretudo nos Tratados Quelim e
Teharot. E várias questões referentes a esses assuntos são tratadas no
Tratado Eduyot. Nós próprios compusemos um comentário sobre a Ordem de
Teharot e não é necessário consultar qualquer outro livro além desse
sobre qualquer assunto relativo à pureza e impureza.

\paragraph{Mergulhar no banho ritual}

Por este preceito somos ordenados a mergulhar nas águas de um banho
ritual, e assim limpar-nos de todo tipo de impurezas que nos tenham
afetado. Este preceito está expresso em Suas palavras, enaltecido seja
Ele, ``E o homem\ldots{} banhará em água toda a sua carne'',\footnote{Levítico
15:16.} a respeito das quais a Tradição diz: ``Água suficiente para
cobrir todo o seu corpo: isto é, a medida de um banho ritual'' ---
exceto no caso de água corrente, para a qual não há medida prescrita,
como está explicado nos detalhes deste preceito.

Uma das cláusulas deste preceito estipula que apenas o ``zab'' deve
purificar-se em água corrente, de acordo com o que diz a Torá: ``E
banhará sua carne nas águas vivas''.\footnote{Ibid., 13.}

Ao tratar da imersão como um preceito positivo, não queremos dizer que
todas as pessoas impuras devam tomar banho de imersão, que todas
as pessoas que usam uma vestimenta devem colocar tsitsit nela, ou
que todos os que tenham uma casa devem fazer um suporte; o que quero
dizer é apenas que pela lei da imersão aquele que desejar livrar-se de
suas impurezas não pode atingir seu objetivo a não ser pela imersão em
água, depois do que ele se tornará puro.

A Sifrá diz: ```E lavar-se-á nas águas';\footnote{Levítico 14:8.} poderíamos
julgar isto como sendo um decreto do Rei. Por isso as Escrituras dizem:
`E depois entrará no acampamento',\footnote{Ibid.} por causa de sua
impureza.\footnote{Entrará no acampamento do qual havia sido excluído por causa de sua
  impureza.} Isto leva ao princípio que eu havia
explicado, ou seja, que a lei de imersão se aplica apenas àquele que
quiser se purificar, e esta lei é um preceito. Não há, contudo, nenhuma
obrigação de banhar-se, e aquele que quiser permanecer impuro e estiver
disposto a privar-se de entrar no acampamento da Presença Divina por
algum tempo tem a liberdade de fazê-lo.

O Livro da Verdade deixa claro que todo aquele que estiver impuro e
fizer uma imersão se purificará, mas sua purificação não estará completa
até o pôr do sol; e de acordo com a interpretação tradicional, ele
deverá estar nu e todo o seu corpo deverá ficar em contato com a água.
Como diz o Talmud, ```Toda sua carne', ou seja, não haver nada entre sua
carne e a água''.

Fica assim claro que este preceito da imersão inclui a regulamentação
do banho ritual, a interposição,\footnote{A interposição de alguma coisa entre o corpo e a água.} e o ``Tebul
yom''. Ele está explicado nos Tratados Mikvaot e Tebul Yom.

\paragraph{Purificar-se da lepra}

Por este preceito somos ordenados a que a purificação da lepra seja
realizada de acordo com as normas estabelecidas nas Escrituras, ou seja,
com pau de cedro, hissopo, carmezim, dois pássaros vivos e águas vivas,
e que eles sejam empregados como está determinado. O homem será
purificado por esse procedimento, como explicam as Escrituras.

Foi explicado, portanto, que, de acordo com nossa Torá, há três
métodos diferentes pelos quais pode ser realizada a purificação: um
geral, e dois aplicáveis cada um apenas a um tipo específico de
impureza. O método geral é pela água, a qual é indispensável para
purificar-se; o segundo é pela água de aspersão e aplica-se
especificamente à impureza adquirida através de um morto; o terceiro,
que consiste de pau de cedro, hissopo, carmezim, dois pássaros vivos e
águas vivas, aplica-se especificamente no caso da lepra.

Todas as normas deste preceito, ou seja, a primeira purificação da
lepra, estão explicadas no Tratado Negaim.

\paragraph{O leproso deve raspar a cabeça}

Por este preceito o leproso é ordenado a raspar a cabeça, e isto
constitui o segundo estágio de sua purificação, como está explicado no
final de Negaim. Este preceito está expresso em Suas palavras,
enaltecido seja Ele, ``E ao sétimo dia raspará todo o seu pelo''.\footnote{Levítico 14:9.} Já nos referimos anteriormente às palavras dos Sábios:
``Três raspam suas cabeças, e para cada um deles
a raspagem constitui um preceito positivo: o Nazir, o leproso e o
Levita''. As normas deste preceito estão explicadas no final de Negaim.

Aqui explicarei por que no caso do leproso contamos a raspagem e a
oferta dos sacrifícios determinados cada um como um preceito individual,
enquanto que no caso do Nazir contamos os dois juntos como um preceito.
É que no caso do leproso não há conexão entre o ato de raspar a cabeça e
o de levar seus sacrifícios, e o objetivo atingido pela raspagem é
diferente do alcançado pelo oferecimento dos sacrifícios, porque sua
purificação depende da raspagem de sua cabeça. No sexto capítulo de
Nazir está dito: ``Como um Nazir difere de um leproso? Sua purificação
depende dos dias, enquanto que a purificação do leproso depende da
raspagem de seus cabelos''. Tendo raspado a cabeça, e tendo completado
sua segunda raspagem, o leproso se purifica e cessa de transmitir o tipo
de impureza que é transmitida pelos animais rastejantes, como está
explicado no final de Negaim; e seu perdão fica em suspenso até que ele
leve seus sacrifícios, da mesma forma que os ciutros cujo perdão fica em
suspenso, como está explicado ali.

Assim, a raspagem da cabeça torna o leproso puro a ponto de que ele
cesse de transmitir o mesmo tipo de impureza que é transmitida por um
animal rastejante, quer ele tenha ou não oferecido seus sacrifícios; e a
oferta dos sacrifícios complementa seu perdão, tal como nos outros casos
em que o perdão fica em suspenso --- a saber, o ``zab'', a ``zaba'', e a
mulher depois do parto. Nós já nos referimos às palavras dos Sábios:
``Há quatro pessoas cujo perdão fica em suspenso'' etc.

No caso de um Nazir, como está explicado ali, o perdão não fica
incompleto, e todo procedimento estabelecido --- raspar a cabeça e
oferecer o sacrifício --- lhe permite beber vinho novamente. E um não é
suficiente sem o outro: a raspagem está ligada ao sacrifício, e o
sacrifício à raspagem, e os dois em conjunto atingem um objetivo único,
que é permitir-lhe as coisas que lhe eram proibidas nos seus dias de
Nazirado. No sexto capítulo de Nazir está dito: ``Se ele raspou seu
cabelo depois de um dos sacrifícios e este foi considerado inválido, a
raspagem de seu cabelo também se torna inválida e seus sacrifícios não
contam. Assim, ficou explicado que a raspagem é uma das condições da
oferenda, e a oferenda é uma das condições da raspagem.

Também foi explicado na Tosseftá que um Nazir que tenha completado seus dias\footnote{Seus dias de Nazirado.} será proibido de raspar a cabeça,
beber vinho e tornar-se impuro pelos mortos até que ele tenha completado todo o procedimento de raspagem em estado de pureza, o qual, como está explicado no sexto capítulo de Nazir, implica que a raspagem seja feita diante da porta da Tenda de
Assinação, que ele jogue seus cabelos debaixo da caldeira e que ele
leve as oferendas, como está explicado nas Escrituras.

Vocês encontrarão que na maioria dos lugares os Sábios denominam
a oferta dos sacrifícios\footnote{Dos Nazirim.} de ``raspagem'', e eles
dizem explicitamente na Mishné: ```Eu serei um Nazir e me comprometo a
raspar a cabeça etc.': a intenção é de levar as ofertas do Nazir e
oferecê-las por si próprio. Assim foi explicado que a raspagem é um
termo alternativo para seu oferecimento de sacrifícios, e a razão disso
é que parte deste último, como explicamos, e é unicamente com a
combinação destes dois que o Nazirado se completa e que o Nazir pode
beber vinho. Mas a raspagem por impureza é apenas uma das leis do
preceito, como explicamos anteriormente.

\paragraph{O leproso deve ser reconhecível}

Por este preceito somos ordenados a que o leproso seja tornado
reconhecível, de forma que as pessoas possam se manter afastadas dele.
Este preceito está expresso em Suas palavras, enaltecido seja Ele, ``E
do leproso que tem a chaga, suas vestes serão rasgadas e seu cabelo não
será cortado, e com seu bigode se cobrirá; e impuro! impuro! clamará''.\footnote{Levítico 13:45.}

A prova de que este é um dos preceitos positivos é encontrada na Sifrá:
``Como foi dito, a respeito do Cohen Gadol que `Seu cabelo não deixará
crescer e suas vestes não romperá'\footnote{Ibid., 21:10.} poder-se-ia pensar que
isto se aplica mesmo se ele tiver contraído uma praga, e que o preceito
`Suas vestes serão rasgadas e seu cabelo não será cortado' se aplica a
todos menos ao Cohen Gadol. Por isso a Torá diz: `Em quem estiver a
praga' etc.; mesmo se ele for um Cohen Gadol, suas roupas devem ser
rasgadas, o seu cabelo deve ficar solto, e ele deve deixar seu cabelo
crescer''.

Está claro que um Cohen Gadol está proibido, através de um preceito negativo, de rasgar suas roupas ou de deixar crescer seu cabelo; e
é um princípio aceito entre nós que toda vez que encontrarmos um
preceito positivo e um negativo,\footnote{Que se apliquem os dois à mesma situação.} se pudermos
cumprir os dois será ótimo; caso contrário o preceito positivo se
sobrepõe ao negativo. Consequentemente, uma vez que está estabelecido
que um Cohen Gadol leproso deve deixar crescer seu cabelo e rasgar
suas roupas, conclui-se que este é um preceito positivo.

Por tradição, outras pessoas impuras também são obrigadas a se tornarem
reconhecíveis a fim de que os outros possam se manter afastados delas. A
Sifrá diz: ``Como sabemos que isto também se aplica a alguém que tenha
se tornado impuro através de um cadáver ou de uma relação com uma
mulher menstruada, e a todos os que possam transmitir impureza aos
outros? As Escrituras dizem: `E impuro! impuro! Clamará'\,''.\footnote{Ibid.,
13:45.} Isto significa que toda pessoa impura deve anunciar sua
impureza, ou seja, deve se tornar reconhecível como pessoa impura, que
transmite impureza com seu contato, de forma a que as pessoas possam
evitá-la.

Ficou claro que a obrigação de um leproso de se tornar reconhecível não
se aplica às mulheres. ``Um homem'', diz o Talmud, ``anda com o cabelo
solto e com as roupas rasgadas, mas uma mulher não anda com o cabelo
solto e com as roupas rasgadas'', embora ela deva cobrir seu lábio
superior, e proclamar sua impureza como as outras pessoas impuras.

\paragraph{As cinzas da vaca vermelha}

Por este preceito somos ordenados a preparar a vaca vermelha de maneira
tal que suas cinzas possam ser utilizadas para fazer o que deve ser
feito a fim de remover a impureza de um cadáver, como disse o
Enaltecido: ``E a congregação dos filhos de Israel a guardará por água
purificadora''.\footnote{Números 19:9.}

As normas deste preceito estão explicadas no Tratado que trata
especificamente deste assunto, que é o Tratado Pará.



\paragraph{A avaliação de uma pessoa}

Por este preceito somos ordenados quanto à lei da avaliação do homem, a
qual estabelece que quando alguém diz ``Eu prometo meu próprio valor''
ou ``Eu prometo o valor de uma determinada pessoa'', se for um homem,
ele deve pagar uma determinada importância, e se for uma mulher, ele
deve pagar urna determinada importância de acordo com a idade, como está
estabelecido nas Escrituras, e de acordo com as posses de que fez a
promessa. Este preceito está expresso em Suas palavras, enaltecido seja
Ele, ``Quando alguém fizer um voto, para oferecer preço de almas\ldots{}
segundo as posses daquele que fez o voto''.\footnote{Levítico 27:2-8.}

As normas deste preceito estão explicadas no Tratado Arakhin.


\paragraph{A avaliação de animais}

Por este preceito somos ordenados quanto à lei da avaliação dos animais
impuros. Ele está expresso em Suas palavras ``Fará apresentar o animal
diante do Cohen. E o avaliará o Cohen.\footnote{Levítico 27:11-12.}

As normas deste preceito estão explicadas em trechos dos Tratados Temurá
e Meilá.

\paragraph{A avaliação de casas}

Por este preceito somos ordenados quanto à avaliação das casas. Ele está
expresso em Suas palavras: ``E quando alguém consagrar a sua casa para
ser santidade ao Eterno, o Cohen a avaliará''.\footnote{Levítico 27:14.}

As normas deste preceito estão explicadas no Tratado Arakhin.

\paragraph{A avaliação dos campos}

Por este preceito somos ordenados quanto à avaliação dos campos. Ele
está expresso em Suas palavras, enaltecido seja Ele, ``E se alguém
consagrar uma parte do campo de sua possessão ao Eterno'' etc.\footnote{Levítico
27:16.} \ldots{} ``E se o campo de sua compra, que não é campo de sua herança,
consagrar ao Eterno'' etc.\footnote{Ibid., 22.} No caso de ``campo de sua
possessão'', ``A tua avaliação será conforme a semente necessária para
semeá-lo'';\footnote{Ibid., 16.} e no caso de ``campo de sua compra'', ``O
Cohen calculará para ele, a conta de sua avaliação''.\footnote{Ibid., 23.}

As normas deste preceito também estão explicadas na íntegra no Tratado Arakhin.

Que ninguém pense que esses quatro tipos de avaliação têm tanto em comum
que deveriam ser contados como um único preceito. Eles são quatro
preceitos separados, cada um com suas regras específicas, embora a
palavra ``avaliação'' seja comum a todos eles. Consequentemente não se
devem contar todos os tipos de avaliação como um único preceito, da
mesma maneira que não se contam todos os tipos de oferenda como um único
preceito. Isso se torna claro ao se estudar cuidadosamente o assunto.

\paragraph{A restituição por sacrilégio}

Por este preceito aquele que sem querer usar a propriedade do Templo ou
comer alguma coisa Sagrada fica obrigado a restituir o que ele usou ou
comeu acrescido de um quinto.\footnote{Um quinto do valor daquilo que deve restituir.} Este preceito está
expresso em Suas palavras, enaltecido seja Ele, ``E pagará o da coisa
sagrada pela qual pecou, acrescentando a quinta parte''\footnote{Levítico 5:16.}
e ``E o homem que comer santidade por erro, acrescentará a quinta parte
do valor desta''.\footnote{Ibid. 22:14.} As normas deste preceito estão
explicadas no Tratado Meilá, e também no Tratado Terumot.

\paragraph{A colheita do quarto ano}

Por este preceito somos ordenados a considerar todo fruto da colheita
do quarto ano como sagrado. Este preceito está expresso em Suas
palavras ``E no quarto ano, será todo o seu fruto santidade de louvores
ao Eterno''.\footnote{Levítico 19:24.} A lei prevê que o proprietário deve
levá-lo a Jerusalém e comê-lo lá, como no caso do Segundo Dízimo. Os
Cohanim não recebem nenhuma parte dele.

O Sifrei diz: ```E as santidades de todo homem dele serão':\footnote{Números 5:10.} as Escrituras separaram todas as Coisa Sagradas e as deram
aos Cohanim, exceto a Ação de Graças,\footnote{Ver preceito positivo 66.} o Sacrifício de Paz, o Sacrifício de Pessach,\footnote{Ver preceito positivo 55 e 56.} o Dízimo do Gado,\footnote{Ver preceito positivo 78.} o segundo Dízimo,\footnote{Ver preceito positivo 128.} e a colheita do quarto ano
das plantas, os quais declarou pertencerem aos proprietários''.

As normas deste preceito estão explicadas integralmente no último
capítulo do Tratado Maasser Sheni.


\paragraph{Peá\star{} para os pobres}

Por este preceito somos ordenados a deixar peá\starr{}
de cereais, frutas e similares. Ele está expresso em Suas palavras,
enaltecido seja Ele, ``Para pobre e o imigrante os deixarás''.\footnote{Levítico 19:10.}

No Tratado Macot está explicado que ``peá'' envolve um preceito negativo
justaposto a um preceito positivo. O preceito negativo está expresso
em Suas palavras ``Não acabarás de segar o canto de teu campo'',\footnote{Ibid.
19.9. Ver o preceito negativo 210.} e o positivo está em Suas palavras: ``Para o pobre e o peregrino os
deixarás''.

As normas deste preceito estão explicadas no Tratado Peá. A Torá limita
sua aplicação à Terra de Israel.

\paragraph{A respiga para os pobres}

Por este preceito somos ordenados a deixar respigas. Ele está expresso
em Suas palavras: ``Nem colhereis a respiga de vossa ceifa; para os
pobres e para o imigrante os deixareis''.\footnote{Levítico 23:22.}

Este preceito também envolve um preceito negativo justaposto a um
preceito positivo, como explicado no Tratado Macot com relação à
``peá''.\footnote{Ver o preceito negativo 211.}


As normas deste preceito estão explicadas no Tratado Peá. A Torá limita
sua aplicação à Terra de Israel.

\paragraph{Deixar a gavela esquecida para os pobres}

Por este preceito somos ordenados a deixar a gavela esquecida. Este
preceito está expresso em Suas palavras, enaltecido seja Ele, ``Quando
segares a messe no teu campo, e esqueceres uma gavela, no campo, não
voltarás a tomá-la; para o imigrante, o órfão, e a viúva será''.\footnote{Deuteronômio 24:19.} As palavras ``Para o imigrante, o órfão, e a viúva
será'' constituem o preceito positivo de
deixar a gavela esquecida, assim como as palavras ``Os deixarás''\footnote{Levítico 19:10.} constituem o preceito positivo, no caso das respigas e da ``peá'', como
foi explicado.

A aplicação deste preceito também está limitada pela Torá à Terra de Israel.\footnote{Neste caso, bem como no caso dos preceitos 120 e 121, as leis
  rabínicas determinam que estas obrigações também devem ser cumpridas
  fora de Israel (Mishné Torá, Leis das Dádivas aos Pobres, Cap. 1,
  Lei 14).}

As normas deste preceito também estão explicadas no Tratado Peá.

\paragraph{Deixar as sobras dos cachos de uva para os pobres}

Por este preceito somos ordenados a deixar para os pobres os cachos de
uva que sobrarem no momento da colheita, chamados ``olelot''. Também a
respeito deles as Escrituras dizem: ``Para o pobre e o imigrante os
deixarás''.\footnote{Levítico 19:10.}

As normas deste preceito estão explicadas no Tratado Peá, e a Torá
limita sua aplicação à Terra de Israel.

\paragraph{Deixar as uvas caídas para os pobres}

Por este preceito somos ordenados a deixar para os pobres o que cair e
ficar separado do cacho durante a vindima. Ele está expresso em Suas
palavras: ``E o bago de tua vinha não recolherás; para o pobre e o
imigrante os deixarás''.\footnote{Levítico 19:10. Ver preceito negativo 213.}

As normas detalhadas deste preceito estão explicadas no Tratado Peá, e a
Torá limita sua aplicação à Terra de Israel.

\paragraph{Levar as primícias ao Santuário}

Por este preceito somos ordenados a separar as primícias e levá-las ao
Santuário. Ele está expresso em Suas palavras, enaltecido seja Ele, ``O
princípio das primícias de tua terra trarás à casa do Eterno, teu
Deus''.\footnote{Êxodo 23:19.}

Está claro que este preceito só é obrigatório durante a existência do
Santuário, e que se aplica apenas às ``Sete Categorias'' de
produtos\footnote{Os sete tipos de produtos pelos quais a terra de Israel era famosa:
  trigo, cevada, uvas, figos, romãs, óleo de oliva, e mel de tâmara.} que crescem na Terra de Israel, na Síria
e na Transjordânia.

As normas deste preceito estão explicadas no Tratado Bicurim, onde foi deixado claro que elas, ou seja, as primícias, são propriedades do Cohen.

\paragraph{A grande Oferta de Elevação}

Por este preceito somos ordenados a separar a grande Oferta de
Elevação. Ele está expresso em Suas palavras, enaltecido seja Ele, ``As
primícias de teu grão\ldots{} darás a ele''.\footnote{Deuteronômio, 18:4.} De acordo
com a Torá, ele é obrigatório apenas na Terra de Israel, e suas normas
estão explicadas no Tratado Terumot.

\paragraph{O primeiro dízimo}

Por este preceito somos ordenados a separar o dízimo do produto da
terra. Ele está expresso em Suas palavras, enaltecido seja Ele, ``Porque
os dízimos dos filhos de Israel, que separarem ao Eterno em oferta''.\footnote{Números 18:24.} As Escrituras explicam que este dízimo pertence aos
Levitas.

As normas deste preceito estão explicadas no Tratado Maasserot. Ele é
chamado de o Primeiro Dízimo, e a Torá só o torna obrigatório dentro da
Terra de Israel.

\paragraph{O segundo dízimo}

Por este preceito somos ordenados a separar o segundo dízimo. Ele está
expresso em Suas palavras, enaltecido seja Ele, ``Certamente separarás o
dízimo de todo o produto das tuas sementes, que o campo produzir de ano
a ano''\footnote{Deuteronômio 14:22.} sobre as quais diz o Sifrei: ```Ano a ano':
isto nos ensina que os dízimos não devem ser deixados de um ano para o
outro. Contudo, as palavras das Escrituras se referem apenas ao segundo
dízimo; como saber que elas devem ser aplicadas aos outros dízimos
também? Porque a Torá diz: `Certamente separarás o dízimo etc.'\,''.

A Torá expõe claramente que este dízimo deve ser levado a Jerusalém,
para lá ser comido pelo seu proprietário. Nós já nos referimos ao que os
Sábios dizem a este respeito.

As Escrituras dão as leis deste preceito em detalhes, dizendo que quando
é impossível levá-lo a Jerusalém devido à distância, ele deve resgatá-lo
e levar seu valor em dinheiro ao Santuário e ali gastá-lo
exclusivamente com comida. Isso está estipulado em Suas palavras,
enaltecido seja Ele, ``E se o caminho te for comprido, de sorte que não o possas levar, por estar longe de ti,'' etc.\footnote{Deuteronômio 14:24.} Outra norma estabelecida pela Torá é que se ele o resgatar para si próprio, ele deverá acrescentar um
quinto.\footnote{Um quinto de seu valor.} Isto está determinado em Suas palavras,
enaltecido seja Ele, ``E se quiser a remir o seu dízimo, acrescentar-lhe-á a quinta parte de seu preço''.\footnote{Levítico 27:31.}
Todas as regras detalhadas deste preceito estão explicadas no Tratado Maasser
Sheni.

Da mesma forma ele é obrigatório, pela Torá, apenas com relação aos
produtos da Terra de Israel, e deve ser comido apenas durante a
existência do Santuário. O Sifrei diz: ``Ela\footnote{A Torá.}
compara o ato de comer os primogênitos ao segundo dízimo: assim como os
primogênitos podem ser comidos apenas durante a existência do
Santuário, o segundo dízimo também só pode ser comido durante a
existência do Santuário''.

\paragraph{O dízimo dos Levitas para os Cohanim ou a Oferta de Elevação}

Por este preceito os Levitas são ordenados a separar um dízimo do dízimo
que eles receberam de Israel, e a dá-lo aos Cohanim. Este preceito
está expresso em Suas palavras, enaltecido seja Ele, ``E aos Levitas
falarás e lhes dirás: Quando tomardes dos filhos de Israel o dízimo que
deles vos dei: por vossa herança, dele separareis, uma oferta para o
Eterno, o dízimo do dízimo''.\footnote{Números 18:26.} As Escrituras explicam que
este dízimo, que é chamado de Oferta de Elevação do Dízimo, deve ser
dado ao Cohen: ``E dareis deles a oferta separada do Eterno, para
Aarão, o Cohen''.\footnote{Ibid., 28.}

As Escrituras explicam que este dízimo deve ser retirado da melhor e
mais selecionada parte: ``De todo o melhor delas, a parte consagrada que
lhe é consagrada''.\footnote{Ibid., 29.} As Escrituras ressaltam então que eles
cometem uma transgressão se não fizerem a seleção dentre a melhor parte:
``Não levareis sobre vós por isso, pecado, quando separardes o melhor
dele''.\footnote{Ibid., 32.} Este é um preceito negativo de exclusão, como se ele
tivesse dito: ``Não haverá pecado quando separardes do melhor''. Daí
deduzimos que se separarmos do pior haverá pecado, e portanto este é um
preceito negativo derivado de um preceito positivo, e portanto não está
contado entre os preceitos negativos: quer dizer, o preceito de fazer a
seleção entre o que há de melhor implica que a escolha não deve ser
feita dentre a pior parte. O Sifrei diz: ``De que modo você conclui que
se fizerem a seleção de outra parte que não a melhor eles cometem um
pecado? Porque as Escrituras dizem: `E não levareis sobre vós por isso,
pecado, quando separardes o melhor dele'\,''.

As normas deste preceito estão explicadas nos Tratados Terumot e
Maasserot, e em diversos lugares de Demai.

\paragraph{O dízimo do homem pobre}

Por este preceito somos ordenados a separar o dízimo para os pobres no
terceiro ano de cada ciclo de Shabat,\footnote{Deve-se deixar que a terra repouse a cada sete anos, formando assim o
  ciclo de Shabat.} e novamente
no terceiro ano depois de cada terceiro ano, ou seja, no se o ano de
cada ciclo de Shabat. Este preceito está expresso em Suas palavras,
enaltecido seja Ele, ``Ao fim de três anos tirarás todos os dízimos de
teu produto'' etc.\footnote{Deuteronômio 14:28.}

A Torá também torna este preceito obrigatório não somente na Terra de
Israel. Suas normas estão explicadas nos Tratados Peá e Maasserot, e
vários assuntos ligados a ele estão espalhados em vários trechos dos
outros Tratados de Zeraim, e nos Tratados Makhshirin e Yadayim.

\paragraph{A declaração do dízimo}

Por este preceito somos ordenados a declarar diante d'Ele, enaltecido
seja Ele, que separamos os dízimos obrigatórios e as Ofertas de
Elevação, e a verbalmente declarar que estamos liberados de nossas
obrigações, assim como nos exoneramos delas de fato. Este preceito,
chamado ``a Declaração do Dízimo'', está expresso em Suas palavras,
enaltecido seja Ele, ``E dirás diante do Eterno, teu Deus: Tirei o que é
consagrado, de minha casa, e também o dei ao Levita, e ao imigrante, e
ao órfão, e à viúva''.\footnote{Deuteronômio 26:13.}

As normas deste preceito, a maneira de proceder a separação e o seu
significado estão explicadas no último capítulo de Maasser Sheni.


\paragraph{A narração ao levar as primícias}

Por este preceito somos ordenados, ao levar as primícias, a narrar as
bondades que Deus, enaltecido seja Ele, nos concedeu, como Ele nos
libertou dos sofrimentos de nosso Patriarca Jacob, e da escravidão e
opressão dos Egípcios; a agradecer-Lhe por isso; e a implorar-Lhe para
perpetuar Suas bênçãos. Este preceito está expresso em Suas palavras,
enaltecido seja Ele, ``Falarás em voz alta e dirás diante do Eterno, teu
Deus: Arameu errante era meu pai''\footnote{Deuteronômio 26:5.} e todo o resto
desta passagem. Este preceito é chamado de o Relato das Primícias. Suas
normas estão explicadas no Tratado Bicurim e no sétimo capítulo de Sotá.
Ele não é obrigatório para as mulheres.

\paragraph{A oferta de massa}

Por este preceito somos ordenados a separar uma Torta (Halá) de cada
massa, e dá-la ao Cohen. Este preceito está expresso em Suas
palavras, enaltecido seja Ele, ``Em primeiro lugar, separareis de vossas
massas, uma torta; como oferta da eira''.\footnote{Números 15:20.}

As normas deste preceito estão explicadas nos Tratados Halá e Orlá, e a
Torá nos obriga a ele somente na Terra de Israel.

\paragraph{Renunciar à produção de sua propriedade no ano de Shabat}

Por este preceito somos ordenados a renunciar a toda a produção de
nossas terras no ano de Shabat,\footnote{O sétimo ano no ciclo de sete anos} e a permitir que
qualquer pessoa recolha tudo o que cresce em nossos campos. Este
preceito está expresso em Suas palavras, enaltecido seja Ele, ``E no
sétimo deixa-la-ás de cultivar, deixa-la-ás de adubar e limpar''.\footnote{Êxodo
23:11.}

A Mekhiltá diz: ``Se o vinhedo e o olival estão incluídos, porque eles
estão mencionados especificamente? Para servirem como analogia: assim
como a obrigação é um preceito positivo específico, cuja violação
acarreta também a transgressão de um preceito negativo, a violação de
qualquer preceito positivo acarreta a infração de um preceito
negativo''.

O significado disto é o que explicarei a seguir. O preceito ``No sétimo
deixa-la-ás de cultivar, deixa-la-ás de adubar e limpar'' abrange toda a
produção da terra durante o Ano de Shabat: figos, uvas, azeitonas,
pêssegos, romãs, trigo, cevada, e os outros. Consequentemente, é um
preceito positivo tratar todos esses tipos de produtos dentro dos
termos da lei de Shabat. Mas as Escrituras depois especificam: ``Assim
farás com tua vinha e teu olival'', embora estes já estejam incluídos
em ``todos'' os produtos da terra. O preceito não se aplica
especificamente ao vinhedo e ao olival, mas nos é ordenado por causa
da advertência das Escrituras quanto a recolher o produto do vinhedo, a
qual está nas palavras: ``As uvas separadas para ti, da tua vinha, não
colherás''.\footnote{Levítico 25:5. Ver o preceito negativo 223.} Assim como no caso do vinhedo, em que é
um preceito positivo declará-lo sem dono e não
fazê-lo\footnote{Não fazê-lo é transgredir um preceito negativo.} é um preceito negativo, assim está claramente expresso que, no caso de tudo o que crescer no sétimo ano, é um preceito positivo declará-lo sem dono e deixar de fazê-lo é um preceito negativo.

Portanto, o caso do olival é o mesmo que o do vinhedo no que se refere a
um preceito positivo e um negativo, e o caso do olival é o mesmo que o
dos outros produtos. Portanto, ficou claro por tudo o que foi exposto
que a renúncia a toda a produção do sétimo ano é um preceito positivo.

As normas deste preceito estão explicadas no Tratado Shebiit, e a Torá
só o torna obrigatório para a produção da Terra de Israel.

\paragraph{O pousio da terra durante o ano de Shabat}

Por este preceito somos ordenados a deixar de cultivar a terra durante
o sétimo ano. Ele está expresso em Suas palavras, enaltecido seja Ele,
``Mesmo no tempo de arar e ceifar descansarás''.\footnote{Êxodo 34:21.} Ele está
repetido várias vezes, como em Suas palavras ``E no sétimo ano, sábado
de descanso para a terra''.\footnote{Levítico 25:4.} Nós já mencionamos que, de
acordo com as palavras dos Sábios, benditas sejam suas memórias, a
palavra ``descanso'' (Shabaton) determina um preceito positivo. E Ele,
louvado seja, também diz ``Descansará a terra, descanso em nome do
Eterno''.\footnote{Ibid., 2.}

As normas deste preceito estão explicadas no Tratado Shebiit e a Torá
não o torna obrigatório a não ser na Terra de Israel.

\paragraph{Santificar o Ano do Jubileu (50 anos)}

Por este preceito somos ordenados a santificar o quinquagésimo ano, ou
seja, a deixar de cultivar a terra durante esse ano, assim como no Ano
de Shabat. Este preceito está expresso em Suas palavras, enaltecido seja
Ele, ``E santificareis o ano quinquagésimo'',\footnote{Levítico 25:10.} sobre as
quais se comenta: ``A Lei do Sétimo Ano é a mesma da do Jubileu''; ou
seja, as Escrituras as colocam em pé de igualdade com relação ao preceito positivo, assim como com relação ao negativo, como vou explicar.\footnote{Ver os preceitos negativos 223 e 226.}

As leis do Jubileu e do Ano de Shabat são iguais no que se refere a
deixar de cultivar a terra e a declarar sem dono tudo o que nela crescer. Essas
duas leis estão compreendidas em Suas palavras: ``E santificareis o ano
quinquagésimo''. As Escrituras explicam que sua santidade consiste em
não ter dono seus frutos e produtos, estando essa obrigação expressa em
Suas palavras: ``Porque jubileu é ele; santidade será para vós; do
campo comereis seu produto''.\footnote{Ibid. 25:12.}

O Jubileu é observado apenas na Terra de Israel, e com a condição de que
cada tribo permaneça em seu próprio lugar, ou seja, que cada uma
permaneça no seu território da Terra de Israel, e que não se misturem
umas com as outras.

\paragraph{Fazer soar o Shofar no décimo dia de Tishrei do Ano do Jubileu}

Por este preceito somos ordenados a fazer soar o Shofar no décimo
dia de Tishrei deste Ano,\footnote{O ano do Jubileu.} e a proclamar por
toda nossa terra a liberdade dos escravos e a liberação, sem pagamento,
de todo escravo hebreu nesse décimo dia de Tishrei. Este preceito
está expresso em Suas palavras, enaltecido seja Ele, ``E farás soar a
voz do Shofar aos dez dias do sétimo mês; no dia das expiações fareis
soar o Shofar em toda a vossa terra''\footnote{Levítico 25:9.} e em Suas
palavras ``E proclamareis liberdade em toda a terra, para todos os seus
moradores''.\footnote{Ibid., 10.}

Tem sido explicado que o Jubileu é como Rosh Hashaná no que se refere a
fazer soar o Shofar e às Bençãos. As normas relativas a fazer soar o
Shofar em Rosh Hashaná estão explicadas no Tratado Rosh Hashaná.

É sabido que a intenção de fazer soar o Shofar no ano do Jubileu é
divulgar amplamente a libertação, e é parte da proclamação, como aparece
em Suas palavras ``E proclamareis a liberdade em toda a terra''. Sua
finalidade é diferente em Rosh Hashaná, quando se faz soar o Shofar
como ``lembrança diante do Eterno'', enquanto que no Jubileu é pela
liberação dos escravos, como explicamos.

\paragraph{A devolução da terra no Ano do Jubileu}

Por este preceito somos ordenados a que todas as terras compradas
retornem aos seus proprietários nesse ano,\footnote{O ano do Jubileu.} e que
elas sejam entregues pelos compradores sem compensação monetária. Esse
preceito está expresso em Suas palavras, enaltecido seja Ele: ``Em toda
terra de vossa possessão, redenção concedereis à terra'',\footnote{Levítico
25:24.} tendo ficado claro pelas Suas Palavras: ``Neste ano do jubileu,
voltareis cada um a sua possessão''\footnote{Ibid., 13.} que o resgate deve
ocorrer nesse Ano.

As Escrituras explicam as regras detalhadas deste preceito, e deixam
claros os direitos do vendedor e do comprador caso ele queira resgatar
sua herança antes do início do Ano do Jubileu. Também deixam claro que
esta lei específica se aplica apenas às terras localizadas fora das
muralhas de uma cidade, e que aldeias e casas construídas em campo
aberto estão sob a mesma lei, uma vez que não estão dentro das muralhas,
assim como pomares e jardins. Essas são as ``casas das aldeias'' a
respeito das quais as Escrituras dizem: ``E as casas das aldeias que não
têm muro ao redor, como os campos da terra serão consideradas; redenção
haverá para elas e no Jubileu sairão do poder do comprador''.\footnote{Ibid., 31.}

As normas deste preceito estão explicadas em Arakhin. Ele também só é
obrigatório na Terra de Israel, e apenas enquanto a lei do Jubileu
estiver em aplicação.

\paragraph{O resgate de propriedades dentro das muralhas da cidade}

Por este preceito somos ordenados a que o resgate das posses vendidas
dentro das muralhas de uma cidade seja válido apenas por um ano
completo e que depois de um ano elas se tornem propriedade do
comprador, e não sejam devolvidas no Jubileu. Este preceito está
expresso em Suas palavras, enaltecido seja Ele, ``E quando o homem
vender casa de moradia numa cidade murada''.\footnote{Levítico 25:29.}

Este preceito é a ``lei das casas numa cidade murada''. Suas normas
estão explicadas no Tratado Arakhin, e ele é obrigatório apenas na Terra
de Israel.

\paragraph{Contar os anos até o Jubileu}

Por este preceito somos ordenados a contar os anos a partir do momento
em que conquistamos a terra e nos tornamos seus proprietários, uma
contagem feita em ciclos de sete anos, até o Ano do Jubileu. Este
preceito, ou seja, a contagem dos anos dos ciclos Sabáticos, é de
responsabilidade do Tribunal, ou seja, do Grande Sanhedrin, o qual deve
contar os cinquenta anos, ano a ano, assim como cada um de nós tem que
contar os dias do ``omer''. Este preceito está expresso em Suas
palavras, enaltecido seja Ele, ``Contarás para ti sete semanas de anos''.\footnote{Levítico 25:8.}

A Sifrá diz: ``Poder-se-ia pensar que se poderiam contar os sete anos de
Shabat sucessivos e proclamar o Jubileu. Por isso a Torá diz `Sete
vezes sete anos'. Esses são dois versículos e a lei só pode ser
compreendida através dos dois juntos''. Quer dizer, a maneira como este
preceito deve ser executado só pode ser entendida através dos dois
versículos: o Sanhedrin tem que contar os anos e os ciclos de Shabat ao
mesmo tempo.

Uma vez que as Escrituras dizem que a lei só pode ser entendida através
dos dois versículos, conclui-se que, definitivamente, só há um preceito;
porque se houvesse dois preceitos --- um para a contagem dos anos e
outro para a contagem dos ciclos sabáticos --- não haveria razão para
dizer ``A não ser através dos dois versículos juntos'', porque dois
preceitos são sempre derivados cada um de um versículo, e a expressão
``A não ser através dos dois versículos juntos'' só é usada com relação
a um único preceito, cujas normas só podem
ser compreendidas por completo através de dois textos. Um exemplo disso
é o caso do primogênito, onde as Escrituras dizem ``Todo o que abre a
matriz será para mim'',\footnote{Exodo 34:19.} ensinando-nos que todo
primogênito, macho ou fêmea, pertence ao Eterno; o versículo
``Separarás\ldots{} macho, para o Eterno''\footnote{Ibid., 13:12.} nos ensina que
todos os machos, sejam primogênitos ou não, pertencem ao Eterno; e a
partir desses dois versículos concluímos o significado do preceito ---
que ele se aplica apenas ao primogênito macho --- como está explicado na
Mekhiltá.

\paragraph{Cancelar as dívidas no Ano de Shabat}

Por este preceito somos ordenados a cancelar todas as nossas
reclamações de dinheiro no Ano de Shabat. Ele está expresso em Suas
palavras, enaltecido seja Ele, ``O que tiveres em poder de teu irmão, o
deixarás''\footnote{Deuteronômio 15:3.} e em Suas palavras ``Este é o modo do
ano sabático: que todo o credor, que emprestou a seu companheiro, o
deixará''.\footnote{Ibid., 2.}

A Tosseftá diz: ``As Escrituras falam de dois tipos de desistência: a
desistência de terra e a desistência de dinheiro''.

A Torá ordena a desistência de dinheiro apenas quando a lei referente
à desistência de terras estiver em vigência, e nesse momento ela a torna
obrigatória em todo lugar.

As normas deste preceito estão explicadas no último capítulo do Tratado
Shebiit.

\paragraph{Cobrar as dívidas dos idólatras}

Por este preceito somos ordenados a cobrar as dívidas do idólatra e a
pressioná-lo para que pague, da mesma forma que somos ordenados a ser
misericordiosos para com o israelita e proibidos de exigir o pagamento
dele.\footnote{Ver o preceito negativo 234.} Ele está expresso em Suas palavras,
enaltecido seja Ele, ``Do estrangeiro idólatra reclamarás'',\footnote{Deuteronômio, 15:3.} a respeito das quais diz o Sifrei: ```Do
estrangeiro idólatra reclamarás' é um preceito positivo''.

\paragraph{A parte do Cohen de cada animal puro que se abate}

Por este preceito somos ordenados a dar ao Cohen o quarto
dianteiro, as duas faces e o estômago de todo animal puro que
abatermos. Este preceito está expresso em Suas palavras, enaltecido
seja Ele, ``E este será o direito dos Cohanim sobre o povo: os que
oferecerem sacrifício, seja boi ou cordeiro''.\footnote{Deuteronômio 18:3.}

As normas deste preceito estão explicadas no décimo capítulo de Hulin;
ele não é obrigatório para os Levitas.

\paragraph{A primeira tosquia deve ser dada ao Cohen}

Por este preceito somos ordenados a separar a primeira tosquia e dá-la
ao Cohen. Este preceito está expresso em Suas palavras, enaltecido
seja Ele, ``A primícia da tosquia de tuas ovelhas, darás a ele''\footnote{Deuteronômio, 18:4.} e é obrigatório apenas na Terra de Israel. Suas
normas estão explicadas no décimo primeiro capítulo de Hulin.

\paragraph{As coisas consagradas}

Por este preceito somos ordenados quanto à lei das coisas consagradas,
ou seja, aquele que consagrar alguma coisa que lhe pertence, dizendo:
``Seja isto consagrado, deve entregá-la ao Cohen, a menos que ele
acrescente explicitamente ``a Deus'', e nesse caso ele deve entregá-la
para ser guardada no Santuário, pois tudo o que for declarado
consagrado, em termos gerais, pertence ao Cohen. Este preceito está
expresso em Suas palavras, enaltecido seja Ele, ``No entanto, toda
consagração que uma pessoa fizer ao Eterno, de tudo o que lhe pertencer,
seja homem, ou animal etc.''.\footnote{Levítico 27:28.}

Suas palavras ``E será o campo, quando sair livre no Jubileu, santidade
ao Eterno, como campo consagrado para o Cohen; a possessão dele
pertencerá aos Cohanim''\footnote{Ibid., 21.} mostram que todas as coisas
consagradas em termos gerais pertencem ao Cohen.

As normas deste preceito estão explicadas no oitavo capítulo de
Arakhin, e no início de Nedarim.

\paragraph{``Shehitá''}

Por este preceito somos ordenados a matar os animais de uma maneira
determinada, antes de comer sua carne, pois só assim ela será alimento
permitido. Este preceito está expresso em Suas palavras, enaltecido seja
Ele, ``Poderás degolar do teu gado, e do teu rebanho, \ldots{} como te
ordenei,\footnote{Deuteronômio 12:21.} a respeito das quais diz o Sifrei:
```Poderás degolar': tal como as ofertas consagradas devem ser abatidas
de uma maneira determinada, assim também os animais abatidos como
alimento devem ser abatidos dessa maneira. `Como te ordenei' nos ensina
que Moisés foi ordenado quanto ao esôfago e à traqueia, e quanto à maior
parte de um deles\footnote{Do esôfago ou da traqueia.} nos pássaros e à maior parte de
ambos no gado''.

Todas as normas e leis sobre este preceito estão explicadas no Tratado
que lida especificamente com este assunto, que é o Tratado Hulin.

\paragraph{Cobrir o sangue de pássaros e animais abatidos}

Por este preceito somos ordenados a cobrir o sangue de um pássaro ou
animal depois de abatido. Este preceito está expresso em Suas palavras,
enaltecido seja Ele, ``Derramará o seu sangue e o cobrirá com pó''.\footnote{Levítico 17:13.}

As normas deste preceito estão explicadas no sexto capítulo de Hulin.

\paragraph{Liberar a mãe quando se pegarem seus filhotes}

Por este preceito somos ordenados a deixar partir do ninho. Ele está
expresso em Suas palavras, enaltecido seja Ele, ``Mas deixarás ir
livremente a mãe, e os filhos poderás tomar para ti''.\footnote{Deuteronômio
22:7.}

As normas deste preceito estão explicadas no último capítulo de Hulin.

\paragraph{Procurar os sinais de pureza determinados no gado e nos animais}

Por este preceito somos ordenados a procurar certos sinais em animais
domésticos e selvagens, ou seja, que eles ruminem o alimento e tenham o
casco totalmente fendido, pois isso faz deles alimento permitido.
Procurar esses sinais nos animais é um preceito positivo, expresso em
Suas palavras, enaltecido seja Ele, ``Estes são os animais que
comereis''.\footnote{Levítico 11:2.}

A Sifrá diz: ```Esses comereis':\footnote{Ibid., 3.} apenas `esses' podem ser
comidos, e não os animais impuros''; quer dizer, todo animal que tiver
esses sinais é alimento permitido, e o animal que não os tiver é
proibido. Temos aqui um preceito negativo derivado de um preceito
positivo, que tem força de um preceito positivo, de acordo com o
princípio que explicamos. É por essa razão que depois dessa sentença a
Sifrá diz: ``Sei apenas que há um preceito positivo; de que modo eu
concluo que também há um preceito negativo? Porque o Talmud diz: `O camelo, que rumina e não tem casco fendido'\,'',\footnote{Ibid., 4.}
como explicarei nos preceitos negativos.\footnote{Ver o preceito negativo 172.}

Portanto, ficou claro que Suas palavras ``Esses comereis'' são um preceito positivo, cujo significado é, como expliquei, que somos obrigados
a procurar esses sinais em todo animal, seja ele doméstico ou selvagem,
e só então ele pode ser comido. O que o preceito prescreve é a
observação deste procedimento.

As normas deste preceito estão explicadas nos Tratados Bekhorot e Hulin.

\paragraph{Procurar os sinais de pureza determinados nos pássaros}

Por este preceito somos ordenados a procurar os sinais nos pássaros,
pois apenas alguns deles são alimento permitido. No caso dos pássaros,
os sinais não estão estipulados na Torá, mas foram obtidos através de
estudo. Quando examinamos todos os tipos declarados individualmente
proibidos, encontramos certos elementos comuns a todos eles e esses são
os sinais dos pássaros impuros.\footnote{Como regra geral, aves de rapina e pássaros selvagens são impuros.
Ver, a esse respeito, Hilkhot Maakhalot Asvrot, 1º capítulo, lei 16.} Este preceito, de examinar os pássaros e
determinar que um é puro e o outro é impuro, é um preceito positivo.

Sifrei diz: ```Toda ave pura, podereis comer',\footnote{Deuteronômio 14:11.} este
é um preceito positivo. Ficou claro, portanto, o que nós assinalamos
acima.

As normas deste preceito estão explicadas no Tratado Hulin.

\paragraph{Procurar os sinais de pureza determinados nos gafanhotos}

Por este preceito somos ordenados quanto aos sinais também nos
gafanhotos. Eles estão descritos na Torá com as seguintes palavras:
``Que tem pernas por cima dos pés''.\footnote{Levítico 11:21.}

A explicação que demos sobre o preceito anterior também é válida para
este. O versículo das Escrituras que se refere a ele é: ``De todo o
réptil alado\ldots{} deles comereis estes''.\footnote{Ibid., 21-22.}

As normas deste preceito estão explicadas no terceiro capítulo do
Tratado Hulin.

\paragraph{Procurar os sinais de pureza determinados nos peixes}

Por este preceito somos ordenados quanto aos sinais nos peixes, que
estão expressos na Torá, em Suas palavras ``Isto comereis de tudo o que
está nas águas'' etc.\footnote{Levítico 11:9.} A Guemará de Hulin diz
explicitamente: ``Aquele que come um peixe impuro viola um preceito
positivo e um preceito negativo'', já que de Suas palavras ``Isto
comereis'' eu concluo que outros peixes não devem ser comidos, e um
preceito negativo derivado de um preceito positivo tem força de preceito
positivo. Fica, portanto, claro que as palavras ``Isto comereis'' são
um preceito positivo. Isto significa, como eu disse, que somos
ordenados a decidir, baseados nesses sinais, que um peixe é alimento
permitido e outro não, como as Escrituras dizem claramente: ``E fareis
separação entre o quadrúpede puro e o impuro''.\footnote{Levítico 20:25.}

A separação só pode ser feita através dos sinais, e portanto os sinais
de cada uma das quatro categorias --- animais domésticos e selvagens,
pássaros, gafanhotos e peixes --- constituem um preceito separado e
diferente. Já mostramos que cada um deles foi considerado como um
preceito positivo.

As normas deste preceito --- a saber, o preceito referente aos sinais
nos peixes --- também estão explicadas no terceiro capítulo do Tratado
Hulin.

\paragraph{Determinar a lua nova}

Por este preceito o Enaltecido nos ordena quanto ao cálculo dos meses e
dos anos. Este é o ``preceito da Santificação da Lua Nova'' e ele está
expresso em Suas palavras, enaltecido seja Ele, ``Este mês seja, para
vós, princípio dos meses''.\footnote{Êxodo 12:2.} Para explicar isto está dito
nos escritos: ``Este testemunho será entregue a vocês''; ou seja, este preceito não é imposto a
todos, como é o caso do Shabat, da Criação, quando todas as pessoas são
obrigadas a contar seis dias, e a descansar no sétimo. Não cabe a cada
um, ao ver a lua nova, decidir que esse dia é o primeiro do mês, nem
fixar o primeiro dia do mês baseado em cálculos aprendidos, nem
intercalar um mês baseado numa primavera tardia ou em outras
considerações que mereçam ser levadas em conta. Esse dever nunca deve
ser cumprido por ninguém a não ser o Grande Tribunal e deve ser
executado na Terra de Israel e em nenhum outro lugar. Portanto, como
não existe nenhum Grande Tribunal hoje em dia, a observação cessou
atualmente entre nós porque não há nenhum Grande Tribunal, assim como
cessou a oferenda de sacrifícios porque o Templo não existe mais.

Neste ponto enganaram-se os descrentes, que aqui no Oriente são chamados
Caraítas; e até alguns Rabanitas não conseguiram perceber o significado
deste ponto e começaram a tatear na mais densa escuridão a esse
respeito. Vocês devem saber que não se permite que os cálculos que
fazemos hoje em dia, e pelos quais podemos determinar as luas novas e as
festividades, sejam feitos em algum outro lugar a não ser na Terra de
Israel; mas num caso de emergência, e na ausência dos Sábios da Terra
de Israel, foi permitido que um Tribunal, que tenha sido habilitado na Terra de Israel, intercale um mês no ano
e determine luas novas fora da Terra de Israel, como o Talmud registra
ter feito o Rabi Akiba. Esse procedimento, contudo, está repleto de grandes
dificuldades, e é sabido que quase sempre houve\footnote{Quase sempre houve Sábios na Terra de Israel.} na Terra de
Israel, e que foram eles, quando estiveram reunidos, que determinaram as luas novas e intercalaram um mês nos anos, de acordo com os métodos corretos.

Um grande e fundamental princípio de nossa fé, que não pode ser
conhecido ou entendido corretamente a não ser através de reflexão
profunda, é que quando hoje em dia, estando fora da Terra de Israel,
calculamos pela tabela de ano bissexto que temos em mãos, e
determinamos que um dia é de Lua Nova e um outro é de Festival, nós o
fazemos não com base em nosso próprio cálculo, e sim porque o Grande
Tribunal da Terra de Israel designou esse dia como o primeiro do mês, ou
como um dia de Festival. Esse dia se tornou o primeiro dia do mês ou um
dia de Festival porque eles assim o decretaram, quer tenha sido sua
decisão baseada em cálculos ou na observação. Isso está de acordo com
nossa Tradição, que interpreta o versículo ``Estas são as solenidades do
Eterno, as santas convocações que proclamareis no seu tempo
determinado''\footnote{Levítico 23:4.} como significando: ``Eu não conheço outras
solenidades a não ser essas'', ou seja, aquelas que foram declaradas
como sendo ``solenidades'', ainda que isso tenha sido feito
involuntariamente, ou sob coação, ou por engano, como a Tradição nos
diz. Hoje nós fazemos cálculos apenas para saber que dia foi fixado
pelos habitantes da Terra de Israel, pois é por este método e não pela
observação da lua nova que eles determinam e estabelecem atualmente. É
na decisão deles que confiamos e não nos nossos cálculos, que nada mais
são do que constatações. Isto deve ser bem compreendido.

Darei uma explicação adicional sobre este assunto. Suponhamos, por
exemplo, que os habitantes da Terra de Israel desaparecessem --- que
Deus nos livre de tal coisa, pois Ele nos prometeu que não tirará nem
apagará da terra o remanescente da nação --- e que não houvesse mais
Tribunal lá, e que fora da Terra não houvesse mais nenhum Tribunal que
tivesse sido habilitado na Terra de Israel: nesse caso nossos cálculos
não nos seriam de nenhuma utilidade porque não devemos fazer cálculos fora da Terra, nem intercalar ou fixar luas novas, a não
ser sob as condições mencionadas, como explicamos, pois ``De Tzion sairá
a Lei''.\footnote{Isa. 2:3.} Se alguém em sã consciência examinar o
que o Talmud diz a este respeito, chegará à conclusão de que nossa
interpretação está correta e não deixa dúvidas.

Nas escrituras há indicações que estabelecem um fundamento para
os princípios em que nos baseamos para reconhecer as luas novas e os
anos bissextos. Baseado num desses trechos, mais especificamente, `` E
guardarás este estatuto em seu prazo, de ano em ano'',\footnote{Êxodo 13:10.} foi
dito:\footnote{Foi dito na Mekhiltá.} Isto nos
ensina que não devemos acrescentar nenhum mês a não ser na época do ano
próxima às solenidades''. Foi dito ainda: ``De que modo concluímos que
só devemos acrescentar um dia ao mês ou santificar a lua nova durante o dia?
Pelas palavras das Escrituras `mi-yamim yamima', a duração de um ano em
`dias'\,''. Sobre as palavras das Escrituras ``Este mês seja para vós
princípio dos meses''\footnote{Ibid. 12:2.} foi dito: ``Você calcula um ano pelos
meses mas não pelos dias'', significando que o que se deve acrescentar é
um mês completo. Também foi dito, a respeito das palavras ``Porém um
mês'':\footnote{Números 11:20.} ``Você calcula um mês pelos dias, não pelas
horas''; e também foi explicado que o versículo ``Estejas alerta desde
antes que chegue o mês da primavera''\footnote{Deuteronômio 16:1.} nos ensina que
em nossos anos devemos levar em consideração as estações do ano, e que
portanto eles devem ser anos solares.

Todas as normas deste preceito estão explicadas na íntegra no primeiro
capítulo de Sanhedrin, no Tratado Rosh Hashaná, e também em Berakhot.

\paragraph{Descansar no Shabat}

Por este preceito somos ordenados a descansar no Shabat. Ele está
expresso em Suas palavras ``E no sétimo dia descansarás'',\footnote{Êxodo 34:21.}
e está repetido várias vezes; o Enaltecido nos diz que descansar de todo
trabalho é uma obrigação aplicável a nós, a nosso gado, e a nossos
empregados.

As normas deste preceito estão explicadas no Tratado Shabat e no Tratado
Yom Tob.

\paragraph{Proclamar a santidade do Shabat}

Por este preceito somos ordenados a recitar determinadas palavras no
início e no final do Shabat, mencionando a grandeza e a alta nobreza do
dia, e a diferença desse dia com relação aos dias da semana que o
precedem e os que o sucedem. Este preceito está expresso em Suas
palavras, enaltecido seja Ele, ``Estejas lembrado do dia de sábado para
santificá-lo'':\footnote{Êxodo 20:8.} ou seja, comemorá-lo proclamando sua
santidade e sua grandeza. Este é o preceito de ``Kidush'',
santificação. A Mekhiltá diz: ```Estejas lembrado do dia de sábado para
santificá-lo': santificá-lo com uma bênção''. E os Sábios dizem
explicitamente: ``Recorda-o sobre o vinho'', e também: ``Santifica-o na
sua chegada e na sua partida'', referindo-se à ``habdala'', que faz
parte de recordar o Shabat como nos foi ordenado.

As normas deste preceito estão explicadas no final de Pessachim e em
vários trechos de Berakhot e de Shabat.

\paragraph{Retirar o fermento}

Por este preceito somos ordenados a remover o fermento de nossas
propriedades no décimo quarto dia de Nissan. Este é o preceito da
Retirada do Fermento, e ele está expresso em Suas palavras, enaltecido seja Ele,
``Mas no primeiro dia cessareis de ter fermento em vossas casas''.\footnote{Êxodo 12:15.}
Os Sábios também o chamam de ``a queima do pão'' (bi'ur), ou seja, o ato
de queimar o pão fermentado. A Guemará de Sanhedrin no Talmud de
Jerusalém diz: ``O pão fermentado envolve ambos um preceito positivo e um negativo: o
preceito positivo é queimá-lo --- `Cessareis de ter fermento em vossas
casas' --- e o negativo está em `Levedura não será encontrada em vossas casas'\,''.\footnote{Ibid., 19.}

As normas deste preceito estão explicadas no início de Pessachim.

\paragraph{Narrar o êxodo do Egito}

Por este preceito somos ordenados a narrar a história do Êxodo do Egito,
com toda a eloquência de que formos capazes, na véspera do décimo quinto
dia de Nissan. Deverá ser elogiado aquele que discorrer sobre este tema,
contando a miséria que nos impuseram os Egípcios e os sofrimentos que
eles nos causaram, e sobre a maneira como o Eterno Se vingou deles,
agradecendo-Lhe, enaltecido seja Ele, por todo o bem com que Ele nos
brindou; como foi dito, ``Todos aqueles que narram longamente a fuga do
Egito merecem ser louvados.''

Nas Escrituras este preceito está expresso em Suas palavras, enaltecido
seja Ele, ``E anunciarás a teu filho naquele dia''.\footnote{Êxodo 13:8.} Sobre
isto foi comentado: ```E anunciarás a teu filho': poder-se-ia pensar que
se deve contar a história do primeiro dia do mês em diante; por isso a
Torá diz `Naquele dia'. As palavras `Naquele dia' poderiam ser
interpretadas como significando durante o dia; por isso a Torá diz
`Por isto' --- uma expressão que não seria usada a não ser no momento em
que o pão ázimo e as ervas amargas estivessem diante de você''.
Portanto, é uma obrigação contá-la somente depois do anoitecer.

A Mekhiltá diz: ``Como foi dito que `E será quando te perguntar teu
filho amanhã etc.'\footnote{Ibid., 14.} poder-se-ia pensar que você deve narrar a
história a seu filho apenas se ele lhe perguntar. Por isso as Escrituras
dizem: `Anunciarás a teu filho' --- mesmo que ele não pergunte.
Novamente poder-se-ia pensar que isto só se aplica àquele que tiver um
filho; de que forma concluímos que ele se aplica também a quem estiver
só ou entre pessoas estranhas? Pelas palavras das Escrituras: `E disse
Moisés ao povo: Recordai este dia'\,'';\footnote{Ibid., 3.} ou seja, Deus nos
mandou que recordassemos, tal como ele nos ordenou com as palavras
``Estejas lembrado do dia de sábado para santificá-lo''.\footnote{Ibid., 20:8.}

Você já está familiarizado com as palavras: ``Mesmo se fôssemos todos
eruditos, homens com conhecimentos, versados na Lei, ainda assim seria
nossa obrigação narrar a fuga do Egito''.

As normas deste preceito estão explicadas no final de Pessachim.

\paragraph{Comer pão ázimo na véspera do décimo quinto dia de Nissan}

Por este preceito somos ordenados a comer pão ázimo na véspera do décimo
quinto dia de Nissan, quer tenha o cordeiro de Pessach sido oferecido ou não. Ele está expresso em Suas palavras, enaltecido seja Ele,
``Pela noite, comereis pães ázimos'',\footnote{Êxodo 12:18.} sobre as quais
comentam: ```Pela noite, comereis pães ázimos': as Escrituras apresentam
isto como uma obrigação''.

Está explicado em Pessachim que comer pão ázimo é obrigatório na primeira
noite (da festividade), e opcional depois disso.\footnote{Não é obrigatório comer-se pão ázimo nos outros dias de Pessach, mas é proibido comer-se pão levedado.}

As normas desse preceito estão explicadas no Tratado Pessachim.

\paragraph{Descansar no primeiro dia de Pessach}

Por este preceito somos ordenados a descansar no primeiro dia de
Pessach. Ele está expresso em Suas palavras, enaltecido seja Ele, ``O
primeiro dia, de santa convocação será para vós''.\footnote{Levítico 23:7.}
Inicialmente, você deve saber que toda vez que o Eterno ordenou uma
santa convocação'' isso é interpretado como significando que o dia deve ser santificado, o que quer dizer que nenhum tipo de trabalho deve ser feito nele, a não ser o que se relaciona com comida, como está explicado nas Escrituras.\footnote{Em Êxodo 12:16.}

Nós já nos referimos ao fato de que foi dito que a palavra
``Shabaton'', ``descanso solene'', constitui um preceito positivo como
se Ele tivesse dito ``descansa'' ou ``descansarás'', que são maneiras de
impor a abstenção do trabalho. Todos os dias das ``épocas
determinadas'', ou seja, dos Festivais, são chamados de ``Sábados do
Eterno''.\footnote{Levítico 23:38.}

Está afirmado em vários trechos do Talmud que ``há um preceito positivo
e um preceito negativo com relação aos Festivais''; quer dizer,
abster-se do trabalho durante todos os Festivais é um preceito positivo,
e executar trabalhos proibidos durante um Festival é um preceito
negativo. De acordo com isso, todo aquele que fizer um trabalho
proibido estará violando ambos um preceito positivo e um negativo.

As normas deste preceito, ou seja, o de descansar, estão explicadas no
Tratado Yom Tob.

\paragraph{Descansar no sétimo dia de Pessach}

Por este preceito somos ordenados a descansar no sétimo dia de
Pessach. Ele está expresso em Suas palavras, enaltecido seja Ele,
``O sétimo dia, de santa convocação é''.\footnote{Levítico 23:8.}

\paragraph{Contar o ``omer''}

Por este preceito somos ordenados a contar o ``omer''. Ele está
expresso em Suas palavras, enaltecido seja Ele, ``E contareis para vós
desde o dia seguinte ao primeiro dia festivo, desde o dia em que
tiverdes trazido o `omer' da movimentação; sete semanas completas
serão''.\footnote{Levítico 23:15.}

Você deve saber que da mesma forma que o Tribunal é obrigado a contar os
anos do Jubileu, ano por ano, e ciclo de Shabat por ciclo de Shabat,
como expliquei anteriormente, assim também nós somos obrigados a contar
os dias do ``omer'', dia por dia, e semana por semana, como está
determinado em Suas palavras ``Contareis cinquenta dias''\footnote{Levítico
23:16.} e ``Sete semanas contarás para ti''.\footnote{Deuteronômio 16:9.}

Assim como contar os anos e os ciclos de Shabat constitui um único
preceito, também no caso do ``omer'' um só preceito determina a
contagem. Todos os meus antecessores o contaram corretamente como um
preceito; e você não deve se deixar confundir pelas palavras: ``É
obrigatório contar os dias, e é obrigatório contar as semanas'' e
considerar que há dois preceitos separados, porque quando o preceito tem
várias partes, o cumprimento de cada uma delas é um preceito. Contudo,
haveria dois preceitos se tivesse sido dito: ``Contar os dias é um
preceito, e contar as semanas é um preceito''. Isto não passará
despercebido a alguém que costuma usar as palavras no seu sentido
preciso. Se você disser ``é obrigatório fazer isto e aquilo'', isto não
significa necessariamente que essa determinada ação seja um preceito
separado.

Há uma prova clara disto no fato de que, ao contar o ``omer'', nós
enumeramos as semanas todas as noites, dizendo ``tantas semanas e tantos
dias'', enquanto que, se as semanas fossem um preceito separado,
deveríamos mencionar o número de semanas apenas nas noites em que se
completa cada semana; e nesse caso haveria duas bênçãos: uma ``Bendito
sejas, o Eterno\ldots{} que\ldots{} nos ordenastes contar os `dias' do `omer'\,''
e outra ``contar as `semanas' do `omer'\,''. Mas não é esse o caso, e o
preceito determina a contagem do ``omer'', seus dias e suas semanas, de
acordo com Seu preceito.

Este preceito não é obrigatório para as mulheres.

\paragraph{Descansar no dia de ``Shabuot''}

Por este preceito somos ordenados a descansar no dia de ``Shabuot''. Ele
está expresso em Suas palavras ``E proclamareis nesse mesmo dia, e
haverá para vós convocação de santidade''.\footnote{Levítico 23:21.}

\paragraph{Descansar no dia de ``Rosh Hashaná''}

Por este preceito somos ordenados a descansar no primeiro dia de
``Thishri''. Ele está expresso em Suas palavras ``No sétimo mês, será
para vós descanso solene''.\footnote{Levítico 23:24.}

Nós já mencionamos que foi dito que a palavra ``Shabaton'', ``descanso
solene'', constitui um preceito positivo.

\paragraph{Jejuar no dia de ``Yom Kipur''}

Por este preceito somos ordenados a jejuar no décimo dia do
``Thishri''. Ele está expresso em Suas palavras, enaltecido seja Ele,
``Afligireis vossas almas'' etc.,\footnote{Levítico 16:29.} que a Sifrá
interpreta desta forma: ```Afligireis vossas almas': aflição com relação àquilo de que dependa a vida, ou seja, abstinência de comer e beber''.

A tradição também proíbe lavar-se, untar-se, usar sapatos e manter
relações conjugais e diz que devemos cessar todas essas atividades
porque, como foi dito, ``Sábado solene é para vós, e afligireis vossas
almas'',\footnote{Ibid., 31.} o que equivale a dizer que é obrigatória a
abstinência tanto de trabalho de toda espécie como de alimentação e
cuidados com o corpo, e é por isso que se usa a expressão ``Shabat
Shabaton'' --- ``um Shabat de descanso solene''.

A Sifrá diz: ``De que forma concluímos que é proibido lavar-se, untar-se
e manter relações conjugais em Yom Kipur? Pelas palavras da Sifrá:
`Sábado solene'; ou seja, devemos abster-nos de todas essas coisas a
ponto de que elas nos aflijam''.

\paragraph{Descansar no dia de ``Yom Kipur''}

Por este preceito somos ordenados a descansar do trabalho de todos os
tipos nesse dia. Ele está expresso em Suas palavras: ``Shabat Shabaton é
para vós''.\footnote{Levítico 16:31.}

Já explicamos várias vezes que foi dito que a expressão ``Shabaton'',
``descanso solene'', constitui um preceito positivo.

\paragraph{Descansar no primeiro dia de ``Sucot''}

Por este preceito somos ordenados a descansar no primeiro dia da Festa
dos Tabernáculos. Ele está expresso em Suas palavras ``No primeiro dia
haverá santa convocação''.\footnote{Levítico 23:35.}

\paragraph{Descansar no dia de ``Shemini Atzeret''}

Por este preceito somos ordenados a descansar no oitavo dia da Festa
dos Tabernáculos. Ele está expresso em Suas palavras, enaltecido seja
Ele, ``No oitavo dia, haverá santa convocação para vós''.\footnote{Levítico 23:36.}

Você deve saber que a mesma lei se aplica a cada um dos seis
dias,\footnote{I. e., o primeiro e o sétimo dias de Pessach, a Festa das
Semanas, Rosh Hashaná, e o primeiro e o oitavo dias dos Tabernáculos.} durante os quais somos obrigados a
descansar, e que nenhum deles está sujeito a algum tipo de restrição que não se aplique aos outros. Também podemos preparar comida em cada um deles. Portanto, as mesmas regras com relação ao ``descanso'' se aplicam a todos os Festivais. Todas as normas a esse
respeito estão explicadas no Tratado Yom Tob.

Contudo, deve ser ressaltado que o descanso imposto em Shabat e em ``Yom
Kipur'' acarreta todas as abstinências e muitas outras mais, já que
nesses dois dias não podemos preparar comida. Há outras coisas que nos
são permitidas num dia de Festival e que nos são proibidas no Shabat,
embora não estejam relacionadas com a preparação de comida, como está
explicado no Tratado Yom Tob.

\paragraph{Morar numa cabana durante os dias de ``Sucot''}

Por este preceito somos ordenados a morar numa cabana por sete dias,
durante toda a Festa. Ele está expresso em Suas palavras, enaltecido
seja Ele, ``Nas cabanas habitareis por sete dias''.\footnote{Levítico 23:42.}

As normas deste preceito estão explicadas no Tratado Sucá. Ele não é
obrigatório para as mulheres.

\paragraph{Pegar um Lulav no ``Sucot''}

Por este preceito somos ordenados a pegar um ramo de palmeira e a
alegrar-nos com ele diante do Eterno durante sete dias. Este preceito
está expresso em Suas palavras, enaltecido seja Ele, ``E tomareis para
vós, no primeiro dia, (o fruto da árvore formosa, palmas de palmeira, e
ramos de murta e de salgueiro de ribeiras, e vos alegrareis diante do
Eterno vosso Deus, por sete dias.)''.\footnote{Levítico 23:40.}

As normas deste preceito estão explicadas no terceiro capítulo do
Tratado Sucá. Lá está explicado que é apenas no Santuário que este
preceito é obrigatório por sete dias; nos outros lugares ele é
obrigatório apenas no primeiro dia, de acordo com a Torá. Ele não é
obrigatório para as mulheres.

\paragraph{Ouvir o Shofar no dia de ``Rosh Hashaná''}

Por este preceito somos ordenados a ouvir o som do Shofar no
primeiro dia de Tishrei. Ele está expresso em Suas palavras ``Dia de
toque do Shofar, será para vós''.\footnote{Números 29:1.}

As normas deste preceito estão explicadas no Tratado Rosh Hashaná. Ele
não é obrigatório para as mulheres.

\paragraph{Dar meio shekel anualmente}

Por este preceito somos ordenados a dar meio
shekel\footnote{Moeda de prata.} todos os anos.\footnote{Ao Santuário.}
Ele está expresso em Suas palavras, enaltecido seja Ele: ``Dará cada um
o resgate de sua alma ao Eterno'',\footnote{Êxodo 30:12.} e ``Isto dará''.\footnote{Ibid., 13-14.} É claro que este preceito não é obrigatório para as mulheres,
porque as Escrituras dizem: ``Cada um que passa para o número dos que
são contados''.\footnote{E o senso militar não incluía mulheres.}

As normas deste preceito estão explicadas no Tratado que lida
especificamente com este assunto, a saber, o Tratado Shekalim. Lá está
explicado que o preceito é obrigatório apenas durante a existência do
Santuário.

\paragraph{Acatar o que dizem os profetas}

Por este preceito somos ordenados a ouvir todo profeta, que a paz esteja
com eles, e a fazer o que quer que ele ordene, mesmo que isso seja contrário a um ou mais preceitos, desde que isso seja temporário e que não
represente uma adição ou subtração permanente, como explicamos na
introdução de nosso \emph{Comentário sobre a Mishné}. O versículo das
Escrituras pelo qual Ele nos impõe isto, enaltecido seja Ele, é: ``A ele
ouvireis'',\footnote{Deuteronômio 18:15.} sobre o qual o Sifrei diz: ```A ele
ouvireis': ainda que ele lhe diga para violar temporariamente um dos
preceitos impostos pela Torá, você deve atendê-lo''. Todo aquele que
transgredir este preceito está sujeito à pena de morte pela mão dos
Céus, como estipulado em Suas palavras, enaltecido seja Ele, ``E
qualquer homem que não ouvir as minhas palavras, que ele falar em Meu
Nome, Eu lhe pedirei contas''.\footnote{Ibid., 19.}

Está dito em Sanhedrin: ``Três pessoas estão sujeitas à morte pela mão
dos Céus: aquele que desobedece a um profeta, um profeta que desobedece
a seu próprio preceito, e o que oculta sua profecia''. Tudo isto se
deduz das palavras ``E qualquer homem que não ouvir (`lo yishma'), as
Minhas palavras etc.'', lendo-se ``lo yishma'' também como ``lo
yishama'', ``não obedecerá'', aplicável ao profeta que desobedecer a
seu próprio preceito, e como ``lo yashmia'', ``não se fará ouvir'',
aplicável ao profeta que ocultar sua profecia.

As normas deste preceito estão explicadas no final de Sanhedrin.

\paragraph{Nomear um rei}

Por este preceito somos ordenados a nomear um rei sobre nós, um
Israelita, que unirá toda a nossa nação e será nosso líder. Este
preceito está expresso em Suas palavras, enaltecido seja Ele,
``Poderás, certamente, pôr sobre ti o rei''.\footnote{Deuteronômio 17:15.}

Nós já nos referimos às palavras do Sifrei: ``Três preceitos foram
impostos aos Israelitas para quando eles chegassem à Terra de Israel:
nomear um rei para si mesmos, construir o Santuário, e aniquilar os
descendentes de Amalec''.

O Sifrei diz ainda: ```Poderás, certamente, pôr sobre ti o rei' é um dos
preceitos positivos'', e explica que isso significa que ele deve ser
temido, e que nosso respeito por ele e estima pela sua grandeza e
supremacia devem ser tão grandes que o coloquem num nível de honra superior ao de todos
os profetas de sua geração. O Talmud diz explicitamente: ``O Rei tem
prioridade sobre o profeta''; e quando esse Rei der uma ordem que não for
conflitante com um preceito da Torá, nós devemos obedecer a seu comando, e ele tem
o direito de matar com a espada todo a ele que lhe desobedecer. Nossos
ante passados aceitaram isso sobre si mesmos ao dizerem: ``Todo aquele que se
rebelar contra teu comando\ldots{} deverá ser morto.\footnote{Josh. 1:18.} A vida de
todo aquele que se rebelar contra a autoridade real, seja ele quem for, está entregue ao rei devidamente nomeado de acordo com a Torá.

As normas deste preceito estão explicadas no segundo capítulo de
Sanhedrin, no início de Queretot, e no sétimo capítulo de Sotá.

\paragraph{Oobedecer ao Grande Tribunal}

Por este preceito somos ordenados a obedecer ao Grande Tribunal e a tudo
o que ele nos ordene com relação ao que é proibido e ao que é permitido. Quanto a isso não há diferença entre uma decisão baseada na
Tradição, uma à qual eles tenham chegado pela aplicação de uma das leis
de interpretação da Torá, e uma sobre a qual eles concordaram a fim de
delimitar alguma determinação da Lei, ou a fim de ir ao encontro de
alguma situação através de uma medida que lhes pareça correta e
calculada, e que reforce a Torá: em todos esses casos somos obrigados a
executar o que eles decidirem e a agir de acordo com suas ordens --- não
podemos desobedecer a eles. Este preceito está expresso em Suas
palavras, enaltecido seja Ele, ``Conforme o mandado da lei que te
ensinarem'',\footnote{Deuteronômio 17:11.} a respeito das quais diz o Sifrei:
```Conforme o juízo que te disserem, farás' é um preceito positivo''.

As normas deste preceito estão explicadas no final de Sanhedrin.

\paragraph{Aceitar a decisão da maioria}

Por este preceito somos ordenados a seguir a maioria caso haja uma
diferença de opinião entre os Sábios com relação a qualquer uma das leis
da Torá. Da mesma forma, se num litígio particular --- por exemplo, num
caso entre Reuben e Simeon --- surgir uma diferença de opiniões entre os
juízes da cidade quanto a ser Simeon ou Reuben o devedor, nós devemos
seguir a maioria. Este preceito está expresso em Suas palavras,
enaltecido seja Ele, ``Inclina-te à maioria''.\footnote{Êxodo 23:2.} Foi dito
explicitamente: ``A maioria é lei decisiva''.

As normas e regulamentos deste preceito estão explicados em vários
trechos de Sanhedrin.

\paragraph{Nomear juízes e oficiais do tribunal}

Por este preceito somos ordenados a nomear juízes que devem impor o
cumprimento dos preceitos da Torá, forçar aqueles que se desgarraram a
voltar a trilhar o caminho da verdade, ordenar que se execute o que é
bom e que se evite o que é ruim e aplicar as penalidades sobre os
transgressores, a fim de que os preceitos e as proibições da Torá não
fiquem entregues à vontade do indivíduo.

Uma das condições deste preceito é que esses juízes devem ser de graus
diferentes, da seguinte forma. Para cada cidade com um número
suficiente de habitantes se nomeia 23 juízes --- para constituir o
Sanhedrin Menor --- que devem se reunir todos no portão da cidade. Em
Jerusalém deve ser nomeado o Grande Tribunal, com setenta juízes, e
acima deles o Chefe da Assembleia, que também é chamado pelos Sábios de
``Nassi'', e eles devem se reunir num local especificamente designado
para eles. Numa cidade cuja população seja numericamente insuficiente
para ter um Sanhedrin Menor serão de qualquer forma nomeados três
juízes que deverão julgar os casos menores e enviar os casos importantes
para o Tribunal Superior.

Também deverão ser nomeados inspetores para visitar os mercados e
supervisionar a conduta das pessoas em suas transações, a fim de que
eles não cometam injustiças nem mesmo em assuntos insignificantes.

Este preceito está expresso em sua ordem, enaltecido seja Ele, ``Juízes
e policiais, designarás para ti, em cada uma de tuas tribos''.\footnote{Deuteronômio 16:18.} O Sifrei diz: ``De que forma sabemos que devemos
nomear um tribunal para todo povo de Israel? Pelas palavras das
Escrituras: `Juízes e policiais, designarás para ti'. De que forma
sabemos que devemos nomear um acima de todos os outros? Pelas palavras `Designarás para ti'. De que forma sabemos
que deve ser nomeado um tribunal para cada tribo? Pelas palavras `Em
cada uma das tuas portas'. Raban Shimeon ben Gamliel diz: ```Em cada uma
de tuas portas; e julgarão': é obrigatório que cada tribo tenha seu
próprio tribunal porque a Torá diz: ``e julgarão o povo'' --- mesmo
contra sua vontade''.

O preceito que nos ordena nomear setenta anciãos está repetido em Suas
palavras, enaltecido seja Ele, a Moisés, ``Ajunta-me (li) setenta homens
dos anciãos de Israel'' etc.\footnote{Números 11:16.} e disseram que toda vez
que está dito para Mim (li), isso quer dizer que é para sempre.
Portanto, ``E Me (li) servirão''\footnote{Êxodo 28:41.} significa que este
preceito é obrigatório para sempre; não é um preceito temporário, mas
sim obrigatório de geração em geração.

Você deve saber que a nomeação de todos esses tribunais, a saber, o
Sanhedrin Maior e Menor, o Tribunal de Três, assim como as outras
nomeações, só podem ter lugar na Terra de Israel, ou não terão
validade. Mas os juízes ordenados na Terra de Israel podem julgar tanto
dentro como fora dela; e esse é o significado das palavras ``O Sanhedrin
tem jurisdição dentro e fora da Terra''. Contudo, eles não podem julgar
casos de pena capital nem dentro nem fora da Terra a não ser durante a
existência do Santuário, como explicamos no início deste trabalho. A
respeito de Suas palavras, enaltecido seja Ele, --- relativas ao
homicídio acidental --- ``E serão estes para vós por estatuto de
julgamento para as vossas gerações, em todas as vossas moradas'',\footnote{Números 35:29.} o Sifrei diz: ```Em todas as vossas moradas' significa
tanto na Terra como fora dela. Poderíamos pensar que as leis sobre as
Cidades de Refúgio também são obrigatórias fora da Terra; por isso a
Torá diz `E serão estes etc.': estas leis, referentes aos Tribunais,
são obrigatórias tanto dentro como fora da Terra de Israel; as outras
referentes às Cidades de Refúgio, são obrigatórias apenas na Terra de
Israel''.

Todas as normas deste preceito estão explicadas no Tratado Sanhedrin.

\paragraph{Tratar as partes com igualdade perante a lei}

Por este preceito os juízes são ordenados a tratar com igualdade todas
as partes, e a permitir que cada um diga o que tem a dizer, quer ele
fale longa ou brevemente. Este preceito está expresso em Suas palavras
``Com justiça julgarás o teu próximo'',\footnote{Levítico 19:15.} que a Sifrá
explica da seguinte forma: ``E proibido permitir a uma pessoa que diga
tudo o que quiser e ordenar a outra que seja breve.'' Este é um dos
aspectos incluídos neste preceito.

Outro aspecto dele é que todo homem que for conhecedor da Lei é obrigado
a proceder a um julgamento se as partes tiverem começado a arguir diante
dele. Os Sábios dizem explicitamente: ``De acordo com as palavras da
Torá, até mesmo uma única pessoa tem competência para julgar casos de
dívida, pois está dito: `Com justiça julgarás o teu próximo'\,''.

Outro aspecto ainda é que um homem é obrigado a julgar seu próximo com
uma inclinação em seu favor, e a sempre interpretar seus atos e
palavras como sendo bons e caritativos.

As intenções deste preceito estão explicadas em diversos trechos do
Talmud.

\paragraph{Testemunhar no tribunal}

Por este preceito somos ordenados a dar ao Tribunal toda e qualquer
prova que tivermos, quer ela arruíne a pessoa julgada ou salve sua vida
ou seu dinheiro. Somos obrigados a prestar testemunho sobre cada aspecto
e a dizer aos juízes o que vimos ou ouvimos. Os Sábios citam como prova
da obrigação de prestar testemunho as Suas palavras, enaltecido seja
Ele, ``Sendo testemunha de um fato, por ter visto ou sabido''.\footnote{Levítico
5:1.} Aquele que violar este preceito e ocultar provas comete pecado
grave, de acordo com Suas palavras, enaltecido seja Ele, ``Se não o
denunciar, levará seu pecado''.\footnote{Ibid.}

Este é o princípio geral. Contudo, se o testemunho que ele omitir é
relativo a dinheiro, e ele o omitir sob juramento, ele será obrigado a
oferecer um Sacrifício de Maior ou Menor Valor, como determinam as
Escrituras, de acordo com as condições expostas em Shabuot.

As normas deste preceito estão expostas em Sanhedrin e em Shabuot.

\paragraph{Investigar o depoimento das testemunhas}

Por este preceito somos ordenados a investigar os depoimentos prestados
pelas testemunhas e examiná-los cuidadosamente antes de aplicar um
castigo ou apresentar uma decisão. Temos que ter a máxima cautela para
não chegar a uma conclusão mal ponderada e precipitada que venha a
prejudicar um inocente. Este preceito está expresso em Suas palavras,
enaltecido seja Ele, ``E indagarás, e investigarás, e perguntarás bem; e
se for verdade, e se for certa a coisa''.\footnote{Deuteronômio 13:15.}

As normas deste preceito e suas subdivisões --- como devem ser
conduzidas as indagações e averiguações, quão cautelosos devemos ser e
de que forma as evidências devem ser aceitas ou rejeitadas, com base nas
investigações --- estão explicadas no Tratado Sanhedrin.

\paragraph{Condenar as testemunhas que prestarem falso testemunho}

Por este preceito somos ordenados a punir as testemunhas que prestarem
falso testemunho com a pena que elas pensaram que seria aplicada pelo
seu testemunho. Este preceito está expresso em Suas palavras, enaltecido
seja Ele, ``Fareis a eles como pensavam fazer a seu irmão''.\footnote{Deuteronômio 19:19.} Esta é a Lei do Falso Testemunho: se seu
testemunho foi calculado para condenar a uma perda monetária, devemos aplicar-lhes uma perda no mesmo valor;
se foi calculado para condenar à morte, elas deverão morrer daquela
maneira;\footnote{O Sanhedrin podia condenar à morte apenas e somente de quatro
maneiras: apedrejamento, queima, decapitação e estrangulamento,
dependendo do crime cometido.}
e se foi calculado para condenar ao açoitamento, elas deverão sofrer
esse castigo.

As normas deste preceito, as dúvidas que surgiram com relação a ele, e a
maneira de provar que os testemunhos são falsos, e que estão, portanto,
sujeitos a esta lei, estão explicadas no Tratado Macot.


\paragraph{Eglá Arufá}

Por este preceito somos ordenados a quebrar o pescoço de uma vaca se
encontrarmos num campo o corpo de um homem assassinado, e se não se
souber quem foi o assassino. Este preceito está expresso em Suas
palavras, enaltecido seja Ele, ``Quando for achada uma pessoa
assassinada, caída no campo etc.''.\footnote{Deuteronômio 21:1.} Esta é a Lei de
Quebrar o Pescoço de uma Vaca. Suas normas estão explicadas no último
capítulo do Tratado Sotá.

\paragraph{Separar seis cidades de refúgio}

Por este preceito somos ordenados a separar seis Cidades de Refúgio que
estejam prontas para receber a quem matar uma pessoa involuntariamente,
e a construir estradas que levem a elas e a nivelar essas estradas, não
deixando nelas nada que possa atrapalhar o fugitivo em sua fuga. Este
preceito está expresso em Suas palavras, enaltecido seja Ele,
``Prepararás o caminho e dividirás em três partes a área de tua terra
etc.''.\footnote{Deuteronômio 19:3.}

As normas deste preceito estão explicadas em Sanhedrin, Macot, Shekalim
e Sotá. Nós já citamos do Sifrei que as leis relativas às Cidades de
Refúgio são obrigatórias apenas na Terra de Israel.

\paragraph{Designar cidades para os Levitas}

Por este preceito somos ordenados a dar aos Levitas cidades para que
habitem nelas,\footnote{Elas eram em número de 42, independentemente das seis Cidades de
Refúgio mencionadas no preceito anterior.} porque eles não receberam nenhum
pedaço da Terra. Ele está expresso em Suas palavras, enaltecido seja
Ele, ``Que deem aos Levitas\ldots{} cidades para habitar''.\footnote{Números 35:2.}

Essas cidades dos Levitas também eram usadas como Cidades de Refúgio,
oferecendo asilo sob condições especiais, como está explicado no Tratado
Macot.

\paragraph{Eliminar o perigo de nossas moradias}

Por este preceito somos ordenados a eliminar todos os obstáculos e
possibilidades de perigo dos lugares em que vivemos: ou seja, construir
muros ou parapeitos nos telhados, poços, fossos e similares, para que
ninguém caia neles ou deles. Da mesma forma, toda estrutura perigosa
deve ser reconstruída ou consertada a fim de afastar todo tipo de
perigo. Este preceito está expresso em Suas palavras ``Farás um
parapeito no teu telhado'',\footnote{Deuteronômio 22:8.}
a respeito das quais diz o Sifrei: ```Farás um parapeito no teu telhado'
é um preceito positivo''.

As normas deste preceito estão explicadas no Tratado Baba Kamma.

\paragraph{Destruir todo tipo de idolatria na Terra de Israel}

Por este preceito somos ordenados a destruir todo tipo de idolatria e
seus templos por todas as maneiras possíveis de destruição e
aniquilação: quebrar, queimar, demolir e rasgar usando, para cada
objeto, o meio apropriado para que a destruição seja feita o mais
completa e rapidamente possível, pois a intenção é que não reste nem
traço dele. Isso está expresso em Suas palavras, enaltecido seja Ele,
``Certamente destruireis dos lugares'',\footnote{Deuteronômio 12:2.} em ``Mas
assim fareis com elas: seus altares derrubareis etc.''\footnote{Ibid., 7:5.} e
novamente em ``Porém seus altares derrubareis''.\footnote{Êxodo 34:13.}

A Guemará de Sanhedrin registra casualmente que a menção de um preceito
positivo relativo à idolatria provocou a seguinte pergunta: ``Como se
pode conceber um preceito positivo com relação à idolatria?''. E o Rabi
Hisdá citou, como explicação: ``Seus altares derrubareis''.

O Sifrei diz: ``De que modo se conclui que se cortar uma
Ashera\footnote{Uma árvore ou bosque devotado à idolatria.}
e se ela tornar a crescer dez vezes, deve-se cortá-la novamente? Pelas
palavras da Torá `Certamente, destruireis'\,''. Também está dito ali:
```E fareis parecer os seus nomes daquele lugar':\footnote{Deuteronômio 12:3.} o
preceito de destruir a idolatria se aplica apenas na Terra de Israel''.

\paragraph{A lei da cidade apóstata}

Por este preceito somos ordenados a matar todos os habitantes de uma
Cidade Apóstata e a queimá-la com tudo o que houver nela. Esta é a Lei
da Cidade Apóstata, e ela está expressa em Suas palavras, enaltecido
seja Ele, ``E queimarás no fogo, a cidade e todo o seu despojo,
inteiramente''.\footnote{Deuteronômio 13:17.}

As normas deste preceito estão explicadas no Tratado Sanhedrin.

\paragraph{A guerra contra as Sete Nações Hereges}

Por este preceito somos ordenados a exterminar as Sete Nações que
habitavam a terra de Canaã,\footnote{I.e., os hiteus, os girgasheus, os emoreus, os cananeus, os periseus,
  os hiveus, os jebuseus, que eram os idólatras habitantes originais da
  Terra de Israel.} porque eles
constituíram a raiz e primeiro fundamento da idolatria. Este preceito
está expresso em Suas palavras, enaltecido seja Ele, ``Mas
destruí-los-as''.\footnote{Deuteronômio 20:17.} Está explicado em vários textos
que o objetivo disso era evitar que imitássemos sua heresia. Há vários
trechos nas Escrituras que nos incitam e insistem veementemente para que
os exterminemos, e a guerra contra eles é obrigatória.

Poder-se-ia pensar que este preceito não é obrigatório para sempre, uma
vez que as sete nações há muito deixaram de existir, mas essa ideia só
seria concebida por alguém que não tivesse compreendido a diferença
entre os preceitos que são obrigatórios através das gerações e os que não o são. Não
se pode dizer que não seja obrigatório para sempre um preceito que
tenha sido cumprido por completo --- alcançando seu objetivo --- mas
cujo cumprimento não tenha sido ligado a um limite determinado de tempo,
porque ele será obrigatório para cada geração em que surja a
possibilidade de executá-lo. Se o Eterno destruísse e exterminasse
completamente os Amalequitas --- e que isso ocorra em breve em nossos
dias, de acordo com Sua promessa, enaltecido seja Ele, ``Pois
extinguirei totalmente a memória de Amalec''\footnote{Êxodo 17:14.} ---
poderíamos então dizer que o preceito ``Apagarás a memória de Amalec''\footnote{Deuteronômio 25:19.} não seria mais obrigatório através das gerações?
Não poderíamos; o preceito é obrigatório através das gerações, e
enquanto existirem descendentes de Amalec eles deverão ser eliminados.
Da mesma forma, no caso das Sete Nações, sua destruição e exterminação é
obrigatória, assim como a guerra contra elas: temos o dever de
exterminá-las e persegui-las através de todas as gerações até que elas
sejam destruídas completamente. Assim fizemos até que sua destruição foi
completada por Davi, e seus remanescentes se dispersaram e se misturaram
com outras nações de tal forma que não restou mais nenhum traço deles.
Mas embora elas tenham desaparecido, isso não significa que o preceito
de exterminá-las não seja obrigatório para sempre --- assim como não
podemos dizer que a guerra contra Amalec não é obrigatória para sempre
--- mesmo depois de elas terem sido extinguidas e destruídas. Não há
nenhuma indicação específica de tempo ou lugar ligada a este preceito, como no caso dos
preceitos especialmente estipulados para serem cumpridos no deserto ou no Egito.
Ao contrário, ele está ligado àqueles a quem ele é imposto, e eles devem
cumpri-lo enquanto exista algum deles.\footnote{I.e., enquanto exista algum daqueles contra quem este preceito é dirigido.}

De um modo geral, é apropriado entender e discernir a diferença
entre um preceito e a ocasião a respeito da qual ele nos é ordenado. Um
preceito pode ser obrigatório para sempre, ainda que as ocasiões deixem de
existir durante um determinado período; mas a falta de ocasião não faz
com que ele deixe de ser um preceito obrigatório através das gerações.
Um preceito deixará de ser obrigatório para sempre quando a situação for
inversa, ou seja, quando tiver sido obrigação nossa executar, sob certas
condições, um determinado ato ou uma determinada ordem que não seja mais
obrigação nossa atualmente, embora essas condições ainda persistam. Um
exemplo disso é o caso do Levita idoso que foi desqualificado para o
serviço no deserto e que hoje está qualificado entre nós, como está
explicado no lugar apropriado. Você deve entender este princípio e
segui-lo à risca.

\paragraph{A extinção de Amalec}

Por este preceito somos ordenados a exterminar, dentre os descendentes
de Esaú, apenas a semente de Amalec, homens e mulheres, jovens e
velhos. Este preceito está expresso em Suas palavras, enaltecido seja
Ele, ``Apagarás a memória de Amalec''.\footnote{Deuteronômio 25:19.}

Nós já citamos do Sifrei: ``Três preceitos foram impostos aos
Israelitas quando eles entraram na Terra de Israel: nomear um rei para
si mesmos, construir o Santuário, e aniquilar os descendentes de
Amalec''. A guerra contra Amalec também é obrigatória.

As normas deste preceito estão explicadas no oitavo capítulo de Sotá.

\paragraph{Recordar os atos nefastos de Amalec}

Por este preceito somos ordenados a recordar o que Amalec nos fez quando
nos atacou sem ter sido provocado. Devemos falar nisso sempre e incitar
o povo a fazer guerra contra ele, e ordenar-lhe que o odeie, a fim de
que esse assunto não seja esquecido e que o ódio por ele não se
enfraqueça nem diminua com o passar do tempo. Este preceito está
expresso em Suas palavras, enaltecido seja Ele, ``Recorda-te do que fez
Amalec''.\footnote{Deuteronômio 25:17.} A esse respeito diz o Sifrei:
```Recorda-te do que te fez Amalec': com palavras; `Não te esquecerás':\footnote{Ibid., 19.} com o coração''; ou seja, você deve falar dessas coisas
para assegurar-se de que o ódio contra Amalec não seja afastado do
coração dos homens. E a Sifrá diz: ```Recorda-te do que te fez Amalec':
poder-se-ia pensar que isto significa `Em teu coração'. Mas `Não te
esquecerás' se refere
ao esquecimento do coração: então como se pode obedecer ao preceito de
`recordar'? Com a palavra''. Veja como o profeta Samuel cumpriu este
preceito:
primeiro ele recordou com palavras e depois deu ordens para que fossem
mortos, segundo Suas palavras ``Eu recordo o que Amalec fez a
Israel''.\footnote{I Sam. 15:2.}

\paragraph{A lei da guerra não obrigatória}

Por este preceito somos ordenados quanto a guerras não obrigatórias contra nações.\footnote{Guerras contra outras nações além daquelas contra quem somos
  ordenados a guerrear.} Caso entremos em guerra contra eles, somos obrigados
a fazer um acordo com eles para poupar suas vidas se eles fizerem as
pazes conosco e nos entregarem suas terras, e nesse caso eles deverão nos pagar
tributos e ser nossos súditos. Este preceito está expresso em Suas
palavras, enaltecido seja Ele, ``Te será tributário ou te servirá''.\footnote{Deuteronômio 20:11.} A esse respeito diz o Sifrei: ``Se eles disserem
`nós concordamos com os tributos mas recusamos a servidão', ou,
`concordamos com a servidão, mas nos recusamos a pagar os tributos', não
devemos concordar: eles devem aceitar as duas condições''. Isto
significa que eles devem pagar um tributo anual a ser determinado pelo
rei daquela ocasião, e obedecer às suas ordens com temor e humildade,
como convém aos súditos. Contudo, se eles não fizerem a paz conosco,
somos ordenados a matar toda a população masculina, jovens e velhos, e a
tomar tudo o que lhes pertence, inclusive suas mulheres. Este preceito
está expresso em Suas palavras, enaltecido seja Ele, ``E se não fizer
paz contigo etc.'';\footnote{Ibid. 12.} tudo isso está na lei da guerra não
obrigatória.

As normas deste preceito estão explicadas no oitavo capítulo de Sotá, e
no segundo capítulo de Sanhedrin.

\paragraph{Nomear um Cohen para a guerra}

Por este preceito somos ordenados a nomear um Cohen para fazer ao
povo o discurso referente à guerra, quando eles forem partir para a
luta, e para mandar de volta todo homem que não estiver apto para a batalha,
seja porque ele tem o coração fraco ou porque seus pensamentos estão
ocupados com alguma outra coisa que possa impedi-lo de se concentrar na
luta, ou seja, com uma das três coisas especificadas nas Escrituras. Só
depois disso é que eles devem entrar em luta. Esse Cohen é chamado
de ``Mashuah Mil-Hama'' para a Guerra. Em seu discurso ele deve dizer o
que está escrito na Torá e acrescentar palavras que incitem as pessoas
à guerra, e que as induzam a entregar suas vidas pelo triunfo da Fé no
Eterno, e pela punição dos ímpios que arruinam a ordem social. Este
preceito está expresso em Suas palavras, enaltecido seja Ele, ``E quando
vos aproximardes à luta, o Cohen se chegará''.\footnote{Deuteronômio 20:2.}

O Cohen então ordena que seja proclamado nas linhas de combate que
devem retornar às suas casas todos aqueles que forem fracos de coração e
os que tiverem construído uma casa e não tenham morado nela, ou que
tiverem plantado um vinhedo e não tenham comido seus frutos, ou que
tenham prometido casamento a uma mulher e não a tenham desposado. Isso
está de acordo com as palavras das Escrituras ``E falarão os policiais'',\footnote{Ibid., 5-8.} sobre as quais a Guemará diz: ``O Cohen fala, e o
policial faz ouvir suas palavras''.

Todo este procedimento --- o discurso do ``Mashuah Mil-Hama'' para a
guerra e sua proclamação entre as linhas de combate --- só é obrigatório
em caso de uma guerra não obrigatória, pois só a ela se aplica esta lei.
No caso de uma guerra obrigatória não há tal procedimento --- nem
discurso nem proclamação --- como se pode verificar no oitavo capítulo
de Sotá, onde as normas deste preceito estão explicadas.

\paragraph{Preparar um lugar separado do acampamento}

Por este preceito somos ordenados a que, quando nossas tropas forem para a guerra, devemos preparar um local fora do acampamento ao qual
eles irão,\footnote{Para fazer suas necessidades.} para que eles não o façam
indiscriminadamente em qualquer lugar ou entre as tendas, como fazem as outras nações. Este preceito está expresso em Suas palavras, enaltecido seja Ele, ``E um lugar (yad) terás para ti, fora
do acampamento, e ali sairás fora'',\footnote{Deuteronômio 23:13.} sobre as quais
o Sifrei diz: ```Yad' significa apenas um lugar, como está dito `E vede!
ele instalou um lugar (yad) para si'\,''.\footnote{I Sam. 15:12.}

\paragraph{Incluir uma estaca entre os utensílios de guerra}

Por este preceito somos ordenados a que cada homem do exército se muna
de um instrumento para cavar como parte de seus utensílios de guerra,
com a qual ele deverá cavar a terra e cobrir o excremento depois de ter
feito suas necessidades no local designado para esse fim, para que não
se veja nenhum traço dos excrementos no solo do acampamento, como Ele
ordenou no início do trecho que começa com as palavras ``Quando te
acampares contra os teus inimigos''.\footnote{Deuteronômio 23:10.} Este preceito está expresso em
Suas palavras, enaltecido seja Ele, ``E uma estaca, terás para ti, entre
os objetos de teu uso (azenecha)'',\footnote{Ibid., 14.} sobre as quais o Sifrei
diz: ```Azenecha' significa apenas o lugar de tuas armas''.

\paragraph{Um ladrão deve devolver o objeto roubado}

Por este preceito somos ordenados a fazer devolver o objeto que alguém
tenha roubado, se ele ainda existir, acrescentando um quinto de seu
valor, ou a reembolsar o seu valor, caso ele tenha sofrido alguma
alteração. Este preceito está expresso em Suas palavras, enaltecido seja Ele,
``Devolverá o que roubou''.\footnote{Levítico 5:23. A lei que diz que se deve acrescentar um quinto do valor se aplica se
  o ladrão tiver jurado em falso a esse respeito.}

Está explicado no Tratado Macot que o preceito negativo referente
ao latrocínio é um preceito negativo justaposto a um preceito positivo.
``O Misericordioso'', diz, ``ordenou `Não extorquirás',\footnote{Ibid., 19:13.} e
`Devolverá o que roubou'\,''.

As normas deste preceito estão explicadas nos últimos capítulos de Baba Kamma.

\paragraph{Tsedaká}

Por este preceito somos ordenados a dar tsedaká,\footnote{Foi usado o termo Hebraico \emph{tsedaká}, e não caridade, pois a raiz da
  palavra \emph{tsedaká} é \emph{tsedek}, ou justiça, o que é mais forte do
  que fazer uma simples caridade. O preceito ordena fazer caridade no
  sentido de fazer justiça.} a sustentar
os necessitados e a aliviar suas cargas. Este preceito está expresso de
várias maneiras em Suas palavras, como por exemplo: ``Abrirás tua mão para teu
irmão, para teu pobre'',\footnote{Deuteronômio 15:11.} e também ``Deterás sua decaída
mesmo se ele é peregrino ou estrangeiro morador da terra e viverá
contigo'',\footnote{Levítico 25:35.} e ainda ``E viverá teu irmão contigo''.\footnote{Ibid., 36.} O significado de todas essas frases é o mesmo, ou seja, que
devemos ajudar os nossos pobres e sustentá-los de acordo com suas
necessidades.

As normas deste preceito estão explicadas em vários lugares,
principalmente em Quetubot e em Baba Batra.

De acordo com a Tradição, mesmo o homem pobre que vive de tsedaká
tem a obrigação de cumprir este preceito; quer dizer, ele deve dar
tsedaká, ainda que mínima, a alguém que seja mais pobre do que ele
ou tão pobre quanto ele próprio.

\paragraph{Bonificar o servo que recobrar sua liberdade}

Por este preceito somos ordenados a bonificar um servo e a ajudá-lo
quando ele for libertado, a fim de que ele não saia de mãos vazias. Este
preceito está expresso em Suas palavras, enaltecido seja Ele,
``Carrega-lo-ás, fornecendo-lhe do teu rebanho, de tua eira, e do teu
depósito de vinho; e do que te abençoou o Eterno, teu Deús, lhe darás''.\footnote{Deuteronômio 15:14.}

As normas deste preceito estão explicadas no primeiro capítulo de Kidushin.

\paragraph{Emprestar dinheiro aos pobres}

Por este preceito somos ordenados a emprestar ao homem pobre para
ajudá-lo a aliviar sua situação. Esta é uma obrigação maior e de maior
peso do que a tsedaká porque o mendigo, cuja necessidade o obriga a
pedir esmola abertamente, não sofre tão grande angústia quanto aquele
que nunca teve que fazer isso e que precise de ajuda para que sua
pobreza não seja descoberta. Este preceito está expresso em Suas
palavras, enaltecido seja Ele, ``Se emprestares dinheiro a Meu povo, ao
pobre que está contigo''.\footnote{Êxodo 22:24.}

A Mekhiltá diz: ``Todo `se' na Torá implica uma opção, com exceção de
três deles, um dos quais está no versículo `Se emprestares dinheiro a
Meu povo'\,''. ``Se emprestares dinheiro'', dizem os Sábios, ``envolve uma
obrigação. Caso você questione isto, e sugira que esta seja uma simples
permissão, as Escrituras dizem mais adiante `Lhe emprestarás o
suficiente para o que lhe faltar',\footnote{Deuteronômio 15:8.} o que é uma
obrigação, não simplesmente uma questão de opção''.

As normas deste preceito também estão explicadas em vários trechos de
Quetubot e Baba Batra.

\paragraph{Cobrar juros do idólatra}

Por este preceito somos ordenados a exigir juros do dinheiro que
emprestarmos a um idólatra, de maneira a não ajudá-lo, nem ser amável
com ele, mas ao contrário, prejudicá-lo, mesmo se lhe fizermos um
empréstimo, o
que nos é proibido fazer no caso de um filho de Israel. Este preceito
está expresso em Suas palavras, enaltecido seja Ele, ``Do estrangeiro poderás
cobrar juros'',\footnote{Deuteronômio 23:1.} e de acordo com a interpretação
tradicional este
é um preceito positivo e não uma questão de opção.\footnote{Aparentemente o autor traduz este versículo como significando ``Você
  emprestará com juros''.}
Isto é o que diz o Sifrei a respeito: ```Do estrangeiro poderás cobrar juros' é um preceito positivo;
`E a teu irmão não pagarás'\footnote{Ver o preceito negativo 235.} é um preceito
negativo''. Neste preceito também
há certas condições estipuladas pelos Sábios que estão explicadas no
Tratado Baba Metzia.

\paragraph{Devolver o penhor ao proprietário necessitado}

Por este preceito somos ordenados a devolver um penhor ao seu
proprietário israelita no momento em que ele o necessitar. Quando o
penhor for algo de que ele precise durante o dia --- como, por exemplo,
as ferramentas de seu trabalho ou ocupação --- ele lhe deve ser
devolvido para uso durante o dia, e guardado em caução durante a noite;
se for algo de que ele precise à noite --- como, por exemplo roupa de
cama ou roupas com as quais ele durma --- ele deve ser devolvido para ser usado à noite, e guardado em caução
durante o dia. A Mekhiltá diz: ```Até pôr-se o sol a devolverás'\footnote{Êxodo
22:25} se refere a uma vestimenta usada durante o dia, que deve ser
devolvida para o dia inteiro. De que forma concluímos que uma vestimenta
usada durante a noite deve ser devolvida para a noite inteira? Pelas
palavras das Escrituras: `Restituir-lhe-ás o penhor ao pôr do sol'.\footnote{Deuteronômio 24:13.} Assim, conclui-se que uma vestimenta de dia deve
ser guardada como penhor durante a noite e devolvida para ser usada
durante o dia, e que uma vestime noturna deve ser guardada como penhor
durante o dia e devolvida para ser usada à noite.


Está explicado na Guemará de Macot que Suas palavras ``Não entrarás em sua casa para lhe tomar o seu penhor''\footnote{Ibid., 10. Ver o preceito negativo 239.} contêm um preceito
negativo justaposto a um preceito positivo, estando este últi mo
expresso em Suas palavras ``Restituir-lhe-ás''. E o Sifrei diz: ``As palavras
`Restituir-lhe-ás' nos
ensinam que o que é usado durante o dia deve ser devolvido para o dia, e
o que é usado durante a noite deve ser devolvido para a noite: um
cobertor durante a noite, e um arado durante o dia''.

As normas deste preceito estão explicadas no nono capítulo de Baba
Metzia.

\paragraph{Pagar os soldos no dia}

Por este preceito somos ordenados a pagar a diária do trabalhador no
mesmo dia, e não adiar o pagamento para outro dia. Este preceito está
expresso em Suas palavras, enaltecido seja Ele, ``No seu dia, lhe
pagarás a sua diária''.\footnote{Deuteronômio 24:15.} De acordo com as normas
deste preceito, um operário que trabalha de dia pode requerer seu
pagamento a qualquer momento da noite, e um trabalhador noturno, a
qualquer momento do dia, como explicarei nos preceitos
negativos.\footnote{Ver o preceito negativo 238.}

As normas deste preceito estão explicadas no nono capítulo do Tratado
Baba Metzia, onde fica claro que ele é obrigatório no caso dos
trabalhadores diaristas, gentis ou Israelitas, e que é um preceito
positivo pagar no momento certo.

\paragraph{Um empregado deve poder comer daquilo com que ele trabalha}

Por este preceito somos ordenados a permitir que um trabalhador coma
durante seu trabalho daquilo com o qual ele está trabalhando, desde que
isso ainda esteja unido ao solo. Este preceito está expresso em Suas
palavras, enaltecido seja Ele, ``Quando entrares na vinha de teu
companheiro, poderás comer uvas\ldots{} quando entrares na seara de teu
companheiro, poderás colher espigas com a tua mão''.\footnote{Deuteronômio
23:25-26.} A Guemará de Baba Metzia explica que desses dois versículos
nós deduzimos que ele pode comer daquilo com que ele estiver trabalhando
e que ainda esteja ligado ao solo; e que nenhum dos dois versículos é
suficiente sem o outro, como no caso mencionado
anteriormente, a respeito do qual citamos que ``estes são dois textos
diferentes e a lei só pode ser compreendida através dos dois juntos''.
Nesse caso o preceito positivo de que um trabalhador deve ter permissão
para comer o que ainda está unido ao solo se deriva de dois versículos,
e os Sábios dizem claramente: ``Eles podem comer de acordo com a lei das
Escrituras etc.''.

As normas deste preceito estão explicadas no sétimo capítulo do Tratado Baba Metzia.

\paragraph{Descarregar um animal cansado}

Por este preceito somos ordenados a descarregar um animal que tenha
sucumbido no campo sob o peso de sua carga. Este preceito está expresso
em Suas palavras, enaltecido seja Ele, ``Quando vires o asno daquele que
te aborrece, prostrado debaixo de sua carga, não te recusarás a
ajudá-lo; auxiliá-lo-ás'',\footnote{Êxodo 23:5.} a respeito das quais a Mekhiltá
diz: ```Auxiliá-lo-ás' se refere a descarregar o peso''. Também está
escrito ali: ``As palavras `Auxilia-lo-ás' nos ensinam que se infringem
ambos, um preceito positivo e um negativo''. Ou seja, somos ordenados a
descarregar o animal e somos proibidos de deixá-lo prostrado sob sua
carga, como explicaremos nos preceitos negativos;\footnote{Ver o preceito negativo 270.}
e aquele que o deixar caído estará infringindo um preceito positivo e um
negativo. Portanto foi explicado que as palavras ``Auxiliá-lo-ás''
contêm um preceito positivo.

As normas deste preceito estão explicadas no segundo capítulo de Baba Metzia.

\paragraph{Ajudar o próximo a levantar sua carga}

Por este preceito somos ordenados a carregar uma carga sobre o animal
ou sobre o homem, sé ele estiver só, depois que ela tiver sido
descarregada por nós ou por outra pessoa. Assim como somos ordenados a
ajudar a descarregar, somos ordenados a ajudar a carregar. Este
preceito está expresso em Suas palavras, enaltecido seja Ele, ``Mas
ajudarás a levantá-los'',\footnote{Deuteronômio 11:4.} sobre as quais diz a
Mekhiltá: ```Ajudarás a levantá-los' se refere a carregar''.

As normas deste preceito estão explicadas no segundo capítulo de Baba
Metzia, onde foi deixado claro que a Torá obriga tanto a carregar como
a descarregar.

\paragraph{Devolver a seu dono o que ele tiver perdido}

Por este preceito somos ordenados a devolver a seu dono o que ele tiver
perdido. Este preceito está expresso em Suas palavras ``Devolvê-lo-ás''\footnote{Êxodo 23:4.} e ``Mas os restituirás a teu irmão''.\footnote{Deuteronômio 22:1.}
Os Sábios dizem explicitamente: ``A devolução de um pertence perdido é
um preceito positivo''. Eles ainda dizem o seguinte, com respeito a um
pertence perdido: ``Somos ensinados que violamos um preceito positivo e
um negativo''. Explicaremos o negativo referente a pertences perdidos no lugar apropriado.\footnote{Ver o preceito negativo 269.}

As normas deste preceito estão explicadas no segundo capítulo de Baba
Metzia.

\paragraph{Repreender o pecador}

Por este preceito somos ordenados a repreender quem estiver pecando ou
estiver inclinado a cometer um pecado, a fim de proibi-lo de agir dessa
forma e a repreendê-lo. Um homem não pode dizer: ``Eu não vou pecar; e
se outra pessoa pecar, isso é um assunto entre ele e Deus''. Tal atitude
é contrária à Torá. Somos ordenados a não pecar e não permitir que
ninguém de nossa nação o faça. Se alguém desejar pecar é dever de cada
um de nós repreendê-lo e evitar que ele o faça, ainda que não haja
evidência de que ele será castigado por isso. Este preceito está
expresso em Suas palavras, enaltecido seja Ele, ``Repreenderás a teu
companheiro''.\footnote{Levítico 19:17.}

Está incluído neste preceito repreender aquele que nos tenha ofendido e
não lhe guardar rancor nem nutrir maus pensamentos a seu respeito. Somos
ordenados a repreendê-lo diretamente, para que nada perdure em nosso
coração contra ele. A Sifrá diz: ``Como sabemos que mesmo que já o
tenhamos repreendido por quatro ou cinco vezes, ainda devemos tornar a
repreendê-lo? Porque a Torá diz `Repreenderás' --- ainda que mil vezes.
Poder-se-ia pensar que ao repreendê-lo se poderia humilhá-lo; o Talmud
diz `E não levarás sobre ti pecado'\,''.

Os Sábios explicam que este preceito é obrigatório para todos, de forma
que até um subalterno é obrigado a repreender um superior, e mesmo que
ele seja amaldiçoado e insultado ele não deve desistir nem deixar de
fazê-lo até que batam nele --- como disseram aqueles que transcreveram a
Tradição: ``Até que ele seja esbofeteado''.

As condições e regulamentos relativos a este preceito estão explicados
em lugares dispersos do Talmud.

\paragraph{Amar o próximo}

Por este preceito somos ordenados a amar uns aos outros da mesma forma
que amamos a nós mesmos, e que o amor e a compaixão de alguém por seu
irmão de fé devem ser iguais ao amor e compaixão que ele tem por si
mesmo com relação a seu dinheiro, seu corpo, e a tudo o que ele possui e
deseja. Tudo aquilo que eu desejar para mim, devo desejar também para
ele; e tudo o que eu não quiser para mim nem para meus amigos também não
devo desejar para ele. Este preceito está expresso em Suas palavras,
enaltecido seja Ele, ``Amarás o teu próximo como a ti mesmo''.\footnote{Levítico
19:18.}

\paragraph{Amar o prosélito}

Por este preceito somos ordenados a amar o prosélito. Ele está expresso
em Suas palavras, enaltecido seja Ele, ``E amareis ao prosélito''.\footnote{Deuteronômio 10:19.} Embora esse prosélito esteja incluído em Israel
--- de tal forma que as palavras ``Amarás o teu próximo como a ti mesmo''\footnote{Levítico 19:18.} se apliquem também a ele ---, o Eterno ordenou que lhe seja
dedicado um amor maior porque ele se converteu à Fé, e acrescentou um
preceito especial em seu favor, assim como fez no caso da advertência
quanto a enganá-lo, em que Ele ordena: ``E não enganareis cada um ao seu
companheiro'',\footnote{Ibid., 25:17. Ver o preceito negativo 251.} e depois acrescenta: ``Ao prosélito não fraudareis''.\footnote{Êxodo 22:20. Ver o preceito negativo 252.} A explicação
dada na Guemará é que ``Aquele que engana um prosélito é culpado de duas
violações: `Não enganareis cada um ao seu companheiro' e `Ao prosélito
não fraudareis'. Também nossa obrigação de amá-lo está em ambos, `Amarás
o teu próximo como a ti mesmo' e `Amareis ao prosélito'\,''. Isto está
claro acima de qualquer dúvida, e não conheço ninguém que, ao enumerar
os preceitos, tenha se enganado a este respeito.

Na maioria dos Midrashot está explicado que o Eterno nos ordenou com
relação aos prosélitos o que Ele nos ordenou com relação a Si mesmo ao
dizer ``E amarás ao Eterno, teu Deus''\footnote{Deuteronômio 6:5.} e ``Amareis ao
prosélito''.

\paragraph{A lei dos pesos e medidas}

Por este preceito somos ordenados a ter pesos, balanças e medidas
exatos, e a regulá-los com extrema precisão. Ele está expresso em Suas
palavras ``Balanças justas, pesos justos, `efá' justa e `hin' justo
tereis para vós'',\footnote{Levítico 19:36.} sobre as quais diz a Sifrá:
```Balanças justas' significa que as balanças devem ser absolutamente
precisas: `pesos justos' significa que os pesos devem ser absolutamente
exatos; `\emph{efá} justa' significa que a `efá' deve ser absolutamente
exata; e `\emph{hin} justo' significa que o `hin' deve ser absolutamente
exato''. Você sabe que a ``efá'' é uma medida para sólidos e o ``hin''
uma medida para líquidos. A mesma regra se aplica a todos esses casos,
embora o tipo de medidas possa variar, porque o que é pesado ou medido é
simplesmente a quantidade de alguma coisa. Todo esse tipo de coisas ---
balanças, pesos, medidas para sólidos e medidas para líquidos --- são
chamadas medidas, e o preceito que nos obriga a regular cada uma delas
com precisão, de acordo com os padrões aprovados, é chamado de preceito
das Medidas.

A Sifrá diz: ``Eu vos tirei da terra do Egito com a condição de que
vocês aceitem sobre si mesmos o preceito sobre as Medidas; pois aquele
que reconhece o preceito das Medidas reconhece através dele o Êxodo do
Egito, e aquele que o negar estará negando\footnote{Também a autenticidade do.} o
Êxodo do Egito''.

As normas deste preceito estão explicadas no quinto capítulo de Baba Batra.

\paragraph{Honrar os eruditos e os idosos}

Por este preceito somos ordenados a respeitar os eruditos e a
levantar-nos diante deles a fim de honrá-los. Ele está expresso em Suas
palavras, enaltecido seja Ele, ``Diante das cãs te levantarás e
honrarás as faces do velho'',\footnote{Levítico 19:32.} sobre as quais a Sifrá diz: ``Te levantarás e honrarás ---
levantar-se para demonstrar respeito''. As normas deste preceito estão
explicadas no primeiro capítulo de Kidushin.

Você deve saber que embora este preceito para respeitar os Eruditos
seja um dever igual para todos, inclusive para um Erudito com relação a
outro de igual conhecimento --- como dizem os Sábios, ``Os eruditos da
Babilônia estavam habituados a levantar-se uns diante dos outros'' ---
ele é especialmente e sobretudo obrigatório para um discípulo, pois ele
deve um respeito muito maior a seu mestre do que a qualquer outro
Erudito, assim como ele tem a obrigação de temê-lo, pois os Sábios
afirmam claramente que nosso dever para com nosso mestre é maior do que
nosso dever para com nosso pai, a quem as Escrituras nos obrigam a
honrar e a temer. E os Sábios dizem explicitamente: ``Seu pai e seu
mestre --- seu mestre tem prioridade''.

Os Sábios também deixam claro que um discípulo está proibido de
contestar seu mestre e por contestar quero dizer se opor a sua decisão,
rejeitar sua opinião e lecionar e instruir sem sua permissão. Ele também
está proibido de brigar e discutir com ele, e de julgá-lo negativamente,
ou seja, atribuir motivações ruins a seus atos ou palavras, pois é
possível que suas intenções\footnote{É possível que as intenções do mestre não sejam as que se pensa.} não
sejam aquelas. No capítulo ``Helek''\footnote{Do Tratado Sanhedrin.} os Sábios
dizem: ``Discordar de seu mestre
é como discordar da `Shekhiná', como está dito em `Quando fizeram brigar
o povo contra o Eterno';\footnote{Números 26:9.} brigar com seu mestre é como
brigar com a `Shekhiná', como está dito em `Estas são as águas de Meribá
por que brigaram os filhos de Israel com o Eterno';\footnote{Ibid., 20:13.}
queixar-se de seu mestre
é como queixar-se da `Shekhiná', como está dito em `Não são sobre nós
vossas queixas, senão sobre o Eterno';\footnote{Êxodo 16:8.}
atribuir\footnote{Atribuir maldade a seu mestre.} a seu mestre é como
atribuir à `Shekhiná', como está dito em `E falou o povo contra Deus e
contra Moisés'\,''.\footnote{Números 21:5.} Tudo isto está perfeitamente claro, pois
embora a discussão de Korah, a briga dos filhos de Israel, e suas
acusações e suspeitas fossem na realidade dirigidas contra Moisés, que
era o Mestre de Israel, as Escrituras os consideram como tendo sido
dirigidas contra Deus. E os Sábios dizem expressamente: ``Que o temor a
seu mestre seja como o temor aos Céus''.

Tudo o que foi exposto foi deduzido do preceito das Escrituras de honrar
os Eruditos e os pais, e ficou claro, pela linguagem do Talmud,
que\footnote{Que o temor por seu mestre.} não é um preceito independente. Você deve
compreender isto.

\paragraph{Honrar os pais}

Por este preceito somos ordenados a honrar nossos pais. Ele está
expresso em Suas palavras, enaltecido seja Ele, ``Honrarás a teu pai e a
tua mãe''.\footnote{Êxodo 20:12.} As normas deste preceito estão explicadas em
vários trechos do Talmud especialmente em Kidushin.

A Sifrá diz: ``O que significa honrar? Assegurar-lhes comida e bebida, roupas e calor, e guiar seus passos.\footnote{Quando os pais estiverem velhos e fracos.}


\paragraph{Respeitar os pais}

Por este preceito somos ordenados a temer nossos pais, a olhá-los com o
respeito devido a alguém de quem se teme o castigo, como o Rei, e a
tratá-los como tratamos aqueles a quem tememos e receamos descontentar.
Ele está expresso em Suas palavras, exaltado seja Ele, ``Cada um a sua
mãe e a seu pai temerá'',\footnote{Levítico 19:3.} sobre as quais diz a Sifrá:
``O que significa temer? Não se sentar em seu assento, nem falar em seu
lugar, nem contradizer suas palavras''.

As normas deste preceito estão explicadas em Kidushin.

\paragraph{``Frutificar e multiplicar''}

Por este preceito somos ordenados a frutificar-nos e multiplicar-nos
para perpetuar a espécie. Esta é a lei da multiplicação; ela está
expressa em Suas palavras, enaltecido seja Ele, ``Frutificai e
multiplicai''.\footnote{Gênesis 1:28.} O Talmud diz que por ocasião do casamento
com uma virgem o noivo está dispensado de recitar o Shemá porque ele
estará ocupado cumprindo um preceito.

As normas deste preceito estão explicadas no sexto capítulo de Yebamot.
Ele não se aplica às mulheres: o Talmud diz explicitamente ``O dever de
frutificar e multiplicar-se recai sobre o homem, não sobre a mulher''.

\paragraph{A lei da consagração pelo casamento}

Por este preceito somos ordenados a desposar uma mulher através de uma
cerimônia de compromisso: seja dando-lhe alguma coisa, ou entregando-lhe
uma certidão de consagração pelo casamento, ou por relação carnal. Este
é o preceito relativo à cerimônia da consagração pelo casamento. Está
dito o seguinte: ``Quando um homem tomar uma mulher e se casar com ela,
etc.''\footnote{Deuteronômio 24:1.} nos ensina que se pode ter uma mulher através da relação
carnal. ``E tendo ela saído da sua casa, poderá ir tornar-se mulher''\footnote{Ibid, 2.} nos ensina que, assim como sua partida se faz por um
documento,\footnote{O documento que concede o divórcio.} ela pode tornar-se esposa de um homem
também por um documento.\footnote{O documento que atesta a consagração pelo casamento.} E aprendemos que uma
mulher pode ser adquirida por dinheiro pelas Suas palavras relativas à
serva hebreia ``Sem dar dinheiro'',\footnote{Êxodo 21:11.} a respeito das quais
diz o Talmud: ``Este senhor não recebe dinheiro, mas há outro senhor que
recebe dinheiro, que é seu pai''. Contudo, a consagração pelo casamento
ordena pela Torá é a da relação carnal, como explicado em vários
trechos dos Tratados Quetubot, Kidushin, e Nidá.

As normas deste preceito estão explicadas na íntegra no Tratado que lida
especificamente com este assunto, que é o Tratado Kidushin.

Os Sábios dizem especificamente que a consagração pelo casamento feita
através da relação carnal é ordenada pela Torá. Assim fica claro que a
origem do preceito relativo à cerimônia da consagração pelo casamento está
nas Escrituras.

\paragraph{O marido deve dedicar-se a sua esposa durante um ano}

Por este preceito o marido é ordenado a dedicar-se a sua esposa durante
um ano inteiro, durante o qual ele não deverá sair do país para viajar
ou ir à guerra, nem assumir qualquer outra obrigação desse tipo. Ele
deverá alegrar-se com sua esposa durante um ano inteiro a partir do dia
de seu casamento. Este preceito está expresso em Suas palavras,
enaltecido seja Ele, ``Livre estará para cuidar de sua casa por um ano,
e alegrará a mulher que tomou''.\footnote{Deuteronômio 24:5.}

As normas deste preceito estão explicadas no oitavo capítulo de Sotá.

\paragraph{A lei da circuncisão}

Por este preceito somos ordenados a circuncidar. Este preceito está
expresso em Suas palavras, enaltecido seja Ele, a Abraham ``Será
circuncidado, entre vós, todo varão''.\footnote{Gênesis 17:10.} A Torá decreta
explicitamente a extinção para aquele que violar este preceito
positivo, com as palavras, enaltecido seja Ele, ``E o varão
incircunciso, que não circuncidar a carne de seu prepúcio, essa alma
será cortada''.\footnote{Ibid., 4. Ou seja, todo Israelita incircunciso se torna culpado e está
sujeito à extinção, quando ele atinge a maioridade. O pai, contudo, não
está sujeito a essa penalidade por não ter incluído seu filho no pacto
de Abraham, embora ele assim esteja transgredindo o mesmo preceito
positivo.}

As normas deste preceito estão explicadas no décimo nono capítulo de
Shabat, e no quarto capítulo de Yebamot. A obrigação de circuncidar um
filho recai sobre o pai e não sobre a mãe, como está explicado em
Kidushin.

\paragraph{A lei do Casamento Levirato}

Por este preceito somos ordenados a que um cunhado tome por esposa a
viúva de seu irmão quando este tiver morrido sem deixar descendentes.
Ele está expresso em Suas palavras, enaltecido seja Ele, ``O irmão de
seu marido estará com ela''.\footnote{Deuteronômio 25:5.}

As normas deste preceito estão explicadas no Tratado Yebamot.

\paragraph{Halitzá}

Por este preceito a mulher de um irmão falecido fica ordenada a
executar Halitzá em seu cunhado, se ele não se casar com ela. Ele
está expresso em Suas palavras, enaltecido seja Ele, ``E lhe descalçará
o sapato do pé''.\footnote{Deuteronômio 25:9.}

As normas deste preceito estão explicadas no Tratado que lida
especialmente com este assunto, que é o Tratado Yebamot.

Você já está familiarizado com a regra estabelecida de que ``O dever de
casar-se com a esposa de um irmão falecido tem prioridade sobre o dever
da Halitzá''. É por esse motivo que o Tratado é chamado ``Yebamot'',
embora ele inclua ambas as leis do casamento levirato e as de
Halitzá.

\paragraph{Um violador deve casar-se com a moça que violentou}

Por este preceito um homem é obrigado a casar-se com a moça que ele
tiver violentado. Ele está expresso em Suas palavras, enaltecido seja
Ele, ``E ela lhe será por mulher, porquanto a afligiu, e não a poderá
despedir por todos os seus dias''.\footnote{Deuteronômio 22:29.} A Guemará de
Macot afirma que o preceito negativo relativo à violação, que é ``E não
a poderá despedir por todos os seus dias,\footnote{Ver o preceito negativo 358.} é um preceito negativo precedido por um positivo'', e assim
disseram.\footnote{Os Sábios.} Afinal isso é um preceito negativo
precedido por um preceito positivo''. Portanto, fica claro que as palavras ``E ela lhe será por mulher'' constitui um preceito positivo.

As normas deste preceito estão explicadas no terceiro e quarto
capítulos de Quetubot.

\paragraph{A lei sobre aquele que difama sua esposa}

Este preceito estabelece a lei relativa a um homem que
difama.\footnote{Difamar a moça com quem ele tenha se casado.} Este preceito ordena que ele seja açoitado
e que fique com essa mulher, uma vez que com relação a isso foi dito ``E
lhe será por mulher, nao a poderá despedir, por todos os seus dias''.\footnote{Deuteronômio 22:19. Ver também o preceito negativo 359.}

A Guemará de Macot afirma que este é um preceito negativo precedido por
um positivo, assim como no caso do violentador.

As normas deste preceito estão explicadas no terceiro e quarto
capítulos de Quetubot.

\paragraph{A lei sobre o sedutor}

Por este preceito somos ordenados quanto à lei sobre o sedutor. Ele está
expresso em Suas palavras, enaltecido seja Ele, ``E quando enganar um
homem a uma virgem etc.''.\footnote{Êxodo 22:15.}

As normas deste preceito estão explicadas no terceiro e quarto
capítulos de Quetubot.

\paragraph{A lei sobre a mulher cativa}

Por este preceito somos ordenados quanto à lei de uma bela
mulher.\footnote{Que tenha sido capturada durante uma guerra.} Ele está expresso em Suas palavras,
enaltecido seja Ele, ``E vires entre os cativos uma mulher formosa''.\footnote{Deuteronômio 21:11.}

As normas deste preceito estão explicadas no início de Kidushin.

\paragraph{A lei do divórcio}

Por este preceito somos ordenados a que, se desejarmos divorciar-nos de
uma mulher, o façamos unicamente através de um documento de divórcio.
Este preceito está expresso em Suas palavras, enaltecido seja Ele,
``Escrever-lhe-á uma carta de divórcio''.\footnote{Deuteronômio 24:1.}

As normas deste preceito, que é a lei do divórcio, estão explicadas por
inteiro no Tratado que lida especificamente com este assunto, e que é o
Tratado Guitin.

\paragraph{A lei sobre uma mulher suspeita de adultério}

Por este preceito somos ordenados quanto à lei da mulher suspeita de ter
cometido adultério. Ele está expresso em Suas palavras, enaltecido seja Ele,
``O homem, quando se desviar sua mulher etc.''.\footnote{Números 5:12.}

As normas deste preceito --- a maneira pela qual ela deve ser forçada a
beber\footnote{Ela deve ser forçada a beber as águas amargas.} e a levar seu sacrifício, e as outras
condições --- estão explicadas no Tratado que lida especificamente com
este assunto, que é o Tratado Sotá.

\paragraph{Açoitar os transgressores de determinados preceitos}

Por este preceito somos ordenados a açoitar com uma correia os
violadores de determinados preceitos. Ele está expresso em Suas
palavras, enaltecido seja Ele, ``O juiz o fará deitar e o fará açoitar
na sua presença''.\footnote{Deuteronômio 25:2.} Quando lidarmos com os preceitos
negativos nós indicaremos quais são os preceitos cuja violação é punida
com o açoitamento.

As normas deste preceito estão explicadas no Tratado Macot.

\paragraph{A lei do homicídio involuntário}

Por este preceito somos ordenados a exilar de sua cidade um homicida
involuntário para uma cidade de refúgio. Ele está expresso em Suas
palavras, enaltecido seja Ele, ``E ficará nela até morrer o Cohen
Gadol''.\footnote{Números 35:25.} sobre as quais diz o Sifrei: ```Ficará nela':
ele nunca poderá sair de lá porque a palavra `nela' significa que lá ele
deverá viver, lá deverá morrer, e lá deverá ser enterrado''.

As normas deste preceito estão explicadas no Tratado Macot.

\paragraph{Executar com a espada os transgressores de determinados preceitos}

Por este preceito somos ordenados a executar com a espada os violadores
de determinados preceitos. Ele está expresso em Suas palavras,
enaltecido seja Ele, ``Serão certamente vingados''.\footnote{Êxodo 21:20.}
Quando tratarmos dos preceitos negativos, mostraremos quais são os
preceitos cuja violação é punida com a decapitação.

As normas deste preceito estão explicadas no sétimo capítulo do Tratado
Sanhedrin.

\paragraph{Estrangular os transgressores de determinados preceitos}

Por este preceito somos ordenados a estrangular os violadores de
determinados preceitos. Ele está expresso em Suas palavras, enaltecido
seja Ele, ``Certamente serão mortos''.\footnote{Levítico 20:10.} Quando tratarmos
dos preceitos negativos mostraremos quais são os preceitos cuja violação
é punida com o estrangulamento.

As normas deste preceito estão explicadas no sétimo capítulo do Tratado
Sanhedrin.

\paragraph{Queimar os transgressores de determinados preceitos}

Por este preceito somos ordenados a queimar os violadores de
determinados preceitos. Ele está expresso em Suas palavras, enaltecido
seja Ele, ``No fogo queimarão a ele e a ela''.\footnote{Levítico 20:14.} Quando
tratarmos dos preceitos negativos mostaremos quais são os preceitos
cuja violação é punida com a morte pelo fogo.

As normas deste preceito estão explicadas no sétimo capítulo do Tratado
Sanhedrin.

\paragraph{Apedrejar os transgressores de determinados preceitos}

Por este preceito somos ordenados a apedrejar os violadores de
determinados preceitos. Ele está expresso em Suas palavras, enaltecido
seja Ele, ``E os apedrejareis, e morrerão''.\footnote{Deuteronômio 22:24.} Quando
tratarmos dos preceitos negativos, mostraremos quais são os preceitos
cuja violação é punida com a morte por apedrejamento.

As normas deste preceito estão explicadas no sexto capítulo de Sanhedrin.


\paragraph{Pendurar os corpos de certos transgressores depois de executados}

Por este preceito somos ordenados a pendurar certos transgressores
executados por ordem do Tribunal. Ele está expresso em Suas palavras,
enaltecido seja Ele, ``O pendurarás num madeiro''.\footnote{Deuteronômio 21:22.}
Quando tratarmos dos preceitos negativos, mostraremos quais são os
preceitos cuja violação acarreta que o corpo seja
pendurado.\footnote{Depois da execução}

As normas deste preceito estão explicadas no sexto capítulo de Sanhedrin.

\paragraph{A lei do enterro}

Por este preceito somos ordenados a enterrar no dia da execução aqueles
que tiverem sido mortos por ordem do Tribunal. Ele está expresso em Suas
palavras, enaltecido seja Ele, ``Certamente enterra-lo-ás no mesmo dia'',\footnote{Deuteronômio 21:23.} sobre as quais o Sifrei diz: `Certamente
enterra-lo-ás' é um preceito positivo.

A mesma lei é obrigatória com relação a todos os outros mortos: todo
Israelita deve ser enterrado no dia de sua morte. É por essa razão que
quando não há ninguém para assistir ao enterro de um corpo ele é
chamado de Corpo de Obrigação Religiosa; ou seja, um corpo cujo enterro
é o dever de cada pessoa, de acordo com Suas palavras, enaltecido seja
Ele, ``Certamente enterra-lo-ás''.

As normas deste preceito estão explicadas no sexto capítulo de Sanhedrin.

\paragraph{A lei do servo hebreu}

Por este preceito somos ordenados quanto à lei do servo hebreu. Ele está
expresso em Suas palavras, enaltecido seja Ele, ``Quando comprares um
escravo hebreu etc.''.\footnote{Êxodo 21:2.}

As normas deste preceito estão claramente explicadas nos versículos da
Torá e todos os seus regulamentos estão no Tratado Kidushin.

\paragraph{O casamento de uma serva hebreia com seu amo ou com o filho dele}

Por este preceito o homem que comprar uma serva hebreia ou o filho dele
são ordenados a casar-se com ela. Este é o preceito relativo aos
esponsais. Os Sábios dizem explicitamente: ``O dever dos esponsais tem
prioridade sobre o dever de resgate, porque o Enaltecido diz: `Que não a
consagrou para si deve remi-la'\,''.\footnote{Êxodo 21:8.}

Você deve estar ciente de que as leis referentes ao servo e à serva
hebreus só estarão em vigor durante a vigência da lei do Jubileu.

As normas deste preceito estão explicadas no primeiro capítulo do
Tratado Kidushin.

\paragraph{O resgate de uma serva hebreia}

Por este preceito somos ordenados quanto ao resgate de uma serva
hebreia. Ele está expresso em Suas palavras, enaltecido seja Ele, ``Deve
remi-la''.\footnote{Êxodo 21:8.}

Há muitas normas, condições e regras referentes a este dever de
resgate. Elas estão todas explicadas no Tratado Kidushin, em que a lei
sobre a serva hebreia está exposta na íntegra.

Explicando Suas palavras, enaltecido seja Ele, ``E se não lhe fizer
estas três coisas'',\footnote{Ibid., 11.} a respeito da serva hebreia, a
Mekhiltá diz que seu dono deve casar-se com ela, ou casá-la com seu
filho, ou remi-la.

\paragraph{A lei sobre o escravo cananeu}

Por este preceito somos ordenados quanto à lei sobre um escravo cananeu;
ela diz que ele deve ser escravo para sempre e que não pode adquirir sua
liberdade a não ser por causa de um dente ou um
olho,\footnote{Se o dono do escravo lhe causar a perda de um dente ou de um olho.} ou por causa de qualquer outro órgão do
corpo que não torne a crescer, de acordo com a interpretação
tradicional. Este preceito está expresso em Suas palavras
``Perpetuamente vos fareis servir deles''\footnote{Levítico 25:46.} e ``E quando
ferir um homem o olho de seu escravo etc.''.\footnote{Êxodo 21:26.}

A Guemará de Guitin diz: ``Todo aquele que liberar seu escravo pagão
estará violando um preceito positivo pois está escrito `Perpetuamente
vos fareis servir deles'\,''. Contudo, a Torá diz que ele poderá obter a
liberdade por causa de um dente ou de um olho.

As normas deste preceito estão explicadas por completo em Kidushin e em
Guitin.

\paragraph{A penalidade por causar ferimentos}

Por este preceito somos ordenados quanto à lei sobre alguém que fere seu
companheiro. Ele está expresso em Suas palavras, enaltecido seja Ele,
``E quando brigarem homens, e ferir um homem a seu próximo etc.''.\footnote{Exodo
21:18.} Essas são chamadas leis sobre Penalidades, todas elas têm sua
base nas Escrituras nas Suas palavras, enaltecido seja Ele, ``Conforme
ele fez, assim lhe será feito'',\footnote{Levítico 24:19.} cujo significado é que
um homem deve pagar a importância equivalente ao dano que causou a seu
companheiro. A Tradição determina que mesmo que ele o tenha apenas
envergonhado, ele deve ser multado na importância equivalente.

Você deve saber que todas essas leis relativas a Penalidades se aplicam
a ferimentos feitos por um homem a outro homem. Da mesma forma existem também leis relativas a ferimentos causados por um animal a uma
pessoa, e vice-versa. Apenas um Tribunal assentado na Terra de Israel
poderá julgar e pronunciar uma sentença referente a essas leis.

A regulamentação deste preceito está explicada no primeiro capítulo de
Baba Kamma.

\paragraph{A lei sobre ferimentos causados por um boi}

Por este preceito somos ordenados quanto à lei do boi. Ele está
expresso em Suas palavras, enaltecido seja Ele, ``Quando marrar um boi
a um homem ou a uma mulher etc.'',\footnote{Êxodo 21:28.} e ``E quando ferir o
boi de um homem ao boi de seu companheiro etc.''.\footnote{Ibid., 35.}

A regulamentação desta lei está explicada nos primeiros seis capítulos
de Baba Kamma.

\paragraph{A lei sobre ferimentos causados por um poço}

Por este preceito somos ordenados quanto à lei do poço. Ele está
expresso em Suas palavras, enaltecido seja Ele, ``E quando um homem
abrir um poço etc.''.\footnote{Êxodo 21:33.}

A regulamentação deste preceito está explicada no terceiro e quarto
capítulos de Baba Kamma.


\paragraph{A lei sobre o roubo}

Por este preceito somos ordenados quanto à lei do ladrão: que devemos
cobrar dele uma restituição em dobro, ou em quádruplo ou em
quíntuplo,\footnote{Êxodo 21:37.} ou que podemos matá-lo se ele for
encontrado arrombando a casa,\footnote{Ibid., 22:1.} ou que podemos
vendê-lo,\footnote{Caso ele não possa fazer a restituição, como lhe é ordenado. Ibid.,
  22:2.} e, de uma maneira geral, que lhe
apliquemos todos os castigos a que estão sujeitos os ladrões, como está
exposto nas Escrituras.

Todos os detalhes desta lei estão explicados no sétimo capítulo de Baba
Kamma, no oitavo capítulo de Sanhedrin, no terceiro capítulo de Baba
Metzia, e em alguns trechos de Quetubot, Kidushin, e Shabuot.

\paragraph{A lei sobre os prejuízos causados por um animal}

Por este preceito somos ordenados quanto à lei sobre o animal que
destrói colheitas. Ele está expresso em Suas palavras, enaltecido seja
Ele, ``Quando um homem fizer pastar num campo ou numa vinha etc.''.\footnote{Êxodo 22:4.}

A regulamentação de toda esta lei está explicada no segundo e sexto
capítulos de Baba Kamma, e no quinto capítulo de Ghitin.


\paragraph{A lei sobre os prejuízos causados pelo fogo}

Por este preceito somos ordenados quanto à lei do fogo. Ele está
expresso em Suas palavras, enaltecido seja Ele, ``Quando houver fogo, e
pegar nos espinhos etc.''.\footnote{Êxodo 22:5.}

A regulamentação desta lei está explicada no segundo e no sexto
capítulos de Baba Kamma.

\paragraph{A lei sobre o depositário não remunerado}

Por este preceito somos ordenados quanto à lei de um depositário não
remunerado. Ele está expresso em Suas palavras, enaltecido seja Ele,
``Quando o homem der ao seu companheiro, dinheiro ou objetos para
guardar etc.''.\footnote{Êxodo 22:6.}

Os detalhes desta lei estão explicados no nono capítulo de Baba Kamma,
no terceiro capítulo de Baba Metzia, e no oitavo capítulo de Shabuot.

\paragraph{A lei sobre o depositário remunerado}

Por este preceito somos ordenados quanto à lei de um depositário
remunerado ou de um arrendador, sendo que uma só lei se aplica a ambos,
como foi explicado pelos Sábios, os quais dizem que há três leis para
regulamentar quatro tipos de depositários. Este preceito está expresso
em Suas palavras, enaltecido seja Ele, ``Quando der o homem a seu
companheiro, asno, boi, carneiro etc.''.\footnote{Exodo 22:9.}

Todos os detalhes desta lei estão explicados no sexto e no nono
capítulo de Baba Kamma, no terceiro e no sexto capítulos de Baba
Metzia, e no oitavo capítulo de Shabuot.

\paragraph{A lei sobre quem pede emprestado}

Por este preceito somos ordenados quanto à lei sobre aquele que pede
emprestado. Ele está expresso em Suas palavras, enaltecido seja Ele, ``E
quando um homem pedir emprestado de seu companheiro etc.''.\footnote{Êxodo
22:13.}

A regulamentação desta lei está explicada no oitavo capítulo de Baba
Metzia e no oitavo capítulo de Shabuot.

\paragraph{A lei de compra e venda}

Por este preceito somos ordenados quanto à lei de compra e venda; ou
seja, o procedimento a seguir pelo vendedor e pelo comprador ao efetuar
uma transação. Aprendemos este procedimento através de Suas palavras,
enaltecido seja Ele, ``E quando fizerdes uma venda a vosso companheiro,
ou comprardes da mão de vosso companheiro etc.'',\footnote{Levítico 25:14.} que
os Sábios interpretam como referindo-se a ``uma mercadoria comprada de mão em mão,
ou seja, por `meshichá'.\footnote{Recibo ou comprovante de uma transação comercial.}

Foi demonstrado que a aquisição por meio de dinheiro fica assegurada
pela lei das Escrituras e que a ``meshichá'' no caso de bens móveis é
apenas uma regulamentação dos Sábios, da mesma forma que entregá-la o
vendedor ao comprador ou levantá-la o comprador. O Talmud diz
explicitamente: ``Assim como eles instituíram a `meshichá' para
compradores, eles também instituíram a `meshichá' para depositários''.

Assim, foi deixado claro que a norma da ``meshichá'' numa venda foi
instituída pelos Sábios, como está explicado no lugar apropriado; mas as
outras formas de procedimento através das quais são adquiridas terras e
outras coisas, a saber, documentos e delimitações, estão baseadas no
versículo.

Os detalhes dessa lei --- ou seja, o procedimento a seguir em cada caso,
ao efetuar uma venda --- estão explicados no primeiro capítulo de
Kidushin, no quarto e oitavo capítulos de Baba Metzia, e no terceiro,
quarto, quinto, sexto e sétimo capítulos de Baba Batra.

\paragraph{A lei sobre os litigantes}

Por este preceito somos ordenados quanto à lei sobre o queixoso
e o acusado. Ele está expresso em Suas palavras, enaltecido seja Ele,
``Sobre toda coisa de delito\ldots{} a respeito da qual se diga `É este'\,''.\footnote{Êxodo 22:8.} A esse respeito a Mekhiltá diz: ```É
este'\footnote{O acusado.} --- admite parte da queixa''.

Esta lei inclui todos os casos de queixas entre os homens que envolvam
confissões ou desmentidos.

Os detalhes desta lei estão explicados no terceiro capítulo de Baba
Kamma, no início e no oitavo capítulo de Baba Metzia, e no quinto, sexto
e sétimo capítulos de Shabuot; muitas perguntas também são encontradas
espalhadas em vários lugares do Talmud.

\paragraph{Salvar a vida do perseguido}

Por este preceito somos ordenados a salvar uma pessoa do perseguidor
que tiver a intenção de matá-la, até mesmo tirando a vida do
perseguidor; ou seja, devemos matar o perseguidor se não pudermos salvar
o perseguido de nenhuma outra forma. Este preceito está expresso em Suas
palavras, enaltecido seja Ele, ``Cortar-lhe-ás a mão, o teu olho não
terá piedade dela''.\footnote{Deuteronômio 25: 11-12.} A esse respeito o Sifrei
diz: ```Pelas suas vergonhas':\footnote{Ibid.} assim como o ato aqui
especificado, por envolver perigo de vida, justifica cortar-se a mão da
mulher, o mesmo princípio deve ser aplicado toda vez que houver risco de
vida. Contudo este versículo nos diz apenas que o homem deve ser salvo
cortando-se a mão da mulher. Como saber se no caso de um homem que não
possa ser salvo cortando-se a mão de alguém, devemos salvá-lo tirando
uma vida? Pelas palavras `O teu olho não terá piedade'\,''.

Portanto, o significado deste preceito foi deixado claro, sendo que as
palavras ``a mulher de um'' são usadas apenas neste caso específico, e
sendo que o verdadeiro significado é que a vida do perseguido deve ser
salva à custa dos membros do perseguidor, e que quando é impossível salvá-lo, a não
ser matando imediatamente o perseguidor, isso deve ser feito.

As normas deste preceito estão explicadas no oitavo capítulo de Sanhedrin.

\paragraph{A lei sobre as heranças}

Por este preceito somos ordenados quanto à lei sobre as heranças. Ele
está expresso em Suas palavras, enaltecido seja Ele, ``Quando um homem
morrer e não tiver filho etc.''.\footnote{Números 27:8.}

Uma das normas desta lei é sem dúvida alguma que o filho primogênito
herde o dobro dos outros, pois esta é uma das leis das
heranças.\footnote{Deuteronômio 21:17.}

As normas deste preceito estão explicadas no oitavo e nono capítulos de
Baba Batra.

\section{Comentários finais de Maimônides sobre os preceitos positivos}

Você deve saber que quando digo, a respeito de cada preceito, ``suas
normas estão explicadas em tal-e-tal lugar'' eu não estou querendo dizer
que o capítulo ou tratado mencionado contém todas as normas daquele
preceito, em seus mínimos detalhes. Eu estou apenas indicando o local
onde se encontram as principais regras e a maioria das normas daquele
preceito, embora haja muitas outras referências relativas a regras
espalhadas em outras partes do Talmud, que eu não menciono
especificamente.

Se você examinar todos os preceitos apresentados até agora, verá que
alguns são obrigatórios a toda a congregação de Israel, de maneira
coletiva, e não a cada pessoa individualmente, como, por exemplo, a
construção do Templo, a nomeação de um rei, e a exterminação da semente
de Amalec. Outros são obrigatórios ao indivíduo que realizou um
determinado ato, ou a quem aconteceu alguma coisa, como, por exemplo,
os sacrifícios oferecidos por quem pecou sem querer, ou um ``zab''; e é
possível que um homem não faça e nem lhe ocorra nenhuma dessas coisas em
toda sua vida. Há entre esses preceitos, como explicamos, determinadas
leis, como a do servo hebreu, e da serva hebreia, a do servo cananeu, a
do depositário não remunerado, a de quem pede emprestado, e outras
mencionadas acima, que pode ser que nunca se apliquem a um determinado
homem, e que pode ser que ele nunca chegue a executar, em toda a sua
vida. Outros preceitos são obrigatórios apenas durante a existência do
Templo, como, por exemplo, as ofertas dos festivais, o comparecimento
diante do Eterno, e a reunião do povo, que nós apresentamos uma a uma.
Outros são obrigatórios apenas para quem tem bens, como, por exemplo,
os dízimos, os sacrifícios de elevação, os presentes prescritos para o
Cohen, e parte para os pobres, tais como as respigas, a gavela
esquecida, ``peá'', e os cachos de uva imperfeitos; e é possível a um
homem ficar isento deles se ele não tiver bens, e passar a vida toda sem
ser obrigado a realizar nenhum dos preceitos desse tipo. A caridade, no
entanto, não pertence a essa categoria, porque ela é uma obrigação até
mesmo para um homem pobre que vive ele próprio de caridade, como
explicamos. Outros preceitos, ainda, são definitivamente obrigatórios a todos os homens, em todos os tempos, em qualquer lugar e em
quaisquer circunstâncias, como por exemplo os tsitsit, os
filactérios, e o Shabat. A esses nós chamamos de preceitos
incondicionais, porque sua obrigatoriedade recai sobre todo israelita
adulto, sempre, em qualquer lugar, e em quaisquer circunstâncias.

Se você refletir sobre os 248 preceitos positivos, descobrirá que os
preceitos ``incondicionais'' são 60, desde que a pessoa sobre quem eles
recaiam se encontre nas mesmas circunstâncias que a maioria das pessoas,
ou seja, que more numa casa da cidade, que se alimente como a maioria
das pessoas, ou seja, com pão e carne, que negocie com as outras
pessoas, que se case com uma mulher e procrie.

Os 60 preceitos positivos, de acordo com a ordem de nossa enumeração,
são: os preceitos positivos 1, 2, 3, 4, 5, 6, 7, 8 e 9. O 10º não é
obrigatório às mulheres, nem o 11º. Os 12, 13, 14, 15 e 18 também não
são obrigatórios para as mulheres. Os 19 e 26 são obrigatórios apenas
aos varões Cohanim. O 32, 54, 73, 94, 143, 146, 147, 149, 150, 152,
154, 155, 156, 157, 158, 159 e 160; o 161 não é obrigatório às mulheres.
O 162, 163, 164, 165, 166 e 167; os 168, 169 e 170 não são obrigatórios
às mulheres. O 172, 175, 184, 195, 197, 206, 207, 208, 209, 210 e 211; o
212 não é obrigatório às mulheres. O 213. Os 214 e 215 só são
obrigatórios aos varões.

Uma mnemônica para o número de preceitos incondicionais é a seguinte: ``Sessenta são as rainhas'',\footnote{Cânticos 6:8.} e o número dos não obrigatórios para as mulheres pode ser lembrado pela expressão ``Que o braço (``yad'')\ldots{} está se fortalecendo'':\footnote{Deuteronômio 32:36.} a palavra ``nashim'' (mulher)
perde seu ``yad'';\footnote{O valor numérico das letras do termo hebraico usado, ``yad'', é 14.} ou então o número dos que são
obrigatórios às mulheres, 46, pode ser lembrado pelo versículo ``Também para você, pelo sangue (``bedam'') de seu pacto'',\footnote{Zacarias 9:11. Isso significa que também ela será redimida pelo mérito do sangue do
  pacto (o preceito da circuncisão). O valor numérico das letras do
  termo hebraico usado aqui, ``bedam'', é 46.} ou seja, a
palavra ``bedam'' indica o número (46) dos preceitos que são incondicionalmente obrigatórios às mulheres e constituem seu pacto específico.

Essas são as observações que achamos necessário registrar na enumeração
dos preceitos positivos.

